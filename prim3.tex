\subsection*{Примечание 3}
\subsubsection*{Еще другие формы, находящиеся в связи с качественной определенностью
величины}

Бесконечно-малое диференциального исчисления есть в своем утвердительном
смысле {\em качественная} определенность величины, а об
этой последней мы показали ближе, что она в этом исчислении наличествует не
только вообще как степенная определенность, но как особенная степенная
определенность отношения некоторой степенной функции к степенному члену
разложения (Entwicklungspotenz)~[ссылка!]. Но качественная определенность
имеется также еще и в дальнейшей, так сказать, более слабой форме, и эта
последняя, равно как связанное с нею употребление бесконечно малых и их
смысл в этом употреблении, должны еще быть рассмотрены в настоящем
примечании.

Исходя из предшествующего, мы должны в этом отношении сперва напомнить, что
различные степенные определения выступают с
{\em аналитической} стороны прежде всего таким образом,
что они оказываются лишь формальными и совершенно
{\em однородными}, означают
{\em числовые величины}, которые как таковые не имеют
вышеуказанного качественного различия друг от друга. Но в приложении к
пространственным предметам аналитическое отношение являет себя во всей
своей качественной определенности, как переход от линейных к плоскостным
определениям, от прямолинейных к криволинейным определениям и~т.~д. Далее,
это приложение влечет за собой то последствие, что пространственные
предметы, согласно своей природе данные в форме
{\em непрерывных} величин, понимаются, как
{\em дискретные}, — плоскость, значит, понимается, как
множество линий, линия, как множество точек и~т.~д. Единственный интерес
такого разложения состоит в определении самих точек, на которые разлагается
линия, линий, на которые разлагается плоскость, и~т.~д., чтобы, исходя из
такого определения, иметь возможность двигаться далее аналитически,~т.~е.,
собственно говоря, арифметически; эти исходные пункты представляют собой
для искомых определений величины те {\em элементы}, из
которых должны быть выведены функция и уравнение для
{\em конкретного}, для
{\em непрерывной} величины. Для решения задач, в
которых по преимуществу оказывается выгодным употреблять этот прием,
требуют, чтобы в виде элемента наличествовало в качестве
{\em исходного} пункта некое
{\em само по себе определенное},
{\em в противоположность} непрямому ходу решения,
поскольку последний может начинать лишь с
{\em пределов}, между которыми лежит то само по себе
определенное, нахождение которого он ставит себе
{\em целью}. Полученный результат сводится в обоих
методах к одному и тому же, если только оказывается возможным найти закон
все дальнейшего и дальнейшего определения, при отсутствии возможности
достигнуть полного,~т.~е. так называемого конечного определения.
{\em Кеплеру} приписывается честь, что ему впервые
пришла в голову мысль прибегнуть к указанному обратному ходу решения и
сделать исходным пунктом дискретное. Его объяснение того, как он понимает
первую теорему {\em архимедова} измерения круга,
выражает это очень просто. Первая теорема Архимеда, как известно, гласит,
что круг равен прямоугольному треугольнику, один катет которого равен
радиусу, а другой "--- длине окружности. Так как Кеплер находит смысл этой
теоремы в том, что {\em окружность} круга содержит в
себе столько же частей, сколько {\em точек},~т.~е.
бесконечно много, из которых каждая может рассматриваться как основание
равнобедренного треугольника, то он этим выражает
{\em разложение непрерывного} в форму
{\em дискретного}. Встречающееся здесь выражение
«бесконечное» еще очень далеко от того определения, которое оно должно
иметь в диференциальном исчислении. Если для таких дискретных найдена
некоторая определенность, функция, то в дальнейшем они должны быть
соединены, должны по существу служить элементами непрерывного. Но так как
никакая сумма точек не образует линии, никакая сумма линий не образует
плоскости, то точки {\em уже с самого начала}
принимаются за линейные, равно как линии за плоскостные. Однако, так как
вместе с тем указанные линейные точки еще {\em не
должны быть линиями}, чем они были бы, если бы их принимали за определенные
количества, то их представляют себе как
{\em бесконечно-малые}. Дискретное способно лишь к
{\em внешнему} объединению, в котором моменты сохраняют
смысл дискретных одних; аналитический переход от последних совершается лишь
к их {\em сумме}, он не есть вместе с тем
геометрический переход от {\em точки} к
{\em линии} и от {\em линии} к
{\em плоскости} и~т.~д. Элементу, имеющему свое
определение как точка или как линия, придается поэтому вместе с тем наряду
с качеством точки еще и качество линейности, а линии "--- еще и качество
плоскости, дабы сумма как сумма маленьких линий оказалась линией и как
сумма маленьких плоскостей —~плоскостью.

Потребность получить этот момент качественного перехода и для этого
прибегнуть к {\em бесконечно-малым} должна быть
рассматриваема как источник всех тех представлений, которые, имея своим
назначением устранить указанные трудности, сами по себе представляют
величайшую трудность. Чтобы сделать излишними эти крайние способы
устранения затруднения, должна была бы иметься возможность показать, что в
самом аналитическом приеме, представляющемся голым
{\em суммированием}, на самом деле уже содержится
{\em умножение}. Но здесь появляется новое допущение,
составляющее основу в этом приложении арифметических отношений к
геометрическим фигурациям, а именно, допущение, что арифметическое
умножение представляет собою также и для геометрического определения
переход в некоторое высшее измерение, что арифметическое умножение величин,
являющихся по своим пространственным определениям
{\em линиями}, есть вместе с тем продуцирование
{\em плоскостного определения}, из линейного; трижды
четыре линейных фута равно 12 линейным футам, но 3 линейных фута,
{\em помноженные} на 4 линейных фута, дают 12
плоскостных и притом квадратных футов, так как в обоих как дискретных
величинах единица "--- одна и та же. {\em Умножение линий
на линии} представляется сначала чем-то бессмысленным, так как умножение
производится вообще над числами,~т.~е. над такими определениями, которые
{\em совершенно однородны} с тем, во что они переходят,
с {\em произведением}, и лишь изменяют свою
{\em величину}. Напротив, то, что называлось бы
умножением линии как таковой на линию "--- это действие называли
{\em ductus lineae in lineam}, равно как
{\em plani in planum}, оно есть также
{\em ductus puncti in lineam}, — есть изменение не
только величины, но изменение их как {\em качественного
определения пространственности}, как измерения; переход линии в плоскость
должен быть понимаем, как выход первой {\em вовне
себя}, равно как выход точки вовне себя есть линия, выход плоскости вовне
себя —~некоторое целое пространство. То же самое получается, когда
представляют себе, что {\em движение} точки образует
линию и~т.~д.; но движение подразумевает определение времени и поэтому
выступает в этом представления лишь как случайное внешнее изменение
состояния; здесь же мы должны брать ту определенность понятия, которую мы
выразили как выход вовне себя "--- качественное изменение "--- и которая
арифметически является умножением единицы (как точки и~т.~д.) на
численность (на линию и~т.~д.). К этому можно еще прибавить то замечание,
что при выходе вовне себя плоскости, что представлялось бы умножением
площади на площадь, получается видимость различия между арифметическим и
геометрическим произведением таким образом, что выход вовне себя плоскости,
как {\em ductus plani in planum}, давал бы
арифметически умножение второго измерения на второе, следовательно,
четырехмерное произведение, которое, однако, геометрическим определением
понижается до трехмерного. Если, с одной стороны, число, так как оно имеет
своим принципом единицу, дает твердое определение для внешне
количественного, то, с другой стороны, свойственное числу продуцирование
настолько же формально; взятое как числовое определение $3 \cdot 3$, помноженное
само на себя, есть $3 \cdot 3 \cdot 3 \cdot 3$; но та же величина, помноженная на себя
как определение площади, удерживается на $3 \cdot 3 \cdot 3$, так как пространство,
представляемое как выход за себя, начинающийся от точки, этой лишь
абстрактной границы, имеет как {\em конкретную}
определенность, начинающуюся с линии, свою истинную границу в третьем
измерении. Упомянутое выше различие могло бы получить действительное
значение в отношении свободного движения, в котором одна сторона,
пространственная, определяется геометрически (в законе Кеплера "---
$s^3 : t^2$), а другая, временная, арифметически.

В чем состоит отличие рассматриваемого здесь качественного от предмета
предыдущего примечания, теперь само собою ясно и без дальнейших объяснений.
В предыдущем примечании качественное заключалось в степенной
определенности; здесь же это качественное, равно как и бесконечно-малое,
есть лишь множитель (в арифметике) относительно произведения, точка
относительно линии, линия относительно плоскости и~т.~д. Долженствующий же
быть сделанным качественный переход от дискретного, на которое, как
представляется, разложена непрерывная величина, к непрерывному
осуществляется как суммирование.

Но что якобы простое суммирование на самом деле содержит в себе умножение,
следовательно, переход из линейного в плоскостное определение, это проще
всего обнаруживается в том способе, каким например показывают, что площадь
трапеции равна произведению суммы ее двух параллельных сторон на половину
высоты. Эту высоту представляют себе лишь как
{\em численность} некоторого множества
{\em дискретных} величин, которые должны быть
суммированы. Эти величины суть линии, лежащие параллельно между теми двумя
ограничивающими трапецию параллельными линиями; их бесконечно много, ибо
они должны составлять площадь, но они суть линии, которые, следовательно,
для того, чтобы быть чем-то плоскостным, должны быть вместе с тем положены
с отрицанием. Чтобы избегнуть трудности, заключающейся в том, что сумма
линий должна дать в результате плоскость, линии сразу же принимаются за
плоскости, но равным образом за
{\em бесконечно-тонкие}, ибо они имеют свое определение
исключительно в линейном элементе (in dem Linearen) параллельных границ
трапеции. Как параллельные и ограниченные другой парой прямолинейных сторон
трапеции они могут быть представлены как члены арифметической прогрессии,
разность которой остается вообще той же самой, но не обязательно должна
быть определена, а первый и последний член которой суть указанные две
параллельные линии; сумма такого ряда равна, как известно,
{\em произведению} этих параллельных линий на
половинную {\em численность} членов. Это последнее
определенное количество называется {\em численностью}
лишь совершенно относительно, лишь сравнительно с представленном о
бесконечно-многих линиях; оно есть вообще определенность величины
некоторого {непрерывного} "--- высоты. Ясно, что то,
что называется суммой, есть вместе с тем {\em ductus
lineae in lineam}, {\em умножение} линейного на
линейное; согласно вышеуказанному определению "--- возникновение плоскостного.
В простейшем случае, в прямоугольнике вообще, каждый из множителей
{\em ab} есть некоторая простая величина; но уже в
дальнейшем, все еще элементарном примере трапеции лишь один множитель есть
простая величина половины высоты, другой же, напротив, определяется через
прогрессию; он тоже есть некоторое линейное, но такое линейное,
определенность величины которого оказывается более запутанной; поскольку
она может быть выражена лишь посредством ряда, постольку интерес к ее
суммированию называется аналитическим,~т.~е. арифметическим; геометрическим
же моментом является здесь умножение, качественный переход от линейного
измерения к плоскостному; один из множителей принимается за
{\em дискретный} лишь в целях арифметического
определения другого, а сам по себе он подобно последнему есть величина
некоторого линейного.

Прием, состоящий в том, чтобы представлять площадь как сумму линий,
употребляется, однако, часто и тогда, когда не имеет места с целью
достижения результата умножение как таковое. Это совершается в тех случаях,
когда дело идет о том, чтобы найти величину, как определенное количество,
не из уравнения, а из пропорции. Известен, например, способ доказательства,
что площадь круга относится к площади эллипса, большая ось которого равна
диаметру этого круга, как большая ось к малой, — способ, состоящий в том,
что каждая из этих площадей принимается за {\em сумму}
принадлежащих ей {\em ординат}; каждая ордината эллипса
относится к соответствующей ординате круга, как малая ось к большой, из
чего заключают, что так же относятся между собою и
{\em суммы} ординат,~т.~е.
{\em площади}. Те, которые при этом желают избегнуть
представления о площади как сумме линий, превращают с помощью обычного,
совершенно излишнего искусственного приема ординаты
{\em в трапеции} бесконечно малой ширины; так как здесь
уравнение есть лишь пропорция, то при этом сравнивается лишь один из двух
линейных элементов площади. Другой элемент площади "--- ось абсцисс
"--- принимается в круге и эллипсе за равный, следовательно, как множитель
арифметического определения величины, за 1, и поэтому пропорция оказывается
всецело зависящей только от отношения одного определяющего момента. Для
{\em представления} площади требуются два измерения; но
{\em определение величины}, как оно дается в этой
пропорции, касается только {\em одного} момента;
поэтому та оказываемая представлению поблажка или помощь, которая состоит в
том, что к этому {\em одному} моменту присоединяется
представление {\em суммы}, есть, собственно говоря,
непонимание того, что здесь требуется для математической определенности.

Данные здесь пояснения доставляют также критерий оценки вышеупомянутого
метода {\em неделимых}, созданного
{\em Кавальери}; метод этот также находит свое
оправдание в данных нами пояснениях, и ему нет надобности прибегать к
помощи бесконечно-малых. Эти неделимые суть линии, когда Кавальери
рассматривает площади, или они суть квадраты, площади кругов, когда он
рассматривает пирамиду или конус, и~т.~д.; принимаемую за определенную
основную линию или основную площадь он называет
{\em правилом}. Это "--- константа, а по своему отношению
к ряду это "--- его первый или последний член; неделимые рассматриваются как
параллельные ей, следовательно, как находящиеся в одинаковом определении по
отношению к фигуре. Общее основоположение
{\em Кавальери} состоит в том (Exerc. Geometr. VI
"--- позднейшее сочинение Exerc. I, стр. 6), что «все как плоские, так и
телесные фигуры {\em относятся} друг к другу, как все
их неделимые, причем эти неделимые
сравниваются~[ссылка!] между собой совокупно, а если у них есть
какая-либо общая пропорция, то в отдельности». — Для этой цели он в
фигурах, имеющих {\em равные} основания и высоты,
сравнивает пропорции между линиями, проведенными параллельно основанию и на
{\em равном расстоянии} от него; все такие линии
некоторой фигуры имеют одинаковое определение и составляют весь ее объем.
Таким образом Кавальери доказывает, например, и ту элементарную теорему,
что параллелограмы, имеющие одинаковую высоту, относятся между собою, как
их основания; каждые две линии, проведенные в обеих фигурах на одинаковом
расстоянии от основания и параллельные ему, относятся между собою, как
основания этих фигур; следовательно, так же относятся между собою и целые
фигуры. В действительности линии не составляют объема фигуры как
{\em непрерывной}, а составляют этот объем, поскольку
он должен {\em определяться} арифметически; линейное
есть тот его элемент, единственно только посредством которого должна быть
постигнута его определенность.

Это приводит нас к тому, {\em чтобы поразмыслить о
различии}, {\em имеющем} место касательно того, в чем
состоит {\em определенность} какой-либо фигуры, а
именно, эта определенность или носит такой характер, как в данном случае
{\em высота} фигуры, или она есть
{\em внешняя граница}. Поскольку она носит характер
{\em внешней границы}, допускается, что
{\em непрерывность} фигуры, так сказать,
{\em следует} равенству или отношению границы;
например, равенство {\em совпадающих} фигур
основывается на совпадении ограничивающих их линий. Но в параллелограмах с
равными высотами и основаниями лишь последняя определенность есть внешняя
граница. Высота, а не вообще параллельность, на которой основано
{\em второе главное определение} фигур, их
{\em отношение}, прибавляет к внешней границе еще
второй принцип определения. Эвклидово доказательство равенства
параллелограммов, имеющих равные высоты и основания, приводит их к
треугольникам, к {\em внешне ограниченным} непрерывным;
в доказательстве же Кавальери, и прежде всего в доказательстве
пропорциональности параллелограмов, граница есть вообще,
{\em определенность величины как таковая}
раскрывающаяся, на любой паре линий, проведенных в обеих фигурах на
одинаковом расстоянии. Эти равные или находящиеся в одинаковом отношении к
основанию линии, взятые {\em совокупно}, дают
находящиеся в одинаковом отношении фигуры. Представление об
{\em агрегате} линий противоречит непрерывности фигуры;
но рассмотрение линий исчерпывает полностью ту определенность, о которой
идет речь. Кавальери часто отвечает на могущее быть выдвинутым возражение,
будто представление о неделимых приводит к тому, что мы должны сравнивать
между собою бесконечные {\em по своей численности}
линии или поверхности (Geom., lib. II, prop. I, Schol.); он проводит
правильное различие, говоря, что он сравнивает между собою не их
численность, которую мы не знаем "--- правильнее сказать: не их численность,
которая, как мы заметили выше, есть вспомогательное пустое представление, а
лишь {\em величину},~т.~е. количественную
определенность как таковую, которая равна занимаемому этими линиями
пространству; так как последнее заключено в границах, то и эта его величина
заключена в тех же границах; {\em непрерывное}, говорит
он, {\em есть не что иное},{\em 
как сами неделимые}; если бы оно было нечто, находящееся
{\em вне их}, то оно не могло бы быть сравниваемо; а
ведь было бы несообразно сказать, что ограниченные непрерывные несравнимы
между собою.

Как видим, Кавальери хочет провести различие между тем, что принадлежит к
{\em внешнему существованию} непрерывного, и тем, в чем
состоит его {\em определенность}, и что единственно и
следует выделять в целях сравнения и для получения теорем о нем. Категории,
которые он употребляет при этом, говоря, что непрерывное
{\em сложено} из неделимых или
{\em состоит} из них и~т.~п., разумеется,
неудовлетворительны, так как при этом приходится утверждать вместе с тем
созерцаемость непрерывного или, как мы сказали выше, его внешнее
существование; вместо того, чтобы сказать, что «непрерывное есть не что
иное, как сами неделимые», было бы правильнее и, стало быть, само по себе
сразу ясно, сказать, что определенность величины непрерывного есть не что
иное, как определенность величины самих неделимых. — Кавальери не придает
никакого значения плохому выводу, что, стало быть, существуют-де большие и
меньшие бесконечные, выводу, делаемому {\em школой}, из
представления, что неделимые составляют непрерывное, и он определенно
выражает далее (Geom., lib. VII, praef.) уверенность в том, что он своим
способом доказательства отнюдь не вынуждается представлять себе непрерывное
сложенным из неделимых; {\em непрерывные лишь следуют
пропорции неделимых}. Он, говорит о своем методе Кавальери, берет агрегаты
неделимых не с той стороны, с какой они кажутся подпадающими под
определение бесконечности, как представляющие собою
{\em бесконечное множество} линий или плоскостей, а
лишь постольку, поскольку они имеют некоторый
{\em определенный характер и природу ограниченности}.
Но чтобы устранить и этот камень преткновения, он все же в специально для
этого прибавленной седьмой книге не жалеет труда, доказать основные теоремы
своей геометрии таким способом, который остается свободным от примеси
бесконечности. — Этот способ сводит доказательства к вышеупомянутой обычной
форме {\em наложения} фигур,~т.~е., как мы заметили
выше, к представлению об определенности как о
{\em внешней пространственной границе}.

Относительно этой формы наложения можно, прежде всего, сделать еще и то
замечание, что она есть, так сказать, ребяческая помощь чувственному
созерцанию. В элементарных теоремах о треугольниках представляют их два
рядом, и, поскольку в каждом из них из шести частей известные три
принимаются равными соответствующим трем частям другого треугольника,
показывается, что такие треугольники совпадают между собою,~т.~е., что
каждый из них имеет равными с другим также и
{\em прочие три} части, так как они вследствие
равенства тех трех первых частей {\em полностью
налагаются друг на друга}. Формулируя это более абстрактно, можно сказать,
что именно вследствие равенства каждой пары соответствующих частей двух
треугольников имеется {\em только один треугольник}; в
последнем три части принимаются нами за {\em уже
определенные}, из чего следует {\em определенность}
также и трех остальных частей. Здесь таким образом показывается, что в трех
частях {\em определенность}
{\em завершена}; стало быть, для определенности как
таковой три остальные части представляют собою некоторое
{\em излишество} "--- {\em излишество
чувственного существования},~т.~е. созерцания непрерывности. Высказанная в
такой форме качественная определенность выступает здесь в своем отличии от
того, что предлежит в созерцании, от целого как некоторого непрерывного
внутри себя; {\em наложение} мешает осознать это
различие.

Вместе с параллельными линиями и в параллелограмах появляется, как мы
заметили, новое обстоятельство, заключающееся отчасти в равенстве одних
только углов, отчасти же в том значении, которое имеет высота фигур, причем
внешние границы последних, стороны параллелограмов, отличны от высоты. При
этом делается явственной имеющаяся здесь двусмысленность, состоящая в
вопросе о том, в какой мере в этих фигурах "--- кроме определенности одной
стороны, основания, которое есть внешняя граница, — следует в качестве
другой определенности принимать s{\em другую внешнюю
границу} (а именно, другую сторону параллелограмм) и в какой мере "--- высоту.
Если даны две такие фигуры, имеющие одинаковые основания и высоты, причем
одна из них прямоугольная, а другая с очень острыми углами (и, стало быть,
с очень тупыми углами на другом конце), то последняя фигура легко может
показаться созерцанию бóльшей, чем первая, поскольку созерцание берет
предлежащую большую сторону как {\em определяющую} и
поскольку оно согласно способу представления Кавальери сравнивает
{\em площади} по некоторому
{\em множеству} параллельных линий, которыми они могут
быть пересечены. Согласно этому способу представления
{\em более длинная} боковая сторона остроугольного
параллелограма могла бы рассматриваться как возможность
{\em бóльшего} количества линий, чем то количество
линий, возможность которого содержится в вертикальной стороне
прямоугольника. Однако, такое представление не служит возражением против
метода Кавальери; ибо представляемое в этих двух параллелограмах с целью
сравнения {\em множество} параллельных линий
предполагает вместе с тем {\em одинаковость их
расстояний} друг от друга или от основания, из чего следует, что
{\em другим определяющим моментом} служит высота, а не
другая сторона параллелограма. Но это, далее, меняется, когда мы сравниваем
между собою два параллелограма, имеющие одинаковые основания и высоты, но
не лежащие в одной плоскости и образующие с третьей плоскостью разные углы;
здесь параллельные сечения, возникающие, когда представляют себе их
пересеченными третьей плоскостью, движущейся параллельно себе самой, уже не
одинаково удалены одно от другого, и эти две площади неравны между собою.
Кавальери тщательно обращает внимание читателя на это различие, которое он
определяет как различие между transitus rectus (прямым переходом) и
transitus obliquus (косвенным переходом) неделимых (как в Exercit I n. XII
и сл., — так уже и в Geometr., I, II) и этим он устраняет поверхностное
недоразумение, могущее возникнуть с этой стороны. Я припоминаю, что
{\em Барроу} в своем вышеупомянутом сочинении (Lect.
Geom., II, p. 21), хотя также пользуется методом неделимых, но, нарушая его
чистоту, соединяет его с перешедшим от него к его ученику Ньютону и к
другим современным ему математикам, в том числе к Лейбницу, допущением
возможности приравнять криволинейный треугольник, как например так
называемый характеристический, прямоугольному, поскольку оба
бесконечно,~т.~е. {\em очень} малы, — я припоминаю, что
Барроу приводит идущее именно в том же направлении возражение
{\em Такэ}, остроумного геометра того времени, также
пользовавшегося новыми методами. Указываемое последним затруднение касается
также вопроса о том, какую линию, — а именно при вычислении
{\em конических} и
{\em сферических} поверхностей "--- следует принимать за
{\em основной момент определения} для рассуждения,
основанного на применении дискретного. Такэ возражает против метода
неделимых, что при вычислении {\em поверхности}
прямоугольного конуса по этому атомистическому методу тот треугольник,
который получается при продольном рассечении конуса, изображается:
составленным из прямых {\em линий}, параллельных
основанию, перпендикулярных к оси и представляющих собою вместе с тем
{\em радиусы тех кругов}, из которых состоит
{\em поверхность }конуса. Но если эта поверхность
определяется как сумма окружностей, а эта сумма определяется из числа их
радиусов,~т.~е. из длины оси конуса, из его высоты, то получаемый результат
противоречит найденной и доказанной Архимедом истине. В ответ на это
возражение Барроу, напротив, показывает, что для определения поверхности
конуса не его ось, а {\em сторона} того треугольника,
который получается при продольном рассечении конуса, должна быть принимаема
за ту линию, вращение которой производит эту поверхность и которая поэтому,
а не ось, должна считаться определенностью величины для множества
окружностей.

Подобного рода возражения или сомнения имеют свой источник единственно
только в употребляемом неопределенном представлении
{\em бесконечного} множества точек, из которых
считается состоящей линия, или линий, из которых считается состоящей
площадь; этим представлением затушевывается существенная определенность
величины линий или площадей. — Целью настоящих примечаний было вскрыть те
{\em утвердительные} определения, которые при различном
употреблении бесконечно-малых в математике остаются, так сказать, на заднем
плане, и освободить их от того тумана, в который их закутывает эта
применяемая в чисто отрицательном смысле категория. В бесконечном ряде, как
например, в архимедовом измерении круга, «бесконечность» не означает ничего
другого, кроме того, что закон дальнейшего определения известен, но так
называемое {\em конечное},~т.~е. арифметическое
выражение, не дано, сведение дуги к прямой линии не может быть
осуществлено; эта несоизмеримость есть их качественное различие.
Качественное различие дискретного и непрерывного вообще равным образом
содержит в себе некоторое отрицательное определение, которое приводит к
тому, что они выступают как несоизмеримые, и влечет за собою бесконечное в
том смысле, что то непрерывное, которое должно быть принимаемо за
дискретное, не должно уже более быть по своей непрерывной определенности
определенным количеством. Непрерывное, которое арифметически должно быть
принимаемо за {\em произведение}, тем самым полагается
в самом себе дискретным, а именно разлагается на те элементы, которые
составляют его множители; в этих множителях заключается определенность его
величины; и именно потому, что они суть эти множители или элементы, они
принадлежат к низшему измерению, а поскольку появляется степенная
определенность, имеют степень низшую, чем та величина, элементами или
множителями которой они являются. Арифметически это различие представляется
чисто количественным различием корня и степени или какой-нибудь другой
степенной определенности. Но если это выражение имеет в виду лишь
количественное как таковое, например $a : a^2$ или $d : a^2 = 2a : a^2 = 2 : a$, или для закона падения тел 
$t : at^2$ , то оно дает лишь ничего не говорящие отношения 
$1 : a$, $2:a$, $1 : at$; в противоположность своему; чисто
количественному определению члены отношения должны были бы быть удерживаемы
врозь своим различным качественным значением, как например в 
$s : at^2$ , где величина выражается как некоторое качество,
как функция величины некоторого другого качества. При этом перед сознанием
стоит исключительно только количественная определенность, над которой без
затруднения производятся подобающие действия, и можно с чистой совестью
умножать величину одной линии на величину другой линии; но в результате
умножения этих самых величин получается вместе с тем качественное
изменение, переход линии в площадь, поскольку появляется некоторое
отрицательное определение; оно и вызывает ту трудность, которая разрешается
посредством усмотрения своеобразной природы этого определения и простой
сути дела; но введением бесконечных, от которых ожидалось ее устранение,
эта трудность скорее только еще более запутывается и оставляется совершенно
неразрешенной.

\chapter*{Третья глава}
\section*{Количественное Отношение}

Бесконечность определенного количества была определена выше так, что она
есть его отрицательное потустороннее, которое оно, однако, имеет в самом
себе. Это потустороннее есть качественное вообще. Бесконечное определенное
количество как единство обоих моментов "--- количественной и качественной
определенностей "--- есть ближайшим образом
{\em отношение}.

В отношении определенное количество уже более не обладает лишь безразличной
определенностью, а качественно определено как безоговорочно соотнесенное со
своим потусторонним. Оно продолжает себя в свое потустороннее; последнее
есть ближайшим образом некоторое {\em другое}
определенное количество вообще. Но по существу они соотнесены друг с другом
не как внешние определенные количества, а {\em каждое
имеет свою определенность в этом соотношении с другим}. Они, таким образом,
в этом своем инобытии возвратились в себя; то, что каждое из них есть, оно
есть в другом; другое составляет определенность каждого из них. —
Выхождение определенного количества за себя теперь уже, стало быть, не
имеет ни того смысла, что оно изменяется лишь в некоторое другое, ни того,
что оно изменяется в свое абстрактное другое, в свое отрицательное
потустороннее, а имеет тот смысл, что в этом другом оно достигает своей
определенности; оно находит {\em самого себя} в своем
потустороннем, которое есть некоторое другое определенное количество.
{\em Качество} определенного количества, определенность
его понятия заключается вообще в том, что оно внешне, и вот теперь, в
отношении, оно {\em положено} так, что оно имеет свою
определенность в своей внешности, в некотором другом определенном
количестве, есть в своем потустороннем то, что оно есть.

Тем соотношением между собой, которое здесь получилось, обладают именно
определенные количества. Это {\em соотношение} само
есть также некоторая величина. Определенное количество не только находится
в {\em отношении}, но {\em оно само
положено как отношение}; оно есть {\em некоторое}
определенное количество вообще, имеющее указанную качественную
определенность {\em внутри себя}. Таким образом, как
отношение оно выражает себя, как замкнутую в себе целостность, и свое
безразличие к границе тем, что оно имеет внешность своей определенности
внутри самого себя, и в этой внешности соотнесено лишь с собою, и,
следовательно, бесконечно в самом себе.

Отношение вообще есть:

1. {\em Прямое} отношение. В нем
{\em качественное} еще не выступает наружу как таковое,
само по себе. Определенное количество положено здесь пока что исключительно
в аспекте определенного количества, положено имеющим свою определенность в
самой своей внешности. — Количественное отношение есть в себе противоречие
внешности и соотношения с самим собою, устойчивости определенных количеств
и отрицания их. Это противоречие снимает себя, поскольку ближайшим образом

2. в {\em обратном} отношении сополагается
{\em отрицание} одного определенного количества как
таковое в изменении другого и изменчивость самого прямого отношения;

3. в {\em степенном} же
{\em отношении} выдвигается соотносящаяся в своем
различии, с самой собою единица как простое самопродуцирование
определенного количества. И наконец, само это качественное, положенное в
простом определении и как тождественное с определенным количеством,
становится {\em мерой}.

О природе излагаемых ниже отношений многое уже было сказано наперед в
предшествующих примечаниях, касающихся бесконечного в количестве,~т.~е.
качественного момента в последнем; теперь остается поэтому лишь разъяснить
абстрактное понятие этих отношений.

\paragraph[А. \ Прямое отношение]{А. \ Прямое отношение}
\ 1. В отношении, которое как непосредственное есть
{\em прямое} отношение, определенность одного
определенного количества заключается в определенности другого определенного
количества, и это взаимно. Имеется лишь {\em одна}
определенность или граница обоих, которая сама есть определенное количество
"--- {\em показатель} отношения.

2. Показатель есть какое-нибудь определенное количество. Но он есть в своей
{\em внешности} соотносящееся с
{\em собою} в самом себе качественно-определенное
количество лишь постольку, поскольку он в нем самом имеет отличие от себя,
свое потустороннее и инобытие. Но это различие определенного количества в
нем {\em самом} есть различие
{\em единицы} и {\em численности};
единица есть самостоятельная определенность (Für-sich-bestimmtsein);
численность же "--- безразличное движение туда и сюда вдоль определенности,
внешнее безразличие определенного количества. Единица и численность были
первоначально моментами определенного количества; теперь в отношении,
которое постольку есть реализованное определенное количество, каждый из его
моментов выступает как некоторое особое
{\em определенное количество}, и оба они "--- как
определения его наличного бытия, как ограничения по отношению к
определенности величины, которая помимо этого есть лишь внешняя,
безразличная определенность.

Показатель есть это различие как простая определенность,~т.~е. он имеет
непосредственно в самом себе значение обоих определений. Он есть,
{\em во-первых}, определенное количество; в этом смысле
он есть численность; если один из членов отношения, принимаемый за единицу,
выражается нумерической единицей "--- а ведь он считается лишь таковой
единицей, — то другой член, численность, есть определенное количество
самого показателя. {\em Во-вторых}, показатель есть
простая определенность как качественное в членах отношения; если
определенное количество одного из членов определено, то и другое
определенное количество определено показателем, и совершенно безразлично,
как определяется первое; оно, как определенное само по себе определенное
количество, уже более не имеет никакого значения и может быть также и любым
другим определенным количеством, не изменяя этим определенности отношения,
которая покоится исключительно на показателе. Одно определенное количество,
принимаемое за единицу, как бы велико оно ни стало, всегда остается
единицей, а другое определенное количество, как бы велико оно при этом
также ни стало, непременно должно оставаться {\em одной
и той же} численностью указанной единицы.

3. Согласно этому оба они составляют, собственно говоря, лишь
{\em одно} определенное количество; одно определенное
количество имеет по отношению к другому лишь значение единицы, а не
численности; другое имеет лишь значение численности; стала быть,
{\em по определенности своего понятия} сами они
{\em не} являются {\em полными}
определенными количествами. Но эта неполнота есть отрицание в них и притом
отрицание не со стороны изменчивости вообще, по которой одно (а каждое из
них есть одно из двух) может принимать всевозможные величины, а со стороны
того определения, что если одно изменяется, то и другое настолько же
увеличивается или уменьшается; это, как мы показали, означает: лишь
{\em одно}, единица, изменяется как определенное
количество, другой же член, численность, остается тем же определенным
количеством {\em единиц}, но и первый член также лишь
сохраняет {\em значение} единицы, как бы он ни
изменялся как определенное количество. Каждый член есть, таким образом,
лишь один из этих двух моментов определенного количества, и
самостоятельность, требующаяся для его своеобразия,
{\em подверглась} в себе
{\em отрицанию}; в этой качественной связи они должны
быть {\em положены} один по отношению к другому как
{\em отрицательные}.

Показатель, по вышесказанному, есть полное определенное количество, так как
в нем сходятся определения {\em обоих} членов
отношения; но на самом деле он как частное сам имеет значение только либо
{\em численности}, либо
{\em единицы}. Нет никакого указания (Bestimmung),
какой из членов отношения должен быть принимаем за единицу и какой за
численность; если один из них, определенное количество
{\em B}, измеряется определенным количеством
{\em A} как единицей, то частное
{\em C} есть численность таких единиц; но если принять
само {\em A} за численность, то частное
{\em C} есть единица, требуемая при численности
{\em A} для определенного количества
{\em B}; тем самым это частное как показатель положено
не как то, чем оно должно быть, — не как то, что определяет отношение или
как его качественная единица. Как последняя оно положено лишь постольку,
поскольку оно имеет значение {\em единства обоих
моментов}, единицы и численности. Так как эти члены отношения, хотя они и
даны как определенные количества такими, какими они должны быть в
развернутом определенном количестве, в отношении, все же при этом даны лишь
в том значении, которое они должны иметь как его члены,~т.~е. суть
{\em неполные} определенные количества и считаются лишь
за один из указанных качественных моментов, то они должны быть положены с
этим их отрицанием; благодаря этому возникает более соответствующее его
определению, более реальное отношение, в котором показатель имеет значение
произведения сторон отношения; согласно этому определению оно есть
{\em обратное отношение}.

\section*{В. Обратное отношение}
\end{document}
