<<Наука логики>> (т.~н. <<Большая логика>>) создана Гегелем в~нюрнбергский
период его жизни. Первая её часть (<<Объективная логика>>, кн.~1 "--- <<Учение
о~бытии>>) вышла в~начале 1812~г., вторая (<<Объективная логика>>, кн.~2 "---
<<Учение о~сущности>>) "--- в 1813~г., третья (<<Субъективная логика>>, или
<<Учение о~понятии>>) "--- в 1816~г. В~1831~г. перед самой своей смертью Гегель
начал подготовку нового издания <<Науки логики>>. Он успел переработать лишь
первую её часть (<<Учение о~бытии>>), значительно расширив её. В~таком
переработанном виде ата часть была издана Л.~фон Гепннигом уже после смерти
Гегеля в~качестве III~тома Собрания ого сочинений~(1833). Две остальные части
составили IV и V тома Собрания сочинении и~вышли в~свет в 1834~г. Все эти три
части были переизданы в 1841~г. В~1923~г. Г.~Лассон выпустил новое,
<<научно-критическое>> издание. В~юбилейном издании Полного собрания сочинений
Гегеля, подготовленном Г.~Глокнером, <<Наука логики>>
составляет IV и~V тома~(1928).

<<Наука логики>> была переведена па ряд западноевропейских языков: итальянский
({\em Hegel}. La scienza della logica. Traduziono italiana con note di Arturo
Moni. Vol.~1---3. Bari, 1925), английский ({\em Hegel}. Science of Logic.
Translated by W.~H.~Johnston and L.~G.~Struthers. Vol.~1 and~2. London, 1929)
и~др.

На русский язык <<Наука логики>> была переведена многократно. Первый её перевод
был осуществлён Н.~Г.~Дебольским (Пг., 1916; 2-е~изд. М., 1929) по изданию
1841~г. Перевод содержит большое количество ошибок, искажений и~неточностей.
В~значительной мере это касается перевода важных философских терминов Гегеля.
В~издании в~сущности отсутствует научный аппарат (в~нём даны лишь небольшой
алфавитный указатель и~краткие объяснения "--- часто неправильные "---
некоторых философских терминов). К~1937~г. Институт философии АН~СССР
подготовил новое издание <<Науки логики>> в~переводе Б.~Г.~Столпнера,
составившее~V и~VI тома Сочинений Гегеля (т.~V "--- <<Учение о~бытии>> и
<<Учение о~сущности>>. М., 1937; т.~VI "--- <<Учение о~понятии>>. М., 1939).
Это издание снабжено обширным научным аппаратом "--- примечаниями и~указателями
(составитель "--- В.~К.~Брушлинский, осуществивший также сверку перевода),
словарём основных терминов гегелевской логики, а~также библиографией изданий
<<Науки логики>> и~списком литературы о~ней. Переводчик и~редактор проделали
большую работу, благодаря которой удалось точнее передать мысль Гегеля, хотя
иногда и~в~ущерб литературной форме изложения. Этот перевод был взят за основу
при подготовке настоящего издания.

\bigskip

\begin{center}
\ding{93}~~~\ding{93}~~~\ding{93}
\end{center}

\bigskip
