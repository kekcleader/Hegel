\clearpage\subsection{Приложения}
\subsubsection{Предисловие Издателя}\label{bkm:Ref474669698}
{\centering
(Леопольда фон Геннинга)
\par}

Автору «Науки логики» не было даровано закончить предпринятую им с большим
рвением новую обработку этого сочинения. Едва он успел написать последние
слова предисловия к I тому нового издания, как он был схвачен той болезнью,
печальный исход которой положил неожиданный конец его дальнейшей
деятельности на пользу столь мощно подвинутой им вперед науки. Хотя из
сравнения прежнего с новым изданием I тома этой «Логики» можно заключить, в
какой степени также и оба другие тома (перепечатываемые ныне с вышедшего в
свет в 1813 и 1816 гг. первого их издания) еще выиграли бы под рукой их
автора в строгости диалектического построения, в определенности выражения и
во внешней доступности, тем не менее нам служит немалым утешением
возможность сказать, что почившему великому учителю, предпринявшему эту
работу не без многолетней подготовки к ней и уже во вполне зрелом возрасте,
удалось уже в первоначальном ее выполнении создать сочинение, за которым
как уже и ныне, так еще более следующими поколениями будет признана слава
покоящегося на прочном фундаменте и во всех главных своих частях мастерски
выполненного органона мыслительного познания. Если, однако, не оказывается
недостатка и в таких друзьях истины, которые с полным признанием того, что
здесь сделано, считают необходимым несколько задержать свое суждение и
вообще не желают ничего знать о {\em готовой} системе
истины, так как, по их мнению, после такой системы для них и для дальнейших
поколений ничего не осталось бы делать (причем они имеют обыкновение
ссылаться на известное изречение
Лессинга)~\label{bkm:Ref474655210},
то эти люди могут для своего успокоения в достаточной мере усмотреть из
предпринятой новой обработки этого сочинения, как обстоит дело с этой
внушающей опасение законченностью науки и каким образом ею нисколько не
исключаются новые достижения и успехи.—В каких границах наш покойный друг в
учениях о {\em сущности} и о
{\em понятии}, составляющих содержание второй и третьей
части настоящего сочинения, предпринял бы новую их обработку и какие новые
развитие и определения получили бы они, это в общих чертах можно усмотреть
из сравнения их с соответствующими отделами вышедшей в 1830 г. третьим
изданием «Энциклопедии философских наук». Из этого сравнения видно, что
автор, строго удерживая великие основные мысли своего сочинения, на
которые, по его собственному скромному заявлению, следует смотреть, как на
общий результат работ его предшественников в области философского познания,
и последовательно проводя метод, справедливо признанный им за единственно
правильный, сумел сохранить в себе потребные для живого прогресса науки
свежесть и подвижность духа. Пусть тем, которые призваны к дальнейшей
разработке нашей науки, служит постоянным образцом эта способность
самоотречения, это мужество разума и неутомимо стремившееся вперед рвение
дорогого учителя; и тогда не может возникнуть никакой основательной жалобы
на оцепенелость науки и на стеснение ее дальнейшего развития.

Задача издателя при перепечатке настоящего сочинения по самому существу дела
могла состоять лишь в тщательном исправлении найденных опечаток и описок, и
в этом отношении он в сомнительных местах позволял себе исключительно лишь
такие изменения, относительно которых он имел право быть вполне уверенным в
согласии автора, если бы была дарована возможность получить это согласие.

{\em Берлин}, 3 мая 1834 г.
