\chapter[{\em Третья глава} Для-Себя-Бытие]{ Третья глава. Для-Себя-Бытие}
В {\em для-себя-бытии качественное бытие завершено}; оно
есть бесконечное бытие. Бытие, которым мы начали, лишено определений.
Наличное бытие есть снятое, но лишь непосредственно снятое бытие. Оно,
таким образом, содержит в себе пока что лишь первое отрицание, которое само
непосредственно. Бытие, правда, также сохранено, и в наличном бытии оба
(т.~е. бытие и отрицание) объединены в простое единство, но как раз поэтому
они сами в себе еще {\em неравны} друг другу и их
единство еще {\em не положено}. Наличное бытие есть
поэтому сфера дифферентности, дуализма, область конечности. Определенность
есть определенность как таковая, некая относительная, а не абсолютная
определяемость. В для-себя-бытии различие между бытием и определенностью
или отрицанием положено и примирено; качество, инобытие, граница, как и
реальность, в-себе-бытие, долженствование и~т.~д. суть несовершенные
внедрения отрицания в бытие, в каковом внедрении еще лежит в основании
различие обоих. Но так как в конечности отрицание перешло в бесконечность,
в {\em положенное} отрицание отрицания, то оно есть
простое соотношение с собою, есть, следовательно, в самом себе примирение с
бытием "--- {\em абсолютная определенность}.

Для-себя-бытие есть, {\em во-первых}, непосредственно
для-себя-сущее, {\em одно}.

{\em Во-вторых}, одно переходит во
{\em множество одних} "--- в
{\em отталкивание}, каковое инобытие одного снимается в
идеальности последнего; это "--- {\em притяжение}.

{\em В-третьих}, оно есть взаимоопределение отталкивания
и притяжения, в котором они погружаются вместе в равновесие, и качество,
доведшее себя в для-себя-бытии до последнего заострения, переходит в
{\em количество}.

\section[А. Для-себя-бытие как таковое]{А. Для-себя-бытие как таковое}
Общее понятие для-себя-бытия получилось. Теперь дело идет только о том,
чтобы доказать, что этому понятию соответствует представление, которое мы
соединяем с выражением «для-себя-бытие», дабы мы имели право употреблять
его для обозначения сказанного понятия. И, повидимому, это так; мы говорим,
что нечто есть для себя, поскольку оно снимает инобытие, свое соотношение и
свою общность с другим, оттолкнуло их от себя, абстрагировалось от них.
Другое имеет для него бытие лишь как некое снятое, как
{\em его момент}. Для-себя-бытие состоит в
{\em таком} выходе за предел, за свое инобытие, что оно
как это отрицание есть бесконечное {\em возвращение} в
себя. — Сознание уже как таковое содержит в себе определение
для-себя-бытия, так как оно представляет себе тот предмет, который оно
ощущает, созерцает,~т.~е. имеет его содержание
{\em внутри себя}, каковое содержание, таким образом,
дано как {\em идеализованное}; в самом своем созерцании
и вообще в своей переплетенности со своим отрицательным, с другим, оно
{\em находится у самого себя}. Для-себя-бытие есть
полемическое, отрицательное отношение к ограничивающему другому и через это
отрицание последнего "--- рефлектированность в себя, хотя
{\em наряду} с этим возвращением сознания в себя и
идеальностью предмета еще сохранилась {\em также и его
реальность}, так как его знают в то же время как
некое внешнее наличное бытие. Сознание есть, таким образом,
{\em являющееся} или, иначе говоря, есть дуализм,
заключающийся в том, что оно, с одной стороны, знает о некотором другом для
него внешнем предмете, а с другой стороны, есть для себя, имеет предмет в
себе идеализованным, находится не только у такового другого, а в нем
находится также и у самого себя. Напротив,
{\em самосознание} есть
{\em для-себя-бытие} как
{\em исполненное} и
{\em положенное}; вышеуказанная сторона соотношения с
некоторым {\em другим}, с внешним предметом устранена.
Самосознание есть, таким образом, ближайший пример наличия бесконечности,
правда, все еще абстрактной бесконечности, которая, однако, вместе с тем
носит характер определения, совершенно иным образом конкретного, чем
для-себя-бытие вообще, бесконечность которого еще всецело имеет
исключительно лишь качественную определенность.

\subsection[a) Наличное бытие и для-себя-бытие]{a) Наличное бытие и для-себя-бытие}
Для-себя-бытие есть, как мы уже указали, погрузившаяся в простое бытие
бесконечность; оно есть наличное бытие, поскольку отрицательная природа
бесконечности, которая есть отрицание отрицания в положенной теперь форме
непосредственности бытия, дана лишь как отрицание вообще, как простая
качественная определенность. Но бытие в такой определенности, в которой оно
есть наличное бытие, также и отлично "--- это сразу явно "--- от самого
для-себя-бытия, которое есть для-себя-бытие лишь постольку, поскольку его
определенность есть сказанное бесконечное. Однако наличное бытие есть
вместе с тем момент самого для-себя-бытия, ибо последнее содержит в себе,
во всяком случае, также и бытие, обремененное отрицанием. Таким образом,
определенность, которая в наличном бытии как таковом есть некоторое
{\em другое} и
{\em бытие-для-другого}, повернута обратно в
бесконечное единство для-себя-бытия, и момент наличного бытия имеется в
для-себя-бытии как {\em бытие-для-одного}.

\subsection[b) Бытие-для-одного]{b) Бытие-для-одного}
Этот момент выражает тот способ, каким конечное есть в своем единстве с
бесконечным или, иначе говоря, имеет бытие как идеализованное.
Для-себя-бытие имеет отрицание не {\em в себе}, как
некоторую определенность или границу и, значит, также и не как соотношение
с некоторым иным, чем оно, наличным бытием. Обозначив этот момент как
{\em бытие-для-одного}, следует сказать, что еще нет
ничего, для которого он был бы, — еще нет того одного, момент которого он
составлял бы. И в самом деле, такого рода одно еще не фиксировано в
для-себя-бытии; то, для чего нечто (а здесь нет никакого нечто) было бы,
то, что вообще должно было бы быть другой стороной, есть равным образом
момент, само есть лишь бытие-для-одного, еще не есть одно. — Следовательно,
еще имеется неразличенность тех двух сторон, которые могут предноситься
нашему умственному взору в бытии-для-одного. Есть лишь
{\em одно} бытие-для-другого, и так как есть лишь
{\em одно} бытие-для-другого, то последнее есть также
лишь бытие-для-одного; оно есть лишь {\em одна}
идеальность того, для чего или в чем некоторое определение должно было бы
быть как момент, и того, что должно было бы быть в нем моментом. Таким
образом, {\em для-одного-бытие} и
{\em для-себя-бытие} не составляют истинных
определенностей в отношении друг друга. Поскольку мы принимаем на одно
мгновенье, что имеется различие, и говорим здесь о некотором
{\em для-себя-сущем}, то само для-себя-сущее как
снятость инобытия соотносится с собою как со снятым другим, стало быть,
есть {\em для-одного}; оно соотносится в своем другом
лишь с собою. Идеализованное необходимо есть
{\em для-одного}, но оно не есть для некоторого
{\em другого}; то одно, для которого оно есть, есть
лишь само же оно. — Следовательно, «я», дух вообще или бог суть
идеализованные, потому что они бесконечны, но они в своей идеальности, как
для-себя-сущие, не разнятся от того, что есть для-одного. Ибо, таким
образом, они были бы лишь непосредственными или, ближе, наличным бытием и
неким бытием-для-другого, потому что то, что есть для них, было бы не они
сами, а некоторое другое, если бы им не был присущ момент бытия-для-одного.
Поэтому бог есть {\em для себя}, поскольку сам он есть
то, что есть {\em для него}.

Для-себя-бытие и для-одного-бытие суть, следовательно, не разные значения
идеальности, а существенные, неразделимые ее моменты.

\subsubsection[Примечание. Выражение: Was für eines?]
{Примечание. Выражение: Was für eines?\pagenote{Выражение «Was
für eines?» берется Гегелем как общий вид или общий тип таких выражений,
как «Was für Einer ist er?» (что он за человек?), «Was für ein Ding ist
das?» (что это за вещь?) и~т.~д. «Was für Eines ist das?» можно передать
по-русски словами: «Это что за штука?».}}

Кажущееся сперва странным выражение нашего языка при вопросе о качестве, was
für ein Ding etwas sei (по-русски: что это за вещь, но буквально это
выражение означает: что есть нечто для одной вещи, и эту двусмысленность
данного выражения использует здесь Гегель. — {\em Перев}.), выделяет
рассматриваемый здесь момент в его рефлексии в себя. Это выражение
идеалистично в своем происхождении, так как оно не спрашивает, что есть эта
вещь {\em А для другой вещи} В, не спрашивает, что есть
этот человек для другого человека, а спрашивает, что
{\em это для одной вещи, для одного человека} (т.~е.
что {\em это за вещь}, {\em за
человек}. — {\em Перев}.), так что это бытие-для-одного вместе с тем
возвратилось в самое эту вещь, в самого этого человека, и то,
{\em что есть}, и то, {\em для
чего} оно есть, есть одно и то же; мы видим здесь тождество, каковым должна
рассматриваться также и идеальность.

Идеальность присуща ближайшим образом снятым определениям, как отличным от
того, {\em в чем} они сняты, каковое, напротив, можно
брать как реальное. Однако таким образом идеализованное оказывается опять
одним из моментов, а реальное "--- другим; но идеальность заключается в том,
что оба определения одинаково суть только для
{\em одного} и считаются лишь за
{\em одно}, каковая одна идеальность тем самым есть
неразличимо реальность. В этом смысле самосознание, дух, бог есть
идеализованное как бесконечное соотношение чисто с собою, — «я» есть для
«я», оба суть одно и то же; «я» названо два раза, но каждое из этих двух
есть лишь для-одного, идеализованно; дух есть лишь для духа, бог есть лишь
для бога и лишь это единство есть бог, бог как дух. — Но самосознание как
сознание вступает в различие между {\em собою} и
некоторым {\em другим} или, иными словами, между своей
идеальностью, в которой оно есть представляющее, и своей реальностью,
поскольку его представление имеет некоторое определенное содержание,
которое имеет еще ту сторону, что его знают как неснятое отрицательное,
как наличное бытие. Однако называть мысль, дух, бога
{\em лишь} идеализованными, значит, исходить из той
точки зрения, на которой конечное наличное бытие представляется реальным, а
идеализованное или бытие-для-одного имеет только односторонний смысл.

В одном из предшествующих примечаний мы указали принцип идеализма и сказали,
что, зная принцип, важно знать относительно того или другого философского
учения, насколько последовательно оно проводит этот принцип. О характере
проведения указанного принципа в отношении той категории, которая нас
сейчас занимает, можно сделать дальнейшее замечание. Последовательность в
проведении этого принципа зависит ближайшим образом от того, остается ли в
данном философском учении самостоятельно существовать наряду с
для-себя-бытием еще и конечное бытие, а затем также и от того, положен ли
уже в самом бесконечном момент «{\em для-одного}»
—~отношение идеализованного к себе как к идеализованному. Так например
элеатское бытие или спинозовская субстанция суть лишь абстрактное отрицание
всякой определенности, причем в них самих идеальность еще не положена. У
{\em Спинозы}, как мы об этом скажем ниже,
бесконечность есть лишь абсолютное {\em утверждение}
некоторой вещи и, следовательно, лишь неподвижное единство; субстанция
поэтому не доходит даже до определения для-себя-бытия и тем менее до
определения субъекта и духа. Идеализм благородного
{\em Мальбранша} более развернут внутри себя; он
содержит в себе следующие основные мысли: так как бог заключает в себе все
вечные истины, идеи и совершенства всех вещей, так что они принадлежат лишь
{\em ему}, то мы их видим только в нем; бог вызывает в
нас наши ощущения предметов посредством действия, в котором нет ничего
чувственного, причем мы воображаем себе, что получаем от предмета не только
его идею, представляющую его сущность, но также и ощущение его
существования («Разыскание истины», Разъяснение относительно природы идей
и~т.~д.). Стало быть, не только вечные истины и идеи (сущности) вещей, но и
их существование есть существование в боге, идеализованное, а не
действительное существование, хотя, как наши предметы, они суть только
{\em для-одного}. Недостающий в спинозизме момент
развернутого и конкретного идеализма здесь имеется налицо, так как
абсолютная идеальность определена как знание. Как ни чист и ни глубок этот
идеализм, все же указанные отношения частью содержат еще в себе много
неопределенного для мысли, частью же их содержание сразу же оказывается
совершенно конкретным (грех и спасение и~т.~д. сразу же появляются в этой
философии). Логическое определение бесконечности, которое должно было бы
быть основой этого идеализма, не разработано самостоятельно, и, таким
образом, этот возвышенный и наполненный идеализм есть, правда, продукт
чистого спекулятивного ума, но еще не чистого спекулятивного, единственно
лишь дающего истинное обоснование мышления.

{\em Лейбницевский} идеализм движется в большей мере в
рамках абстрактного понятия. — {\em Лейбницевская
представляющая} сущность, {\em монада}, по существу
идеализованна. Представление есть некое для-себя-бытие, в котором
определенности суть не границы и, следовательно, не некоторое наличное
бытие, а лишь моменты. Представливание есть, правда, также и некое более
конкретное определение, но здесь оно не имеет никакого иного значения,
кроме значения идеальности, ибо и то, что вообще лишено сознания, есть у
Лейбница представляющее, перципирующее. В этой системе инобытие, стало
быть, снято; дух и тело или вообще монады суть не другие друг для друга,
они не ограничивают друг друга, не воздействуют друг на друга; здесь вообще
отпадают все те отношения, в основании которых лежит некоторое наличное
бытие. Многообразие есть лишь идеализованное и внутреннее, монады остаются
в нем лишь соотнесенными с самими собою, изменения развиваются внутри
монады и не суть соотношения последней с другими. То, что со стороны
реального определения берется нами как некоторое налично сущее соотношение
монад друг с другом, есть независимое, лишь
{\em одновременное} становление, заключенное в
для-себя-бытии каждой из них. — То обстоятельство, что существуют
{\em многие монады}, что их, следовательно, определяют
также и как другие, не касается самих монад; это "--- имеющее место вне них
размышление некоторого третьего; {\em в самих себе} они
не суть {\em другие по отношению друг к другу};
для-себя-бытие сохраняется чисто, без примеси некоторого находящегося
{\em рядом} существования. — Но тем самым явствует
вместе с тем и незавершенность этой системы. Монады суть представляющие
таким образом лишь {\em в себе} или
{\em в боге} как монаде монад, или
{\em также в системе}. Инобытие также имеется, где бы
оно ни имело место, в самом ли представлении, или как бы мы ни определили
то третье, которое рассматривает их как другие, как многие. Множественность
их существования лишь исключена и притом только на мгновение, монады лишь
путем абстрагирования положены как такие, которые суть не-другие. Если
некое третье полагает их инобытие, то некое третье также и снимает их
инобытие; но все это {\em движение, которое делает их
идеализованными}, совершается вне их. Однако так как нам могут напомнить о
том, что это движение мысли само имеет место лишь внутри некоторой
представляющей монады, то мы должны указать вместе с тем на то, что как раз
{\em содержание} такого мышления
{\em само в себе внешне себе}. Переход от единства
абсолютной идеальности (монады монад) к категории абстрактного (лишенного
соотношений) {\em множества} наличного бытия
совершается непосредственно, не путем постижения в понятии (совершается
посредством представления о сотворении), и обратный переход от этого
множества к тому единству совершается столь же абстрактно. Идеальность,
представливание вообще, остается чем-то формальным, равно как формальным
остается и то представливание, которое интенсифицировано до сознания. Как в
вышеприведенном замечании
Лейбница\pagenote{См. прим. \ref{magleibn} к стр. \pageref{bkm:bm31}.} о магнитной игле, которая, если
бы обладала сознанием, рассматривала бы свое направление к северу как
определение своей свободы, сознание мыслится лишь как односторонняя форма,
безразличная к своему определению и содержанию, так и идеальность в монадах
есть лишь некая остающаяся внешней для множественности форма. Идеальность,
согласно Лейбницу, имманентна им, их природа состоит в представливании; но
способ их поведения есть, с одной стороны, их гармония, не имеющая места в
их наличном бытии, — она поэтому предустановлена; с другой стороны, это их
{\em наличное бытие} не понимается Лейбницем ни как
бытие-для-другого, ни еще шире как идеальность, а определено лишь как
абстрактная множественность. Идеальность множественности и дальнейшее ее
определение в гармонию не имманентно самой этой множественности и не
принадлежит ей самой.

Другого рода идеализм, как, например, кантовский и фихтевский, не выходит за
пределы {\em долженствования} или
{\em бесконечного прогресса} и застревает в дуализме
наличного бытия и для-себя-бытия. В этих системах вещь-в-себе или
бесконечный толчок, правда, вступает непосредственно в «я» и становится
лишь неким «{\em для последнего}» (для «я»); однако
толчок этот исходит от некоторого свободного инобытия, которое пребывает во
веки веков как отрицательное в-себе-бытие. Поэтому «я», правда,
определяется в этого рода идеализме как идеализованное, как для-себя-сущее,
как бесконечное соотношение с собою; однако
{\em для-одного-бытие} не завершено до исчезновения
того потустороннего или направления в потустороннее.

\subsection[c) Одно]{c) Одно}
Для-себя-бытие есть простое единство самого себя и своего момента,
бытия-для-одного. Имеется лишь одно определение "--- свойственное снятию
соотношение с самим собою. {\em Моменты}
для-себя-бытия, слившись, погрузились в {\em отсутствие
различий}, которое есть непосредственность или бытие, но
{\em непосредственность}, основанная на отрицании,
положенном как ее определение. Для-себя-бытие есть, таким образом,
{\em для-себя-сущее}, и ввиду того, что в этой
непосредственности исчезает его внутреннее значение, оно есть совершенно
абстрактная граница самого себя "--- {\em одно}.

Можно здесь наперед обратить внимание читателя на ту трудность, которая
заключается в последующем изложении {\em развития}
одного, и на причину этой трудности. {\em Моменты},
составляющие {\em понятие} одного как для-себя-бытия, в
нем {\em разъединяются} (treten auseinander). Эти
моменты таковы: (1) отрицание вообще; (2) два отрицания, (3) стало быть,
отрицания двух, которые суть {\em одно и то же} и (4)
которые безоговорочно противоположны; (5) соотношение с собою, тождество
как таковое; (6) {\em отрицательное} соотношение и,
однако, с {\em самим собою}. Эти моменты здесь
разъединяются вследствие того, что в для-себя-бытии как сущем-для-себя
привходит форма {\em непосредственности},
{\em бытия}; благодаря этой непосредственности каждый
момент {\em полагается}, как
{\em некое особое (eigene) сущее определение}; и тем не
менее, они также и нераздельны. Приходится, следовательно, о каждом
определении высказывать также и ему противоположное; это-то противоречие
при абстрактном {\em характере моментов} и составляет
указанную трудность.

\section[B. Одно и многое]{B. Одно и многое}
Одно есть простое соотношение для-себя-бытия с самим собою, в каковом
соотношении моменты этого для-себя-бытия совпали, и потому в сказанном
соотношении для-себя-бытие имеет форму
{\em непосредственности}, и его моменты становятся
поэтому теперь {\em налично сущими}.

Как соотношение {\em отрицательного} с собою, одно есть
процесс определения, а как соотношение {\em с собою}
оно есть бесконечное {\em самоопределение}. Но
вследствие теперешней непосредственности эти
{\em различия} уже более не положены лишь как моменты
{\em одного и того же} самоопределения, а положены
вместе с тем также и как {\em сущие}.
{\em Идеальность} для-себя-бытия как тотальность
превращается, таким образом, во-первых, в
{\em реальность} и притом в наиабстрактнейшую,
наипрочнейшую, как {\em одно}. В
{\em одном} для-себя-бытие есть
{\em положенное} единство бытия и наличного бытия как
абсолютное соединение соотношения с другим и соотношения с собою; но кроме
того появляется также и определенность бытия в
{\em противоположность} определению
{\em бесконечного отрицания}, в противоположность
самоопределению, так что то, что одно есть {\em в
себе}, оно есть теперь только {\em в нем} и, стало
быть, отрицательное есть некое другое как отличное от него. То, что
обнаруживает себя {\em имеющимся} как отличное от него,
есть его собственное самоопределение; его единство с собою, взятое как
отличное от него, понижено до {\em соотношения} и, как
{\em отрицательное} единство, оно есть отрицание самого
себя как некоторого {\em другого},
{\em исключение} одного как некоторого
{\em другого} из себя, из одного.

\subsubsection[a) Одно в нём самом]{a) Одно в нём самом}
В нем самом одно вообще {\em есть}; это его бытие есть
не наличное бытие, не определенность как соотношение с другим, не характер;
оно есть состоявшееся отрицание этого круга категорий. Одно, следовательно,
не способно становиться другим; оно {\em неизменно}.

Оно неопределенно, однако уже более не таким образом, как бытие; его
неопределенность есть определенность, которая есть соотношение с самим
собою, абсолютная определенность; это
—~{\em положенное} внутри-себя-бытие.
{\em Как} то, что согласно своему понятию есть
соотносящееся с собою отрицание, оно имеет различие внутри себя "--- имеет
некоторое направление вовне, от себя к другому, каковое направление,
однако, непосредственно повернуто назад и возвратилось в себя, так как
согласно этому моменту самоопределения нет никакого другого, к которому оно
устремлялось бы.

В этой простой непосредственности исчезло опосредствование наличного бытия и
самой идеальности, исчезли, стало быть, всякие различия и всякое
многообразие. В нем нет {\em ничего}; это
{\em ничто}, абстракция соотношения с самим собою,
отлично здесь от самого внутри-себя-бытия; оно есть
{\em положенное} ничто, так как это внутри-себя-бытие
уже более не есть простое нечто, а имеет определением то, что оно как
опосредствование конкретно; ничто же как абстрактное, хотя и тождественно с
одним, разнится, однако, от его определения. Это ничто, положенное, таким
образом, как имеющее место {\em в одном}, есть ничто
как {\em пустота}. — Пустота есть таким образом
качество одного в его непосредственности.

\subsubsection[b) Одно и пустота]{b) Одно и пустота}
Одно есть пустота, как абстрактное соотношение отрицания с самим собою. Но
от простой непосредственности, от того бытия одного, которое также и
утвердительно, пустота как ничто безоговорочно разнится, а так как они
находятся в {\em одном} соотношении, а именно, в
соотношении самого одного, то их разница
{\em положена}. Но, разнствуя от сущего, ничто как
пустота находится {\em вне} сущего одного.

Для-себя-бытие, определяя себя, таким образом, как одно и пустоту, вновь
достигло некоторого {\em наличного бытия}. — Одно и
пустота имеют своей общей простой почвой отрицательное соотношение с собою.
Моменты для-себя-бытия выступают из этого единства, становятся внешними
себе; так как через {\em простое} единство моментов
привходит определение {\em бытия}, то оно (простое
единство) тем самым понижает само себя до {\em одной}
стороны и, следовательно, до наличного бытия, и тем самым его другое
определение, отрицание вообще, равным образом становится рядом как наличное
бытие [самого] ничто, как пустота.

\subsubsection[Примечание. Атомистика]{Примечание. Атомистика}

Одно в этой форме наличного бытия есть та ступень категории, которую мы
встречаем у древних как {\em атомистический принцип},
согласно которому сущность вещей составляют {\em атом}
и {\em пустота} ($\tau \acute{o}$
$\acute{\alpha} \tau o \mu o \nu $ или $\tau \acute{\alpha}$
$\acute{\alpha} \tau o \mu \alpha $ $\chi \alpha \iota $
$\tau \acute{o}$ $\chi \varepsilon \nu \acute{o} \nu $).
Абстракция, созревшая до этой формы, получила бóльшую определенность, чем
{\em бытие} Парменида и
{\em становление} Гераклита. Насколько
{\em высоко} поднимается эта абстракция, делая эту
простую определенность одного и пустоты принципом всех вещей, сводя
бесконечное многообразие мира к этой простой противоположности и
отваживаясь познать и объяснить его из нее, настолько же
{\em легко} для представляющего рефлектирования
представлять себе, что вот {\em здесь} находятся атомы,
а {\em рядом} с ними "--- пустота. Неудивительно поэтому,
что атомистический принцип сохранялся во все времена; такое же тривиальное
и внешнее отношение составности, которое должно еще прибавиться, чтобы была
достигнута видимость некоторого конкретного и некоторого многообразия,
столь же популярно, как и сами атомы и пустота. Одно и пустота есть
для-себя-бытие, наивысшее качественное внутри-себя-бытие, опустившееся до
полной {\em внешности}; непосредственность или бытие
одного ввиду того, что оно есть отрицание всякого инобытия, положено так,
чтобы не быть более определимым и изменчивым; для его абсолютной
неподатливости всякое определение, многообразие, всякая связь остается,
следовательно, всецело внешним соотношением.

У тех мыслителей, которые впервые выдвинули указанный атомистический
принцип, он, однако, не застрял в этом внешнем своем характере, а имел
помимо своего абстрактного еще и некоторое спекулятивное определение,
заключающееся в том, что {\em пустота} была ими познана
как {\em источник движения}, что является совершенно
другим отношением между атомами и пустотой, чем голая рядоположность этих
двух определений и их безразличие друг к другу. Утверждение, что пустота
есть источник движения, имеет не тот малозначительный смысл, что нечто
может вдвинуться лишь в пустоту, а не в уже наполненное пространство, так
как в последнем оно уже не находило бы открытого для него места; в этом
смысле пустота была бы лишь предпосылкой или условием, а не
{\em основанием} движения, равно как и само движение
предполагается при этом имеющимся налицо и забывается существенное "--- его
основание. Воззрение, согласно которому пустота составляет основание
движения, заключает в себе ту более глубокую мысль, что в отрицательном
вообще лежит основание становления, беспокойства самодвижения, причем,
однако, отрицательное следует понимать как истинную отрицательность
бесконечного. — Пустота есть {\em основание движения}
лишь как {\em отрицательное} соотношение одного со
своим {\em отрицательным}, с одним,~т.~е. с самим
собою, которое, однако, положено как налично сущее.

Но помимо этого спекулятивного смысла дальнейшие определения древних
относительно формы атомов, их положения, направления их движения довольно
произвольны и внешни; при этом они находятся в прямом противоречии с
основным определением атомов. Атомами, принципом величайшей внешности и,
следовательно, величайшего отсутствия понятия болеет физика в учении о
молекулах, частицах, равно как и та наука о государстве, которая исходит из
единичной воли индивидуумов.

\subsection[c) Многие одни "--- Отталкивание]{c) Многие одни "--- Отталкивание}
Одно и пустота составляют для-себя-бытие в его ближайшем наличном бытии.
Каждый из этих моментов имеет своим определением отрицание и вместе с тем
положен как некоторое наличное бытие. Взятые со стороны первого, одно и
пустота есть {\em соотношение} отрицания с отрицанием
как соотношение некоторого другого со своим другим; одно есть отрицание в
определении бытия, пустота "--- отрицание в определении небытия. Но одно есть
по существу лишь соотношение с собою, как соотносящее
{\em отрицание},~т.~е. оно само есть то, чем пустота
должна быть вне его. Но оба {\em положены} также и как
утвердительное {\em наличное бытие}, одно "--- как
для-себя-бытие как таковое, другое "--- как неопределенное наличное бытие
вообще, причем оба соотносятся друг с другом как с некоторым
{\em другим наличным бытием}. Для-себя-бытие одного
есть, однако, существенно идеальность наличного бытия и другого; оно
соотносится со своим другим не как с некоторым другим, а лишь как
{\em с собою}. Но так как для-себя-бытие фиксировано
как одно, как для-себя-{\em сущее}, как
{\em непосредственно} имеющееся налицо, то его
{\em отрицательное} соотношение с
{\em собою} есть вместе с тем соотношение с некоторым
{\em сущим}, а так как это соотношение также и
отрицательно, то то, с чем для-себя-бытие соотносится, остается
определенным как некоторое {\em наличное бытие} и
некоторое {\em другое}; как представляющее собою по
существу соотношение {\em с самим собою}, другое есть
не неопределенное отрицание как пустота, а есть равным образом
{\em одно}. Одно есть, следовательно,
{\em становление многими одними}.

Но, собственно говоря, это не становление, так как становление есть переход
{\em бытия} в {\em ничто};
напротив, {\em одно} становится лишь
{\em одним} же. Одно, соотнесенное, содержит в себе
отрицательное как соотношение и потому имеет это отрицательное
{\em в} нем самом. Вместо становления здесь,
следовательно, имеется, во-первых, собственное имманентное соотношение
одного; и, во-вторых, поскольку это соотношение есть отрицательное, а одно
есть вместе с тем сущее, постольку одно отталкивает само себя
{\em от себя}. Отрицательное соотношение одного с собою
есть, следовательно, {\em отталкивание}.

Это отталкивание как полагание {\em многих одних} через
само одно есть собственный выход одного вне себя, но выход, к таким лежащим
вне его, которые сами суть лишь одно. Это "--- отталкивание согласно
{\em понятию}, {\em в себе} сущее
отталкивание. Второе отталкивание отлично от этого и есть, прежде всего,
предносящееся представлению внешней рефлексии отталкивание не как
порождение многих одних, а лишь как взаимное неподпускание пред-положенных,
уже {\em имеющихся} одних. Следует затем посмотреть,
каким образом первое, {\em в себе} сущее отталкивание
определяет себя ко второму, внешнему.

Прежде всего следует установить, какими определениями обладают многие одни
как таковые. Становление многими или продуцированность многих
непосредственно исчезает как полагаемость; продуцированные суть одни не для
другого, а соотносятся бесконечно с самими собою. Одно отталкивает от себя
лишь само {\em себя}, оно, следовательно, не
становится, а {\em уже есть}. То, что мы представляем
себе как оттолкнутое, равным образом есть некоторое
{\em одно}, некоторое {\em сущее}.
Отталкивание и отталкиваемость принадлежат обоим одинаковым образом и не
составляют никакого различия между ними.

Одни суть, таким образом, {\em пред-положенные} в
отношении друг друга "--- {\em положенные} отталкиванием
одного от самого себя, {\em наперед} положенные как
{\em не} положенные; их положенность снята, они суть
{\em сущие} в отношении друг друга как соотносящиеся
лишь с собою.

Множественность представляется, стало быть, не неким
{\em инобытием}, а неким совершенно внешним одному
определением. Одно, отталкивая само себя, остается соотношением с самим
собою, как и то одно, которое принимается ближайшим образом за
отталкиваемое. Что одни суть {\em другие} в отношении
друг друга, что они объединены в определении множественности, не касается,
стало быть, одних. Если бы множественность была некоторым соотношением
самих одних друг с другом, то они взаимно ограничивали бы себя и имели бы в
самих себе утвердительно некоторое бытие-для-другого. Их соотношение "--- а
последнее они имеют благодаря их сущему {\em в себе}
единству, — как оно здесь {\em положено}, определено
как отсутствие всякого соотношения; оно есть опять-таки положенная ранее
{\em пустота}. Последняя есть их граница, но граница
внешняя им, в которой они не должны быть {\em друг для
друга}. Граница есть то, в чем ограничиваемые столь же
{\em суть}, сколь и {\em не суть};
но пустота определена как чистое небытие, и лишь это составляет их границу.

Отталкивание одного от самого себя есть раскрытие того, что одно есть в
себе, но бесконечность как {\em развернутая} есть здесь
{\em вышедшая вне себя бесконечность}; она вышла вне
себя вследствие непосредственности бесконечного, одного. Она есть столь же
некое простое соотношение одного с одним, сколь и, наоборот, абсолютное
отсутствие соотношений одного; она есть первое со стороны простого
утвердительного соотношения одного с собою; она есть последнее со стороны
того же соотношения как отрицательного. Или, иначе говоря, множественность
одного есть собственное полагание одного; одно есть не что иное, как
{\em отрицательное} соотношение одного с собою, и это
соотношение, стало быть, само одно, есть многие одни. Но вместе с тем
множественность безоговорочно внешня одному, ибо одно именно и есть снятие
инобытия, отталкивание есть его соотношение с собою и простое равенство с
самим собою. Множественность одних есть бесконечность как простодушно
производящее себя противоречие.

\subsubsection[Примечание. Лейбницевская монада]
{Примечание. Лейбницевская монада}

Мы упомянули выше о {\em лейбницевском идеализме}. Здесь
мы можем прибавить, что этот идеализм, исходя из мысли о
{\em представляющей монаде}, которую он определяет как
для-себя-сущую, дошел лишь до только что рассмотренного нами отталкивания,
и притом, лишь до {\em множественности} как таковой, в
которой каждый из одних есть лишь сам по себе, безразличен к наличному
бытию и для-себя-бытию других одних или, иначе говоря, других вообще нет
для одного. Монада есть сама по себе весь замкнутый мир; ни одна монада не
нуждается в других. Но это внутреннее многообразие, которым она обладает в
своем представлении, ничего не меняет в ее определении "--- быть для-себя.
Лейбницевский идеализм берет {\em множественность}
непосредственно как нечто {\em данное} и не постигает
ее как некое {\em отталкивание} монады; для него
поэтому множественность имеется лишь со стороны ее абстрактной внешности.
{\em Атомистика} не обладает понятием идеальности; она
понимает одно не как нечто такое, что имеет {\em в
самом себе} оба момента, момент для-себя-бытия и момент для-него-бытия,
понимает его, стало быть, не как идеализованное, а лишь как просто, сухо
для-себя-сущее. Но она идет дальше исключительно только безразличной
множественности; атомы вступают в дальнейшее определение в отношении друг
друга, хотя это происходит, собственно говоря, непоследовательно, между тем
как, напротив, в указанной безразличной независимости монад множественность
остается как неподвижное {\em основное определение},
так что их соотношение имеет место лишь в монаде монад или в
рассматривающем их философе.


\bigskip

\section[C. Отталкивание и притяжение]{C. Отталкивание и притяжение}
\subsection[a) Исключение одного]{a) Исключение одного}
Многие одни суть сущие; их наличное бытие или соотношение друг с другом есть
не-соотношение, оно им внешне; это "--- абстрактная пустота. Но они сами суть
это отрицательное соотношение с собою
лишь\pagenote{В немецком тексте
(как в издании Глокнера, так и в издании Лассона) вместо слова «nur» (лишь)
стоит «nun» (теперь). По-видимому, это опечатка.} как соотношение с
{\em сущими} другими; это "--- выше вскрытое противоречие,
бесконечность, положенная в непосредственности бытия. Тем самым
отталкивание непосредственно {\em преднаходит} то, что
им отталкивается. Взятое в этом определении, оно есть
{\em исключение}; одно отталкивает только им не порожденные, 
им не положенные многие одни. Это отталкивание как взаимное
или всестороннее, — относительно, ограничено бытием одних.

Множественность есть ближайшим образом неположенное инобытие; граница есть
лишь пустота, лишь то, в чем {\em нет} одних. Но они
{\em суть} также и в границе; они суть в пустоте или,
иначе говоря, их отталкивание есть их {\em общее
соотношение}.

Это взаимное отталкивание есть положенное {\em наличное
бытие} многих одних; оно не есть их для-себя-бытие, по которому они были бы
различены, как многое, лишь в некотором третьем, а их собственное,
сохраняющее их различие. — Они взаимно отрицают друг друга, полагают одно
другое как такое, которое есть лишь для {\em одного}.
Но они вместе с тем также и {\em отрицают}, что
{\em они суть лишь для-одного}; они
{\em отталкивают} эту свою
{\em идеальность} и {\em обладают
бытием}. — Таким образом, разлучены те моменты, которые в идеальности
безоговорочно соединены. Одно есть в своем для-себя-бытии также и
{\em для-одного}, но это одно, для которого оно есть,
есть само же оно; его различение от себя непосредственно снято. Но во
множественности различенное одно обладает бытием. Бытие-для-одного, как оно
определено в исключении, есть поэтому некоторое бытие-для-иного. Таким
образом, каждое из них отталкивается некоторым другим, снимается им и
превращается в такое одно, которое есть не для-себя, а для-одного, и притом
есть другое одно.

Для-себя-бытие многих одних оказывается согласно этому их самосохранением
благодаря опосредствованию их отталкивания в отношении друг друга, в
котором они взаимно снимают одно другое и полагают другие как некое голое
бытие-для-иного. Но вместе с тем оно (самосохранение) состоит в том,
чтобы отталкивать эту идеальность и полагать одни не сущими 
для-некоторого-иного. Но это самосохранение одних через их отрицательное
соотношение друг с другом есть, наоборот, их разложение.

Одни не только {\em суть}, но и сохраняют себя своим
взаимным исключением. Во-первых, то, благодаря чему они должны были бы
найти прочную опору их разнствования, защищающую их от того, чтобы
оказаться отрицаемыми, есть их {\em бытие} и притом их
{\em в-себе}{}-бытие, противостоящее их соотношению с
другим; это в-себе-бытие состоит в том, что они суть
{\em одни}. Но {\em все суть такое
одно}; вместо того чтобы иметь в своем в-себе-бытии твердую точку, на
которую опиралась бы их разница, они оказываются в нем
{\em одним и тем же}. Во-вторых, их наличное бытие и их
взаимоотношение,~т.~е. их {\em полагание самих себя как
одних}, есть взаимное отрицание; но это равным образом есть
{\em одно и то же} определение всех, которым они,
следовательно, полагают себя, наоборот, как тождественные, равно как и
благодаря тому, что они суть в себе одно и то же, их идеальность,
долженствующая быть положенной другими, есть {\em их
собственная} идеальность, которую они, стало быть, столь же мало
отталкивают. — Они суть, таким образом, по своему бытию и полаганию лишь
{\em одно} утвердительное единство.

Это рассмотрение одних, приводящее к заключению, что они по обоим своим
определениям "--- как поскольку они суть, так и поскольку они соотносятся друг
с другом "--- оказываются лишь одним и тем же и неразличимыми, есть лишь наше
сравнивание. — Но следует также посмотреть, что в их
{\em взаимоотношении положено} в них же самих. — Они
{\em суть}: эта предпосылка предполагается в указанном
взаимоотношении "--- и они суть лишь постольку, поскольку они взаимно отрицают
друг друга и вместе с тем не подпускают к самим себе этой своей
идеальности, своей отрицаемости,~т.~е. отрицают взаимное отрицание. Но они
суть лишь постольку, поскольку они отрицают: таким образом, когда
отрицается это их отрицание, отрицается также и их бытие. Правда, ввиду
того, что они {\em суть}, они этим отрицанием не
подверглись бы отрицанию, оно есть для них лишь нечто внешнее; это
отрицание другого отскакивает от них и попадает, коснувшись их, лишь в их
поверхность. Однако лишь благодаря отрицанию других одних они возвращаются
в самих себя; они имеют бытие лишь как это опосредствование: это их
возвращение есть их самосохранение и их для-себя-бытие. Так как их
отрицание не имеет никакого эффекта вследствие противодействия, которое
оказывают сущие как таковые или как отрицающие, то они не возвращаются в
себя, не сохраняют себя и не суть.

Выше мы уже выяснили, что одни суть одно и то же, каждое из них есть такое
же {\em одно}, как и другое. Это "--- не только наше
соотнесение, внешнее сведение вместе, а само отталкивание есть соотнесение;
одно, исключающее одни, соотносит само себя с ними, с одними,~т.~е. с самим
собою. Отрицательное отношение одних друг к другу есть, следовательно, лишь
некое {\em слияние с собою}. Это тождество, в которое
переходит их отталкивание, есть снятие той их разницы и внешности, которую
они как исключающие должны были, наоборот, отстоять друг против друга.

Это самополагание многих одних в единое одно
(dies sich in-Ein-Eines-setzen der vielen Eins)
есть {\em притяжение}.

\subsubsection[Примечание. Положение о единстве одного и многого]
{Примечание. Положение о единстве одного и многого}

Самостоятельность, доведенная до того последнего заострения, которое мы
видим в для-себя-сущем одном, есть абстрактная, формальная
самостоятельность, сама себя разрушающая; это "--- величайшее, упорнейшее
заблуждение, принимающее себя за высшую истину. В своих более конкретных
формах она выступает как абстрактная свобода, как чистое «я», а затем,
далее, как нравственное зло. Это "--- свобода, впавшая в такую ошибку, что
полагает свою сущность в этой абстракции и ласкает себя мыслью, будто в
этом замыкании в себя (Bei-sich-Sein) она обретает себя в чистом виде.
Говоря определеннее, эта самостоятельность есть заблуждение, заключающееся
в том, что смотрят как на отрицательное на то и относятся как к
отрицательному к тому, что есть ее собственная сущность. Она есть, таким
образом, отрицательное отношение к самой себе, которое, желая обрести
собственное бытие, разрушает его, и это его деяние представляет собою лишь
проявление ничтожества этого деяния. Примирение заключается в признании,
что то, против чего направлено отрицательное отношение, есть, наоборот, его
сущность, заключается лишь {\em в отказе} от
отрицательности {\em своего} для-себя-бытия, вместо
того чтобы крепко держаться за это последнее.

Древнее изречение гласит, что {\em одно} есть
{\em многое} и что в особенности
{\em многое есть одно}. По поводу этого изречения мы
должны повторить сделанное выше замечание, что истина одного и многого,
выраженная в предложениях, выступает в неадекватной форме, что эту истину
нужно понимать и выражать лишь как некое становление, как некий процесс,
отталкивание и притяжение, а не как бытие, взятое в предложении как
покоящееся единство. Выше мы упомянули и напомнили о диалектике
{\em Платона} в «Пармениде» касательно дедукции многого
из одного, а именно, из предложения, гласящего: одно есть. Внутренняя
диалектика понятия была нами указана; всего легче понимать диалектику
положения, гласящего, что {\em многое есть одно}, как
внешнюю рефлексию, и она имеет право быть здесь внешней, поскольку и
предмет, {\em многие}, есть то, что внешне друг другу.
Это сравнение многих между собою сразу дает тот результат, что одно всецело
определено лишь как другое; каждое есть одно, каждое есть одно из многих,
исключает другие, — так что они безоговорочно суть лишь одно и то же,
безоговорочно имеется налицо лишь {\em одно}
определение. Это "--- {\em факт}, и дело идет лишь о
понимании этого простого факта. Рассудок упрямо противится этому пониманию
лишь потому, что ему предносится, и притом правильно,
{\em также} и различие; но последнее так же не отпадает
вследствие сказанного факта, как и, обратно, этот факт не перестает
существовать, несмотря на различие. Можно было бы, следовательно, утешить
рассудок касательно его здравомысленного понимания факта различия, указав
ему, что и различие появится снова.

\subsection[b) Единое одно притяжения]{b) Единое одно притяжения}
Отталкивание есть саморасщепление одного ближайшим образом на многие,
отрицательное отношение которых бессильно, так как они предполагают друг
друга как сущие; оно есть лишь {\em долженствование}
идеальности; реализуется же последняя в притяжении. Отталкивание переходит
в притяжение, многие одни в единое одно. Эти два определения, отталкивание
и притяжение, ближайшим образом различаются, первое как реальность одних,
второе "--- как их положенная идеальность. Притяжение относится к отталкиванию
таким образом, что оно имеет последнее своей
{\em предпосылкой}. Отталкивание доставляет материю для
притяжения. Если бы не было никаких одних, то нечего было бы притягивать.
Представление о непрерывном притяжении, о непрерывном потреблении одних,
предполагает столь же непрерывное порождение одних; чувственное
представление о пространственном притяжении оставляет существовать поток
притягиваемых одних; вместо атомов, исчезающих в притягивающей точке,
выступает из пустоты другое множество атомов, и это, если угодно,
продолжается до бесконечности. Если бы притяжение было завершено,~т.~е.,
если бы мы представили себе, что многие приведены в точку единого одного,
то имелось бы лишь некое косное одно, уже не было бы более притяжения.
Налично сущая в притяжении идеальность заключает в себе еще также и
определение отрицания самой себя, те многие одни, соотношение с которыми
она составляет, и притяжение неотделимо от отталкивания.

Притяжение ближайшим образом равно присуще каждому из многих
{\em непосредственно} имеющихся одних; никакое из них
не имеет преимущества перед другим; в последнем случае имелось бы
равновесие в притяжении, собственно говоря, равновесие самих же притяжения
и отталкивания и косный покой без налично сущей идеальности. Но здесь не
может быть речи о преимуществе одного такого одного перед другим, что
предполагало бы некоторое определенное различие между ними, а, наоборот,
притяжение есть полагание имеющейся неразличности одних. Только само
притяжение есть впервые {\em полагание} некоего
отличного от других одного; они суть лишь непосредственные одни,
долженствующие сохранять себя через отталкивание; а через их положенное
отрицание возникает одно притяжения (das Eins der Attraktion), каковое одно
поэтому определено как опосредствованное, как
{\em одно, положенное как одно}. Первые одни как
непосредственные не возвращаются в своей идеальности назад в себя, а имеют
ее в некотором другом.

Но единое одно есть реализованная, положенная в одном идеальность; оно
притягивает через посредство отталкивания. Оно содержит это
опосредствование в самом себе {\em как свое
определение}. Оно, таким образом, не поглощает в себя притягиваемых одних
как в некоторую точку,~т.~е. оно не упраздняет их абстрактным образом. Так
как оно содержит в своем определении отталкивание, то последнее вместе с
тем сохраняет в нем одни как многие. Оно, так сказать, через свое
притяжение ставит нечто перед собою, приобретает некоторый объем или
наполнение. В нем, таким образом, есть вообще единство отталкивания и
притяжения.

\subsection[c) Соотношение отталкивания и притяжения]{c) Соотношение отталкивания и притяжения}
Различие между {\em одним} и
{\em многими} определилось в различие их
{\em соотношения} друг с другом, которое разложено на
два соотношения, на отталкивание и притяжение, каждое из коих сначала стоит
самостоятельно вне другого, но так, что они, однако, по существу связаны
вместе. Их еще неопределенное единство должно получить более определенные
очертания.

Отталкивание как основное определение одного выступает первым и
{\em непосредственным}, так же, как и его одни, хотя и
порожденные им, но вместе с тем положенные как непосредственные, и тем
самым оно выступает как безразличное к притяжению, которое привходит к
нему, как к такому пред-положенному, внешним образом. Напротив, притяжение
не предполагается отталкиванием, так что к полаганию его и его бытию первое
не должно быть причастно,~т.~е. отталкивание не есть уже в самом себе
отрицание самого себя, одни не суть уже в самих себе подвергшиеся
отрицанию. Таким образом, мы имеем отталкивание абстрактно, само по себе,
равно как и притяжение имеет по отношению к одним как
{\em сущим} аспект некоторого непосредственного
наличного бытия и привходит к ним спонтанно (von sich aus), как некоторое
другое.

Если мы согласно этому возьмем голое отталкивание так просто, само по себе,
то оно будет рассеянием многих одних в неопределенную даль, лежащую вне
сферы самого отталкивания, ибо оно состоит в отрицании соотношения многих
одних друг с другом; отсутствие соотношения есть его, взятого абстрактно,
определение. Но отталкивание не есть только пустота; одни, как не имеющие
соотношений, не отталкивают, не исключают, что составляет их определение.
Отталкивание есть, по существу хотя и отрицательное, но все же
{\em соотношение}; взаимное неподпускание и избегание
не есть освобождение от того, что не подпускается и чего избегают;
исключающее еще находится {\em в связи с тем}, что из
него исключается. Но этот момент соотношения есть притяжение,
следовательно, притяжение в самом отталкивании. Оно есть отрицание того
абстрактного отталкивания, по которому одни были бы лишь соотносящимися с
собою, сущими, а не исключающими.

Но поскольку исходным пунктом служило отталкивание налично сущих одних и,
стало быть, притяжение также положено приходящим к нему внешним образом, то
при всей их нераздельности они все же еще удерживались одно вне другого как
разные определения. Теперь, однако, оказалось, что не только отталкивание
предполагается притяжением, но что имеет место также и обратное соотношение
отталкивания с притяжением, и первое равным образом имеет свою предпосылку
в последнем.

Согласно этому определению они нераздельны и вместе с тем каждое из них
определено по отношению к другому как долженствование и предел. Их
долженствование есть их абстрактная определенность как
{\em сущих в себе}, которая, однако, вместе с тем
безоговорочно выпирается за себя и соотносится с другой определенностью и
таким образом каждое из них имеет бытие через посредство своего
{\em другого} как другого; их самостоятельность состоит
в том, что они в этом опосредствовании положены друг для друга как некий
другой процесс определения, отталкивание как полагание многих, притяжение
как полагание одного, притяжение, вместе с тем как отрицание многих, а
отталкивание как отрицание их идеальности в одном "--- состоит в том, что
также и притяжение есть притяжение лишь
{\em посредством} отталкивания, а отталкивание есть
отталкивание лишь посредством притяжения. Но что в этом процессе
определения опосредствование с собою через {\em другое}
на самом деле, наоборот, отрицается, и каждое из этих определений есть
опосредствование себя с самим собою, это вытекает из более близкого их
рассмотрения и приводит их обратно к единству их понятия.

Во-первых, что каждое предполагает {\em само себя},
соотносится в своей предпосылке лишь с собою, это уже подразумевается в 
состоянии (in dem Verhalten) пока что еще относительных отталкивания и
притяжения.

Относительное отталкивание есть взаимное неподпускание
{\em имеющихся налицо} многих одних, которые, как
предполагается, преднаходят друг друга как непосредственные. Но что имеются
многие одни, в этом ведь и состоит самое отталкивание; предпосылка, которую
оно якобы имеет, есть лишь его собственное полагание. Далее, определение
{\em бытия}, которое якобы принадлежит одним сверх
того, что они суть положенные, — определение, благодаря которому они
оказались бы {\em предшествующими}, равным образом
принадлежит отталкиванию. Отталкивание есть то, через что одни проявляют и
сохраняют себя как одни, то, через что они как таковые
{\em имеют бытие}. Их бытием и служит само
отталкивание; последнее, таким образом, не есть некое относительное к
некоторому другому наличному бытию, а относится всецело лишь к самому себе.

Притяжение есть полагание одного как такового, реального одного, в отношении
которого многие в их наличном бытии определяются, как лишь идеализованные и
исчезающие. Таким образом, притяжение сразу же предполагает само себя, а
именно, предполагает себя в том определении других одних, согласно которому
они суть идеализованные, тогда как помимо этого эти другие одни должны были
бы быть для-себя-сущими, а {\em для других},
следовательно, также и для какого бы то ни было притягивающего
—~отталкивающими. В противовес этому определению отталкивания они получают
идеальность не через отношение к притяжению, а она уже предпослана, есть
{\em в себе} сущая идеальность одних, так как они как
одни "--- включая и то одно, которое представляют себе как притягивающее "--- не
отличны друг от друга, суть одно и то же.

Это самопредпосылание обоих определений, каждого из них самого по себе,
означает, далее, то, что каждое из них содержит в себе другое как момент.
Самопредпосылание вообще есть в то же время полагание себя как
{\em отрицательного} себя (das Negative seiner), —
отталкивание; а то, что здесь предпосылается, есть
{\em то же самое}, что и предпосылающее, — притяжение.
Что каждое из них есть {\em в себе} лишь момент,
означает, что каждое из них спонтанно (aus sich selbst) переходит в
другое, отрицает себя в самом себе и полагает себя как другое самого себя.
Поскольку одно как таковое есть выхождение вне себя, поскольку оно само
состоит лишь в том, что оно полагает себя как другое, как множественное, а
множественное также состоит лишь в том, что оно сжимается в себя и полагает
себя как свое другое, как одно, и поскольку именно в этом они соотносятся
лишь с самими собою и каждое из них продолжает себя в своем другом,
постольку, следовательно, выхождение вне себя (отталкивание) и полагание
себя как одного (притяжение) уже в самих себе имеются нераздельными. Но в
относительных отталкивании и притяжении,~т.~е. в таком отталкивании и таком
притяжении, которое предполагает непосредственные,
{\em налично сущие} одни,
{\em положено}, что каждое из них есть это отрицание
себя в самом себе и тем самым также и продолжение себя в свое другое.
{\em Отталкивание} налично сущих одних есть
самосохранение одного путем взаимного неподпускания других, так что (1)
другие одни отрицаются {\em в нем} "--- это есть аспект
его наличного бытия или его бытия-для-другого; но этот аспект есть,
следовательно, притяжение как идеальность одних; и (2) одно есть
{\em в себе}, без соотношения с другими; но «в себе» не
только вообще давно уже перешло в для-себя-бытие, но и
{\em в себе}, по своему определению, одно есть
сказанное становление многими. — {\em Притяжение}
налично сущих одних есть их идеальность и полагание одного, в чем оно,
стало быть, как отрицание и продуцирование одного снимает само себя и как
полагание одного оказывается отрицанием самого себя в самом себе,
оказывается отталкиванием.

Этим развитие для-себя-бытия завершено и дошло до своего результата. Одно,
как соотносящееся {\em с самим собой бесконечным
образом},~т.~е., как положенное отрицание отрицания, есть опосредствование
в том смысле, что оно себя, как свое абсолютное (т.~е. абстрактное)
{\em инобытие} (т.~е.
{\em многие}), отталкивает от себя и, соотносясь с этим
своим небытием отрицательно, снимая его, именно в этом соотношении есть
лишь соотношение с самим собою; и одно есть лишь это становление, в котором
исчезло определение, что оно {\em начинается},~т.~е.
положено как непосредственное, сущее, и что оно равным образом и как
результат восстановило себя, сделавшись снова одним,~т.~е. таким же
{\em непосредственным}, исключающим одним; процесс,
который оно есть, повсюду полагает и содержит его в себе лишь как некоторое
снятое. Снятие, определившееся сначала лишь в относительное снятие, в
{\em соотношение} с другим налично сущим (каковое
соотношение, следовательно, само есть различное отталкивание и
притяжение), оказывается также и переходящим в бесконечное соотношение
опосредствования через отрицание внешних соотношений непосредственного и
налично сущего. При этом оно имеет своим результатом именно то становление,
которое ввиду неустойчивости его моментов есть опадание или, вернее,
слияние с собою, переход в простую непосредственность. Это бытие по тому
определению, которое оно теперь {\em получило},
{\em есть количество}.

Если обозреть вкратце моменты этого {\em перехода
качества в количество}, то окажется, что качественное имеет своим основным
определением бытие и непосредственность, в которой граница и определенность
так тождественны с бытием данного нечто, что вместе с их изменением
исчезает и само нечто; {\em положенное} таким образом,
оно определено как конечное. Вследствие непосредственности этого единства,
в котором {\em различие} исчезло, но в котором, как в
единстве {\em бытия} и {\em ничто},
оно в себе имеется, это различие как {\em инобытие}
вообще имеет место {\em вне} вышеупомянутого единства.
Это соотношение с другим противоречит непосредственности, в которой
качественная определенность есть соотношение с собою. Это инобытие
снимается в бесконечности для-себя-бытия, реализовавшего различие, которое
оно имеет в отрицании отрицания {\em у} и
{\em внутри} самого себя, сделав его одним и многим и
их соотношениями, и возведшего качественное в истинное, уже не
непосредственное, а положенное как согласующееся с собою, единство.

Это единство есть, стало быть, ($\alpha $) {\em бытие}
лишь как {\em утвердительное},~т.~е. как
опосредствованная с собою через отрицание отрицания
{\em непосредственность}; бытие положено как единство,
{\em проходящее сквозь} свои определенности, границы
и~т.~д., которые положены в нем как снятые. ($\beta $)
{\em Наличное бытие}; оно есть по такому определению
отрицание или определенность как момент утвердительного бытия; однако она
уже более не есть непосредственная, а есть рефлектированная в себя,
соотносящаяся не с иным, а с собою определенность, — безоговорочная
определенность, {\em в-себе}{}-определенность, — одно;
инобытие как таковое само есть для-себя-бытие. ($\gamma $)
{\em Для-себя-бытие} как то продолжающееся сквозь
определенность бытие, в котором одно и в-себе-определенность сами положены
как снятые. Одно вместе с тем определено как вышедшее за себя и как
{\em единство}; стало быть, одно,~т.~е. безоговорочно
определенная граница положена как граница, которая не есть граница, — как
граница, которая есть в бытии, но безразлична для него.


\subsubsection[Примечание. Кантовское построение материи из сил притяжения и отталкивания]
{Примечание. Кантовское построение материи из сил притяжения и отталкивания}

На притяжение и отталкивание, как известно, обыкновенно смотрят, как на
{\em силы}. Следует сравнить это их определение и
связанные с ним отношения с теми понятиями, которые у нас получились о них.
— В сказанном представлении они рассматриваются как самостоятельные, так
что они соотносятся друг с другом не по своей природе,~т.~е. каждое из них
не есть лишь переходящий в свою противоположность момент, а прочно остается
перед лицом другого тем же, что раньше. Их, далее, представляют себе, как
сходящиеся в некотором {\em третьем}, в
{\em материи}, но сходящиеся таким образом, что это их
схождение в одно (In-Eins-Werden) не считается их истиной, а каждое
признается неким первым и само по себе сущим, материя же или ее
определения "--- положенными и произведенными ими. Когда говорят, что материя
{\em обладает внутри себя} силами, то под этим их
единством разумеют некоторую их связь, причем они вместе с тем
предполагаются как сущие внутри себя и свободные друг от друга.

Как известно, {\em Кант} конструировал
{\em материю из сил отталкивания и притяжения}, или, по
крайней мере, как он выражается, дал метафизические элементы этой
конструкции. — Не безынтересно будет рассмотреть ближе эту конструкцию. Это
{\em метафизическое} изложение предмета, который, как
казалось, не только сам, но и в своих определениях принадлежит лишь области
{\em опыта}, замечательно отчасти тем, что оно как
попытка понятия дало, по крайней мере, толчок новейшей философии природы,
философии, которая не делает основой науки природу как нечто чувственно
данное восприятию, а познает ее определения из абсолютного понятия; отчасти
же оно интересно также и потому, что часто еще и теперь не идут дальше
кантовской конструкции и считают ее философским началом и основой физики.

Такого рода существование, как чувственная материя, не есть, правда, предмет
логики; она столь же мало является таковым, как и пространство и
пространственные определения. Но и в основе сил притяжения и отталкивания,
поскольку они понимаются как силы чувственной материи, лежат рассмотренные
здесь чистые определения одного и многих, равно как и их взаимоотношений,
которые я назвал отталкиванием и притяжением, потому что эти названия ближе
всего подходят.

Кантовский прием в дедукции материи из сказанных сил, который он называет
{\em конструкцией}, оказывается при более близком
рассмотрении не заслуживающим этого имени, если только не называть
конструкцией всякого рода рефлексию, хотя бы даже анализирующую, как и в
самом деле позднейшие натурфилософы называли
{\em конструкцией} также и самое плоское
рассуждательство и самую неосновательную смесь произвольного фантазирования
и лишенной мысли рефлексии "--- смесь, в которой в особенности пользовались,
протаскивая их повсюду, так называемыми факторами отталкивательной и
притягательной силы.

Прием Канта именно, в сущности говоря, {\em аналитичен},
а не конструктивен. Он уже {\em предполагает
представление материи} и затем спрашивает, какие требуются силы для того,
чтобы получить ее предполагаемые определения. Таким образом оказывается,
что он, с одной стороны, требует силы притяжения, так как при наличии
одного лишь отталкивания, без притяжения, не могло бы, собственно говоря,
быть никакой материи («Основные начала естествознания», стр. 53 и сл.).
Отталкивание же он, с другой стороны, также выводит из материи и указывает
в качестве его основания то обстоятельство, {\em что мы
представляем себе материю непроницаемой}, так как под таким именно
определением она являет себя {\em чувству осязания},
через которое, дескать, она нам открывается. Отталкивание потому нами сразу
же мыслится в {\em понятии} материи, что оно
(отталкивание) непосредственно дано вместе с ней, притяжение же, напротив,
мы прибавляем к ней посредством {\em умозаключений}. Но
и в основании этих умозаключений также лежит только что высказанное
соображение, что материя, которая обладала бы единственно лишь
отталкивательной силой, не исчерпывала бы того, что мы представляем себе
под материей. Совершенно очевидно, что здесь перед нами тот образ действия
рефлектирующего об опыте познания, который сначала
{\em воспринимает} в явлении те или другие определения,
кладет их затем в основание и принимает для так называемого
{\em объяснения} их соответствующие
{\em основные материи} или
{\em силы}, долженствующие произвести эти определения
явления.

Касательно указанного различия тех способов, какими познание находит в
материи силу отталкивания и силу притяжения, Кант замечает далее, что сила
притяжения все-таки тоже {\em принадлежит} к понятию
материи, {\em хотя она и не содержится в нем}. Кант
подчеркивает это последнее выражение. Но нельзя усмотреть, в чем тут
различие, ибо определение, принадлежащее к
{\em понятию} некоторой вещи, поистине
{\em необходимо должно содержаться в нем}.

Затруднение, заставляющее Канта прибегнуть к этой пустой уловке, состоит
здесь именно в том, что Кант с самого начала односторонне включает в
понятие материи единственно лишь определение
{\em непроницаемости}, которое мы согласно ему
{\em воспринимаем} посредством
{\em чувства осязания}, вследствие чего сила
отталкивания, как неподпускание некоторого другого к себе, дана-де
непосредственно. Но если далее говорится, что материя не может
{\em существовать} без притяжения, то в основании этого
утверждения лежит заимствованное из восприятия представление о материи;
определение притяжения должно, следовательно, равным образом встретиться
нам в восприятии. И мы действительно воспринимаем, что материя, кроме
своего для-себя-бытия, которое устраняет (aufhebt) бытие-для-другого
(оказывает сопротивление), обладает также и некоторым
{\em соотношением для-себя-сущих друг с другом},
пространственным {\em протяжением} и
{\em связностью} и в виде неподатливости (Starrheit),
твердости (Festigkeit) обладает очень прочной связностью. Объясняющая
физика требует для разрыва и~т.~д. тела наличия такой силы, которая
превосходила бы {\em притяжение} его частей друг к
другу. Из этого восприятия рефлексия может столь же непосредственно вывести
силу притяжения или принять ее как {\em данную}, как
она это сделала с силой отталкивания. И в самом деле, когда мы
рассматриваем те кантовские умозаключения, из которых согласно ему
выводится сила притяжения (доказательство теоремы, что возможность материи
требует силы притяжения как второй основной силы; там же), то мы
убеждаемся, что они не заключают в себе ничего другого, кроме того
соображения, что при одном только отталкивании материя не могла бы быть
{\em пространственной}. Так как материя предполагается
наполняющей пространство, то ей приписывается непрерывность, как основание
которой и принимается сила притяжения.

Хотя такая так называемая конструкция материи обладает в лучшем случае
аналитической заслугой, которая еще кроме того умаляется нечеткостью
изложения, мы все же должны признать весьма ценной основную мысль познать
материю из этих двух противоположных определений как из ее основных сил.
Кант старается главным образом об изгнании вульгарно-механических способов
представления, которые не идут дальше одного определения "--- непроницаемости,
{\em для-себя-сущей точечности}, и делают чем-то
{\em внешним} противоположное определение,
{\em соотношение} материи внутри себя или
{\em соотношение} друг с другом нескольких материй,
рассматриваемых в свою очередь как особенные одни, — об изгнании того
способа представления, который, как говорит Кант, не соглашается признать
никаких других движущих сил, кроме сил, движущих посредством давления и
толчка, следовательно, лишь посредством воздействия извне. Это носящее
{\em внешний} характер познание предполагает, что
движение как нечто внешнее для материи всегда уже
{\em имеется налицо}, и не помышляет о том, чтобы
понимать его как нечто внутреннее и постигать его в самой материи, которая
благодаря отсутствию такого понимания признается сама по себе неподвижной и
косной. Этой точке зрения предносится лишь обычная механика, а не
имманентное и свободное движение. — Хотя Кант устраняет сказанный внешний
характер, превращая притяжение ({\em соотношение}
материй друг с другом, поскольку эти материи принимаются отделенными друг
от друга, или {\em соотношение} материи вообще в ее
вне-себя-бытии) в {\em силу самой материи}, все же
принимаемые им две основные силы остаются, с другой стороны, внутри материи
внешними друг другу и, сами по себе, самостоятельными
{\em в отношении друг друга}.

Точно так же, как оказалось неосновательным то самостоятельное различие этих
двух сил, которое приписывается им с точки зрения указанного познания,
должно оказаться неосновательным и всякое другое различие, проводимое в
отношении их содержательного определения как нечто
{\em якобы неподвижное}, так как они, как они были
рассмотрены выше в их истине, суть лишь моменты, переходящие друг в друга.
— Теперь я рассмотрю эти дальнейшие различительные определения, как их
устанавливает Кант.

А именно, он определяет силу притяжения как
{\em проникающую} силу, благодаря которой одна материя
может {\em непосредственно} действовать на части другой
также и за пределами поверхности соприкосновения, отталкивательную же силу
он, напротив, определяет как {\em поверхностную} силу,
посредством которой материи могут действовать друг на друга только в общей
им поверхности соприкосновения. Довод, приводимый им в пользу того, что
отталкивание есть только поверхностная сила, гласит следующим образом:
«Каждая из {\em соприкасающихся} частей ограничивает
сферу действия другой, и отталкивательная сила не могла бы привести в
движение более отдаленную часть без посредства промежуточных частей;
проходящее поперек через них непосредственное действие одной материи на
другую посредством сил расширения (так называются здесь силы отталкивания)
невозможно» (см. там же, «Пояснения и добавления», стр. 67).

Мы должны сразу же напомнить о том, что, поскольку принимаются
{\em более близкие} или {\em более
отдаленные} части материи, постольку и {\em по
отношению к притяжению} равным образом возникает
{\em различие}: один атом, правда, действует на
{\em другой}, но {\em третий},
более отдаленный, между которым и первым, притягивающим, находится
{\em другой} атом, должен был бы сначала вступить в
сферу притяжения лежащего между ними, более близкого к нему атома, и первый
атом, следовательно, не мог бы оказывать на третий
{\em непосредственного} простого действия, из чего
вытекает, что действие силы притяжения есть такое же опосредствованное, как
и действие силы отталкивания. И далее: {\em истинное
проникание} силы притяжения должно было бы состоять только в том, что все
части материи {\em сами по себе} суть притягивающие, а
не в том, что известное их количество ведет себя пассивно и только один
атом активен. — Непосредственно же или, иначе говоря, по отношению к самой
силе отталкивания мы должны заметить, что в приведенной цитате говорится о
{\em соприкасающихся} частях и, следовательно, о
{\em сплошности} и
{\em непрерывности готовой} материи, не позволяющей
отталкиванию пройти через нее. Но эта компактность материи, в которой части
{\em соприкасаются} и уже не разделены более пустотой,
предполагает {\em устраненность} (Aufgehobensein)
{\em силы отталкивания}; соприкасающиеся части должны
быть признаны согласно господствующему здесь чувственному представлению об
отталкивании такими частями, которые не отталкивают друг друга. Из этого,
следовательно, вытекает совершенно тавтологически, что там, где мы
принимаем небытие отталкивания, отталкивание не может иметь места. Но из
этого ничего дальше не следует касательно определения силы отталкивания.
Если же мы еще подумаем о том, что соприкасающиеся части соприкасаются лишь
постольку, поскольку они еще держатся {\em вне друг
друга}, то мы убедимся, что сила отталкивания находится тем самым не только
на поверхности материи, но и внутри той сферы, которая якобы есть лишь
сфера притяжения.

Далее Кант принимает определение, что «через посредство силы притяжения
материя лишь {\em занимает некоторое пространство, не
наполняя его}» (там же); «так как материя не наполняет пространства
посредством силы притяжения, то последняя может действовать через
{\em пустое пространство}, ибо не имеется промежуточной
материи, которая ставила бы ей границы». — Это различие носит
приблизительно такой же характер, как вышеприведенное: там определение
принадлежит к понятию некоторой вещи, но не содержится в нем; здесь материя
лишь {\em занимает} некоторое пространство, но не
{\em наполняет} его. Там получается, что посредством
{\em отталкивания}, если мы остановимся на его первом
определении, нами одни отталкиваются и {\em соотносятся
друг с другом} лишь отрицательно, а именно {\em через
пустое пространство}. Здесь же получается, что как раз
{\em притягательная сила} сохраняет пространство
пустым; она {\em не наполняет} пространство посредством
своего соотнесения атомов,~т.~е. она удерживает атомы в
{\em отрицательном}
{\em соотношении} друг с другом. — Как видим, здесь
Кант, приписывая силе притяжения как раз то, что он согласно первому
определению приписывал противоположной силе, бессознательно натыкается на
то, что лежит в природе вещей. В процессе установления различия этих двух
сил получилось, что одна сила перешла в другую. — Так посредством
отталкивания материя, согласно Канту, {\em наполняет},
напротив, некоторое пространство и, следовательно, посредством него
исчезает то пустое пространство, которое сила притяжения оставляет
существовать. И в самом деле, отталкивание, устраняя пустое пространство,
тем самым устраняет отрицательное соотношение атомов или одних,~т.~е. их
отталкивание,~т.~е. отталкивание определено как противоположность самого
себя.

К этому стиранию различий присоединяется еще и та путаница, что, как мы уже
заметили вначале, кантовское изображение противоположных сил аналитично, и
во всем этом изложении материя, которая еще должна быть выведена из ее
элементов, уже выступает как готовая и конституированная. В дефиниции
поверхностной и проникающей сил обе принимаются как движущие силы,
посредством которых {\em материи} могут действовать тем
или иным образом. — Они, следовательно, изображаются здесь не как силы,
посредством которых материя впервые получает существование, а как такие
силы, посредством которых она, уже готовая, лишь приводится в движение. Но
поскольку речь идет о силах, посредством которых различные материи
воздействуют друг на друга и движут друг друга, это есть нечто совершенно
другое, чем то определение и то соотношение, которое они должны были иметь
как моменты материи.

Такую же противоположность, как силы притяжения и отталкивания, представляют
собою в дальнейшем определении
{\em центростремительная} и
{\em центробежная} силы. Сначала кажется, что эти силы
являют существенное различие, так как в их сфере имеется неподвижное единое
одно, центр, по отношению к которому другие одни ведут себя как не
для-себя-сущие, и мы можем поэтому приводить в связь различие указанных сил
с этим предполагаемым различием между центральным одним и другими одними,
которые неподвижны по отношению к этому центральному одному. Но поскольку
этими силами пользуются для объяснения, для каковой цели принимают, как
принималось прежде относительно сил отталкивания и притяжения, что они
находятся в обратном количественном отношении, так что одна увеличивается с
уменьшением другой, постольку явление движения, для объяснения которого
{\em они} принимаются, и его неравенство должны еще
только оказаться их результатом. Однако достаточно только вникнуть в первое
попавшееся изображение какого-нибудь явления, например, неровной скорости,
которой обладает планета на ее пути вокруг ее центрального тела, стоит
только вникнуть в объяснение этого явления противоположностью этих сил,
чтобы сразу увидеть господствующую здесь путаницу и невозможность
разъединить их величины, так что всегда приходится принимать возрастающей
также и ту силу, которая в объяснении принимается убывающей, и обратно.
Чтобы сделать сказанное наглядным, потребовалось бы более пространное
изложение, чем то, которое мы можем дать здесь, но все необходимое будет
дано далее, когда дойдем до изложения {\em обратного
отношения}.

\bigskip
