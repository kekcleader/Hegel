\chapter[{\em Вторая глава} Определенные Сущности или Определения Рефлексии]{Вторая глава. Определенные Сущности или Определения Рефлексии}
Рефлексия есть определенная рефлексия; тем самым
сущность есть определенная сущность —~Wesenheit.

Рефлексия есть {\em свечение сущности в себе самой}.
Сущность как бесконечное возвращение в себя есть не непосредственная, а
отрицательная простота; она есть движение через различенные моменты,
абсолютное опосредствование с собой. Но она светит в эти свои моменты; они
поэтому сами суть рефлектированные в себя определения.

Сущность есть, {\em во-первых}, простое соотношение с
собой самой, чистое {\em тождество}. Это есть то ее
определение, со стороны которого она скорее есть отсутствие определений.

{\em Во-вторых}, подлинным определением служит
{\em различие}, и притом отчасти как внешнее или
безразличное различие, {\em разность} вообще, отчасти
же как противоположная разность или как
{\em противоположность}.

{\em В-третьих}, как
{\em противоречие}, противоположность рефлектируется в
самое себя и переходит обратно в свое {\em основание}.


\subsubsection[Примечание Определения рефлексии в форме предложений]
{Примечание Определения рефлексии в форме предложений}

{\em Определения рефлексии} обыкновенно брались ранее в
{\em форме предложений}, в которых о них высказывалось,
что они {\em значимы относительно всего}. Эти
предложения считались {\em всеобщими законами
мышления}, которые, дескать, лежат в основании всякого мышления, в себе
самих абсолютны и недоказуемы, но признаются и принимаются непосредственно
и без возражения за истинные всяким мышлением, как только оно поймет их
смысл.

Так например, существенное определение {\em тождества}
получает выражение в предложении: «{\em все равно себе
самому}»; $А=А$. Или отрицательно: «$А$ не может
быть одновременно $А$ и не $-А$».

Прежде всего нельзя усмотреть, почему лишь эти простые определения рефлексии
должны быть облечены в эту особенную форму, а не также и другие категории,
как например все определенности сферы бытия. Тогда получились бы, например,
следующие предложения: всё {\em есть}, всё обладает
{\em наличным бытием} и~т.~д.; или: всё обладает
некоторым {\em качеством},
{\em количеством} и~т.~д. Ибо бытие, наличное бытие
и~т.~д., как логические определения, суть вообще предикаты
{\em всего}. Категория согласно этимологии этого слова
и согласно дефиниции, данной Аристотелем, есть то, что говорится,
утверждается о сущем. — Однако всякая определенность бытия есть по существу
переход в противоположное; отрицательное всякой определенности столь же
необходимо, как и она сама; как непосредственным определенностям, каждой из
них непосредственно противостоит другая. Поэтому если эти категории
облекаются в такие предложения, то появляются также и противоположные
предложения; и те и другие предложения выступают с одинаковой
необходимостью и, как непосредственные утверждения, по меньшей мере
одинаково правомерны. Одно предложение требовало бы тогда доказательства
своей истинности вопреки другому, и потому указанным утверждениям уже не
мог бы быть присущ характер непосредственно истинных и неопровержимых
законов (Sätze) мышления.

Напротив, определения рефлексии не имеют качественного характера. Они суть
определения, соотносящиеся с собой и тем самым вместе с тем не имеющие
определенности по отношению к другому. Далее, так как это —~такие
определенности, которые в самих себе суть
{\em соотношения}, то в них постольку уже содержится
форма предложения. Ибо предложение отличается от суждения главным образом
тем, что в первом {\em содержанием} служит само
{\em соотношение} или, иначе говоря, содержание есть
некое {\em определенное соотношение}. Напротив,
суждение помещает содержание в предикат, как некую всеобщую определенность,
которая стоит особо и отлична от своего соотношения, от простой связки.
Если нам нужно превратить предложение в суждение, то мы превращаем
определенное содержание, когда оно, например, заключается в каком-нибудь
глаголе, в причастие, чтобы таким образом отделить друг от друга само
определение и его соотношение с субъектом. Напротив, для рефлективного
определения как рефлектированной в себя положенности подходит форма самого
предложения. —Однако, так как они высказываются как
{\em всеобщие законы мышления}, то они нуждаются еще в
некотором субъекте своего соотношения, и этим субъектом служит «все» или
«$А$», которое означает то же самое, что и «всякое бытие».

С одной стороны, эта форма предложения есть нечто излишнее; рефлективные
определения должны быть рассмотрены сами по себе. Далее, в этих
предложениях есть та превратная сторона, что они имеют субъектом «бытие»,
«всякое нечто». Они этим снова возрождают бытие и высказывают рефлективные
определения —~тождество и~т.~д. — \ о некотором нечто, как имеющееся в нем
качество, они высказывают это не в спекулятивном смысле, но в том смысле,
что нечто как субъект остается в некотором таком качестве как
{\em сущее}, а не в том смысле, что оно перешло в
тождество и~т.~д. как в свою истину и свою сущность.

Наконец, хотя рефлективные определения и имеют форму равенства самим себе и
поэтому форму несоотнесенности с другим и свободы от противоположения, тем
не менее, как это выяснится из их ближайшего рассмотрения, — \ или, скажем
иначе, как это непосредственно явствует из них самих, как тождества,
различия, противоположения, — они суть
{\em определенные \ по отношению }друг к другу; они,
следовательно, не освобождены этой своей формой от рефлексии, перехода и
противоречия. Те {\em несколько предложений}, которые
устанавливаются как {\em абсолютные законы мышления},
оказываются поэтому при ближайшем рассмотрении
{\em противоположными друг другу}; они противоречат
друг другу и взаимно упраздняют одно другое. — Если все
{\em тождественно} с собой, то оно не
{\em разно}, не
{\em противоположно}, не имеет
{\em основания}. Или, если принимается, что нет
{\em двух одинаковых вещей}, т.~е. что все
{\em разнится} друг от друга, то $А$ не равно
$А$, то $А$ также и не противоположно и~т.~д. Принятие
каждого из этих предложений не допускает принятия других. — При чуждом
мысли рассмотрении этих предложений они просто перечисляются
{\em одно за другим}, так что они представляются не
имеющими никакого соотношения друг с другом; это рассмотрение имеет в виду
лишь их рефлектированность в себя, не принимая во внимание их другого
момента, {\em положенности} или их
{\em определенности} как таковой, которая увлекает их в
переход и в их отрицание.

\section[А. Тождество]{А. Тождество}
1. Сущность есть простая непосредственность, как
снятая непосредственность. Ее отрицательность есть ее бытие; она равна
самой себе в своей абсолютной отрицательности, в силу которой инобытие и
соотношение с другим сами в себе безоговорочно исчезли в чистое
саморавенство. Сущность есть, следовательно, простое
{\em тождество} с собой.

Это тождество с собой есть {\em непосредственность}
рефлексии. Оно есть не такое равенство с собой, как
{\em бытие} или также и
{\em ничто}, а то равенство с собой, которое
устанавливает себя к единству, причем это установление не есть
восстановление из некоторого другого, а чистое установление из себя самого
и в себе самом; {\em существенное} тождество. Постольку
оно не есть {\em абстрактное} тождество, или, иначе
говоря, не возникло через относительное подвергание отрицанию, происшедшее
вне его самого и лишь отделившее от него то, что от него отлично, в
остальном же оставившее это отличное по-прежнему, как нечто
{\em сущее}. Дело обстоит не так, а так, что бытие и
всякая определенность бытия сняли себя не относительно, а в самих себе, и
эта простая отрицательность, отрицательность бытия в себе, и есть само
тождество.

Постольку последнее есть еще вообще то же самое, что и сущность.


\subsubsection[Примечание 1 Абстрактное тождество]
{Примечание 1 Абстрактное тождество}

Мышление, держащееся в рамках внешней рефлексии и не знающее ни о каком
другом мышлении, кроме как о внешней рефлексии, не доходит до познания
тождества в том понимании, которое мы только что дали, или, что то же
самое, до познания сущности. Такое мышление всегда имеет в виду лишь
абстрактное тождество, и кроме него и рядом с ним —~различие. Оно полагает,
что разум есть не более, как ткацкий станок, на котором основа —~скажем,
тождество —~и уток —~различие —~внешним образом соединяются и переплетаются
между собой; или что разум, подвергая анализу получившуюся таким путем
ткань, сперва выделяет особо тождество, а {\em затем}
опять-таки сохраняет {\em рядом с ним} также и
различие, так что он представляет собой то установление равенства, то
{\em опять-таки} также и установление неравенства —~он
есть установление равенства, поскольку
{\em абстрагируются} от различия, и установление
неравенства, поскольку абстрагируются от установления равенства. — Нужно
оставить совершенно в стороне эти заверения и мнения о том, что делает
разум, так как они в некоторой мере лишь
{\em историчны}, и рассмотрение всего, что есть,
{\em в нем самом} показывает, наоборот, что оно в своем
равенстве с собой неравно себе и противоречиво, а в своем различии, в своем
противоречии тождественно с собой, и что в нем самом совершается это
движение перехода одного из этих определений в другое; и это —~так именно
потому, что каждое из них есть в самом себе противоположность самого себя.
Понятие тождества, заключающееся в том, чтобы быть простой соотносящейся с
собой отрицательностью, не есть продукт внешней рефлексии, а получилось на
самом бытии, тогда как, напротив, то тождество, которое находится вне
различия, и то различие, которое находится вне тождества, суть продукты
внешней рефлексии и абстракции, произвольно задерживающейся на этой точке
зрения безразличной разности.

2. Это тождество есть ближайшим образом сама сущность, а еще не ее
определение, есть вся рефлексия, а не различенный ее момент. Как абсолютное
отрицание оно есть то отрицание, которое непосредственно отрицает себя
само; некоторое небытие и различие, которое исчезает в своем возникновении,
или, иначе говоря, некоторое различение, которым ничего не различается и
которое непосредственно разрушается в себе самом. Различение есть полагание
небытия как небытия другого. Но небытие другого есть снятие другого и,
стало быть, самого различения. Но различение, таким образом, здесь имеется,
как соотносящаяся с собой отрицательность, как некое небытие, которое есть
небытие самого себя, есть такое небытие, которое имеет свое небытие не в
некотором другом, а в самом себе. Имеется, следовательно, соотносящееся с
собой, рефлектированное различие, или чистое,
{\em абсолютное различие}.

Или, иначе сказать, тождество есть рефлексия в себя самого, которая
представляет собою такую рефлексию в себя лишь как внутреннее отталкивание,
а это отталкивание есть отталкивание лишь как рефлексия в себя, — оно
(тождество) есть отталкивание, непосредственно вбирающее себя обратно в
себя. Оно тем самым есть тождество как тождественное с собой различие. Но
различие тождественно с собой лишь постольку, поскольку оно есть не
тождество, а абсолютное нетождество. Но нетождество абсолютно постольку,
поскольку оно не содержит в себе ничего из своего другого, а содержит
только само себя, т.~е. поскольку оно есть абсолютное тождество с собой.

Тождество есть, следовательно, {\em в себе самом}
абсолютное нетождество. Но оно есть также и
{\em определение} тождества в противоположность
нетождеству. Ибо как рефлексия в себя оно полагает себя как свое
собственное небытие; оно есть целое, но как рефлексия оно полагает себя,
как свой собственный момент, как положенность, возвращение из которой в
себя оно представляет собой. Лишь таким образом, лишь как момент, оно
впервые есть тождество как таковое, как
{\em определение} простого равенства с самим собой,
противостоящее абсолютному различию.


\subsubsection[Примечание 2 Первый первоначальный закон мышления: начало тождества]
{Примечание 2 Первый первоначальный закон мышления: начало тождества}

Я рассмотрю ближе в этом примечании тождество, как
{\em предложение о тождестве}, которое обыкновенно
приводится, как {\em первый закон мышления}.

Это предложение в его положительном выражении $A=A$ есть,
скажем прежде всего, не более, как выражение пустой
{\em тавтологии}. Было поэтому правильно замечено, что
этот закон мышления {\em бессодержателен} и никуда
далее не ведет. Таково то пустое тождество, за которое продолжают крепко
держаться те, которые принимают его, как таковое, за нечто истинное, и
всегда поучительно сообщают: тождество не есть разность, тождество и
разность разны. Они не видят, что уже этим они сами говорят, что
{\em тождество есть некоторая разность}; ибо они
говорят, что тождество {\em разнится} от разности; так
как вместе с тем они необходимо должны согласиться, что природа тождества
именно такова, то из этого вытекает, что тождество не внешним образом, а в
нем самом, в своей природе таково, что оно разно. — Но далее, так как они
крепко держатся за это неподвижное тождество, имеющее свою
противоположность в разности, то они не видят, что они тем самым делают его
односторонней определенностью, которая, как таковая, не имеет истинности.
Они соглашаются, что предложение о тождестве выражает лишь одностороннюю
определенность, содержит в себе лишь {\em формальную},
некую {\em абстрактную},
{\em неполную истину}. — Но из этого правильного
суждения непосредственно вытекает, что {\em истина
достигает полноты лишь в единстве тождества с разностью} и тем самым
состоит только в этом единстве. Так как они утверждают, что указанное
неподвижное тождество несовершенно, то это означает, что их мысли
предносится, как нечто совершенное, та целостность, в сравнении с которой
тождество несовершенно. Но так как, с другой стороны, тождество
фиксируется, как абсолютно отделенное от разности, и принимается в этой
отделенности за некое существенное, значимое, истинное, то в этих
сталкивающихся утверждениях нельзя усмотреть ничего другого, кроме
неспособности свести вместе эти две мысли —~мысль о том, что тождество, как
абстрактное тождество, существенно, и мысль о том, что оно, как таковое,
также и несовершенно, — нельзя усмотреть ничего другого, кроме отсутствия
сознания о том отрицательном движении, каковым в этих утверждениях
изображается само тождество. — Или, иначе говоря, поскольку они выражаются
так, что тождество есть {\em существенное тождество}
как {\em отделенность} от разности или в
{\em отделенности от разности}, то это выражение
непосредственно представляет собой высказанную истину его (тождества), а
именно, что оно состоит в том, чтобы быть
{\em отделенностью} как таковой, или иметь бытие по
существу в {\em отделенности}, т.~е., что оно
{\em есть не нечто самостоятельное} (für sich), а
{\em момент отделенности}.

Что же касается дальнейшего удостоверения абсолютной
{\em истинности предложения} о тождестве, то это
удостоверение основывается на {\em опыте} постольку,
поскольку ссылаются на опыт каждого сознания, которое, дескать, как только
ему высказывают это предложение $-А$ есть $А$,
{\em дерево есть дерево}, — непосредственно соглашается
с ним и удовлетворяется тем, что это предложение, как непосредственно ясное
само по себе, не нуждается ни в каком другом обосновании и доказательстве.

С одной стороны, эта ссылка на опыт, что-де всякое сознание всегда признает
это предложение, есть просто фраза. Ибо этим конечно не хотят сказать, что
проделали эксперимент с абстрактным предложением $А=А$,
обращаясь к каждому сознанию. Постольку эта ссылка на действительно
произведенный опыт несерьезна, и она есть лишь
{\em уверение}, что если бы произвели этот опыт, то его
результатом оказалось бы всеобщее признание. — Если же оказалось бы, что
имеют в виду не абстрактное предложение как таковое, а это предложение в
{\em конкретном применении}, из которого абстрактное
предложение должно еще быть {\em развито}, то указанное
утверждение о его всеобщности и непосредственности состояло бы в том, что
всякое сознание и притом в каждом своем высказывании
{\em кладет} его в {\em основание}
или, иначе говоря, что оно {\em скрыто} содержится в
каждом высказывании. Однако {\em конкретное} и
{\em применение} ведь именно и состоит в
{\em соотношении} простого
{\em тождественного} с некоторым
{\em отличным} от него
{\em многообразием}. Выраженное, как предложение,
конкретное было бы ближайшим образом синтетическим предложением. Из самого
конкретного или из выражающего его синтетического предложения абстракция
могла бы, правда, добыть посредством анализа предложение о тождестве; но на
самом деле она не оставила бы {\em опыт} таким, каков
он есть, а {\em изменила} бы его; ибо
{\em опыт}, наоборот, содержал в себе тождество в
единстве с разностью и есть {\em непосредственное
опровержение} утверждения, будто абстрактное тождество, как таковое, есть
нечто истинное, ибо во всяком опыте мы встречаем прямую противоположность
этому, а именно, тождество, лишь соединенное с разностью.

Но, с другой стороны, опыт с чистым предложением о тождестве проделывается
даже весьма и весьма, часто, и в этом опыте достаточно ясно обнаруживается,
как смотрят на истину, которую оно в себе содержит. А именно, если
например, на вопрос: {\em что такое растение?} дают
ответ: {\em растение есть растение}, то все общество,
на котором испытывается истинность такого рода предложения, одновременно и
признает ее и столь же единогласно заявляет, что этим
{\em ничего} не сказано. Если кто-нибудь открывает рот
и обещает указать, что такое бог, а затем говорит: бог есть бог, то
слушатели оказываются обманутыми в своих ожиданиях, ибо они надеялись
услышать некоторое {\em разнящееся определение}; и если
это предложение есть абсолютная истина, то нужно сказать, что такая
абсолютная болтовня ценится весьма низко; ничто не считается более скучным
и несносным, чем беседа, пережевывающая лишь одно и то же, — чем такого
рода речь, которая, однако, якобы есть истина.

При более близком рассмотрении этой скуки, вызываемой такой истиной, мы
видим, что начало: «{\em растение есть}», делает
приготовления к тому, чтобы {\em что-то} сказать,
сообщить какое-то дальнейшее определение. Но так как затем лишь повторяется
то же самое, то произошло, наоборот, нечто противоположное этим
приготовлениям: {\em ничего} не было сказано. Такая
{\em тождественная} речь
{\em противоречит}, следовательно,
{\em самой себе}. Тождество, вместо того, чтобы быть в
самом себе истиной и абсолютной истиной, есть, поэтому, наоборот, нечто
противоположное; вместо того чтобы быть неподвижным простым, оно есть
выхождение вне себя к разложению самого себя.

{\em В той форме предложения}, в которой выражено
тождество, заключается, следовательно, {\em более} чем
простое, абстрактное тождество. В ней заключается то чистое движение
рефлексии, в котором другое выступает лишь как видимость, как
непосредственное исчезание. «$А$ есть» представляет собою начало,
при котором уму предносится некое разное, к которому должен быть совершен
выход; но до разного не доходят; $А$ есть $А$, разность есть
лишь исчезание; движение возвращается в само себя. — На форму предложения
можно смотреть, как на скрытую необходимость присоединить к абстрактному
тождеству еще и тот прибавок, который состоит в указанном движении. — Таким
образом, присоединяется также некоторое $А$ или растение или
какой-нибудь другой субстрат, который, как бесполезное содержание, не имеет
никакого значения; но это содержание составляет ту разность, которая —~так
кажется —~случайно сюда присоединилась. Если вместо $А$ и всякого
другого субстрата берут само тождество —~«тождество есть тождество», — то и
в этом случае признано, что вместо него можно равным образом брать всякий
другой субстрат. Поэтому, если уже ссылаться на то, что показывает явление,
то оно показывает, что в выражении тождества непосредственно встречается
также и разность, — или выразим его, согласно предыдущему, определеннее:
оно показывает, что это тождество есть ничто, что оно есть отрицательность,
абсолютное различие от самого себя.

Другое выражение начала тождества: {\em $А$ не может быть
одновременно $А$ и не $-А$}, имеет отрицательную форму; оно называется
{\em началом противоречия}. Обычно не дают никакого
оправдания относительно того, каким образом присоединяется к тождеству
{\em форма отрицания}, которой это предложение
отличается от предыдущего. — Но эта форма получается оттого, что тождество,
как чистое движение рефлексии, есть простая отрицательность, которую
приведенное второе выражение предложения содержит в себе в более развитом
виде. Высказывается $А$ и не $-А$, чисто другое того
$А$; но это не $-А$ появляется только для того, чтобы
исчезнуть. Тождество, следовательно, выражено в этом предложении как
отрицание отрицания. $А$ и не $-А$ различны; эти различные
соотнесены с одним и тем же $А$. Тождество, следовательно,
изображается здесь как {\em эта различенность в одном
соотношении} или как {\em простое различие в них же
самих}.

Из этого явствует, что само начало тождества, а еще больше —~начало
противоречия, имеют не только {\em аналитическую}, но и
{\em синтетическую} природу. Ибо последнее содержит в
своем выражении не только пустое, простое равенство с собой и не только
{\em вообще другое} этого равенства, но даже
{\em абсолютное неравенство},
{\em противоречие в себе}. Само же начало тождества
содержит в себе, как мы обнаружили относительно него, рефлективное
движение, тождество, как исчезание инобытия.

Вывод, получающийся из этого рассмотрения, заключается, стало быть, в том,
что, {\em во-первых}, начало тождества или
противоречия, взятое в том смысле, что оно должно выражать как истину, лишь
абстрактное тождество в противоположность различию, не есть закон мышления,
а есть, наоборот, противоположность такого закона; и что,
{\em во-вторых}, эти начала содержат в себе более, чем
ими {\em хотят сказать}, а именно, эту
противоположность, само абсолютное различие.

\section[В. Различие]{В. Различие}
\subsection[1. Абсолютное различие]{1. Абсолютное различие}
Различие есть та отрицательность, которая
присуща рефлексии в себя; ничто, высказываемое посредством тождественной
речи; существенный момент самого тождества, которое в одно и то же время
определяет себя как отрицательность самого себя и различено от различия.

1. Это различие есть различие {\em в себе и для себя},
{\em абсолютное} различие,
{\em различие сущности}. — Оно есть различие в себе и
для себя, не различие через нечто внешнее, а
{\em соотносящееся с собой} и, следовательно,
{\em простое }различие. — Существенно важно понимать
абсолютное различие как {\em простое}. В абсолютном
различии между $А$ и не $-А$ именно
{\em простое «не»}, как таковое, и составляет это
различие. Самое различие есть простое понятие. Две вещи, как говорится,
различны между собою {\em тем}, что они и~т.~д.
«{\em Тем}» это означает: в одном и том же отношении,
на почве того же самого определения. Это различие есть
{\em различие рефлексии}, а не
{\em инобытие наличного бытия}. Одно наличное бытие и
другое наличное бытие положены, как находящиеся друг вне друга; каждое из
определенных одно но отношению к другому наличных бытий обладает
самостоятельным {\em непосредственным бытием}.
Напротив, {\em другое сущности} есть другое в себе и
для себя, а не другое, как другое некоторого находящегося вне его другого,
простая определенность в себе. Равным образом в сфере наличного бытия
инобытие и определенность оказались по своей природе простой
определенностью, тождественной противоположностью; но это тождество
оказалось лишь {\em переходом} одной определенности в
другую. Здесь же, в сфере рефлексии, различие выступает, как
рефлектированное различие, положенное так, как оно есть в себе.

2. Различие в себе есть соотносящееся с собою различие; таким образом, оно
есть отрицательность самого себя, различие не от некоторого другого,
{\em а себя от самого себя}; оно есть не оно само, а
свое другое. Различное же от различия есть тождество. Различие,
следовательно, есть само же оно и тождество. Оба вместе составляют
различие; оно есть целое и его момент. — Можно также сказать, что различие
как простое не есть различие; оно впервые таково лишь в соотношении с
тождеством; но вернее будет сказать, что оно как различие содержит в себе и
тождество и само это соотношение. — Различие есть целое и свой собственный
{\em момент}; и тождество точно так же есть свое целое
и свой момент. — Это должно быть рассматриваемо, как существенная природа
рефлексии и как {\em определенная первооснова всякой
деятельности и самодвижения}. — Различие, равно как и тождество, делают
себя {\em моментами} или
{\em положенностью}, потому что они как рефлексия суть
отрицательное соотношение с самим собою.

Различие, взятое, таким образом, как единство себя и тождества, есть
{\em в самом себе определенное} различие. Оно не есть
переход в некоторое другое, не есть соотношение с другим, находящимся вне
его; оно имеет свое другое, тождество, в себе самом, точно так же как и
тождество, вступив в определение различия, не потеряло себя в нем как в
своем другом, а сохраняется в нем, есть его рефлексия в себя и его момент.

3. Различие имеет оба момента, тождество и различие; оба суть, таким
образом, некоторая {\em положенность}, определенность.
Но в этой положенности каждый момент есть
{\em соотношение с самим собою}. Один момент,
тождество, сам, есть непосредственно момент рефлексии в себя; но точно так
же и другой момент, различие, есть различие в себе, рефлектированное
различие. Различие, поскольку оно имеет два таких момента, которые сами
суть рефлексии в себя, есть {\em разность}.

\subsection[2. Разность]{2. Разность}
1. Тождество {\em распадается} в самом себе на разность, так как
оно, как абсолютное различие в самом себе, полагает себя как отрицательное
себя, и эти его моменты, само оно и отрицательное его, суть рефлексии в
себя, тождественны с собой; или, иначе говоря, именно потому, что оно
непосредственно само снимает свой процесс отрицания и в своем
{\em определении рефлектировано в себя}. Различное
{\em пребывает} (besteht) как безразличное друг к другу
разное, так как оно тождественно с собою, так как тождество составляет его
почву и стихию; или, иначе говоря, разное есть то, что оно есть, именно
лишь в своей противоположности, в тождестве.

Разность составляет инобытие как инобытие рефлексии. Другое наличного бытия
имеет непосредственное {\em бытие} своим основанием, в
котором пребывает отрицательное. В рефлексии же тождество с собою,
рефлектированная непосредственность, составляет пребывание отрицательного и
его безразличие.

Моментами различия служат тождество и само различие. Они разные, как
рефлектированные в самих себя, {\em соотносящиеся с
собою}; таким образом, {\em в определении тождества}
они суть соотношения лишь с собой; тождество не соотнесено с различием и
различие не соотнесено с тождеством; поскольку, таким образом, каждый из
этих моментов соотнесен лишь с собой, они {\em не
определены} по отношению друг в другу. — Так как они, таким образом, не
суть различные в себе же самих, то {\em различие} им
{\em внешне}. Разные, следовательно, относятся друг к
другу, не как тождество и различие, а лишь как
{\em разные} вообще, которые безразличны по отношению
друг к другу и к своей определенности.

2. В разности как безразличии различия {\em рефлексия}
стала вообще {\em внешней} себе; различие есть лишь
некоторая {\em положенность} или снятое различие, но
оно само есть вся рефлексия. — При ближайшем рассмотрении оказывается, что
оба, тождество и различие, как только что определилось, суть рефлексии;
каждое из них есть единство самого себя и своего другого; каждое из них
есть целое. Но тем самым определенность, заключающаяся в том, что каждое из
них есть {\em только} тождество или
{\em только} различие, снята. Они не суть качества,
потому что их определенность через рефлексию в себя дана (ist) вместе с тем
только как отрицание. Имеется, следовательно, такое двоякое:
{\em рефлексия в себя}, как таковая, и определенность
как отрицание или {\em положенность}. Положенность есть
внешняя себе рефлексия; она есть отрицание как отрицание; следовательно,
{\em в себе} она, правда, есть соотносящееся с собой
отрицание и рефлексия в себя, но лишь в себе; она есть соотношение с собой
как с чем-то внешним.

Рефлексия в себе и внешняя рефлексия суть тем самым те два определения, в
виде которых положили себя моменты различия, тождество и различие. Они суть
самые эти моменты, поскольку они теперь определились. —
{\em Рефлексия в себе} есть тождество, но определенное
так, что оно безразлично к различию, т.~е. не совершенно лишено различия, а
относится к нему, как тождественное с собою; она есть
{\em разность}. Тождество так рефлектировалось в себя,
что оно, собственно говоря, есть {\em единая} рефлексия
в себя обоих моментов; оба суть рефлексии в себя. Тождество есть эта единая
рефлексия обоих, которая содержит в себе различие, лишь как безразличное
различие, и которая есть разность вообще. — Напротив,
{\em внешняя рефлексия} есть их
{\em определенное} различие, не как абсолютная
рефлексия в себя, а как такое определение, к которому сущая в себе
рефлексия безразлична; оба его момента, тождество и само различие, суть,
таким образом, внешне положенные, а не в себе и для себя сущие определения.

Это-то внешнее тождество есть {\em одинаковость}, а
внешнее различие —~{\em неодинаковость}.—
{\em Одинаковость} есть, правда, тождество, но лишь как
некоторая положенность, тождество, которое не есть в себе и для себя. —
Точно так же {\em неодинаковость} есть, правда,
различие, но как некоторое внешнее различие, которое не есть в себе и для
себя различие самого неодинакового. Одинаково ли какое-нибудь нечто с
другим нечто или нет, — это не касается ни того, ни другого нечто; каждое
из них соотнесено лишь с собою, есть само по себе то, что оно есть.
Тождество или нетождество, как одинаковость и неодинаковость, есть
соображение некоторого третьего, имеющее место вне их.

3. Внешняя рефлексия соотносит разное с одинаковостью и неодинаковостью. Это
соотнесение, {\em сравнение}, переходит туда и обратно
от одинаковости к неодинаковости и от последней к первой. Но это
перемежающееся соотнесение одинаковости и неодинаковости внешне самим этим
определениям; да их и соотносят не друг с другом, а каждое самое по себе
лишь с некоторым третьим. В этом чередовании каждая выступает
непосредственно сама по себе. — Внешняя рефлексия как таковая внешня самой
себе; определенное различие есть подвергшееся отрицанию абсолютное
различие; оно, стало быть, не просто, не есть рефлексия в себя, а имеет
последнюю вне себя; его моменты поэтому распадаются и соотносятся с
противостоящей им рефлексией в себя тоже как внешние друг другу.

В отчужденной от себя рефлексии одинаковость и неодинаковость появляются,
стало быть, как такие определения, которые сами не соотнесены друг с
другом, и она {\em разлучает} их, соотнося их с
{\em одним и тем же} посредством выражений
«{\em постольку}», «{\em с этой
стороны}» и «{\em в таком-то отношении}».
Следовательно, разные, представляющие собой то одно и то же, с чем
соотносят оба определения —~одинаковость и неодинаковость, —
{\em с одной стороны} одинаковы между собой, а
{\em с другой стороны} неодинаковы и,
{\em поскольку} они одинаковы,
{\em постольку} они не неодинаковы.
{\em Одинаковость} соотносится лишь с собою, и
{\em неодинаковость} есть также лишь неодинаковость.

Но этим своим отделением друг от друга они только упраздняются. Как раз то,
что должно было не подпускать к ним противоречие и разложение, а именно то
обстоятельство, что нечто в {\em одном отношении
одинаково} с каким-либо другим нечто, {\em в другом же
}{\em отношении неодинаково с ним}, — как раз это
недопускание соединения одинаковости и неодинаковости есть их разрушение.
Ибо оба они суть определения различия; они суть соотношения друг с другом,
соотношения, состоящие в том, что одно есть то, что другое не есть;
одинаковое не есть неодинаковое, и неодинаковое не есть одинаковое; обоим
им существенно это соотношение, и вне его они не имеют никакого значения;
как определения различия каждое из них есть то, что оно есть, как
{\em различенное} от своего другого. Но в силу их
безразличия друг к другу одинаковость соотнесена лишь с собою и
неодинаковость равным образом есть сама по себе свое собственное «в
таком-то отношении» (eigene Rücksicht) и самостоятельная рефлексия; каждая,
стало быть, одинакова с самой собою; различие исчезло, так как они не имеют
никакой определенности по отношению друг к другу; или, иначе говоря, каждая
есть тем самым только одинаковость.

Это безразличное «в таком-то отношении» (Rücksicht) или внешнее различие
упраздняет, стало быть, само себя и есть своя отрицательность в самом себе.
Внешнее различие есть та отрицательность, которая в процессе сравнения
принадлежит сравнивающему. Сравнивающее переходит от одинаковости к
неодинаковости и от этой последней обратно к первой, заставляет,
следовательно, одну исчезать в другой и есть на деле
{\em отрицательное единство обоих}. Это единство
находится ближайшим образом по ту сторону сравниваемого, равно как и по ту
сторону моментов сравнения, как некоторое субъективное, совершающееся вне
их действие. Но это отрицательное единство, как оказалось, есть на самом
деле природа самих одинаковости и неодинаковости. Как раз то
самостоятельное «в таком-то отношении» (Rücksicht), каким оказывается
каждая из них, и есть, наоборот, их соотношение с собою, упраздняющее их
различность и, стало быть, их самих.

С этой стороны одинаковость и неодинаковость, как моменты внешней рефлексии
и как внешние самим себе, исчезают вместе в свою одинаковость. Но это их
{\em отрицательное} единство, далее, также и
{\em положено} в них; а именно, они имеют
{\em сущую в себе} рефлексию вне их, или, иначе говоря,
суть одинаковость и неодинаковость {\em некоторого
третьего}, некоторого другого, чем они сами. Таким образом, одинаковое не
есть одинаковое с самим собой; и неодинаковое, как неодинаковое не с самим
собой, а с некоторым неодинаковым с ним, само есть одинаковое. Одинаковое и
неодинаковое есть, следовательно, {\em неодинаковое с
самим собой}. Каждое из них есть, стало быть, рефлексия, заключающаяся в
том, что одинаковость есть и она сама, и неодинаковость, а неодинаковость
есть и она сама, и одинаковость.

Одинаковость и неодинаковость составляли сторону
{\em положенности} по отношению к сравниваемому или
разному, которое определилось по отношению к ним, как
{\em в-себе-сущая рефлексия}. Но это разное тем самым
также потеряло свою определенность по отношению к ним. Как раз одинаковость
и неодинаковость, определения внешней рефлексии, суть та лишь в-себе-сущая
рефлексия, которой должно было быть разное как таковое, его (разного) лишь
неопределенное различие. {\em В-себе-сущая} рефлексия
есть соотношение с собой без отрицания, абстрактное тождество с собой и
стало быть как раз сама положенность. — То, что только разно, переходит,
следовательно, в силу положенности в отрицательную рефлексию. Разное есть
такое различие, которое только положено, следовательно, различие, которое
не есть различие, следовательно, отрицание себя в самом себе. Таким
образом, сами одинаковость и неодинаковость, положенность, возвращаются
через безразличие или в-себе-сущую рефлексию обратно в отрицательное
единство с собой, в такую рефлексию, которая в самой себе есть различие
одинаковости и неодинаковости. Разность,
{\em безразличные} стороны которой вместе с тем суть
безоговорочно лишь {\em моменты}, как моменты
{\em одного} отрицательного единства, есть
{\em противоположность}.


\subsubsection[Примечание Начало разности]
{Примечание Примечание Начало разности}

Разность, как и тождество, выражается в особом предложении. Впрочем, эти два
предложения удерживаются в безразличной разности по отношению друг к другу,
так что каждое предложение признается верным само по себе безотносительно к
другому.

{\em Все вещи разны} или: {\em нет
двух вещей, которые были бы одинаковы}. — Это предложение на самом деле
противоположно предложению о тождестве, ибо оно высказывает: $А$
есть некоторое разное, следовательно, $А$ также и не есть
$А$; или выражая это иначе: $А$ неодинаково с некоторым
другим; таким образом, оно есть не $А$ вообще, а, наоборот,
некоторое определенное $А$. В тождественном предложении вместо
$А$ можно поставить всякий другой субстрат, но $А$, как
неодинаковое, уже больше не может быть заменено всяким другим. Оно, правда,
есть по смыслу предложения некоторое отличное {\em не
от себя}, а лишь {\em от другого}; но эта отличность,
эта разность есть его собственное определение. Как тождественное с собой
$А$, оно есть неопределенное; но как определенное, оно есть
противоположность этого; оно уже содержит в себе не исключительно только
тождество с собою, а также и некоторое отрицание, тем самым некоторую
отличность (разность) самого себя от себя.

Что все вещи разнятся друг от друга, есть совершенно излишнее предложение;
ибо во множественном числе слова «вещи» уже непосредственно подразумевается
множественность и совершенно неопределенная разность. — Но предложение: нет
двух вещей, которые были бы вполне одинаковы, выражает больше, а именно
{\em определенную} разность. Две вещи суть не только
две (нумерическая множественность есть только однородность), а они разны
через {\em некоторое определение}. Предложение,
гласящее, что нет двух вещей одинаковых между собой, поражает представление
и, согласно известному анекдоту, поразила его также при одном дворе, где
Лейбниц его изложил и тем побудил дам искать среди листьев дерева, не
найдут ли они два одинаковых. — Блаженные времена для метафизики, когда ею
занимались при дворе и когда не требовалось никаких других усилий для
исследования ее положений, кроме сравнивания листьев на дереве! —~Причина,
почему это положение представляется удивительным, лежит в сказанном выше, а
именно в том обстоятельстве, что {\em два} или
нумерическая множественность еще не заключает в себе
{\em определенной} разности, и что разность, как
таковая, в ее абстрактности ближайшим образом безразлична к одинаковости и
неодинаковости. Представление, переходя также и к процессу определения,
берет самые эти моменты, как безразличные друг к другу, полагая, что для
определения достаточно одного из них без другого, достаточно
{\em голой одинаковости} вещей
{\em без неодинаковости} или что вещи разны, хотя бы
они были только нумерически множественными, разными вообще, а не
неодинаковыми. Напротив, предложение о разности гласит, что вещи разнятся
между собою через неодинаковость, что им настолько же присуще определение
неодинаковости, насколько и определение одинаковости, ибо лишь оба эти
определения вместе впервые составляют определенное различие.

Это предложение, провозглашающее, что всем вещам присуще определение
неодинаковости, нуждалось бы в доказательстве; оно не может быть
выставлено, как непосредственное положение, ибо даже обычный способ
познания для связывания разных определений в одном синтетическом
предложении требует, чтобы привели доказательство или показали некоторое
третье, в котором они опосредствованы. Это доказательство должно было бы
показать переход тождества в разность, а затем переход последней в
определенную разность, в неодинаковость. Но этого доказательства обычно не
дают; оно получилось у нас тем, что мы показали, что разность или внешнее
различие есть на самом деле рефлектированное в себя различие, различие в
нем же самом, что безразличное пребывание разного есть голая положенность и
тем самым не внешнее, безразличное различие, а
{\em единое} соотношение обоих моментов.

В этом заключается также разложение и ничтожность{\em 
предложения о разности}. Две вещи не могут быть вполне одинаковы; таким
образом, они одновременно и одинаковы и неодинаковы; одинаковы уже тем, что
они суть вещи или вообще две, ибо каждая есть, как и другая, некоторая вещь
и некоторое одно, каждая, следовательно, есть то же самое, что другая;
неодинаковы же они по предположению. Тем самым имеется определение, что оба
момента, одинаковость и неодинаковость, разны {\em в
одном и том же}, или, иначе говоря, что распадающееся различие есть вместе
с тем одно и то же соотношение. Тем самым это определение перешло в
{\em противоположение}.

Правда, «вместе», утверждаемое относительно этих двух предикатов, не сливает
их в одно благодаря прибавлению \ слова «поскольку», — благодаря
утверждению, что две вещи, {\em поскольку} они
одинаковы, {\em постольку} не неодинаковы, или: с одной
{\em стороны} и в одном
{\em отношении} одинаковы, а с другой
{\em стороны} и в другом
{\em отношении} неодинаковы. Этим из вещи удаляют
единство одинаковости и неодинаковости; и то, что было бы ее собственной
рефлексией и рефлексией одинаковости и неодинаковости в себе, фиксируется,
как некоторая внешняя самой вещи рефлексия. Но ведь это означает, что
последняя-то именно и различает {\em в одной и той же
деятельности} эти две стороны, одинаковость и неодинаковость, и стало быть,
содержит их обе в {\em одной} деятельности, заставляет
светиться и рефлектирует одну в другую. — Обычное же нежничание с вещами,
заботящееся лишь о том, чтобы они не противоречили себе, забывает здесь,
как и в других случаях, что этим противоречие не разрешается, а лишь
отодвигается куда-то в другое место, в субъективную или вообще внешнюю
рефлексию, и что последняя на самом деле заключает в себе в одном единстве,
как снятые и соотнесенные друг с другом, оба момента, которые этим
удалением и перемещением провозглашаются как исключительно только
{\em положенность}.
 

\subsection[3. Противоположность]{3. Противоположность}
В противоположности {\em определенная рефлексия} (различие) завершена.
Противоположность есть единство тождества и разности; его моменты суть в
едином тождестве разные; таким образом, они суть
{\em противоположные}.

{\em Тождество} и {\em различие}
суть моменты различия, заключенные внутри него самого; они суть
{\em рефлектированные} моменты его единства.
{\em Одинаковость же и неодинаковость} суть ставшая
внешней рефлексия; тождество этих моментов с собой есть безразличие каждого
из них не только к различающемуся от него, но и к в-себе-и-для-себя-бытию
как таковому —~есть тождество с собой, противостоящее рефлектированному в
себя тождеству; оно есть, следовательно, не рефлектированная в себя
{\em непосредственность}. Положенность сторон внешней
рефлексии есть поэтому некоторое {\em бытие}, равно как
и их неположенность есть некоторое {\em небытие}.

При более близком рассмотрении моментов противоположности оказывается, что
они представляют собою рефлектированную в себя положенность или определение
вообще. Положенность есть одинаковость и неодинаковость; обе они,
рефлектированные в себя, составляют определения противоположности. Их
рефлексия в себя состоит в том, что каждый из них есть в себе же самом
единство одинаковости и неодинаковости. Одинаковость имеет бытие лишь в
рефлексии, сравнивающей со стороны неодинаковости, и тем самым одинаковость
опосредствована ее другим безразличным моментом; и точно так же
неодинаковость имеет бытие лишь в том же рефлектирующем соотнесении, в
котором имеет бытие одинаковость. — Каждый из этих моментов есть,
следовательно, в своей определенности целое. Он есть целое, поскольку он
содержит также и свой другой момент; но это его другое есть некоторое
безразлично {\em сущее}; таким образом, каждый момент
содержит в себе соотношение со своим небытием и есть лишь рефлексия в себя
или целое, как существенно соотносящееся со своим небытием.

Эта рефлектированная в себя {\em одинаковость} с собою,
содержащая в самой себе соотношение с неодинаковостью, есть
{\em положительное}; равным образом
{\em неодинаковость}, содержащая в себе самой
соотношение со своим небытием, с одинаковостью, есть
{\em отрицательное}. — Или, скажем иначе, оба эти
определения суть положенность; и поскольку различенная определенность
берется, как различенное {\em определенное соотношение}
положенности {\em с собой}, противоположность есть, с
одной стороны, {\em положенность}, рефлектированная в
свою {\em одинаковость с собой}, а, с другой стороны,
{\em положенность}, рефлектированная в свою
неодинаковость с собой, — {\em положительное} и
{\em отрицательное}. —
{\em Положительное} есть положенность, как
рефлектированная в одинаковость с собой; но рефлектированное есть
положенность, т.~е. отрицание, как отрицание; таким образом, эта рефлексия
в себя имеет своим определением соотношение с другим.
{\em Отрицательное} есть положенность, как
рефлектированная в неодинаковость. Но положенность есть сама же
неодинаковость. Эта рефлексия есть, стало быть, тождество неодинаковости с
собой самой и абсолютное соотношение с собой. — Обоим, следовательно,
присущи: положенности рефлектированной в одинаковость с собой
—~неодинаковость, а положенности рефлектированной в неодинаковость с собой
—~одинаковость.

Положительное и отрицательное суть, таким образом, ставшие самостоятельными
стороны противоположности. Они самостоятельны, так как они суть рефлексия
{\em целого} в себя, и они принадлежат к
противоположности, поскольку именно
{\em определенность}{}-то и рефлектирована, как целое,
в себя. В силу своей самостоятельности они составляют определенную
{\em в себе} противоположность. Каждое из них есть оно
же само и свое другое; и потому каждое имеет {\em свою
определенность} не в чем-либо другом, {\em а в себе
самом}. — Каждое соотносится с самим собою, лишь соотносясь со своим
другим. Это имеет двоякий аспект; каждое есть соотношение со своим
небытием, как снятие внутри себя этого инобытия; таким образом, его небытие
есть лишь момент внутри его. Но, с другой стороны, положенность стала здесь
бытием, безразличным пребыванием; содержащееся в каждом из них его другое
есть поэтому также и небытие того, в чем оно, как сказано, содержится лишь
как момент. Каждое имеется поэтому лишь постольку, поскольку имеется его
{\em небытие}, и притом в тождественном соотношении.

Определения, образующие положительное и отрицательное, состоят,
следовательно, в том, что положительное и отрицательное суть, во-первых,
абсолютные {\em моменты} противоположности; их
пребывание есть нераздельно {\em единая} рефлексия; в
едином опосредствовании каждое есть через небытие своего другого,
следовательно, через свое другое или через свое собственное небытие. —
Таким образом, они суть {\em противоположные} вообще;
или, иначе сказать, {\em каждое} есть лишь
противоположное другого; первое еще не есть положительное, а второе еще не
есть отрицательное, но оба они суть отрицательные друг относительно друга.
Каждое, таким образом, {\em есть} вообще,
{\em во-первых}, постольку,
{\em поскольку есть другое}; оно есть то, что оно есть,
через другое, через свое собственное небытие; оно есть лишь
{\em положенность};
{\em во-вторых}, оно есть постольку,
{\em поскольку другого нет}; оно есть то, что оно есть,
через небытие другого; оно есть {\em рефлексия} в себя.
— Но и та и другая обусловленность суть {\em единое}
опосредствование противоположности вообще, в котором они
{\em суть} вообще лишь
{\em положенные}.

Но, {\em далее}, эта голая положенность рефлектирована
вообще в себя; положительное и отрицательное по этому моменту
{\em внешней рефлексии безразличны} к тому первому
тождеству, в котором они суть лишь моменты; или, иначе говоря, поскольку
эта первая рефлексия есть собственная рефлексия положительного и
отрицательного в самое себя, и каждое есть своя положенность в нем же
самом, то каждое из них безразлично к этой своей рефлексий в свое небытие,
к своей собственной положенности. Обе стороны суть, таким образом, только
разные, и поскольку их определенность, заключающаяся в том, что они
положительны и отрицательны, образует их положенность друг относительно
друга, то каждая из этих сторон определена так не в ней самой, а есть лишь
определенность вообще; поэтому, хотя каждой стороне присуща одна из
определенностей —~положительного или отрицательного, но они могут быть
переставляемы, и каждая сторона носит такой характер, что ее можно
одинаково принимать как за положительную, так и за отрицательную.

Но положительное и отрицательное не есть,
{\em в-третьих}, ни только некоторое положенное, ни
просто безразличное, а дело обстоит так, что их положенность или
{\em соотношение с другим в некотором единстве, которым
не являются они сами, вобрана обратно в каждое из них}. Каждое из них в нем
самом положительно и отрицательно; положительное и отрицательное есть
рефлективное определение в себе и для себя; лишь в этой рефлексии
противоположного внутрь себя оно положительно и отрицательно. Положительное
содержит в себе самом то соотношение с другим, которое составляет
определенность положительного; равным образом и отрицательное не есть
отрицательное по отношению к некоторому другому, а тоже имеет в себе самом
ту определенность, в силу которой оно отрицательно.

Таким образом, каждое из них есть самостоятельное, сущее особо единство с
собой. Положительное есть, правда, положенность, но так, что для него
положенность есть лишь положенность, как снятая. Оно есть
{\em непротивоположное}, есть снятая противоположность,
но как сторона самой противоположности. — Как положительное, нечто, правда,
определено в соотношении с некоторым инобытием, но так, что его природа
состоит в том, чтобы не быть положенным; оно есть рефлексия в себя,
отрицающая инобытие. Но и его другое, отрицательное, само уже больше не
есть положенность или момент, а есть самостоятельное
{\em бытие}; таким образом, отрицающая рефлексия
положительного внутрь себя определена, как
{\em исключающая} из себя это свое
{\em небытие}.

Точно так же и отрицательное, как абсолютная рефлексия, есть не
непосредственное отрицательное, а отрицательное, как снятая положенность;
оно есть отрицательное в себе и для себя, положительно покоящееся на самом
себе. Как рефлексия в себя, оно отрицает свое соотношение с другим; его
другое есть положительное, некоторое самостоятельное бытие; его
отрицательное соотношение с последним состоит поэтому в том, что оно
исключает его из себя. Отрицательное есть пребывающее особо
противоположное, в противоположность положительному, которое есть
определение снятой противоположности. Отрицательное есть покоящаяся на себе
{\em вся противоположность}, противоположная
тождественной с собой \ положенности.

Положительное и отрицательное суть, стало быть, положительное и
отрицательное не только {\em в себе}, но в себе и для
себя. {\em В себе} они таковы, поскольку абстрагируются
от их исключающего соотношения с другим, в берут их лишь по их определению.
Нечто положительно или отрицательно {\em в себе}, когда
оно должно быть определяемо так {\em не только
относительно другого}. Но когда берут положительное или отрицательное не
как положенность и тем самым не как противоположное, тогда каждое есть
непосредственное, {\em бытие} и
{\em небытие}. Но положительное и отрицательное суть
моменты противоположности; их в-себе-бытие составляет лишь форму их
рефлектированности в себя. Нечто есть положительное
{\em в себе}, вне соотношения с отрицательным; и нечто
есть {\em отрицательное в себе}, вне соотношения с
положительным~\pagenote{В немецком тексте
(как в издании Глокнера, так и в издании Лассона) вместо слова
«положительным» стоит слово «отрицательным». По-видимому, это опечатка.};
в этом определении держатся лишь абстрактного момента этой
рефлектированности. Но {\em в-себе-сущее} положительное
или отрицательное означает по существу, что быть противоположным не есть ни
только момент, ни нечто принадлежащее области сравнений, а есть
{\em собственное} определение сторон противоположности.
Они, следовательно, положительны или отрицательны
{\em в себе} не вне соотношения с другим, но так, что
{\em это соотношение}, и притом, как исключающее,
составляет их определение или в-себе-бытие; здесь, стало быть, они суть
положительные и отрицательные также в себе и для себя.


\subsubsection[Примечание Противоположные величины арифметики]
{Примечание Противоположные величины арифметики}

Здесь нужно кое-что сказать о понятии
{\em положительного} и
{\em отрицательного}, как оно встречается нам в
{\em арифметике}. Оно предполагается там известным; но
так как его понимают не в его определенном различии, то оно не свободно от
неразрешимых затруднений и запутанности. Только что получились оба
{\em реальных} определения положительного и
отрицательного —~помимо простого понятия их противоположения,— состоящие в
том, что, {\em во-первых}, в основании лежит лишь
разное, непосредственное наличное бытие, простую рефлексию которого внутрь
себя различают от его положенности, от самого противоположения. Последнее
поэтому имеет силу только как не в-себя-и-для-себя-сущее, и хотя оно
принадлежит разному, так, что каждое из разных есть противоположное вообще,
однако, вместе с тем это разное остается самостоятельным и безразличным к
противоположению, и все равно, какое из обоих противоположных разных
считать положительным или отрицательным. — Но,
{\em во-вторых}, положительное есть положительное в
себе самом, а отрицательное —~отрицательное в себе самом, так что разное не
безразлично к противоположению, а последнее есть его определение в себе и
для себя. — Обе эти формы положительного и отрицательного встречаются сразу
уже в первых определениях, в которых они употребляются в арифметике.

Во-первых, $+a$ и $-a$
суть {\em противоположные величины вообще}:
$а$ есть лежащая в основании обоих
{\em в-себе-сущая единица}, которая безразлична к
самому противоположению и служит здесь без всякого дальнейшего понятия
мертвой основой. Правда, $-a$ означает
отрицательное, $+a$ —~положительное, но
{\em каждое из них} есть столь же
{\em противоположное, как и другое}.

Далее, $а$ есть не только
{\em простая}, лежащая в основании единица, но как
$+а$ и  $-а$, она есть
рефлексия этих противоположных в себя; имеются {\em два
разных $а$}, и безразлично, какую из них обозначают, как положительное или
отрицательное; оба обладают самостоятельным устойчивым наличием и
положительны.

Взятые с той первой стороны ; или в $-8+3$ три единицы положительные в 3,
отрицательны в 8. Противоположные упраздняются в своем соединении. Если
проделан час пути на восток, и точно такой же путь обратно на запад, то
последний путь упраздняет первый; сколько есть долгов, на столько меньше
имущества и сколько есть имущества, столько вычеркивается долгов. Вместе с
тем ни час пути на восток не есть сам по себе положительный путь, ни путь
на запад не есть сам по себе отрицательный путь; а эти направления
безразличны к сказанной определенности противоположности; лишь некоторое
третье, имеющее место вне их соображение делает одно из этих направлений
положительным, а другое отрицательным. Равным образом и долги не суть
отрицательное сами по себе; они таковы лишь по отношению к должнику; для
заимодавца они суть его положительное имущество; это —~некоторая сумма
денег (или чего бы то ни было, обладающего известной ценностью), которая
становится долгом или имуществом по соображениям, имеющим место вне ее.

Противоположные, правда, упраздняют себя в своем соотношении, так что
результат равен нулю; но в них имеется равным образом и их
{\em тождественное соотношение}, безразличное в самой
противоположности; таким образом, они составляют
{\em одно}. Как было упомянуто о сумме денег, она есть
лишь {\em одна} сумма, или $а$
есть лишь одно $а$ и в
$+а$ и в $-а$; равным
образом, упомянутый выше путь есть лишь один отрезок пути, а не два пути, —
один на восток, а другой на запад. Точно так же и ордината
$у$ — одна и та же, на какой бы стороне оси мы ее ни
взяли; в этом смысле ; она только ордината как таковая; имеется только
{\em одно} определение и {\em один}
закон ординаты.

Но, далее, два противоположных суть не одно безразличное, а
{\em два безразличных}. А именно, они, как
противоположные, суть также и рефлектированные в себя, и таким образом
остаются разными.

Так, в выражении  $-8+3$ дано вообще 11 единиц;
$+y$ и $-у$ суть
ординаты на противоположных сторонах оси, на которых каждая есть наличное
бытие, безразличное к этой границе и в своей противоположности; таким
образом, .— Равным образом, путь, проделанный на восток и на запад, есть
сумма двойного усилия или сумма двух периодов времени. Точно так же в
политической экономии определенное количество денег или ценностей есть как
средство существования не только это одно количество, а оно удвоено; оно
есть средство существования и для заимодавца, и для должника.
Государственное имущество исчисляется не только как сумма наличных денег и
других недвижимых и движимых ценностей, имеющихся в государстве, и тем
паче, не как сумма, остающаяся свободной после вычитания пассивного
имущества из активного; а капитал, хотя бы его активное и пассивное
определение сводились к нулю, остается,
{\em во-первых}, положительным капиталом, как ;
во-вторых же, поскольку он самым различным образом является пассивным
капиталом, дается и снова дается в заем, он оказывается благодаря этому,
весьма многообразным средством.

Но противоположные величины суть не только, с одной стороны, просто
противоположные вообще, а, с другой, реальные или безразличные. Дело
обстоит так, что хотя само определенное количество есть безразлично
ограниченное бытие, мы все же встречаем в нем также и положительное в себе
и отрицательное в себе. Например, $а$, поскольку
оно не имеет знака, считается за положительное, если перед ним требуется
поставить знак. Если бы оно должно было посредством знака стать лишь
противоположным вообще, то его одинаково можно было бы принять и за 
$-а$. Но положительный знак дается ему
непосредственно, так как положительное само по себе имеет своеобразное
значение непосредственного, как тождественного с собой, в отличие от
противоположения.

Далее, когда положительные и отрицательные величины складываются или
вычитаются, то они принимаются за сами по себе положительные и
отрицательные, а не за становящиеся такими лишь внешним образом через
отношение сложения и вычитания. В выражении $8-(-3)$ первый минус
{\em противополагается} восьми, а второй минус $(-3)$
есть противоположный {\em в себе}, вне этого отношения.

Ближе обнаруживается это в умножении и делении; здесь положительное следует
брать по существу, как {\em непротивоположное},
отрицательное же, как противоположное, а не принимать обоих определений
одинаково лишь за противоположные вообще. Так как учебники при
доказательстве правил о знаках в обоих этих действиях не идут дальше
понятия противоположных величин вообще, то эти доказательства страдают
неполнотой и запутываются в противоречиях. — Но плюс и минус в умножении и
делении получают более определенное значение положительного и
отрицательного в себе, так как взаимное отношение множителей, заключающееся
в том, что они суть по отношению друг к другу единица и численность, не
есть просто отношение увеличения и уменьшения, как при сложении и
вычитании, а имеет качественный характер, вследствие чего плюс и минус тоже
получают качественное значение положительного и отрицательного. — Если не
принимать во внимание этого определения и исходить только из понятия
противоположных величин, легко можно вывести ложное заключение, что если 
$-a\cdot+a=-a^2$, то наоборот $+a\cdot-a=+a^2$. Так как один из множителей означает численность, а
другой — единицу, причем за первую принимается обыкновенно первый множитель,
то оба выражения  $-a\cdot+a$ и $+a\cdot-a$ различаются тем,
что в первом \ $+а$ есть единица и
$-а$ численность, а во втором наоборот. По поводу
первого обыкновенно говорят, что если $+а$ должно
быть взято $-а$ раз, то я беру
$+а$ не просто $а$ раз, а
вместе с тем противоположным ему образом, т.~е.  раз 
$+a$~\pagenote{В немецком тексте
наоборот: «$+~а$ раз $–~а$». По-видимому, это опечатка.}
поэтому, так как мы имеем тут {\em $+~а$}, то его следует
брать отрицательно, и произведение есть . Если же во втором случае
$-а$ должно быть взято
$+а$ раз, то это $-а$
равным образом следовало бы брать не $-а$ раз, а в
противоположном ему определении, именно $+а$ раз.
Следовательно, рассуждая, как и в первом случае, произведение должно было
бы быть  $+a^2$. — То же самое должно было бы иметь место и при
делении.

Это заключение необходимо, поскольку плюс и минус берутся лишь как
противоположные величины вообще; минусу в первом случае приписывается сила
изменять плюс; во втором же случае плюс не должен был бы иметь такой силы
над минусом, несмотря на то, что он так же, как и последний, есть
{\em противоположное} определение величины. И в самом
деле, плюс не обладает такой силой, потому что он должен быть взят здесь по
своему качественному определению относительно минуса, поскольку множители
относятся между собою качественно. Постольку, следовательно, отрицательное
есть здесь противоположное в себе, противоположное, как таковое, а
положительное есть неопределенное, безразличное вообще; правда, оно есть
также и отрицательное, но отрицательное другого, а не в себе самом. —
Определение, как отрицание, получается, стало быть, лишь через
отрицательное, а не через положительное.

Точно так же и  потому, что отрицательное $а$ должно
быть взято не просто противоположным образом (а ведь именно так оно должно
было бы быть взято при умножении на $-а$), а
отрицательным образом. Отрицание же отрицания есть положительное.

\section[С. Противоречие]{С. Противоречие}
1. {\em Различие} вообще
содержит в себе обе свои стороны как {\em моменты}; в
{\em разности} они {\em равнодушно}
распадаются врозь; в {\em противоположности} как
таковой они суть стороны различия, определенные лишь одна через другую, —
суть, стало быть, лишь моменты; но они вместе с тем также и определены в
самих себе, как безразличные по отношению друг к другу и взаимно
исключающие: они суть {\em самостоятельные определения
рефлексии}.

Одна сторона есть {\em положительное}, другая же
—~{\em отрицательное}, но первая есть положительное в
самом себе, а последняя —~отрицательное в самом себе. Безразличной
самостоятельностью каждое в отдельности обладает благодаря тому, что оно
содержит в самом себе соотношение со своим другим моментом; таким образом,
оно есть {\em вся} целиком, замкнутая внутри себя
противоположность. — Как такое целое каждое опосредствовано с собой
{\em своим другим} и {\em содержит}
в себе это другое. Но оно, далее, опосредствовано с собою
{\em небытием своего другого}; таким образом оно есть
особо сущее единство и {\em исключает} из себя другое.

Так как самостоятельное определение рефлексии исключает другое в том же
самом отношении, в котором оно содержит в себе это другое и благодаря этому
самостоятельно, то оно в своей самостоятельности исключает из себя свою
собственную самостоятельность; ибо последняя состоит в том, что она
содержит в себе свое другое определение и единственно только благодаря
этому не есть соотношение с некоторым внешним; но столь же непосредственно
эта самостоятельность состоит также и в том, что она есть она же сама и
исключает из себя отрицательное по отношению к себе определение.
Самостоятельное определение рефлексии есть, таким образом,
{\em противоречие}.

Различие вообще есть уже противоречие {\em в себе}; ибо
оно есть {\em единство} таких, которые суть лишь
постольку, поскольку они {\em не суть одно}, и
{\em разъединение} таких, которые суть лишь как
разъединенные {\em в одном и том же отношении}. Но
положительное и отрицательное суть {\em положенное}
противоречие, так как они как отрицательные единства сами суть полагание
самих себя, и в этом полагании каждое есть упразднение себя и полагание
своей противоположности. — Они составляют определяющую рефлексию, как
{\em исключающую}; так как исключение есть
{\em единое} различение и каждое из различенных, как
исключающее, само есть все исключение, то каждое исключает себя в себе
самом.

Если будем рассматривать каждое из этих двух самостоятельных определений
рефлексии в отдельности, то положительное есть
{\em положенность}, как рефлектированное в
{\em равенство с собой}, — положенность, которая не
есть соотношение с некоторым другим, стало быть, устойчивое наличие,
поскольку положенность {\em снята} и
{\em исключена}. Но тем самым положительное обращает
себя в {\em соотношение некоторого небытия}, в
некоторую {\em положенность}. — Таким образом, оно есть
то противоречие, что оно, как полагание тождества с собой, обращает само
себя через {\em исключение} отрицательного в
{\em отрицательное} чего-то (von einem) и,
следовательно, в то другое, которое оно из себя исключает. Последнее, как
исключенное, положено свободным от исключающего и, стало быть, —
рефлектированным в себя и тем, что само исключает. Таким образом,
исключающая рефлексия есть полагание положительного, как исключающего
другое, так что это полагание есть непосредственно полагание того своего
другого, которое исключает его.

В этом состоит абсолютное противоречие положительного; но это противоречие
есть непосредственно и абсолютное противоречие отрицательного; полагание их
обоих есть единая рефлексия. — Отрицательное, рассматриваемое особо, наряду
с положительным, есть положенность, как рефлектированная в
{\em неравенство с собой}, отрицательное, как
отрицательное. Но отрицательное само есть неравное, небытие некоторого
другого; следовательно, рефлексия в его неравенство есть скорее его
соотношение с самим собой. — Отрицание {\em вообще}
есть отрицательное, как качество, или, иначе говоря,
{\em непосредственная} определенность; но
отрицательное, {\em как отрицательное}, соотнесено со
своим отрицательным, со своим другим. Если берут это отрицательное лишь как
тождественное с первым, то оно, равно как и первое, лишь непосредственно;
они, таким образом, понимаются не как другие по отношению друг к другу,
стало быть, не как отрицательные; отрицательное есть вообще не
непосредственное. — Но далее, так как каждое есть вместе с тем то же самое,
что и другое, то это соотношение неравных есть вместе с тем их
тождественное соотношение.

Тут, следовательно, получается то же самое противоречие, которым раньше
оказалось положительное, а именно, положенность или отрицание как
соотношение с собой. Но положительное есть лишь {\em в
себе} это противоречие; напротив, отрицательное есть
{\em положенное} противоречие; ибо в своей рефлексии в
себя, заключающейся в том, что оно есть в себе и для себя отрицательное
или, иначе говоря, что оно, как отрицательное, тождественно с собой, — в
этой рефлексии в себя оно имеет то определение, что оно есть
нетождественное, исключение тождества. Оно состоит в том, чтобы быть
{\em тождественным с собой вопреки тождеству} и тем
самым исключать через свою исключающую рефлексию само себя из себя.

Отрицательное есть, следовательно, целое (как противоположение,
самодовлеющее) противоположение, есть абсолютное,
{\em не соотносящееся с иным} различие; это различие
как противоположение исключает из себя тождество; но тем самым оно
исключает само себя, ибо, как {\em соотношение с
собой}, оно определяет себя, как то самое тождество, которое оно исключает.


\bigskip

2. Противоречие {\em разрешается}.

В исключающей самоё себя рефлексии, которую мы рассматривали раньше,
положительное и отрицательное снимают в своей самостоятельности самих себя;
каждое из них есть безоговорочно переход или, вернее, перемещение себя в
свою противоположность. Это безостановочное исчезание противоположных в них
самих есть {\em ближайшее единство}, получающееся через
противоречие; это единство есть {\em нуль}.

Но противоречие содержит в себе не только
{\em отрицательное}, но также и
{\em положительное}; или, иначе говоря, исключающая
самоё себя рефлексия есть вместе с тем {\em полагающая}
рефлексия. Результат противоречия представляет собой не только нуль. —
Положительное и отрицательное составляют
{\em положенность} самостоятельности; отрицание их ими
же самими снимает {\em положенность} самостоятельности.
Это-то и есть то, что поистине идет ко дну в противоречии, погружается в
основание. (На немецком языке непереводимая игра слов. Zu Grunde geht: идет
ко дну, погибает, а буквально: идет к основанию, погружается в основание. В
дальнейшем Гегель употребляет это и аналогичные выражения то в одном, то в
другом из этих смыслов, а чаще в обоих смыслах вместе. Перевод этого
выражения двумя выражениями \ «идет ко дну, погружается в основание»
передает, насколько возможно, оба смысла. —
{\em Перев}.)~\pagenote{Об игре слов в
немецком выражении «zu Grunde gehen», как его употребляет Гегель, см.
замечания Энгельса в письмах Конраду Шмидту от 1 ноября 1891 г. и от 4
февраля 1892 г. ({\em Маркс} и {\em Энгельс}, Письма, под ред.
Адоратского, изд. 4-е, М.—Л. 1931, стр. 393–394).}.

Рефлексия в себя, через которую стороны противоположности обращают себя в
самостоятельные соотношения с собой, есть ближайшим образом их
самостоятельность как {\em различенных} моментов; они,
таким образом, суть лишь {\em в себе} эта
самостоятельность, ибо они суть еще противоположные; и то обстоятельство,
что они таковы {\em в себе}, составляет их
положенность. Но их исключающая рефлексия снимает эту положенность, делает
их для-себя-сущими самостоятельными, такими самостоятельными, которые
самостоятельны не только {\em в себе}, а через их
отрицательное соотношение с их другим; их самостоятельность, таким образом,
также и положена. Но, далее, через это свое полагание они делают себя
некоторой положенностью. Они {\em пускают себя ко дну,
в основание}, определяя себя, как тождественное с собой, но в этом
тождественном скорее как отрицательное, как такое тождественное с собой,
которое есть соотношение с другим.

Но при ближайшем рассмотрении оказывается, что эта исключающая рефлексия
есть не только такое формальное определение. Она есть
{\em в-себе}{}-сущая самостоятельность и представляет
собою снятие этой положенности, и лишь через это снятие —~для-себя-сущее и
в самом деле самостоятельное единство. Через снятие инобытия или
положенности оказывается, правда, снова наличной положенность,
отрицательное некоторого другого. Но на самом деле это отрицание не есть
снова лишь первое, непосредственное соотношение с другим, не есть
положенность, как снятая непосредственность, а как снятая положенность.
Исключающая рефлексия самостоятельности, будучи исключающей, делает себя
положенностью, но есть вместе с тем также и снятие своей положенности. Она
есть снимающее соотношение с собой; в этом соотношении она,
{\em во-первых}, снимает отрицательное и,
{\em во-вторых}, полагает себя как отрицательное; и
это-то и есть впервые то отрицательное, которое она снимает; в снимании
отрицательного она одновременно и полагает и снимает его.
{\em Само исключающее определение} есть для себя, таким
образом, то {\em другое}, отрицание которого она есть;
снятие этой положенности не есть поэтому снова положенность, как
отрицательное некоторого другого, а есть слияние с самим собою,
положительное единство с собою. Самостоятельность, таким образом, есть
через {\em свое собственное} отрицание возвращающееся в
себя единство, так как она возвращается в себя через отрицание
{\em своей} положенности. Она есть единство сущности,
заключающееся в том, что она тождественна с собою не через отрицание
некоторого другого, а через отрицание самой себя.

3. С этой положительной стороны, с которой самостоятельность в
противоположении, как исключающая рефлексия, делает себя положенностью и
вместе с тем снимает ее, противоположность не только погрузилась
{\em в основание}, но и возвратилась
{\em в свое основание}. — Исключающая рефлексия
самостоятельной противоположности делает последнюю некоторым отрицательным,
лишь положенным; она этим понижает в ранге свои сначала самостоятельные
{\em определения}, положительное и отрицательное,
превращает их в такие определения, которые суть
{\em лишь определения}; и когда, таким образом,
положенность делается положенностью, она (положенность) вообще возвратилась
в свое единство с собой; она есть {\em простая
сущность}, но сущность как {\em основание}. Через
снятие противоречащих себе в самих себе определений сущности последняя
восстановлена, однако, с тем определением, что она есть исключающее
единство рефлексии, есть простое единство, которое определяет само себя как
отрицательное, но в этой положенности непосредственно равно самому себе и
слилось с собой.

Следовательно, самостоятельная противоположность
{\em возвращается} сначала через свое противоречие в
основание; она есть то первое, непосредственное, с которого начинают, и
снятая противоположность или снятая положенность сама есть некоторая
положенность. Тем самым {\em сущность как основание
есть некоторая положенность, некоторое ставшее}. Но и наоборот, оказалось
положенным лишь то, что противоположность или положенность есть некоторое
снятое, имеет бытие лишь как положенность. Сущность как основание есть,
следовательно, исключающая рефлексия таким образом, что она делает самоё
себя положенностью, что противоположность, с которой раньше начали и
которая была непосредственным, есть лишь положенная, определенная
самостоятельность сущности, и что она есть лишь нечто снимающее себя в
самом себе, сущность же есть нечто рефлектированное в своей определенности
внутрь себя. Сущность как основание исключает
{\em себя} из самой себя, она полагает
{\em себя}; ее положенность —~которая есть исключенное,
— имеет бытие лишь как положенность, как тождество отрицательного с самим
собой. Это самостоятельное есть отрицательное,
{\em положенное} как отрицательное, есть некое
противоречащее самому себе, которое поэтому непосредственно остается в
сущности как в своем основании.

Разрешенное противоречие есть, следовательно, основание, сущность как
единство положительного и отрицательного. В противоположности определение
доросло до самостоятельности; основание же есть эта завершенная
самостоятельность; отрицательное есть в нем самостоятельная сущность, но
как отрицательное; таким образом, оно есть в такой же степени
положительное, как и тождественное с собой в этой отрицательности. Поэтому
в основании противоположность и ее противоречие столь же упразднены, как и
сохранены. Основание есть сущность как положительное тождество с собой,
однако, такое тождество, которое вместе с тем соотносится с собой как
отрицательность, следовательно, определяет и делает себя исключенной
положенностью; но эта положенность есть вся самостоятельная сущность, и
сущность есть основание как тождественная с самой собою и положительная в
этом своем отрицании. Противоречащая себе самостоятельная противоположность
была уже, следовательно, сама основанием; прибавилось лишь определение
единства с самим собой, которое появляется благодаря тому, что
самостоятельные противоположные снимают каждое само себя и превращают себя
в свое другое, следовательно, идут ко дну, погружаются в основание, но в
этом погружении сливаются вместе с тем лишь с самими собой, следовательно,
в своей гибели, т.~е. в своей положенности или в отрицании, скорее впервые
представляют собою рефлектировавшую в себя, тождественную с собой сущность.


\subsubsection[Примечание 1 Единство положительного и отрицательного]
{Примечание 1 Единство положительного и отрицательного}

{\em Положительное и отрицательное есть одно и то же}.
Это выражение принадлежит {\em внешней рефлексии},
поскольку она производит {\em сравнение} этих двух
определений. Но следует не только производить внешнее сравнение между этими
определениями, равно как и между другими категориями, а их надлежит
рассматривать в них же самих, т.~е. нужно рассмотреть, в чем состоит их
собственная рефлексия. Относительно последней же оказалось, что каждое из
них есть по существу просвечивание себя в другом и даже полагание себя как
другого.

Но представлению, поскольку оно не рассматривает положительного и
отрицательного так, как они суть в себе и для себя, можно во всяком случае
рекомендовать произвести сравнение для того, чтобы обратить его внимание на
несостоятельность этих различенных, которые оно признает прочно
противостоящими друг другу. Ничтожного опыта в рефлектирующем мышлении
будет достаточно, чтобы увидеть, что если нечто было определено, как
положительное, то, когда идут от этой основы дальше, это положительное
непосредственно под руками превращается в отрицательное, и, наоборот,
определенное, как отрицательное, превращается в положительное, и чтобы
убедиться в том, что рефлектирующее мышление запутывается в этих
определениях и становится самопротиворечивым. Те, которые незнакомы с
природой этих определений, придерживаются того мнения, что эта путаница
есть нечто неправомерное, которого не должно быть, и приписывают ее
субъективной погрешности. Этот переход и на самом деле остается голой
путаницей, поскольку нет сознания необходимости этого превращения. — Но и
внешняя рефлексия легко может сообразить, что, во-первых, положительное
есть не некоторое непосредственно тождественное, а отчасти некоторое
противоположное по отношению к отрицательному, и что оно лишь в этом
соотношении обладает значением, следовательно, отрицательное само заключено
{\em в его }{\em понятии}, отчасти
же, что оно в себе самом есть соотносящееся с собой отрицание голой
положенности или отрицательного, и, следовательно, само есть внутри себя
{\em абсолютное отрицание}. — Равным образом
отрицательное, противостоящее положительному, имеет смысл лишь в указанном
соотношении с этим своим другим; оно, следовательно, содержит последнее
{\em в своем понятии}. Но отрицательное обладает и без
соотношения с положительным некоторым {\em собственным
устойчивым наличием}; оно тождественно с собой; но таким образом оно само
есть то, чем должно было быть положительное.

Противоположность между положительным и отрицательным понимается
преимущественно в том смысле, что первое (хотя оно по своему названию
выражает {\em положенность}) есть некоторое
объективное, а последнее —~нечто субъективное, принадлежащее лишь некоторой
внешней рефлексии, ничуть не касающееся \ в-себе-и-для-себя-сущего
объективного и совершенно не существующее для последнего. И в самом деле,
если отрицательное есть не что иное, как абстракция, созданная субъективным
произволом, или определение, являющееся продуктом некоторого внешнего
сравнения, то оно, разумеется, не существует для объективного
положительного, т.~е. последнее не соотнесено в самом себе с такой пустой
абстракцией; но в таком случае определение, что оно есть некоторое
положительное, также лишь внешне ему. — Так, чтобы привести пример
неподвижной противоположности этих определений рефлексии, мы укажем, что
{\em свет} принимается вообще за нечто лишь
положительное, а {\em тьма} за нечто лишь
отрицательное. Но свет в своем бесконечном распространении и силе своей
раскрывающей и животворяющей деятельности обладает по существу природой
абсолютной отрицательности. Напротив, тьма, как нечто немногообразное или,
иначе говоря, как не различающее само себя внутри себя лоно порождения,
есть простое тождественное с собой, положительное. Ее принимают за
исключительно лишь отрицательное в том смысле, что она, как голое
отсутствие света, совершенно не существует для последнего, так что этот
последний, соотносясь с нею, соотносится не с некоторым другим, а чисто с
самим собой, и, следовательно, она лишь исчезает перед ним. Но, как
известно, свет помутняется тьмой, сереет; и помимо этого чисто
количественного изменения он претерпевает также и качественное изменение,
состоящее в том, что благодаря соотношению с нею он определяется в цвет. —
\ \ Подобным же образом, например, и {\em добродетель}
также не существует без борьбы; она скорее представляет собой высшую,
завершенную борьбу; таким образом, она есть не только положительное, но и
абсолютная отрицательность; она также есть добродетель не только в
{\em сравнении} с пороком, а есть
{\em в самой себе} противоположение и борьба с ее
противоположностью. Или, наоборот, {\em порок} не есть
только {\em отсутствие} добродетели —~ведь и невинность
есть такое отсутствие —~и отличается от добродетели не только для внешней
рефлексии, а в самом себе противоположен ей, он есть нравственное
{\em зло}. Нравственное зло состоит в самодовлении, в
сопротивлении добру; оно есть положительная отрицательность. Невинность же,
как отсутствие и добра и зла, безразлична к обоим определениям, не есть ни
положительное, ни отрицательное. Но вместе с тем это отсутствие надлежит
также брать как определенность, и, с одной стороны, ее следует
рассматривать, как положительную природу чего-то, а, с другой стороны, она
соотносится с некоторым противоположным, и все существа выходят из своего
состояния невинности, из своего безразличного тождества с собой,
соотносятся через самих себя со своим другим и вследствие этого пускают
себя ко дну или, в положительном смысле, возвращаются в свое основание. —
{\em Истина} также есть нечто положительное, как
соответствующее объекту знание; но она есть это равенство с собой лишь
постольку, поскольку знание отнеслось отрицательно к другому, пронизало
собою объект и сняло отрицание, которым он является.
{\em Заблуждение} есть нечто положительное, как мнение
касательно того, что не есть само по себе сущее, — мнение, знающее и
отстаивающее себя. Неведение же есть либо нечто безразличное к истине и
заблуждению и, стало быть, не определенное ни как положительное, ни как
отрицательное, так что определение его, как некоторого отсутствия,
принадлежит внешней рефлексии, либо же, как объективное, как собственное
определение некоторого существа, оно есть влечение, направленное против
себя, некоторое отрицательное, содержащее в себе положительное направление.
— Одно из важнейших познаний состоит в усмотрении и удержании того взгляда
на эту природу рассмотренных определений рефлексии, что их истина состоит
лишь в их соотношении друг с другом, и тем самым состоит в том, что каждое
из них в самом своем понятии содержит другое; без этого познания нельзя,
собственно говоря, сделать и шагу в философии.


\subsubsection[Примечание 2 Начало исключенного третьего]
{Примечание 2 Начало исключенного третьего}

Определение противоположения также было превращено в некоторое предложение,
в так называемое {\em начало исключенного третьего}.

{\em Нечто есть либо А либо не-А, нет третьего}.

Это предложение означает, {\em во-первых}, что все есть
некоторое {\em противоположное}, нечто
{\em определенное} либо как положительное, либо как
отрицательное. — Это —~важное положение, имеющее свою необходимость в том,
что тождество переходит в разность, а последняя в противоположение. Однако
это предложение обыкновенно понимают не в указанном смысле, а оно якобы
должно означать, что из всех возможных предикатов вещи присущ либо сам
данный предикат, либо его небытие. Противоположное означает здесь лишь
отсутствие или, вернее, {\em неопределенность}, в
предложение так незначительно, что не стоит труда высказывать его. Если мы
берем определения «сладкое», «зеленое», «четырехугольное», — а нам говорят,
что мы должны брать все предикаты, — и затем высказываем о духе, что он
либо сладок, либо несладок, либо зеленый, либо незеленый и~т.~д., то это
тривиальность, которая ни к чему не приводит. Определенность, предикат
соотносится с чем-то; нечто определено, — так высказывает наше предложение;
последнее должно было бы затем содержать в себе по существу требование,
чтобы определенность определилась ближе, чтобы она стала определенностью
{\em в себе}, противоположением. Но вместо этого
предложение лишь переходит в вышеуказанном тривиальном смысле от
определенности к ее небытию вообще, возвращается назад к неопределенности.

Предложение об исключенном третьем отличается, далее, от рассмотренного выше
предложения о тождестве или противоречии, которое гласило так: же
существует ничего такого, что было бы {\em вместе} и
$А$ и не $-А$. Предложение об исключенном третьем утверждает,
что {\em нет} ничего такого, что не было бы
{\em ни} $А$, {\em ни}
не $-А$, что нет такого третьего, которое было бы безразлично к
противоположности. На самом же деле {\em имеется} в
самом этом предложении третье, которое безразлично к противоположности, а
именно, в нем имеется само $А$. Это $А$ не есть ни
$+А$ ни $-А$, но вместе с тем оно есть и $+А$ и
$-А$. — Нечто, которое якобы должно быть либо $+А$ либо
не $-А$, соотнесено, стало быть, как с $+А$, так и с
не $-А$; и опять-таки утверждают, что, будучи соотнесено с
$А$, оно {\em не} соотнесено с не $-А$,
равно как оно якобы {\em не} соотнесено с $А$,
если оно соотнесено с не $-А$. Само нечто есть, следовательно, то
третье, которое якобы исключено. Так как противоположные определения как
положены в нечто, так и суть в этом полагании снятые, то третье, которое
здесь имеет образ какого-то мертвенного нечто, есть, если его взять
поглубже, то единство рефлексии, в которое, как в основание, возвращается
противоположение.


\subsubsection[Примечание 3 Начало противоречия]
{Примечание 3 Начало противоречия}

Если первые определения рефлексии —~тождество, разность и противоположение
—~нашли каждое свое выражение в особом предложении, то тем паче должно было
бы быть сформулировано в виде предложения то определение, в которое они
переходят, как в свою истину, а именно,
{\em противоречие}. Так что надо было бы сказать:
{\em все вещи противоречивы в самих себе}; и притом в
том смысле, что это предложение выражает по сравнению с прочими истину и
сущность вещей. — Противоречие, выступающее в противоположении, есть лишь
развитое ничто, ничто, содержащееся в тождестве и встретившееся нам в
выражении, что предложение о тождестве {\em ничего} не
говорит. Это отрицание определяет себя в дальнейшем в разность и в
противоположение, которое теперь представляет собою положенное
противоречие.

Но одним из основных предрассудков прежней логики и обычного представления
является взгляд, будто противоречие не есть такое же существенное и
имманентное определение, как тождество; больше того, если уже речь идет об
иерархии и оба определения мы должны фиксировать, как раздельные, то
следовало бы признать противоречие более глубоким и более существенным. Ибо
по сравнению с ним тождество есть лишь определение простого
непосредственного, определение мертвенного бытия; противоречие же есть
корень всякого движения и жизненности; лишь поскольку нечто имеет в самом
себе противоречие, оно движется, обладает импульсом и деятельностью.

Противоречие обыкновенно, во-первых, устраняют из вещей, из сущего и
истинного вообще, утверждая, что {\em нет ничего
противоречивого}. Во-вторых, противоречие, напротив того, выталкивается в
субъективную рефлексию, которая своим соотнесением и сравниванием его якобы
впервые создает. Но и в этой рефлексии его тоже нет по-настоящему; ибо
{\em противоречивого}, как уверяют, нельзя ни
{\em представить} себе, ни
{\em помыслить}. Противоречие признается вообще, будь
это противоречие в действительном или в мыслящей рефлексии, случайностью,
как бы аномалией и преходящим пароксизмом болезни.

Что касается утверждения, что противоречия нет, что оно не есть нечто
существующее, то такого рода заверение не должно причинять нам забот;
абсолютное определение сущности должно оказаться во всяком опыте, во всяком
действительном, равно как во всяком понятии. Выше, говоря о
{\em бесконечном}, представляющем собою противоречие,
как оно обнаруживается в сфере бытия, мы уже указали на нечто подобное. Но
обыденный опыт сам высказывает, что {\em существует} по
меньшей мере {\em множество} противоречивых вещей,
противоречивых учреждений и~т.~д., противоречие которых находится не только
в некоторой внешней рефлексии, а в них самих. Но, далее, противоречие не
следует принимать только за какую-то аномалию, встречающуюся лишь кое-где:
оно есть отрицательное в его существенном определении, принцип всякого
самодвижения, состоящего не в чем ином, как в некотором изображении
противоречия. Само внешнее чувственное движение есть его непосредственное
наличное бытие. Нечто движется не поскольку оно в этом «теперь» находится
здесь, а в другом «теперь» там, а лишь поскольку оно в одном и том же
«теперь» находится здесь и не здесь, поскольку оно в этом «здесь»
одновременно и находится и не находится. Надлежит согласиться с древними
диалектиками, что противоречия, которые они нашли в движении, действительно
существуют; но из этого не следует, что движения нет, а наоборот, что
движение есть само {\em существующее} противоречие.

Равным образом внутреннее, подлинное самодвижение,
{\em импульс} вообще (устремление или напряжение
монады, энтелехия абсолютно простого существа) состоит не в чем ином, как в
том, что в одном и том же отношении существуют нечто
{\em в самом себе} и его отсутствие,
{\em отрицательное его самого}. Абстрактное тождество с
собой еще не есть жизненность; но в силу того, что положительное есть в
самом себе отрицательность, оно выходит вне себя и начинает изменяться.
Нечто, следовательно, жизненно лишь постольку, поскольку оно содержит в
себе противоречие и притом есть та сила, которая в состоянии вмещать в себе
это противоречие и выдерживать его. Но если нечто существующее не способно
в своем положительном определении вместе с тем охватывать свое
отрицательное определение и удерживать одно в другом, если оно не способно
иметь в самом себе противоречие, то оно не есть само живое единство, не
есть основание, а идет в противоречии ко дну. —
{\em Спекулятивное мышление} состоит лишь в том, что
мышление удерживает противоречие и в нем—само себя, а не в том, что оно
допускает, как это происходит с представлением, чтобы это противоречие
господствовало над ним и растворяло его определения лишь в другие или в
ничто.

Если в движении, импульсе и~т.~п. противоречие скрыто для представления за
{\em простотой} этих определений, то, \ напротив, в
{\em определениях отношения} противоречие выступает
непосредственно. Если взять какие угодно тривиальнейшие примеры: верх и
низ, правое и левое, отец и сын и~т.~д. до бесконечности, то все они
содержат противоположность в одном определении. Верх
{\em есть} то, что {\em не} есть
низ; верх определен лишь так, чтобы не быть низом, и
{\em есть} лишь постольку,
{\em поскольку} есть низ, и наоборот; в одном
определении заключается его противоположность. Отец есть другое сына, а сын
—~другое отца, и каждый имеет бытие лишь как это другое другого; и вместе с
тем одно определение имеется лишь в соотношении с другим; их бытие есть
{\em единое} наличие. Отец есть нечто особое также и
вне соотношения с сыном; но при этом он не отец, а мужчина вообще; равным
образом верх и низ, правое и левое суть также и рефлектированное в себя,
суть нечто и вне соотношения, но в таком случае они суть лишь места вообще.
— Противоположные постольку содержат в себе противоречие, поскольку они в
одном и том же отношении суть $-а$) соотносящиеся
друг с другом отрицательно или {\em взаимно
упраздняющие} друг друга и b) {\em безразличные} друг к
другу. Представление, переходя к моменту
{\em безразличия} этих определений, забывает в нем их
отрицательное единство, и тем самым сохраняет их лишь как разные вообще, в
каковом определении правое уже не есть правое, левое уже больше не есть
левое и~т.~д. Но поскольку представление имеет перед собою в самом деле
правое и левое, оно имеет перед собой эти определения, как отрицающие себя
одно в другом, и вместе с тем и как не отрицающие себя в этом единстве, а
каждое, как безразличное сущее особо.

Поэтому представление имеет, правда, повсюду своим содержанием противоречие,
но не доходит до его осознания; оно (представление) остается внешней
рефлексией, переходящей от одинаковости к неодинаковости или от
отрицательного соотношения к рефлектированности различенных определений
внутрь себя. Внешняя рефлексия сопоставляет эти два определения внешним
образом и имеет в виду {\em лишь их}, а не
{\em переход}, который существенен и содержит в себе
противоречие. — {\em Остроумная} рефлексия —~скажем
здесь и о ней —~состоит, напротив, в улавливании и высказывании
противоречия. Хотя она не выражает {\em понятия} вещей
и их отношений, а имеет своим материалом и содержанием лишь определения
представления, она все же приводит их в такое соотношение, которое содержит
в себе их противоречие и {\em дает тем самым их понятию
просвечивать сквозь это последнее}. — Но {\em мыслящий}
разум заостряет, так сказать, притупившееся различие разного, голое
многообразие представления, до {\em существенного}
различия, до {\em противоположности}. Лишь доведенные
до заостренности противоречия, многообразные впервые становятся подвижными
и живыми по отношению друг к другу и получают в нем ту отрицательность,
которая есть имманентная пульсация самодвижения и жизненности.

Относительно {\em онтологического доказательства бытия
божия} мы уже указали, что положенное в нем в основание определение есть
«{\em совокупность всех реальностей}». Относительно
этого определения обыкновенно прежде всего показывают, что оно
{\em возможно}, так как оно, мол, не содержит в себе
{\em противоречия}, потому что реальность берется в
этом доказательстве, лишь как беспредельная реальность. Мы заметили выше,
что этим указанная совокупность превращается в простое неопределенное бытие
или, если реальности берутся на самом деле, как многие определенные
реальности, в совокупность всех отрицаний. Если мы возьмем более
определенно различие реальности, то оно превращается из разности в
противоположность, и тем самым в противоречие, а совокупность всех
реальностей вообще в абсолютное противоречие внутри самого себя. Обычный
horror (страх), который представляющее, не-спекулятивное мышление
испытывает перед противоречием, как природа перед vacuum (пустотой),
отвергает это следствие; ибо это мышление останавливается на одностороннем
рассмотрении {\em разрешения} противоречия в
{\em ничто} и не познает его положительной стороны, по
которой оно становится {\em абсолютной деятельностью} и
абсолютным основанием.

Из рассмотрения природы противоречия получился вообще тот вывод, что если в
той или иной вещи можно обнаружить некоторое противоречие, то это само по
себе еще не есть, так сказать, изъян, дефект или погрешность этой вещи.
Наоборот, каждое определение, каждое конкретное, каждое понятие есть по
существу единство различных и могущих быть различенными моментов, которые
через {\em определенное, существенное различие}
переходят в противоречащие. Это противоречивое во всяком случае разрешается
в ничто, оно возвращается в свое отрицательное единство. Вещь, субъект,
понятие оказываются теперь самим этим отрицательным единством; оно есть
нечто в себе самом противоречивое, но вместе с тем также и
{\em разрешенное противоречие}: оно есть
{\em основание}, содержащее в себе и носящее свои
определения. Вещь, субъект или понятие, будучи рефлектированы в своей сфере
внутрь себя, суть свое разрешенное противоречие, но вся их сфера есть
опять-таки некоторая {\em определенная, разная}; таким
образом, она есть конечная, а это означает
{\em противоречивая}. Не она сама есть разрешение этого
более высокого противоречия, а она имеет своим отрицательным единством,
своим основанием некоторую более высокую сферу. Конечные вещи в их
безразличном многообразии состоят поэтому вообще в том, что они
противоречивы в самих себе, {\em надломлены внутри себя
и возвращаются в свое основание}. — Как это будет выяснено далее, истинное
умозаключение от конечного и случайного к абсолютно необходимому существу
состоит не в том, чтобы умозаключать от конечного и случайного, как от
{\em лежащего и остающегося лежать в основании бытия},
а в том, что (как это даже непосредственно подразумевается в понятии
«{\em случайности}») умозаключают от лишь преходящего,
противоречащего себе {\em в самом себе бытия} к
абсолютно необходимому бытию, или, лучше сказать, состоит скорее в том, что
показывают, что случайное бытие возвращается в самом себе в свое основание,
в котором оно снимает себя, и что, далее, оно через это возвращение
полагает основание лишь так, что оно скорее делает само себя положенным. В
обычном умозаключении {\em бытие} конечного выступает,
как основание абсолютного; именно потому, что
{\em есть} конечное, есть и абсолютное. Но истина
состоит в том, что именно потому, что конечное есть в самой себе
противоречивая противоположность, потому, что оно
{\em не есть}, есть абсолютное. В первом смысле
умозаключение гласит так: {\em бытие} конечного есть
{\em бытие} абсолютного; в последнем же смысле оно
гласит: {\em небытие} конечного есть
{\em бытие} абсолютного.
