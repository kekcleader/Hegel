\textstyleEndnoteText{Немецкое слово «Begriff»
очень удобно для того, чтобы толковать его так, как это
делает Гегель, т.~е. в смысле тотальности определений или совокупности всех
моментов рассматриваемого предмета: «Begriff» происходит от
глагола «begreifen», который означает «обнимать собою», «охватывать
воедино», а затем «понимать, постигать умом». От глагола «begreifen»
происходит, далее, существительное «Inbegriff», означающее
«совокупность, суть, содержание, сущность». У Гегеля слово «Begriff»
нередко употребляется в смысле, близком к значению слова «Inbegriff».
Вообще «Begriff» имеет у Гегеля по преимуществу
}\textstyleEndnoteTextEmphasis{объективное}\textstyleEndnoteText{
значение (объективное в смысле объективного и абсолютного
идеализма), в отличие от того обычного субъективного смысла этого слова,
который имеет в виду «понятие» как идею, образуемую человеческим умом, как
умственное отображение предмета в человеческом сознании. Например, на
стр.~\pageref{bkm:bm01a} Гегель пишет: «понятие есть ушедшая
внутрь себя всеобщая сущность какой-нибудь вещи». Ср. замечание Энгельса:
«в суждении понятия о субъекте высказывается, в какой мере он соответствует
своей всеобщей природе или, как говорит Гегель, своему понятию» (Engels,
Anti-Duhring und Dialektik der Natur, M. — L. 1935, S.
662).}\textstyleEndnoteText{ }}

\begin{enumerate}
\item 
\label{bkm:Ref484771221}\textstyleEndnoteText{К
стр.~\pageref{bkm:bm02}. — Клопшток в поэме «Мессиада»,
песнь 7-я.}\textstyleEndnoteText{ }}
\item 
\label{bkm:Ref484771234}\textstyleEndnoteText{К
стр.~\pageref{bkm:bm03}. — Т.~е. причина и действие,
активная субстанция и пассивная субстанция.}\textstyleEndnoteText{ }}
\item 
\label{bkm:Ref484771243}\textstyleEndnoteText{К
стр.~\pageref{bkm:bm04}. — Гегель имеет в виду высказывания
Фихте, который в «Первом введении в наукоучение» (1797) писал: «Всякий
последовательный догматик —~неизбежно фаталист; он не
отрицает того факта сознания, что мы считаем себя свободными; ибо это было
бы противно разуму; но он доказывает из своего принципа ложность этого
мнения. Он совершенно отрицает самостоятельность Я, на которой строит свое
учение идеалист, и делает его простым продуктом вещей, случайной
принадлежностью мира; последовательный догматик —~неизбежно
материалист. Он }\textstyleEndnoteTextEmphasis{мог бы быть
опровергнут только из постулата свободы и самостоятельности Я: но это
—~то самое, что он отрицает}\textstyleEndnoteText{» (J. G
Fichte, Werke, hrsg. v. Medicus, Bd III, S. 14–15; в русском
издании Избранных сочинений Фихте, т.~I, M. 1916, это место находится на
стр.~421). Что под «последовательным догматизмом» Фихте имел в виду
философию Спинозы, видно из многих других мест его сочинений, например, из
следующего места в «Основах общего наукоучения» (1794): «Поскольку
догматизм может быть последовательным, спинозизм является наиболее
последовательным его продуктом»
[}\textstyleEndnoteTextEmphasis{Fichte}\textstyleEndnoteText{, Werke, Bd.
I, S. 314; в русском издании —~стр.~96).}}
\end{enumerate}

\textstyleEndnoteText{До Фихте о строгой внутренней
последовательности спинозизма и о невозможности его опровергнуть без
обращения к другим принципам говорил Ф.-Г. Якоби.} }

\begin{enumerate}
\item 
\label{bkm:Ref484771255}\textstyleEndnoteText{К
стр.~\pageref{bkm:bm05}. — В
немецком тексте: «die ursprungliche Sache».
Гегель намекает на этимологию немецкого слова «Ur-sache»
(причина; буквально: перво-вещь).}\textstyleEndnoteText{ }}
\item 
\label{bkm:Ref484771272}\textstyleEndnoteText{К
стр.~\pageref{bkm:bm06}. — Гегель, повидимому, имеет в виду
иррационализм Фридриха-Генриха Якоби (1743–1819). Ср. замечания Гегеля об
этом философе в «Малой логике»
(}\textstyleEndnoteTextEmphasis{Гегель}\textstyleEndnoteText{,
Соч., т.~I, стр.~114–118) и в «Истории философии» (т.~XI,
стр.~405–415).}\textstyleEndnoteText{ }}
\item 
\label{bkm:Ref484771290}\textstyleEndnoteText{К
стр.~\pageref{bkm:bm07}. — Имеется в виду трансцендентальная
философия Канта.}\textstyleEndnoteText{ }}
\item 
\label{bkm:Ref484771295}\textstyleEndnoteText{К
стр.~\pageref{bkm:bm08}. — Краткая, но чрезвычайно глубокая
критика гегелевского учения о понятии, как о «конкретной тотальности»,
«образующей (или «порождающей») из себя реальность», дана Марксом в третьей
главе «Введения к Критике политической экономии». Маркс вышелушивает
рациональное зерно, имеющееся в гегелевском учении о восхождении от
абстрактного к конкретному. Он показывает, что научное мышление
действительно движется от абстрактных определений к «конкретной
тотальности» (Маркс даже сам употребляет несколько раз этот гегелевский
термин). Но вместе с тем Маркс с гениальной силой вскрывает фундаментальную
ложь в построениях Гегеля, притом ложь двоякого рода: 1) неверно, будто
«реальное есть результат мышления, синтезирующего воедино свои определения
внутри самого себя, углубляющегося в себя и движущегося из самого себя»: в
действительности «метод восхождения от абстрактного к конкретному есть лишь
способ, при помощи которого мышление усваивает себе конкретное,
}\textstyleEndnoteTextEmphasis{воспроизводит}\textstyleEndnoteText{
его духовно как конкретное, а отнюдь не процесс возникновения
самого конкретного»; 2) неверно, будто конкретная тотальность определений
мысли является продуктом такого «понятия, которое мыслит
}\textstyleEndnoteTextEmphasis{вне}\textstyleEndnoteText{
созерцания и представления или над ними и само порождает
себя»: в действительности эта конкретная тотальность представляет собою
«продукт }\textstyleEndnoteTextEmphasis{переработки
представления и созерцания в понятия}\textstyleEndnoteText{»
(}\textstyleEndnoteTextEmphasis{Marx}\textstyleEndnoteText{, Zur Kritik der
politischen Oekonomie, M. — L. 1934, S. 236–237). Там же
Маркс показывает, что соответствие между ходом абстрактного мышления, с
одной стороны, и действительным историческим процессом, идущим от простого
к сложному, с другой стороны, хотя и имеет место в общем и целом, но не
может быть сведено к простому тождеству, так как в действительности дело
обстоит гораздо сложнее.} }
\item 
\label{bkm:Ref484771376}\textstyleEndnoteText{К
стр.~\pageref{bkm:bm09}. — В действительности это место
находится на стр.~82 второго издания «Критики чистого разума». См.
}\textstyleEndnoteTextEmphasis{Кант}\textstyleEndnoteText{,
Критика чистого разума, пер. Н. Лосского, Пгр. 1915,
стр.~64.}\textstyleEndnoteText{ }}
\item 
\label{bkm:Ref484771406}\textstyleEndnoteText{К
стр.~\pageref{bkm:bm10}. —
}\textstyleEndnoteTextEmphasis{Кант}\textstyleEndnoteText{,
Критика чистого разума, стр.~82–83 по 2-му немецкому изданию.
Курсив принадлежит Гегелю.}\textstyleEndnoteText{ }}
\item 
\label{bkm:Ref484771431}\label{bkm:Ref484792854}\textstyleEndnoteText{К
стр.~\pageref{bkm:bm11}. — Ср. замечание Маркса о том, что
«до Гегеля логики по профессии упускали из вида формальное содержание (den
Forminhalt) различных типов суждений и умозаключений»
(}\textstyleEndnoteTextEmphasis{Маркс}\textstyleEndnoteText{,
Das Kapital, Bd. 1, Hamburg}\textstyleEndnoteText{ 1867, S. 21),
т.~е. то
}\textstyleEndnoteTextEmphasis{реальное
содержание}\textstyleEndnoteText{, которое имеется в
}\textstyleEndnoteTextEmphasis{логической
форме}\textstyleEndnoteText{ суждений и
умозаключений.}\textstyleEndnoteText{ }}
\item 
\label{bkm:Ref484771774}\textstyleEndnoteText{К
стр.~\pageref{bkm:bm12}. — Ср. слова Энгельса:
«Диалектическая логика, в противоположность старой, чисто формальной
логике, не довольствуется тем, чтобы перечислить и сопоставить без связи
формы движения мышления, т.~е. различные формы суждения и умозаключения.
Она, наоборот, выводит эти формы одну из другой, устанавливает между ними
отношение субординации, а не координации, она развивает высшие формы из
низших»
(}\textstyleEndnoteTextEmphasis{Энгельс}\textstyleEndnoteText{,
Диалектика природы, М. 1936,
стр.~100).}\textstyleEndnoteText{ }}
\item 
\label{bkm:Ref484771795}\textstyleEndnoteText{К
стр.~\pageref{bkm:bm13}. — Немецкий текст тут испорчен.
Напечатано: «weil
}\textstyleEndnoteTextEmphasis{sie}\textstyleEndnoteText{ unmittelbar das
An-und-fursichsein
}\textstyleEndnoteTextEmphasis{ist}\textstyleEndnoteText{».
Между тем, ни в этой, ни в предыдущей фразе нет ни одного
существительного женского рода, к которому могло бы относиться местоимение
«sie». Приходится прибегать к конъектурам. Лассон предлагает читать:
«es...ist». Покойный Б. Г. Столпнер предлагал:
«sie...}\textstyleEndnoteText{sind». Нам представляется
более обоснованным чтение: «er (т.~е. der Begriff).. ist».
Перевод сделан в соответствии с этой последней конъектурой, в подтверждение
которой можно сослаться на два обстоятельства: 1) под словами «seine
Unterschiede», непосредственно предшествующими вышеприведенному
придаточному предложению, могут иметься в виду только «различия
}\textstyleEndnoteTextEmphasis{понятия}\textstyleEndnoteText{»;
2) третья фраза следующего абзаца начинается словами: «Weil
er (т.~е. der Begriff) das An-und-fursichsein ist». }}
\item 
\label{bkm:Ref484771931}\textstyleEndnoteText{К
стр.~\pageref{bkm:bm14}. — Ср. «Учение о бытии»,
стр.~62–63.}\textstyleEndnoteText{ }}
\item 
\label{bkm:Ref484771963}\textstyleEndnoteText{К
стр.~\pageref{bkm:bm15}. — Имеется в виду «философия
тожества» Шеллинга и его последователей. Ср. примечание 93 к т.~I «Науки
логики».}\textstyleEndnoteText{ }}
\item 
\label{bkm:Ref484771982}\textstyleEndnoteText{К
стр.~\pageref{bkm:bm16}. — «Разность» (Verschiedenheit) в
смысле многообразия, как совокупность разных видов одного
рода.}\textstyleEndnoteText{ }}
\item 
\label{bkm:Ref484772004}\textstyleEndnoteText{К
стр.~\pageref{bkm:bm17}. — «В понятии» означает у Гегеля то
же самое, что «в себе», т.~е. в неразвернутом виде, в возможности. См.,
например, т.~I «Науки логики», стр.~79 («в себе или в понятии») и стр.~153
(«в себе или в возможности»).}\textstyleEndnoteText{ }}
\item 
\label{bkm:Ref484772028}\textstyleEndnoteText{К
стр.~\pageref{bkm:bm18}. — Гегель здесь опять дает
неправильное истолкование учению Спинозы об отношении между субстанцией и
атрибутами (а также и модусами). См. примечание 92 к т.~I «Науки логики»,
где приводятся цитаты из сочинений Спинозы, показывающие, что атрибуты, по
учению Спинозы, не только
}\textstyleEndnoteTextEmphasis{мыслятся}\textstyleEndnoteText{
интеллектом (рассудком) как составляющие сущность субстанции,
но и объективно
}\textstyleEndnoteTextEmphasis{присущи}\textstyleEndnoteText{
субстанции безотносительно к
интеллекту.}\textstyleEndnoteText{ }}
\item 
\label{bkm:Ref484772047}\textstyleEndnoteText{К
стр.~\pageref{bkm:bm19}. — Имеются в виду Шеллинг и
шеллингианцы, а также Якоби и романтики. См. примечание 78 к т.~I «Науки
логики».}\textstyleEndnoteText{ }}
\item 
\label{bkm:Ref484772066}\textstyleEndnoteText{К
стр.~\pageref{bkm:bm20}. — См. примечание 18 к т.~I «Науки
логики».}\textstyleEndnoteText{ }}
\item 
\label{bkm:Ref484772084}\textstyleEndnoteText{К
стр.~\pageref{bkm:bm21}. — См. т.~I «Науки логики»,
стр.~164–169.}\textstyleEndnoteText{ }}
\item 
\label{bkm:Ref484772116}\textstyleEndnoteText{К
стр.~\pageref{bkm:bm22}. — См. выше,
стр.~\pageref{bkm:bm22a}.}\textstyleEndnoteText{ }}
\item 
\label{bkm:Ref484772133}\textstyleEndnoteText{К
стр.~\pageref{bkm:bm23}. — См. выше,
стр.~\pageref{bkm:bm23a}.}\textstyleEndnoteText{ }}
\item 
\label{bkm:Ref484772165}\textstyleEndnoteText{К
стр.~\pageref{bkm:bm24}. — В немецком издании 1816 г. стоит
«Abtraction» (sic!). В изданиях 1834 и 1841~гг., а также у Лассона
напечатано «Abstraktion». Мы считаем более вероятным чтение
«Attraktion».}\textstyleEndnoteText{ }}
\item 
\label{bkm:Ref484772182}\textstyleEndnoteText{К
стр.~\pageref{bkm:bm25}. — Латинское слово «abstractio»
означает «отвлечение» в смысле «удаления, оттаскивания, отделения». В этом
смысле Гегель и говорит здесь о «единичном» или «отдельном» (das Einzelne),
что оно представляет собой процесс самоотделения от других отдельных
предметов. Гегелевский термин «das Einzelne» вообще можно было бы всюду
переводить словом «отдельное» (как это в ряде мест делает В. И. Ленин), но
во многих случаях (особенно в главе об умозаключении, где Гегель применяет
буквенные схемы
}\textstyleEndnoteTextEmphasis{Е}\textstyleEndnoteText{ —~
}\textstyleEndnoteTextEmphasis{О}\textstyleEndnoteText{
—~}\textstyleEndnoteTextEmphasis{В}\textstyleEndnoteText{
и~т.~д.) удобнее пользоваться словом «единичное». Поэтому мы
и остановились на этом последнем слове для передачи термина «das
Einzelne».}\textstyleEndnoteText{ }}
\item 
\label{bkm:Ref484772231}\textstyleEndnoteText{К
стр.~\pageref{bkm:bm26}. — Гегель хочет сказать, что
«суждение» (das Urteil) этимологически означает в немецком языке
«перводеление» (Ur-teilen). В действительности это не так. Слово «Urteil»
представляет собой существительное, соответствующее глаголу «erteilen», и
первоначально означает, собственно говоря, «das, was erteilt wird», т.~е.
«то, что предоставляется, присуждается, постановляется» (судьей,
начальником, }\textstyleEndnoteText{законодателем). См.
}\textstyleEndnoteTextEmphasis{Kluge}\textstyleEndnoteText{, Etymo-
logisehes Worterbuch der deutschen Sprache, 9-te Aufl., Berlin und Leipzig
1921, S. 470. Даваемое Гегелем произвольное толкование
значения слова «Urteil» (не находящее ни малейшего подтверждения в истории
немецкого языка) нужно Гегелю для того, чтобы облегчить себе переход от
«понятия» (в узком смысле «понятия как такового») к «суждению», которое он
трактует как некоторое
}\textstyleEndnoteTextEmphasis{объективное}\textstyleEndnoteText{
(объективное в смысле объективного и абсолютного идеализма)
отношение между единичным и всеобщим (или между особенным и всеобщим, или,
наконец, между единичным и особенным). Неправильное толкование этимологии
слова «Urteil» в смысле «перводеления» встречается также у Шеллинга в его
вышедшей в 1800 г. «Системе трансцендентального идеализма»
[}\textstyleEndnoteTextEmphasis{Schelling}\textstyleEndnoteText{, Werke,
hrsg. v. O. Weiss, Bd II, S. 181).}}
\item 
\label{bkm:Ref484772286}\textstyleEndnoteText{К
стр.~\pageref{bkm:bm27}. — В немецком тексте всех изданий
напечатано: «Aber
}\textstyleEndnoteTextEmphasis{der}\textstyleEndnoteText{ Begriff...gibt
erst das Pradikat». Повидимому, это опечатка вместо «Aber
}\textstyleEndnoteTextEmphasis{den}\textstyleEndnoteText{
Begriff...»}\textstyleEndnoteText{ }}
\item 
\label{bkm:Ref484772335}\textstyleEndnoteText{К
стр.~\pageref{bkm:bm28}. — Это —~фактическая
ошибка, так как Аристотель умер на 63 году своей жизни (384–322 до
н.~э.).}\textstyleEndnoteText{ }}
\item 
\label{bkm:Ref484772364}\textstyleEndnoteText{К
стр.~\pageref{bkm:bm29}. — \ В немецком тексте всех изданий
напечатано: «bestimmte». Повидимому, это опечатка вместо:
«besondere».}\textstyleEndnoteText{ }}
\item 
\label{bkm:Ref484772393}\textstyleEndnoteText{К
стр.~\pageref{bkm:bm30}. — В изданиях 1816 и 1834~гг.:
«Zusammenfassen». В издании 1841~г., а также и у Лассона (у последнего без
указания разночтений): «Zusaramentreffen». Мы переводим согласно тексту
изданий 1816 и 1834~гг.}\textstyleEndnoteText{ }}
\item 
\label{bkm:Ref484772407}\textstyleEndnoteText{К
стр.~\pageref{bkm:bm31}. — В «Диалектике природы» Энгельс
всю главу «Большой логики» о Суждении, занимающую в издании 1841~г.
(которым пользовался Энгельс) страницы 63–115 (в настоящем русском издании
стр.~\pageref{bkm:bm31a} —~\pageref{bkm:bm31b}), называет
«гениальной», отмечая «внутреннюю истину и необходимость» даваемой Гегелем
классификации суждений, несмотря на всю ее «сухость» и частичную
«произвольность на первый взгляд». И Энгельс дает в связи с этой
классификацией блестящий образчик того,
}\textstyleEndnoteTextEmphasis{как}\textstyleEndnoteText{
надо «переворачивать» Гегеля с головы на ноги, извлекая
гениальные зерна истины из насквозь идеалистических, абстрактных и темных
рассуждений Гегеля, не свободных к тому же от того, что Ленин называл
«данью старой, формальной логике»
(}\textstyleEndnoteTextEmphasis{Ленин}\textstyleEndnoteText{,
Философские тетради, стр.~171). Дело в том, что Гегель в
главе о Суждении ухитрился втиснуть в свои четыре основных вида суждений
традиционную формально-логическую классификацию суждений по количеству,
качеству, отношению и модальности, — ближайшим образом в том
ее виде, в каком она дается у Канта (см.
}\textstyleEndnoteTextEmphasis{Кант}\textstyleEndnoteText{,
Критика чистого разума, пер. Лосского, Пгр. 1915, стр.~70–73;
ср.
}\textstyleEndnoteTextEmphasis{Кант}\textstyleEndnoteText{,
Логика, пер. Маркова, Пгр. 1915, стр.~93–101). Разбирая
гегелевскую классификацию суждений, Энгельс показывает (на конкретном
примере исторического развитая человеческих суждений о превращении одних
форм движения в другие), что «то, что у Гегеля является развитием
мыслительной формы суждения как такового, выступает здесь перед нами как
развитие наших, покоящихся на
}\textstyleEndnoteTextEmphasis{эмпирической}\textstyleEndnoteText{
основе, теоретических знаний о природе движения вообще»
(}\textstyleEndnoteTextEmphasis{Engels}\textstyleEndnoteText{,
Anti-Duhring und Dialektik der Natur, M. — L. 1935, S. 663).
При этом Энгельс сводит четыре основных группы суждений
гегелевской классификации к
}\textstyleEndnoteTextEmphasis{трем}\textstyleEndnoteText{
основным видам, объединяя гегелевские суждения рефлексии и
суждения необходимости в один вид суждений особенности (или частности).
Соответственно с этим гегелевские суждения наличного бытия характеризуются
у Энгельса как суждения единичности (или отдельности), а гегелевские
суждения понятия —~как суждения всеобщности. Относительно
расшифровки одного места из этого замечательного отрывка «Диалектики
природы» см. рецензию В. Брушлинского на новое немецкое издание
«Анти-Дюринга» и «Диалектики природы» в журнале «Под знаменем марксизма»,
1937, № 7, стр.~171–172.}\textstyleEndnoteText{ }}
\item 
\label{bkm:Ref484772436}\textstyleEndnoteText{К
стр.~\pageref{bkm:bm32}. — Ср. приведенное в примечании
\ref{bkm:Ref484792854} замечание Маркса о «формальном
содержании» суждений и умозаключений.}\textstyleEndnoteText{ }}
\item 
\label{bkm:Ref484772458}\textstyleEndnoteText{К
стр.~\pageref{bkm:bm33}. — См. т.~I «Науки логики»,
стр.~394–399.}\textstyleEndnoteText{ }}
\item 
\label{bkm:Ref484772480}\textstyleEndnoteText{К
стр.~\pageref{bkm:bm34}. — См. выше,
стр.~\pageref{bkm:bm34a}.}\textstyleEndnoteText{ }}
\item 
\label{bkm:Ref484772504}\textstyleEndnoteText{К
стр.~\pageref{bkm:bm35}. — См. выше,
стр.~\pageref{bkm:bm35a}.}\textstyleEndnoteText{ }}
\item 
\label{bkm:Ref484772536}\textstyleEndnoteText{К
стр.~\pageref{bkm:bm36}. — Для обозначения диалектических
категорий «всеобщее», «особенное» и «единичное», являющихся, как отмечает
Энгельс (}\textstyleEndnoteTextEmphasis{Engels}\textstyleEndnoteText{,
Dialektik der Natur, M.  —~L. 1935, S. 664), теми «тремя
определениями, в которых движется все Учение о понятии», Гегель пользуется
немецкими словами «Аllgemeines», «Besonderes», «Einzelnes».
Для обозначения же трех видов суждения рефлексии,
соответствующих принятой в формальной логике классификации суждений «по
количеству», Гегель употребляет заимствованные из латинского языка
прилагательные «singulär», «partikulär», «universell». Этого
терминологического различения придерживается и Энгельс в цитированном нами
в примечании \ref{bkm:Ref484772407} отрывке о классификации
суждений. Поэтому}\textstyleEndnoteText{ мы сочли
необходимым провести это различение и в русском переводе. Термины
«сингулярный», «партикулярный», «универсальный» удобны еще и потому, что
пользование ими облегчает переход к таким выражениям, как
«партикуляризация», «партикуляризировать», служащим для перевода
гегелевских «Partikularisation», «partikularisieren».}}
\end{enumerate}

\textstyleEndnoteText{Что касается немецких слов «Allgemeines»
и «Besonderes», то интересное указание на их первоначальное значение
имеется в письме Маркса Энгельсу от 26 марта 1868~г. Маркс пишет: «Что
сказал бы старый Гегель, если бы он на том свете узнал, что
}\textstyleEndnoteTextEmphasis{das Allgemeine}\textstyleEndnoteText{
(}\textstyleEndnoteTextEmphasis{общее}\textstyleEndnoteText{)
на немецком и на северных наречиях означает не что иное, как
общинную землю, а что }\textstyleEndnoteTextEmphasis{das
Sundre}\textstyleEndnoteText{
—~}\textstyleEndnoteTextEmphasis{Besondre}\textstyleEndnoteText{
(}\textstyleEndnoteTextEmphasis{особенное}\textstyleEndnoteText{)
есть не что иное, как выделенное из общей земли особое
владение? Таким образом, ведь совершенно же очевидно, что логические
категории возникают из «нашего общения» [из общественных отношений]»
(}\textstyleEndnoteTextEmphasis{Маркс и
Энгельс}\textstyleEndnoteText{, Письма, пер. Адоратского,
М. — Л. 1931, стр.~232). }}

\begin{enumerate}
\item 
\label{bkm:Ref484772564}\textstyleEndnoteText{К
стр.~\pageref{bkm:bm37}. — «Всеобщее рефлексии» (das
Allgemeine der Reflexion), т.~е. всеобщее, как оно дано в рефлексии,
противопоставляется у Гегеля «всеобщему понятия» (das Allgemeine des
Begriffes), т.~е. всеобщему, как оно дано в
понятии.}\textstyleEndnoteText{ }}
\item 
\label{bkm:Ref484772654}\textstyleEndnoteText{К
стр.~\pageref{bkm:bm38}. — Слово «всякость» не вполне
передает тот смысл, который вкладывается Гегелем в немецкое слово
«Allheit»: «всякость» образована от слова «всякий», между тем как тут идет
речь о производном от слова «все» (alle). «Allheit» означает у Гегеля
эмпирическую }\textstyleEndnoteTextEmphasis{совокупность
всех}\textstyleEndnoteText{, «эмпирическую всеобщность», как
он сам поясняет на стр.~\pageref{bkm:bm38a}, — такую «форму
всеобщности, на которую обыкновенно раньше всего набредает рефлексия»
(}\textstyleEndnoteTextEmphasis{Гегель}\textstyleEndnoteText{,
Соч., т.~I, стр.~283).}\textstyleEndnoteText{ }}
\item 
\label{bkm:Ref484772671}\textstyleEndnoteText{К
стр.~\pageref{bkm:bm39}. — Ср. т.~I «Науки логики»,
стр.~222.}\textstyleEndnoteText{ }}
\item 
\label{bkm:Ref484772700}\textstyleEndnoteText{К
стр.~\pageref{bkm:bm40}. — Это —~семь
основных цветов солнечного спектра согласно Ньютону (см.
}\textstyleEndnoteTextEmphasis{Ньютон}\textstyleEndnoteText{,
Оптика, пер. с прим. С. И. Вавилова, М. — Л.
1927).}\textstyleEndnoteText{ }}
\item 
\label{bkm:Ref484772712}\textstyleEndnoteText{К
стр.~\pageref{bkm:bm41}. — Гегель имеет в виду гетевское
учение о цветах, согласно которому «для порождения цвета нужны свет и мрак,
светлое и темное, или, пользуясь более общей формулой, свет и не-свет»
[}\textstyleEndnoteTextEmphasis{Goethes}\textstyleEndnoteText{ Sämtliche
Werke, Jubiläums-Ausgabe, Stuttgart und Berlin, Cotta 1902, Bd. 40, S. 73).
Трактат Гете «Zur Farbenlehre», откуда взята эта цитата,
вышел в 1810 г. В «Философии природы» Гегель развивает теорию цветов,
близкую к гетевской. Об ошибочности этой теории цветов см. примечание А. А.
Максимова к стр.~249 «Философии природы»
(}\textstyleEndnoteTextEmphasis{Гегель}\textstyleEndnoteText{,
Соч., т.~II, М. — Л. 1934, стр.~601). В
«Диалектике природы» Энгельса мы читаем: «Гегель построил теорию света и
цветов из чистой мысли и при этом впадает в
}\textstyleEndnoteTextEmphasis{грубейшую
эмпирию}\textstyleEndnoteText{ доморощенного
}\textstyleEndnoteText{филистерского опыта (хотя, впрочем, с
известным основанием, так как этот пункт тогда еще не был выяснен),
например, когда он выдвигает против Ньютона смешивание красок, практикуемое
живописцами»
(}\textstyleEndnoteTextEmphasis{Engels}\textstyleEndnoteText{,
Anti-Duhring und Dialektik der Natur, M. — L. 1935, S.
681).}\textstyleEndnoteText{ }}
\item 
\label{bkm:Ref484772721}\textstyleEndnoteText{К
стр.~\pageref{bkm:bm42}. — В
немецкой тексте: «Von diesen aber muss die
eine...» Под «diesen» Гегель, повидимому,
имеет в виду 1) род (die Gattung) в его простом единстве с
самим собой и 2) род в его расчлененности на виды, а под «die eine»
—~род в первом из этих двух его
аспектов.}\textstyleEndnoteText{ }}
\item 
\label{bkm:Ref484772739}\textstyleEndnoteText{К
стр.~\pageref{bkm:bm43}. — Согласно гете-гегелевской теории
цветов к двум основным противоположным цветам
—~}\textstyleEndnoteTextEmphasis{желтому}\textstyleEndnoteText{
(в основе которого лежит светлое) и
}\textstyleEndnoteTextEmphasis{синему}\textstyleEndnoteText{
(в основе которого лежит темное) присоединяется еще третий
—~}\textstyleEndnoteTextEmphasis{зеленый}\textstyleEndnoteText{,
представляющий собой «простою смесь, обыкновенную
нейтральность желтого и синего»
(}\textstyleEndnoteTextEmphasis{Гегель}\textstyleEndnoteText{,
Философия природы, М. — Л. 1934,
стр.~266–267).}\textstyleEndnoteText{ }}
\item 
\label{bkm:Ref484772753}\textstyleEndnoteText{К
стр.~\pageref{bkm:bm44}. — Это опять выпад против
ньютоновской теории цветов, в которой фиолетовый, оранжевый, темносиний и
голубой цвета рассматриваются как самостоятельные, первоначальные цвета,
занимающие определенные места в спектре.}\textstyleEndnoteText{ }}
\item 
\label{bkm:Ref484772854}\textstyleEndnoteText{К
стр.~\pageref{bkm:bm45}. — Дело в том, что в разделительном
суждении род содержит в себе принцип диференциации
}\textstyleEndnoteTextEmphasis{только для
видов}\textstyleEndnoteText{, а не для единичностей
}\textstyleEndnoteText{(не для индивидов) Эти последние
лежат еще }\textstyleEndnoteTextEmphasis{за
пределами}\textstyleEndnoteText{ того процесса определения
или диференциации, который имеет место в «объективной всеобщности»,
составляющей содержание разделительного суждения. Поэтому, поскольку
ассерторическое суждение (с его «конкретной всеобщностью») по Гегелю
непосредственно вырастает из разделительного суждения, постольку в нем еще
нет необходимой внутренней связи между единичным (индивидом) и всеобщим
(понятием). }}
\item 
\label{bkm:Ref486285580}\label{bkm:Ref484772979}\textstyleEndnoteText{К
стр.~\pageref{bkm:bm46}. — Немецкое слово «Schluss»
можно переводить трояким образом: 1) «умозаключение», 2)
«заключение» и 3) «силлогизм». В настоящем переводе «Schluss»
чаще всего передается словом «умозаключение», в отдельных
случаях —~словом «силлогизм» (особенно когда речь идет о
«среднем термине силлогизма»). Всюду пользоваться термином «силлогизм» для
перевода немецкого «Schluss» неудобно по той причине, что в
русской философской литературе слово «силлогизм» употребляется в более
узком смысле умозаключения от общего к частному, между тем как у Гегеля
речь идет также и об индуктивных умозаключениях, умозаключениях по аналогии
и~т.~д. Что касается термина «заключение», то его пришлось оставить для
перевода немецких терминов «Schlusssatz» и «Konklusion»,
поскольку слово «вывод» не всегда пригодно для передачи этих
терминов и служит для перевода слова «Folgerung».
Необходимость переводить «Schluss» через
«умозаключение» вызывается еще и тем, что глагол «schliessen»
в большинстве случаев можно переводить только через
«умозаключать», так как перевод его через «заключать» привел бы к
шероховатостям и недоразумениям.}}
\end{enumerate}

\textstyleEndnoteText{Необходимо, однако, отметить, что
русское слово «умозаключение» не вполне соответствует немецкому слову
«Schluss», особенно в том значении этого последнего, которое
ему придает Гегель. Для Гегеля «умозаключение» (так же, как и «понятие» и
«суждение») имеет прежде всего
}\textstyleEndnoteTextEmphasis{объективное}\textstyleEndnoteText{
значение (объективное в смысле объективного и абсолютного
идеализма). Он рассматривает «den Schluss» не как нечто
такое, что имеет место в «уме», а как объективное соотношение моментов
самогó предмета или самогó понятия (это для Гегеля одно и то же).
Соответственно этому он толкует слово «Schluss» как
«Zusammenschliessen » («смыкание воедино»,
«сключение»).}\textstyleEndnoteText{ }}

\begin{enumerate}
\item 
\label{bkm:Ref484778605}\textstyleEndnoteText{К
стр.~\pageref{bkm:bm47}. — Под этим «претендующим на
разумность познанием» (так же как и под «обыденной болтовней о разуме» в
предыдущем предложении) имеется в виду «философия веры» Фридриха-Генриха
Якоби (174}\textstyleEndnoteText{3–1819), центральная мысль
которой заключалась в метафизическом противопоставлении
}\textstyleEndnoteTextEmphasis{рассудочному}\textstyleEndnoteText{
знанию знания непосредственного, иррационального,
мистического, не допускающего обоснования и доказательств. Это
непосредственное иррациональное знание Якоби обозначал терминами «вера»,
«разум», «чувство», «духовное чутье», «откровение». }}
\item 
\label{bkm:Ref484778678}\textstyleEndnoteText{К
стр.~\pageref{bkm:bm48}. — Гегель намекает на то, что
латинское слово «concretus» происходит от глагола
«concrescere», первоначальное значение которого
—~«срастаться, сращиваться».}\textstyleEndnoteText{ }}
\item 
\label{bkm:Ref484778655}\textstyleEndnoteText{К
стр.~\pageref{bkm:bm49}. — Т.~е. «единичное
—~особенное —~всеобщее». В «Малой логике»
Гегель дает такой пример: «Эта роза красна, красное есть цвет; роза,
следовательно, обладает цветом»
(}\textstyleEndnoteTextEmphasis{Гегель}\textstyleEndnoteText{,
Соч , т.~I, стр.~291).}\textstyleEndnoteText{ }}
\item 
\label{bkm:Ref484778694}\textstyleEndnoteText{К
стр.~\pageref{bkm:bm50}. — Это —~известное
место из «Первой аналитики» Аристотеля (в т.~I академического Берлинского
издания 1831~г., под ред. И. Беккера, стр.~25b, строки
32–35) в несколько вольном переводе Гегеля. Точнее это место гласит: «Если
три термина так относятся друг к другу, что последний имеется во всем
среднем термине, а этот средний термин либо имеется, либо отсутствует во
всем первом, то в отношении крайних терминов необходимо имеет место полный
силлогизм».}\textstyleEndnoteText{ }}
\item 
\label{bkm:Ref485477266}\label{bkm:Ref484778728}\textstyleEndnoteText{К
стр.~\pageref{bkm:bm51} —~ Эта «вторая фигура» умозаключения
соответствует «третьей фигуре» Аристотеля, точно так же как «третья фигура»
Гегеля соответствует «второй фигуре» Аристотеля.}}
\end{enumerate}

\textstyleEndnoteText{В «Малой логике» Гегель пишет формулу
своей «второй фигуры» наоборот:
«}\textstyleEndnoteTextEmphasis{B}\textstyleEndnoteText{
—~}\textstyleEndnoteTextEmphasis{Е}\textstyleEndnoteText{
—~}\textstyleEndnoteTextEmphasis{О}\textstyleEndnoteText{»
(см.
}\textstyleEndnoteTextEmphasis{Гегель}\textstyleEndnoteText{,
Соч., т.~I, стр.~293). Такое начертание встречается и
в}\textstyleEndnoteText{ «Большой логике» на стр.~137, 138 и
152. Дело в том, что для Гегеля основным и решающим в умозаключении
является именно средний термин как
}\textstyleEndnoteTextEmphasis{опосредствующий}\textstyleEndnoteText{
крайние термины, тогда как расстановка крайних терминов
(какой из них стоит на первом месте и какой на последнем) не может служить
основанием для классификации силлогизмов. }}

\begin{enumerate}
\item 
\label{bkm:Ref485477304}\label{bkm:Ref484778769}\textstyleEndnoteText{К
стр.~\pageref{bkm:bm52}. — В этом абзаце Гегель имеет в виду
практикуемое в формальной логике «сведение» модусов третьей (а равно и
второй) фигуры к модусам первой фигуры. Для иллюстрации возьмем
какой-нибудь тривиальный пример умозаключения третьей (по Гегелю
—~второй) фигуры: «птицы имеют когти; птицы суть двуногие
существа; следовательно, некоторые двуногие существа имеют когти». Средним
термином в этом силлогизме служат «птицы»; бóльшим термином служит
«обладание когтями», а меньшим термином —~«двуногость». Для
сведения этого силлогизма к первой фигуре надо перевернуть меньшую посылку
(«птицы суть двуногие существа») или, выражаясь языком школьной логики,
«обратить ее посредством ограничения». Тогда силлогизм примет такой вид:
«птицы имеют когти; некоторые двуногие существа суть птицы; следовательно,
некоторые двуногие существа имеют когти». Ввиду того что крайние термины
«обладание когтями» и «двуногость» находятся во внешнем, безразличном
отношении друг к другу, они могут меняться местами, и заключение может с
таким же правом гласить: «некоторые снабженные когтями животные имеют две
ноги».}}
\end{enumerate}

\textstyleEndnoteText{Чтобы более наглядно выявить характер
}\textstyleEndnoteTextEmphasis{единичности}\textstyleEndnoteText{,
который по Гегелю присущ среднему термину рассматриваемой
фигуры, возьмем еще такой пример: «Харьков лежит на 50-й параллели;
Харьков —~большой город; следовательно, некоторые большие
города лежат на 50-й параллели, или: некоторые лежащие на 50-й параллели
города имеют большие размеры».}}


\textstyleEndnoteText{Необходимо, впрочем, отметить, что хотя
Гегель и намекает здесь на формально-логическое «сведение» одной фигуры к
другой, но сам он придает силлогистическим фигурам совершенно другой смысл,
чем какой они имеют в формальной логике. Для Гегеля суть дела состоит в
том, какое из трех «определений понятия» в том или ином случае служит
«средним термином»; т.~е. выполняет функцию опосредствования. Поэтому
приведенные нами примеры (так же как и пример в нижеследующем примечании
\ref{bkm:Ref485476947}) иллюстрируют не гегелевское учение о
фигурах силлогизма, а только гегелевские намеки на формально-логическую
трактовку этих фигур. Гегель указывает, что те три обособленные
предложения, из которых конструируются школьные силлогизмы, представляют
собой лишь внешнюю, субъективную форму (см. в тексте,
стр.~\pageref{bkm:bm52a}). Сам он приводит такие примеры
истинного силлогизма и его трех фигур: 1) взаимоотношения между «тремя
членами философской науки, т.~е. логической идеей, природой и духом»
(}\textstyleEndnoteTextEmphasis{Гегель}\textstyleEndnoteText{,
Соч., т.~I, стр.~294–295), 2) взаимоотношения между членами
солнечной системы
(}\textstyleEndnoteTextEmphasis{Гегель}\textstyleEndnoteText{,
Соч., т.~II, стр.~135–136), 3) взаимоотношения
между элементами государства
(}\textstyleEndnoteTextEmphasis{Гегель}\textstyleEndnoteText{,
Соч., т.~I, стр.~310; см. также ниже в тексте,
стр.~\pageref{bkm:bm52b}}\textstyleEndnoteText{ —~\pageref{bkm:bm52c})
и~т.~д. }}

\begin{enumerate}
\item 
\label{bkm:Ref484778821}\textstyleEndnoteText{К
стр.~\pageref{bkm:bm53}. — В «Малой логике» Гегель пишет
формулу своей «третьей фигуры» наоборот:
}\textstyleEndnoteTextEmphasis{О}\textstyleEndnoteText{
—~}\textstyleEndnoteTextEmphasis{В}\textstyleEndnoteText{
—~}\textstyleEndnoteTextEmphasis{Е}\textstyleEndnoteText{
(см.
}\textstyleEndnoteTextEmphasis{Гегель}\textstyleEndnoteText{,
Соч., т.~I, стр.~294, а также выше, примечание
\ref{bkm:Ref485477266}). Этой последней формулой пользуется
Маркс при характеристике товарно-денежного обращения. Маркс пишет: в
процессе обращения «Т —~Д —~Т оба крайние
члена Т находятся, под углом зрения формы, не в одинаковом отношении к Д.
Первый Т относится к деньгам как особенный товар к всеобщему товару, между
тем как деньги относятся ко второму Т как всеобщий товар к единичному
товару. Следовательно, абстрактно-логически Т —~Д
—~Т может быть сведено к форме силлогизма О
—~В —~Е, где особенность образует первый
крайний член, всеобщность —~связывающий средний член и
единичность —~последний крайний член»
(}\textstyleEndnoteTextEmphasis{Маркс}\textstyleEndnoteText{,
К критике политической экономии, Партиздат, 1935,
стр.~98).}\textstyleEndnoteText{ }}
\item 
\label{bkm:Ref484778834}\label{bkm:Ref485476947}\textstyleEndnoteText{К
стр.~\pageref{bkm:bm54}. — Опять намек на практикуемое в
школьной логике «сведение» модусов второй (по Гегелю третьей) фигуры к
модусам первой фигуры }\textstyleEndnoteText{(ср. примечание
\ref{bkm:Ref485477304}). Возьмем тривиальный пример: «рыбы
не имеют легких; киты имеют легкие; следовательно, киты не суть рыбы». Для
сведения этого силлогизма к силлогизму первой фигуры нужно перевернуть
большую посылку. Тогда мы получим: «животные, обладающие легкими, не суть
рыбы; киты обладают легкими; следовательно, киты не рыбы». В
рассматриваемой фигуре заключение всегда имеет форму отрицательного
суждения. Поэтому в нем всегда можно сделать «обращение»: субъект поставить
на место предиката, а предикат —~на место субъекта. Вместо
«киты не суть рыбы» получим: «рыбы не суть киты». Об этом безразличном
отношении между субъектом и предикатом
}\textstyleEndnoteTextEmphasis{заключения}\textstyleEndnoteText{
Гегель и говорит в следующей фразе
текста.}\textstyleEndnoteText{ }}
\item 
\label{bkm:Ref484778920}\textstyleEndnoteText{К
стр.~\pageref{bkm:bm55}. — См. т.~I «Науки логики»,
стр.~9.}\textstyleEndnoteText{ }}
\item 
\label{bkm:Ref484778947}\textstyleEndnoteText{К
стр.~\pageref{bkm:bm56}. — См. примечание
\ref{bkm:Ref484772654}.}\textstyleEndnoteText{ }}
\item 
\label{bkm:Ref484778985}\textstyleEndnoteText{К
стр.~\pageref{bkm:bm57}. — Под
«}\textstyleEndnoteTextEmphasis{E}\textstyleEndnoteText{»
Гегель имеет здесь в виду
}\textstyleEndnoteTextEmphasis{совокупность}\textstyleEndnoteText{
всех единичных какого-нибудь рода. Пользуясь примером,
приводимым Гегелем в следующей фразе, можно вместо
«}\textstyleEndnoteTextEmphasis{В}\textstyleEndnoteText{
—~}\textstyleEndnoteTextEmphasis{Е}\textstyleEndnoteText{»
подставить такое суждение: «Четвероногие животные суть: лев,
слон, медведь, лошадь и~т.~д.». Для бóльшей наглядности продолжим этот
пример. Пусть второй посылкой будет суждение: «лев, слон, медведь, лошадь
и~т.~д. имеют хвост». Тогда заключение будет гласить: «все четвероногие
имеют хвост».}\textstyleEndnoteText{ }}
\item 
\label{bkm:Ref484779009}\textstyleEndnoteText{К
стр.~\pageref{bkm:bm58}. — Ср. замечание Энгельса о том, что
постоянные перевороты в индуктивных классификациях животного и
растительного мира служат «прекрасным подтверждением гегелевского положения
о том, что индуктивное умозаключение по
}\textstyleEndnoteText{существу является проблематическим»
(}\textstyleEndnoteTextEmphasis{Engels}\textstyleEndnoteText{, Dialektik
der Natur, M. — \ L. 1935, S. 653).}\textstyleEndnoteText{ }}
\item 
\label{bkm:Ref484779035}\textstyleEndnoteText{К
стр.~\pageref{bkm:bm59}. — Примером категорического
умозаключения может служить такой силлогизм: «роза есть растение; растение
нуждается во влаге; следовательно, роза нуждается во влаге». Или: «роза
есть растение; растение есть организм; следовательно, роза есть
организм».}\textstyleEndnoteText{ }}
\item 
\label{bkm:Ref484779051}\textstyleEndnoteText{К
стр.~\pageref{bkm:bm60}. — Речь идет о среднем термине
категорического умозаключения.}\textstyleEndnoteText{ }}
\item 
\label{bkm:Ref484779077}\textstyleEndnoteText{К
стр.~\pageref{bkm:bm61}. — Момент непосредственности выражен
здесь словом «есть», момент опосредствования —~словом
«следовательно».}\textstyleEndnoteText{ }}
\item 
\label{bkm:Ref484779089}\textstyleEndnoteText{К
стр.~\pageref{bkm:bm62}. — Для бóльшей наглядности возьмем
простенький пример, аналогичный тому примеру, которым Гегель иллюстрирует
тождество причины и действия в «Учении о сущности»: «если идет дождь, то на
улице мокро; дождь идет; следовательно, на улице мокро». Гегель указывает,
что «та же самая вода, которая составляет дождь, и есть мокротá» (см. т.~I
«Науки логики», стр.~453). Поэтому тут и получается тождество
опосредствующего («дождь идет») и опосредствованного («на улице
мокро»).}\textstyleEndnoteText{ }}
\item 
\label{bkm:Ref484779124}\textstyleEndnoteText{К
стр.~\pageref{bkm:bm63}. — Это намерение Гегель выполнил в
1829~г. в виде «Лекций о доказательствах бытия божия». Перед самой своей
смертью, осенью 1831~г., он собирался издать их отдельной книгой, но не
успел сделать этого. Любопытна первая фраза этих лекций: «Эти лекции могут
рассматриваться как дополнение к логике: они трактуют о некоторой
своеобразной форме тех основных моментов, которые выступают в логике»
(}\textstyleEndnoteTextEmphasis{Hegel}\textstyleEndnoteText{, Die Beweise
vom Dasein Gottes, Neu hrsg. v. Lasson, Leipzig 1930, S. 1).
Здесь, пожалуй, ярче всего проявляется связь гегелевской
логики с самой настоящей поповщиной.}\textstyleEndnoteText{ }}
\item 
\label{bkm:Ref484779373}\textstyleEndnoteText{К
стр.~\pageref{bkm:bm64}. — Т.~е., «так как ветра долгое
время не было» (буквально: «так как ветер долгое время
}\textstyleEndnoteTextEmphasis{был}\textstyleEndnoteText{
без того, чтобы
}\textstyleEndnoteTextEmphasis{существовать}\textstyleEndnoteText{»),}\textstyleEndnoteText{
}}
\item 
\label{bkm:Ref484779387}\textstyleEndnoteText{К
стр.~\pageref{bkm:bm65}. — Т.~е. «он долгое время не писал
мне» (буквально: «он долгое время был без того, чтобы писать
мне»).}\textstyleEndnoteText{ }}
\item 
\label{bkm:Ref484779404}\textstyleEndnoteText{К
стр.~\pageref{bkm:bm66}. — См. выше,
стр.~\pageref{bkm:bm66a} —~\pageref{bkm:bm66b} (в конце
параграфа о проблематическом суждении).}\textstyleEndnoteText{ }}
\item 
\label{bkm:Ref484779429}\textstyleEndnoteText{К
стр.~\pageref{bkm:bm67}. — Имеется в виду субъективный
идеализм Фихте.}\textstyleEndnoteText{ }}
\item 
\label{bkm:Ref484779460}\label{bkm:Ref485478669}\textstyleEndnoteText{К
стр.~\pageref{bkm:bm68}. — В издании 1841~г. (а также и у
Лассона) вместо «bewirkt wird» (как напечатано в изданиях 1816 и 1834 гг.)
по ошибке напечатано «Bеstimmt wird». То обстоятельство, что Лассон не
указывает этого места в своем перечне разночтений, свидетельствует о том,
что он в основу своего издания}\textstyleEndnoteText{
положил текст издания 1841~г. и недостаточно тщательно сверил
его с изданиями 1816 и 1834~гг., содержащими меньшее количество ошибок и
опечаток. Об этом свидетельствует также и случай, указанный в примечании
\ref{bkm:Ref484772393}, и ряд других мест текста. Ср. также
примечание \ref{bkm:Ref484779591}.}\textstyleEndnoteText{ }}
\item 
\label{bkm:Ref484779529}\textstyleEndnoteText{К
стр.~\pageref{bkm:bm69}. — Для иллюстрации категорий
«сообщения» и «распределения» какой-нибудь определенности можно
воспользоваться примером, приводимым у Гегеля в параграфе о
«специфицирующей мере» (см. т.~I «Науки логики», стр.~269–270). Нагретый
воздух
}\textstyleEndnoteTextEmphasis{сообщает}\textstyleEndnoteText{
теплоту находящимся в нем предметам. Эти последние
воспринимают температуру окружающего их воздуха в меру своей теплоемкости.
Температура воздуха
}\textstyleEndnoteTextEmphasis{распределяется}\textstyleEndnoteText{
между находящимися в нем предметами. Она в одно и то же время
становится
}\textstyleEndnoteTextEmphasis{всеобщей}\textstyleEndnoteText{
(}\textstyleEndnoteTextEmphasis{генерализируется}\textstyleEndnoteText{),
распространяясь на все другие находящиеся здесь тела, и
}\textstyleEndnoteTextEmphasis{партикуляризируется}\textstyleEndnoteText{,
будучи воспринимаема каждым телом по-разному, в меру своей
специфической теплоемкости.}\textstyleEndnoteText{ }}
\item 
\label{bkm:Ref484779549}\textstyleEndnoteText{К стр:
\pageref{bkm:bm70}. — В немецком тексте всех изданий
напечатано: «Ег ist der Schlusssatz». Повидимому, это
опечатка, вместо «Es ist...»}\textstyleEndnoteText{ }}
\item 
\label{bkm:Ref484779576}\textstyleEndnoteText{К
стр.~\pageref{bkm:bm71}. — Гегель имеет в виду главным
образом то понятие судьбы, которое выступает в античной трагедии. В т.~III
«Эстетики», в главе о драматической поэзии, Гегель более подробно
останавливается на характеристике этой «судьбы». Он возражает против
понимания этой судьбы как слепого рока. «Разумность судьбы, —
говорит Гегель, — заключается именно в том,
что высшая сила, властвующая над отдельными богами и людьми, не может
потерпеть, чтобы силы, односторонне делающие себя самостоятельными и этим
преступающие границы своего права, равно как и проистекающие отсюда
конфликты, получили устойчивое существование»
(}\textstyleEndnoteTextEmphasis{Hegel}\textstyleEndnoteText{, Sämtliche
Werke, hrsg. v. Glockner, Bd. XIV, S. 554). И далее: «В
античной трагедии спасает и отстаивает гармонию нравственной субстанции
против напора делающих себя самостоятельными и потому вступающих в коллизию
частных сил вечная справедливость как
}\textstyleEndnoteTextEmphasis{абсолютная мощь
судьбы}\textstyleEndnoteText{, и она, в силу внутренней
разумности ее управления, доставляет нам удовлетворение зрелищем самой
гибели индивидуумов» (там же, стр.~572).}\textstyleEndnoteText{ }}
\item 
\label{bkm:Ref484779591}\textstyleEndnoteText{К
стр.~\pageref{bkm:bm72}. — В
изданиях 1816 и 1834~гг.:
«Andererseits aber ist
}\textstyleEndnoteTextEmphasis{es}\textstyleEndnoteText{ das Aufheben...».
В издании 1841~г., а также у Лассона (у последнего
—~без указания разночтений) пропущено слово «es». Ср.
примечания \ref{bkm:Ref484772393} и
\ref{bkm:Ref485478669}.}\textstyleEndnoteText{ }}
\item 
\label{bkm:Ref484779619}\textstyleEndnoteText{К
стр.~\pageref{bkm:bm73}. — Как ни темны эти рассуждения
Гегеля о «центре» и «центральности», в них есть определенное рациональное
зерно. Под «центром» и «центральностью» Гегель имеет в виду прежде всего
центр тяжести небесных тел, например, земли. Всякое движение, совершающееся
на поверхности земли или в окружающей землю атмосфере, находится в
необходимой и существенной зависимости от силы тяготения, направленной к
центру земли. То сопротивление движущемуся телу, которое оказывается
воздухом и поверхностью земли, в свою очередь обусловлено в конечном счете
силой тяготения и является существенно-необходимым моментом всякого
движения, рассматриваемого в земной механике, — таким
моментом, от которого нельзя отвлечься, не впадая в «пустую абстракцию».
Рациональное зерно этих рассуждений Гегеля выступает особенно явственно,
если их сопоставить со следующими рассуждениями Энгельса в «Диалектике
природы»: «Возьмем, — пишет Энгельс, —
какую-нибудь телесную массу на самой нашей земле. Благодаря
тяжести она связана с землей, подобно тому как Земля, с своей стороны,
связана с Солнцем; но в отличие от земли эта масса неспособна на свободное
планетарное движение. Она может быть приведена в движение только при помощи
внешнего толчка. Но и в этом случае, по миновании толчка, ее движение
вскоре прекращается либо благодаря действию одной лишь тяжести, либо же
благодаря этому действию в соединении с сопротивлением среды, в которой
движется наша масса. Однако }\textstyleEndnoteTextEmphasis{и
это сопротивление является в последнем счете действием
тяжести}\textstyleEndnoteText{, без которой Земля не имела
бы никакой сопротивляющейся среды, никакой атмосферы на своей поверхности»
}\textstyleEndnoteText{(}\textstyleEndnoteTextEmphasis{Энгельс}\textstyleEndnoteText{,
Диалектика природы, Партиздат, 1936, стр.~133–134.
Подчеркнуто мной. —
\ }\textstyleEndnoteTextEmphasis{В}\textstyleEndnoteText{.
}\textstyleEndnoteTextEmphasis{Б}\textstyleEndnoteText{.) }}
\item 
\label{bkm:Ref484779643}\textstyleEndnoteText{К
стр.~\pageref{bkm:bm74}. — Это —~гегелевская
«третья фигура» силлогизма
(}\textstyleEndnoteTextEmphasis{Е}\textstyleEndnoteText{
—~}\textstyleEndnoteTextEmphasis{В}\textstyleEndnoteText{
—~}\textstyleEndnoteTextEmphasis{О}\textstyleEndnoteText{),
причем в роли единичности (отдельности) выступают вторичные
или относительные центры, в роли всеобщности —~первичный или
абсолютный центр тяжести, а в роли особенности (частности)
—~несамостоятельные объекты. В «Философии природы» Гегель
применяет эту логическую схему к взаимоотношению между членами солнечной
системы. Солнце выступает в роли всеобщности, Земля (а также и другие
планеты) —~в роли единичности, а «несамостоятельные тела»
(Луна и кометы) —~в роли особенности. См.
}\textstyleEndnoteTextEmphasis{Гегель}\textstyleEndnoteText{,
Соч., т.~II, стр.~136.}\textstyleEndnoteText{ }}
\item 
\label{bkm:Ref484779685}\textstyleEndnoteText{К
стр.~\pageref{bkm:bm75}. — Это —~гегелевская
«вторая фигура» силлогизма
(}\textstyleEndnoteTextEmphasis{О}\textstyleEndnoteText{
—~}\textstyleEndnoteTextEmphasis{Е}\textstyleEndnoteText{
—~}\textstyleEndnoteTextEmphasis{В}\textstyleEndnoteText{).
В немецком тексте всех изданий конец этой фразы грамматически
искажен. Напечатано: «deren... Vereinzelung von ihr getragen
}\textstyleEndnoteTextEmphasis{werden}\textstyleEndnoteText{».
Здесь надо либо вместо «werden» поставить «wird» (как это и
сделано в нашем переводе), либо после слова «Vereinzelung» прибавить слова
вроде «und Unselbstandigkeit» (или: «und
Aeusserlichkeib»).}\textstyleEndnoteText{ }}
\item 
\label{bkm:Ref484779698}\textstyleEndnoteText{К
стр.~\pageref{bkm:bm76}. — Это —~гегелевская
«первая фигура» силлогизма,
(}\textstyleEndnoteTextEmphasis{Е}\textstyleEndnoteText{
—~}\textstyleEndnoteTextEmphasis{О}\textstyleEndnoteText{
—~}\textstyleEndnoteTextEmphasis{В}\textstyleEndnoteText{),
которую Гегель называет также «формальным
силлогизмом».}\textstyleEndnoteText{ }}
\item 
\label{bkm:Ref484779715}\textstyleEndnoteText{К
стр.~\pageref{bkm:bm77}. — Здесь опять речь идет о «третьей
фигуре» силлогизма, о которой упоминалось выше (см. примечание
\ref{bkm:Ref484779643}).}\textstyleEndnoteText{ }}
\item 
\label{bkm:Ref484779727}\textstyleEndnoteText{К
стр.~\pageref{bkm:bm78}. — Умозаключение «третьей фигуры»
(}\textstyleEndnoteTextEmphasis{Е}\textstyleEndnoteText{
—~}\textstyleEndnoteTextEmphasis{В}\textstyleEndnoteText{
—~}\textstyleEndnoteTextEmphasis{О}\textstyleEndnoteText{).}\textstyleEndnoteText{
}}
\item 
\label{bkm:Ref484779740}\textstyleEndnoteText{К
стр.~\pageref{bkm:bm79}. — Умозаключение «второй фигуры»
(}\textstyleEndnoteTextEmphasis{В}\textstyleEndnoteText{
—~}\textstyleEndnoteTextEmphasis{Е}\textstyleEndnoteText{
—~}\textstyleEndnoteTextEmphasis{О}\textstyleEndnoteText{).}\textstyleEndnoteText{
}}
\item 
\label{bkm:Ref484779753}\textstyleEndnoteText{К
стр.~\pageref{bkm:bm80}. — Умозаключение «первой фигуры»
(}\textstyleEndnoteTextEmphasis{Е}\textstyleEndnoteText{
—~}\textstyleEndnoteTextEmphasis{О}\textstyleEndnoteText{
—~}\textstyleEndnoteTextEmphasis{В}\textstyleEndnoteText{).}\textstyleEndnoteText{
}}
\item 
\label{bkm:Ref484779784}\textstyleEndnoteText{К
стр.~\pageref{bkm:bm81}. — Имеется в виду так называемое
«сродство душ». Ср. роман Гете «Die Wahlverwandschaften»
(«Сродство душ»), написанный великим поэтом в
1809~г.}\textstyleEndnoteText{ }}
\item 
\label{bkm:Ref484779797}\textstyleEndnoteText{К
стр.~\pageref{bkm:bm82}. — Слово «основание» (Basis) берется
здесь в смысле химического основания, причем, однако, этот смысл
метафорически переносится также и на другие, не химические предметы и даже
на личности. В химии основаниями называются такие вещества, которые при
соединении с кислотами дают соли. Типическими представителями оснований в
химии начала XIX в. считались щелочи (например, едкое кали, едкий натр),
являющиеся соединениями кислорода с металлами.}\textstyleEndnoteText{ }}
\item 
\label{bkm:Ref484779821}\textstyleEndnoteText{К
стр.~\pageref{bkm:bm83}. — Эту фразу (начиная со слов:
«Механизм показывает себя...») цитирует в «Диалектике природы» Энгельс,
снабжая ее следующим замечанием: «Беда однако в том, что механизм (также
материализм 18-го века) не может выбраться из абстрактной необходимости, а
потому также и из случайности. Для него тот факт, что материя развивает из
себя мыслящий мозг человека, есть чистая случайность, хотя и необходимо
обусловленная шаг за шагом там, где это происходит. В действительности же
материя приходит к развитию мыслящих существ в силу самой своей природы, а
потому это с необходимостью и происходит во всех тех случаях, когда имеются
налицо соответствующие условия (не обязательно везде и всегда одни и те
же)» (}\textstyleEndnoteTextEmphasis{Engels}\textstyleEndnoteText{,
Dialektik der Natur, M. — L. 1935, S. 654). Этим замечанием
Энгельс показывает: 1) что под гегелевской категорией «механизма»
скрывается главным образом антителеологический детерминизм французских
материалистов XVIII в. и 2) что этот детерминизм имел абстрактный,
метафизический характер, в чем и состояла его историческая ограниченность и
недостаточность. Любопытно, что идеалист Гегель, провозглашая телеологию
более высокой точкой зрения, чем механизм, все же вынужден признать
известные «формальные» (как он выражается) преимущества за
механизмом.}\textstyleEndnoteText{ }}
\item 
\label{bkm:Ref484779833}\textstyleEndnoteText{К
стр.~\pageref{bkm:bm84}. — Эту фразу (начиная со слов: «Этот
принцип...») цитирует в «Диалектике природы» Энгельс, сопровождая ее
следующим замечанием: «При этом опять-таки колоссальная расточительность
природы с веществом и движением. В солнечной системе имеются может быть
самое большее только три планеты, на которых, при теперешних условиях,
возможно существование жизни и мыслящих существ. И ради них весь этот
громадный аппарат!»
(}\textstyleEndnoteTextEmphasis{Engels}\textstyleEndnoteText{, Dialektik
der Natur, M. — L. 1935, S. 655). Этим замечанием Энгельс
вскрывает несостоятельность того телеологического воззрения на природу,
согласно которому
}\textstyleEndnoteTextEmphasis{целью}\textstyleEndnoteText{
природы является существование органической жизни и мыслящих
существ. Ср. примечание \ref{bkm:Ref484779821}. С точки
зрения диалектического материализма категория цели имеет место только в
области сознательной деятельности людей.}\textstyleEndnoteText{ }}
\item 
\label{bkm:Ref484779856}\textstyleEndnoteText{К
стр.~\pageref{bkm:bm85}. — См.
}\textstyleEndnoteTextEmphasis{Кант}\textstyleEndnoteText{,
Критика способности суждения, пер. Соколова, Спб. 1898,
стр.~274.}\textstyleEndnoteText{ }}
\item 
\label{bkm:Ref484779906}\textstyleEndnoteText{К
стр.~\pageref{bkm:bm86}. — Гегель определяет «идею» как
адекватное понятие, как субъект-объект, как единство понятия и
действительности (или единство понятия и объективности). О том, что Гегель
разумеет под «понятием», см. примечание 1. Для правильного понимания того
содержания, которое скрывается в гегелевской категории «идеи», огромное
значение имеют указания Ленина, который дает
}\textstyleEndnoteTextEmphasis{двоякую}\textstyleEndnoteText{
расшифровку этой категории: 1) идея~=~познание человека
(}\textstyleEndnoteTextEmphasis{Ленин}\textstyleEndnoteText{,
Философские тетради, М. 1936, стр.~187) и 2) идея~=~сама
природа (там же, стр.~198) или идея~=~объективная действительность
(стр.~192). Этот двуединый характер гегелевской идеи, с одной стороны, таит
в себе глубочайшую фальшь гегелевского идеализма  —~учение о
тождестве мышления и бытия, сводящее бытие к мышлению, а с другой стороны,
здесь же скрывается гениальная догадка Гегеля относительно
«}\textstyleEndnoteTextEmphasis{общих законов движения
мира}\textstyleEndnoteText{ и
}\textstyleEndnoteTextEmphasis{мышления}\textstyleEndnoteText{»
(}\textstyleEndnoteTextEmphasis{Ленин}\textstyleEndnoteText{,
Философские тетради, стр.~170). Эту вторую сторону
гегелевской «идеи» имеет в виду Ленин, когда он по поводу гегелевского
Введения к отделу «Идея» замечает, что «Гегель гениально
}\textstyleEndnoteTextEmphasis{угадал}\textstyleEndnoteText{
диалектику вещей (явлений, мира,
}\textstyleEndnoteTextEmphasis{природы}\textstyleEndnoteText{)
в диалектике понятий» (там же, Стр.~189). Этой же стороны
дела касается и Энгельс в своих «Примечаниях» к «Анти-Дюрингу». Говоря о
заслугах новейшей «идеалистической, но вместе с тем и диалектической
философии, в особенности Гегеля», Энгельс замечает: «Несмотря на
бесчисленные произвольные и фантастические построения этой философии,
несмотря на идеалистическую, на голову поставленную форму ее конечного
результата —~единства мышления и бытия, —
нельзя отрицать того, что она доказала на множестве примеров,
взятых из самых разнообразных отраслей знания, аналогию между процессами
мышления и процессами в области природы и истории —~и
обратно, и господство одинаковых законов для всех этих процессов»
(}\textstyleEndnoteTextEmphasis{Энгельс}\textstyleEndnoteText{,
Анти-Дюринг, Партиздат, 1936, стр.~244). Вот почему Ленин
указывает, что именно во Введении к отделу «Идея» —~этому
завершающему отделу всей гегелевской Логики —~у Гегеля
«...замечательно гениально показано совпадение, так сказать, логики и
гносеологии»
(}\textstyleEndnoteTextEmphasis{Ленин}\textstyleEndnoteText{,
Философские тетради, стр.~185), т.~е. логики, как науки о
законах развития всех материальных, природных и духовных вещей, и
гносеологии, как науки о законах развития человеческого
познания.}\textstyleEndnoteText{ }}
\item 
\label{bkm:Ref484779980}\textstyleEndnoteText{К
стр.~\pageref{bkm:bm87}. — «Weil ihr kein kongruierender Gegenstand in der
Sinnenwelt gegeben werden konne» —~ почти дословное
воспроизведение кантовской формулировки из «Критики чистого разума», 2-е
нем. изд., стр.~383.}\textstyleEndnoteText{ }}
\item 
\label{bkm:Ref484779997}\textstyleEndnoteText{К
стр.~\pageref{bkm:bm88}. — Цитата взята (с маленькими
несущественными изменениями) из «Критики чистого разума», 2-е нем. изд.,
стр.~373. Курсив принадлежит Гегелю.}\textstyleEndnoteText{ }}
\item 
\label{bkm:Ref484780017}\textstyleEndnoteText{К
стр.~\pageref{bkm:bm89}. — «Kongruieren»: опять намек на
Канта (см. примечание}\textstyleEndnoteText{ \ref{bkm:Ref484779980}).
Самый термин «Kongruieren» заимствован из геометрии, где он
обозначает совпадение при наложении друг на друга двух равных и подобных
фигур («конгруэнтные фигуры»). }}
\item 
\label{bkm:Ref484780056}\textstyleEndnoteText{К
стр.~\pageref{bkm:bm90}. — Выражение «субъект-объект»
встречается как у Фихте, так и у Шеллинга. У Фихте мы находим его,
например, в его «System der Sittenlehre» 1798~г. (см. Fichte, Werke, hrsg.
v. Medicus, Bd. II, S. 436, 454,}\textstyleEndnoteText{ 524, 531).
Шеллинг употребляет это выражение, например, в «Системе
трансцендентального идеализма» 1800~г.
(}\textstyleEndnoteTextEmphasis{Schelling}\textstyleEndnoteText{, Werke,
hrsg. v. O. Weiss, Bd. II, S. 47, 63) и в диалоге «Бруно»,
вышедшем в 1802~г. (См.
}\textstyleEndnoteTextEmphasis{Шеллинг}\textstyleEndnoteText{,
Философские исследования о сущности человеческой свободы.
Бруно или о божественном и естественном начале вещей, Спб. 1908,
стр.~163).}}
\item 
\label{bkm:Ref484780102}\textstyleEndnoteText{К
стр.~\pageref{bkm:bm91}.  —~Это, повидимому, намек на
Шеллинга, причем слово «гений» имеет в виду «Систему трансцендентального
идеализма» Шеллинга (1800~г.), а слово «число» —~его
натурфилософию (ср. примечание 41 к т.~I «Науки логики»). В «Системе
трансцендентального идеализма» Шеллинг утверждает, что абсолютное дано
человеку в интеллектуальном и эстетическом созерцании и что для этого
созерцания требуется особого рода «гениальность» (das Genie),
понятие которой вводится у Шеллинга следующим образом:
«Подобно тому, как именуется роком та сила, которая через нашу свободную
деятельность без нашего ведома и даже наперекор нашему желанию осуществляет
цели, }\textstyleEndnoteTextEmphasis{о которых люди не
думали}\textstyleEndnoteText{, — так мы обозначаем темным
понятием
}\textstyleEndnoteTextEmphasis{гениальности}\textstyleEndnoteText{
то непостижимое начало, которое без всякого содействия
свободы и до известной степени даже вопреки последней придает осознанному
объективность, тогда как в области свободы то, что соединено в продукте,
вечно убегает от самого себя»
(}\textstyleEndnoteTextEmphasis{Schelling}\textstyleEndnoteText{, Werke,
hrsg. v. Weiss. Bd. II,}\textstyleEndnoteText{ 3. 290). В
другом месте Шеллинг определяет гениальность как «совпадение
бессознательной и сознательной деятельности» (там же, стр.~298). Можно
думать, что на Шеллинга (а именно, на его романтическое превознесение
искусства как «всеобщего органа философии») намекают и слова «как простой
образ» (als blosses Bild) двумя строками выше, тем более что
дальнейшее определение этого «образа» словами «без стремления и движения»
напоминает следующие слова Шеллинга из той же «Системы трансцендентального
идеализма»: «Всякое стремление к продуцированию прекращается, как только
продукт завершен; все противоречия устранены, все загадки разрешены» (там
же, стр.~289). — С другой стороны, гегелевские слова «как
простой
}\textstyleEndnoteTextEmphasis{образ}\textstyleEndnoteText{,
тусклый и бессильный» напоминают известное место из «Этики»
Спинозы, где Спиноза полемизирует с теми, кто «смотрит на идеи как на немые
изображения на картине и под влиянием этого предрассудка не видит, что
идея, поскольку она есть идея, заключает в себе утверждение или отрицание»
(схолия к теореме 49 второй части). }}
\item 
\label{bkm:Ref484780166}\textstyleEndnoteText{К
стр.~\pageref{bkm:bm92}. — \ «Das Urteil» в смысле
«перводеления» (см. примечание
\ref{bkm:Ref484772231}).}\textstyleEndnoteText{ }}
\item 
\label{bkm:Ref484780194}\textstyleEndnoteText{К
стр.~\pageref{bkm:bm93}. — По поводу этого перехода от жизни
к познанию Энгельс делает следующее замечание в «Диалектике природы»:
«Когда Гегель переходит от жизни к познанию через посредство оплодотворения
(размножения), то в этом находится уже в зародыше теория развития, учение о
том, что раз дана органическая жизнь, то она должна развиться путем
развития поколений до породы мыслящих существ»
(}\textstyleEndnoteTextEmphasis{Энгельс}\textstyleEndnoteText{,
Диалектика природы, Партиздат,}\textstyleEndnoteText{ 1936,
стр.~46). }}
\item 
\label{bkm:Ref484780232}\textstyleEndnoteText{К
стр.~\pageref{bkm:bm94}. — К этому месту относится следующее
замечание Энгельса из «Диалектики природы»: «Внутренняя цель в организме
прокладывает себе затем согласно Гегелю (V, 244) дорогу через посредство
}\textstyleEndnoteTextEmphasis{влечения}\textstyleEndnoteText{.
Не слишком убедительно это. Влечение должно, по Гегелю,
привести отдельное живое существо более или менее в гармонию с его
понятием. Отсюда ясно, насколько вся эта
}\textstyleEndnoteTextEmphasis{внутренняя
цель}\textstyleEndnoteText{ сама представляет собой
идеологическое определение. И однако в этом суть Ламарка»
(}\textstyleEndnoteTextEmphasis{Engels}\textstyleEndnoteText{, Dialektik
der Natur, M. — L. 1935, S. 655). Даваемая Энгельсом ссылка
«V, 244» означает т.~V немецкого собрания сочинений Гегеля, 2-е изд.
(Берлин 1841), стр.~244. Если Энгельс пользовался 2-м изданием сочинений
Гегеля, то Ленин писал свои «Философские тетради» на основе 1-го издания. В
первом издании V тома (1834~г.) указанное Энгельсом место находится на
стр.~251–252.}\textstyleEndnoteText{ }}
\item 
\label{bkm:Ref484780250}\textstyleEndnoteText{К
стр.~\pageref{bkm:bm95}. — Под «образом» (Gestalt) Гегель
здесь понимает «животный субъект как целое, взятое
}\textstyleEndnoteTextEmphasis{только в его соотношении с
самим собой}\textstyleEndnoteText{» (см. «Философию
природы», § 353). Помещая в скобках латинское слово «insectum», которое
означает «надрезанное», а затем «насекомое», Гегель намекает на то, что
характерной }\textstyleEndnoteText{особенностью насекомых
является разделение или рассечение их тела на голову, грудь и брюшко
(отсюда и самое название этого класса живых существ). В «Философии природы»
Гегель, трактуя о трех основных функциях организма
—~чувствительности, раздражимости и воспроизведении
(воспроизведение берется здесь у Гегеля в смысле постоянного
воспроизведения живым организмом всех тканей и соков, входящих в его
состав), — \ отмечает, что эти функции в рассеченном виде
представлены у насекомых, у которых, дескать, «голова является центром
чувствительности, грудь —~раздражимости, брюшко
—~воспроизведения»
(}\textstyleEndnoteTextEmphasis{Гегель}\textstyleEndnoteText{,
Соч., т.~II, стр.~464).}\textstyleEndnoteText{ }}
\item 
\label{bkm:Ref484780296}\textstyleEndnoteText{К
стр.~\pageref{bkm:bm96}. — Гегель здесь передает (большею
частью словами самого Канта) содержание кантовской критики
«психологического паралогизма», занимающей страницы 400–411 второго
немецкого издания «Критики чистого разума», причем больше всего выписок
Гегель делает со страницы 404. Курсив везде принадлежит Гегелю. В
«Диалектике природы» Энгельс делает следующее замечание по поводу этого
места «Логики» Гегеля: «Ценная самокритика кантовской
}\textstyleEndnoteTextEmphasis{вещи-в-себе}\textstyleEndnoteText{,
показывающая, что Кант терпит крушение также и по вопросу о
мыслящем Я, в котором он тоже находит некоторую непознаваемую вещь-в-себе.
Гегель, V, 256 и сл.»
(}\textstyleEndnoteTextEmphasis{Engels}\textstyleEndnoteText{, Dialektik
der Natur, M. — L. 1935, S. 655). Даваемая Энгельсом ссылка
(«V, 256 и сл.») указывает страницы второго немецкого издания V
тома Собрания сочинений Гегеля. В первом издании этого тома
указанное место находится на стр.~264–268.}\textstyleEndnoteText{ }}
\item 
\label{bkm:Ref484780310}\textstyleEndnoteText{К
стр.~\pageref{bkm:bm97}. — Выражения «неудобство»
(Unbequemlichkeit) и «круг» (Zirkel) в применении к самосознанию
употребляются Кантом на стр.~404 второго издания «Критики чистого
разума».}\textstyleEndnoteText{ }}
\item 
\label{bkm:Ref484780324}\textstyleEndnoteText{К
стр.~\pageref{bkm:bm98}. — Выражения «Я как субъект
сознания», «Я может употреблять себя только в качестве субъекта суждения»,
«созерцание, через которое Я было бы дано как объект», —
взяты со страницы 411 второго издания «Критики чистого
разума».}\textstyleEndnoteText{ }}
\item 
\label{bkm:Ref484780672}\textstyleEndnoteText{К
стр.~\pageref{bkm:bm99}. — «Опровержение мендельсоновского
доказательства пребывающего характера души» (т.~е. ее неуничтожимости или
бессмертия) занимает страницы 413–415 второго издания «Критики чистого
разума».}\textstyleEndnoteText{ }}
\item 
\label{bkm:Ref484780756}\textstyleEndnoteText{К
стр.~\pageref{bkm:bm100}. — Имеется в виду «критическая»
философия Канта.}\textstyleEndnoteText{ }}
\item 
\label{bkm:Ref484780835}\textstyleEndnoteText{К
стр.~\pageref{bkm:bm101}. — В сложении соединяются вместе
два (если брать простейший случай) каких-нибудь числа, которые не
обязательно равны друг другу: поэтому, как более общий случай, Гегель берет
сложение двух
}\textstyleEndnoteTextEmphasis{неравных}\textstyleEndnoteText{
чисел (например, 5+7). В умножении одно и то же число
прибавляется к самому этому числу: поэтому Гегель рассматривает умножение
как сложение
}\textstyleEndnoteTextEmphasis{равных}\textstyleEndnoteText{
чисел (например, 2·~5~=~5~+~5). При возведении в квадрат
(который, по Гегелю, представляет собою
}\textstyleEndnoteTextEmphasis{основную}\textstyleEndnoteText{
математическую степень) имеет место равенство между тем
числом, которое возводится в квадрат (по Гегелю это
—~«}\textstyleEndnoteTextEmphasis{единица}\textstyleEndnoteText{»),
и тем числом, которое служит
«}\textstyleEndnoteTextEmphasis{численностью}\textstyleEndnoteText{»,
т.~е. множителем (например, 5\textsuperscript{2} =
5·~5~=~5+5+5+5+5). Подробнее об арифметических действиях
Гегель говорит в I части «Науки логики» (см. т.~I «Науки логики»,
стр.~158–164).}\textstyleEndnoteText{ }}
\item 
\label{bkm:Ref484780819}\textstyleEndnoteText{К
стр.~\pageref{bkm:bm102}. — См.
}\textstyleEndnoteTextEmphasis{Кант}\textstyleEndnoteText{,
Критика чистого разума, 2-е нем. изд.,
стр.~15–16.}\textstyleEndnoteText{ }}
\item 
\label{bkm:Ref484780858}\textstyleEndnoteText{К
стр.~\pageref{bkm:bm103}. — Под «десятичными числами»
(Dezimalzahlen) Гегель имеет здесь в виду любые
}\textstyleEndnoteTextEmphasis{многозначные}\textstyleEndnoteText{
числа по десятичной системе счисления, в том числе и
многозначные десятичные дроби.}}
\item 
\label{bkm:Ref484781128}\textstyleEndnoteText{К
стр.~\pageref{bkm:bm104}. — Гегель, повидимому, имеет в виду
«Арифметические исследования» (Disquisitiones arithmeticae) Гаусса,
вышедшие в 1801~г.}\textstyleEndnoteText{ }}
\item 
\label{bkm:Ref484781151}\textstyleEndnoteText{К
стр.~\pageref{bkm:bm105}. — В немецком тексте всех изданий
напечатано «in welchem». Повидимому, это —~опечатка, вместо
«in welcher».}}
\item 
\label{bkm:Ref484781181}\textstyleEndnoteText{К
стр.~\pageref{bkm:bm106}. — Ср. рассуждения Маркса о методе
политической экономии во «Введении к Критике политической экономии». См.
примечание \ref{bkm:Ref484771295}.}}
\item 
\label{bkm:Ref484781259}\textstyleEndnoteText{К
стр.~\pageref{bkm:bm107}. — По Гегелю число (и вообще
количество, величина) есть нечто внешнее, безразличное для предмета (в
известных границах, конечно). Из области чисел заимствуются определения для
классификации, например, в ботанических классификациях Линнея, где
принципом для подразделения на виды нередко служат такие признаки, как
число тычинок и~т.~д.}\textstyleEndnoteText{ }}
\item 
\label{bkm:Ref484781310}\textstyleEndnoteText{К
стр.~\pageref{bkm:bm108}. — Ср. следующие замечания Энгельса
в «Анти-Дюринге»: «Математические аксиомы представляют собой выражения
крайне скудного умственного содержания, которое математика должна
заимствовать у логики. Их можно свести к двум следующим аксиомам:}}

\begin{enumerate}
\item 
\textstyleEndnoteText{Целое больше части. Это положение есть
чистая тавтология...}}
\item 
\textstyleEndnoteText{Если две величины равны третьей, то они
равны между собой. Это положение, как показал еще Гегель, представляет
собой умозаключение, за правильность которого ручается логика; оно значит
доказывается, хотя и вне области чистой математики. Прочие аксиомы о
равенстве и неравенстве являются просто логическим развитием этого
умозаключения»
[}\textstyleEndnoteTextEmphasis{Энгельс}\textstyleEndnoteText{,
Анти-Дюринг, Партиздат, 1936, стр.~27). То умозаключение, о
котором здесь говорит Энгельс, составляет у Гегеля «четвертую фигуру»
«умозаключения наличного бытия» и называется у Гегеля «математическим» или
«чисто-количественным» умозаключением. Его схема:
«}\textstyleEndnoteTextEmphasis{B}\textstyleEndnoteText{
—~}\textstyleEndnoteTextEmphasis{B}\textstyleEndnoteText{
—~}\textstyleEndnoteTextEmphasis{B}\textstyleEndnoteText{». }}
\end{enumerate}
\item 
\label{bkm:Ref484781428}\textstyleEndnoteText{К
стр.~\pageref{bkm:bm109}. — Греческое слово «лемма» означает
«заимствованное положение», т.~е. такое положение, обоснование которого
дано в другой науке.}\textstyleEndnoteText{ }}
\item 
\label{bkm:Ref484781445}\textstyleEndnoteText{К
стр.~\pageref{bkm:bm110}. — См. выше,
стр.~\pageref{bkm:bm110a} —~\pageref{bkm:bm110b}. Ср.
примечание \ref{bkm:Ref484781259}.}\textstyleEndnoteText{ }}
\item 
\label{bkm:Ref484781469}\textstyleEndnoteText{К
стр.~\pageref{bkm:bm111}. — В немецком тексте всех изданий
стоит «Seiten» (стороны). Повидимому, это опечатка вместо
«Satze» или «Lehrsatze».}\textstyleEndnoteText{ }}
\item 
\label{bkm:Ref484781485}\textstyleEndnoteText{К
стр.~\pageref{bkm:bm112}. — По Гегелю «единообразное начало»
(das Gleichförmige) представлено в данном случае
прямоугольным треугольником, о котором Гегель выше говорит, что он есть
«наиболее простой в своих различиях и потому наиболее правильный
треугольник».}\textstyleEndnoteText{ }}
\item 
\label{bkm:Ref484781520}\textstyleEndnoteText{К
стр.~\pageref{bkm:bm113} —~ В немецком тексте всех изданий
вместо «und» напечатано «mit». Это —~явная
опечатка или описка.}}
\item 
\label{bkm:Ref484781593}\textstyleEndnoteText{К
стр.~\pageref{bkm:bm114}.~— В немецком тексте всех изданий
напечатано: «aus welchem die Vermittlung... zurückgefuhrt wird».
Грамматически такая конструкция невозможна. Приходится
либо вместо «aus welchem»
читать «auf welchen», либо
вместо «zurückgeführt» —~«abgeleitet». Перевод
сделан в соответствии со второй конъектурой.}\textstyleEndnoteText{ }}
\item 
\label{bkm:Ref484781676}\textstyleEndnoteText{К
стр.~\pageref{bkm:bm115}. — Имеются в виду Кант и его
сторонники.}\textstyleEndnoteText{ }}
\item 
\label{bkm:Ref484781764}\textstyleEndnoteText{К
стр.~\pageref{bkm:bm116}. — \ Циркумвалляционной линией в
старой фортификации называлась блокадная линия, имевшая своим назначением
не допустить противника прийти на выручку осаждаемой неприятельской
крепости (от латинского circumvallare —~окружать, обносить
валом).}\textstyleEndnoteText{ }}
\item 
\label{bkm:Ref484781786}\textstyleEndnoteText{К
стр.~\pageref{bkm:bm117}. — Полное заглавие: «Метафизические
основоначала естествознания» («Metaphysische Anfangsgründe der
Naturwissenschaft»). Вышли в 1786~г.}\textstyleEndnoteText{
}}
\item 
\label{bkm:Ref484781811}\textstyleEndnoteText{К
стр.~\pageref{bkm:bm118}. — Гегель указывает страницы по
1-му немецкому изданию «Феноменологии духа» (1807). В 1-м издании II тома
Собрания сочинений Гегеля (1832~г.) указанное Гегелем место находится на
страницах 45}\textstyleEndnoteText{1–475, в издании Лассона
(1921~г.) —~на страницах 388–408. В русском переводе
«Феноменологии духа» (Спб. }\textstyleEndnoteText{1913)
этому соответствуют страницы 272–287. В этом отделе
«Феноменологии» Гегель рассматривает «моральное мировоззрение», имея в виду
моральную философию Канта и Фихте. При этом Гегель подвергает ее
обстоятельной имманентной критике, язвительно вскрывая таящиеся в ней
противоречия. }}
\item 
\label{bkm:Ref484781854}\textstyleEndnoteText{К
стр.~\pageref{bkm:bm119}. — Говоря здесь о «сознании в
собственном смысле», Гегель имеет в виду то предметное (направленное на
внешние предметы) сознание, которое он рассматривает в первом отделе своей
«Феноменологии духа». Подобным же образом то «самосознание», о котором
упоминает предыдущая фраза текста, соответствует второму отделу
«Феноменологии», трактующему о самосознании, предметом которого является,
во-первых, оно само, а во-вторых, объекты чувственной достоверности и
чувственного восприятия.}\textstyleEndnoteText{ }}
\item 
\label{bkm:Ref484781880}\textstyleEndnoteText{К
стр.~\pageref{bkm:bm120}. — \ Об «абсолютной идее» как
завершении «Логики» Гегеля мы имеем высказывания Энгельса и Ленина, на
первый взгляд кажущиеся прямо противоположными, а в действительности
прекрасно дополняющие друг друга. Энгельс указывает, что конечная точка
«Логики», абсолютная идея, «абсолютна лишь постольку, поскольку Гегель
абсолютно ничего не может}\textstyleEndnoteText{ сказать о
ней» (}\textstyleEndnoteTextEmphasis{Engels}\textstyleEndnoteText{, L.
Feuerbach, Moskau, 1932, S. 19). Ленин пишет: «Замечательно,
что вся глава об «абсолютной идее» почти ни словечка не говорит о боге
(едва ли не один раз случайно вылезло «божеское» «понятие») и кроме того
—~}\textstyleEndnoteTextEmphasis{это}\textstyleEndnoteText{
NB —~почти не содержит специфически
}\textstyleEndnoteTextEmphasis{идеализма}\textstyleEndnoteText{,
а главным своим предметом имеет
}\textstyleEndnoteTextEmphasis{диалектический
метод}\textstyleEndnoteText{»
(}\textstyleEndnoteTextEmphasis{Ленин}\textstyleEndnoteText{,
Философские тетради, М. 1936, стр.~225). Эти две оценки, как
сказано, прекрасно дополняют друг друга. Дело в том, что об абсолютной идее
}\textstyleEndnoteTextEmphasis{как
таковой}\textstyleEndnoteText{, т.~е. как о каком-то
}\textstyleEndnoteTextEmphasis{особом}\textstyleEndnoteText{
абсолютном содержании, Гегель, действительно, не в состоянии
ничего сказать, поскольку содержанием этой абсолютной идеи оказывается не
что иное, как }\textstyleEndnoteTextEmphasis{весь процесс
развертывания логических категорий}\textstyleEndnoteText{,
который служил предметом рассмотрения всех предыдущих глав
«Логики». В главе об абсолютной идее Гегель бросает, как он сам выражается,
«}\textstyleEndnoteTextEmphasis{ретроспективный
взгляд}\textstyleEndnoteText{» (см.
}\textstyleEndnoteTextEmphasis{Гегель}\textstyleEndnoteText{,
Соч., т.~I, стр.~341) на весь этот уже рассмотренный им
процесс и формулирует общие черты, характеризующие
}\textstyleEndnoteTextEmphasis{форму}\textstyleEndnoteText{
этого процесса, — иначе говоря, формулирует
основные черты
}\textstyleEndnoteTextEmphasis{диалектического
метода}\textstyleEndnoteText{, представляющего собою, по
выражению Гегеля, «имманентную душу самого содержания» (см. т.~I «Науки
логики», стр.~10). Поэтому и получается, что глава об абсолютной идее
имеет, как указывает Ленин, «своим главным предметом
}\textstyleEndnoteTextEmphasis{диалектический
метод}\textstyleEndnoteText{». В предисловии к
«Феноменологии духа» Гегель выставил положение о том, что «истина
—~это целое» (см. примечание \ref{bkm:Ref485481734}),
причем это целое понимается у Гегеля как
}\textstyleEndnoteTextEmphasis{процесс}\textstyleEndnoteText{.
Последняя глава «Науки логики» и подводит итоги этому
процессу, бросая общий взгляд назад на весь пройденный путь и формулируя
основные принципы диалектического движения понятий, являющегося, по Гегелю,
вместе с тем диалектическим движением самой действительности.}}
\item 
\label{bkm:Ref485416891}\textstyleEndnoteText{К
стр.~\pageref{bkm:bm121}. — Это место в сокращенном виде
цитируется у Маркса в «Нищете философии», после чего Маркс замечает: «Итак,
что же такое абсолютный метод? Абстракция движения. Что такое абстракция
движения? Движение в абстрактном виде. Что такое движение в абстрактном
виде? Чисто логическая формула движения, или движение чистого разума. В чем
состоит движение чистого разума? В том, что он полагает себя,
противополагает себя самому себе и слагается с самим собою, в том, что он
формулируется в тезис, антитезис и синтезис, или, наконец, в том, что он
себя утверждает, отрицает и отрицает свое отрицание». И~т.~д. (См.
}\textstyleEndnoteTextEmphasis{Маркс}\textstyleEndnoteText{,
Нищета философии, Партиздат, 1937, стр.~76–77.) В своей
полемике против Прудона Маркс касается главным образом внешней стороны
гегелевского «абсолютного метода», поскольку Прудон пытался использовать
(притом весьма неудачно) именно эту внешнюю сторону идеалистической
диалектики Гегеля.}\textstyleEndnoteText{ }}
\item 
\label{bkm:Ref484781953}\textstyleEndnoteText{К
стр.~\pageref{bkm:bm122}. — В немецком тексте всех изданий
стоит: «weder… gesetzt sind». Так как этому «weder»
нигде не соответствует второе отрицание «noch»,
то вместо «weder» надо читать
«nicht».}\textstyleEndnoteText{ }}
\item 
\label{bkm:Ref484781976}\textstyleEndnoteText{К
стр.~\pageref{bkm:bm123}. — Имеется в виду кинический
философ Диоген из Синопа (414–323~гг. до н.~э). Греческое слово «киник»
(или «циник») происходит от «кион» —~собака. Поэтому Гегель
и называет здесь Диогена «Diogenes der Hund», желая этим
вместе с тем подчеркнуть свое отрицательное отношение к этому
философу.}\textstyleEndnoteText{ }}
\item 
\label{bkm:Ref484781997}\textstyleEndnoteText{К
стр.~\pageref{bkm:bm124}. — Немецкий
текст первого издания
(1816~г.) испорчен: «Das Positive in
}\textstyleEndnoteTextEmphasis{seinem}\textstyleEndnoteText{ Negativen, dem
Inhalt der Voraussetzung im Resultate festzuhalten...»
Издание 1834~г., а за ним и все последующие прибавляют
запятую после слова «Voraussetzung», но смысл от этого не
выигрывает, а проигрывает. Гораздо более соответствует смыслу всего
контекста такая конъектура: вместо «dem Inhalt» читать «den
Inhalt». Перевод сделан соответственно этой последней
конъектуре.}\textstyleEndnoteText{ }}
\item 
\label{bkm:Ref484782039}\textstyleEndnoteText{К
стр.~\pageref{bkm:bm125}. — Тот силлогизм, о котором здесь
говорит Гегель, различая его первую и вторую посылку, можно для большей
наглядности представить с помощью схемы
«}\textstyleEndnoteTextEmphasis{П}\textstyleEndnoteText{
—~}\textstyleEndnoteTextEmphasis{О}\textstyleEndnoteText{
—~}\textstyleEndnoteTextEmphasis{О}\textstyleEndnoteText{»,
где
«}\textstyleEndnoteTextEmphasis{П}\textstyleEndnoteText{»
означает «положительное», а
«}\textstyleEndnoteTextEmphasis{О}\textstyleEndnoteText{»
—~«отрицательное». Первую посылку образует соотношение между
«}\textstyleEndnoteTextEmphasis{П}\textstyleEndnoteText{»
и
«}\textstyleEndnoteTextEmphasis{О}\textstyleEndnoteText{»,
вторую —~соотношение между
«}\textstyleEndnoteTextEmphasis{О}\textstyleEndnoteText{»
и
«}\textstyleEndnoteTextEmphasis{О}\textstyleEndnoteText{».
Эти два
«}\textstyleEndnoteTextEmphasis{О}\textstyleEndnoteText{»
отличаются друг \ от друга. Первое
«}\textstyleEndnoteTextEmphasis{О}\textstyleEndnoteText{»
есть опосредствующий средний термин, и в нем Гегель в свою
очередь различает два момента, как это более подробно выясняется из
дальнейших рассуждений Гегеля: 1) простое, первое, формальное или
абстрактное отрицание и 2) абсолютное или второе отрицание. Этот второй
момент, имеющий внутри себя противоречие, Гегель характеризует как
«диалектическую душу» всего истинного, как «наивнутреннейший источник»
всякой деятельности и всякого самодвижения. Что же касается того
«}\textstyleEndnoteTextEmphasis{О}\textstyleEndnoteText{»,
которое образует другой крайний термин рассматриваемого
силлогизма, то оно, как отрицание отрицания, есть восстановление первого
непосредственного
(«}\textstyleEndnoteTextEmphasis{П}\textstyleEndnoteText{»),
однако такое восстановление, которое делает его единством
непосредственного и опосредствованного.}\textstyleEndnoteText{ }}
\item 
\label{bkm:Ref484782093}\textstyleEndnoteText{К
стр.~\pageref{bkm:bm126}. — \ Выше, на
стр.~\pageref{bkm:bm126a} Гегель говорил о другой
«бесконечной заслуге» Канта —~идее критического исследования
определений мысли. Здесь же Гегель имеет в виду главным образом
триадичность кантовской таблицы категорий, в которой «третья категория
возникает всегда из соединения второй и первой категории одного и того же
класса» (см.
}\textstyleEndnoteTextEmphasis{Кант}\textstyleEndnoteText{,
Критика чистого разума, пер. Лосского, Пгр. 1915, стр.~11).
Еще бóльшую роль играет триада в философии Фихте. Триадическая схема
«тезис —~антитезис —~ синтез» фигурирует уже
в написанной в 1794~г. книге Фихте «Основа общего наукоучения» (см.
}\textstyleEndnoteTextEmphasis{Фихте}\textstyleEndnoteText{,
Избр. соч., т.1, М. 1916, стр.~91 и
др.).}\textstyleEndnoteText{ }}
\item 
\label{bkm:Ref484782271}\textstyleEndnoteText{К
стр.~\pageref{bkm:bm127}. — Гегель имеет ввиду Шеллинга и, в
особенности, его последователей и поклонников. О Шеллинге Гегель в «Истории
философии» замечает, что его философия страдает «формализмом внешнего
конструирования по некоторой наперед принятой схеме»
(}\textstyleEndnoteTextEmphasis{Гегель}\textstyleEndnoteText{,
Соч., т.~XI, стр.~504), а именно, по схеме троичности (там
же, стр.~506 и 510). О поклонниках Шеллинга Гегель отзывается еще более
резко. «Вся эта манера, — говорит он об их манере
философствовать, — представляет собой такой жалкий
формализм, такое бессмысленное смешение обыденнейшей эмпирии и
поверхностнейших идеальных определений, какой только когда-либо
существовал... Философия благодаря этому... сделалась предметом
пренебрежения и презрения» (там же, стр.~511–512).}\textstyleEndnoteText{
}}
\item 
\label{bkm:Ref484782311}\textstyleEndnoteText{К
стр.~\pageref{bkm:bm128}. — В смысле все более и более
глубокого понимания этого начала, его истинной природы и его истинного
значения.}\textstyleEndnoteText{ }}
\item 
\label{bkm:Ref484782325}\textstyleEndnoteText{К
стр.~\pageref{bkm:bm129}. — Немецкий текст гласит: «es
braucht nicht depreziert zu werden, dass man ihn nur
}\textstyleEndnoteTextEmphasis{provisorisch}\textstyleEndnoteText{ und
}\textstyleEndnoteTextEmphasis{hypothetisch}\textstyleEndnoteText{ gelten
lassen möge».Слово «deprezieren» имеет в
немецком языке два значения: 1) извиняться и 2) обесценивать. Если в этой
фразе вместо «es» читать «еr» (т.~е. der
Anfang), то слово «deprezieren» можно
истолковать во втором его значении. Тогда перевод будет гласить: «нет нужды
обесценивать это начало, утверждая, что его можно принимать лишь
}\textstyleEndnoteTextEmphasis{провизорно}\textstyleEndnoteText{
и
}\textstyleEndnoteTextEmphasis{гипотетически}\textstyleEndnoteText{».
Ленин, повидимому, склонялся к этому последнему истолкованию
(см. «Ленинский сборник», IX, М. — Л. 1929, стр.~296).}}
\end{enumerate}

\textstyleEndnoteText{Гегель имеет здесь в виду кантианца
Рейнгольда, о котором он упоминает в т.~I «Науки логики», как о защитнике
того взгляда, что «философия должна начинать лишь с некоторого
}\textstyleEndnoteTextEmphasis{гипотетически}\textstyleEndnoteText{
и
}\textstyleEndnoteTextEmphasis{проблематически}\textstyleEndnoteText{
истинного и что философствование поэтому может быть сначала
лишь исканием» (см. \ т.~I «Науки логики», стр.~45; ср.
}\textstyleEndnoteTextEmphasis{Гегель}\textstyleEndnoteText{,
Соч., т.~I, стр.~28). О Рейнгольде и его «Материалах для
краткого обзора состояния философии в начале XIX столетия» (первая тетрадь
вышла в 1801 г.) Гегель подробно говорит в своей первой печатной работе
«Различие между системами философии Фихте и Шеллинга» (Иена
1801).}\textstyleEndnoteText{ }}

\begin{enumerate}
\item 
\label{bkm:Ref484782343}\textstyleEndnoteText{К
стр.~\pageref{bkm:bm130}. — Намек на «критическую» философию
Канта.}\textstyleEndnoteText{ }}
\item 
\label{bkm:Ref484782376}\textstyleEndnoteText{К
стр.~\pageref{bkm:bm131}. — Гегель хочет сказать, что только
в самом конце изложения логика достигает полного познания самой себя,
своего предмета и своего «понятия». Начинается же это самопознание логики
(самопознание познания) с самых первых шагов логической науки, с категорий
бытия и ничто. Во введении к «Науке логики» Гегель указывает, что его
«объективная логика есть подлинная критика» «чистых форм мысли»,
рассматриваемых «}\textstyleEndnoteTextEmphasis{в их
особенном содержании}\textstyleEndnoteText{» (т.~I «Науки
логики», стр.~39–40). В этом заключается одно из существенных отличий
гегелевской логики от «критицизма» Канта, который требует критического
исследования познавательных форм
}\textstyleEndnoteTextEmphasis{до}\textstyleEndnoteText{
самого познания, рассматривая эти формы как пустые априорные
формы, извне накладываемые нами на внешнее им содержание. По Гегелю,
логические формы составляют
«}\textstyleEndnoteTextEmphasis{живой дух
действительного}\textstyleEndnoteText{»
(}\textstyleEndnoteTextEmphasis{Гегель}\textstyleEndnoteText{,
Соч., т.~I, стр.~267). Ленин, выписывая эту формулировку
Гегеля, отмечает на полях: «общие законы
}\textstyleEndnoteTextEmphasis{движения мира и
мышления}\textstyleEndnoteText{»
(}\textstyleEndnoteTextEmphasis{Ленин}\textstyleEndnoteText{,
Философские тетради, М. 1936, стр.~170). То же самое отмечает
и Энгельс, усматривая великую заслугу диалектики Гегеля в том, что она
показала наличие }\textstyleEndnoteTextEmphasis{одних и тех
же}\textstyleEndnoteText{ законов в трех различных сферах: в
природе, истории и мышлении (см примечания к «Анти-Дюрингу»). Эти общие
законы движения мира и мышления и составляют
}\textstyleEndnoteTextEmphasis{подлинное}\textstyleEndnoteText{
содержание Логики Гегеля. Отсюда и получается (как говорит
Ленин), что «в этом }\textstyleEndnoteTextEmphasis{самом
идеалистическом}\textstyleEndnoteText{ произведении Гегеля
}\textstyleEndnoteTextEmphasis{всего
меньше}\textstyleEndnoteText{ идеализма,
}\textstyleEndnoteTextEmphasis{всего
больше}\textstyleEndnoteText{ материализма. «Противоречиво»,
но факт!»
(}\textstyleEndnoteTextEmphasis{Ленин}\textstyleEndnoteText{,
Философские тетради, стр.~225). Конечно, для того чтобы
усмотреть и понять этот факт, необходимо предварительно очистить
гегелевскую логику от обволакивающей ее «мистики идей», поповщины, остатков
формализма, пустой игры в диалектику и~т.~д.}\textstyleEndnoteText{ }}
\item 
\label{bkm:Ref484782398}\textstyleEndnoteText{К
стр.~\pageref{bkm:bm132}. — Этот переход от абсолютной идеи
к природе имеет у Гегеля две стороны: 1) мистическую или теологическую и 2)
рациональную. Мистической стороны этого перехода Энгельс касается в
«Людвиге Фейербахе», говоря, что у Гегеля «сотворение мира принимает еще
гораздо более несуразный и невозможный вид, чем в христианстве» (вспомним,
что сам Гегель определяет содержание логики как «изображение бога, каков он
есть в своей вечной сущности
}\textstyleEndnoteTextEmphasis{до
сотворения}\textstyleEndnoteText{ природы и конечного
духа», — т.~I «Науки логики», стр.~28; цитата из Энгельса
взята по немецкому изданию «Л. Фейрбаха», Москва 1932, стр.~28)
Рациональную сторону перехода от идеи к природе отмечает в своем Конспекте
Ленин. Выписав гегелевскую фразу: «А именно... есть
}\textstyleEndnoteTextEmphasis{природа}\textstyleEndnoteText{»,
Ленин замечает: «Эта фраза на
}\textstyleEndnoteTextEmphasis{последней}\textstyleEndnoteText{,
[213]-ой }\textstyleEndnoteText{странице
}\textstyleEndnoteTextEmphasis{Логики}\textstyleEndnoteText{
архизамечательна. Переход логической идеи к
}\textstyleEndnoteTextEmphasis{природе}\textstyleEndnoteText{.
Рукой подать к материализму. Прав был Энгельс, что система
Гегеля перевернутый материализм»
(}\textstyleEndnoteTextEmphasis{Ленин}\textstyleEndnoteText{,
Философские тетради, М. 1936, стр.~224). С этим указанием
Ленина интересно сопоставить замечания Маркса в его
«Экономическо-философских рукописях 1844 года». Маркс пишет: Абсолютная
идея, как результат всей логики Гегеля, «в свою очередь снимает самое себя,
если только она не хочет снова проделать сначала весь процесс абстракции и
удовольствоваться тем, чтобы быть тотальностью всех абстракций или
постигающей себя абстракцией. Но абстракция, постигающая себя как
абстракцию, знает, что она есть ничто: она должна отказаться от себя,
отказаться от абстракции, и таким образом она приходит к такой сущности,
которая есть ее прямая противоположность, — к
}\textstyleEndnoteTextEmphasis{природе}\textstyleEndnoteText{.
Вся логика представляет собой, стало быть, доказательство
того, что абстрактное мышление, рассматриваемое само по себе (оторванно от
}\textstyleEndnoteTextEmphasis{действительного}\textstyleEndnoteText{
духа и от
}\textstyleEndnoteTextEmphasis{действительной}\textstyleEndnoteText{
природы), есть ничто, что абсолютная идея, взятая сама по
себе (как нечто самостоятельное по отношению к природе и духу), есть ничто
и что только
}\textstyleEndnoteTextEmphasis{природа}\textstyleEndnoteText{
есть нечто» (}\textstyleEndnoteTextEmphasis{Marx
—~Engels}\textstyleEndnoteText{, Gesamtausgabe, hrsg. v. Adoratskij, Erste
Abteilung, Bd. III, Berlin 1932, S. 168–169).}\textstyleEndnoteText{ }}
\item 
\label{bkm:Ref484782429}\textstyleEndnoteText{К
стр.~\pageref{bkm:bm133}. — Имеется в виду «Философия духа»
как третья часть гегелевской системы.}}
\end{enumerate}
\clearpage\setcounter{page}{1}\subsubsection[Перевод
важнейших
терминов«Науки
логики»
Гегеля]{Перевод
важнейших терминов\newline
«Науки логики» Гегеля}
\hypertarget{Toc486359860}{}
\bigskip

\begin{multicols}{2}

Allgemeine, das —~всеобщее }

Allgemeinheit —~всеобщность }

Allheit —~всякость (см. прим. \ref{bkm:Ref484772654}) }

Am{}-Etwas{}-Sein —~бытие-в-нечто }

Andere, das —~другое, иное }

Anderssein —~инобытие, инаковость }

Aiujerswerden —~иностановление, становление иным }

An ihm —~в нем }

Anschauung —~созерцание }

An-sich —~в себе }

Ansichsein —~в-себе-бытие }

An-und-fur-sich-sein —~в-себе-и-для-себя-бытие }

Anzahl —~численность }

Attraktion —~притяжение }

Aufheben —~1) снимание, снятие (как технический термин
гегелевской философии); 2) упразднение, устранение (во всех остальных
случаях) }

Auseinandersein —~внеположность }

Aussereinander, das —~внеположность }

Aussereinandersein —~бытие-вне-друг-друга, внеположность }

Ausser-sich-sein —~вне{}-себя{}-бытие }

Äusserung —~проявление во-вне }

\bigskip

Begriff —~понятие (см. прим. \ref{bkm:Ref485481734}) }

Bei sich —~у себя }

Bei-sich-sein —~у-себя-бытие (иногда: замыкание в себя) }

Beschaffenheit —~характер }

Besondere, das —~особенное }

Bestehen —~устойчивое наличие, устойчивое существование (в
отдельных случаях: составленность, состояние, существование) }

Bestimmtheit —~определенность (иногда: определенный характер) }

Bestimmtsein —~определенность, определяемость }

Bestimmtwerden —~определяемость }

Bestimmung —~определение (иногда в смысле назначения) }

Beziehung —~соотношение (иногда: соотнесение) }

Böse, das —~(нравственное) зло}

\bigskip

Dasein —~1) наличное бытие (как технический термин); 2)
существование (во всех остальных случаях) }

Definition —~дефиниция }

Denkbestimmungen —~ определения мысли, мыслительные
определения }

Denkformen —~формы мысли }

Diese, das —~этость }

Differenz —~различие, небезразличие, диферентность }

Ding —~вещь }

Ding-an-sich —~вещь{}-в{}-себе }

Dingheit —~вещность }

Diskretion —~дискретность}

\bigskip

Eine, das —~единое }

Eines —~одно }

Einheit —~1) единица (как единица измерения и как момент
числа); 2) единство }

Eins —~одно, единое }

Einteilung —~деление, подразделение }

Einzelne, das —~единичное }

Element —~стихия }

Endlichkeit —~конечность, конечное }

Entäusserung —~отчуждение }

Entgegensetzung —~противоположение }

Entstehen —~возникновение }

Entwicklung —~1) развитие, развертывание; 2) разложение в ряд
(в математике) }

Entwicklungspotenz —~степенной член разложения (см прим. 53 к
т.~I) }

Existenz —~существование}

\bigskip

Für-Eines-Sein —~бытие-для-одного (см прим. 33 к т.~I) }

Für-es-sein —~для-него-бытие }

Fur-sich —~1) для себя (как технический термин); 2) особо,
само по себе, отдельно, самостоятельно (во всех остальных случаях) }

Fur-sich-sein —~для-себя-бытие}

\bigskip

Gedankending —~вещь, сочиненная мыслью, нечто лишь мысленное,
а не реальное; голая абстракция }

Gegensatz —~противоположность (в отдельных случаях:
противоречие) }

Gehalt —~содержимое, содержательность (иногда: содержание) }

Geist —~дух }

Gemüt —~душа }

Gesetzisein —~положенность }

Gewissheit —~достоверность, уверенность }

Gleichheit —~равенство, одинаковость }

Grad —~градус, степень }

Grenze —~1) граница (как технический термин логики Гегеля); 2)
предел (в математике) }

Grösse —~величина }

Grund —~основание (Zu Grunde gehen —~идти ко
дну, погружаться в основание, см. прим. 81 к т.~I) }

Grundlage —~основа }

Gute, das —~добро, благо}

\bigskip

Ideelle, das —~идеализованное }

Identität —~тождество }

Indifferenz —~индиференция (см. прим. 69 к т.~I) }

Inhalt —~содержание }

Inhärenz —~присущность }

In-sich-Sein —~внутри-себя-бытие}

\bigskip

Jenseits, das —~потустороннее}

\bigskip

Kontinuität —~непрерывность}

\bigskip

Leere, das —~пустота }

Lehrsatz —~теорема}

\bigskip

Masslose, das —~безмерное }

Mehrheit —~многость }

Mitte (die) des Schlusses~—~средний термин силлогизма}

\bigskip

Negativität —~отрицательность }

Nichtdasein —~неимение наличного бытия }

Nichtidentität —~нетождество }

Nichtsein —~небытие }

Nichtunterschiedensein —~неразличность}

\bigskip

Punktualitat  —~точечность}

\bigskip

Quantität —~количество }

Quantum —~определенное количество}

\bigskip

Räsonnement —~рассуждение (иногда: рассуждательство) }

Rasonnieren —~рассуждательство, рассуждение }

Reflektiertsein —~рефлектированность }

Reflex —~отражение }

Reflexion —~рефлексия (см. прим 78 к т.~I; в отдельных случаях
переводится через «соображение») }

Reflexionsbestimmungen —~рефлективные (рефлексивные)
определения, определения рефлексии }

Regel —~правило (см прим 61 к т.~I) }

Repulsion —~отталкивание}

\bigskip

Sache —~1) мыслимая вещь; 2) суть; 3) вещь, предмет }

Satz —~предложение, положение, начало (например, «начало
противоречия») }

Schein —~видимость (в отдельных случаях переводится через
«свечение») }

Scheinen —~свечение, свечение видимостью, излучение видимости
(см. прим. 78 к т.~I) }

Schluss —~умозаключение, силлогизм (см. прим.
\ref{bkm:Ref486285580}) }

Schlusssatz —~заключение }

Schranke —~предел }

Seele —~душа }

Seelending —~душа-вещь }

Sein —~бытие }

Selbst, das —~самость }

Selbstbewegung —~самодвижение }

Setzen —~полагание }

Sollen —~долженствование }

Sprung —~скачок }

Staatsökonomie —~политическая экономия }

Stetigkeit —~непрерывность }

Subsumtion —~подведение под более общее, подчинение более
общему}

\bigskip

Totalität —~тотальность, целостность, целокупность }

Trieb —~влечение (в отдельных случаях: стремление, движущее
начало, импульс) }

Triplizität —~троичность, тройственность}

\bigskip

Ungleichheit —~неравенство, неодинаковость }

Universum —~вселенная, универсум }

Unmittelbare, das —~непосредственное }

Unterschied —~различие }

Unterschiedensein —~различность }

Urteil —~суждение (см. прим. \ref{bkm:Ref484772231})}

\bigskip

Vergehen —~прехождение }

Verhältnis —~отношение}

Vermittelung —~опосредствование }

Vernunft —~разум }

Verschiedenheit —~разность }

Verstand —~рассудок (в отдельных случаях: смысл) }

Vielheit —~множественность }

Voraussetzen —~предполагать, предполагать (в смысле «полагать
наперед», «предпосылать») }

Voraussetzung —~предположение, предпосылка, пред-положение }

Vorstellen —~представливание, представление }

Vorstellung —~представление}

\bigskip

Wahrheit —~истина, истинность}

Werden —~становление}

Wesen —~сущность}

Wesenheit —~1) определенная сущность (как особая категория);
2) сущность (в остальных случаях)}

Wesentlichkeit —~существенность}

Widerspruch —~ противоречие}

Wollen, das —~воля}

\bigskip

Ziel —~цель стремлений}

Zusammenfallen —~сжиматься, свертываться}

Zusammensetzung —~составность}

Zweck —~цель }
\end{multicols}
\begin{multicols}{2}

\bigskip
\end{multicols}
\clearpage\setcounter{page}{1}\subsubsection[Библиография
]{Библиография }
\hypertarget{Toc486359861}{}\paragraph[I.
Немецкие
издания
большой
логики
]{I. Немецкие издания большой логики }
\begin{enumerate}
\item 
\textstyleEndnoteText{Wissenschaft der Logik. Von D.
}\textstyleEndnoteTextEmphasis{Ge. Wilh. Friedr.
Hegel}\textstyleEndnoteText{, Professor und Rector am Königl. Bayerischen
Gymnasium zu Ntirnberg. Erster Band: Die objective Logik, Nurnberg 1812.}}
\end{enumerate}

\textstyleEndnoteTextEmphasis{Idem}\textstyleEndnoteText{., Erster Band: Die
objective Logik; Zweites Buch: Die Lehre vom Wesen, Nurnberg 1813.}}

\textstyleEndnoteTextEmphasis{Idem}\textstyleEndnoteText{., Zweiter Band:
Die subjective Logik oder Lehre vom Begriff, Nurnberg 1816.}}

\begin{enumerate}
\item 
\textstyleEndnoteTextEmphasis{G. W. F. Hegel's}\textstyleEndnoteText{ Werke,
Vollständige Ausgabe durch einen Verein von Freunden des Verewigten.}}
\end{enumerate}

\textstyleEndnoteText{Band III: Wissenschaft der Logik, Herausgegeben von
Leopold von Henning, Teil I: Die objective Logik; Abteilung I: Die Lehre
vom Sein, Berlin 1833.}}

\textstyleEndnoteText{Band IV: Die objective Logik; Abteilung 2: Die Lehre
vom Wesen, Berlin 1834.}}

\textstyleEndnoteText{Band V: Wissenschaft der Logik; Teil 2: Die subjective
Logik oder die Lehre vom Begriff, Berlin 1834.}}

\begin{enumerate}
\item 
\textstyleEndnoteTextEmphasis{G. W. F. Hegel's}\textstyleEndnoteText{ Werke,
Vollständige Ausgabe durch einem Verein von Freunden des Verewigten.}}
\end{enumerate}

\textstyleEndnoteText{Band III: Wissenschaft der Logik, Herausgegeben von
Leopold von Henning; Teil I: Die objective Logik; Abteilung I: Die Lehre
vom Sein, 2-te unveränderte Auflage, Berlin 1841.}}

\textstyleEndnoteText{Band IV: Die objective Logik; Abteilung 2: Die Lehre
vom Wesen, 2-te unveränderte Auflage, Berlin 1841.}}

\textstyleEndnoteText{Band V: Wissenschaft der Logik; Teil 2: Die subjective
Logik oder die Lehre vom Begriff, 2-te unveränderte Auflage, Berlin 1841.
}}

\begin{enumerate}
\item 
\textstyleEndnoteTextEmphasis{Hegel}\textstyleEndnoteText{, Wissenschaft der
Logik, Herausgegeben von Georg Lasson; Erster Teil (Die Lehre vom Sein),
Leipzig 1923.}}
\end{enumerate}

\textstyleEndnoteTextEmphasis{Idem}\textstyleEndnoteText{., Zweiter Teil
(Die Lehre vom Wesen. Die Lehre vom Begriff), Leipzig 1923. }}

\begin{enumerate}
\item 
\textstyleEndnoteTextEmphasis{Hegel}\textstyleEndnoteText{, Sämtliche Werke,
Jubiläumsausgabe, herausgegeben von H. Glockner.}}
\end{enumerate}

\textstyleEndnoteText{Band IV: Wissenschaft der Logik; Teil I: Die objective
Logik, Stuttgart 1928.}}

\textstyleEndnoteText{Band V: Wissenschaft der Logik; Teil 2: Die subjective
Logik, Stuttgart 1928. }}

\begin{enumerate}
\item 
\textstyleEndnoteTextEmphasis{Hegel}\textstyleEndnoteText{, Wissenschaft der
Logik, herausgegeben von Georg Lasson, Erster Teil (Die Lehre vom Sein),
2-te Auflage, Leipzig 1933.}}
\end{enumerate}

\textstyleEndnoteTextEmphasis{Idem}\textstyleEndnoteText{., Zweiter Teil
(Die Lehre vom Wesen. Die Lehre vom Begriff), 2-te Auflage, Leipzig 1934.
}}

\begin{enumerate}
\item 
\textstyleEndnoteTextEmphasis{Hegel}\textstyleEndnoteText{, Sämtliche Werke,
Jubiläumsausgabe, herausgegeben von H. Glockner.}}
\end{enumerate}

\textstyleEndnoteText{Band IV: Wissenschaft der Logik; Teil I: Die objective
Logik, 2-te Auflage, Stuttgart 1936.} }

\paragraph[II.
Переводы
большой
логики
на
другие
языки
]{II. Переводы большой логики на другие языки }
\begin{enumerate}
\item 
\textstyleEndnoteTextEmphasis{Гегель}\textstyleEndnoteText{,
Наука логики, пер. с немецкого Н. Г. Дебольского, ч. I.
Объективная логика; кн. 1: Учение о бытии, Пг. 1916.}}
\end{enumerate}

\textstyleEndnoteText{То же, ч. I., кн. 2: Учение о сущности,
Пг. 1916.}}


\textstyleEndnoteText{То же, ч. II. Субъективная логика или
учение о понятии, Пг. 1916. }}

\begin{enumerate}
\item 
\textstyleEndnoteTextEmphasis{Hegel}\textstyleEndnoteText{, La scienza della
logica. Traduzione italiana con note di Arturo Moni, Vol. 1–3, Bari 1925.
}}
\item 
\textstyleEndnoteTextEmphasis{Гегель}\textstyleEndnoteText{,
Наука логики, Пер. с немецкого Н. Г. Дебольского
(перепечатано на правах рукописи с издания 1916 г.), Москва, Профком
слушателей Института Красной профессуры, 1929 (издано одним томом с
раздельной пагинацией по трем частям). }}
\item 
\textstyleEndnoteTextEmphasis{Hegel}\textstyleEndnoteText{, Science of
Logic, Translated by W. H. Johnston and L. G. Struthers, With an
introduction by Haldane of Cloan, vol. I (The doctrine of being) and vol. 2
(The doctrine of essence. The doctrine of the notion), London 1929. }}
\item 
\textstyleEndnoteTextEmphasis{Hegel's}\textstyleEndnoteText{ doctrine of
formal logic, being a translation of the first section of the subjective
logic, with introduction and notes by H. S. Macran, Oxford 1912. }}
\item 
\textstyleEndnoteTextEmphasis{Hegel's}\textstyleEndnoteText{ logic of world
and idea, being a translation of the 2-nd and 3-d parts of the subjective
logic, with introduction on Idealism limited and absolute by H. S. Macran,
Oxford 1929. }}
\end{enumerate}
\paragraph[III.
Маркс,
Энгельс,
Ленин
о
логике
Гегеля
]{III. Маркс, Энгельс, Ленин о логике Гегеля }
{\centering
\textstyleEndnoteText{А) \ Маркс и Энгельс}
\par}

\begin{enumerate}
\item 
\textstyleEndnoteTextEmphasis{Маркс}\textstyleEndnoteText{,
Из критики философии права Гегеля (Соч., т.~I, Госполитиздат,
1938, стр.~588). }}
\item 
\textstyleEndnoteTextEmphasis{Энгельс}\textstyleEndnoteText{,
Шеллинг и откровение (Соч., т.~II, М. — Л.
1931, стр.~129, 139–141). }}
\item 
\textstyleEndnoteTextEmphasis{Маркс}\textstyleEndnoteText{
и
}\textstyleEndnoteTextEmphasis{Энгельс}\textstyleEndnoteText{,
Святое семейство (Соч., т.~III, M. — Л. 1929,
стр.~114, 168). }}
\item 
\textstyleEndnoteTextEmphasis{Маркс}\textstyleEndnoteText{,
Подготовительные работы для «Святого семейства» (Соч.,
т.~III, М. — \ Л. 1929, стр.~632, 650–652).
}}
\item 
\textstyleEndnoteTextEmphasis{Маркс}\textstyleEndnoteText{
и
}\textstyleEndnoteTextEmphasis{Энгельс}\textstyleEndnoteText{,
Немецкая идеология (Соч., т.~IV, стр.~100, 102, 213, 222,
246–247, 257). }}
\item 
\textstyleEndnoteTextEmphasis{Маркс}\textstyleEndnoteText{,
Нищета философии (Соч., т. V, стр.~361–363). }}
\item 
\textstyleEndnoteTextEmphasis{Маркс}\textstyleEndnoteText{,
К критике политической экономии (Соч., т.~XII, ч. 1-я,
стр.~80). }}
\item 
\textstyleEndnoteTextEmphasis{Маркс}\textstyleEndnoteText{,
Введение к «Критике политической экономии» (Соч., т.~XII, ч.
1-я, стр.~191–192). }}
\item 
\textstyleEndnoteTextEmphasis{Энгельс}\textstyleEndnoteText{,
Рецензия на книгу Маркса «К критике политической экономии»
(Соч., т.~XI, ч. 2-я, стр.~358–359). }}
\item 
\textstyleEndnoteTextEmphasis{Маркс}\textstyleEndnoteText{,
Капитал, т.~I (Соч., т.~XVII, стр.~19–20, 199, 288, 339). }}
\item 
\textstyleEndnoteTextEmphasis{Маркс}\textstyleEndnoteText{,
Капитал, т.~I, изд. 1-е, СПБ 1872, стр.~13, 15, 17, 19. }}
\item 
\textstyleEndnoteTextEmphasis{Маркс}\textstyleEndnoteText{,
Капитал, т.~III, Партиздат, 1936, стр.~686. }}
\item 
\textstyleEndnoteTextEmphasis{Энгельс}\textstyleEndnoteText{,
Карл Маркс о капитале (Соч., т.~XIII, ч. 1-я, стр.~258). }}
\item 
\textstyleEndnoteTextEmphasis{Энгельс}\textstyleEndnoteText{,
К жилищному вопросу (Соч., т.~XV, стр.~60). }}
\item 
\textstyleEndnoteTextEmphasis{Энгельс}\textstyleEndnoteText{,
Анти-Дюринг, Госполитиздат, 1938, стр.~30, 34, 37–38, 39, 43,
44, 49, 55, 94, 99, 101, 104, 106, 118, 270, 273, 281, 282, 285, 294, 295,
322. }}
\item 
\textstyleEndnoteTextEmphasis{Энгельс}\textstyleEndnoteText{,
Диалектика природы, Партиздат, 1936, стр.~3, 6, 7, 8, 9, 10,
46, 47, 70, 73, 81, 82, 85, 100–103, 109, 111, 112, 113, 114, 116, 122,
125, 127, 129, 194. }}
\item 
\textstyleEndnoteTextEmphasis{Энгельс}\textstyleEndnoteText{,
Л. Фейербах, Госполитиздат, 1938, стр.~6, 9, 19, 35. }}
\item 
\textstyleEndnoteTextEmphasis{Энгельс}\textstyleEndnoteText{,
Введение к англ. изданию «Развития социализма» (Соч., т.~XVI,
ч. 2-я, стр.~293). }}
\end{enumerate}

\textstyleEndnoteText{Кроме того, следующие письма: }}

\begin{enumerate}
\item 
\textstyleEndnoteText{Энгельс Марксу от 19. XI. 1844 (Соч.,
т.~XXI, стр.~7). }}
\item 
\textstyleEndnoteText{Маркс Энгельсу от 13. XI. 1857 (Соч.,
т.~XXII, стр.~251). }}
\item 
\textstyleEndnoteText{Маркс Энгельсу от 8. XII. 1857 (Соч.
т.~XXII, стр.~266). }}
\item 
\textstyleEndnoteText{Маркс Энгельсу от 14. I. 1858 (Соч.,
т.~XXII, стр.~290–291). }}
\item 
\textstyleEndnoteText{Маркс Энгельсу от 1. II. 1858 (Соч.,
т.~XXII, стр.~299). }}
\item 
\textstyleEndnoteText{Маркс Лассалю от 31. V. 1858 (Соч.,
т.~XXV, стр.~229). }}
\item 
\textstyleEndnoteText{Энгельс Марксу от 14. VII. 1858 (Соч.,
т.~XXII, стр.~345–346). }}
\item 
\textstyleEndnoteText{Маркс Энгельсу от 28. XI. 1860 (Соч.,
т.~XXII, стр.~542). }}
\item 
\textstyleEndnoteText{Маркс Энгельсу от 27. II. 1861 (Соч.,
т.~XXIII, стр.~15). }}
\item 
\textstyleEndnoteText{Маркс Энгельсу от 9. XII. 1861 (Соч.,
т.~XXIII, стр.~51). }}
\item 
\textstyleEndnoteText{Энгельс Альберту Ланге от 29. III. 1865
(Соч., т.~XXV, стр.~452), }}
\item 
\textstyleEndnoteText{Маркс Энгельсу от 19. VIII. 1865 (Соч.,
т.~XXIII, стр.~305). }}
\item 
\textstyleEndnoteText{Энгельс-Марксу от 16. VI. 1867 (Соч.,
т.~XXIII, стр.~414–415). }}
\item 
\textstyleEndnoteText{Маркс Энгельсу от 22. VI. 1867 (Соч.,
т.~XXIII, стр.~417). }}
\item 
\textstyleEndnoteText{Маркс Кугельману от 6. III. 1868 (Соч.,
т.~XXV, стр.~516). }}
\item 
\textstyleEndnoteText{Маркс Энгельсу от 25. III. 1868 (Соч.,
т.~XXIV, стр.~34–35). }}
\item 
\textstyleEndnoteText{Маркс Энгельсу от 14. IV. 1870 (Соч.,
т.~XXIV, стр.~318). }}
\item 
\textstyleEndnoteText{Маркс Кугельману от 27. VI. 1870 (Соч.,
т.~XXVI, стр.~58). }}
\item 
\textstyleEndnoteText{Маркс Энгельсу от 31. V. 1873 (Соч.,
т.~XXIV, стр.~415). }}
\item 
\textstyleEndnoteText{Энгельс Марксу от 21. IX. 1874 (Соч.,
т.~XXIV, стр.~442). }}
\item 
\textstyleEndnoteText{Энгельс Марксу от 18. VIII. 1881 (Соч.,
т.~XXIV, стр.~531–532). }}
\item 
\textstyleEndnoteText{Энгельс Конраду Шмидту от 27. X. 1890
(Маркс и Энгельс. Письма, пер. Адоратского, изд. 4-е, М. —
Л. 1931, стр.~386). }}
\item 
\textstyleEndnoteText{Энгельс К. Шмидту от 1. VII. 1891
(Письма, стр.~386). }}
\item 
\textstyleEndnoteText{Энгельс К. Шмидту от 1. XI. 1891
(Письма, стр.~292–294). }}
\item 
\textstyleEndnoteText{Энгельс К. Шмидту от 4. II. 1892
(Письма, стр.~394). }}
\end{enumerate}
{\centering
\textstyleEndnoteText{Б) \ Ленин}
\par}

\begin{enumerate}
\item 
\textstyleEndnoteText{Что такое «друзья народа» (Соч., т.~I,
стр.~80, 82, 84). }}
\item 
\textstyleEndnoteText{Шаг вперед, два шага назад (Соч., т.~VI,
стр.~325–326). }}
\item 
\textstyleEndnoteText{Марксизм и ревизионизм (Соч., т.~XII,
стр.~184–185). }}
\item 
\textstyleEndnoteText{Материализм и эмпириокритицизм (Соч.,
т.~XIII, стр.~103–104, 154, 156, 186–187, 190, 253, 276). }}
\item 
\textstyleEndnoteText{Письмо Горькому от 16. XI. 1909 (Соч.,
т.~XIV, стр.~186). }}
\item 
\textstyleEndnoteText{Памяти Герцена (Соч., т.~XV, стр.~464).
}}
\item 
\textstyleEndnoteText{Три источника и три составных части
марксизма (Соч., т.~XVI, стр.~350). }}
\item 
\textstyleEndnoteText{Карл Маркс (Соч., т.~XVIII, стр.~10–12).
}}
\item 
\textstyleEndnoteText{Крах II Интернационала (Соч., т.~XVIII,
стр.~247). }}
\item 
\textstyleEndnoteText{Еще раз о профсоюзах (Соч., т.~XXVI,
стр.~134–135). }}
\item 
\textstyleEndnoteText{О значении воинствующего материализма
(Соч., т.~XXVII, стр.~187–188). }}
\item 
\textstyleEndnoteText{Конспект книги Гегеля «Наука логики»
(Философские тетради, Госполитиздат, 1938, стр.~87–228). }}
\item 
\textstyleEndnoteText{Замечания на книгу Ж. Ноэля «Логика
Гегеля» (Философские тетради, стр.~229–236). }}
\item 
\textstyleEndnoteText{План диалектики (логики) Гегеля
(Философские тетради, стр.~237–242). }}
\item 
\textstyleEndnoteText{Конспект книги Гегеля «Лекции по истории
философии» (Философские тетради, стр.~274 и 283). }}
\item 
\textstyleEndnoteText{К вопросу о диалектике (Философские
тетради, стр.~323–328). }}
\item 
\textstyleEndnoteText{Заметки по отзывам немецких,
французских, английских и итальянских авторов о логике Гегеля (Философские
тетради, стр.~422–425).} }
\end{enumerate}
\paragraph[IV Из
буржуазной
литературы
о
логике
Гегеля
]{IV Из буржуазной литературы о логике Гегеля }
\begin{enumerate}
\item 
\textstyleEndnoteTextEmphasis{Baillie, J. B.,}\textstyleEndnoteText{ The
origine and significance of Hegel's Logic, London 1901. }}
\item 
\textstyleEndnoteTextEmphasis{Emge, K. A.,}\textstyleEndnoteText{ Hegels
Logik und die Gegenwart, Karlsruhe 1927. }}
\item 
\textstyleEndnoteTextEmphasis{Hibben, J. G}\textstyleEndnoteText{., Hegel's
Logic, an essay of an interpretation, New York 1902. }}
\item 
\textstyleEndnoteTextEmphasis{Mac-Taggart}\textstyleEndnoteText{, A
commentary on Hegel's Logic, Cambridge 1910. }}
\item 
\textstyleEndnoteTextEmphasis{Michelet}\textstyleEndnoteText{ und
}\textstyleEndnoteTextEmphasis{Haring}\textstyleEndnoteText{,
Historisch-kritische Darstellung der dialektischen Methode Hegels, Leipzig
1888. }}
\item 
\textstyleEndnoteTextEmphasis{Noël, G}\textstyleEndnoteText{., La logique de
Hegel, Paris 1897. }}
\item 
\textstyleEndnoteTextEmphasis{Schmitt, E. H}\textstyleEndnoteText{., Das
Geheimniss der Hegelschen Dialektik, beleuchtet vom konkret-sinnlichen
Standpunkte, Halle 1888. }}
\item 
\textstyleEndnoteTextEmphasis{Stirling. J. H}\textstyleEndnoteText{., The
secret of Hegel, two volumes, London 1865. }}
\item 
\textstyleEndnoteTextEmphasis{Wallace, W}\textstyleEndnoteText{.,
Prolegomena to the study of Hegel's philosophy and especially of his Logic,
Oxford and London 1894.}}
\end{enumerate}
\clearpage\setcounter{page}{1}
\bigskip

\subsection[Оглавление]{Оглавление}

\bigskip


\bigskip

{\centering
\textit{\textcolor[rgb]{0.0,0.0,0.039215688}{Гегель}}\textcolor[rgb]{0.0,0.0,0.039215688}{
\ «НАУКА ЛОГИКИ» T. II. Субъективная логика}
\par}

\setcounter{tocdepth}{7}
\renewcommand\contentsname{}
\tableofcontents

\bigskip

\clearpage\setcounter{page}{1}
\bigskip
\end{document}
}
