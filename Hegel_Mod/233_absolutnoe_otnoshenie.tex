\subsubsection{Третья глава. Абсолютное Отношение}
Абсолютная необходимость не есть ни
{\em необходимое}, ни тем менее
{\em некоторое} необходимое, а есть
{\em необходимость} —~бытие всецело как рефлексия. Она
есть отношение, так как она есть такое различение, моменты которого суть
сами вся ее тотальность; они, таким образом, абсолютно
{\em устойчиво-наличны}, но так, что образуют лишь
единое устойчивое наличие и различие есть только
{\em видимость} развертывания, и эта видимость есть
само абсолютное. — Сущность как таковая есть рефлексия или свечение
видимостью; но сущность как абсолютное отношение есть
{\em видимость, положенная как
}{\em видимость}, которая, как это соотнесение с собой,
есть {\em абсолютная действительность}. — Абсолютное,
развернутое сначала {\em внешней рефлексией},
развертывает теперь, как абсолютная форма или как необходимость, само себя;
это развертывание самого себя есть его полагание самого себя, и оно
{\em есть} лишь это самополагание. — Подобно тому как
{\em свет} в природе не есть ни нечто, ни вещь, а его
бытие есть лишь его свечение, так и проявление есть равная самой себе
абсолютная действительность.

Стороны абсолютного отношения не суть поэтому
{\em атрибуты}. В атрибуте абсолютное светится лишь в
одном из своих моментов, как в чем-то
{\em пред-положенном} и воспринятом
{\em внешней рефлексией}. Но
{\em развертывается-то} абсолютное
{\em абсолютной необходимостью}, которая тождественна с
собой, как определяющая самое себя. Так как она есть свечение, положенное
как видимость, то стороны этого отношения суть
{\em тотальности}, потому что они даны (sind) как
видимость; ибо, как видимость, различия суть они же сами и их
противоположное или, иначе говоря, они суть целое; и обратно, они суть
видимость таким образом потому, что они суть тотальности. Это различение
или свечение абсолютного есть, таким образом, лишь тождественное полагание
самого себя.

Это отношение в его непосредственном понятии есть отношение
{\em субстанции} и
{\em акциденций}, непосредственное исчезание и
становление абсолютной видимости в самом себе. Когда субстанция определяет
себя как {\em для-себя-бытие}, противостоящее
некоторому {\em другому}, или, иначе говоря, когда
абсолютное отношение выступает как реальное, то это отношение есть
{\em отношение причинности}. Наконец, когда последнее,
как соотносящееся с собой, переходит во
{\em взаимодействие}, то тем самым абсолютное
отношение, по содержащимся в нем определениям, также и
{\em положено}; это {\em положенное
единство} себя в своих {\em определениях},
{\em которые сами положены как целое}, и тем самым
положены также и как определения, есть затем
{\em понятие}.

\paragraph[А. \ Отношение субстанциальности]{А. \ Отношение
субстанциальности}
Абсолютная необходимость есть абсолютное
отношение, так как она есть не {\em бытие} как таковое,
а такое {\em бытие}, которое есть,
{\em потому что} оно есть, бытие как абсолютное
опосредствование себя с самим собою. Это бытие есть
{\em субстанция}; как окончательное единство сущности и
бытия она есть бытие во {\em всяком} бытии; она не есть
ни нерефлектированное непосредственное, ни нечто абстрактное, стоящее
позади существования и явления, а есть сама непосредственная
действительность и притом как абсолютная рефлектированность в себя, как
в-себе-и-для-себя-сущее {\em устойчивое наличие}. —
Субстанция, как это единство бытия и рефлексии, есть по существу их
{\em свечение} и
{\em положенность}. Свечение есть
{\em соотносящееся с собой} свечение; таким образом,
оно {\em имеет бытие}; это бытие есть субстанция как
таковая. И обратно, это бытие есть лишь тождественная с собой
{\em положенность}; таким образом, оно есть
{\em светящаяся видимостью тотальность,
акцидентальность.}

Это свечение есть тождество как тождество формы, — единство возможности и
действительности. Это единство есть, во-первых,
{\em становление}, случайность, как сфера возникновения
и прехождения; ибо по определению
{\em непосредственности} соотношение возможности и
действительности есть {\em непосредственное
превращение} их друг в друга как {\em сущих},
превращение каждого из них в свое другое как лишь
{\em другое} для него. — Но так как бытие есть
видимость, то их соотношение есть также соотношение тождественных или
светящихся видимостью друг в друге, рефлексия. Движение акцидентальности
представляет поэтому в каждом из своих моментов свечение
{\em категорий} бытия и
{\em рефлективных определений} сущности друг в друге. —
Непосредственное {\em нечто} имеет некоторое
{\em содержание}; его непосредственность есть вместе с
тем рефлектированное безразличие к форме. Это содержание определено, а так
как эта определенность есть определенность бытия, то нечто
{\em переходит} в некоторое другое. Но качество есть
также и определенность рефлексии; оно есть, таким образом, безразличная
{\em разность}. Но последняя оживляется (begeistet
sich) так, что становится {\em противоположением} и
возвращается в основание, которое есть {\em ничто}, но
также и {\em рефлексия в себя}. Последняя снимает себя;
но она сама есть рефлектированное в-себе-бытие, и таким образом, она есть
возможность, а это в-себе-бытие в своем переходе, который есть также и
рефлексия в себя, есть {\em необходимое
действительное}.

Это движение акцидентальности есть {\em деятельность}
субстанции как {\em спокойное выплывание ее самой}. Она
деятельна не {\em по отношению} к нечто, а лишь по
отношению к себе, как к простому, не оказывающему сопротивления элементу.
Снятие чего-либо {\em пред-положенного} есть исчезающая
видимость; лишь в таком действии, которое {\em снимает}
непосредственное, возникает само это непосредственное или, иначе говоря,
имеет бытие указанное свечение; начинание с себя самого есть только впервые
полагание того самого, с которого начинают.

Субстанция как это тождество свечения есть тотальность целого и охватывает
собою акцидентальность, а акцидентальность есть вся субстанция сама.
Различие, состоящее в ее распадении на {\em простое
тождество} {\em бытия} и на
{\em смену} акциденций в этом тождестве, есть некоторая
форма ее видимости. {\em Первое} есть лишенная формы
{\em субстанция}
{\em представливания}, для которого видимость
определилась не как видимость, а которое крепко держится, как за некоторое
абсолютное, за такое неопределенное тождество, которое не имеет истинности
в представляет собою лишь определенность
{\em непосредственной} действительности или также
{\em в-себе-бытия} или возможности, — за те определения
формы, которые имеют место в акцидентальности.

Другое определение, {\em смена акциденций}, есть
абсолютное {\em единство формы} акцидентальности,
субстанция как {\em абсолютная мощь}. — Прехождение
акциденции есть ее возвращение, как действительности, в себя, как в свое
в-себе-бытие или в свою возможность; но это ее в-себе-бытие есть само лишь
положенность; поэтому оно есть также и действительность, а так как эти
определения формы суть в такой же мере определения содержания, то это
возможное есть и по содержанию некоторое иначе определенное действительное.
Переводя возможность в действительность с ее содержанием, субстанция
проявляет себя как {\em творящую мощь}, а возвращая
действительное в возможность, она проявляет себя как
{\em разрушающую} мощь. Но и то и другое тождественно:
творчество разрушает, разрушение —~творит; ибо отрицательное и
положительное, возможность и действительность, абсолютно соединены в
субстанциальной необходимости.

{\em Акциденции} как таковые, — а их
{\em много}, так как множественность есть одно из
определений бытия, — {\em не имеют власти} одна над
другой. Они суть сущие или для-себя-сущие нечто, существующие вещи с
многообразными свойствами или целые, состоящие из частей, самостоятельные
части, силы, нуждающиеся в возбуждении друг другом и имеющие друг друга
условием. Если при этом кажется, что одно такое акцидентальное обнаруживает
некоторую власть над другим, то на самом деле оба их объемлет собою власть
субстанции, которая, как отрицательность, полагает неодинаковые значения,
определяя одно как преходящее, а другое —~с другим содержанием и как
возникающее, или, иначе сказать, определяя первое, как переходящее в свою
возможность, а второе, как переходящее при этом в действительность;
субстанция вечно раздваивается на эти различия формы и содержания и вечно
очищает себя от этой односторонности, но в самом этом очищении впадает
опять в определение и раздвоение. — Одна акциденция вытесняет,
следовательно, другую лишь потому, что ее собственное
{\em существование} (Subsistiren) само есть та
тотальность формы и содержания, в которой равно исчезают и она и ее другая.

Вследствие этого {\em непосредственного тождества} и
наличия субстанции в акциденциях здесь еще нет
{\em реального} различия. В этом
{\em первом} определении субстанция еще не проявлена по
всему своему понятию. Если различают субстанцию, как тождественное с собой
{\em в-себе-и-для-себя-бытие}, от нее же самой, как
тотальности {\em акциденций}, то она как
{\em мощь} есть нечто
{\em опосредствующее}. Эта мощь есть
{\em необходимость}, положительное
{\em упорное пребывание} акциденций в их
отрицательности и их голая {\em положенность} в их
устойчивом наличии; этот {\em средний термин} есть,
стало быть, единство субстанциальности и акцидентальности, и его
{\em крайние термины} не имеют своего собственного
устойчивого наличия. Субстанциальность есть поэтому лишь отношение как
непосредственно исчезающее, она не соотносится с собой
{\em как отрицательное}, и, будучи непосредственным,
единством мощи с самой собой, имеет бытие в {\em форме}
лишь {\em своего тождества}, а не своей
{\em отрицательной сущности}; лишь один момент, а
именно, отрицательное или различие, есть безоговорочно исчезающий; другой
же момент, т.~е. тождественное, не есть таковой. — Это можно рассматривать
также и следующим образом. Видимость или акцидентальность есть, правда,
благодаря мощи, {\em в себе} субстанция, но она не
{\em положена} как эта тождественная с собой видимость;
таким образом, субстанция имеет своим образом (Gestalt) или своей
положенностью лишь акцидентальность, а не самое себя; она не есть
субстанция {\em как} субстанция. Отношение
субстанциальности есть, следовательно, субстанция ближайшим образом лишь в
том смысле, что субстанция {\em открывает} себя как
{\em формальную мощь}, различия которой не
субстанциальны; она есть на самом деле лишь
{\em внутреннее} акциденций, и последние имеют бытие
\ лишь {\em в субстанции}. Или, иначе говоря, это
отношение есть лишь светящаяся видимостью тотальность как
{\em становление}; но она в такой же мере есть также и
рефлексия; акцидентальность, которая {\em в себе} есть
субстанция, именно поэтому также и {\em положена} как
таковая; таким образом, она {\em определена}, как
соотносящаяся с собой {\em отрицательность}, в отличие
от себя, определенной, как соотносящееся с собой простое
{\em тождество} с собой; и она есть
{\em для-себя-сущая}, {\em мощная
субстанция}. Таким образом, отношение субстанциальности
{\em переходит} в {\em отношение
причинности}.

\paragraph[В. \ Отношение причинности]{В. \ Отношение причинности}
Субстанция есть мощь и мощь
{\em рефлектированная в себя}, не просто переходящая,
но и полагающая {\em определения} и
{\em отличающая их от себя}. Как соотносящаяся с самой
собой в своем процессе определения, {\em она сама} есть
то, что она полагает как отрицательное или, иначе говоря, что она делает
{\em положенностью}. Последняя, стало быть, есть вообще
снятая субстанциальность, нечто лишь положенное
—~{\em действие}; а для-себя-сущая субстанция есть
{\em причина}.

Это отношение причинности есть ближайшим образом лишь указанное
{\em отношение причины и действия}; таким образом, оно
есть {\em формальное отношение причинности}.

\subparagraph[а) \ Формальная причинность]{а) \ Формальная причинность}
1. Причина есть нечто
{\em первоначальное} по сравнению с действием. —
Субстанция есть, как мощь, {\em свечение видимостью},
или {\em обладает} акцидентальностью. Но она, как мощь,
есть в своей видимости также и рефлексия в себя; таким образом, она
{\em развертывает} свой переход, и
{\em это свечение определено как видимость}, или, иначе
сказать, акциденция {\em положена} как некоторое лишь
{\em положенное}. — Однако в своем процессе определения
субстанция исходит не от акцидентальности, как будто бы последняя была
{\em наперед} некоторым
{\em другим} и лишь теперь положена как определенность,
а обе суть единая деятельность. Субстанция, как мощь,
{\em определяет себя}; но этот процесс определения сам
есть непосредственно снятие процесса определения и возвращение.
{\em Она определяет себя, — она}, определяющее, есть,
таким образом, {\em непосредственное} и то, что само
уже определено. Следовательно, определяя {\em себя},
она полагает это уже {\em определенное} как
{\em определенное}; она, таким образом, сняла
положенность и возвратилась в себя. — И обратно, это возвращение, так как
оно есть {\em отрицательное} соотношение субстанции с
собой, само есть некоторый {\em процесс определения}
или отталкивание субстанции от себя самой; благодаря этому возвращению
{\em возникает} то определенное, с которого, как
кажется, начинает субстанция и которое, как преднайденное определенное, она
теперь, как кажется, полагает в качестве такового. — Таким образом,
абсолютная деятельность есть {\em причина} —~мощь
субстанции в {\em ее истине}, как проявление, которое
непосредственно также и {\em развертывает} в его
становлении то, что есть {\em в себе}, акциденцию
(которая есть положенность), {\em полагает} ее как
{\em положенность}, —
{\em действие}. — Последнее есть, следовательно,
{\em во-первых}, то же самое, что и акцидентальность в
отношении субстанциальности, а именно, субстанция как
{\em положенность}; но,
{\em во-вторых}, акциденция как таковая субстанциальна
лишь благодаря своему исчезанию, лишь как преходящая; как действие же она
есть положенность как тождественное с собой; причина проявлена в действии
как вся субстанция, а именно, как рефлектированная в себя в самой
положенности как таковой.

2. Этой рефлектированной в себя {\em положенности},
определенному как определенному, противостоит субстанция как
{\em неположенное} первоначальное. Так как она, как
абсолютная мощь, есть возвращение в себя, но само это возвращение \ есть
{\em процесс определения}, то она уже больше не есть
простое «{\em в себе}» своей акциденции, но также
{\em и положена} как это в-себе-бытие. Поэтому лишь как
причина субстанция обладает впервые
{\em действительностью}. Но эта действительность.
заключающаяся в том, что {\em в-себе-бытие} субстанции,
ее определенность внутри отношения субстанциальности, теперь уже положена
как {\em определенность}, — эта действительность есть
{\em действие}; поэтому та действительность, которую
субстанция имеет как причина, она имеет {\em лишь в
своем действии}. — Это —~та {\em необходимость},
которая есть причина. — Она есть {\em действительная}
субстанция, так как субстанция, как мощь, определяет самое себя; но она
есть вместе с тем причина, так как она развертывает эту определенность или
полагает ее как положенность; таким образом, она полагает свою
действительность как положенность или как действие. Последнее есть другое
причины, положенность по отношению к первоначальному, и
{\em опосредствовано} через последнее. Но причина, как
необходимость, вместе с тем упраздняет этот свой процесс опосредствования и
есть в {\em процессе определения} себя самой (как
первоначально соотносящееся с собой в
{\em противоположность} опосредствованному) возвращение
в себя; ибо положенность определена {\em как}
положенность и, стало быть, тождественна с собой; поэтому лишь в своем
действии причина есть истинно действительное и тождественное с собой. —
Действие {\em необходимо} потому, что оно есть именно
проявление причины, или, иначе говоря, есть та необходимость, которая есть
причина. — Лишь как эта необходимость, причина есть нечто самодвижущее,
начинает спонтанейно, не будучи возбуждена некоторым другим, и есть
{\em самостоятельный источник, порождение из себя}; она
необходимо должна {\em действовать}; ее
первоначальность состоит в том, что ее рефлексия в себя есть определяющее
полагание и что, обратно, и то и другое составляют единое единство.

{\em Действие поэтому не содержит в себе вообще ничего
такого, чего не содержит в себе причина.} И обратно,
{\em причина не содержит в себе ничего такого, чего нет
в ее действии}. Причина есть причина лишь постольку, поскольку она
порождает некоторое действие; и {\em причина есть не
что другое, как это определение, состоящее в том, чтобы иметь некоторое
действие, }а {\em действие есть не что другое, как то,
что оно имеет некоторую причину}. В самой причине как таковой заключается
ее действие, и в самом действии —~причина; поскольку причина еще не
действовала бы или поскольку она перестала бы действовать, постольку она не
была бы причиной, и действие, поскольку его причина исчезла, уже больше не
есть действие, а представляет собою некоторую безразличную
действительность.

3. В этом {\em тождестве} причины и действия снята
теперь та форма, через которую они различаются между собою как в-себе-сущее
и как положенное. Причина {\em потухает} в своем
действии; тем самым потухло также и действие, ибо оно есть лишь
определенность причины. Эта потухшая в действии причинность есть тем самым
некоторая {\em непосредственность}, безразличная к
отношению причины и действия и содержащая в себе это отношение внешним
образом.

\subparagraph[b) \ Определенное отношение причинности]{b) \ Определенное
отношение причинности}
1. {\em Тождество} причины с
собой в ее действии есть снятие ее мощи и отрицательности, и потому есть
безразличное к различиям формы единство,
{\em содержание}. — Последнее поэтому лишь
{\em в себе} соотнесено с формой, в данном случае —~с
причинностью. Они тем самым положены как {\em разные},
и форма есть по отношению к содержанию лишь непосредственно действительная
форма, некоторая {\em случайная} причинность.

Далее, содержание, взятое, таким образом, как определенное, есть некоторое
разное содержание в себе самом; и причина определена по своему содержанию,
а тем самым и действие также определено таким образом. — Так как здесь
рефлектированность есть также и непосредственная действительность, то
содержание есть постольку {\em действительная}, но
{\em конечная субстанция}.

Таково теперь {\em отношение причинности} в его
реальности и {\em конечности}. Как формальное, оно есть
бесконечное отношение абсолютной мощи, содержанием которой служит чистое
проявление или необходимость. Напротив, как конечная причинность, оно имеет
некоторое {\em данное} содержание и исчерпывается
внешним различием, имеющимся на этом тождественном, которое в своих
определениях есть одна и та же субстанция.

В силу такого {\em тождества содержания}, эта
причинность представляет собой {\em аналитическое}
предложение. {\em Одна и та же вещь} оказывается в
одном случае причиной, а в другом действием, там —~самостоятельным
устойчивым наличием, здесь —~положенностью или определением в некотором
другом. Так как эти определения формы суть
{\em внешняя} рефлексия, то, когда определяют некоторое
явление как действие и восходят от него к его причине для того, чтобы
постигнуть и объяснить его, это представляет собою
{\em по существу дела} тавтологическое рассмотрение
{\em субъективным} рассудком; одно и то же содержание
лишь повторяется дважды; в причине не имеется ничего, чего нет в действии.
— Например, дождь есть причина мокроты, которая есть его действие;
«{\em дождь производит мокроту}», это есть
аналитическое предложение; та же самая вода, которая составляет дождь, и
есть мокрота; но как дождь, эта вода существует в форме особой вещи, а как
мокрота или влажность, она есть как бы прилагательное, нечто положенное,
которое, как предполагается, уже не имеет своего устойчивого наличия в себе
самом; и как одно, так и другое определение одинаково внешни воде. —
Подобным же образом причина {\em этого цвета} есть
нечто окрашивающее, {\em пигмент}, который есть одна и
та же действительность, выступающая в одном случае во внешней ей форме
чего-то действующего, т.~е. как связанная внешним образом с отличным от нее
действующим, а во втором случае —~в столь же внешнем для нее определении
некоторого действия. — Причиной какого-либо
{\em деяния} служит внутреннее умонастроение некоторого
действующего субъекта, которое, как внешнее наличное бытие, сообщаемое ему
(этому умонастроению) поступком, представляет собою то же самое содержание
и ту же самую ценность. Если {\em движение} какого-либо
тела рассматривается, как действие, то причиной его служит некоторая
{\em толкающая} сила; но и до и после толчка имеется
одно и то же количество движения, одно и то же существование, содержавшееся
в толкающем теле и сообщенное им толкаемому; и сколько оно сообщает,
столько же оно само и теряет.

Причина, например живописец или толкающее тело, имеет, правда,
{\em еще и другое} содержание; живописец имеет еще и
другое содержание помимо красок и их формы, соединяющей краски и образующей
из них картину, а толкающее тело имеет еще другое содержание помимо
движения определенной силы и определенного направления. Но это дальнейшее
содержание есть случайный придаток, не касающийся причины; какие бы другие
качества живописец ни содержал в себе, если мы берем его абстрагируясь от
того обстоятельства, что он есть живописец данной картины, — это не входит
в состав нашей картины; лишь те из его свойств, которые сказываются в
{\em действии}, присущи ему
{\em как причине}; по прочим же своим свойствам он не
есть причина. Точно так же, есть ли толкающее теш о камень или дерево,
зеленое ли оно, желтое и~т.~п., это не соучаствует в его толчке, и
{\em постольку} это тело не есть причина.

По поводу {\em этой тавтологичности} отношения
причинности следует заметить, что оно может показаться не содержащим в себе
тавтологии в тех случаях, когда указываются не ближайшие, а
{\em отдаленные причины} какого-либо действия.
Изменение формы, претерпеваемое лежащей в основании вещью в этом переходе
через многие промежуточные члены, прикрывает собою то тождество, которое
при этом удерживается самой вещью. Вместе с тем в этом умножении причин,
вдвигающихся между нею и окончательным действием, она связывается с другими
вещами и обстоятельствами, так что не то первое, которое признается в этом
случае причиной, а лишь все эти многие причины
{\em вместе} взятые заключают в себе полное действие. —
Так например, если человек попал в обстоятельства, при которых развился его
талант, вследствие того, что он потерял своего отца, убитого пулей в
сражении, то можно указывать на этот выстрел (или, если итти еще далее
назад, на войну или некоторую причину войны и~т.~д. до бесконечности), как
на причину мастерства этого человека. Но ясно, что, например, не этот
выстрел есть сам по себе причина, а причиной служит лишь соединение его с
другими действующими определениями. Или, правильнее сказать, он вообще есть
не причина, а лишь отдельный {\em момент} в
{\em обстоятельствах}, сделавших
{\em возможным} результат.

Затем следует главным образом обратить еще внимание на
{\em недопустимое применение} отношения причинности к
{\em обстоятельствам физико-органической и духовной
жизни}. Здесь то, что называется причиной, оказывается, конечно, имеющим
другое содержание, чем действие, {\em но это происходит
оттого}, что то, что действует на живое, самостоятельно определяется,
изменяется и преобразуется последним, {\em ибо живое не
дает причине дойти до ее действия}, т.~е. упраздняет ее как причину. Так
например, недопустимо говорить, что пища есть
{\em причина} крови, или что такие-то кушанья или
холод, сырость суть {\em причины} лихорадки и~т.~п.;
так же недопустимо указывать на климат Ионии как на
{\em причину} творений Гомера, или на честолюбие Цезаря
как на {\em причину} падения республиканского строя в
Риме. Вообще в {\em истории} действуют и определяют
друг друга духовные массы и индивидуумы; природе же духа еще в более
высоком смысле, чем характеру живого вообще, скорее свойственно
{\em не принимать в себя другого первоначального} или,
иначе говоря, не допускать продолжения в нем (в духе) какой-либо причины,
а, наоборот, прерывать и преобразовывать ее. — Но такого рода отношения
принадлежат области {\em идеи} и имеют быть рассмотрены
лишь тогда, когда дойдем до нее. — Здесь же можно заметить еще то, что,
поскольку допускается отношение причины и действия хотя бы и не в
собственном смысле, действие не может быть
{\em больше}, чем причина; ибо действие есть не что
иное, как проявление причины. В истории стало обычным острое словцо, что
{\em из малых причин происходят большие действия}, и
поэтому для объяснения широко и глубоко захватывающего события приводят
какой-нибудь {\em анекдот} как первую причину. Такая
так называемая причина должна рассматриваться лишь как
{\em повод}, лишь как {\em внешнее
возбуждение}, в котором {\em внутренний дух} события и
не нуждался бы или вместо которого он мог бы воспользоваться бесчисленным
множеством других поводов, чтобы начать с них в явлении, пробить себе путь
и проявиться. Скорее наоборот, {\em лишь самим этим
внутренним духом события} такого рода нечто, само по себе мелкое и
случайное, {\em было определено} к тому, чтобы стать
вызвавшим его поводом. Эта {\em арабескная манера}
излагать историю, согласно которой из ничтожного зыбкого стебля вырастает
какой-либо большой образ (Gestalt), представляет собою поэтому хотя и
остроумную, но в высшей степени поверхностную трактовку истории. В этом
возникновении великого из малого имеет, правда, вообще место операция
оборачивания, которую дух проделывает над внешним; но именно потому это
внешнее не есть {\em причина внутри его} (духа), или,
иначе сказать, самое это оборачивание снимает отношение причинности.

2. Но эта {\em определенность} отношения причинности,
состоящая в том, что содержание и форма разны и безразличны, простирается
далее. {\em Определение формы} есть также и
{\em определение содержания}; причина и действие, обе
стороны отношения, суть поэтому так же и {\em другое
содержание}. Или, иначе говоря, содержание, ввиду того, что оно дано лишь
как содержание некоторой формы, имеет в себе самом различие этой формы и
существенно разно. Но так как эта его форма есть отношение причинности,
которое есть некоторое тождественное в причине и действии содержание, \ то
разное {\em содержание} связано
{\em внешним образом}, с одной стороны, с
{\em причиной}, а с другой —~с
{\em действием}; оно, стало быть,
{\em не входит} само в
{\em процесс} \ {\em действия} (in
das Wirken) и в {\em отношение}.

Таким образом, это внешнее содержание лишено отношения; оно есть
{\em некоторое непосредственное существование}; или,
иначе говоря, так как оно как содержание есть
{\em в-себе-сущее} тождество причины и действия, то оно
равным образом есть {\em непосредственное, сущее}
тождество. Это есть поэтому некоторая {\em вещь},
имеющая многообразные определения своего наличного бытия и
{\em между прочим} также и то определение, что она в
{\em каком-нибудь отношении} есть причина или также и
действие. Определения формы —~причина и действие —~имеют в этой вещи свой
{\em субстрат}, т.~е. свое существенное устойчивое
наличие, и притом каждое из них свое особое, ибо их тождество есть их
устойчивое наличие; но вместе с тем она [вышеуказанная вещь] есть их
непосредственное устойчивое наличие, а не их устойчивое наличие как
единство формы или как отношение.

Но эта вещь есть не только субстрат, а также и субстанция, ибо она есть
тождественное устойчивое наличие лишь {\em как
устойчивое наличие отношения}. Далее, она есть
{\em конечная} субстанция, ибо она определена как
непосредственная {\em по отношению} к ее причинности.
Но она вместе с тем обладает причинностью, так как она равным образом есть
тождественное лишь как тождественное этого отношения. — Как причина, этот
субстрат есть отрицательное соотношение с {\em собой}.
Но он сам, с которым он соотносится, есть,
{\em во-первых}, некоторая положенность, так как он
определен как {\em непосредственно}{}-действительное;
эта положенность, как содержание, есть какое-либо определение вообще. —
{\em Во-вторых}, {\em причинность}
ему внешня; {\em она, стало быть, сама образует его
положенность}. Поскольку же он есть причинная субстанция, его причинность
состоит в том, что он соотносится отрицательно с собой и, следовательно, со
своей положенностью и с внешней причинностью. Процесс действия этой
субстанции начинает поэтому с некоторого внешнего, освобождается от этого
внешнего определения, и его возвращение в себя есть сохранение своего
непосредственного существования и снятие своей положенной причинности и тем
самым своей причинности вообще.

Так например, находящийся в движении камень есть причина; его движение есть
некоторое обладаемое им определение, помимо которого, однако, он содержит в
себе еще многие другие определения цвета, фигуры и~т.~д., не входящие в
состав его причинности. Так как его непосредственное существование отделено
от его соотношения формы, а именно, от причинности, то последняя есть нечто
{\em внешнее}; его движение и присущая ему в этом
движении причинность суть в нем лишь
{\em положенность}. — Но причинность есть также и
{\em его собственная} причинность; это сказывается в
том, что его субстанциальное устойчивое наличие есть его тождественное
соотношение с собой, а последнее теперь определено как положенность; оно
есть, следовательно, вместе с тем {\em отрицательное
соотношение} с собой. — Поэтому присущая камню причинность, направляющаяся
на себя как на положенность или как на некоторое внешнее, состоит в том,
что она снимает это внешнее и возвращается в себя через его
{\em удаление}; тем самым она постольку состоит не в
том, чтобы быть тождественной с собой {\em в своей
положенности}, а лишь в том, что она восстанавливает
{\em свою абстрактную первоначальность}.—Или, например,
дождь есть причина мокроты, которая есть та же самая вода, что и он. Эта
вода имеет определение быть дождем и причиной вследствие того, что оно
положено в ней некоторым другим; другая сила (или что бы там ни было)
подняла ее в воздух и собрала в такую массу, тяжесть которой заставляет ее
падать вниз. Ее удаление от земли есть определение, чуждое ее
первоначальному тождеству с собой, т.~е. тяжести; ее причинность состоит в
том, что она устраняет это чуждое ей определение и вновь восстанавливает
первоначальное тождество, но тем самым также и упраздняет свою причинность.

Рассматриваемая теперь {\em вторая определенность}
причинности касается {\em формы}; это отношение есть
{\em причинность как внешняя себе самой}, как такая
{\em первоначальность}, которая вместе с тем есть в ней
самой также и {\em положенность} или
{\em действие}. Это соединение противоположных
определений в {\em сущем} субстрате образует собой
{\em бесконечный регресс} от причин к причинам. —
Начинают с действия; оно как таковое имеет некоторую причину, последняя в
свою очередь имеет некоторую причину и~т.~д. Почему причина имеет в свою
очередь некоторую причину? То есть почему {\em та самая
сторона}, которая ранее была определена {\em как
причина}, теперь определяется как {\em действие},
вследствие чего возникает вопрос о некоторой новой причине? —~Потому, что
причина есть вообще некоторое {\em конечное,
определенное}; она определена как {\em один} момент
формы, противостоящий действию; таким образом, она имеет свою
определенность или отрицание вне себя; но именно вследствие этого сама она
{\em конечна}, имеет {\em свою
определенность в самой себе} и тем самым представляет собою
{\em положенность} или
{\em действие}. Это ее тождество также положено, но оно
есть некоторое {\em третье}, непосредственный субстрат;
причинность внешня самой себе потому, что здесь ее
{\em первоначальность} есть
{\em непосредственность}. Различие формы есть поэтому
первая определенность, {\em определенность}, еще
{\em не положенная как} определенность; оно есть
{\em сущее инобытие}. Конечная рефлексия, с одной
стороны, задерживается на этом непосредственном, отстраняет от него
единство формы и признает, что оно {\em в одном
отношении} есть причина, а {\em в другом} —~действие; с
другой стороны, она перемещает единство формы в
{\em бесконечное} и продолжающимся без конца все далее
и далее движением выражает свое бессилие достигнуть и удержать его.

Точно так же обстоит дело с {\em действием}; или,
правильнее сказать, {\em бесконечный прогресс от
действия к действию} есть совершенно то же, что и
{\em регресс от причины к причине}. В последнем
{\em причина} становилась
{\em действием}, в свою очередь имеющим некоторую
{\em другую} причину; и точно так же обратно
{\em действие} становится
{\em причиной}, в свою очередь имеющей некоторое
{\em другое} действие. — Рассмотренная нами
определенная причина начинает с какой-либо внешности и в своем действии не
возвращается обратно в себя {\em как причина}, а скорее
теряет в нем причинность. Но и, обратно, действие приходит к такому
субстрату, который есть субстанция, первоначально соотносящееся с собой
устойчивое наличие; в нем поэтому эта положенность становится
{\em положенностью}, т.~е. эта субстанция, когда в ней
полагается действие, {\em ведет себя как причина}. Но
то первое действие или, иначе сказать, та положенность, которая
{\em внешним образом} приходит к субстанции, есть нечто
{\em другое}, чем второе,
{\em производимое ею} действие; ибо это второе действие
определено как {\em ее рефлексия}
{\em в себя}, а то первое —~как нечто
{\em внешнее в ней}. — Но так как здесь каузальность
есть внешняя самой себе причинность, то и в своем действии она
{\em не} возвращается {\em обратно
в себя}; она становится в этом действии {\em внешней}
себе; {\em ее} действие становится снова положенностью
в некотором субстрате как в некоторой {\em другой
субстанции}, которая, однако, точно так же делает эту положенность
положенностью, или, иначе говоря, проявляет себя как причину, снова
отталкивает от себя свое действие и~т.~д. до дурной бесконечности.

3. Теперь мы должны посмотреть, что получилось в результате движения
определенного отношения причинности. — Формальная причинность потухает в
действии; благодаря этому {\em возникла
тождественность} обоих этих моментов, но тем самым лишь
{\em как} единство {\em в себе}
причины и действия, в котором соотношение формы внешне. — Вследствие этого
указанная тождественность также {\em непосредственна}
по обоим определениям непосредственности, во-первых, как
{\em в-себе-бытие}, некоторое такое
{\em содержание}, в котором причинность протекает
внешне; {\em во-вторых}, как некоторый
{\em существующий} субстрат, которому
{\em присущи} причина и действие, как различенные
определения формы. Последние суть при этом {\em в себе}
одно, но в силу этого {\em в-себе-бытия} или внешности
формы каждое из них внешне самому себе и поэтому в своем
{\em единстве} с другим определено относительно него
также и как {\em другое}. Поэтому, хотя причина имеет
действие и {\em вместе с тем сама есть действие}, а
действие не только имеет некоторую причину, но и
{\em само есть также и причина}, однако действие,
которое {\em имеет} причину, и действие,
{\em которое есть причина}, разны, а равным образом
разны между собой причина, {\em имеющая} действие, и
причина, {\em которая есть действие}.

Но через движение определенного отношения причинности получилось теперь то,
что причина {\em не только потухает} в действии, а тем
самым потухает и действие, — как это было в формальной причинности, — но
что причинность в {\em своем потухании}, т.~е. в
действии, вновь {\em возникает} и что
{\em действие}, {\em исчезая} в
причине, равным образом вновь {\em возникает} в ней.
Каждое из этих определений {\em упраздняет себя в своем
полагании} и {\em полагает себя в своем упразднении};
здесь имеется не {\em внешний переход} причинности от
одного субстрата на другой, а это их {\em становление
другими} есть вместе с тем их {\em собственное
полагание}. Причинность, следовательно,
{\em предполагает} сама себя или
{\em обусловливает} себя. Тождество, бывшее прежде лишь
{\em в-себе-сущим} тождеством, субстратом, теперь
{\em определено} поэтому как
{\em пред-положение} или
{\em положено в противоположность действующей}
причинности, и {\em рефлексия}, бывшая прежде лишь
{\em внешней} для тождественного, теперь находится к
нему в {\em отношении}.

\subparagraph[с) \ Действие и противодействие]{с) \ Действие и
противодействие}
Причинность есть
{\em пред-полагающее} делание. Причина
{\em обусловлена}; она есть отрицательное соотношение с
собой как пред-положенное, как внешнее другое, которое
{\em в себе}, но лишь {\em в себе},
есть сама причинность. Это другое есть, как оказалось, то
{\em субстанциальное тождество}, в которое переходит
формальная причинность, определившая себя теперь
{\em против этого субстанциального тождества} как его
отрицательное. Или, иначе говоря, оно есть то же самое, что субстанция
причинного отношения, но как такая субстанция, которой противостоит мощь
акцидентальности как то, что само есть
{\em субстанциальная деятельность}. — Это
—~{\em пассивная} субстанция. —
{\em Пассивным} является непосредственное или такое
в-себе-сущее, которое не есть также и {\em для себя}, —
чистое бытие или сущность, которой присуща лишь эта определенность
{\em абстрактного тождества с собой}. — Пассивной
субстанции противостоит соотносящаяся с собой отрицательно
{\em деятельная} субстанция. Последняя есть причина,
поскольку она в определенной причинности через отрицание самой себя
восстановила себя вновь из действия;— нечто такое, что в своем инобытии или
как непосредственное ведет себя по существу как
{\em полагающее} и опосредствует себя с собой через
свое отрицание. Поэтому причинность здесь уже больше не имеет никакого
{\em субстрата}, которому она была бы
{\em присуща}, и есть не определение формы по отношению
к этому тождеству, а сама она есть субстанция, или, иначе говоря,
первоначальное есть только причинность. —
{\em Субстрат} есть та пассивная субстанция, которую
причинность пред-положила себе.

Итак, эта причина {\em действует}, ибо она есть
отрицательная власть {\em над самой собой}; вместе с
тем она есть свое же {\em пред-положенное}; таким
образом, она действует на себя как {\em на некоторое
другое}, на {\em пассивную субстанцию}. — Тем самым
она, {\em во-первых}, {\em снимает
инобытие} последней и возвращается в ней обратно в себя; во-вторых, она
{\em определяет} ее, полагает это снятие своего
инобытия или возвращение в себя как некоторую
{\em определенность}. Эта положенность, поскольку она
есть вместе с тем ее возвращение в себя, есть ближайшим образом
{\em ее действие}. Но и обратно, так как она, как
пред-полагающая, определяет самое себя, как свое другое, то она полагает
действие {\em в другой}, пассивной
{\em субстанции}. — Или, иначе говоря, так как
пассивная субстанция сама есть нечто {\em двоякое}, а
именно, некоторое самостоятельное {\em другое} и вместе
с тем некоторое {\em пред-положенное} и в себе уже
{\em тождественное} с действующей причиной, то и
процесс действия последней сам есть нечто двоякое; он есть то и другое
сразу: снятие {\em ее определенности} (т.~е. ее условия
или снятие самостоятельности пассивной субстанции) и снятие действующей
причиной своего тождества, с пассивной субстанцией, так что действующая
причина тем самым {\em пред-полагает} себя или, иначе
говоря, полагает себя как {\em другое}. — Благодаря
последнему моменту пассивная субстанция
{\em сохраняется}; то первое снятие ее представляется
по отношению к этому моменту вместе с тем также и так, что
{\em лишь некоторые ее определения} снимаются и что
тождество ее с действующей причиной появляется в ее действии внешним
образом.

Постольку она испытывает {\em насилие}. — Насилие есть
{\em явление мощи или мощь как нечто внешнее}. Но мощь
есть нечто внешнее лишь постольку, поскольку причинная субстанция в ее
процессе действия, т.~е. в полагании себя самой, есть вместе с тем
пред-полагающая, т.~е. полагает самоё себя как снятую. Поэтому и обратно,
дело насилия есть также и дело мощи. Насильственная причина действует
только на ею же самой пред-положенное другое; ее действие на это последнее
есть отрицательное соотношение {\em с собой} или
проявление {\em самой себя}. Пассивное есть такое
самостоятельное, которое есть лишь некоторое
{\em положенное}, некоторое надломленное внутри самого
себя, — такого рода действительность, которая есть условие и притом теперь
уже условие в его истине, а именно, некоторая действительность, которая
есть лишь возможность, или, обратно, некоторое
{\em в-себе-бытие}, которое есть лишь
{\em определенность в-себе-бытия}, лишь пассивно.
Поэтому над тем, что подвергается насилию, не только можно, но и
{\em необходимо должно} совершать насилие; то, что
имеет возможность совершать насилие над другим, имеет эту возможность лишь
потому, что в совершаемом им насилии его собственная мощь
{\em проявляет} как себя, так и другое. Пассивная
субстанция {\em полагается} насилием лишь как то, что
она есть {\em поистине}: так как она есть простое
положительное или непосредственная субстанция, то именно потому она есть
лишь нечто {\em положенное}. То
«{\em пред}», которое она есть как условие (т.~е. то
обстоятельство, что она, как условие, есть нечто пред-положенное. —
Перев.), есть та видимость непосредственности, которую действующая
причинность сдирает с нее.

Поэтому пассивная субстанция, подвергаясь воздействию некоторого другого по
отношению в ней насилия, получает лишь должное.
{\em Теряет} она при этом указанную
{\em непосредственность},
{\em чуждую ей} субстанциальность. То, что она
{\em получает} как нечто
{\em чуждое}, а именно, то обстоятельство, что ее
определяют как некоторую {\em положенность}, есть ее
собственное определение. — Но когда ее полагают в ее положенности или в
{\em ее собственном} определении, то она этим скорее не
снимается, а {\em сливается таким образом лишь с самой
собой} и есть, следовательно, {\em в своей
определяемости первоначальность}. — Следовательно, пассивная субстанция, с
одной стороны, {\em сохраняется} или
{\em полагается} активной субстанцией, а именно,
поскольку последняя делает самое себя снятой; но, с другой стороны, то
обстоятельство, что пассивная субстанция сливается с самой собой и тем
самым делает себя чем-то первоначальным и
{\em причиной}, есть {\em дело
самой этой пассивной }{\em субстанции}.
{\em Полагаемость} некоторым другим и собственное
{\em становление} есть одно и то же.

Благодаря тому, что пассивная субстанция теперь сама превращена в причину,
действие в ней, {\em во-первых}, снимается; в этом
состоит вообще ее {\em противодействие}. Она есть
{\em в себе} положенность, как пассивная субстанция;
положенность была также и {\em положена} внутри ее
другою субстанциею, поскольку, именно, она восприняла в себя
{\em действие} этой последней. В ее противодействии
заключается поэтому равным образом двоякое, а именно, во-первых, что то,
что она есть {\em в себе},
{\em полагается}, и, во-вторых, что то, какою она
{\em полагается}, оказывается ее
{\em в-себе-бытием}; она есть
{\em в себе положенность}, поэтому она получает
некоторое действие от другой; но эта положенность есть, обратно,
{\em ее} собственное в-себе-бытие: таким образом, это
есть {\em ее} собственное действие, она сама
оказывается причиной.

{\em Во-вторых}, противодействие направлено
{\em против первой действующей причины}. Ибо то
действие, которое субстанция, бывшая прежде пассивной, снимает внутри себя,
есть именно как раз вышеупомянутое действие первой действующей причины. Но
причина имеет свою субстанциальную действительность лишь в своем действии;
так как последнее снимается, то снимается ее причинная субстанциальность.
Это происходит, во-первых, {\em в себе через ее самое},
поскольку она делает себя действием; в этом тождестве исчезает ее
отрицательное определение, и она становится чем-то пассивным; во-вторых,
это происходит {\em через бывшую прежде пассивной}, а
теперь воздействующую ответно субстанцию, \ которая снимает действие первой
причины. — Правда, в {\em определенной причинности}
субстанция, на которую оказывается воздействие, в свою очередь также
становится причиной, и, стало быть, действует
{\em против} того обстоятельства, что в ней было
положено некоторое {\em действие}. Но она не
действовала обратно {\em против той причины}, а
полагала свое действие опять-таки в {\em некоторую
другую} субстанцию, вследствие чего появлялся бесконечный прогресс
действий; ибо здесь причина тождественна с собой в своем действии пока что
лишь {\em в себе}, и потому она, с одной стороны, в
своем {\em покое} исчезает в некотором
{\em непосредственном} тождестве, а, с другой стороны,
пробуждается вновь в некоторой {\em другой} субстанции.
— Напротив, в обусловленной причинности причина
{\em соотносит} себя в действии
{\em с самой собой}, так как действие есть ее другое
как условие, как пред-{\em положенное}, и ее
воздействие благодаря этому есть столь же
{\em становление}, сколь и полагание и
{\em снимание другого}.

Она, далее, тем самым ведет себя как пассивная субстанция; но, как оказалось
из предшествующего, последняя через произведенное на нее действие
{\em возникает} как причинная субстанция. Та первая
причина, которая действует первой и получает свое действие в себя обратно
как противодействие, тем самым снова выступает как причина, вследствие чего
процесс действия, оказывающийся в конечной причинности дурно-бесконечным
прогрессом, {\em поворачивается обратно} и становится
возвращающимся назад в себя, бесконечным
{\em взаимодействием}.

\paragraph[С. \ Взаимодействие]{С. \ Взаимодействие}
В конечной причинности сторонами отношения
служат субстанции, которые относятся друг к другу как воздействующие.
{\em Механизм} состоит в этой
{\em внешности} причинности, в том, что
{\em рефлексия} причины в ее действии
{\em в себя} есть вместе с тем некоторое отталкивающее
{\em бытие}, или, иначе говоря, что в том
{\em тождестве с собой}, которое свойственно причинной
субстанция в ее действии, она столь же непосредственно остается чем-то
{\em внешним} себе, и действие
{\em перешло} на некоторую
{\em другую субстанцию}. Теперь во взаимодействии этот
механизм снят; ибо оно заключает в себе,
{\em во-первых}, {\em исчезновение}
вышеупомянутого первоначального {\em устойчивого
сохранения непосредственной} субстанциальности, а,
{\em во-вторых}, {\em возникновение
причины}, и тем самым содержит в себе
{\em первоначальность} как
{\em опосредствующую} себя с собой через свое
{\em отрицание}.

Ближайшим образом взаимодействие представляется взаимной причинностью
{\em предположенных},
{\em обусловливающих} друг друга
{\em субстанций}; каждая есть относительно другой
{\em одновременно} и активная и пассивная
{\em субстанция}. Поскольку обе тем самым и активны и
пассивны, постольку всякое различие между ними уже снято; оно есть
совершению прозрачная видимость; они суть субстанции лишь постольку,
поскольку они суть тождество активного и пассивного. Само взаимодействие
есть поэтому все еще лишь {\em пустой вид и способ}; и
теперь еще требуется только некоторое внешнее суммирование того, что уже и
имеется {\em в себе и положено}. Во-первых, теперь
находятся друг с другом в соотношении уже не
{\em субстраты}, а субстанции; в движении обусловленной
причинности упразднилась еще остававшаяся
{\em предположенная непосредственность}, и
{\em обусловливающий} характер причинной активности
заключается еще только в {\em воздействии} или в
{\em собственной} пассивности. Но это воздействие
исходит, далее, не от {\em другой} первоначальной
субстанции, а как раз от такой причинности, которая обусловлена
воздействием или, иначе говоря, которая есть некоторое
{\em опосредствованное}. Это ближайшим образом
{\em внешнее}, которое приходит в причину и составляет
сторону ее пассивности, опосредствовано поэтому {\em ею
самою}; оно порождено ее собственной активностью и есть тем самым
{\em положенная самой ее активностью пассивность}. —
Причинность обусловлена и обусловливает;
{\em обусловливающее} есть
{\em пассивное}, но в той же мере
{\em пассивно и обусловленное}. Это обусловливание или
пассивность есть {\em отрицание} причины ею же самою,
так как она существенным образом делает себя
{\em действием} и именно благодаря этому есть причина.
{\em Взаимодействие} есть поэтому лишь сама
причинность; причина не только {\em имеет} некоторое
действие, а в действии она {\em как причина} находится
в соотношении с самой собою.

Благодаря этому причинность возвращена к {\em ее
абсолютному понятию} и вместе с тем она дошла до самого
{\em понятия}. Она есть ближайшим образом реальная
необходимость, абсолютное {\em тождество} с собой, так
что различие необходимости и соотносящиеся в ней друг с другом определения
суть по отношению друг к другу субстанции или
{\em свободные действительности}. Необходимость есть,
таким образом, {\em внутреннее тождество}; причинность
есть его проявление, в котором ее видимость
{\em субстанциального инобытия} сняла себя и
необходимость возведена в {\em свободу}. — Во
взаимодействии первоначальная причинность представляется как
{\em возникновение} из ее отрицания, т.~е. из
пассивности, и как {\em прехождение} в этой
пассивности, — представляется как {\em становление}, но
так, что это становление есть вместе с тем в такой же мере лишь
{\em свечение видимостью}; переход в
{\em другое} есть рефлексия в себя само; то
{\em отрицание}, которое есть основание причины, есть
ее {\em положительное слияние} с самой собой.

\label{bkm:bm93a}В этом слиянии необходимость и причинность, следовательно,
исчезли; они содержат в себе и то и другое:
{\em непосредственное тождество}, как
{\em связь} и {\em соотношение}, и
{\em абсолютную субстанциальность различенных} и, стало
быть, их абсолютную {\em случайность}, — содержат в
себе первоначальное {\em единство} субстанциальных
{\em различий}, — следовательно, абсолютное
противоречие. Необходимость есть бытие, {\em потому
что} оно есть; единство с самим собою бытия, имеющего себя
{\em основанием}. Но и обратно, так как оно имеет
некоторое основание, то оно не есть бытие, а есть безоговорочно только
{\em видимость, соотношение} или
{\em опосредствование}. Причинность есть этот
{\em положенный} переход первоначального бытия,
{\em причины}, в видимость или голую
{\em положенность}, и, наоборот, положенности в
первоначальность; но {\em самое тождество} бытия и
видимости есть пока еще {\em внутренняя} необходимость.
Эту {\em внутренность} или это в-себе-бытие движение
причинности снимает; тем самым утрачивается субстанциальность стоящих в
отношении сторон и раскрывается необходимость. Необходимость становится
{\em свободой} не в силу того, что она исчезает, а
только в силу того, что ее пока еще {\em внутреннее}
тождество {\em проявляется}, причем это проявление
представляет собою тождественное движение различенного внутрь себя самого,
рефлексию в себя видимости как видимости. — Вместе с тем, обратно, и
{\em случайность} благодаря этому становится
{\em свободой}, так как стороны необходимости, имеющие
образ таких действительностей, которые сами по себе свободны и не светятся
друг в друге, теперь {\em положены как тождество}, так
что эти тотальности рефлексии-в-себя теперь также и
{\em светятся} {\em как
тождественные} в своем различии или, иначе сказать, положены лишь как одна
и та же рефлексия.

Поэтому абсолютная субстанция, отличая себя от себя как абсолютная форма,
уже больше не отталкивает себя от себя как необходимость, равно как и не
распадается, как случайность, на безразличные, внешние друг другу
субстанции, а {\em диференцирует} себя,
{\em с одной стороны}, на такую тотальность (то, что
прежде было пассивной субстанцией), которая есть нечто первоначальное как
рефлексия из определенности внутрь себя, как простое целое, содержащее
внутри самого себя свою {\em положенность} и
{\em положенное} как
{\em тождественное в этой положенности с самим собой},
— на {\em всеобщее}, а, {\em с
другой стороны}, на тотальность (то, что прежде было причинной субстанцией)
как на рефлексию равным образом из определенности внутрь себя к
отрицательной определенности, которая таким образом, как
{\em тождественная} с собой
{\em определенность}, равным образом есть целое, но
положена как {\em тождественная с собой
отрицательность},— на {\em единичное}. Но
непосредственно ввиду того, что {\em всеобщее}
тождественно с собой лишь постольку, поскольку оно содержит в себе
{\em определенность} {\em как
снятую} и, следовательно, есть отрицательное как отрицательное,
непосредственно ввиду этого оно есть {\em та же самая
отрицательность}, что и {\em единичность}, а
единичность, так как она равным образом есть определенное определенное,
отрицательное как отрицательное, есть непосредственно
{\em то же самое тождество}, что и
{\em всеобщность}. Это их
{\em простое} тождество есть
{\em особенность}, которая содержит в себе от
единичного момент {\em определенности},
{\em а} от всеобщего момент
{\em рефлексии в себя}, содержит их в себе в
непосредственном единстве. Эти три тотальности суть поэтому одна и та же
рефлексия, которая, как {\em отрицательное соотношение
с собой}, диференцирует себя на те два определения, но как на
{\em вполне прозрачное различие}, а именно на
{\em определенную простоту} и на
{\em простую определенность}, которые суть их одно и то
же тождество. — Это есть {\em понятие}, царство
{\em субъективности} или
{\em свободы}.


\bigskip

