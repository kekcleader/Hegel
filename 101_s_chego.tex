Только в новейшее время зародилось сознание, что
нахождение {\em начала} в философии представляет собой
какие-то трудности, и основание этой трудности, равно как и возможность
решить эту трудную задачу, служили предметом многократного обсуждения.
Начало философии должно быть чем-то или {\em опосредствованным} или
{\em непосредственным}; и легко показать, что оно не
может быть ни тем, ни другим; стало быть, и тот и другой способ начинать
находит свое опровержение.

{\em Принцип} какого-нибудь философского учения тоже
означает некое начало, но не столько субъективное, сколько
{\em объективное} начало, начало
{\em всех вещей}. Принцип есть некое так или иначе
определенное {\em содержание}, "--- вода, единое, Нус,
идея, субстанция, монада\pagenote{Имеются в виду
философские учения Фалеса (вода), Парменида (единое), Анаксагора (нус),
Платона (идея), Спинозы (субстанция), Лейбница (монада).}
и~т.~д.; или, если он касается природы познания и, следовательно, по смыслу
данного философского учения представляет собою скорее только некий
критерий, чем некое объективное определение "--- мышление, созерцание, <<я>>,
сама субъективность, "--- то также и здесь интерес направлен на определение
содержания. Вопрос же о начинании как таковом оставляется, напротив, без
внимания и считается безразличным, как нечто субъективное в том смысле, что
дело идет о случайном способе изложения, стало быть, и потребность найти
то, с чего следует начинать, представляется незначительной по сравнению с
потребностью найти принцип, ибо, как кажется, единственно это интересно,
единственно в принципе заключается самая {\em суть};
нам интересно знать, что есть {\em истина}, что есть
{\em абсолютное основание} всего.

Но современное затруднение, причиняемое вопросом о начале, проистекает из
более широкой потребности, еще незнакомой тем, которые заботятся
догматически о том, чтобы доказать свой принцип, или скептически о том,
чтобы найти некий субъективный критерий для опровержения догматического
философствования, и совершенно отрицаемой теми, которые как бы выпаливают
из пистолета, прямо начиная с своего внутреннего откровения, с веры,
интеллектуального созерцания и~т.~д., и претендуют, что стоят выше
{\em метода} и логики. Если прежнее абстрактное
мышление сначала интересуется только принципом как
{\em содержанием}, в дальнейшем же процессе развития
вынуждается обратить внимание также и на другую сторону, на то, как
действует {\em познавание}, то теперешнее мышление
понимает также и {\em субъективное} делание как
существенный момент объективной истины и возникает потребность в соединении
метода с содержанием, {\em формы с принципом}. Таким
образом получается требование, чтобы {\em принцип} был
также началом и чтобы то, что представляет собою prius (первое) для
мышления, было также {\em первым} в {\em ходе движения} мышления.

Здесь мы должны только рассмотреть, каким является
{\em логическое} начало. Два возможных понимания его
характера мы уже назвали выше, а именно, его можно понимать как результат,
полученный опосредствованно, или как подлинное начало, как
непосредственное. Вопрос, являющийся столь важным для современного
образования, есть ли знание истины непосредственное, всецело зачинающее
знание, некая вера или же опосредствованное знание, "--- этот вопрос не должен
рассматриваться здесь. Поскольку можно давать обсуждение этого вопроса
{\em предварительно}, мы это сделали в другом месте (в
моей <<Энциклопедии философских наук>>, изд.~3-е <<Предварительное понятие>>,
§61 и~сл.). Здесь мы приведем оттуда лишь то замечание, что
{\em нет} ничего ни на небе, ни в природе, ни в духе,
ни где бы то ни было, что не содержало бы в себе столь же
непосредственность, сколь и опосредствование, так что эти два определения
оказываются {\em нераздельными} и
{\em неразделимыми}, и указанная противоположность
между ними являет себя чем-то ничтожным. Что же касается научного
рассмотрения, то в каждом логическом предложении мы встречаем определения
непосредственности и опосредствования и, следовательно, выяснение их
противоположности и их истины. Поскольку по отношению к мышлению, знанию,
познанию эта противоположность получает более конкретный вид
непосредственного или опосредствованного {\em знания},
постольку природа познания вообще рассматривается в рамках науки логики, а
рассмотрение познания в его дальнейшей конкретной форме есть дело науки о
духе и феноменологии последнего. Но желать еще {\em до}
науки уже получить полную ясность относительно познания равносильно
требованию, чтобы оно подверглось обсуждению {\em вне}
науки; но {\em вне} науки этого во всяком случае нельзя
сделать научно, а здесь мы стремимся единственно только к научности.

Начало есть {\em логическое} начало, поскольку мы его
должны сделать в стихии свободно для себя сущего мышления, в
{\em чистом знании}. {\em Опосредствовано} оно, стало быть, тем, что чистое
знание есть последняя, абсолютная истина
{\em сознания}. Мы заметили во введении, что
{\em феноменология духа} есть наука о сознании,
изображение того, что сознание имеет своим результатом
{\em понятие} науки, т.~е. чистое знание. Постольку
логика имеет своей предпосылкой науку о являющемся духе, содержащую в себе
и вскрывающую необходимость той точки зрения, которая представляет собой
чистое знание, равно как и вообще ее опосредствование, и тем самым дающую
доказательство ее истинности. В этой науке о являющемся духе исходят из
эмпирического, {\em чувственного} сознания, а последнее
есть настоящее {\em непосредственное} знание; там же
разъясняется, как обстоит дело с этим непосредственным знанием. Иное
сознание, как, например, вера в божественные истины, внутренний опыт,
знание посредством внутреннего откровения и~т.~д., оказывается после
небольшого размышления очень неподходящим для того, чтобы его приводить в
качестве представителя непосредственного знания. В указанном исследовании
непосредственное сознание является первым и непосредственным также и в
науке, и служит, стало быть, предпосылкой; в логике же предпосылкой служит
то, что оказалось результатом этого исследования, "--- идея как чистое знание.
{\em Логика есть чистая наука}, т.~е. чистое знание во
всем объеме его развития. Но эта идея определилась в вышеуказанном
результате как достоверность, ставшая истиной, достоверность, которая, с
одной стороны, уже больше не стоит наряду с предметом, а вобрала его внутрь
себя, знает его как то, что есть сама же она, и которая, с другой стороны,
отказалась от знания о себе, как о чем-то таком, что противостоит
предметному и есть лишь его уничтожение, отчуждена от этой субъективности и
есть единство со своим отчуждением.

Для того чтобы, исходя из этого определения чистого знания, начало
оставалось имманентным самой науке, не надо делать ничего другого, как
рассматривать или, правильнее, отстранив всякие размышления, всякие мнения,
которых придерживаются вне этой науки, воспринимать то,
{\em что имеется налицо}.

Чистое знание, как {\em слившееся в это единство}, сняло
всякое отношение к иному и к опосредствованию; оно есть то, что лишено
различий; это лишенное различий, следовательно, само перестает быть
знанием; имеется теперь только {\em простая
непосредственность}.

<<Простая непосредственность>> сама есть рефлективное выражение и имеет в виду
различие от опосредствованного. В своем истинном выражении указанная
простая непосредственность есть поэтому {\em чистое
бытие}. Подобно тому, как <<{\em чистое} знание>> не
должно означать ничего другого, кроме знания как такового, взятого
совершенно абстрактно, так и {\em чистое} бытие не
должно означать ничего другого, кроме {\em бытия}
вообще; {\em бытие} "--- и ничего больше, бытие без
всякого дальнейшего определения и наполнения.

Здесь выходит, что бытие служит началом, возникшим через опосредствование и
притом через такое опосредствование, которое есть вместе с тем снимание
самого себя; вместе с тем здесь имеется предпосылка о чистом знании, как
результате конечного знания, сознания. Но если не делать никаких
предпосылок, а {\em непосредственно} брать само начало,
то последнее будет определяться лишь тем, что оно есть начало логики,
мышления, взятого само по себе. Имеется только решение, которое можно
рассматривать также и как произвол, а именно решение рассматривать
{\em мышление как таковое}. Таким образом, начало должно
быть {\em абсолютным}, или, что здесь равнозначно,
абстрактным началом; оно, таким образом, {\em ничего не
должно предполагать}, ничем не должно быть опосредствовано, не должно также
иметь никакого основания; оно, наоборот, само должно быть основанием всей
науки. Оно поэтому должно быть всецело {\em неким}
непосредственным или, вернее, лишь самим
{\em непосредственным}. Как оно не может иметь какого
бы то ни было определения по отношению к иному, так оно не может иметь
никаких определений также и внутри себя, не может заключать в себе какого
бы то ни было содержания, ибо такого рода содержание было бы различением и
соотнесением разного друг с другом, было бы, следовательно, неким
опосредствованием. Началом, стало быть, оказывается {\em чистое бытие}.

Изложив то, что ближайшим образом касается только самого этого
наипростейшего, логического начала, мы можем еще прибавить следующие
дальнейшие соображения. Последние, однако, могут служить не столько
разъяснением и подтверждением данного выше изложения (которое само по себе
закончено), сколько скорее вызываются лишь представлениями и соображениями,
которые могут загородить нам дорогу еще до того, как приступим к делу, но
которые, однако, подобно всем другим предшествующим изучению науки
предрассудкам, должны находить свое разрешение в самой науке, и поэтому,
собственно говоря, здесь следовало бы, указывая на это, лишь призвать
читателя к терпению.

Усмотрение того обстоятельства, что абсолютно истинное представляет собою
результат и что, наоборот, всякий результат предполагает некое первое
истинное, которое, однако, именно потому, что оно есть первое,
рассматриваемое с объективной стороны, не необходимо, и, с субъективной
стороны, не познано, "--- усмотрение этого обстоятельства привело в новейшее
время к мысли, что философия должна начинать лишь с некоторого
{\em гипотетически и проблематически} истинного и что
философствование поэтому может быть сначала лишь исканием. Этот взгляд
{\em Рейнгольд} многократно выдвигал в позднейшие годы
своего философствования, и нельзя не отдать справедливости этому взгляду,
не признать, что в его основе лежит истинный интерес к вопросу о
спекулятивной природе философского {\em начала}. Разбор
этого взгляда является вместе с тем поводом к тому, чтобы дать
предварительное разъяснение смысла логического поступательного движения
вообще, ибо указанный взгляд с самого же начала принимает во внимание это
движение. И притом это последнее представляют себе так, что в философии
движение вперед есть скорее возвращение назад и обоснование, благодаря
которому только и получается уверенность, что то, с чего начали, не есть
только принятое произвольно, а в самом деле есть частью
{\em истинное}, частью {\em первое истинное}.

Нужно признаться, что здесь перед нами то существенное соображение "--- в более
определенном виде оно получится в рамках самой логики, "--- что движение
вперед есть {\em возвращение назад} в {\em основание}, к
{\em первоначальному и истинному}, от которого зависит
то, с чего начинают, и которым на деле это последнее порождается. "--- Так
например, сознание на своем пути от непосредственности, которой оно
начинает, приводится обратно к абсолютному знанию, как к своей
наивнутреннейшей {\em истине}. Это
{\em последнее}, основание, и есть также и то, из чего
происходит {\em первое}, выступившее сначала как
непосредственное. "--- Так, и в еще большей мере, абсолютный дух, получающийся
как конкретная и последняя высшая истина всякого бытия, познается, как
свободно отчуждающий себя в {\em конце} развития и
отпускающий себя, чтобы принять образ
{\em непосредственного} бытия, как решающийся сотворить
мир, содержащий в себе все то, что входило в развитие, предшествовавшее
этому результату, и что благодаря этому обратному положению превращается
вместе со своим началом в нечто, зависящее от результата, как от принципа.
Существенным для науки является не столько то, что началом служит нечто
непосредственное, а то, что все ее целое есть в самом себе круговорот, в
котором первое становится также и последним, а последнее также и первым.

Поэтому оказывается, с другой стороны, столь же необходимым рассматривать
как {\em результат} то, во что движение возвращается
обратно, как в свое {\em основание}. С этой точки
зрения первое есть также и основание, а последнее нечто выводное; так как
мы исходим из первого и путем правильных заключений приходим к последнему,
как к основанию, то это основание есть результат. Далее,
{\em поступательное движение} от того, что составляет
начало, должно быть рассматриваемо как дальнейшее его определение, так что
начало продолжает лежать в основании всего последующего и не исчезает из
него. Движение вперед состоит не в том, что выводится лишь некое
{\em иное} или совершается переход в некое истинно
иное, а, поскольку такой переход имеет место, он вместе с тем снова
снимает себя. Таким образом, начало философии есть наличная во всех
дальнейших развитиях и сохраняющаяся основа, есть то, что остается всецело
имманентным своим дальнейшим определениям.

Благодаря именно такому поступательному движению начало теряет то, что в нем
есть одностороннего вследствие этой определенности, вследствие того, что
оно есть некое непосредственное и абстрактное вообще; оно становится неким
опосредствованным, и линия научного поступательного движения тем самым
превращает себя в {\em круг}. Вместе с тем оказывается,
что то, с чего начинают, еще не познается поистине в начале, так как оно в
нем еще есть неразвитое, бессодержательное, и что лишь наука, и притом во
всем ее развитии, есть его завершенное, содержательное и теперь только
истинно обоснованное познание.

Но то обстоятельство, что только {\em результат}
оказывается абсолютным основанием, вовсе не означает, что поступательное
шествие этого познавания есть нечто предварительное или нечто
проблематическое и гипотетическое. Это поступательное шествие познания
должно определяться природой вещей и самого содержания. Указанное выше
начало не есть ни нечто произвольное и лишь временно предположенное, ни
нечто такое, что появляется произвольно и относительно чего просят читателя
принять его как предположение, но относительно чего все же оказывается
впоследствии, что поступили правильно, сделав его началом. Здесь дело не
обстоит так, как в тех построениях, которые приходится делать для
доказательства геометрической теоремы: относительно такого построения
оказывается лишь в конце, после того, как получилось доказательство, что мы
хорошо сделали, что провели именно эту линию и что затем начали в самом
доказательстве со сравнения между собою этих линий или углов: само же по
себе проведение этих линий или сравнивание их между собою нам этого не
показывает.

Вот почему в предшествующем мы указали {\em основание}
того обстоятельства, что в чистой науке начинают с чистого бытия,
непосредственно в самой же этой науке. Это чистое бытие есть то единство, в
которое возвращается чистое знание, или же, если будем считать, что мы
должны продолжать отличать само последнее как форму от его единства, то
чистое бытие есть содержание этого чистого знания. Это та сторона,
сообразно которой это {\em чистое бытие}, это абсолютно
непосредственное есть также и абсолютно опосредствованное. Но столь же
существенным образом оно должно быть взято только в своей односторонности,
как чисто непосредственное, должно быть взято таковым
{\em именно потому}, {\em что} оно
здесь берется как начало. Поскольку оно не было бы этой чистой
неопределенностью, постольку оно было бы определенным, и мы бы уже его
брали как опосредствованное, уже развитое далее; всякое определенное
содержит в себе некое {\em иное}, присоединяющееся к
некоему первому. Следовательно, это {\em природа самого
начала} требует, чтобы оно было бытием и ничем иным кроме этого. Оно
поэтому не нуждается для своего вступления в философию в каких бы то ни
было других подготовлениях, не нуждается в каких бы то ни было посторонних
размышлениях или исходных пунктах.

Из того, что начало есть начало философии, также, собственно говоря, нельзя
почерпать какого бы то ни было {\em более детального}
(nähere) его {\em определения} или, иначе говоря,
нельзя почерпать какого бы то ни было
{\em положительного} содержания для этого начала. Ибо
философия здесь, в самом начале, где еще нет самой вещи, есть пустое слово
или какое-нибудь принятое как предпосылка, неоправданное представление.
Чистое знание дает лишь то отрицательное определение, что начало должно
быть {\em абстрактным} началом. Поскольку мы берем
чистое бытие как {\em содержание} чистого знания,
последнее должно отойти от своего содержания, дать ему действовать
самостоятельно и не определять его далее. Или, иначе говоря, так как чистое
бытие должно быть рассматриваемо как единство, в которое сжалось знание на
своей высшей точке единения с объектом, то знание исчезло в это единство,
ничем не отличается от него и, следовательно, не оставило для него никакого
определения. Да и вне этого знания нет никакого нечто или содержания,
которым можно было бы воспользоваться, чтобы, начав с него, получить более
определенное начало.

Но мы могли бы опустить даже определение {\em бытия},
принятое нами доселе в качестве начала, так что оставалось бы лишь
требование совершить некоторое чистое начало. В таком случае не было бы
ничего другого, кроме самого {\em начала}, и нам
приходилось бы только посмотреть, что оно такое. Эту позицию мы могли бы
вместе с тем предложить в качестве полюбовной сделки тем, которые по каким
бы то ни было соображениям остаются недовольными тем, что логика начинает с
бытия, и, еще более того, недовольны результатом, к которому приводит это
бытие, а именно тем, что бытие переходит в ничто, отчасти же вообще не
хотят знать о каком-либо другом начале науки, кроме
{\em предположения} некоторого
{\em представления}, каковое представление затем
{\em анализируется}, так что результат такого анализа
служит первым определенным понятием в науке. Также и при этом способе
действия мы не получили бы никакого особенного предмета, потому что начало
в качестве начала {\em мышления} должно быть совершенно
абстрактным, совершенно всеобщим, должно быть всецело формой без всякого
содержания; у нас, следовательно, не было бы ничего другого, кроме
представления о голом начале как таковом. Нам, стало быть, следует только
посмотреть, что мы имеем в этом представлении.

Есть пока что ничто, и должно возникнуть нечто. Начало есть не чистое ничто,
а такое ничто, из которого должно произойти нечто; бытие, стало быть, уже
содержится также и в начале. Начало, следовательно, содержит в себе и то и
иное, бытие и ничто; оно есть единство бытия и ничто или, иначе говоря,
оно есть небытие, которое есть вместе с тем бытие, и бытие, которое есть
вместе с тем небытие.

Далее, бытие и ничто имеются в начале как
{\em различные}, ибо начало указывает на нечто иное;
оно есть некое небытие, соотнесенное с бытием как с неким иным;
начинающегося еще {\em нет}, оно лишь направляется к
бытию. Следовательно, начало содержит в себе бытие как некое такое, которое
отдаляется от небытия или, иначе говоря, упраздняет это последнее как
нечто, противоположное ему.

Но, далее, то, что начинается, уже {\em есть}, но в
такой же мере также его еще {\em нет}.
Противоположности, бытие и небытие, находятся, следовательно, в нем в
непосредственном соединении, или, иначе говоря, начало есть их
{\em неразличенное единство}.

Стало быть, анализ начала дал бы нам понятие единства бытия и небытия или,
выражая это в более рефлектированной форме, понятие единства различности и
неразличности; или, выражая это еще иначе, понятие тождества тождества и
нетождества\pagenote{Выражение
<<тождество тождества и нетождества>> очень характерно для Гегеля. Оно
встречается уже в ранней работе Гегеля <<Различие между философскими
системами Фихте и Шеллинга>> (1801~г.). См. Hegels Werke, hrsg. von Lasson,
Bd.~I, S.~77 или Werke, hrsg. von Glöckner, Bd.~I, S.~124. Это выражение
является как бы общей формулой гегелевской трактовки единства
противоположностей (согласно которой в {\em тождестве} противоположных
определений сохраняется, в снятом виде, также и {\em различие} между
ними), "--- в отличие от непосредственного, неподвижного тождества
противоположностей у Шеллинга, а также от <<совпадения противоположностей в
боге>> у Николая Кузанского, Гамана и других мистиков.}.
Это понятие можно было бы рассматривать как первую наичистейшую, т.~е.
наиабстрактнейшую дефиницию абсолютного, и оно в самом деле было бы
таковой, если бы дело шло вообще о форме дефиниций и о названии
абсолютного. В этом смысле указанное выше абстрактное понятие было бы
первой дефиницией этого абсолютного, а все дальнейшие определения и
развития лишь его более определенными и богатыми дефинициями. Но пусть
решат те, которые потому недовольны {\em бытием} как
началом, что оно переходит в ничто и что из этого возникает единство бытия
и ничто, будут ли они более довольны этим началом, начинающим с
представления о {\em начале}, и с анализа последнего,
который несомненно правилен, но также приводит к единству бытия и ничто, "---
пусть решат, будут ли они более довольны этим началом, чем тем, что началом
будет взято бытие.

Но мы должны сделать еще дальнейшее замечание об этом способе действия.
Вышеуказанный анализ предполагает известным представление начала; таким
образом мы поступили здесь по примеру других наук. Последние предполагают
существование своего предмета и принимают в виде уступки со стороны
приступающего к нему, что каждый имеет о нем то же самое представление и
может найти в нем приблизительно те же определения, которые они там и сям
получают и указывают посредством анализа, сравнения и прочих рассуждений о
нем. Но то, что представляет собою абсолютное начало, также должно быть
чем-то ранее знакомым; если оно есть конкретное и, следовательно,
многообразно определенное внутри себя, то это
{\em соотношение}, которое оно есть
{\em внутри себя}, предполагается чем-то знакомым; оно,
следовательно, выдано за нечто {\em непосредственное},
{\em но на самом деле оно не есть таковое}, ибо оно
есть лишь соотношение различных, содержит, следовательно, в себе
опосредствование. Далее, в конкретном появляются случайность и
произвольность анализа и разных процессов определения. Какие в конце концов
получатся определения, это зависит от того, что каждый
{\em преднаходит} в своем непосредственном случайном
представлении. Содержащееся в некоем конкретном, в некоем синтетическом
единстве соотношение представляет собою
{\em необходимое} соотношение лишь постольку, поскольку
оно не преднайдено, а порождено собственным движением моментов,
возвращающим их обратно в это единство, движением, представляющим собою
противоположность аналитическому способу действия, деланию, внешнему самой
вещи, совершающемуся лишь в субъекте.

Это влечет за собою также и тот более определенный вывод, что то, с чего
следует начинать, не может быть чем-то конкретным, чем-то таким, что
содержит некое соотношение {\em внутри самого себя}.
Ибо нечто такое предполагает, что внутри его самого имеется некоторое
опосредствование и переход от некоторого первого к некоторому иному,
результатом чего является ставшее простым конкретное. Но начало не должно
само уже быть некоторым первым и некоторым иным; то, что есть внутри себя
некоторое первое {\em и} некоторое иное, уже содержит
в себе совершившееся дальнейшее продвижение. То, с чего начинают, само
начало, мы должны понимать, как нечто не поддающееся анализу, должны брать
его в его простой, ненаполненной непосредственности, следовательно, как
{\em бытие}, как нечто совершенно пустое.

Если кто-либо, выведенный из терпения рассматриванием абстрактного начала,
скажет, что не нужно начинать с начала, а прямо с самой
{\em сути}, с самого предмета рассмотрения, то мы на
это ответим, что этот предмет есть не что иное, как указанное пустое бытие,
ибо, что такое предмет рассмотрения, это должно выясниться именно только в
ходе самой науки, не может предполагаться известным до нее.

Какую бы иную форму мы ни брали, чтобы получить иное начало, нежели
пустое бытие, оно (это иное начало) одинаково будет страдать указанным
недостатком. Тем, которые остаются недовольными этим началом, мы предлагаем
самим взяться за разрешение этой задачи: пусть попробуют начинать
как-нибудь иначе, чтобы при этом избежать этих недостатков.

Но нельзя совсем не упомянуть об оригинальном начале философии, приобревшем
большую известность в новейшее время, о начале с
<<я>>\pagenote{Имеется в виду философия И.~Г.~Фихте.}.
Оно получилось отчасти на основании того соображения, что из первого
истинного должно быть выведено все дальнейшее, и отчасти вследствие
потребности, чтобы {\em первое} истинное было чем-то
известным и, больше того, чем-то {\em непосредственно
достоверным}. Это начало не есть в общем случайное представление, такое
представление, которое у одного субъекта может носить один характер, а у
другого субъекта "--- другой. Ибо <<я>>, это непосредственное самосознание, само
представляется ближайшим образом отчасти чем-то непосредственным, отчасти
чем-то в гораздо более высоком смысле известным, чем какое-либо иное
представление. Все другое известное, хотя и принадлежит к <<я>>, есть,
однако, еще некое отличное от него и, следовательно, сразу же случайное
содержание; <<я>> же, напротив, есть простая достоверность самого себя. Но
<<я>> вообще есть {\em вместе с тем} также и некое
конкретное или, скорее, <<я>> есть наиконкретнейшее, есть сознание себя, как
бесконечно многообразного мира. Для того, чтобы <<я>> было началом и
основанием философии, требуется отделение этого конкретного, требуется тот
абсолютный акт, которым <<я>> очищается от самого себя и вступает в свое
сознание как абстрактное <<я>>. Но теперь оказывается, что это чистое <<я>> не
есть ни некое непосредственное, ни то знакомое, обычное <<я>> нашего
сознания, из которого, как нам говорили, непосредственно и для каждого
человека должна исходить наука. Указанный акт был бы, собственно говоря, не
чем иным, как возвышением на точку зрения чистого знания, на которой
исчезает различие между субъективным и объективным. Но ввиду требования,
чтобы это возвышение носило столь
{\em непосредственный} характер, оно есть лишь некий
субъективный постулат. Для того, чтобы этот постулат явил себя истинным
требованием, следовало бы раньше показать и изобразить поступательное
движение конкретного <<я>> в нем же самом, по его собственной необходимости,
движение от непосредственного сознания до чистого знания. Без этого
объективного движения чистое знание, также и в том случае, когда его
определяют {\em как интеллектуальное созерцание},
представляется произвольной точкой зрения или даже одним из эмпирических
{\em состояний} сознания, относительно которого важно
решить, не обстоит ли дело так, что один человек
{\em преднаходит} или может вызывать его в себе, а
другой "--- нет. Но поскольку это чистое <<я>> должно быть существенным чистым
знанием, чистое же знание не имеется в индивидуальном сознании
непосредственно, а полагается в нем только абсолютным актом самовозвышения,
то как раз теряется то преимущество, которое, как утверждают, возникает из
этого начала философии, а именно, это начало перестает быть чем-то всецело
всем знакомым, тем, что каждый непосредственно находит в себе и что он
может сделать исходным пунктом дальнейших размышлений; указанное чистое <<я>>
есть в его абстрактной сущности скорее нечто, незнакомое обычному сознанию,
нечто такое, чего оно не преднаходит в себе. Тем самым скорее появляется
невыгода иллюзии, получается, что речь идет якобы о чем-то знакомом, о <<я>>
эмпирического самосознания, между тем как на самом деле речь идет о чем-то
далеком этому сознанию. Определение чистого знания как <<я>> заставляет
непрерывно вспоминать о субъективном <<я>>, об ограниченностях которого мы
должны забыть, и сохраняет представление, будто положения и отношения,
которые получаются в дальнейшем развитии <<я>>, имеют место в обычном
сознании и могут быть преднайдены там, так как ведь оно и есть то,
относительно чего их высказывают. Это смешение порождает вместо
непосредственной ясности скорее лишь еще более кричащую путаницу и полную
дезориентировку, а уж в умах людей внестоящих оно вызвало грубейшие
недоразумения.

Что же касается, далее, вообще {\em субъективной}
определенности <<я>>, то это правда, что чистое знание освобождает <<я>> от
связанного с ним ограниченного смысла, согласно которому оно находит в
некотором объекте свою непреодолимую противоположность. Но как раз по этой
же причине было бы по меньшей мере {\em излишне}
сохранять еще эту субъективную позицию и определение чистой сущности как
<<я>>. Но следует прибавить, что это определение не только влечет за собою
вышеуказанную мешающую двусмысленность, но, как оказывается при ближайшем
рассмотрении, оно действительно остается субъективным <<я>>. Действительное
развитие той науки, которое исходит из <<я>>, показывает, что объект имеет и
сохраняет в ней определение чего-то постоянно остающегося неким
{\em иным} для <<я>>, что, следовательно, <<я>>, из
которого в ней исходят, не есть чистое знание, поистине преодолевшее
противоположность сознания, а еще находится в плену у явления.

При этом мы должны сделать еще то существенное замечание, что если бы
{\em в себе} <<я>> и могло быть определено как чистое
знание или интеллектуальное созерцание и можно было бы утверждать, что оно
есть начало, то ведь наука имеет дело не с тем, что имеется
{\em в себе} или {\em внутренно}, а
с существованием внутреннего {\em в мышлении} и с тем
{\em определенным характером} (Bestimmtheit), который
носит такое внутреннее в этом существовании. Но то, что в
{\em начале} науки имеется от интеллектуального
созерцания или "--- если предмет последнего получает название вечного,
божественного, абсолютного "--- то, что в начале науки имеется от вечного и
абсолютного, не может быть чем-либо иным, как первым, непосредственным,
простым определением. Какое бы мы ему ни дали более богатое название, чем
то содержание, которое мы выражаем голым <<бытием>>, во внимание все же может
быть принято такого рода абсолютное только таковым, каковым оно вступает в
{\em мыслительное} знание и в словесное выражение этого
знания. <<Интеллектуальное созерцание>> означает, правда, крутое отстранение
опосредствования и доказывающей, внешней рефлексии. Но то большее, чем
простая непосредственность, которое подразумевается под этим выражением,
есть нечто конкретное, нечто, содержащее в себе разные определения. Однако
высказывание и изображение такого конкретного есть, как мы уже указали
выше, некое опосредствующее движение, начинающее с
{\em одного} из определений и переходящее к иному
определению, хотя последнее также и возвращается к первому; это "--- движение,
которое вместе с тем не должно быть произвольным или ассерторическим.
Поэтому в таком изображении {\em начинают} не с самого
конкретного, а движение имеет своим исходным пунктом лишь простое
непосредственное. И кроме того, если делают началом конкретное, то
недостает доказательства, в котором нуждается соединение содержащихся в
конкретном определений.

Следовательно, если в выражении <<абсолютное>> или <<вечное>> или <<бог>>
(а~бесспорнейшее право имел бы {\em бог}, чтобы начинали
именно с него), если в созерцании их или мысли о них
{\em имеется больше} содержания, чем в чистом бытии, то
нужно, чтобы содержащееся в них {\em вступило} в знание
мыслительное, а не представляющее; как бы ни было богато заключающееся в
них содержание, все же определение, которое
{\em первым} вступает в область знания, есть некое
простое; ибо лишь в простом нет ничего более, кроме чистого начала; только
непосредственное просто, ибо лишь в непосредственном нет еще перехода от
одного к другому. Стало быть, что бы ни высказывали о бытии или что бы ни
содержалось в более богатых формах нашего представления об абсолютном или
боге, это все же в начале "--- лишь пустое слово и имеется лишь бытие. Это
простое, не имеющее никакого дальнейшего значения, это пустое есть, стало
быть, безусловно начало философии.

Это усмотрение само столь просто, что указанное начало как таковое не
нуждается ни в каком подготовлении или дальнейшем введении, и это наше
предварительное рассуждение о нем не могло иметь в виду ввести указанное
начало, а скорее ставило себе целью устранить всякую предварительность.

\section[Общее подразделение бытия]{Общее подразделение бытия}
Бытие, {\em во-первых}, определено вообще по отношению к иному.

Оно, {\em во-вторых}, определяет себя внутри самого себя.

{\em В-третьих}, когда отбрасывается это предварительное
подразделение, бытие есть та абстрактная неопределенность и
непосредственность, в которой оно должно служить началом.

Со стороны {\em первого} определения бытие
отмежевывается от {\em сущности}, поскольку оно в
дальнейшем своем развитии являет свою целокупность лишь как
{\em одну} сферу понятия и противопоставляет ей как
момент некоторую другую сферу.

Со стороны {\em второго} определения оно есть та сфера,
в которую входят определения и все движение его рефлексии. В ней бытие
положит себя в трех следующих определениях

I "--- как {\em определенность}, как таковую: {\em качество;}

II "--- как {\em снятую} определенность: {\em величина, количество;}

III "--- как {\em качественно} определенное {\em количество: мера.}

Это деление, как мы заметили во введении относительно всех этих делений
вообще, представляет собою только предварительное указание. Его
определениям предстоит возникнуть впервые лишь из движения самого бытия,
дать себе через это движение дефиницию и оправдать себя. Об отступлении
этого деления от обычного перечня категорий, а именно, как количества,
качества, отношения и модальности, которые, впрочем, должны были служить у
{\em Канта} только заголовками для его категорий, а на
самом деле сами суть категории, только более всеобщие, "--- об этом
отступлении здесь не стоит говорить, так как все изложение покажет, каковы
вообще наши отступления от обычного порядка и значения категорий.

Здесь можно сделать лишь то замечание, что обычно определение
{\em количества} излагают раньше определения
{\em качества}, и притом это делается, подобно большей
части того, что имеет место в обычной трактовке категорий, без всяких
дальнейших объяснений. Мы уже показали, что началом служит бытие как
{\em таковое} и, значит, качественное бытие. Из
сравнения качества с количеством легко увидеть, что качество есть по
природе вещей первое. Ибо количество есть ставшее уже отрицательным
качество; {\em величина} есть определенность, которая
уже больше не едина с бытием, а отлична от него, есть снятое, ставшее
безразличным качество. Она включает в себя изменчивость бытия, не изменяя
самой вещи, бытия, определением которого она служит; качественная же
определенность, напротив, едина со своим бытием, она не выходит за его
пределы и не находится также внутри его, а есть его непосредственная
ограниченность. Поэтому качество как
{\em непосредственная} определенность есть первое по
порядку, и с него следует начинать.

{\em Мера} есть {\em отношение}, но
не отношение вообще, а определенным образом отношение друг к другу качества
и количества; категории, которые Кант объединяет под названием <<отношение>>,
найдут себе место совсем в другом отделе.

Меру можно, если угодно, рассматривать также и как некоторую модальность. Но
так как последняя у {\em Канта} уже больше не
представляет собою определение содержания, а касается лишь отношения
последнего к мышлению, к субъективному, то это "--- совершенно гетерогенное,
сюда не принадлежащее отношение.

Третье определение {\em бытия} входит в состав отдела о
качестве, поскольку оно (бытие) как абстрактная непосредственность снижает
себя до некоторой единичной определенности, противостоящей внутри его сферы
иным его определенностям.

\bigskip
