\documentclass[b5paper, 11pt, twoside, onecolumn, openany]{memoir}

%%% PACKAGES
%%%------------------------------------------------------------------------

\usepackage[utf8]{inputenc}
\usepackage[T2A]{fontenc}
\usepackage[greek,english,main=russian]{babel}
\usepackage[final,letterspace=150]{microtype} % Less badboxes
\usepackage{bookmark}
\usepackage{footnotehyper}
\usepackage{multicol}
\usepackage{fancyhdr} % Required by poemscol
\usepackage{poemscol}
\usepackage{geometry}
\usepackage{pifont}
\usepackage{xcolor}

% \usepackage{kpfonts} %Font

\usepackage{amsmath,amssymb,mathtools} % Math

% \usepackage{tikz} % Figures
\usepackage{graphicx} % Include figures

%%% PAGE LAYOUT
%%%------------------------------------------------------------------------

\setlrmarginsandblock{0.11\paperwidth}{*}{1} % Left and right margin
\setulmarginsandblock{0.14\paperwidth}{*}{1} % Upper and lower margin
\checkandfixthelayout

\clubpenalty=10000
\widowpenalty=10000
\raggedbottom

%%% SECTIONAL DIVISIONS
%%%------------------------------------------------------------------------

\maxsecnumdepth{section} % Subsections (and higher) are numbered
\setsecnumdepth{section}

\makeatletter %
\makechapterstyle{standard}{
  \setlength{\beforechapskip}{0.8\baselineskip}
  \setlength{\midchapskip}{0\baselineskip}
  \setlength{\afterchapskip}{0.5\baselineskip}
  \renewcommand{\chapterheadstart}{\vspace*{\beforechapskip}}
  \renewcommand{\chapnamefont}{\centering\normalfont\Large}
  \renewcommand{\printchaptername}{\chapnamefont \@chapapp}
  \renewcommand{\chapternamenum}{\space}
  \renewcommand{\chapnumfont}{\normalfont\Large}
  \renewcommand{\printchapternum}{\chapnumfont \thechapter}
  \renewcommand{\afterchapternum}{\par\nobreak\vskip \midchapskip}
  \renewcommand{\printchapternonum}{\vspace*{\midchapskip}\vspace*{5mm}}
  \renewcommand{\chaptitlefont}{\centering\bfseries\normalsize}
  \renewcommand{\printchaptertitle}[1]{\chaptitlefont ##1}
  \renewcommand{\afterchaptertitle}{\par\nobreak\vskip \afterchapskip}

  % Part titles
  \renewcommand{\beforepartskip}{}
  \renewcommand{\afterpartskip}{\bigskip}

  \renewcommand{\clearforchapter}{}
}
\makeatother

\chapterstyle{standard}

\newcommand{\styledsubsubsection}[2][]{%
  \subsubsection[#1]{\normalsize\mdseries\em #2}}

\newcommand{\addtocspace}[1]{%
  \addtocontents{toc}{\vspace{#1}}}

\newcommand{\hegpart}[2][]{%
  \addtocontents{toc}{\vspace{4mm}%
    {\centering\bfseries\em #1\\\vspace{2mm}}}%
  \part[\small\MakeUppercase{#2}]%
       {\begin{Spacing}{1.6}\small\textsc{\lsstyle\mdseries%
        #1}\nopagebreak\\\normalsize\MakeUppercase{#2}\end{Spacing}}}

\newcommand{\hegchapter}[2][]{%
  \chapter[{\em #1.} #2]{%
    \fontsize{11}{18}\selectfont{\em\mdseries #1}\nopagebreak\\%
    \vspace{1mm}\\%
    \MakeUppercase{#2}%
    \nopagebreak%
    \vspace{2mm}\\}%
    \addtocspace{1mm}%
    \thispagestyle{plain}}

\newcommand{\hegsection}[2][]{%
  \section[#1]{\small\MakeUppercase{#2}}}

\newcommand{\hegremark}[3][]{%
  \subsubsection[#1. #2]{\centering {\lsstyle\mdseries #1}\nopagebreak\\%
  \vspace{1mm}\\%
  \footnotesize #3}}

\setsecheadstyle{\normalfont\bfseries\centering}
\setsubsecheadstyle{\normalfont\normalsize\bfseries\centering}
\setparaheadstyle{\normalfont\normalsize\bfseries}
\setparaindent{0pt}\setafterparaskip{0pt}

\usepackage{indentfirst}
\setlength\parindent{6mm}

%%% FLOATS AND CAPTIONS
%%%------------------------------------------------------------------------

\makeatletter % You do not need to write [htpb] all the time
\renewcommand\fps@figure{htbp}
\renewcommand\fps@table{htbp}
\makeatother

\captiondelim{\space } % A space between caption name and text
\captionnamefont{\small\bfseries} % Font of the caption name
\captiontitlefont{\small\normalfont} % Font of the caption text

\changecaptionwidth % Change the width of the caption
\captionwidth{1\textwidth}

%%% ABSTRACT
%%%------------------------------------------------------------------------

\renewcommand{\abstractnamefont}{\normalfont\small\bfseries} % Abstract title
\setlength{\absleftindent}{0.1\textwidth} % Width of abstract
\setlength{\absrightindent}{\absleftindent}

\tightlists

%%% HEADER AND FOOTER
%%%------------------------------------------------------------------------

\makepagestyle{standard} % Make standard pagestyle

\makeatletter % Define standard pagestyle
\makeevenfoot{standard}{}{}{}
\makeoddfoot{standard}{}{}{}
\makeevenhead{standard}{\bfseries\thepage\normalfont\qquad\small\leftmark}{}{}
\makeoddhead{standard}{}{}{\small\rightmark\qquad\bfseries\thepage}
% \makeheadrule{standard}{\textwidth}{\normalrulethickness}
\makeatother

\makeatletter
\makepsmarks{standard}{
\createmark{chapter}{both}{nonumber}{\@chapapp\ }{ \quad }
\createmark{chapter}{right}{nonumber}{}{ \quad }
\createmark{section}{right}{nonumber}{}{ \quad }
\createmark{subsection}{right}{nonumber}{}{ \quad }
\createplainmark{toc}{both}{\contentsname}
\createplainmark{lof}{both}{\listfigurename}
\createplainmark{lot}{both}{\listtablename}
\createplainmark{bib}{both}{\bibname}
\createplainmark{index}{both}{\indexname}
\createplainmark{glossary}{both}{\glossaryname}
}
\makeatother                               %

\makepagestyle{chap} % Make new chapter pagestyle

\makeatletter
\makeevenfoot{chap}{}{}{} % Define new chapter pagestyle
\makeoddfoot{chap}{}{}{}
\makeevenhead{chap}{}{}{}
\makeoddhead{chap}{}{}{}
% \makeheadrule{chap}{\textwidth}{\normalrulethickness}
\makeatother

\nouppercaseheads
\pagestyle{standard}   % Choosing pagestyle and chapter pagestyle
\aliaspagestyle{chapter}{chap}

%%% END NOTES
%%%------------------------------------------------------------------------
\makepagenote
\continuousnotenums
\notepageref

\let\oldpagenote\pagenote%
  \renewcommand{\pagenote}[1]{%
  {\fontsize{8}{8}\selectfont%
  \:\oldpagenote{#1}}}

\let\oldfootnote\footnote%
  \renewcommand{\footnote}[1]{%
  \,\oldfootnote{~#1}}

\renewcommand{\noteidinnotes}[2]{%
  \hspace{9mm}\textsuperscript{#1}\hspace{1.5mm}}
\renewcommand{\pageinnotes}[1]{%
  К~\hyperref[#1]{стр.~\pageref*{#1}.} ---}

\renewcommand*{\thefootnote}{\ding{93}}

\renewcommand*{\footnoterule}{%
  \kern-3pt%
  \hrule width 0.17\columnwidth
  \kern 2.6pt
  \vspace{1mm}}

\renewcommand*{\notedivision}{}

\renewcommand*{\pagenotesubhead}[3]{}

\renewcommand*{\pagenotesubheadstarred}[3]{}

%%% TEXT
%%%------------------------------------------------------------------------

\renewcommand{\baselinestretch}{0.82}
\setPagenoteSpacing{0.82}
\setFloatSpacing{0.82}

%%% NEW COMMANDS
%%%------------------------------------------------------------------------

\newcommand{\hm}[1]{#1\nobreak\discretionary{}{\hbox{\ensuremath{#1}}}{}}

\maxtocdepth{paragraph} % ToC depth
\settocdepth{paragraph}
\addto\captionsrussian{\renewcommand{\contentsname}{ОГЛАВЛЕНИЕ}}
\setlength{\cftbeforepartskip}{0.5mm}

\AtEndDocument{\addtocontents{toc}{\par}} % Add a \par to the end of the TOC

\usepackage{hyperref}   % Internal hyperlinks
\hypersetup{
  pdfborder={0 0 0},    % No borders around internal hyperlinks
  pdfauthor={ФРА}       % author
}
\usepackage{memhfixc}

\author{Г.~В.~Ф. Гегель}
\title{Наука логики}
\date{}

\renewcommand{\partnumberline}[1]{} % Remove part number in ToC
\renewcommand{\cftpartdotsep}{\cftdotsep} % Part dots in ToC
\renewcommand{\chapternumberline}[1]{} % Remove chapter number in ToC
\renewcommand{\cftchapterdotsep}{\cftdotsep} % Chapter dots in ToC

\renewcommand{\printpartname}{}
\renewcommand{\thepart}{}

\renewcommand{\printchaptername}{}
\renewcommand{\thechapter}{}

\makeatletter
\renewcommand{\@seccntformat}[1]{}
\makeatother

% Remove section numbers in ToC
\let\oldcftsf\cftsectionfont% save definition of \cftsectionfont
\let\oldcftspn\cftsectionafterpnum% and of \cftsectionafterpnum
\renewcommand*{\cftsectionfont}{%
\let\oldnl\numberline% save definition of \numberline
\renewcommand*{\numberline}[1]{}% change it
\oldcftsf} % use original \cftsectionfont
\renewcommand*{\cftsectionafterpnum}{%
\let\numberline\oldnl% % restore orginal \numberline
\oldcftspn} % use original \cftsectionafterpnum

\cftpagenumbersoff{book}
\renewcommand*{\cftbookfont}{\hfil}
\renewcommand*{\cftbookafterpnum}{\hfil}

\begin{document}

\frontmatter
\pagestyle{empty}

{\centering
  {\Large Г.~В.~Ф.~ГЕГЕЛЬ} \\
  \vspace{130pt}
  \textbf{\Huge \textcolor{red}{\fontsize{50pt}{60pt}\selectfont НАУКА ЛОГИКИ}} \\
  \vspace{30pt}
  {\huge Издание в~продолжениях. Выпуск 2} \\
  \vspace{60pt}
  {\Large Том~I. Объективная логика} \\
  \vspace{8pt}
  {\large Книга~I. Учение о~бытии} \\
  \vspace{8pt}
  {\large Глава~II. Наличное бытие} \\
  \vspace{210pt}
  {\tiny ЛАТВИЙСКОЕ ИЗДАТЕЛЬСТВО НАУЧНО-ПОЛИТИЧЕСКОЙ ЛИТЕРАТУРЫ} \\
  \vspace{10pt}
  {\small \textit{РИГА \ \ \ 2025}} \\
\par}

\mainmatter

\pagestyle{plain}

\hegpart[Первый отдел]{Определённость (качество)}
{\hfill \small \em (продолжение)}

\hegchapter[Вторая глава]{Наличное бытие}
Наличное бытие есть {\em определенное} бытие; его определенность есть
{\em сущая} определенность, {\em качество}. Своим качеством {\em нечто}
противостоит {\em иному}, оно {\em изменчиво} (ver\-änder\-lich)
[точнее "--- {\em способно стать иным}. "--- {\em Перев.}] и {\em конечно},
определено безоговорочно отрицательно не только в отношении иного, но и
в~сам\'{о}м себе. Это его отрицание по отношению прежде всего к конечному нечто
есть {\em бесконечное;} абстрактная противоположность, в~которой выступают эти
определения, разрешается в не~имеющую противоположности бесконечность,
в~{\em для-себя-бытие}.

Таким образом рассмотрение наличного бытия
распадается на следующие три раздела:

A) {\em Наличное бытие как таковое,}

B) {\em Нечто и иное, конечность,}

С) {\em Качественная бесконечность.}

\hegsection[А. Наличное бытие как таковое]{А. Наличное бытие как таковое}

В~наличном бытии

a) {\em как таковом} следует прежде всего различать его определенность

b) как {\em качество}. Последнее же следует брать и в одном, и в другом
определении наличного бытия: как {\em реальность} и как {\em отрицание}.
Но в~этих определённостях наличное бытие также и рефлектировано в~себя,
и положенное как таковое оно есть

c) {\em нечто}, налично сущее.

\subsection[a) Наличное бытие вообще]{a) Наличное бытие вообще}

Из становления происходит наличное бытие. Наличное бытие есть простая единость
бытия и ничто. Из-за этой простоты оно имеет форму некоего
{\em непосредственного}. Его опосредствование, становление, лежит позади него;
оно сняло себя, и наличное бытие представляется поэтому неким первым, из
которого исходят. Оно выступает прежде всего в~одностороннем определении
{\em бытия;} другое содержащееся в~нем определение, {\em ничто}, равным
образом проявится в~нем, проявится в~противоположность первому.

Оно есть не голое бытие, а~{\em наличное бытие;} взятое этимологически, Dasein
(наличное бытие) означает бытие в известном {\em месте;} но представление
о~пространстве здесь не приложимо. Наличное бытие есть вообще по своему
становлению {\em бытие} с некоторым {\em небытием}, так что это небытие принято
в простое единство с бытием. {\em Небытие}, принятое в бытие, таким образом,
что конкретное целое имеет форму бытия, непосредственности, составляет
{\em определенность} как таковую.

\label{bkm:bm25a}Это {\em целое} имеет равным образом форму, т.~е.
{\em определённость} бытия, ибо бытие равным образом явило себя в становлении
имеющим характер всего лишь момента, представляющим собой некое снятое,
отрицательно-определенное\pagenote{Эта фраза в издании 1833~г. (фототипически
воспроизведенном Глокнером в 1928~г.) дана с несколько необычными знаками
препинания, чт\'{о} дало повод Лассону изменить пунктуацию, прибавив тире перед
словами <<denn Sein hat\ldots>>. Тем самым глагол <<ist>> приобрел в главном
предложении значение связки, тогда как согласно пунктуации, даваемой в издании
1833~г., его следует понимать в смысле самостоятельного глагола (<<имеется
в~форме>> или, как переведено у Б.~Г.~Столпнера, <<имеет форму>>). Если принять
пунктуацию Лассона, то всю эту фразу надо перевести так: <<Это {\em целое}
равным образом в форме, т.~е. {\em определенности} бытия (ибо бытие равным
образом явило себя в становлении имеющим характер всего лишь момента) есть
нечто снятое, отрицательно-определенное>>. Сопоставление этого места с
серединой \pageref{bkm:bm25a}~стр. (<<Что целое, единство бытия и ничто, имеет
{\em одностороннюю определенность} бытия, "--- это является внешней
рефлексией>>) заставляет предпочесть интерпретацию Б.~Г.~Столпнера.}; но таково
оно {\em для нас, в нашей рефлексии;} оно еще не {\em положено} в себе самом.
Определенность же наличного бытия как таковая есть положенная определенность,
на чт\'{о} указывает также и выражение <<{\em наличное бытие}>>. "--- Следует
всегда строго различать между тем, чт\'{о} есть для нас, и тем, чт\'{о}
положено; лишь то, чт\'{о} {\em положено} в известном понятии, входит в
развертывающее рассмотрение его, в состав его содержания. Определенность же,
еще не положенная в нем самом "--- все равно, касается ли она природы самого
понятия или она есть внешнее сравнение, "--- принадлежит нашей рефлексии;
обращение внимания читателя на определенность последнего рода может лишь
служить к уяснению того пути, который представится нам в самом ходе развития
понятия, или же являться предварительным намеком на этот путь. Что целое,
единство бытия и ничто, имеет {\em одностороннюю определенность} бытия, "---
это является внешней рефлексией. В отрицании же, в нечто и {\em ином} и~т.~д.
эта односторонняя определенность дойдет до того, чтобы выступить как
{\em положенная}. "--- Мы должны были здесь обратить внимание на это различие;
но давать отчет обо всем, чт\'{о} рефлексия может позволить себе заметить,
не~следует, ибо это привело бы к пространности изложения, к предвосхищению
того, чт\'{о} должно получиться в самом предмете. Если такого рода рефлексии
могут служить к облегчению обозревания и тем самым и понимания, то они, однако,
также влекут за собой ту невыгоду, что они выглядят неоправданными
утверждениями, основаниями и основами последующего. Не следует поэтому
придавать им большее значение, чем то, которое они должны иметь, и надлежит
отличать их от того, чт\'{о} составляет момент в дальнейшем ходе развития
самого предмета.

Наличное бытие соответствует {\em бытию} предшествующей сферы; однако, бытие
есть неопределенное, в~нем вследствие этого не получается никаких определений.
Наличное же бытие есть некоторое определенное бытие, некоторое
{\em конкретное;} поэтому в нем сразу же открываются несколько определений,
различные отношения его моментов.

\subsection[b) Качество]{b) Качество}

Ввиду непосредственности, в которой бытие и ничто
{\em едины} в наличном бытии, они не выходят за пределы
друг друга; сколь далеко наличное бытие есть сущее, столь же далеко оно
есть небытие, определённо. Бытие не есть {\em всеобщее},
определенность не есть {\em особенное}. Определенность
еще {\em не отделилась от бытия;} она, правда, уже
больше и не будет отделяться от него, ибо лежащее отныне в основании
истинное есть единство небытия с бытием; на нем как на основании получаются
все дальнейшие определения. Но здесь то соотношение, в котором
определенность находится с бытием, есть непосредственное единство обоих,
так что еще не положено никакого различения между ними.

Определенность, так самодовлеюще (für sich) изолированная, как
{\em сущая} определенность, есть
{\em качество} "--- некое совершенно простое,
непосредственное. {\em Определенность} вообще есть
более всеобщее, которое в одинаковой мере может быть также и
количественным, равно как и определенным еще далее. Ввиду этой простоты
нечего более сказать о качестве как таковом.

Но наличное бытие, в котором содержатся как ничто, так и бытие, само
является масштабом для односторонности качества как лишь
{\em непосредственной} или {\em сущей} определенности. Качество должно быть
положено также и в определении ничто, благодаря чему непосредственная или
{\em сущая} определенность полагается как некая
различенная, рефлектированная определенность, и таким образом, ничто как
определенность некоторой определенности есть также некое рефлектированное,
некое {\em отрицание}. Качество, взятое с той стороны,
что оно, будучи различенным, признается {\em сущим},
есть {\em реальность;} оно же, обремененное некоторым
отрицанием, есть {\em отрицание} вообще; это "--- также
некоторое качество, но такое, которое признается недостатком и определится
в дальнейшем как граница, предел.

Оба суть наличное бытие; но в {\em реальности} как качестве с ударением на то,
что оно есть {\em сущее}, запрятано то обстоятельство, что оно содержит в~себе
определенность и, следовательно, также и отрицание; реальность признается
поэтому чем-то только положительным, из которого исключены отрицание,
ограниченность, недостаток. Отрицание, взятое как голый недостаток, было бы
то~же, что ничто; но оно есть некоторое наличное бытие, некоторое качество,
только с~ударением на небытие.

\hegremark[Примечание]{Реальность и отрицание}{[Реальность и отрицание]}

\label{bkm:bm73a}Реальность может показаться словом, имеющим разнообразные
значения, так как оно употребляется для выражения разных и даже противоположных
определений. В философском смысле говорят, например, об {\em исключительно
эмпирической} реальности, как о лишенном ценности существовании (Dasein). Но
когда говорят о мыслях, понятиях, теориях, что они {\em лишены реальности}, то
это означает, что они не обладают {\em действительностью}, хотя {\em в~себе}
или в понятии идея, например, платоновской республики может, дескать, быть
истинной. Здесь не отрицается за идеей ее ценность, и оставляют стоять
{\em наряду} с реальностью также и ее. Но сравнительно с так называемыми
{\em голыми} идеями, с~{\em голыми} понятиями, реальное признается единственно
истинным. "--- Смысл, в котором внешнему существованию приписывается решение
вопроса об истинности некоторого содержания, столь же односторонен, как
односторонни те, которые представляют себе, что для идеи, сущности или даже
внутреннего чувства безразлично внешнее существование (Dasein), и которые даже
считают, что они тем превосходнее, чем более они отдалены от реальности.

По поводу термин <<реальность>> мы должны коснуться прежнего метафизического
{\em понятия бога}, которое преимущественно клали в основание так называемого
онтологического доказательства бытия божия. Бог определялся как
{\em совокупность всех реальностей}, и об этой совокупности говорилось, что она
не заключает в себе противоречия, что ни одна из реальностей не упраздняет
другую; ибо реальность следует понимать лишь как некоторое совершенство, как
некое {\em утвердительное}, не содержащее в себе никакого отрицания.
Реальности, стало быть, не противоположны и не противоречат друг другу.

\label{bkm:bm73b}При таком понимании реальности предполагают, что она
остается еще и тогда, когда отмысливают всякое отрицание; однако этим снимается
всякая определенность реальности. Реальность есть качество, наличное бытие; тем
самым она содержит в себе момент отрицательного, и лишь благодаря этому она
есть то определенное, которое она есть. В~так называемом
{\em эминентном}\pagenote{Схоластический термин <<эминентно>> (emi\-nen\-ter)
обозначает наивысшую степень какого-нибудь качества, свойства или какой-нибудь
реальности.} {\em смысле} или как {\em бесконечная} "--- в обычном значении
этого слова "--- т.~е. в~том смысле, в котором ее якобы следует понимать, она
расширяется до неопределенности и теряет свое значение. Божественная благость,
утверждали, не есть благость в обычном смысле, а в~эминентном; она не отлична
от правосудия, а~{\em умеряется} ({\em лейбницевское} примиряющее выражение),
ею, равно как и, наоборот, правосудие умеряется благостью; таким образом,
благость уже перестает быть благостью и правосудие "--- правосудием. Могущество
бога, говорят, умеряется его мудростью, но, таким образом, оно уже
не~могущество как таковое, ибо оно подчинено мудрости; мудрость бога,
утверждают, расширяется до могущества, но, таким образом, она исчезает как
определяющая цель и меру мудрость. Истинное понятие бесконечного и его
{\em абсолютное} единство "--- то понятие, к~которому мы придем позднее, "---
нельзя понимать как {\em умерение, взаимное ограничение} или {\em смешение;}
это "--- поверхностное, остающееся неопределенно туманным соотношение, которым
может удовлетворяться лишь чуждое понятию представление. Реальность, как ее
берут в~вышеуказанной дефиниции бога, т.~е. реальность как определенное
качество, выведенное за пределы его определенности, перестает быть реальностью;
оно превращается в абстрактное бытие; бог как {\em чисто}~реальное во всем
реальном или как {\em совокупность} всех реальностей есть то же самое лишенное
определения и содержания, что и~пустое абсолютное, в~котором все есть одно.

Если же, напротив, брать реальность в ее определенности, то ввиду того, что
она по существу содержит в себе момент отрицательного, совокупность всех
реальностей оказывается также совокупностью всех отрицаний, совокупностью всех
противоречий; она, скажем примерно, превращается в абсолютное {\em могущество},
в котором все определенное поглощается; но так как само оно существует лишь
постольку, поскольку оно имеет рядом с собою нечто еще не упраздненное им, то,
когда его мыслят расширенным до осуществленного, беспредельного могущества, оно
превращается в абстрактное ничто. То реальное во всяком реальном, {\em бытие}
во всяком {\em наличном бытии}, которое якобы выражает понятие бога, есть
не~что иное, как абстрактное бытие, есть то же самое, чт\'{о} и ничто.

Определенность есть отрицание, положенное как утвердительное, "--- это и есть
положение Спинозы: Omnis deter\-minatio est negatio (всякое определение есть
отрицание)\pagenote{\label{omnisnote1}Выражение <<omnis
deter\-mina\-tio est nega\-tio>> у~Спинозы нигде не встречается. Встречается:
deter\-mina\-tio est nega\-tio~"--- в письме Яриху Иеллесу от 2~июня 1674~г.
(см. Спиноза, Переписка, М. 1932, стр.~173, письмо~50). Из всего контекста
письма, а также из сопоставлений с другими местами сочинений Спинозы
явствует, что слово <<deter\-minatio>> означает в данном случае не
<<определение>>, а <<ограничение>>. Гегель вообще в значительной мере
произвольно толковал философию Спинозы. Основное искажение, которое он,
вслед за Шеллингом, допускал в оценке философии Спинозы, состоит в том, что
он не желал видеть в нем материалиста.}. Это положение имеет
бесконечную важность; только следует сказать, что отрицание как таковое
есть бесформенная абстракция. Но не следует обвинять спекулятивную
философию в том, что для нее отрицание или ничто есть последнее слово; оно
является для нее столь же мало последним словом, как и реальность
"--- последней истиной.

Необходимым выводом из положения, гласящего, что определенность есть
отрицание, является {\em единство спинозовской
субстанции} или существование лишь одной субстанции.
{\em Мышление} и {\em бытие} или
протяжение, эти два определения, которые Спиноза именно имеет перед собою,
он должен был слить воедино (in Eins setzen) в этом единстве, ибо как
определенные реальности они суть отрицания, бесконечность которых есть их
единство; согласно дефиниции, даваемой Спинозой, о чем будет сказано далее,
бесконечность чего-либо есть его утверждение. Он поэтому их понимал как
атрибуты, т.~е. как такие, которые не обладают особым существованием,
бытием в себе и для себя, а имеют бытие лишь как снятые, как моменты; или,
правильнее сказать, они для него даже и не моменты, ибо субстанция
совершенно лишена определений в самой себе, а атрибуты, равно как и модусы,
суть различения, делаемые внешним рассудком. "--- Это положение точно так же
не допускает субстанциальности индивидуумов. Индивидуум есть соотношение с
собою благодаря тому, что он ставит границы всему другому; но эти границы
суть тем самым также и границы его самого, суть соотношения с другим; он не
имеет своего наличного бытия в самом себе. Индивидуум, правда, есть нечто
{\em большее}, чем только всесторонне ограниченное, но
это <<большое>> принадлежит другой сфере понятия; в метафизике бытия он есть
некое всецело определенное; и против того, чтобы нечто подобное, чтобы
конечное как таковое существовало в себе и для себя, выступает, предъявляя
свои права, определенность именно как отрицание и увлекает его в то же
отрицательное движение рассудка, которое заставляет все исчезать в
абстрактном единстве, в субстанции.

Отрицание непосредственно противостоит реальности; в дальнейшем, в сфере
собственно рефлектированных определений, оно противопоставляется
{\em положительному}, которое есть рефлектирующая на
отрицание реальность, "--- реальность, в которой как бы
{\em светится} то отрицательное, которое в реальности
как таковой еще запрятано.

Качество есть преимущественно лишь с~той стороны {\em свойство}, с~какой оно
в~некотором {\em внешнем соотношении} показывает себя {\em имманентным
определением}. Под свойствами, например, трав понимают определения, которые
не~только вообще {\em свойственны} некоторому нечто, а свойственны ему как раз
постольку, поскольку это нечто через них своеобразно {\em сохраняет} себя
в~отношении с~иным, не~дает внутри себя воли чужим положенным в~нем
воздействиям, а само {\em показывает} в ином {\em силу} своих собственных
определений, хотя и не~отстраняет от~себя этого иного. Напротив, более
спокойные определенности, как, например, фигуру, внешний вид, не~называют
свойствами, как, впрочем, и качествами, поскольку их представляют себе
изменчивыми, не~тождественными с~{\em бытием}.

{\em Quali\-erung} (качествование) или {\em Inquali\-erung} (вкачествование)
"--- специфическое выражение философии Якова {\em Бемё}, философии, проникающей
вглубь, но в смутную глубь, "--- означает движение некоторого качества
(кислого, терпкого, огненного качества и~т.~д.) в самом себе, поскольку оно в
своей отрицательной природе (в~своей {\em Qual}\pagenote{Яков Бёме полагал, что
немецкое слово Qual (м\'{у}ка) и латинское слово <<qua\-li\-tas>> (качество)
происходят от одного и того же корня. В~действительности корни у них разные.},
м\'{у}ке) выделяется из иного и укрепляется, поскольку оно вообще есть свое
собственное беспокойство в самом себе, в соответствии с которым оно порождает
и сохраняет себя лишь в борьбе.

\subsection[c) Нечто]{c) Нечто}

В наличном бытии мы различили его определенность, как качество; в последнем,
как налично сущем, {\em есть} различие, "--- различие реальности и отрицания.
Насколько эти различия имеются в наличном бытии, настолько же они вместе с~тем
ничтожны и сняты. Реальность сама содержит в~себе отрицание, есть наличное,
а~не неопределенное, абстрактное бытие. И~точно так же отрицание есть наличное
бытие; оно "--- не то ничто, которое должно было оставаться абстрактным, а оно
здесь положено, как оно есть в себе, как сущее, принадлежащее к наличному
бытию. Таким образом, качество вообще не отделено от наличного бытия, которое
есть лишь определенное, качественное бытие.

Это снятие различения есть больше, чем голый отказ от него и внешнее новое
отбрасывание его или простой возврат к простому началу, к наличному бытию как
таковому. Различие не может быть отброшено, ибо оно {\em есть}. Следовательно,
фактически имеющимся оказывается наличное бытие вообще, различие в нем,
и снятие этого различия; не наличное бытие, лишенное различий, как вначале,
а наличное бытие как {\em снова}~равное самому~себе
{\em благодаря снятию различия}, как простота наличного бытия,
{\em опосредствованная} этим снятием. Эта снятость различия есть
собственная определенность наличного бытия. Таким образом, оно есть
{\em внутри-себя-бытие;} наличное бытие есть {\em налично сущее, нечто}.

Нечто есть {\em первое отрицание отрицания} как простое
сущее соотношение с собою. Наличное бытие, жизнь, мышление и~т.~д.
существенно определяются в {\em налично сущее},
{\em живое}, {\em мыслящее} (в <<я>>)
и~т.~д. Это определение имеет величайшую важность: благодаря ему не
останавливаются, как на всеобщностях, на наличном бытии, жизни, мышлении
и~т.~д.; не останавливаются также и на божестве (вместо бога).
Представление справедливо считает {\em нечто} некоторым
{\em реальным}. Однако {\em нечто}
есть еще очень поверхностное определение, подобно тому, как
{\em реальность} и {\em отрицание},
наличное бытие и его определенность, хотя уже более не суть пустые бытие и
ничто, все же суть совершенно абстрактные определения. Вследствие этого они
и являются самыми ходячими выражениями, и философски необразованная
рефлексия чаще всего пользуется ими, вливает в них свои различения и мнит,
что в них она обладает чем-то вполне хорошо и твердо определенным. "---
Отрицание отрицания есть как {\em нечто} лишь начало
субъекта, "--- внутри-себя-бытие, пока что лишь совершенно неопределенное. Оно
определяет себя в дальнейшем прежде всего как сущее для себя, затем
продолжает определять себя и далее до тех пор, пока оно не получит впервые
в понятии конкретную напряженность субъекта. В основании всех этих
определений лежит отрицательное единство с собою. Но при этом следует
различать между отрицанием как {\em первым}, как
отрицанием {\em вообще}, и вторым, отрицанием
отрицания, которое есть конкретная, {\em абсолютная}
отрицательность, точно так же, как первое отрицание есть, напротив, лишь
{\em абстрактная} отрицательность.

{\em Нечто} есть {\em сущее} как отрицание отрицания; ибо последнее есть
восстановление простого соотношения с собою; но тем самым нечто есть точно так
же и {\em опосредствование себя с самим собою}. Уже в простоте [категории]
нечто, а затем еще определеннее в для-себя-бытии, субъекте и~т.~д. имеется
опосредствование себя с самим собою; оно имеется уже и в становлении, но в нем
оно есть лишь совершенно абстрактное опосредствование. В нечто опосредствование
с~{\em собою положено}, поскольку нечто определено как простое
{\em тождественное}. "--- Можно обратить внимание читателя на присутствие
опосредствования вообще в противовес утверждению о якобы голой
непосредственности знания, в которой опосредствования якобы совершенно нет; но
в дальнейшем нет нужды обращать особенное внимание читателя на момент
опосредствования, ибо он находится везде и повсюду, в каждом понятии.

Это опосредствование с собою, которым нечто является {\em в~себе}, взятое лишь
как отрицание отрицания, не имеет своими сторонами каких-либо конкретных
определений; таким образом, оно сжимается в простое единство, которое есть
{\em бытие}. Нечто {\em есть}, и оно ведь {\em есть} также и налично сущее;
оно, далее, есть {\em в~себе} также и {\em становление}, которое, однако, уже
более не имеет своими моментами только бытие и ничто. Один из них "--- бытие
"--- есть теперь наличное бытие и, далее, налично сущее; второй есть также
некое {\em налично сущее}, но определенное как отрицание нечто (Nega\-tives des
Etwas), "--- как {\em иное}. Нечто как становление есть переход, моменты
которого сами суть нечто и который поэтому есть {\em изменение}, "--- есть
ставшее уже {\em конкретным} становление. "--- Но нечто изменяется сначала лишь
в своем понятии; оно, таким образом, еще не {\em положено} как опосредствующее
и опосредствованное; пока что оно положено, как просто сохраняющее себя в своем
соотношении с собою, а отрицание его "--- как некоторое также качественное, как
только некоторое {\em иное} вообще.

\hegsection[B. Конечность]{B. Конечность}

a) Нечто {\em и} иное; они ближайшим образом равнодушны друг к другу; иное есть
тоже некоторое непосредственно налично сущее, некоторое нечто; отрицание, таким
образом, имеет место вне их обоих. Нечто есть {\em в~себе} в противоположность
к своему {\em бытию-для-иного}. Но определенность принадлежит также и к его
<<{\em в~себе}>> и~есть

b) его {\em определение}, переходящее также в {\em характер}, который, будучи
тождественным с первым, составляет имманентное и вместе с тем подвергшееся
отрицанию бытие-для-иного, составляет {\em границу} нечто, которая

c) есть имманентное определение самого нечто, и последнее есть,
следовательно, {\em конечное}.

В первом отделе, в котором мы рассматривали {\em наличное бытие} вообще,
последнее как взятое в начальной стадии рассмотрения, имело определение
{\em сущего}. Моменты его развития, качество и нечто, суть поэтому также
утвердительные определения. Напротив, в этом отделе развивается заключающееся
в наличном бытии отрицательное определение, которое там еще было только
отрицанием вообще, {\em первым} отрицанием, а теперь определилось далее до
{\em внутри-себя-бытия} нечто, до отрицания отрицания.

\subsection[a) Нечто и иное]{a) Нечто и иное}

1. Нечто и иное суть, {\em во-первых}, оба налично сущие или {\em нечто}.

{\em Во-вторых}, каждое из них есть также {\em иное}. Безразлично, которое из
них мы называем сначала и лишь потому именуем {\em нечто} (по-латыни, когда они
встречаются в предложении вместе, оба называются aliud, или <<один другого>>
"--- alius alium, а~когда идет речь об отношении взаимности, аналогичным
выражением служит alter alterum). Если мы некоторое наличное бытие называем
$A$, а~другое~$B$, то $B$ определено ближайшим образом как иное. Но $A$ есть
также и столь же иное этого~$B$. Оба суть одинаковым образом {\em иные}. Для
фиксирования различия и того нечто, которое следует брать как утвердительное,
служит слово <<{\em это}>>. Но <<{\em это}>> именно и выражает, что это
различение и выделение одного нечто есть субъективное обозначение, имеющее
место вне самого нечто. В этом внешнем показывании и заключается вся
определенность; даже выражение <<{\em это}>> не содержит в себе никакого
различия; каждое нечто есть столь же <<{\em это}>>, сколь и иное. Мы
{\em мним}, что словом <<{\em это}>> мы выражаем нечто совершенно определенное;
но мы при этом упускаем из виду, что язык как произведение рассудка выражает
лишь всеобщее; исключение составляет только {\em имя} некоторого единичного
предмета, но индивидуальное имя есть нечто бессмысленное в том смысле, что оно
не есть выражение всеобщего, и по этой же причине оно представляется чем-то
лишь положенным, произвольным, как и на самом деле собственные имена могут быть
произвольно приняты, даны или также изменены.

Таким образом, {\em инобытие} представляется определением, чуждым определенному
таким образом наличному бытию, или, иначе говоря, иное выступает {\em вне}
данного наличного бытия; это представляют себе так, что отчасти некоторое
наличное бытие определяется нами как иное только через {\em сравнение},
производимое некоторым третьим, отчасти же это наличное бытие определяется нами
как иное только из-за иного, находящегося вне его, но само по себе {\em оно}
не~таково. Вместе с тем, как мы уже заметили, каждое наличное бытие
определяется также и для представления в равной мере и как некоторое иное
наличное бытие, так что не остается ни одного наличного бытия, которое было бы
определено лишь как наличное бытие, не было бы вне некоторого наличного бытия,
и, следовательно, само не было бы некоторым иным.

Оба определены и как {\em нечто} и как {\em иное}, суть, значит,
{\em одно и то же}, и между ними еще нет никакого различия. Но эта
{\em однозначность} (Diesel\-bigkeit) определений также имеет место только во
внешней рефлексии, в~{\em сравнении} их друг с другом; но в том виде, в каком
пока что положено {\em иное}, оно само по себе, правда, находится в соотношении
с нечто, однако вместе с тем оно есть также и
{\em само по себе вне последнего}.

{\em В-третьих}, следует поэтому брать {\em иное} как изолированное,
в соотношении с самим собою, брать {\em абстрактно}
как иное (\textgreek{τὸ~ἕτερον}) Платона, который
противопоставляет его {\em единому} как один из
моментов тотальности и, таким образом, приписывает
{\em иному} собственно ему принадлежащую {\em природу}. Таким образом,
{\em иное}, понимаемое единственно как таковое, есть
не иное некоторого нечто, а иное в нем самом, т.~е. иное самого себя.
"--- {\em Физическая природа} есть такое по своему
определению иное; она есть {\em иное духа}. Это ее
определение есть, таким образом, пока что голая относительность, которой
выражается не качество самой природы, а лишь внешнее ей соотношение. Но так
как дух есть истинное нечто, а природа поэтому есть в себе же самой лишь
то, что она есть в отношении к духу, то ее качество постольку, поскольку
она берется сама по себе, именно и состоит в том, что она есть в самой себе
иное, {\em вне себя сущее} (в~определениях пространства, времени, материи).

Иное само по себе есть иное в самом себе и, следовательно, иное самого
себя есть, таким образом, иное иного, "--- стало быть, всецело неравное
внутри себя, отрицающее себя, {\em изменяющееся}. Но
оно вместе с тем также и остается тождественным с собою, ибо то, во что оно
изменилось, есть {\em иное}, которое помимо этого не
имеет никаких других дальнейших определений. А то, что изменяется,
определено быть иным не каким-нибудь другим образом, а тем же самым; оно
поэтому {\em сливается} в том ином, в которое оно
переходит, лишь {\em с самим собою}. Таким образом, оно
положено как рефлектированное в себя со снятием инобытия; оно есть
{\em тождественное} с собою нечто, по отношению к
которому, следовательно, инобытие, составляющее вместе с тем его момент,
есть некое отличное от него, не принадлежащее ему самому как такому нечто.

2. Нечто {\em сохраняется} в своем неимении наличного
бытия (Nicht\-dasein), оно существенно {\em едино} с ним
и существенно {\em не~едино} с ним. Оно, следовательно,
находится в {\em соотношении} со своим инобытием; оно не есть просто
свое инобытие. Инобытие в одно и то же время и содержится в~нем
и еще {\em отделено} от него. Оно есть {\em бытие-для-иного}.

Наличное бытие как таковое есть непосредственное, безотносительное; или,
иначе говоря, оно есть в определении {\em бытия}. Но
наличное бытие, как включающее в себя небытие, есть
{\em определенное}, подвергшееся внутри себя отрицанию
бытие, а затем, ближайшим образом "--- иное; но так как оно вместе с тем
также и сохраняется в своей подвергнутости отрицанию, то оно есть лишь
{\em бытие-для-иного}.

Оно сохраняется в своем неимении наличного бытия и есть бытие; но не бытие
вообще, а как соотношение с собою {\em в
противоположность} своему соотношению с иным, как равенство с собою в
противоположность своему неравенству. Таковое бытие есть
{\em в-себе-бытие}.

Бытие-для-иного и в-себе-бытие составляют {\em два
момента} нечто. Здесь мы имеем перед собою {\em две
пары} определений: 1) {\em нечто} и
{\em иное;} 2) {\em бытие-для-иного} и {\em в-себе-бытие}. В первых имеется
безотносительность их определенности: нечто и иное не связаны друг с
другом. Но их истиной служит соотношение между ними; бытие-для-иного и
в-себе-бытие суть поэтому указанные определения, положенные как
{\em моменты} одного и того же, как определения,
которые суть соотношения и остаются в своем единстве, в единстве наличного
бытия. Каждое из них, следовательно, само содержит в себе вместе с тем
также и свой разнствующий от него момент.

Бытие и ничто в том их единстве, которое есть наличное бытие, уже более не
суть бытие и ничто. Таковы они только вне своего единства. В их беспокойном
единстве, в становлении, они суть возникновение и прехождение. "--- Бытие в
нечто есть {\em в-себе-бытие}. Бытие, соотношение с
собою, равенство с собою, теперь уже более не непосредственно, а есть
соотношение с собою лишь как небытие инобытия (как рефлектированное в себя
наличное бытие). И точно так же небытие, как момент нечто в этом единстве
бытия и небытия, есть не неимение наличного бытия вообще, а иное, и,
говоря определеннее, по {\em различению} от него бытия
оно есть вместе с тем {\em соотношение} со своим
неимением наличного бытия, бытие-для-иного.

Тем самым {\em в-себе-бытие} есть, во-первых,
отрицательное соотношение с неимением наличного бытия, оно имеет инобытие
вне себя и противоположно ему; поскольку нечто есть
{\em в~себе}, оно изъято из инобытия и бытия-для-иного.
Но, во-вторых, оно имеет инобытие также и в самом себе, ибо оно
само {\em есть небытие} бытия-для-иного.

Но {\em бытие-для-иного} есть, во-первых, отрицание
простого соотношения бытия с собою, соотношения, которым ближайшим образом
должно быть наличное бытие и нечто; поскольку нечто есть в ином
или для иного, оно лишено собственного бытия. Но, во-вторых,
оно не есть неимение наличного бытия (Nicht\-dasein), как чистое ничто. Оно
есть неимение наличного бытия (Nicht\-dasein), указующее на в-себе-бытие, как
на свое рефлектированное в себя бытие, равно как и наоборот, в-себе-бытие
указует на бытие-для-иного.

3. Оба момента суть определения одного и того же, а именно определения
нечто. Нечто есть {\em в~себе}, поскольку оно ушло из
бытия-для-иного, возвратилось в себя. Но нечто имеет также некоторое
определение или обстоятельство {\em в~себе} (здесь
ударение падает на <<в>>) или {\em в~нем}, поскольку это
обстоятельство есть {\em в~нем} внешним образом, есть
бытие-для-иного.

Это ведет к некоторому дальнейшему определению.
{\em В-себе-бытие} и бытие-для-иного ближайшим
образом разны, но то обстоятельство, что нечто имеет
{\em то же самое}, {\em что оно
есть в себе}, также и {\em в~нем}, и что, наоборот, то,
что оно есть как бытие-для-иного, оно есть также и в себе "--- в этом
состоит тождество в-себе-бытия и бытия-для-иного, согласно тому
определению, что само нечто есть тождество (ein und dasselbe) обоих
моментов, и что они, следовательно, в нем нераздельны. "--- Формально это
тождество получается уже в сфере наличного бытия, но более определенное
выражение оно получит при рассмотрении сущности и затем "--- отношения
{\em внутреннего} (der Inner\-lich\-keit) и
{\em внешнего} (Äusser\-lich\-keit), а определеннее всего
оно выявится при рассмотрении идеи как единства понятия и действительности.
"--- Обыкновенно мнят, что словами <<{\em в~себе}>> мы
высказываем нечто высокое, точно так же, как словом
<<{\em внутреннее}>>; но на самом деле то, что нечто есть
{\em только в себе}, есть также {\em только в нем;} <<в себе>> есть только
абстрактное и, следовательно, внешнее определение. Выражения: <<в~нем
ничего нет>>, <<в этом что-то есть>>, содержат в себе, хотя и смутно, тот
смысл, что то, чт\'{о} {\em в человеке} есть (an einem), принадлежит также
и к его {\em в-себе-бытию}, к его внутренней истинной ценности.

Можно указать, что здесь уясняется смысл
{\em вещи-в-себе}, которая есть очень простая
абстракция, но в продолжение долгого времени слыла очень важным
определением, как бы чем-то аристократическим, точно так же, как положение,
гласящее, что мы не знаем, каковы вещи в себе, признавалось
многозначительной мудростью. "--- Вещи называются вещами-в-себе, поскольку мы
абстрагируемся от всякого бытия-для-иного, поскольку мы их мыслим без
всякого определения, как представляющие собою ничто. В этом смысле нельзя,
разумеется, знать, {\em что такое}
вещь-{\em в-себе}. Ибо вопрос,
{\em что такое?} требует, чтобы были указаны
{\em определения;} но так как те вещи, относительно
которых требуется, чтобы были указаны определения, должны быть вместе с тем
{\em вещами-в-себе}, т.~е. как раз не обладать никакими
определениями, то в вопрос бессмысленным образом вложена невозможность
ответить на него или же (если все-таки пытаются ответить) на него дают только
нелепый ответ. "--- Вещь в себе есть то же самое, чт\'{о} то абсолютное,
о котором знают только то, что все в нем едино. Мы поэтому знаем очень
хорошо, чт\'{о} представляют из себя эти вещи-в-себе; они как таковые суть
не~что иное, как не имеющие истинности, пустые абстракции. Но что такое
поистине вещь в себе, чт\'{о} поистине есть в себе, "--- изложением этого
является логика, причем, однако, под <<{\em в-себе}>> понимается нечто
лучшее, чем абстракция, а именно, то, чт\'{о} нечто есть в своем понятии;
но последнее конкретно внутри себя, как понятие вообще постижимо и, как
определенное и связь своих определений, внутри себя познаваемо.

В-себе-бытие имеет прежде всего своим противостоящим моментом
бытие-для-иного; но в-себе-бытию противопоставляется также и
{\em положенность}. Это выражение, правда,
подразумевает также и бытие-для-иного, но оно определительно разумеет уже
происшедший поворот назад того, чт\'{о} не есть в себе, в то, чт\'{о} есть его
в-себе-бытие, в чем оно {\em положительно}.
{\em В-себе-бытие} должно быть обычно понимаемо как
абстрактный способ выражения понятия; {\em полагание},
собственно говоря, относится уже к сфере сущности, объективной рефлексии;
основание {\em полагает} то, чт\'{о} им обосновывается;
причина, больше того, {\em производит} некоторое
действие, некоторое наличное бытие, самостоятельность которого
{\em непосредственно} отрицается и смысл которого
заключается в том, что оно имеет свою {\em суть}
(Sache), свое бытие в ином. В сфере бытия наличное бытие лишь
{\em происходит} из становления, иначе говоря, вместе с нечто
положено иное, вместе с конечным "--- бесконечное, но конечное не
производит бесконечного, не {\em полагает} его. В сфере
бытия {\em самоопределение} понятия само есть лишь
{\em в~себе} "--- и соответственно с этим оно называется
переходом. Рефлектирующие определения бытия, как, например, нечто и иное
или конечное и бесконечное, хотя они по существу указывают друг на друга,
или суть как бытие-для-иного, также считаются как
{\em качественные} существующими особо;
{\em иное есть}, конечное считается так же
{\em непосредственно сущим} и прочно стоящим особо, как
и бесконечное; их смысл представляется завершенным также и без их иного.
Напротив, положительное и отрицательное, причина и действие, хотя они также
берутся как сущие изолированно, все же не имеют вместе с тем никакого
смысла друг без друга; {\em в них самих} имеется
отблеск своего иного, каждое из них как бы светится в своем ином. "--- В
разных кругах определения и в особенности в развитии
изложения, или, точнее, в поступательном движении понятия по направлению к
своему изложению, главное заключается в том, чтобы всегда вполне различать
между тем, чт\'{о} еще есть {\em в~себе}, и тем, чт\'{о}
{\em положено}, каковы определения, как они суть в
понятии, и каковы они, как положенные или сущие-для-иного. Это
"--- различение, которое принадлежит только диалектическому развитию,
различение, которого не знает метафизическое философствование "--- к
последнему принадлежит также и критическая философия; дефиниции метафизики,
равно как и ее предпосылки, различения и следствия, имеют целью делать
утверждения и выводы лишь относительно лишь {\em сущего} и
притом {\em в-себе-сущего}.

В единстве нечто с собою {\em бытие-для-иного}
тождественно со своим <<{\em в~себе}>>; бытие-для-иного
есть, {\em таким образом}, в [самом] нечто.
Рефлектированная таким образом в себя определенность тем самым есть снова
{\em простое сущее}, есть, следовательно, снова
качество, "--- {\em определение}.

\subsection[b) Определение, характер и граница]{b) Определение, характер и граница}

<<{\em В~себе}>>, в которое нечто рефлектировано внутри
себя из своего бытия-для-иного, уже более не есть абстрактное <<в себе>>, а
как отрицание его бытия-для-иного, опосредствовано последним, которое
таким образом составляет его момент. Оно есть не только непосредственное
тождество нечто с собою, а то тождество, через которое нечто есть также и
{\em в~нем} то, чт\'{о} оно есть {\em в~себе;}
бытие-для-иного есть {\em в~нем}, потому что
<<{\em в~себе}>> есть его снятие, есть выхождение
{\em из него} в себя; но оно есть в нем также уже и
потому, что оно абстрактно, следовательно, существенно обременено
отрицанием, бытием-для-иного. Здесь имеется не только качество и
реальность, сущая определенность, но и
{\em в-себе-сущая} определенность, и ее развертывание
состоит в том, чтобы {\em положить} ее как эту
рефлектированную в себя определенность.

1. Качество, которое есть <<в~себе>>
в простом нечто и сущностно находится
в единстве с другим моментом этого нечто, с
{\em в-нем-бытием}, можно назвать его
{\em определением}, поскольку различают это слово в
точном его значении от {\em определенности} вообще.
Определение есть утвердительная определенность как в-себе-бытие, которому
нечто в своем наличном бытии, противодействуя своей переплетенности с
иным, которым оно быть бы определено, остается адекватным,
удерживаясь в своем равенстве с собою и проявляя это последнее в своем
бытии-для-иного. Нечто {\em осуществляет} свое
определение (назначение)\footnote{Немецкое слово Bestimmung
означает как определение, так и назначение. "--- {\em Перев.}},
поскольку дальнейшая определенность,
многообразно вырастающая ближайшим образом на почве его отношения к
иному, становится его полнотой (Fülle) в соответствии с его
в-себе-бытием. Определение подразумевает, что то, чт\'{о} нечто есть
{\em в~себе}, есть также и {\em в~нем}.

{\em Определением человека} служит мыслящий разум;
мышление вообще есть его простая {\em определенность},
ею он отличается от животного; он есть мышление {\em в~себе},
поскольку оно отличается также и от его бытия-для-иного, от его
собственной природности и чувственности, которыми он непосредственно связан
с иным. Но мышление есть также и {\em в~нем:} сам
человек есть мышление, он {\em налично сущ} как
мыслящий, оно есть его существование и действительность; и далее: так как
мышление есть в его наличном бытии и его наличное бытие есть в мышлении, то
оно {\em конкретно}, его следует брать с содержанием и
наполнением, оно есть мыслящий разум и таким образом оно есть
{\em определение} человека. Но даже это определение
есть опять-таки лишь {\em в~себе} как некоторое
{\em долженствование}, т.~е. оно вместе с включённым
в его <<в-себе>> наполнением дано в форме <<в~себе>>
вообще, в {\em противоположность} не включённому
в него наличному бытию, которое вместе с тем еще
есть внешне противостоящая ему чувственность и природа.

2. Наполнение в-себе-бытия определенностью также отлично от той
определенности, которая есть лишь бытие-для-иного и остается вне
определения. Ибо в области [категорий] качества различия сохраняют даже в
их снятости непосредственное качественное бытие в отношении друг друга. То,
чт\'{о} нечто имеет {\em в~нем}, таким образом разделяется,
и оно есть с этой стороны внешнее наличное бытие нечто, каковое наличное
бытие также есть {\em его} наличное бытие, но не
принадлежит его в-себе-бытию. Определенность, таким образом, есть
{\em характер}.

Нося тот или иной характер, нечто подвергается воздействию внешних влияний
и обстоятельств. Это внешнее соотношение, от которого зависит характер, и
определяемость иным представляются чем-то случайным. Но
качество какого-нибудь нечто в том-то и состоит, чтобы быть предоставленным
этой внешности и обладать некоторым характером.

Поскольку нечто изменяется, изменение имеет место в характере; последний
есть {\em в} нечто то, чт\'{о} становится иным.
Само нечто сохраняет себя в изменении, которое затрагивает только эту
непостоянную поверхность его инобытия, а не его определение.

Определение и характер таким образом отличны друг от друга; со стороны
своего определения нечто безразлично к своему характеру. Но то, чт\'{о} нечто
имеет {\em в~нем}, есть связующий их средний термин
этого силлогизма. Но {\em бытие-в-нечто}
(Am-Еtwas-Sein) оказалось, наоборот, распадающимся на указанные два крайних
термина. Простой средний термин есть
{\em определенность} как таковая; к ее тождеству
принадлежит как определение, так и характер. Но определение переходит само
по себе в характер и характер сам по себе "--- в определение. Это вытекает из
предыдущего; связь мыслей, говоря более точно, такова: поскольку то, чт\'{о}
нечто {\em есть в себе}, есть также и
{\em в~нем}, оно обременено бытием-для-иного;
определение как таковое открыто, следовательно, отношению к иному.
Определенность есть в то же время момент, но вместе с тем содержит
качественное различие, состоящее в том, что оно разнится от в-себе-бытия,
есть отрицание нечто, некоторое иное наличное бытие. Определенность,
включающая в себя таким образом иное, соединенная с в-себе-бытием, вводит
инобытие во в-себе-бытие или, иначе говоря, в определение, которое, таким
образом, понижается до характера. "--- Наоборот, бытие-для-иного,
изолированное и положенное само по себе в форме характера, есть в нем (в
нечто) то же, что иное как таковое, иное в нем (в ином) самом, т.~е.
иное самого себя; но, таким образом, оно есть
{\em соотносящееся с собою} наличное бытие, есть, таким
образом, в-себе-бытие с некоторой определенностью, стало быть,
{\em определение}. "--- Следовательно, поскольку оба
должны быть вместе с тем удержаны друг вне друга, характер, являющийся
обоснованным в некотором внешнем, в ином вообще,
{\em зависит} также и от определения, и идущий от
чужого процесс определения определен вместе с тем собственной имманентной
определенностью данного нечто. Но, далее, характер принадлежит к тому, чт\'{о}
нечто есть в себе; вместе со своим характером изменяется и нечто.

Это изменение нечто уже более не есть первое изменение нечто, изменение
исключительно по своему бытию-для-иного; то первое изменение было только
в себе сущим, принадлежащим внутреннему понятию; теперь же изменение есть
также и положенное в нечто. "--- Само нечто определено далее, и отрицание
положено как имманентное ему, как его развитое
{\em внутри-себя-бытие}.

Переход определения и характера друг в друга есть ближайшим образом снятие
их различия; тем самым положено наличное бытие или нечто вообще, а так как
оно есть результат указанного различия, обнимающего собою также и
качественное инобытие, то имеются два нечто, но не только вообще иные по
отношению друг к другу, так что это отрицание оказалось бы в таком случае
еще абстрактным и находило бы место лишь в нашем сравнивании их между
собою, а это отрицание теперь имеется как
{\em имманентное} этим ничто. Они как
{\em налично сущие} безразличны друг к другу. Но теперь
это их утверждение уже более не есть непосредственное, каждое из них
соотносится с самим собою через {\em посредство} снятия
того инобытия, которое в определении рефлектировано во в-себе-бытие.

Таким образом, нечто относится к иному {\em из самого
себя} [спонтанно], ибо инобытие положено в нем как его собственный
момент; его внутри-себя-бытие объемлет собою отрицание, через посредство
которого оно теперь вообще обладает своим утвердительным наличным бытием.
Но это иное также и качественно отлично от последнего и, следовательно,
положено вне нечто. Отрицание своего иного есть лишь качество данного
нечто, ибо оно есть нечто именно как это снятие своего иного. Итак,
собственно говоря, только теперь иное настоящим образом само
противополагает себя некоторому наличному бытию; первому нечто иное
противополагается лишь внешним образом или, иначе говоря, так как они на
самом деле находятся во взаимной связи безоговорочно, т.~е. по своему
понятию, то эта связь заключается в том, что наличное бытие
{\em перешло} в инобытие, нечто
{\em перешло} в иное, заключается в том, что нечто,
как и иное, есть иное. Поскольку же внутри-себя-бытие есть небытие
инобытия, которое в нем содержится, но вместе с тем, как сущее, отлично от
него, постольку само нечто есть отрицание,
{\em прекращение в нем иного;} оно
положено, как относящееся к нему отрицательно и тем самым сохраняющее себя;
"--- это иное, внутри-себя-бытие данного нечто, как отрицание отрицания,
есть его {\em в-себе-бытие}, и, вместе с тем, это
снятие есть {\em в~нем} как простое отрицание, а
именно, как отрицание им внешнего ему другого нечто. Одна и та же единая их
определенность, с одной стороны, тождественна с внутри-себя-бытием этих
нечто как отрицание отрицания, а, с другой стороны, вместе с тем, поскольку
эти отрицания противостоят одно другому как иные нечто, она,
исходя из них же самих смыкает их и также отделяет их друг от друга,
так как каждое из них отрицает иное; это "--- {\em граница}.

3. {\em Бытие-для-иного} есть неопределенная,
утвердительная общность нечто со своим иным; в границе же выдвигается
{\em небытие}-для-иного, качественное отрицание
иного, которое (иное) благодаря этому не подпускается к
рефлектированному в себя нечто. Мы должны присмотреться к развертыванию
этого понятия, каковое развертывание впрочем скорее оказывается
запутанностью и противоречием. Последнее сразу же сказывается в том, что
граница, как рефлектированное в себя отрицание данного нечто, содержит в
себе {\em идеализованно} моменты нечто и иного, и они
же как различенные моменты вместе с тем положены в сфере наличного бытия
как {\em реальные},
{\em качественно различные}.

$\alpha$) Нечто, следовательно, есть непосредственное соотносящееся
с~собою наличное бытие и имеет границу ближайшим образом как границу
в~отношении иного; она есть небытие иного, а не самого нечто; последнее
ограничивает в ней свое иное. "--- Но иное само есть некоторое нечто вообще;
стало быть, граница, которую нечто имеет в отношении к иному, есть граница
также и иного как нечто, граница этого нечто, которой оно не подпускает к себе
первое нечто, как {\em свое} иное, или, иначе говоря, она есть
{\em небытие этого первого нечто;} таким образом, она не есть только небытие
иного, а есть небытие как одного, так и иного нечто и, значит, небытие
[всякого] {\em нечто} вообще.

Но она есть существенно также и небытие иного; таким образом, нечто вместе
с~тем {\em есть} благодаря своей границе. Будучи ограничивающим, нечто, правда,
понижается до того, что само оно оказывается ограничиваемым, "--- однако его
граница как прекращение другого в нём, вместе с тем сама есть лишь бытие этого
нечто; {\em последнее есть благодаря ей то, чт\'{о} оно есть, имеет в ней свое
качество}. "--- Это отношение есть внешнее проявление того обстоятельства, что
граница есть простое или {\em первое} отрицание, иное же есть вместе с тем
отрицание отрицания, внутри-себя-бытие данного нечто.

Нечто как непосредственное наличное бытие есть, следовательно, граница
в отношении другого нечто, но оно имеет ее {\em в~себе самом} и есть нечто
через ее опосредствование, которое есть также и его небытие. Она есть то
опосредствование, через которое нечто и иное
{\em столь же суть, сколь и не~суть}.

$\beta$) Поскольку нечто {\em и~есть} и {\em не~есть} в своей
границе и эти моменты суть некоторое непосредственное, качественное различие,
постольку неимение наличного бытия (Nicht\-dasein) нашим нечто и его наличное
бытие оказываются друг вне друга. Нечто имеет свое наличное бытие {\em вне}
(или, как это себе также представляют, {\em внутри}) своей границы; и точно так
же и иное есть вне ее, так как оно есть нечто. Она есть {\em середина между}
ними, в которой они прекращаются. Они имеют свое {\em наличное бытие по ту
сторону} друг друга и {\em их границы;} граница как небытие каждого из них есть
иное в отношении обоих.

В~силу такого различия между нечто и его границей {\em линия} представляется
линией лишь вне своей границы, точки; {\em плоскость} представляется плоскостью
вне линии; {\em тело} представляется телом лишь вне ограничивающей его
плоскости. "--- Это есть тот аспект, в котором граница прежде всего
воспринимается представлением, этим вне-себя-бытием понятия и в этом же аспекте
она берется преимущественно в пространственных предметах.

$\gamma$) Но, далее, нечто, как оно есть вне границы, есть
неограниченное нечто, лишь наличное бытие вообще. Таким образом, оно не
отлично от своего иного; оно есть лишь наличное бытие, имеет,
следовательно, одно и то же определение со своим иным; каждое из них есть
лишь нечто вообще или, иначе говоря, каждое есть иное; оба суть, таким
образом, {\em одно и то же}. Но это их сначала лишь
непосредственное наличное бытие теперь положено с определенностью, как
границей, в которой оба суть то, что они суть, в различенности друг от
друга. Но она точно так же, как и наличное бытие, есть
{\em общее} им обоим различие, их единство и различие.
Это двоякое тождество обоих "--- наличное бытие и граница "--- подразумевает, что
нечто имеет свое наличное бытие только в границе и что, так как и граница и
непосредственное наличное бытие вместе с тем суть отрицания друг друга, то
нечто, которое есть только в своей границе, в такой же мере отделяет себя
от самого себя, указует дальше себя, на свое небытие, и высказывает
последнее как свое бытие, переходя, таким образом, в последнее. Чтобы
применить это к предыдущему примеру, следует сказать, что
{\em одно} определение нашего нечто состоит в том, что
нечто есть то, чт\'{о} оно есть, только в своей границе; следовательно,
{\em точка} есть граница
{\em линии} не только таким образом, что последняя лишь
прекращается в точке, и что линия как наличное бытие есть вне точки;
{\em линия} есть граница
{\em плоскости} не только таким образом, что последняя
лишь прекращается в линии (это точно так же применимо к
{\em плоскости}, как к границе
{\em тела}). А в точке линия также и
{\em начинается;} точка есть абсолютное начало линии.
Даже и в том случае, когда линию представляют себе продолженной в обе ее
стороны безгранично, или, как обыкновенно выражаются, бесконечно, точка
составляет ее {\em элемент}, подобно тому как линия
составляет элемент плоскости, а плоскость "--- элемент тела. Эти
{\em границы} суть {\em принцип}
того, чт\'{о} они ограничивают, подобно тому, как единица, например, как сотая,
есть граница, но вместе с тем также и элемент всей сотни.

{\em Другим} определением служит беспокойство нашего
нечто "--- беспокойство, состоящее в том, что оно в своей границе, в которой
оно пребывает, представляет собою {\em противоречие},
заставляющее его выходить дальше самого себя. Так например, точка есть
диалектика самой себя, заставляющая ее стать линией; линия "--- диалектика,
заставляющая стать плоскостью, плоскость "--- диалектика, заставляющая стать
целостным пространством. Вторая дефиниция, которую дают линии, плоскости и
всему пространству, гласит поэтому, что через
{\em движение} точки возникает линия, через движение
линии возникает плоскость и~т.~д. Но на это
{\em движение} точки, линии и~т.~д. смотрят как на
нечто случайное или как на нечто такое, чт\'{о} мы только представляем себе.
Однако от этого взгляда, собственно говоря, отказываются уже тогда, когда
признают, что определения, из которых, согласно этой дефиниции, возникают
линии и~т.~д., суть их {\em элементы} и принципы, а
последние суть не~что иное, как вместе с тем и их границы; возникновение,
таким образом, рассматривается не как случайное или лишь представляемое.
Что точка, линия, поверхность сами по себе, противореча себе, суть начала,
которые сами отталкиваются от себя, и что точка, следовательно, сама собою,
через свое понятие, переходит в линию, {\em движется в
себе} и заставляет возникнуть линию и~т.~д., "--- это лежит в понятии
имманентной данному нечто границы. Однако само применение должно
рассматриваться не здесь, а там, где будем трактовать о пространстве; чтобы
здесь только намекнуть на это применение, скажем, что точка есть совершенно
абстрактная граница, но {\em в некотором наличном
бытии;} последнее берется здесь еще совершенно неопределенно; оно есть так
называемое абсолютное, т.~е. абстрактное
{\em пространство}, безоговорочно непрерывная
внеположность. Тем самым, что граница не есть абстрактное отрицание, а есть
отрицание {\em в этом наличном бытии}, тем самым, что
она есть {\em пространственная} определенность, "--- точка
пространственна и представляет собою противоречие между абстрактным
отрицанием и непрерывностью и, значит, совершающийся и совершившийся
переход в линию и~т.~д., как и на самом деле [в~реальном мире]
{\em нет} ни точки, ни линии, ни поверхности.

Нечто вместе со своей имманентной границей, положенное, как противоречие
самого себя, в силу которого оно выводится и гонится вне себя, есть
{\em конечное}.

\subsection[c) Конечность]{c) Конечность}

Наличное бытие определённо; нечто имеет некоторое качество, и оно в последнем
не только определённо, но и ограниченно; его качество есть его граница;
обремененное границей нечто сначала остается утвердительным, спокойным
наличным бытием. Но когда это отрицание развито так, что противоположность
между его наличным бытием и отрицанием как имманентной ему границей сама
есть внутри-себя-бытие этого нечто, и последнее, таким образом, есть лишь
становление в нем самом, "--- когда это отрицание так развито, оно составляет
его (этого нечто) конечность.

Когда мы говорим о вещах, что {\em они конечны}, то мы
подразумеваем под этим, что они не только имеют некоторую определенность,
что качество есть не только реальность и сущее-в-себе определение, что они не
только ограничены, "--- а как таковые они еще обладают наличным бытием вне
своей границы, "--- но что, наоборот, небытие составляет их природу, их бытие.
Конечные вещи {\em суть}, но их соотношение с самими
собою состоит в том, что они соотносятся с самими собою как
{\em отрицательные}, что они именно в этом соотношении
с самими собою гонят себя дальше себя, дальше своего бытия. Они
{\em суть}, но истиной этого бытия служит их
{\em конец}. Конечное не только изменяется, как нечто
вообще, а {\em преходит;} и это не только возможно, что
оно преходит, так что оно могло бы быть, не преходя, а бытие конечных вещей
как таковое состоит в том, что они носят в себе зародыш прехождения, как
свое внутри-себя-бытие, что час их рождения есть час их смерти.

\styledsubsubsection%
[$\alpha$. Непосредственность конечности]%
{\centering $\alpha$. Непосредственность конечности}

Мысль о конечности вещей влечет за собой эту скорбь по той причине, что эта
конечность есть доведенное до последнего заострения качественное отрицание
и что в простоте такого определения им уже более не оставлено никакого
утвердительного бытия, {\em отличного} от их
определения к гибели. Вследствие этой качественной простоты отрицания,
возвратившегося к абстрактной противоположности ничто и прехождения, с
одной стороны, и бытия "--- с другой, конечность есть наиболее упрямая
категория рассудка; отрицание вообще, характер, граница уживаются со своим
иным, с наличным бытием; даже от абстрактного ничто, взятого само по
себе, как абстракция, готовы отказаться; но конечность есть
{\em фиксированное в себе} отрицание и поэтому резко
противостоит своему утвердительному. Конечное, правда, не сопротивляется
тому, чтобы его приводили в движение, оно само и состоит в том, что оно
предназначено к своему концу, но лишь к своему концу; оно есть упорный
отказ от того, чтобы его утвердительно приводили к его утвердительному, к
бесконечному, недопущение того, чтобы его приводили в связь с последним.
Оно, следовательно, положено нераздельным со своим ничто, и этим отрезан
путь к какому бы то ни было его примирению со своим иным, с
утвердительным. Определение конечных вещей не простирается далее их
{\em конца}. Рассудок никак не хочет отказаться от этой
скорби о конечности, делая небытие определением вещей и вместе с тем
{\em непреходящим} и
{\em абсолютным}. Их преходимость могла бы прейти лишь
в их ином, в утвердительном; тогда их конечность отделилась бы от них. Но
она есть их неизменное качество, т.~е. не переходящее в свое иное, т.~е.
в свое утвердительное; {\em таким образом, она вечна}.

Это "--- весьма важное соображение; но что конечное абсолютно, "--- это такая
точка зрения, которую, разумеется, вряд ли какое-либо философское учение
или какое-либо воззрение или рассудок позволят навязать себе; можно
сказать, что, наоборот, в утверждении о конечном определенно заключается
противоположный взгляд: конечное есть ограниченное преходящее; конечное
есть {\em только} конечное, а не непреходящее; это
заключается непосредственно в его определении и выражении. Но важно знать,
настаивает ли это воззрение на том, чтобы мы не шли дальше
{\em бытия конечности} и рассматривали
{\em преходимость} как остающуюся существовать, или же
оно признает, что {\em преходимость} и
{\em прехождение преходят}? Что последнее не имеет
места, это как раз фактически утверждается тем воззрением на конечное,
которое делает {\em прехождение последним словом} о
конечном. Оно определенно утверждает, что конечное непримиримо и
несоединимо с бесконечным, что конечное безоговорочно противоположно
бесконечному. Бесконечному это воззрение приписывает бытие, абсолютное
бытие; конечное, таким образом, остается в отношении к нему фиксированным
как его отрицательное; несоединимое с бесконечным, оно остается абсолютным
на своей собственной стороне; оно могло бы получить утвердительность от
утвердительного, от бесконечного и, таким образом, оно прешло бы; но
соединение-то с последним именно и объявляется невозможным. Если верно, что
оно пред лицом бесконечного не пребывает, а преходит, то, как мы сказали
раньше, последнее слово о нем есть прехождение, а не утвердительное,
которым могло бы быть лишь прехождение прехождения. Если же конечное
преходит не в утвердительном, а его конец понимается как
{\em ничто}, то мы снова оказываемся у того первого,
абстрактного ничто, которое само давно прешло.

Однако в этом ничто, которое должно быть {\em только}
ничто и которому вместе с тем приписывают некоторое существование, а
именно, существование в мышлении, представлении или речи, мы встречаем то
же самое противоречие, которое только что было указано в конечном, с той
только разницей, что в абстрактном ничто это противоречие только
{\em встречается}, а в конечности оно
{\em решительно выражено}. Там оно представляется
субъективным, здесь же утверждают, что конечное
{\em противостоит} бесконечному
{\em во веки веков}, {\em есть} в
себе ничтожное и есть {\em как} в себе ничтожное. Это
нужно осознать; и развертывание конечного показывает, что оно в самом себе,
как это внутреннее противоречие, рушится внутри себя, но при этом
действительно разрешает указанное противоречие, обнаруживая, что оно не
только преходяще и преходит, но что прехождение, ничто не есть нечто
окончательное, а само преходит.

\bigskip

\styledsubsubsection%
[$\beta$. Предел и долженствование]%
{\centering $\beta$. Предел и долженствование}

Хотя абстрактно это противоречие сразу же содержится в том, что
{\em нечто} конечно, или, иначе говоря, что конечное
есть, однако {\em нечто} или бытие теперь уже более не
положено абстрактно, а рефлектировано в себя и развито, как
внутри-себя-бытие, имеющее в себе некоторое определение и характер, и, еще
определеннее, оно развито так, что имеет границу в нем самом, которая,
будучи имманентной этому нечто и составляя качество его внутри-себя-бытия,
есть конечность. Мы должны посмотреть, какие моменты содержатся в этом
понятии конечного нечто.

Определение и характер оказались {\em сторонами} для
внешней рефлексии. Но первое уже содержало инобытие, как принадлежащее к
<<{\em в~себе}>> данного нечто. Внешность инобытия есть,
с одной стороны, в собственной внутренности нечто, а, с другой стороны, она
как внешность остается отличной от последней, она еще есть внешность как
таковая, но {\em в} (an) нечто. Но так как, далее,
инобытие как {\em граница} само определено как
отрицание отрицания, то имманентное нашему нечто инобытие положено как
соотношение обеих сторон, и единство нашего нечто с собою, которому (нечто)
принадлежит как определение, так и характер, оказывается его обращенным
против самого себя соотношением, отрицающим в нем его имманентную границу
соотнесением его в-себе-сущего определения с этой границей. Тождественное с
собою внутри-себя-бытие соотносится, таким образом, с самим собою как со
своим собственным небытием, однако как отрицание отрицания, как отрицающее
это свое небытие, которое вместе с тем сохраняет в нем наличное бытие, ибо
оно есть качество его внутри-себя-бытия. Собственная граница данного нечто,
положенная, таким образом, им как такое отрицательное, которое вместе с тем
существенным образом есть, есть не только граница как таковая, а
{\em предел}. Но предел не есть только положенное как
подвергнутое отрицанию. Отрицание обоюдоостро, поскольку положенное им как
отрицаемое есть {\em граница}. А именно, последняя есть
вообще нечто такое, что обще данному нечто и его иному; она есть также
определенность {\em в-себе-бытия} определения как
такового. Это в-себе-бытие, следовательно, как отрицательное соотношение со
своей также и отличной от него границей, с собою как пределом, есть
{\em долженствование}.

Для того, чтобы граница, которая есть вообще в нечто, была пределом, оно
необходимо должно вместе с тем внутри самого себя переступать ее, в самом
себе соотноситься {\em с нею как с некоторым не-сущим}.
Наличное бытие нашего нечто лежит спокойно-равнодушно, как бы
{\em подле} своей границы. Но нечто переступает свою
границу лишь постольку, поскольку оно есть ее снятость, отрицательное по
отношению к ней в-себе-бытие. А так как она в самом
{\em определении} имеет бытие как предел, то нечто тем
самым {\em переступает через самого себя}.

Долженствование содержит, следовательно, двоякое определение: содержит,
{\em во-первых}, его как в-себе-сущее определение,
противостоящее отрицанию, но, {\em во-вторых}, содержит
это же определение как некоторое небытие, которое как предел отлично от
него, но вместе с тем само есть в-себе-сущее определение.

Итак, конечное определилось как соотношение его определения с границей,
первое есть в этом соотношении {\em долженствование}, а
последняя есть {\em предел}. Оба суть, таким образом,
моменты конечного; тем самым оба, как долженствование, так и предел, сами
конечны. Но лишь предел {\em положен} как конечное;
долженствование ограничено лишь в себе, следовательно, лишь для нас.
Благодаря своему соотношению с ему самому уже имманентной границей оно
ограничено, но эта его ограниченность закутана во в-себе-бытие, ибо по
своему наличному бытию, т.~е. по своей определенности, противостоящей
пределу, долженствование положено как в-себе-бытие.

То, чт\'{о} должно быть, {\em есть} и, вместе с тем,
{\em не есть}. Если бы оно {\em было}, оно тогда не только
{\em должно было бы быть}. Следовательно,
долженствование имеет по существу своему некоторый предел. Этот предел не
есть некое чуждое; {\em то}, чт\'{о} {\em лишь} должно быть, есть
{\em определение}, которое теперь положено таковым,
каково оно есть в самом деле, а именно, как то, чт\'{о} есть вместе с тем лишь
некоторая определенность.

В-себе-бытие, присущее нашему нечто в его определении, низводит себя,
следовательно, до уровня {\em долженствования} тем, что
то самое, чт\'{о} составляет его в-себе-бытие, дано (ist) в одном и том же
отношении как {\em небытие} и притом таким образом, что
во внутри-себя-бытии, в отрицании отрицания, означенное в-себе-бытие как
одно отрицание (отрицающее) есть единство с другим отрицанием, которое как
качественно другое есть вместе с тем граница, благодаря чему указанное
единство дано как {\em соотношение} с нею. Предел
конечного не есть некое внешнее, а его собственное определение есть также и
его предел; и последний есть как сам он, так и долженствование; он есть
общее обоим, или, вернее, то, в чем оба тождественны.

Но, далее, как долженствование конечное выходит {\em за}
свой предел; та же самая определенность, которая есть его отрицание, также
и снята и, таким образом, есть его в-себе-бытие; его граница также и не
есть его граница.

Как {\em долженствование} нечто, следовательно,
{\em выше своего предела}, но и наоборот, лишь
{\em как долженствование} оно имеет свой
{\em предел;} оба нераздельны. Нечто имеет предел
постольку, поскольку оно в своем определении имеет отрицание, а определение
есть также и снятость предела.

\paragraph[Примечание. Долженствование]%
{\centering {\lsstyle\mdseries Примечание}\nopagebreak\\\vspace{1mm}
\footnotesize [Долженствование]\\\vspace{2mm}\\}

Долженствование играло недавно большую роль в философии, преимущественно в
том, что касается морали, а также и в метафизике вообще, как последнее и
абсолютное понятие о тождестве в-себе-бытия или соотношения с
{\em самим собою} и {\em определенности} или границы.

<<{\em Ты можешь, потому что ты должен}>>\pagenote{Имеется в виду
кантианец Карл Шмид (1761---1812), философскую точку зрения которого Шиллер
в~своем сатирическом стихотворении <<Die Phi\-lo\-sophen>> выразил
следующим образом:\par
Auf theo\-reti\-schem Feld ist weiter nichts mehr zu finden,\par
Aber der prakti\-sche Satz gilt doch: du kannst, denn du sollst!\\
(<<В области теоретического разума нельзя ничего больше найти, но остается в
силе положение практического разума: ты можешь, потому что ты должен!>>).
У~самого Канта рассуждения на тему <<ты можешь, потому что ты должен>>,
встречаются как в <<Критике чистого разума>> (пер. Лосского, Пгр. 1915,
стр.~442), так и в <<Критике практического разума>> (пер. Соколова, Спб. 1897,
стр.~37 и др.), а также и в других его произведениях.} "--- это выражение,
которое должно было много говорить уму, содержится в понятии долженствования.
Ибо долженствование есть выход за предел; граница в нем снята, в-себе-бытие
долженствования есть, таким образом, тождественное соотношение с собою, и,
следовательно, есть абстракция представления: <<мочь, быть в состоянии>>
(Abstraktion des Könnens). "--- Но столь же правильно и обратное утверждение:
{\em ты не можешь именно потому, что ты должен}. Ибо в
долженствовании содержится также и предел как предел; вышеуказанный
формализм возможности имеет в этом пределе некоторую противостоящую ему
реальность, некоторое качественное инобытие, и их взаимоотношение есть
противоречие, означает, следовательно, не быть в состоянии или, вернее,
невозможность.

В долженствовании начинается выхождение за конечность, бесконечность.
Долженствование есть то, чт\'{о} в дальнейшем [логическом] развитии оказывается
со стороны вышеуказанной невозможности прогрессом в бесконечность.

Мы можем здесь ближе подвергнуть критике два предрассудка касательно формы
{\em предела} и {\em долженствования}. Во-первых, обыкновенно придают
{\em большое} значение пределам мышления, разума
и~т.~д. и утверждают, что наш разум, наше мышление,
{\em не в состоянии} выйти за эти пределы. В этом
утверждении сказывается отсутствие сознания того, что, определяя нечто как предел,
мы тем самым уже вышли за этот предел. Ибо некоторая определенность, граница,
определяется как предел лишь в противоположность к его иному вообще, как
к его {\em неограниченному;} иное некоторого предела
именно и есть {\em выход} за него. Камень, металл не
выходят за свой предел, потому что {\em для них} он не
есть предел. Но если при таких всеобщих положениях рассудочного мышления,
как утверждение о невозможности выйти за предел, мышление не хочет
потрудиться рассмотреть, что содержится в понятии, то можно отослать
читателя к действительности, в которой подобного рода положения оказываются
самым что ни на есть недействительным. Именно вследствие того, что мышление
{\em должно} быть чем-то более высоким, чем
действительность, {\em должно} оставаться вдали от нее,
витать в высших областях, вследствие того, следовательно, что само оно
определено как некоторое {\em долженствование}, "---
именно вследствие этого оно, с одной стороны, не движется вперед по
направлению к понятию, а, с другой стороны, оказывается в такой же мере
неистинным по отношению к действительности, в какой оно неистинно по
отношению к понятию. "--- Так как камень не мыслит и даже не ощущает, то его
ограниченность не есть {\em для него} предел, т.~е. она
не есть в нем отрицание для ощущения, представления, мышления и~т.~д.,
которыми он не обладает. Но даже и камень как некоторое нечто заключает в
себе различие между своим определением или своим в-себе-бытием и своим
наличным бытием, и постольку он тоже выходит за свой предел; понятие,
которое он есть в себе, содержит в себе тождество с его иным. Если он
есть способное к окислению [химическое] основание, то он окисляется,
нейтрализуется и~т.~д. В окислении, нейтрализации и~т.~д. его предел,
состоящий в том, что он имеет наличное бытие лишь как [химическое]
основание, снимается; он выходит за него; и точно так же и кислота снимает
свой предел, состоящий в том, что она имеет бытие только как кислота, и в
ней, равно как и в щелочном основании, имеется в такой мере
{\em долженствование} выйти за свой предел, что лишь
силой их можно заставить оставаться "--- безводными, т.~е. в чистом виде, не
нейтральными "--- кислотой и щелочным основанием.

Но если некоторое существование содержит понятие не только как абстрактное
в-себе-бытие, а как для-себя-сущую целостность, как влечение, как жизнь,
ощущение, представление и~т.~д., то оно само изнутри самого себя
осуществляет бытие за своим пределом и переход дальше его. Растение выходит
за предел "--- быть зародышем, и точно так же оно переходит за предел "--- быть
цветком, плодом, листом; зародыш становится развитым растением, цветок
отцветает и~т.~д. Чувствующее существо в пределах голода, жажды и~т.~д.
есть стремление выйти за эти пределы, и оно осуществляет этот выход. Оно
ощущает {\em боль}, и ощущение боли есть прерогатива
чувствующей природы. В его самости есть некоторое отрицание, и это
отрицание определено в его чувстве {\em как некоторый
предел} именно потому, что ощущающее существо обладает чувством своей
{\em самости}, которая есть целостность, выходящая за
пределы указанной определенности. Если бы оно не выходило за эту
определенность, оно не ощущало бы ее как свое отрицание и не испытывало бы
боли. "--- Но разум, мышление не может, дескать, выйти за предел, он, который
есть {\em всеобщее}, само по себе вышедшее за пределы
особенности, т.~е. {\em всякой} особенности,
представляющее собою лишь выхождение за предел? "--- Правда, не всякий переход
за предел и не всякое бытие за пределом есть истинное освобождение от него,
истинное утверждение; уже само долженствование есть такое несовершенное
выхождение за предел, есть вообще абстракция. Но указания на совершенно
абстрактное всеобщее достаточно, чтобы служить противовесом такому же
абстрактному заверению, что нельзя выйти за предел, или, пожалуй,
достаточно уже указания на бесконечное вообще, чтобы служить противовесом
заверению, что нельзя выйти за пределы конечного.

Можно при этом упомянуть об одном кажущемся остроумным замечании
{\em Лейбница}, что если бы магнит обладал сознанием,
то он считал бы свое направление к северу определением своей воли, законом
своей свободы\pagenote{\label{magleibn}Лейбниц приводит
этот пример с магнитной стрелкой в <<Теодицее>> (часть~I, \S~50). Ср.
замечания Плеханова об этом примере в <<К вопросу о развитии монистического
взгляда на историю>> (Соч., т.~VII, стр.~142---144).}. Верно как раз
обратное. Если бы магнит обладал сознанием и, значит, волей и свободой, то
он был бы мыслящим; и тем самым пространство как
{\em всеобщее} было бы для него объемлющим все
направления, и потому {\em одно} направление к северу
было бы для него пределом его свободы, подобно тому как для человека быть
удерживаемым на одном месте есть предел, а для растения "--- нет.

С другой стороны, {\em долженствование} есть выхождение
за предел, но такое, которое само есть лишь
{\em конечное выхождение}. Оно имеет поэтому свое место
и свою значимость в области конечного, где оно твердо держится в-себе-бытия
против ограниченного и отстаивает его как правило и существенное против
ничтожного. Обязанность (die Pflicht) есть некоторое
{\em долженствование}, обращенное против особенной
воли, против эгоистического вожделения и произвольного интереса; воле,
поскольку она в своей подвижности может изолироваться от истинного,
напоминают о последнем как о некотором долженствовании. Те, которые ставят
долженствование как принцип морали так высоко и полагают, будто непризнание
долженствования последним словом и истинным приводит к разрушению морали,
равно как резонеры, рассудок которых доставляет себе постоянное
удовлетворение тем, что он имеет возможность выставлять против всего
существующего какое-нибудь долженствование и тем самым проявлять свое
притязание на лучшее знание, и которые поэтому тоже не желают, чтобы их лишили
долженствования, не замечают того, что для интересующих их конечных
областей жизни долженствование признается вполне имеющим силу. "--- Но в самой
действительности вовсе не обстоит так печально с разумностью и законом,
чтобы им приходилось быть только {\em долженствующими}
быть, "--- на этом останавливается лишь абстракция в-себе-бытия, "--- равно как и
неверно, что долженствование, взятое в самом себе, есть нечто пребывающее
во веки веков, что было бы тем же самым, как если бы конечность была
абсолютной. Кантовская и фихтевская философии выдают
{\em долженствование} за высший пункт разрешения
противоречий разума, но это, наоборот, есть точка зрения, не желающая выйти
из области конечного и, следовательно, из противоречия.

\styledsubsubsection%
[$\gamma$. Переход конечного в бесконечное]%
{\centering $\gamma$. Переход конечного в бесконечное}

Долженствование, взятое само по себе, содержит в~себе предел, а~предел "---
долженствование. Их взаимоотношение есть само конечное, содержащее их оба
в~своем внутри-себя-бытии. Эти моменты его определения качественно
противоположны; предел определен как отрицание долженствования,
а~долженствование "--- как отрицание предела. Таким образом конечное есть
внутреннее самопротиворечие; оно снимает себя, преходит. Но этот его результат,
отрицательное вообще, есть $\alpha$)~самое его {\em определение;} ибо оно есть
отрицательное отрицательного. Конечное, таким образом, не прешло в прехождении;
оно ближайшим образом лишь стало некоторым {\em другим} конечным, которое,
однако, есть также прехождение как переход в некоторое другое конечное и~т.~д.,
можно сказать "--- до {\em бесконечности}. Но $\beta$)~рассматривая ближе этот
результат, мы убеждаемся, что в~своем прехождении, этом отрицании самого себя,
конечное достигло своего в-себе-бытия, оно в~этом прехождении
{\em слилось с самим собою}. Каждый из его моментов содержит в~себе именно этот
результат; долженствование выходит за предел, т.~е. за себя само; но~вне этого
долженствования или его иное есть лишь сам предел. Предел~же указывает
на непосредственный выход самого себя к~своему иному, которое есть
долженствование, а~последнее есть то~же самое раздвоение {\em в-себе-бытия} и
{\em наличного бытия}, чт\'{о} и предел, есть то~же, что и он; выходя за себя,
оно поэтому точно так~же лишь сливается с~собою. Это {\em тождество с~собою},
отрицание отрицания, есть утвердительное бытие, есть, таким образом, иное
конечного, долженствующего иметь своей определенностью первое отрицание;
это~иное есть~{\em бесконечное}.

\hegsection[C. Бесконечность]{C. Бесконечность}

Бесконечное в его простом понятии можно, прежде всего, рассматривать как новую
дефиницию абсолютного; как лишенное определений соотношение с собою, оно
(абсолютное) положено как {\em бытие} и {\em становление}. Формы
{\em наличного бытия} выпадают из ряда определений, которые могут быть
рассматриваемы как дефиниции абсолютного, ибо формы указанной сферы, взятые
сами по себе, непосредственно положены лишь как определенности, как конечные
вообще. Бесконечное же признается безоговорочно абсолютным, так как оно явно
определено как отрицание конечного, и в бесконечном, следовательно, явно
выраженным образом принимается во внимание ограниченность, которой могли бы
обладать бытие и становление, хотя сами в себе они не обладают никакой
ограниченностью и не обнаруживают таковой, "--- принимается во внимание
ограниченность и отрицается наличие таковой в~нем.

Но тем самым бесконечное на самом деле отнюдь еще не избавлено от
ограниченности и конечности. Главное состоит в том, чтобы отличить истинное
понятие бесконечности от дурной бесконечности, бесконечное разума от
бесконечного рассудка; однако последнее есть {\em оконеченное} бесконечное, и
мы увидим, что, удерживая бесконечное чистым от конечного и вдали от него, мы
его как раз лишь оконечиваем.

Бесконечное есть:

a) в {\em простом определении} утвердительное как отрицание конечного;

b) но оно тем самым находится во {\em взаимоопределении
с конечным} и есть абстрактное, {\em одностороннее бесконечное;}

c) оно есть само снятие этого бесконечного, а равно и конечного, как
{\em единый} процесс, "--- есть {\em истинное бесконечное}.

\subsection[a) Бесконечное вообще]{a) Бесконечное вообще}

Бесконечное есть отрицание отрицания, утвердительное,
{\em бытие}, которое, выйдя из ограниченности, вновь
восстановило себя. Бесконечное {\em есть}, и оно есть в
более интенсивном смысле, чем первое непосредственное бытие; оно есть
истинное бытие, восстание из предела. При слове <<бесконечное>> для души и
для духа {\em восходит} его свет, ибо в нем дух не
только находится абстрактно у себя, а поднимается к самому себе, к свету
своего мышления, своей всеобщности, своей свободы.

Сначала выяснилось по отношению к понятию бесконечного, что наличное бытие
в~своем в-себе-бытии определяет себя как конечное и выходит за предел. Природа
самого конечного в~том и состоит, чтобы выходить за себя, отрицать свое
отрицание и становиться бесконечным. Бесконечное, стало быть, не стоит над
конечным, как нечто само по себе готовое, так что выходило бы, что конечное
имеет и сохраняет место {\em вне} его или под ним. Равным образом дело не
обстоит так, что лишь {\em мы}, как некоторый субъективный разум, выходим за
пределы конечного, переходим в бесконечное. Так например, представляют себе
дело, когда говорят, что бесконечное есть понятие разума, и мы посредством
разума возвышаемся над земным и бренным; тут выходит так, что это совершается
без всякого ущерба для конечного, которого вовсе не касается это остающееся для
него внешним возвышение. Но поскольку само конечное поднимается до
бесконечности, оно отнюдь не принуждается к этому чуждой силой, а его
собственная природа состоит в том, чтобы соотноситься с собою как с пределом
"--- и притом как с пределом как таковым, так и с пределом как долженствованием
"--- и выходить за этот предел, или, вернее, его природа состоит в том, чтобы
оно как соотношение с собою подвергло отрицанию этот предел и вышло за него. Не
в упразднении конечности вообще рождается (wird) бесконечность вообще, а
конечное только и состоит в~том, что само оно через свою природу становится
бесконечным. Бесконечность есть его {\em утвердительное определение},
то, чт\'{о} оно поистине есть в~себе.

Таким образом, конечное исчезло в бесконечном,
и то, чт\'{о} {\em есть}, есть лишь {\em бесконечное}.

\subsection[b) Взаимоопределение конечного и бесконечного]%
{b) Взаимоопределение конечного и бесконечного}

Бесконечное {\em есть;} в этой непосредственности оно
вместе с тем есть {\em отрицание} некоторого
{\em иного}, конечного. Будучи, таким образом,
{\em сущим} и вместе с тем {\em небытием} некоторого
{\em иного}, оно впало обратно в категорию нечто как
некоторого определенного вообще; говоря точнее, так как оно есть
рефлектированное в себя, получающееся посредством снятия определенности
вообще наличное бытие и, следовательно, {\em положено}
как отличное от своей определенности наличное бытие, то оно снова впало в
категорию нечто, имеющего некоторую границу. По этой определенности
конечное противостоит бесконечности как {\em реальное
наличное бытие;} таким образом, они находятся в качественном
{\em соотношении} как
{\em остающиеся} вне друг друга:
{\em непосредственное бытие} бесконечного снова
пробуждает {\em бытие} его отрицания, конечного,
которое, как сначала казалось, исчезло в бесконечном.

Но бесконечное и конечное не только находятся в этих категориях соотношения;
обе стороны определены далее так, чтобы быть в отношении друг друга лишь
{\em иными}. А именно, конечность есть предел,
положенный как предел, есть наличное бытие, положенное с
{\em определением} переходить в свое
{\em в-себе-бытие},
{\em становиться} бесконечным. Бесконечность есть ничто
конечного, его {\em в-себе-бытие} и
{\em долженствование}, но последнее дано (ist) вместе с
тем как рефлектированное в себя, как выполненное долженствование, как лишь
с самим собою соотносящееся, совершенно утвердительное бытие.
В~бесконечности имеется то удовлетворение, что всяческая определенность,
изменение, всякий предел, а с ним и само долженствование исчезли, положены
как упраздненные, как ничто конечного. Как такое отрицание конечного
определено в-себе-бытие, которое, таким образом, как отрицание отрицания
утвердительно внутри себя. Однако это утверждение есть, как качественно
{\em непосредственное} соотношение с собою,
{\em бытие;} вследствие этого бесконечное сведено к той
категории, что ему противостоит конечное как некое
{\em иное;} его отрицательная природа положена как
{\em сущее}, следовательно, как первое и
непосредственное отрицание. "--- Бесконечное, таким образом, обременено
противоположностью к конечному, которое как иное остается вместе с тем
определенным, реальным наличным бытием, хотя оно в своем в-себе-бытии, в
бесконечном, положено вместе с тем как упраздненное; последнее есть
не-конечное, "--- некое бытие в определенности отрицания. В сопоставлении с
конечным, с кругом сущих определенностей, реальностей, бесконечное есть
неопределенное пустое, потустороннее конечного, имеющего свое в-себе-бытие
не в своем наличном бытии, которое есть некоторое определенное бытие.

Бесконечное, сопоставленное таким образом с конечным, положенное во взаимном
качественном соотношении {\em иных}, должно быть
названо {\em дурным бесконечным}, бесконечным
{\em рассудка}, который считает его высшей, абсолютной
истиной. Те противоречия, в которые он впадает со всех сторон, как только
он берется за применение и объяснение этих своих категорий, должны были бы
заставить его осознать, что, полагая, что он достиг своего удовлетворения в
примирении истины, он на самом деле пребывает в непримиренном,
неразрешенном, абсолютном противоречии.

Это противоречие сразу же сказывается в том, что наряду с бесконечным
остается конечное как наличное бытие; имеются, таким образом,
{\em две} определенности;
{\em даны} (имеются) два мира, бесконечный и конечный,
и в их соотношении бесконечное есть лишь {\em граница}
конечного и, следовательно, {\em само} есть лишь
определенное, {\em конечное бесконечное}.

Это противоречие развивает свое содержание до более выразительных форм. "---
Конечное есть реальное наличное бытие, которое, таким образом, остается и
тогда, когда мы переходим к его небытию, к бесконечному. Последнее, как мы
показали, имеет своей определенностью в отношении конечного лишь первое,
непосредственное отрицание, равно как и конечное в отношении указанного
отрицания имеет, как подвергшееся отрицанию, лишь значение некоторого
{\em иного} и поэтому еще есть нечто. Следовательно,
когда поднимающийся над этим конечным миром рассудок восходит к своему
наивысшему, к бесконечному, этот конечный мир остается существовать как
некое посюстороннее, так что бесконечное лишь становится над конечным,
{\em отделяется} от него, и тем самым конечное как раз
отделяется от бесконечного. Они {\em ставятся в
различные} места; конечное как здешнее наличное бытие, а бесконечное, хотя
оно и есть <<{\em в~себе}>> конечного, все же как некое
потустороннее перемещается в смутную, недостижимую даль,
{\em вне} которой находится и остается конечное.

Отделенные таким образом друг от друга, они столь же существенно
{\em соотнесены} друг с другом как раз разлучающим их отрицанием. Это
соотносящее их "--- рефлектированные в себя нечто "--- отрицание есть их
взаимная граница одного относительно иного, и притом таким образом, что каждое
из них имеет ее не только в отношении иного в~нем, а отрицание есть их
{\em в-себе-бытие;} каждое из них, таким образом, имеет границу в самом себе,
взятом особо, в его отделенности от иного. Но эта граница имеет бытие как
первое отрицание; таким образом, оба суть ограниченные, конечные в самих себе.
Однако каждое из них, как утвердительно соотносящееся с собою, есть также и
отрицание своей границы. Таким образом, оно непосредственно отталкивает ее от
себя как свое небытие, и, будучи качественно отделенным от нее, оно ее полагает
как некоторое {\em иное бытие}, вне себя; конечное полагает свое небытие как
это бесконечное, а последнее полагает таким же образом конечное. Что от
конечного необходимо, т.~е. благодаря определению конечного, совершается
переход к бесконечному и что конечное тем самым возводится во в-себе-бытие,
с~этим легко соглашаются, поскольку конечное, хотя и определено как устойчивое
наличное бытие, определено, однако, вместе с тем {\em также} и как ничтожное
{\em в~себе}, следовательно, по самому своему определению разлагающееся,
а~бесконечное, хотя и определено как обремененное отрицанием и границей,
определено, однако, вместе с тем также и как сущее {\em в~себе}, так что эта
абстракция соотносящегося с собою утверждения составляет его определение и,
следовательно, согласно последнему в нем не заключено конечное наличное бытие.
Но мы показали выше, что само бесконечное получает утвердительное бытие лишь
{\em посредством} отрицания как отрицания отрицания и что это его утверждение,
взятое как лишь простое, качественное бытие, понижает содержащееся в нем
отрицание до простого, непосредственного отрицания и тем самым "--- до
определенности и границы, которая затем как противоречащая его в-себе-бытию
вместе с тем исключается из него, полагается как не ему принадлежащая, а,
наоборот, противоположная его в-себе-бытию, полагается как конечное. Таким
образом, поскольку каждое из них в самом себе и в силу своего определения есть
полагание своего другого, они {\em нераздельны}. Но это их единство
{\em скрыто} в~их качественном инобытии; оно есть {\em внутреннее}, которое
только лежит в основании.

Этим определен способ проявления указанного единства; положенное
в~{\em наличном бытии}, оно дано (ist) как превращение или переход конечного
в~бесконечное, и наоборот; так что бесконечное в конечном и конечное
в~бесконечном, другое в другом, лишь {\em выступает}, т.~е. каждое из них есть
некое собственное {\em непосредственное} возникновение в ином и их соотношение
есть лишь внешнее.

Процесс их перехода [друг в друга] имеет детально следующий вид. Совершается
выхождение за пределы конечного, переход в бесконечное. Это выхождение
представляется внешним действием. Что возникает в этой потусторонней для
конечного пустоте? Что в ней положительного? Вследствие нераздельности
бесконечного и конечного (или, иначе говоря, вследствие того, что это
находящееся на своей стороне бесконечное само ограничено) возникает
граница; бесконечное исчезло, и появилось его иное, конечное. Но это
появление конечного представляется чем-то внешним для бесконечного, а
новая граница "--- чем-то таким, что не возникает из самого бесконечного, а
само есть такое же преднайденное. Перед нами, таким образом, впадение снова
в прежнее, тщетно снятое определение. Но эта новая граница сама в свою
очередь есть лишь нечто такое, что должно быть снято или, иначе говоря, что
следует преступить. Стало быть, снова возникла пустота, ничто, в котором мы
равным образом встречаем указанную определенность, некоторую новую границу
"--- {\em и так далее до бесконечности}.

Имеется {\em взаимоопределение конечного и бесконечного;} конечное конечно лишь
в соотношении с долженствованием или с бесконечным, а бесконечное бесконечно
лишь в соотношении с конечным. Они нераздельны и вместе с тем суть всецело иные
в отношении друг друга; каждое из них имеет в нем самом свое иное; таким
образом, каждое есть единство себя и своего иного, и есть в своей
определенности наличное бытие, состоящее в том, чтобы {\em не}~быть тем, что
оно само есть и что есть его иное.

Это взаимоопределение, отрицающее само себя и свое отрицание, и есть то, что
выступает как {\em прогресс в бесконечность}, который в столь многих образах
и~применениях признается {\em последним словом}, дальше которого уже не идут,
ибо, дойдя до этого <<{\em и так далее} до бесконечности>>, мысль имеет
обыкновение останавливаться, достигнув своего конца. "--- Этот прогресс
проявляется всюду, где {\em относительные} определения доводятся до их
противопоставления, так что они находятся в нераздельном единстве, и~тем не
менее каждому в отношении другого приписывается самостоятельное существование
(Dasein). Этот прогресс есть поэтому {\em противоречие}, которое не разрешено,
а лишь высказывается постоянно, как просто {\em имеющееся налицо}.

Имеется некое абстрактное выхождение, которое остается неполным, так как
{\em не~выходят дальше самого этого выхождения}. Имеется бесконечное; за него,
правда, выходят, ибо полагают некоторую новую границу, но тем самым, как раз
наоборот, лишь возвращаются к конечному. Эта дурная бесконечность есть в себе
то же самое, что продолжающееся во веки веков {\em долженствование;} она, хотя
и есть отрицание конечного, не может, однако, истинно освободиться от него; это
конечное снова выступает {\em в~ней же самой} как ее иное, потому что это
бесконечное имеет бытие лишь как находящееся {\em в~соотношении} с другим для
него конечным. Прогресс в бесконечность есть поэтому лишь повторяющаяся
одинаковость, одно и то же скучное чередование этого конечного и бесконечного.

Бесконечность бесконечного прогресса остается обремененной конечным как
таковым, ограничена им и сама {\em конечна}. Но этим
она на самом деле была бы положена как единство конечного и бесконечного.
Однако указанное единство не делается предметом размышления. Тем не менее,
только оно-то и вызывает в конечном бесконечное и в бесконечном конечное;
оно есть, так сказать, движущая пружина бесконечного прогресса. Он есть
{\em внешняя сторона} сказанного единства, в которой
застревает представление; последнее застревает в этом продолжающемся во
веки веков повторении одного и того же чередования, в пустом беспокойстве
выхождения во вне, за границу к бесконечности, выхождения, которое
{\em находит} в этом бесконечном новую границу, но
столь же мало может удержаться на этой границе, как и на бесконечном. Это
бесконечное имеет твердую детерминацию некоего
{\em потустороннего}, которое не может быть достигнуто
потому, что оно не {\em должно} быть достигнуто, так
как не хотят отказаться от определенности потустороннего, от
{\em сущего} отрицания. По этому определению оно имеет
на противоположной стороне конечное как некое
{\em посюстороннее}, которое столь же мало может
возвыситься до бесконечности именно потому, что оно имеет эту детерминацию
некоторого {\em иного} и, следовательно, детерминацию
продолжающегося во веки веков, все снова и снова порождающего себя в своем
потустороннем (и притом порождающего себя как отличное от него)
{\em наличного бытия}\pagenote{Немецкий текст
этой фразы после слова <<следовательно>> испорчен. Чтобы придать ему какой бы
то ни было смысл, приходится прибегать к тем или другим конъектурам.
В~настоящем переводе положена в основу конъектура Б.~Г.~Столпнера, который
вместо слов <<hiermit ein Peren\-niren\-des>> предлагает читать <<hiermit eines
peren\-niren\-den>>. Лассон предлагает другую конъектуру: он изменяет порядок
слов и вставляет неопределенный член <<eines>>. Если принять конъектуру
Лассона, то это место надо будет перевести следующим образом: <<\ldotsи,
следовательно, некоторого такого {\em наличного бытия}, которое
порождает себе в своем потустороннем опять-таки нечто продолжающееся во
веки веков, и притом порождает его как от него отличное>>.}.

\subsection[c) Утвердительная бесконечность]{c) Утвердительная бесконечность}

В~показанном нами переходящем туда и сюда взаимоопределении конечного и
бесконечного их истина уже {\em имеется} в себе, и
требуется лишь воспринять то, что имеется. Это качание туда и сюда
составляет внешнюю реализацию понятия. В ней
{\em положено}, но {\em внешним
образом}, одно вне иного, то, что содержится в понятии; требуется лишь
сравнение этих разных моментов, в котором получается
{\em единство}, дающее само понятие.
{\em Единство} бесконечного и конечного "--- мы на это
часто указывали, но здесь следует в особенности напомнить об этом "--- есть
неудачное выражение для единства, каково оно есть поистине; но и устранение
этого неудачного определения должно иметься в этом лежащем перед нами
внешнем проявлении (Äusserung) понятия.

Взятое по своему ближайшему, лишь непосредственному определению, бесконечное
имеет бытие (ist) только как {\em выход за конечное;}
оно есть по своему определению отрицание конечного; таким образом, и
конечное имеет бытие (ist) только как то, за что следует выйти, как
отрицание себя в самом себе, отрицание, которое есть бесконечность.
{\em В каждом из них заключается}, следовательно,
{\em определенность другого}, причем по смыслу
бесконечного прогресса они исключены друг из друга и лишь попеременно
следуют одно за другим; одно не может быть положено и мыслимо (gefasst) без
другого, бесконечное "--- без конечного и конечное "--- без бесконечного. Когда
{\em высказывают}, что такое бесконечное, а именно, что
оно есть отрицание {\em конечного}, то одновременно
{\em высказывается} само конечное; и
{\em обойтись} без него при определении бесконечного
{\em нельзя}. Нужно только
{\em знать, что высказываешь}, чтобы найти в
бесконечном определение конечного. Относительно же конечного, с другой
стороны, сразу соглашаются, что оно есть ничтожное; но именно его
ничтожность и есть бесконечность, от которой оно равным образом неотделимо.
"--- Может показаться, что это понимание берет их по их
{\em соотношению с их иным}. Следовательно, если их
брать {\em безотносительно}, так что они будут
соединены лишь союзом <<и>>, то они будут противостоять друг другу, как
самостоятельные, каждое из которых есть только в самом себе. Посмотрим,
какой характер они, взятые таким способом, будут носить. Бесконечное,
поставленное таким образом, есть {\em одно из этих
двух;} но как {\em лишь} одно из двух, оно само
конечно, оно "--- не целое, а лишь одна сторона; оно имеет свою границу в
противостоящем; таким образом, оно есть {\em конечное
бесконечное}. Имеются лишь {\em два конечных}. Как раз
в том обстоятельстве, что бесконечное, таким образом,
{\em отделено} от конечного, поставлено, следовательно,
как {\em одностороннее}, и заключается его конечность
и, стало быть, его единство с конечным. "--- Конечное со своей стороны, как
поставленное само по себе, в отдалении от бесконечного, есть
{\em то соотношение с собою}, в котором удалена его
относительность, зависимость, его преходимость; оно есть те же самые
самостоятельность и утверждение себя, которыми должно быть бесконечное.

Оба способа рассмотрения, имеющие, как кажется сначала, своим исходным
пунктом разные определенности, поскольку первый брал лишь
{\em соотношение} друг с другом конечного и
бесконечного, каждого с его иным, а второй якобы удерживает их в их
полной отделенности друг от друга, приводят к одному и тому же результату.
Бесконечное и конечное, взятые по их {\em соотношению}
друг с другом, которое как будто внешне для них, но на самом деле для них
существенно, и без которого ни одно из них не есть то, что оно есть,
содержат, таким образом, свое иное в своем собственном определении, и
точно так же каждое, взятое {\em особо},
рассматриваемое в самом {\em себе}, заключает в себе
свое иное, как свой собственный момент.

Это и дает приобретшее дурную славу единство конечного и бесконечного
"--- единство, которое само есть бесконечное, охватывающее собою само себя и
конечность, "--- следовательно, бесконечное в другом смысле, чем в том,
согласно которому конечное отделено от него и поставлено на другой стороне;
так как они должны быть также и различны, то каждое, как мы показали
раньше, есть само в себе единство обоих; таким образом, получаются два
таких единства. То, что обще тому и другому, т.~е. единство этих двух
определенностей, полагает их, как единство, ближайшим образом подвергшимися
отрицанию, так как каждое берется как долженствующее быть тем, что оно есть
в их различности; в своем единстве они, следовательно, теряют свою
качественную природу. Это "--- очень важное соображение против представления,
которое не хочет отказаться от того, чтобы в единстве бесконечного и
конечного удерживать их в том качестве, которое они должны иметь, взятые
вне друг друга, и которое (представление) поэтому видит в сказанном
единстве только противоречие, а не также и разрешение последнего путем
отрицания качественной определенности их обоих. Таким образом,
фальсифицируется это ближайшим образом простое, всеобщее единство
бесконечного и конечного.

Но, далее, так как они должны быть взяты также и как различные, то единство
бесконечного [и~конечного], которое (единство) каждый из этих моментов есть
сам, определено в каждом из них различным образом. То, что по своему
определению есть бесконечное, имеет в себе (an ihm) отличную от себя
конечность; первое есть <<{\em в~себе}>> (das Ansich) в
этом единстве, а конечность есть лишь определенность, граница в нем; но это
"--- такая граница, которая есть его всецело иное, его
противоположность. Его определение, которое есть в-себе-бытие как таковое,
портится примесью такого рода качества; оно есть, таким образом,
{\em оконеченное бесконечное}. Подобным же образом, так
как конечное как таковое есть лишь не-в-себе-бытие, но согласно сказанному
единству заключает в себе также и свою противоположность, то оно
возвеличивается превыше своей ценности и притом, можно сказать,
возвеличивается бесконечно; оно полагается, как
{\em обесконеченное} конечное.

Таким же образом, как раньше рассудок извращал простое единство, он
теперь извращает и двоякое единство бесконечного и конечного.
Это и здесь также происходит потому, что в одном из этих двух единств
бесконечное берется не как подвергшееся отрицанию, а, наоборот, как
в-себе-бытие, в котором, следовательно, не должны быть положены
определенность и предел; в-себе-бытие этим-де унижается и портится.
Обратно, конечное равным образом фиксируется, как не подвергшееся
отрицанию, хотя в себе ничтожное, так что оно в своей связи с бесконечным
возводится в то, что оно не {\em есть}, и тем самым ему
в противоположность его не исчезнувшему, а, наоборот, вековечному
определению придается характер бесконечности.

Фальсификация, которую проделывает рассудок касательно конечного и
бесконечного и которая состоит в том, что он фиксирует их взаимоотношение
как качественную разность и утверждает, что они в своем определении
раздельны и притом абсолютно раздельны, "--- эта фальсификация основывается на
забвении того, что представляет собою понятие этих моментов для самого же
рассудка. Согласно этому понятию единство конечного и бесконечного не есть
ни внешнее сведение их вместе, ни ненадлежащее, противное их определению
соединение, в котором связывались бы в себе раздельные и противоположные,
самостоятельные в отношении друг друга, сущие и, стало быть, несовместимые
[определения], а каждое есть само в себе это единство, и притом лишь как
{\em снятие} самого себя, снятие, в котором ни одно не
имеет перед другим преимущества в-себе-бытия и утвердительного наличного
бытия. Как мы показали раньше, конечность имеет бытие лишь как выход за
себя; в ней, следовательно, содержится бесконечность, иное ее самой.
И~точно так же бесконечность имеет бытие лишь как выход за конечное. В ней,
следовательно, существенно содержится ее иное, и она есть, следовательно,
в ней же самой иное самой себя. Конечное не снимается бесконечным как вне
его имеющейся силой, а его собственная бесконечность состоит в том, что оно
снимает само себя. Это снятие есть, стало быть, не изменение или инобытие
вообще, не снятие [данного] {\em нечто}. То, в чем
конечное снимает себя, есть бесконечное как отрицание конечности; но
последняя сама давно уже есть лишь наличное бытие, определенное как
некоторое {\em небытие}. Следовательно, это только
{\em отрицание снимает себя в отрицании}. Точно так же
бесконечность со своей стороны определена как отрицательное конечности и
тем самым определенности вообще, "--- как бессодержательное (leere)
потустороннее; его снятие себя в конечном есть возвращение из
бессодержательного бегства, {\em отрицание} такого
потустороннего, которое есть некоторое
{\em отрицательное} в самом себе.

Стало быть, в обоих имеется здесь налицо одно и то же отрицание отрицания.
Но это отрицание отрицания есть {\em в~себе}
соотношение с самим собою, утверждение, однако как возвращение к самому
себе, т.~е., через {\em опосредствование}, которое есть
отрицание отрицания. Эти-то определения следует по существу иметь в виду;
второе же, что следует иметь в виду, "--- это то, что они в бесконечном
прогрессе также и {\em положены}, и тот способ, каким
они положены, а именно, следует иметь в виду, что они положены еще не в
своей последней истине.

Здесь, {\em во-первых}, оба, как бесконечное, так и
конечное, подвергаются отрицанию, "--- совершается одинаковым образом выход
как за конечное, так и за бесконечное; {\em во-вторых},
они полагаются также и как различные, каждое после другого, полагаются как
сами по себе положительные. Мы выделяем, таким образом, эти два
определения, сравнивая их между собою, точно так же, как мы в сравнении,
внешнем сравнении, отделили друг от друга два способа рассмотрения
"--- рассмотрение конечного и бесконечного в их соотношении и рассмотрение
каждого из них, взятого само по себе. Но бесконечный прогресс выражает еще
нечто большее: в нем положена также и {\em связь} также
и различных, однако ближайшим образом она еще положена только как переход и
чередование. Нам следует в простом размышлении лишь разглядеть то, что
здесь на самом дело имеется.

Сначала можно брать то отрицание конечного и бесконечного, которое положено
в бесконечном прогрессе, как простое, следовательно, брать их как
внеположные, лишь следующие друг за другом. Если начнем с конечного, то
совершается выход за границу, конечное подвергается отрицанию,
потустороннее этого конечного, бесконечное, имеется следовательно теперь
налицо, но в последнем снова возникает граница; таким образом, имеется
выход за бесконечное. Это двойное снятие, однако, частью положено вообще
лишь как некоторое внешнее событие (Geschehen) и чередование моментов,
частью же еще не положено как одно единство; каждое из этих выхождений есть
особый разбег, новый акт, так что они, таким образом, лишены связи друг с
другом. "--- Но в бесконечном прогрессе налицо также и их соотношение.
Имеется, во-первых, конечное; засим совершается выхождение за него; это
отрицательное или потустороннее конечного есть бесконечное; в-третьих,
совершают снова выход, выходят также и за это отрицание, возникает новая
граница, опять некоторое конечное. "--- Это "--- полное, замыкающее само себя
движение, пришедшее к тому, что составляло начало. Возникает то же самое,
из чего исходили, т.~е. конечное восстановлено; последнее, следовательно,
слилось с самим собою, снова нашло в своем потустороннем лишь само себя.

То же самое происходит и с бесконечным. В бесконечном, в потустороннем
данной границы, возникает лишь новая граница, которую постигает та же самая
участь "--- подвергнуться отрицанию в качестве конечного. Что, таким образом,
снова имеется, это то же самое бесконечное, которое перед тем исчезло в
новой границе. Бесконечное поэтому указанным снятием его, этой новой
границей, не выталкивается дальше за последнюю, оно не удалено ни от
конечного, "--- ибо последнее и состоит лишь в том, что оно переходит в
бесконечное, "--- ни от себя самого, ибо оно прибыло к себе.

Таким образом, оба, конечное и бесконечное, суть
{\em движение}, состоящее в возвращении к себе через
свое отрицание; они имеют бытие (sind) лишь как
{\em опосредствование} внутри себя, и утвердительное
обоих содержит в себе отрицание обоих и есть отрицание отрицания. "--- Они,
таким образом, суть {\em результат} и, стало быть, не
то же самое, чем они были в определении их
{\em начала}, "--- конечное не есть со своей стороны
некоторое {\em наличное бытие}, а бесконечное не есть
некоторое {\em наличное бытие} или
{\em в-себе-бытие} по ту сторону наличного бытия, т.~е.
определенного как конечное. Против единства конечного и бесконечного
рассудок столь энергично восстает только потому, что он предполагает предел
и конечное, равно как и в-себе-бытие {\em вековечными;}
тем самым он {\em упускает из виду} отрицание обоих,
фактически имеющиеся в бесконечном прогрессе, равно как и то, что они
встречаются в последнем лишь как моменты некоторого целого и что каждое из
них выступает наружу лишь через посредство своего противоположного, а по
существу также и через посредство снятия своего противоположного.

Когда мы в предыдущем рассматривали ближайшим образом возвращение к себе
как, с одной стороны, возвращение к себе конечного, а с другой стороны
"--- возвращение к себе бесконечного, то в самом этом результате
обнаруживается некоторая неправильность, находящаяся в связи с только что
порицавшейся нами неудачностью [выражения: единство бесконечного и
конечного]: в первый раз взято {\em исходным пунктом}
конечное, а во второй раз "--- бесконечное, и только благодаря этому возникают
{\em два} результата. Но на самом деле совершенно
безразлично, какое из них мы берем как начало и, следовательно, само собою
отпадает то различие, которое породило {\em двоякость}
результата. Это равным образом положено в неограниченной по направлению
обеих сторон линии бесконечного прогресса, в котором с одинаковым
чередованием имеется каждый из моментов, и является совершенно внешним
делом, за какой из них и где именно мы возьмемся, чтобы сделать его
началом. "--- Они различаются в этом бесконечном прогрессе, но равным образом
одно есть лишь момент другого. Поскольку они оба, конечное и бесконечное,
сами суть моменты прогресса, они суть {\em сообща
конечное}, а поскольку они столь же сообща подвергаются отрицанию и в нем и
в результате, то этот результат как отрицание указанной конечности обоих
истинно именуется бесконечным. Их различие есть, таким образом, тот
{\em двоякий смысл}, который они оба имеют. Конечное
имеет тот двоякий смысл, что оно, во-первых, есть лишь конечное
{\em наряду} с бесконечным, которое ему противостоит, и
что оно, во-вторых, есть {\em вместе} и конечное и
противостоящее ему бесконечное. Бесконечное также имеет тот двоякий смысл,
что оно есть, во-первых, {\em один} из этих двух
моментов, "--- таким образом, оно есть дурное бесконечное "--- и, во-вторых, оно
есть то бесконечное, в котором оба, оно само и его другое, суть лишь
моменты. Следовательно, на самом деле бесконечное, взятое таковым, как оно
подлинно имеется, есть процесс, в котором оно понижает себя до того, чтобы
быть лишь {\em одним} из своих определений,
противостоять конечному, и, значит, быть самому лишь одним из конечных, а
затем снимает это различие себя от себя самого, превращает его в
утверждение себя и есть через это опосредствование
{\em истинно бесконечное}.

Это определение истинно бесконечного не может быть облечено в уже
отвергнутую нами {\em формулу единства} конечного и
бесконечного; {\em единство} есть абстрактное,
неподвижное саморавенство, и моменты тогда также оказываются неподвижно
сущими. Бесконечное же, подобно своим двум моментам, есть, наоборот, по
существу лишь {\em становление}, но становление, теперь
{\em далее определенное} в своих моментах. Становление
имеет сначала своими определениями абстрактное бытие и ничто; затем оно как
изменение имеет своими моментами налично сущие, т.~е. нечто и другое;
теперь же как бесконечное оно имеет своими моментами конечное и
бесконечное, которые сами суть становящиеся.

Это бесконечное как возвращенность в себя, соотношение себя с самим собою,
есть {\em бытие}, но не лишенное определений
абстрактное бытие, ибо оно положено отрицающим отрицание; оно,
следовательно, есть также и {\em наличное бытие}, ибо
оно содержит в себе отрицание вообще и, стало быть, определенность. Оно
{\em есть} и оно {\em есть здесь},
налично, присутствует. Только дурное бесконечное есть
{\em потустороннее}, ибо оно представляет собою
{\em лишь} отрицание конечного, положенного как
{\em реальное;} таким образом, оно есть абстрактное,
первое отрицание; будучи определено {\em лишь} как
отрицательное, оно не имеет в себе утверждения
{\em наличного бытия;} фиксированное как только
отрицательное, оно даже {\em не должно} быть
{\em тут} "--- оно должно быть недостижимым. Но эта
недостижимость есть не его величие (Hoheit), а его недостаток, который
имеет свое последнее основание в том, что
{\em конечное} как таковое удерживается как
{\em сущее}. Неистинное есть недостижимое; и легко
усмотреть, что такое бесконечное неистинно. "--- Образом прогресса в
бесконечность служит {\em прямая линия}, только на
обеих границах которой лежит бесконечное и всегда лишь там, где ее "--- а она
есть наличное бытие "--- нет, и которая выходит
{\em вовне} к этому своему неимению наличного бытия
(Nicht\-dasein), т.~е. выходит вовне в неопределенность; истинная же
бесконечность, обратно в себя загибающаяся, имеет своим образом
{\em круг}, достигшую себя линию, которая замкнута и
всецело налична, не имеет ни {\em начального}, ни {\em конечного} пункта.

Истинная бесконечность, взятая, таким образом, вообще как
{\em наличное бытие}, положенное как
{\em утвердительное} в противоположность абстрактному
отрицанию, есть {\em реальность} в более высоком
смысле, чем та реальность, которая была {\em просто}
определена раньше; она получила здесь некоторое конкретное содержание. Не
конечное есть реальное, а бесконечное. Так и в дальнейшем реальность
определяется как сущность, понятие, идея и~т.~д. Однако при рассмотрении
более конкретного излишне повторять такие более ранние, более абстрактные
категории, как реальность, и применять их для характеристики более
конкретных определений, чем то, что они суть сами в себе. Такое повторение,
как, например, в том случае, когда говорят,
что сущность "--- или идея "--- есть
реальное, вызывается тем, что для некультивированного мышления самые
абстрактные категории, например бытие, наличное бытие, реальность,
конечность, суть наиболее привычные.

Здесь повторение категории реальности вызывается более определенным поводом,
так как то отрицание, в отношении которого она есть утвердительное, есть
здесь отрицание отрицания, и, стало быть, она сама противополагается той
реальности, которая есть конечное наличное бытие. "--- Отрицание определено,
таким образом, как идеальность; идеализованное\footnote{{\em Идеальное}
имеет дальнейшее, более определенное значение (прекрасного и
того, что ведет к последнему), чем {\em идеализованное;} первому здесь еще
не место; поэтому мы здесь употребляем выражение: <<{\em идеализованное}>>.
В~отношении к реальности это различие в словоупотреблении не имеет место;
<<das Reelle>> и <<das Reale>> употребляются приблизительно в одном и том же
значении. Выяснение оттенков этих двух выражений в их отличии друг от друга
не представляет интереса.} есть конечное, как оно есть в истинном
бесконечном "--- как некоторое определение, содержание, которое различено, но
не есть нечто {\em самостоятельно сущее}, а имеет бытие
как {\em момент}. Идеальность имеет этот более
конкретный смысл, который не вполне выражен отрицанием конечного наличного
бытия. Но в отношении реальности и идеальности противоположность между
конечным и бесконечным понимают так, что конечное считается реальным, а
бесконечное идеализованным; как и в дальнейшем, понятие рассматривается как
некоторое идеализованное и притом как некоторое
{\em лишь} идеализованное, наличное же бытие вообще
рассматривается, наоборот, как реальное. При таком понимании, разумеется,
нисколько не поможет то, что мы имеем для обозначения указанного
конкретного определения отрицания особое слово <<идеализованное>>; в этой
противоположности снова возвращаются к односторонности абстрактного
отрицания, которая присуща дурному бесконечному, и упорно настаивают на
утвердительном наличном бытии конечного.

\subsection[Переход]{Переход}

Идеальность может быть названа {\em качеством}
бесконечности; по существу она есть процесс
{\em становления} и тем самым некоторый переход,
подобный переходу становления в наличное бытие, и теперь следует указать
характер этого перехода. Как снятие конечности, т.~е. и конечности как
таковой, и равным образом лишь противостоящей ей, лишь отрицательной
бесконечности, это возвращение в себя есть {\em соотношение} с самим собой,
{\em бытие}. Так как в этом бытии есть отрицание, то
оно есть {\em наличное бытие}, но так как, далее, это отрицание есть
по существу отрицание отрицания, соотносящееся с собою отрицание, то
оно есть то наличное бытие, которое именуется {\em для-себя-бытием}.

\hegremark[Примечание 1]{Бесконечный прогресс}{[Бесконечный прогресс]}

Бесконечное "--- взятое в обычном смысле, в смысле дурного бесконечного "--- и
{\em прогресс в бесконечность} как {\em долженствование} суть выражение
{\em противоречия}, которое выдает само себя за
{\em разрешение} и за последнее слово. Это бесконечное
есть первое возвышение чувственного представления над конечным, возвышение
его в область мысли, имеющей, однако, своим содержанием лишь ничто, некое
{\em нарочито} положенное как не-сущее, "--- есть бегство
за пределы ограниченного, не концентрирующееся на самом себе и не умеющее
возвратить отрицательное к положительному. Эта
{\em незавершенная рефлексия} имеет перед собою
полностью оба определения истинно бесконечного:
{\em противоположность} между конечным и бесконечным и
{\em единство} конечного и бесконечного, но
{\em не сводит вместе этих двух мыслей}. Одна мысль
неразлучно приводит за собою другую, эта же рефлексия лишь
{\em чередует} их. Изображение этого чередования,
бесконечный прогресс, появляется повсюду, где не хотят выбраться из
противоречия {\em единства} двух определений и их
{\em противоположности}. Конечное есть снятие самого
себя, оно заключает в себе свое отрицание, бесконечность: это "--- их
{\em единство}. Затем совершается выход
{\em вовне} за конечное к бесконечному, как к
потустороннему конечного: это "--- их {\em разъединение}.
Но за бесконечным есть другое конечное; выход за конечное, бесконечность,
содержит в себе конечность: это "--- их {\em единство}. Но
это конечное есть также некое отрицание бесконечного: это "--- их
{\em разъединение}. и~т.~д. "--- Так, например, в
причинном отношении причина и действие нераздельны: причина, которая не
производила бы никакого действия, не была бы причиной, равно как действие,
которое не имело бы причины, уже не было бы действием. Это отношение
приводит таким образом к бесконечному прогрессу
{\em причин и действий}. Нечто определено как причина,
но последняя как конечное (а конечна она, собственно говоря, как раз
вследствие ее отделения от действия) сама имеет причину, т.~е. она есть
также действие; следовательно, {\em то самое}, что
раньше было определено как причина, определено также и как действие; это
"--- {\em единство} причины и действия. Но определяемое
теперь как действие, опять-таки имеет некоторую причину, т.~е. причину
следует {\em отделить} от ее действия и положить как
отличное от него нечто. Эта новая причина сама однако есть только действие;
это "--- {\em единство} причины и действия. Она имеет
своей причиной некоторое другое; это
"--- {\em разъединение} сказанных двух определений,
и~т.~д. до {\em бесконечности}.

Этому прогрессу можно, таким образом, придать более своеобразную форму.
Выдвигается утверждение, что конечное и бесконечное суть одно единство; это
ложное утверждение должно быть исправлено противоположным утверждением: они
всецело разны и противоположны друг другу. Это утверждение должно быть
вновь исправлено утверждением о их единстве в том смысле, что они
неразделимы, что в одном определении заключено другое, и~т.~д., до
бесконечности. "--- Легко исполнимое требование, предъявляемое к тому, кто
хочет проникнуть в природу бесконечного, заключается в том, что он должен
сознавать, что бесконечный прогресс, развитое бесконечное рассудка, носит
характер {\em чередования} обоих определений,
{\em чередования единства} и
{\em раздельности} обоих моментов, а затем должен он
иметь дальнейшее сознание того, что это единство и эта раздельность сами
нераздельны.

Разрешением этого противоречия служит не признание
{\em одинаковой правильности} и одинаковой
неправильности обоих утверждений "--- это будет лишь другой формой остающегося
противоречия, "--- а {\em идеальность} обоих определений,
в каковой они в своем различии как взаимные отрицания суть лишь
{\em моменты;} вышеуказанное монотонное чередование
есть фактически отрицание как {\em единства}, так и
{\em раздельности} их. В нем (в этом чередовании)
фактически имеется также и показанное нами выше, а именно: конечное, выходя
за себя, впадает в бесконечное, но оно также и выходит за последнее,
находит себя порожденным снова, а, стало быть, сливается в этом выхождении
за себя лишь с самим собою, и это равным образом происходит и с
бесконечным, так что из этого отрицания отрицания получается
{\em утверждение}, каковой результат, стало быть,
оказывается их истиной и изначальным значением. Таким образом, в этом бытии
как {\em идеальности} отличных друг от друга
[определений] противоречие не исчезло абстрактно, а разрешено и примирено,
и мысли оказываются не только полными, но также и
{\em сведенными вместе}. Природа спекулятивного
мышления являет себя здесь, как на вполне развитом примере, в своем
определенном виде; она состоит единственно в схватывании противоположных
моментов в их единстве. Так как каждый из них являет себя в себе же, и
притом фактически, имеющим в самом себе свою противоположность и в ней
сливающимся с самим собою, то утвердительная истина есть это движущееся
внутри себя единство, объединение обеих мыслей, их бесконечность, "--- есть
соотношение с самим собою, не непосредственное, а бесконечное.

Многие, уже несколько более освоившиеся с философией, часто полагали
сущность и задачу философии в разрешении вопроса,
{\em каким образом бесконечное выходит из себя и
приходит к конечности}. Это, полагают они, не может быть сделано
{\em постижимым}. То бесконечное, к понятию которого мы
пришли, получит дальнейшие определения в ходе последующего изложения, и на
нем (на этом бесконечном) требуемое этими людьми будет показано во всем
многообразии форм, а именно, будет показано, {\em каким
образом} это бесконечное, если угодно так выражаться,
{\em приходит к конечности}. Здесь же мы рассматриваем
этот вопрос лишь в его непосредственности и имея в виду ранее рассмотренный
смысл, в котором обыкновенно понимают слово <<бесконечное>>.

От ответа на этот вопрос зависит, как утверждают, вообще решение вопроса,
{\em существует ли философия}, и, делая вид, что еще
видят в нем вопрос, ждущий своего разрешения, задающие его полагают вместе
с тем, что они обладают в самом этом вопросе некоторого рода каверзным
вопросом, неодолимым талисманом, служащим верной и обеспечивающей защитой
от утвердительного ответа и тем самым от философии и достижения ее. "--- И
относительно других предметов также требуется известное развитие для того,
чтобы уметь {\em задавать вопросы;} тем паче оно
требуется в отношении философских предметов, чтобы получить другой ответ,
чем тот, что вопрос никуда не годится.

При задавании таких вопросов взывают обыкновенно к чувству справедливости,
говоря, что дело не в том, какие употребляют слова, а что, выражают ли
вопрос так или этак, "--- все равно понятно, о чем идет речь. Употребление в
этом вопросе выражений, заимствованных из области чувственного
представления, как, например, <<выходить>> и~т.~п., возбуждает подозрение,
что он возник на почве обычного представления и что для ответа на него
также ожидают представлений, имеющих хождение в обыденной жизни, и образов
чувственной метафоры.

Если вместо бесконечного взять бытие вообще, то кажется, что легче постичь
процесс {\em определения бытия}, наличие в нем
некоторого отрицания или конечности. Хотя само бытие есть неопределенное, в
нем, однако, непосредственно не выражено, что оно есть противоположность
определенного. Напротив, бесконечное содержит эту мысль в явно выраженном
виде; оно есть {\em не}{}-конечное. Единство конечного
и бесконечного кажется, следовательно, непосредственно исключенным; поэтому
незавершенная рефлексия упорнее всего не приемлет этого единства. Но мы уже
показали, да и без дальнейшего углубления в определение конечного и
бесконечного непосредственно ясно, что бесконечное в том смысле, в котором
его берет сказанная незавершенная рефлексия, "--- а именно в смысле чего-то
противостоящего конечному, "--- как раз в силу того, что оно противостоит
последнему, имеет в нем свое другое и уже потому ограничено и само конечно,
есть дурное бесконечное. Поэтому ответ на вопрос,
{\em каким образом бесконечное становится конечным},
заключается в том, что {\em нет} такого бесконечного,
которое {\em сначала} бесконечно и которому только
потом приходится стать конечным, выйти к конечности, но что оно уже само по
себе столь же конечно, сколь и бесконечно. Так как вопрос принимает, что, с
одной стороны, бесконечное стоит особо и что, с другой стороны, конечное,
которое вышло из него, чтобы стать разлученным с ним, или которое, откуда
бы оно ни пришло, обособлено и отделено от него, "--- что такое конечное
поистине реально, то следовало бы скорее сказать, что
{\em непостижимым} является именно эта разлученность.
Ни такое конечное, ни такое бесконечное не имеют истинности, а неистинное
{\em непостижимо}. Но нужно также сказать, что они
{\em постижимы}. Рассмотрение их, даже взятых так, как
они даны в представлении, устанавливающее, что в каждом заключено
определение другого, простое усмотрение этой их нераздельности означает
постижение их: {\em эта нераздельность} есть
{\em их понятие}. "--- Напротив, принимая
{\em самостоятельность} вышесказанных конечного и
бесконечного, этот вопрос выставляет неистинное содержание и уже заключает
в себе неистинное соотношение между ними. На него поэтому не следует
отвечать, а следует, наоборот, отринуть содержащиеся в нем ложные
предпосылки, т.~е. следует отринуть самый вопрос. Вопрос об истинности
вышесказанных конечного и бесконечного изменяет точку зрения на них, и это
изменение переносит на самый первый вопрос то смущение, которое он хотел
вызвать. Наш {\em вопрос} оказывается чем-то
{\em новым} для рефлексии, являющейся источником
первого вопроса, так как в таком рефлектировании нет того спекулятивного
устремления, которое само по себе и прежде, чем соотносить между собой
определения, добивается познать, представляют ли из себя эти определения,
взятые так, как они предпосланы, нечто истинное. Но поскольку познана
неистинность вышеуказанного абстрактного бесконечного, а также и
неистинность долженствующего равным образом продолжать стоять на своей
стороне конечного, мы должны сказать относительно этого выхождения
конечного из бесконечного, что бесконечное
{\em выходит} к конечному потому, что оно, если его
понимают, как абстрактное единство, не имеет в себе истинности, не имеет
устойчивого существования, равно как и, наоборот, конечное
{\em входит} в бесконечное вследствие той же причины,
вследствие своей ничтожности. Или, правильнее будет сказать, что
бесконечное извечно выходит из себя к конечности, что его (точно так же,
как и чистого {\em бытия}) безоговорочно
{\em нет} самого по себе, без его другого {\em в нем же самом}.

Тот вопрос, каким образом бесконечное выходит из себя к конечному, может
содержать еще ту дальнейшую предпосылку, что бесконечное
{\em в~себе} включает в себя конечное и, стало быть,
есть в себе единство самого себя и своего другого, так что трудность
состоит по существу в их {\em разъединении}, которое
противоречит принятому в качестве предпосылки единству обоих. В этой
предпосылке та противоположность [обоих определений], за которую крепко
держатся, получает только другой вид; {\em единство} и
{\em различение} разлучаются и изолируются друг от
друга. Но если берут первое не как абстрактное, неопределенное единство, а
(как в указанной предпосылке) уже как определенное единство
{\em конечного} и {\em бесконечного}, то здесь уже имеется также и
различение обоих, "--- различение, которое, таким образом, вместе с тем не
есть предоставление им быть раздельно самостоятельными, а оставляет их в
единстве, как {\em идеализованные}. Это {\em единство} конечного и
бесконечного и их {\em различение} суть та же самая нераздельность,
что конечность и бесконечность.

\hegremark[Примечание 2]{Идеализм}{[Идеализм]}

Положение, гласящее, что {\em конечное идеализованно},
составляет {\em идеализм}. Философский идеализм состоит
не в чем другом, как в том, что конечное не признается истинно сущим.
Всякая философия есть по существу идеализм или по крайней мере имеет его
своим принципом, и вопрос затем заключается лишь в том, насколько этот
принцип действительно проведен. Философия есть столь же идеализм, как и
религия, ибо религия столь же мало признает конечность истинным бытием,
некоторым окончательным, абсолютным или, иначе говоря, некоторым
неположенным, несотворенным, вечным. Противоположение идеалистической и
реалистической философии не имеет поэтому никакого значения. Философия,
которая приписывала бы конечному существованию (Dasein) как таковому
истинное, последнее, абсолютное бытие, не заслуживала бы названия
философии. Первоначала древних или новых философских учений "--- вода или
материя или атомы "--- суть {\em мысли}, всеобщее,
идеализованное, а не вещи, как мы их непосредственно преднаходим, т.~е.
вещи в чувственной единичности; даже фалесовская вода тоже не есть такая
вещь; ибо, хотя она есть также и эмпирическая вода, она кроме того есть
вместе с тем <<{\em в~себе}>> или
{\em сущность} всех других вещей, и эти последние суть
не самостоятельные, не обоснованные внутри себя, а
{\em положены} проистекающими из другого, из
воды, т.~е. суть идеализованные. Назвав только что принцип, всеобщее,
{\em идеализованным}, как еще с большим правом должны
быть названы {\em идеализованными} понятие, идея, дух,
и, говоря затем, что единичные чувственные вещи в свою очередь имеют бытие
как {\em идеализованные} в принципе, в понятии, а еще
больше "--- в~духе, как снятые в~них, мы должны предварительно обратить
внимание читателя на ту же двусторонность, которая обнаружилась также и при
трактовании бесконечного, а именно, что {\em то}
идеализованным оказывается конкретное, истинно-сущее,
{\em то} идеализованным оказываются равным образом его
моменты в том смысле, что они сняты в нем; на самом же деле имеется только
единое конкретное целое, от которого моменты неотделимы.

Под идеализованным обыкновенно разумеют форму {\em представления} и
{\em идеализованным} называют то, что есть вообще
{\em в}~моем представлении или {\em в}~понятии, {\em в}~идее,
{\em в}~воображении и~т.~д., так что идеализованное
признается вообще также и фантазиями "--- представлениями, которые, как
предполагается, не только отличаются от реального, но по существу и
{\em не}~должны быть реальными. Дух в самом деле есть
вообще настоящий {\em идеалист;} в нем, уже как в
ощущающем и представляющем, а еще более, поскольку он мыслит и постигает в
понятиях, содержание имеет бытие не как так называемое
{\em реальное существование} (Dasein) "--- в~простоте <<я>>
такого рода внешнее бытие есть лишь снятое, оно есть {\em для меня}, оно
{\em идеализованно} во мне. Этот субъективный идеализм,
высказывается ли он и устанавливается как бессознательный идеализм сознания
вообще или сознательно как принцип, имеет в виду лишь ту
{\em форму} представления, по которой некоторое
содержание есть {\em мое} содержание. Систематический
субъективный идеализм утверждает относительно этой формы, что она есть
единственно истинная, исключающая форму объективности или реальности, форму
{\em внешнего существования} сказанного содержания.
Такой идеализм формален, так как он не обращает внимания на
{\em содержание} представления или мышления, каковое
содержание может при этом оставаться в представлении или мышлении всецело в
своей конечности. С принятием такого идеализма ничего не теряется, как
потому, что сохраняется реальность этого конечного содержания, наполненное
конечностью существование, так и потому, что, поскольку абстрагируются от
него, оно {\em само по себе} не должно иметь для нас
никакой важности; с принятием этого идеализма ничего также и не
выигрывается, именно потому, что ничего не теряется, так как <<я>>,
представление, дух остается наполненным тем же конечным содержанием.
Противоположность формы субъективности и объективности есть, разумеется,
один из видов конечности. Но {\em содержание}, как оно
воспринимается в ощущение, созерцание или же в более абстрактную стихию
представления, мышления, содержит в себе массу видов конечности, которые с
исключением лишь сказанного одного вида, формы субъективного и
объективного, еще совершенно не устранены и тем более не отпадают сами собою.

\bigskip


\clearpage

\chapter[\mdseries Примечания]{ПРИМЕЧАНИЯ}
<<Наука логики>> (т.~н. <<Большая логика>>) создана Гегелем в нюрнбергский
период его жизни. Первая ее часть (<<Объективная логика>>, кн.~1 "--- <<Учение
о бытии>>) вышла в начале 1812~г., вторая (<<Объективная логика>>, кн.~2 "---
<<Учение о сущности>>) "--- в 1813~г., третья (<<Субъективная логика>>, или
<<Учение о~понятии>>) "--- в 1816~г. В~1831~г. перед самой своей смертью Гегель
начал подготовку нового издания <<Науки логики>>. Он успел переработать лишь
первую ее часть (<<Учение о бытии>>), значительно расширив ее. В~таком
переработанном виде ата часть была издана Л.~фон Гепннигом уже после смерти
Гегеля в качестве III~тома Собрания ого сочинений~(1833). Две остальные части
составили IV и V тома Собрания сочинении и вышли в свет в 1834~г. Все эти три
части были переизданы в 1841~г. В~1923~г. Г.~Лассон выпустил новое,
<<научно-критическое>> издание. В~юбилейном издании Полного собрания сочинений
Гегеля, подготовленном Г.~Глокнером, <<Наука логики>>
составляет IV и~V тома~(1928).

<<Наука логики>> была переведена па ряд западноевропейских языков: итальянский
({\em Hegel}. La scienza della logica. Traduziono italiana con note di Arturo
Moni. Vol.~1---3. Bari, 1925), английский ({\em Hegel}. Science of Logic.
Translated by W.~H.~Johnston and L.~G.~Struthers. Vol.~1 and~2. London, 1929)
и~др.

На русский язык <<Наука логики>> была переведена многократно. Первый ее перевод
был осуществлен Н.~Г.~Дебольским (Пг., 1916; 2-е~изд. М., 1929) по изданию
1841~г. Перевод содержит большое количество ошибок, искажений и неточностей.
В~значительной мере это касается перевода важных философских терминов Гегеля.
В~издании в сущности отсутствует научный аппарат (в~нем даны лишь небольшой
алфавитный указатель и краткие объяснения "--- часто неправильные "---
некоторых философских терминов). К~1937~г. Институт философии АН~СССР
подготовил новое издание <<Науки логики>> в~переводе Б.~Г.~Столпнера,
составившее~V и~VI тома Сочинений Гегеля (т.~V "--- <<Учение о~бытии>> и
<<Учение о~сущности>>. М., 1937; т.~VI "--- <<Учение о~понятии>>. М., 1939).
Это издание снабжено обширным научным аппаратом "--- примечаниями и указателями
(составитель "--- В.~К.~Брушлинский, осуществивший также сверку перевода),
словарем основных терминов гегелевской логики, а~также библиографией изданий
<<Науки логики>> и списком литературы о~ней. Переводчик и редактор проделали
большую работу, благодаря которой удалось точнее передать мысль Гегеля, хотя
иногда и в~ущерб литературной форме изложения. Этот перевод был взят за основу
при подготовке настоящего издания.

\bigskip

\begin{center}
\ding{93}~~~\ding{93}~~~\ding{93}
\end{center}

\bigskip

\begin{footnotesize}
\printpagenotes
\end{footnotesize}


\clearpage
\tableofcontents*
\clearpage

\end{document}
