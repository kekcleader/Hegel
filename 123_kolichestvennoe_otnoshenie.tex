Бесконечность определённого количества была определена выше так, что она
есть его отрицательное потустороннее, которое оно, однако, имеет в самом
себе. Это потустороннее есть качественное вообще. Бесконечное определённое
количество как единство обоих моментов "--- количественной и качественной
определённостей "--- есть ближайшим образом {\em отношение}.

В отношении определённое количество уже более не обладает лишь безразличной
определённостью, а качественно определено как безоговорочно соотнесённое со
своим потусторонним. Оно продолжает себя в своё потустороннее; последнее есть
ближайшим образом некоторое {\em другое} определённое количество вообще. Но по
существу они соотнесены друг с другом не как внешние определённые количества, а
{\em каждое имеет свою определённость в этом соотношении с другим}. Они, таким
образом, в этом своём инобытии возвратились в~себя; то,~чт\'{о} каждое из них
есть, оно есть в другом; другое составляет определённость каждого из них. "---
Выхождение определённого количества за себя теперь уже, стало быть, не имеет ни
того смысла, что оно изменяется лишь в некоторое другое, ни того, что оно
изменяется в своё абстрактное другое, в своё отрицательное потустороннее,
а~имеет тот смысл, что в этом другом оно достигает своей определённости; оно
находит {\em самого себя} в своём потустороннем, которое есть некоторое другое
определённое количество. {\em Качество} определённого количества,
определённость его понятия заключается вообще в~том, что оно внешне, и вот
теперь, в отношении, оно {\em положено} так, что оно имеет свою определённость
в своей внешности, в некотором другом определённом количестве, есть в~своём
потустороннем то, чт\'{о} оно есть.

Тем соотношением между собой, которое здесь получилось, обладают именно
определённые количества. Это {\em соотношение} само есть также некоторая
величина. Определённое количество не только находится в {\em отношении,} но
{\em оно само положено как отношение;} оно есть {\em некоторое} определённое
количество вообще, имеющее указанную качественную определённость
{\em внутри себя}. Таким образом, как отношение оно выражает себя, как
замкнутую в себе целостность, и своё безразличие к границе тем, что оно имеет
внешность своей определённости внутри самого себя, и в этой внешности
соотнесено лишь с собою, и, следовательно, бесконечно в самом себе.

Отношение вообще есть:

1) {\em прямое} отношение. В~нём {\em качественное} ещё не выступает наружу как
таковое, само по себе. Оно положено здесь пока что
исключительно в виде (Weise) определённого количества, положено имеющим свою
определённость в сам\'{о}й своей внешности. "--- Количественное отношение есть
в~себе противоречие между внешностью и соотношенем с самим собою, между устойчивым существованием
определённых количеств и отрицанием их. Это противоречие снимает себя, поскольку
ближайшим образом

2) в~{\em непрямом} отношении полагается {\em отрицание} одного определённого
количества как таковое также при изменении другого и изменчивость самог\'{о}
прямого отношения;

3) в~{\em степенн\'{о}м} же {\em отношении} соотносящаяся в~своём различии
с~самой собою единица выступает как простое самопродуцирование определённого
количества. И,~наконец, само это качественное, положенное в простом определении
и как тождественное с определённым количеством, становится {\em мерой}.

О~природе излагаемых ниже отношений многое уже было сказано наперёд
в~предшествующих примечаниях, касающихся бесконечного в количестве, т.~е.
качественного момента в последнем; теперь остаётся поэтому лишь разъяснить
абстрактное понятие этих отношений.

\subsection{Прямое отношение}

1. В~отношении, которое как непосредственное есть {\em прямое} отношение,
определённость одного определённого количества заключается в определённости
другого определённого количества, и это взаимно. Имеется лишь {\em одна}
определённость или граница обоих, которая сама есть определённое количество
"--- {\em показатель} отношения.

2. Показатель есть какое-нибудь определённое количество. Но он есть в~своей
{\em внешности} соотносящееся с {\em собою} в самом себе
качественно-определённое количество лишь постольку, поскольку он в~нём самом
имеет отличие от себя, своё потустороннее и инобытие. Но это различие
определённого количества в нём {\em самом} есть различие {\em единицы}
и~{\em численности;} единица есть самостоятельная определённость
(Für-sich-bestimmt\-sein); численность же "--- безразличное движение туда и
сюда вдоль определённости, внешнее безразличие определённого количества.
Единица и~численность были первоначально моментами определённого количества;
теперь в~отношении, которое постольку есть реализованное определённое
количество, каждый из его моментов выступает как некоторое особое
{\em определённое количество,} и оба они "--- как определения его наличного
бытия, как ограничения по отношению к определённости величины, которая помимо
этого есть лишь внешняя, безразличная определённость.

Показатель есть это различие как простая определённость, т.~е. он имеет
непосредственно в самом себе значение обоих определений. Он есть, {\em
во-первых,} определённое количество; в этом смысле он есть численность; если
один из членов отношения, принимаемый за единицу, выражается нумерической
единицей "--- а ведь он считается лишь таковой единицей, "--- то другой член,
численность, есть определённое количество самого показателя. {\em Во-вторых,}
показатель есть простая определённость как качественное в членах отношения;
если определённое количество одного из членов определено, то и другое
определённое количество определено показателем, и совершенно безразлично, как
определяется первое; оно, как определённое само по себе определённое
количество, уже более не имеет никакого значения и может быть также и любым
другим определённым количеством, не изменяя этим определённости отношения,
которая покоится исключительно на показателе. Одно определённое количество,
принимаемое за единицу, как бы велико оно ни стало, всегда остаётся единицей,
а~другое определённое количество, как бы велико оно при этом также ни стало,
непременно должно оставаться {\em одной и той же} численностью указанной
единицы.

3. Согласно этому оба они составляют, собственно говоря, лишь {\em одно}
определённое количество; одно определённое количество имеет по отношению
к~другому лишь значение единицы, а не численности; другое имеет лишь значение
численности; стала быть, {\em по определённости своего понятия} сами они
{\em не}~являются {\em полными} определёнными количествами. Но эта неполнота
есть отрицание в них и притом отрицание не со стороны изменчивости вообще, по
которой одно (а~каждое из них есть одно из двух) может принимать всевозможные
величины, а со стороны того определения, что если одно изменяется, то и другое
настолько же увеличивается или уменьшается; это, как мы показали, означает:
лишь {\em одно,} единица, изменяется как определённое количество, другой же
член, численность, остаётся тем же определённым количеством {\em единиц,} но и
первый член также лишь сохраняет {\em значение} единицы, как бы он ни изменялся
как определённое количество. Каждый член есть, таким образом, лишь один из этих
двух моментов определённого количества, и самостоятельность, требующаяся для
его своеобразия, {\em подверглась} в себе {\em отрицанию;} в этой качественной
связи они должны быть {\em положены} один по отношению к другому как
{\em отрицательные}.

Показатель, по вышесказанному, есть полное определённое количество, так как
в~нём сходятся определения {\em обоих} членов отношения; но на самом деле он
как частное сам имеет значение только либо {\em численности,} либо
{\em единицы}. Нет никакого указания (Bestimmung), какой из членов отношения
должен быть принимаем за единицу и какой за численность; если один из них,
определённое количество $B$, измеряется определённым количеством $A$ как
единицей, то частное $C$ есть численность таких единиц; но если принять само
$A$ за численность, то частное $C$ есть единица, требуемая при численности $A$
для определённого количества $B${\em ;\,} тем самым это частное как показатель
положено не как то, чем оно должно быть, "--- не как то, чт\'{о} определяет
отношение или как его качественная единица. Как последняя оно положено лишь
постольку, поскольку оно имеет значение {\em единства обоих моментов,} единицы
и численности. Так как эти члены отношения, хотя они и даны как определённые
количества такими, какими они должны быть в развёрнутом определённом
количестве, в отношении, всё же при этом даны лишь в том значении, которое они
должны иметь как его члены, т.~е. суть {\em неполные} определённые количества
и~считаются лишь за один из указанных качественных моментов, то они должны быть
положены с этим их отрицанием; благодаря этому возникает более соответствующее
его определению, более реальное отношение, в котором показатель имеет значение
произведения сторон отношения; согласно этому определению оно есть
{\em обратное отношение}.

\subsection{Обратное отношение}

1. Отношение, как оно получилось теперь, есть {\em снятое} прямое отношение;
оно было {\em непосредственным} и, стало быть, ещё не истинно определённым;
теперь же определённость привзошла к нему так, что показатель считается
произведением, единством единицы и численности. Со стороны его
непосредственности его можно было (как было показано выше) принимать
безразлично и за единицу и за численность, вследствие чего он и был лишь
определённым количествам вообще и, стало быть, преимущественно численностью;
одна сторона была единицей, и её следовало принимать за одно, а другая сторона
была её неизменной численностью, которая вместе с тем была и показателем;
качество последнего состояло, следовательно, лишь в том, что это определённое
количество принималось за неизменное или, вернее, неизменное понималось лишь в
смысле определённого количества.

В обратном же отношении показатель как определённое количество равным образом
есть некое непосредственное и нечто, принимаемое за неизменное. Но это
определённое количество не есть {\em неизменная численность} по отношению к
другому члену {\em отношения,} принимаемому за {\em единицу;} это в
предшествующем неизменное отношение теперь скорее, наоборот, положено как
изменчивое; когда в качестве одного из членов обратного отношения берут
какое-нибудь другое определённое количество, то другой член отношения уже более
не остаётся {\em той же самой численностью} единиц первого члена. В~прямом
отношении эта единица есть лишь общее обоих членов; она как таковая
продолжается в другой член, в численность; сама численность, взятая особо, или,
иначе говоря, показатель, безразлична к единице.

Но при той определённости отношения, какую мы имеем теперь, численность как
таковая изменяется по отношению к единице, относительно которой она составляет
другую сторону отношения; если мы принимаем за {\em единицу} какое-нибудь
другое определённое количество, то численность становится другой. Поэтому, хотя
показатель также и здесь есть лишь некоторое непосредственное, лишь произвольно
принимаемое за неизменное определённое количество, но он не сохраняется как
таковое в стороне отношения, и эта сторона, а тем самым и прямое отношение
сторон изменчиво. Поэтому в рассматриваемом теперь отношении показатель, как
определяющее определённое количество, положен отрицательным по отношению к~себе
как к определённому количеству отношения, положен тем самым как качественный,
как граница, так что, следовательно, качественное выступает особо, отличным от
количественного. "--- В~прямом отношении {\em изменение} обоих членов есть лишь
{\em одно} изменение определённого количества, каковым принимается единица,
представляющая собою общее обеих сторон отношения, и, следовательно, во сколько
раз одна сторона увеличивается или уменьшается, во столько же раз увеличивается
или уменьшается также и другая; само отношение безразлично к этому изменению;
последнее внешне ему. В~обратном же отношении изменение, хотя оно по
безразличному количественному моменту также произвольно, удерживается
{\em внутри отношения,} и также и это произвольное количественное выхождение
подвергается ограничению отрицательной определённостью показателя как
некоторой границей.

2. Следует рассмотреть эту качественную природу обратного отношения ещё ближе,
а~именно в~её реализации, и разъяснить содержащуюся в ней переплетенность
утвердительного с отрицательным. "--- Определённое количество положено здесь
как то, чт\'{о} качественно определяет определённое количество, т.~е. само
себя, как представляющее себя в нём [в~самом себе] своей границей. Тем самым
оно есть, {\em во-первых,} некоторая непосредственная величина как
{\em простая} определённость, {\em целое} как {\em сущее,} утвердительное
определённое количество. Но, {\em во-вторых,} эта непосредственная
{\em определённость} есть вместе с~тем {\em граница;} для этого оно различено
на два определённых количества, которые ближайшим образом суть другие
относительно друг друга; но как их качественная определённость, и притом эта
определённость как полная, оно есть единство единицы и численности,
произведение, множителями которого они служат. Таким образом, показатель их
отношения, с одной стороны, тождественен в~них с собою и есть то их
утвердительное, благодаря чему они суть определённые количества; с~другой
стороны, он, как положенное в них отрицание, есть в~них то {\em единство,}
вследствие которого каждое, представляющее собою ближайшим образом некоторое
непосредственное, ограниченное определённое количество вообще, вместе с тем
есть такое ограниченное, что оно только {\em в~себе тождественно} со своим
другим. {\em В-третьих,} он как простая определённость есть отрицательное
единство этого своего различения на два определённых количества и граница их
взаимного ограничения.

Согласно этим определениям оба момента {\em ограничивают} друг друга внутри
показателя, и один момент есть отрицательное другого, так как показатель есть
их определённое единство; один момент становится во столько раз меньше, во
сколько другой становится больше; каждый имеет свою величину постольку,
поскольку он заключает в себе величину другого, именно ту величину, которая
недостаёт другому. Каждая величина продолжает себя, таким образом,
{\em отрицательно} в другую; сколько численности есть в~ней, столько она
устраняет в другой как численности, и она есть то, чт\'{о} она есть, только
через отрицание или границу, которая полагается в ней другою. Каждая, таким
образом, {\em содержит} в себе также и другую и измеряется ею, ибо каждая
должна быть только тем определённым количеством, которым не является другая;
для значения каждой из них величина другой необходима и, стало быть, от неё
неотделима.

Эта непрерывность каждой в другой составляет момент единства, благодаря
которому они находятся в отношении "--- момент {\em единой} определённости,
простой границы, которая есть показатель. Это единство, целое, образует
{\em в-себе-бытие} каждой из сторон отношения, от какового в-себе-бытия отлична
её~{\em наличная} величина, по которой каждая сторона есть лишь постольку,
поскольку она отнимает у другой часть их общего в-себе-бытия "--- целого. Но
она может отнять у другой лишь столько, сколько нужно для того, чтобы сделать
себя равной этому в-себе-бытию. Она имеет свой максимум в показателе, который
по указанному второму определению есть граница их взаимного ограничения. А~так
как каждая есть момент отношения лишь постольку, поскольку она ограничивает
другую и, стало быть, ограничивается другою, то, делаясь равною своему
в-себе-бытию, она утрачивает это своё определение; при этом не только другая
величина становится нулём\pagenote{Здесь слово <<нуль>> употребляется Гегелем в
фигуральном смысле "--- в том смысле, что сторона обратного отношения перестаёт
быть стороной отношения, если она становится равной показателю.
В~математическом же смысле, если мы возьмём обратное отношение, показателем
которого является произведение членов отношения ($xy\hm=C$), и приравняем один
из членов отношения этому произведению (например, $x\hm=C$), то другой член
отношения будет не нулём, а единицей ($y\hm=1$). В~арифметическом обратном
отношении (о~котором здесь у Гегеля ещё нет речи и формулой которого является
$x\hm+y\hm=C$), действительно, если $x\hm=C$, то $y\hm=0$.}, но и она сама
исчезает, так как она, согласно предположению, есть не голое определённое
количество, а она должна быть тем, чт\'{о} она как таковое есть лишь как
такого рода момент отношения. Таким образом, каждая сторона отношения есть
противоречие между определением её как её в-себе-бытия, т.~е. единства того
целого, которым служит показатель, и определением её как момента отношения; это
противоречие есть {\em бесконечность,} снова появившаяся в новой, своеобразной
форме.

Показатель есть {\em граница} членов своего отношения, внутри которой они
обратно друг другу увеличиваются и уменьшаются, причём они не могут стать
равными показателю по той утвердительной определённости, которая свойственна
ему как определённому количеству. Таким образом, как граница их взаимного
ограничения, он есть $\alpha$)~их {\em потустороннее,} к которому они могут
{\em бесконечно} приближаться, но которого они не могут достигнуть. Эта
бесконечность, с которой они к нему приближаются, есть дурная бесконечность
бесконечного прогресса; она сама конечна, имеет свой предел в своей
противоположности, в конечности каждого члена и самого показателя, и есть
поэтому лишь {\em приближение}. Но $\beta$)~дурная бесконечность вместе
с тем здесь {\em положена} как то, чт\'{о} она есть {\em поистине,} а именно,
лишь как {\em отрицательный момент} вообще, по которому показатель есть
относительно различенных определённых количеств отношения {\em простая граница}
как в-себе-бытие, с которым соотносят их конечность как безоговорочно
изменчивое, но которое как их отрицание остаётся безоговорочно отличным от них.
Это бесконечное, к которому они могут лишь приближаться, в таком случае
наличествует также и как {\em утвердительное посюстороннее;} это "--- простое
определённое количество показателя. В~показателе достигнута та потусторонность,
которой обременены стороны отношения; он есть {\em в~себе} единство обеих или
тем самым он есть в себе другая сторона каждой из них; ибо каждая имеет лишь
столько величины, сколько её не имеет другая, вся её определённость лежит,
таким образом, в другой, и это её в-себе-бытие есть как утвердительная
бесконечность просто показатель.

3. Но тем самым получился переход обратного отношения в некоторое другое
определение, чем то, которым оно первоначально обладало. Последнее состояло
в~том, что некоторое определённое количество как непосредственное вместе с~тем
имеет то соотношение с другим, что становится тем больше, чем последнее
становится меньше, есть то, чт\'{о} оно есть, лишь через отрицательное
отношение к~другому; и равным образом состояло в том, что некоторая третья
величина есть общий предел этого их увеличения. Это изменение, в
противоположность к качественному как {\em твёрдой, неизменной} границе,
составляет здесь их своеобразие; они имеют определение {\em переменных}
величин, для которых то неизменное есть некоторое бесконечное потустороннее.

Но определения, которые обнаружились перед нами и которые мы должны свести
воедино, заключаются не только в том, что это бесконечное потустороннее есть
вместе с тем некоторое имеющееся налицо и какое-нибудь конечное определённое
количество, а и в том, что его неизменность "--- вследствие которой оно есть
такое бесконечное потустороннее по отношению к количественному и которая есть
качественная сторона бытия лишь как абстрактное соотношение с самою собою "---
развилась в опосредствование себя с самим собою в своём другом, в конечных
членах отношения. Всеобщий момент этих определений заключается в том, что
вообще целое как показатель есть граница взаимного ограничения обоих членов,
что, стало быть, положено {\em отрицание отрицания,} а~тем самым бесконечность,
{\em утвердительное} отношение к самому себе. Более определённый момент
заключается в том, что {\em в~себе} показатель как произведение уже есть
единство единицы и численности, а каждый из обоих членов отношения есть лишь
один из этих двух моментов, благодаря чему показатель, следовательно, включает
их в себя и {\em в~себе} соотносится в них с самим собою. Но в обратном
отношении различие развилось в характеризующую количественное бытие
{\em внешность} и качественное дано не только как неизменное, а также не только
как лишь непосредственно включающее в себя моменты, а как смыкающееся
{\em с~собою} в {\em вовне-себя-сущем инобытии}. Это определение и выделяется
как результат в обнаружившихся доселе моментах. А~именно, показатель
оказывается в-себе-бытием, моменты которого реализованы в определённых
количествах и в их изменчивости вообще. Безразличие их величин в их изменении
представляется в виде бесконечного прогресса; в~основании этого лежит то, что
в~их безразличии их определённость как раз и состоит в том, чтобы иметь свою
величину в величине другого и, стало быть, $\alpha$)~по утвердительной стороне
их определённого количества быть {\em в~себе} полным показателем. И~точно так
же они имеют $\beta$)~своим отрицательным моментом, своим взаимным
ограничиванием величину показателя; их граница есть его граница. То
обстоятельство, что они уже больше не имеют никакой другой имманентной границы,
никакой\pagenote{В~немецком тексте вместо <<keine>> (никакой) стоит <<eine>>.
По-видимому, это опечатка.} твёрдой непосредственности, положено в бесконечном
прогрессе их наличного бытия и их ограничения, в отрицании всякой особенной
величины. Это отрицание есть согласно этому {\em отрицание} того вне-себя-бытия
показателя, которое изображено в них, и он, т.~е. тот, который сам вместе с~тем
представляет собой некоторое определённое количество вообще и также и развёрнут
в определённые количества, тем самым положен, как сохраняющийся, сливающийся
с~собою в отрицании их безразличного существования, положен, таким образом, как
определяющий это выхождение за себя.

Отношение определилось таким образом в {\em степенн\'{о}е отношение}.

\subsection{Степенн\'{о}е отношение}

1. Определённое количество, полагающее себя в своём инобытии тождественным
с~собою, определяющее само своё выхождение за себя, достигло для-себя-бытия.
Таким образом оно представляет собой некоторую качественную целокупность,
которая, поскольку она полагает себя как развёрнутую, имеет своими моментами
определения понятия числа "--- единицу и численность; в обратном отношении
численность есть некоторое такое множество, которое ещё не определено самой
единицей как таковою, а определено откуда-то извне, некоторым третьим; теперь
же численность положена как определяемая лишь ею же. Это происходит
в~степенн\'{о}м отношении, где единица, которая сама по себе есть численность,
есть вместе с тем численность в отношении себя как единицы. Инобытие,
численность единиц, есть самая же единица. Степень есть некоторое множество
единиц, каждая из которых есть самое это множество. Определённое количество как
безразличная определённость изменяется; но поскольку это изменение есть
возведение в степень, это его инобытие ограничено исключительно самим собою.
"--- Таким образом, определённое количество положено в степени, как
возвратившееся в себя само; оно непосредственно есть оно само и также
и своё инобытие.

{\em Показатель} этого отношения уже более не есть некоторое непосредственное
определённое количество, как в прямом, а также и в обратном отношении. Он имеет
в степенн\'{о}м отношении совершенно {\em качественную} природу, есть та
{\em простая} определённость, что численность есть сама\'{я} же единица и что
определённое количество {\em тождественно} в своём инобытии с самим собою.
В~этом обстоятельстве заключается вместе с тем та сторона его
{\em количественной} природы, что граница или отрицание не положена как
непосредственно сущее, а наличное бытие положено, как продолженное в своё
инобытие; ибо истина качества заключается именно в~том, что оно есть
количество, непосредственная определённость как снятая.

2. Степенн\'{о}е отношение представляется сначала некоторым внешним изменением,
которому подвергают какое-нибудь определённое количество; но оно имеет ту более
тесную связь с~{\em понятием} определённого количества, что последнее в~том
наличном бытии, до которого оно развилось в указанном отношении, достигло этого
понятия, полностью реализовало его; это отношение есть изображение того,
чт\'{о} определённое количество есть {\em в~себе,} и выражает ту его
определённость или то {\em качество,} которым оно отличается от другого.
Определённое количество есть {\em безразличная, положенная} как {\em снятая,}
определённость, т.~е. определённость как граница, которая также и не есть
граница, продолжается в своё инобытие, остаётся, следовательно, в нём
тождественной с самой собой; таким оно {\em положено} в степенн\'{о}м
отношении; его инобытие, выхождение за само себя в некоторое другое
определённое количество, определено им же самим.

Сравнивая между собой этапы этой реализации в рассмотренных доселе отношениях,
мы видим, что качество определённого количества, заключающееся в том, что оно
положено как своё собственное отличие от самого себя, состоит вообще в том,
чтобы быть отношением. Как прямое отношение, оно есть таковое положенное
различие пока что лишь вообще или непосредственно, так что его соотношение с
самим собою, которое оно, как показатель, имеет относительно своих различий,
признаётся лишь неизменностью некоторой численности единиц. В~обратном
отношении определённое количество есть в отрицательном определении некоторое
своё отношение к себе самому, к себе, как к своему отрицанию, в котором оно,
однако, имеет своё численное значение; как утвердительное соотношение с собою,
оно есть такой показатель, который как определённое количество есть
определяющий свои моменты лишь {\em в~себе}. В~степенн\'{о}м же отношении оно
наличествует в различии {\em как различии себя от самого себя}. {\em Внешность}
определённости есть качество определённого количества: теперь эта внешность
положена, таким образом, соответственно его понятию, как его собственный
процесс определения, {\em как} его соотношение с самим собою, его
{\em качество}.

3. Но тем, что определённое количество {\em положено} так, как оно
соответствует своему понятию, оно перешло в другое определение или, как это
можно также выразить, его {\em определение} теперь дано (ist) также и как
{\em определённость,} его {\em в-себе-бытие} дано (ist) также и как
{\em наличное бытие}. Оно есть {\em определённое количество,} поскольку
внешность или безразличие к тому, как оно определено (то обстоятельство, что
оно есть то, чт\'{о}, как говорится, может быть увеличено или уменьшено),
значимо и положено лишь {\em просто} или, иначе говоря, {\em непосредственно;}
оно стало своим другим, т.~е. качеством, поскольку указанная внешность теперь
положена, как опосредствованная через него самого, положена как момент так, что
оно именно {\em в~ней же соотносится с самим собой,} есть бытие как качество.

Итак, первоначально количество как таковое выступает как нечто противостоящее
качеству. Но само количество есть {\em некоторое} качество, соотносящаяся
с собою определённость вообще, отличённая от другой для неё определённости,
от качества как такового. Однако, оно не только есть {\em некоторое} качество,
а истина самого качества есть количество; качество явило себя переходящим
в количество. И~обратно: количество в своей истине есть возвратившаяся в себя
самое, небезразличная внешность. Таким образом, оно есть само качество, так что
качество как таковое не есть ещё что-то помимо этого определения. "--- Для того
чтобы была {\em положена} целокупность как таковая, требуется {\em двойной}
переход, не только переход одной определённости в её другую, но также и переход
этой другой, возвращение её в первую. Через первый переход тождество этих двух
определённостей имеется пока что лишь {\em в~себе;} "--- качество содержится
в количестве, которое однако вместе с тем есть пока что ещё односторонняя
определённость. Что последняя, наоборот, также содержится в первой, что она
также есть лишь снятая, это получается во втором переходе, "--- в~её
возвращении в первую. Это замечание о необходимости {\em двойного} перехода
имеет большую важность для всего научного метода.

Определённое количество теперь уже не как безразличное или внешнее определение,
а так, что оно как такое определение снято и есть качество и то, благодаря чему
нечто есть то, чт\'{о} оно есть, "--- это истина определённого количества,
{\em мера}\pagenote{Этот абзац переведён Б.~Г.~Столпнером (стр.~377)
в~соответствии со вторым немецким изданием. В~настоящем томе перевод даётся
по изданию Лассона в соответствии с первым немецким изданием.}.

\setsecnumdepth{subsection}
\subremark{}

Выше, в примечаниях о количественно бесконечном, было разъяснено, что
последнее, равно как и трудности, возникающие относительно него, имеют своё
происхождение в {\em качественном} моменте, обнаруживающемся в количественном,
и [разъяснено также], каким образом особенно качественная сторона
степенн\'{о}го отношения получает многообразное развитие и становится
усложнённой; как на основной недостаток, служащий
помехой усвоению понятия, было указано на то, что при рассмотрении бесконечного
останавливаются только на отрицательном [его] определении, на том, что оно есть
отрицание определённого количества, и не идут дальше, не устанавливают того
простого, утвердительного определения, что оно есть качественное. "--- Здесь
нам остаётся сделать ещё одно замечание о происходившем в философии
примешивании форм количественного к чистым качественным формам мышления.
С~особенным усердием применяли в новейшее время
к~{\em определениям понятия степенн\'{ы}е отношения}\pagenote{Гегель имеет
в~виду философию Шеллинга.}. Понятие в своей непосредственности было названо
{\em первой} степенью, понятие в своём инобытии или различии, в существовании
его моментов "--- {\em второй,} а понятие в своём возвращении в себя или, иначе
говоря, понятие как целостность "--- {\em третьей} степенью. "--- Как
возражение против этого сразу приходит в голову, что категория <<степень>>,
употребляемая таким образом, есть категория, существенно принадлежащая области
определённого количества; говорившие об этих Potenzen не имели в виду potentia,
\textgreek{δόναμις} Аристотеля (Potenz по-немецки означает и степень и
возможность, потенцию. "--- {\em Перев.}). Таким образом, степенн\'{о}е
отношение выражает определённость как различие, взятое так, как оно есть
в~{\em особенном понятии} определённого количества, выражает, как это различие
достигает своей истины, но не выражает его, взятого так, как оно есть в понятии
как таковом. Определённое количество содержит в себе отрицательность,
принадлежащую к природе понятия, ещё вовсе не как положенную в своеобразном
определении последнего; различия, присущие определённому количеству, суть
поверхностные определения для самого понятия; они ещё весьма далеки от того,
чтобы быть определёнными так, как они определены в понятии. Как раз в детском
периоде философствования числа "--- а~первая, вторая и~т.~д. степень не имеют
в~этом отношении никакого преимущества перед числами "--- употреблялись,
например, {\em Пифагором} для обозначения всеобщих существенных различий. Это
было подготовительной ступенью к чистому, мыслительному пониманию; лишь после
Пифагора были изобретены, т.~е. были осознаны {\em особо} сами определения
мысли. Но возвращаться от последних назад к числовым определениям "--- это
свойственно чувствующему себя бессильным мышлению, которое в противоположность
существующей философской культуре, привыкшей к определениям мысли, прибавляет
к~своему бессилию прямо-таки смешное желание выдавать эту слабость за нечто
новое, возвышенное и за прогресс.

Поскольку выражение понятий через степени применяется лишь как {\em символ,}
против этого приходится столь же мало возражать, как против употребления чисел
или другого рода символов для выражения понятия; но вместе с тем против этого
приходится возражать столь же много, как против всякой символики вообще, при
помощи которой нам предлагают изображать чистые понятийные или, иначе говоря,
философские определения. Философия не нуждается в такой помощи, не нуждается ни
в помощи, приходящей из области чувственного мира, ни в помощи, приходящей от
представляющей силы воображения, ни даже в помощи из находящихся на её
собственной почве подчинённых сфер, определения которых вследствие этой
подчинённости не подходят для более высоких её кругов и для целого. Последнее
происходит вообще в тех случаях, когда применяют категории конечного к
бесконечному; привычные определения силы или субстанциальности, причины и
действия и~т.~п. равным образом суть лишь символы для выражения, например,
жизненных или духовных отношений, т.~е. суть неистинные определения
применительно к последним, а тем паче это справедливо о применении степеней
определённого количества и числовых степеней к таким и вообще к спекулятивным
отношениям. "--- Если хотят употреблять числа, степени,
математически-бесконечное и тому подобное не как символы, а как формы для
философских определений и, стало-быть, как философские формы, то следовало бы
прежде всего вскрыть их философское значение, т.~е. их понятийную
определённость. А~если это сделают, то они сами окажутся излишними
обозначениями; определённость понятия сама себя обозначает, и её обозначение
является единственно правильным и подходящим. Употребление указанных форм
представляет собою поэтому не~что иное, как только удобное средство избавить
себя от труда понимания, указания и оправдания определений понятия.

\bigskip
