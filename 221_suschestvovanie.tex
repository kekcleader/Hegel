\clearpage
\part[{\em Второй отдел} Явление]{Второй отдел. Явление}

{\em Сущность должна являться.}
Бытие есть абсолютная абстракция; эта отрицательность есть для него не некое
внешнее, а оно есть бытие и ничего другого, кроме бытия, есть лишь эта
абсолютная отрицательность. Из-за этой отрицательности бытие дано (ist)
лишь как снимающее себя бытие и есть {\em сущность}. Но
и обратно, сущность как простое равенство с собою есть равным образом
{\em бытие}. Учение о бытии содержит в себе первое
предложение: {\em бытие есть сущность}. Второе
предложение: {\em сущность есть бытие}, составляет
содержание первого отдела учения о сущности. Но это бытие, которым делает
себя сущность, есть {\em существенное бытие,}
{\em существование}, состоявшийся выход из
отрицательности и внутренности.

Таким образом, сущность {\em является}. Рефлексия есть
{\em свечение} сущности {\em внутри
ее самой}, излучение ею видимости внутри ее самой. Определения рефлексии
замкнуты в единство, всецело лишь как положенные, снятые; или, иначе
говоря, она есть сущность, непосредственно тождественная с собой в своей
положенности. Но поскольку сущность есть основание, она определяет себя
реально через свою снимающую самое себя или возвращающуюся в себя
рефлексию; далее, так как это определение или инобытие соотношения
основания снимается в рефлексии основания и становится существованием, то
определения формы приобретают тем самым элемент самостоятельного
устойчивого наличия. Их {\em видимость}
совершенствуется до того, что становится {\em явлением}.

Сущность, достигшая непосредственности, есть
{\em ближайшим образом существование}, а как
неразличенное единство сущности с ее непосредственностью, она есть
существующее или {\em вещь}. Вещь, правда, содержит в
себе рефлексию, но отрицательность рефлексии ближайшим образом угасла в
непосредственности вещи; однако, так как основание вещи есть по существу
рефлексия, то непосредственность вещи снимается; вещь делает себя некоторой
положенностью.

Таким образом, она есть, {\em во-вторых},
{\em явление}. Явление есть то, что вещь есть в себе,
или истина последней. Но это лишь положенное, рефлектированное в инобытие
существование есть равным образом и выход за себя, за существование, взятое
в его бесконечности; миру явления противостоит рефлектированный в себя,
{\em сущий в себе мир}.

Но являющееся и существенное бытие безоговорочно соотнесены друг с другом.
Таким образом, существование есть, {\em в-третьих},
существенное {\em отношение}; являющееся обнаруживает
существенное, и последнее имеет бытие в своем явлении. "--- Отношение есть еще
неполное соединение рефлексии в инобытие и рефлексии в себя; полное
взаимопроникновение {\em обеих} есть {\em действительность}.

\chapter[{\em Первая глава} Существование]{Первая глава Существование}

Подобно тому, как предложение основания гласит:
{\em все, что есть, имеет некоторое основание}, или,
иными словами, {\em есть нечто положенное,
опосредствованное}, следовало бы также выставить предложение о
существовании и выразить его следующим образом:
{\em все, что есть, существует}. Истина бытия состоит
не в некотором первом непосредственном, а в том, что оно есть перешедшая в
непосредственность сущность.

Но, далее, если сказано было также: {\em все, что
существует, имеет основание и обусловлено}, то следовало бы в такой же мере
сказать: {\em оно не имеет основания и безусловно}. Ибо
существование есть непосредственность, возникшая из снятия того
опосредствования, которое соотносит свои термины при помощи категорий
основания и условия, "--- непосредственность, которая в своем возникновении
снимает самое это возникновение.

Поскольку здесь можно упомянуть доказательства о
{\em существовании} бога, то следует наперед указать,
что кроме непосредственного {\em бытия}, во-первых, и,
во-вторых, {\em существования}, бытия, возникающего из
сущности, есть еще, далее, бытие, возникающее из понятия, "---
{\em объективность}. "--- Процесс доказательства есть
вообще {\em опосредствованное познание}. Разные виды
бытия требуют или содержат в себе свои особые виды опосредствования;
поэтому и природа процесса доказательства по отношению к каждому из них
также различна. {\em Онтологическое доказательство}
хочет исходить из понятия; оно кладет в основание совокупность всех
реальностей, а затем подводит существование под понятие реальности. Таким
образом, это доказательство представляет собою опосредствование, имеющее
характер умозаключения и здесь еще не подлежащее рассмотрению. Уже
выше\pagenote{См. <<Учение о бытии>>,
стр.~\pageref{bkm:bm85a}"---\pageref{bkm:bm85b}.}
мы обратили внимание на возражение {\em Канта} против
этого доказательства, и указали, что Кант разумеет под
{\em существованием определенное} наличное бытие, через
которое нечто вступает в контекст совокупного опыта, т.~е. в определение
некоторого {\em инобытия} и в соотношении с
{\em другим}. Таким образом, нечто как существующее
опосредствовано другим, и существование есть вообще сторона его
опосредствования. Но в том, что Кант называет понятием, а именно, в нечто,
поскольку его берут лишь как просто {\em соотносящееся
с собой}, или, иначе говоря, в представлении как таковом нет его
опосредствования; в абстрактном тождестве с собой отброшено
противоположение. Онтологическое доказательство должно было бы показать,
что абсолютное понятие, а именно понятие бога приходит к определенному
наличному бытию, к опосредствованию, или, иначе говоря, имело бы целью
доказать, каким образом простая сущность опосредствует себя с
опосредствованием. Это выполняется путем вышеуказанного подведения
существования под его всеобщее, а именно под реальность, которая
принимается за средний термин между богом в его понятии, с одной стороны, и
существованием, с другой. "--- Об этом опосредствовании, поскольку оно имеет
форму умозаключения, мы здесь, как сказано, не будем говорить. Но каков
истинный характер этого опосредствования сущности с существованием, "--- это
вытекает из предшествующего изложения. Природа самого доказательства
подлежит рассмотрению в учении о познании. Здесь нужно лишь указать на то,
что относится к природе опосредствования вообще.

Доказательства существования бога указывают
{\em основание} для этого существования. Это основание
не должно быть объективным основанием существования бога; ибо существование
бога есть в себе и для себя самого. Поэтому оно есть лишь
{\em основание для познания}. Тем самым оно вместе с
тем выдает себя за нечто такое, что {\em исчезает} в
том предмете, который сначала кажется обоснованным им. Так вот основание,
почерпнутое из случайности мира, содержит в себе возвращение этой
случайности в абсолютную сущность; ибо случайное есть
{\em лишенное основания} в себе самом и снимающее себя.
Абсолютная сущность, стало быть, при этом способе доказательства на самом
деле возникает из того, что лишено основания; основание снимает само себя,
и тем самым исчезает также и видимость приписанного богу отношения, будто
он есть нечто обоснованное в некотором другом. Это опосредствование есть
поэтому истинное опосредствование. Но указанной доказывающей рефлексии эта
природа ее опосредствования остается неизвестной; она принимает себя, с
одной стороны, за нечто только субъективное и тем самым отстраняет свое
опосредствование от самого бога, а с другой стороны, именно поэтому не
познает, что это опосредствующее движение имеет место в
{\em самой сущности}, не познает также и того, как оно
имеет место в этой сущности. Поистине же с этим опосредствованием дело
обстоит так, что оно есть и то и другое сразу, есть опосредствование как
таковое, но вместе с тем и субъективное, внешнее, а именно, внешнее себе
опосредствование, которое {\em снова снимает себя в нем
самом}. В~вышеуказанном же изложении существование получает неправильную
трактовку, так что оно представляется лишь
{\em опосредствованным} или положенным.

С другой стороны, нельзя также рассматривать существование как нечто
исключительно только {\em непосредственное}. Будучи
взято в определении непосредственности, постижение существования бога
объявлялось чем-то недоказуемым, а знание об этом существовании "--- некоторым
{\em только} непосредственным сознанием,
{\em верой}. Знание якобы должно прийти к тому
результату, что оно {\em ничего не знает}, т.~е. что
оно само снова {\em отказывается} от своего
{\em опосредствующего} движения и встречающихся в нем
определений. Это выявилось уже в предыдущем; но следует прибавить, что
рефлексия, оканчивая снятием себя самой, в силу этого еще не имеет своим
результатом {\em ничто}, так что оказалось бы, что
положительное знание о сущности, как
{\em непосредственное} соотношение с ней,
{\em не связано} с вышеуказанным результатом и
представляет собой особо возникающий, лишь с себя начинающий акт; дело
обстоит так, что самый этот конец, это уничтожение опосредствования,
погружение его в основание есть вместе с тем то
{\em основание}, из которого происходит
непосредственное. Язык [немецкий], как выше было указано, соединяет
значение этого {\em уничтожения} и
{\em основания}; так например, говорится, что сущность
бога есть {\em пучина} (Abgrund. Буквально "--- отсутствие
основания. "--- Перев.) для конечного разума. Она действительно такова,
поскольку этот разум отказывается в ней от своей конечности и погружает в
нее свое опосредствующее движение; но эта {\em пучина},
это отрицательное основание есть вместе с тем
{\em положительное} основание возникновения сущего, в
себе самой непосредственной сущности; опосредствование есть
{\em существенный момент}. Опосредствование основанием
снимает себя, но не оставляет основание внизу так, чтобы то, что из него
возникает, было чем-то {\em положенным}, имеющим свою
сущность где-то в другом месте, а именно, в основании; это основание есть,
как пучина, исчезнувшее опосредствование, и, наоборот, лишь исчезнувшее
опосредствование есть вместе с тем основание и лишь через это отрицание оно
есть равное самому себе и непосредственное.

Таким образом, существование не следует здесь понимать, как
{\em предикат} или
{\em определение} сущности, так что предложение о нем
гласило бы: сущность существует или {\em обладает}
существованием. Дело обстоит так, что сущность перешла в существование;
существование есть ее абсолютное отчуждение, и про нее нельзя сказать, что
она осталась по ту сторону этого отчуждения. Предложение о существовании
поэтому гласило бы: сущность есть существование; она не отлична от своего
существования. "--- Сущность {\em перешла} в
существование, поскольку сущность как основание уже не отличается от себя
как обоснованного, или поскольку это основание сняло себя. Но это отрицание
столь же существенным образом есть ее полагание или безоговорочно
положительная непрерывность с собой самой; существование есть рефлексия
{\em основания} в себя, его тождество с самим собой,
достигнутое им в своем отрицании, следовательно, такое опосредствование,
которое положило себя тождественным с собой и которое тем самым есть
непосредственность.

Так как существование есть по существу
{\em тождественное с собой опосредствование}, то оно
имеет {\em определения} опосредствования
{\em в нем} [в себе самом] но так, что они вместе с тем
рефлектированы в себя и обладают существенным и непосредственным устойчивым
наличием. Как непосредственность, полагающая себя через снятие,
существование есть отрицательное единство и внутри-себя-бытие; оно
определяет себя поэтому непосредственно как некоторое
{\em существующее} и как
{\em вещь}.


\section[А. Вещь и ее свойства]{А. Вещь и ее свойства}

Существование как {\em существующее} положено в форме отрицательного
единства, каково оно и есть по существу. Но это отрицательное единство есть
ближайшим образом лишь {\em непосредственное}
определение и тем самым то одно, которое характеризует
{\em нечто} вообще. Но существующее нечто отлично от
сущего нечто. Первое есть по существу такая непосредственность, которая
возникла через рефлексию опосредствования в само себя. Таким образом,
существующее нечто есть некоторая {\em вещь}.

Между {\em вещью} и ее
{\em существованием} проводится различие подобно тому,
как можно проводить различие между {\em нечто} и его
{\em бытием}. Вещь и существующее есть непосредственно
одно и то же. Но так как существование не есть первая непосредственность
бытия, а заключает в себе самом момент опосредствования, то его определение
в вещь и различение между ними есть, собственно говоря, не переход, а
анализ, и существование, как таковое, само содержит в себе это различение в
моменте своего опосредствования, "--- различие между
{\em вещью-в-себе} и {\em внешним} существованием.


\subsection[а) Вещь-в-себе и существование]{а) Вещь-в-себе и существование}

1.{\em Вещь-в-себе} есть
существующее, как имеющееся благодаря снятому опосредствованию
{\em существенное непосредственное}. При этом для
вещи-в-себе столь же существенно и опосредствование; но указанное различие
в этом первом или непосредственном существовании распадается на
{\em безразличные определения}. Одна сторона, а именно,
опосредствование вещи, есть ее {\em не-рефлектированная
непосредственность}, следовательно, ее бытие вообще, которое, так как оно
вместе с тем определено как опосредствование, есть некоторое
{\em другое} для себя самого, внутри себя {\em многообразное} и {\em внешнее
наличное бытие}. Но оно есть не только наличное бытие, а находится в
соотношении со снятым опосредствованием и существенной непосредственностью;
поэтому оно есть наличное бытие как
{\em несущественное}, как положенность."--- (Если
различают вещь от ее существования, то она есть нечто
{\em возможное}, вещь представляемая или сочиненная
мыслью вещь, которая как таковая не должна вместе с тем непременно быть
существующей. Однако об определении возможности и о противоположности вещи
и ее существования будет сказано далее.) "--- Но вещь-в-себе и
опосредствованное бытие вещи оба содержатся в существовании и оба
существуют; вещь-в-себе существует и есть существенное существование вещи,
опосредствованное же бытие есть ее несущественное существование.

{\em Вещь-в-себе} как простая рефлектированность
существования в себя не есть основание несущественного наличного бытия; она
есть неподвижное, неопределенное единство именно потому, что ей свойственно
определение быть снятым опосредствованием и потому лишь
{\em основой} этого наличного бытия. Поэтому же и
рефлексия как наличное бытие, опосредствующее себя через другое, имеет
место {\em вне вещи-в-себе}. Последняя не должна в ней
самой иметь никакого определенного многообразия и потому приобретает
последнее, лишь {\em будучи перенесена во внешнюю
рефлексию}, но при этом остается к нему безразличной (вещь-в-себе имеет
цвет, лишь будучи поднесена к глазу, запах "--- к носу; и~т.~п.). Ее различия
суть лишь различные отношения к ней некоторого другого, они суть
определенные соотношении с вещью-в-себе, сообщаемые себе этим другим, а не
ее собственные определения.

2. Это другое есть рефлексия, которая, будучи определена, как внешняя,
{\em во-первых}, {\em внешня себе
самой} и есть определенное многообразие.
{\em Во-вторых}, она внешня существенно-существующему и
{\em соотносится} с ним, как со своей абсолютной
{\em предпосылкой}. Но оба эти момента внешней
рефлексии, ее собственное многообразие и ее соотношение с другой для нее
вещью-в-себе, суть одно и то же. Ибо это существование внешне лишь
постольку, поскольку оно соотносит себя с существенным тождеством, как с
{\em некоторым другим}. Поэтому многообразие не имеет
собственного самостоятельного устойчивого наличия по ту сторону
вещи-в-себе, а оказывается лишь видимостью по сравнению с нею, оказывается
в своем необходимом соотношении с нею лишь преломляющимся в ней рефлексом.
Таким образом, различия имеются как соотношение некоторого другого с
вещью-в-себе; но это другое есть отнюдь не некоторое устойчиво наличное
само по себе, а лишь соотношение с вещью-в-себе; вместе с тем однако оно
есть лишь отталкивание от нее и, таким образом, лишенное опоры отбрасывание
себя в себя самого.

Вещи же в-себе, так как она есть существенное тождество существования, не
присуща поэтому эта несущественная рефлексия, и последняя рушится внутри
себя самой вне вещи-в-себе. Она идет ко дну и тем самым сама становится
существенным тождеством или вещью-в-себе. "--- Это может быть рассматриваемо
также и следующим образом: лишенное сущности существование имеет свою
рефлексию в себя в вещи-в-себе; оно соотносится с нею прежде всего, как со
{\em своим} другим; но как другое по отношению в тому,
что есть в себе, оно есть лишь снятие себя самого и становление
в-себе-бытием. Тем самым вещь-в-себе тождественна с внешним существованием.

Это проявляется в вещи-в-себе следующим образом. Вещь-в-себе есть
{\em соотносящееся с собой}, существенное
существование; она лишь постольку есть тождество с собой, поскольку в ней
содержится отрицательность рефлексии в себя самое; то, что представлялось
внешним ей существованием, есть поэтому момент в ней самой. Поэтому она
есть также отталкивающая себя от себя вещь-в-себе, которая,
{\em таким образом, относится к себе, как к некоторому
другому}. Тем самым имеются теперь несколько вещей-в-себе, находящихся
между собой в соотношения внешней рефлексии. Это несущественное
существование есть их отношение друг к другу, как к другим; но оно, далее,
существенно для них самих, или, иначе говоря, это несущественное
существование, кружась внутри себя, есть вещь-в-себе, но
{\em другая} вещь-в-себе, чем та первая; ибо та первая
есть непосредственная существенность, последняя же происходит из
несущественного существования. Однако эта другая вещь-в-себе есть лишь
некоторое {\em другое} вообще; ибо как тождественная с
собой вещь она не имеет никакой дальнейшей определенности по отношению в
первой; она, как и первая, есть рефлексия несущественного существования в
себя. Определенность разных вещей-в-себе по отношению друг к другу имеет
поэтому место во внешней рефлексии.

3. Эта внешняя рефлексия есть теперь отношение вещей-в-себе друг к другу,
{\em их взаимное опосредствование} как других.
Вещи-в-себе суть, таким образом, крайние члены некоторого умозаключения,
средний член которого составляет их внешнее существование, то
существование, через которое они суть другие друг для друга и различные.
Это их различие имеет место лишь в {\em их
соотношении}; они как бы лишь высылают определения от своей поверхности в
свое соотношение с другими, к которому они, как абсолютно рефлектированные
в себя, остаются безразличными. "--- Это отношение составляет теперь
тотальность существования. Вещь-в-себе находится в соотношении с некоторой
внешней для нее рефлексией, в которой она обладает многообразными
определениями; это "--- ее отталкивание себя от самой себя в другую вещь-в-себе;
это отталкивание есть ее отбрасывание себя в самое себя, поскольку каждая
есть некоторая другая лишь как светящая себе вновь из другой; она имеет
свою положенность не в себе самой, а в другой, определена лишь через
определенность другой; эта другая точно так же определена лишь через
определенность первой. Но так как {\em обе} вещи-в-себе
тем самым имеют разность не в самих себе, а каждая лишь в другой, то они не
суть различные; вещь-в-себе, которая должна относиться к другому крайнему
члену, как к некоторой другой вещи-в-себе, относится к ней как к чему-то
неразличному от нее самой, и внешняя рефлексия, которая должна была бы
составлять опосредствующее соотношение между крайними членами, есть
отношение вещи-в-себе лишь к самой себе или, иначе говоря, есть по существу
ее рефлексия в себя; она тем самым есть в-себе-сущая определенность или
определенность вещи-в-себе. Вещь-в-себе обладает ею, следовательно, не в
некотором внешнем для нее соотношении с другой вещью-в-себе и этой другой с
нею; определенность есть не только поверхность вещи-в-себе, а есть
существенное опосредствование ее с собою, как с некоторым другим. "--- Обе
вещи-в-себе, которые должны были бы составлять крайние члены соотношения,
так как они в себе не должны иметь никакой определенности по отношению друг
к другу, {\em на самом деле }{\em сливаются воедино}; имеется лишь
{\em одна} вещь-в-себе, относящаяся во внешней
рефлексии к самой себе, и ее {\em собственное
соотношение с собой, как с некоторой другой}, и составляет ее определенность.

Эта определенность вещи-в-себе есть {\em свойство} вещи.


\subsection[b) Свойство]{b) Свойство}

{\em Качество} есть
{\em непосредственная} определенность данного нечто,
само то отрицательное, через которое бытие есть нечто. Таким же образом,
{\em свойство} вещи есть отрицательность рефлексии,
через которую существование вообще есть некоторое существующее и, как
простое тождество с собою, {\em вещь-в-себе}. Но
отрицательность рефлексии, снятое опосредствование, само есть по существу
опосредствование и соотношение, соотношение не с другим вообще, подобно
качеству как не-рефлектированной определенности, а
{\em с собой}, как с некоторым другим, или, иными
словами, такое {\em опосредствование}, которое
непосредственно есть {\em также и тождество с собой}.
Абстрактная вещь-в-себе сама есть это возвращающееся в себя из другого
отношение; она вследствие этого {\em определена в самой
себе}; но ее определенность есть {\em характер},
который как таковой сам есть {\em определение}, а, как
отношение к другому, {\em не переходит} в инобытие и не
подлежит {\em изменению}.

Вещь обладает {\em свойствами}; они суть,
{\em во-первых}, ее определенные соотношения с
{\em другим}; свойство имеется лишь как некоторый
способ отношения друг к другу; оно поэтому есть внешняя рефлексия и аспект
(die Seite) положенности вещи. Но, {\em во-вторых},
вещь в этой положенности есть {\em в себе}; она
сохраняет себя в соотношении с иным; следовательно, если существование
отдает себя в жертву становлению бытия и изменению, то это затрагивает лишь
некоторую поверхность вещей; свойство не теряется в этом изменении. Вещь
обладает свойством вызывать то или другое в ином и проявляться своеобразно
в своем соотношении с другими вещами. Она обнаруживает это свойство лишь
при условии наличия соответствующего характера другой вещи, но оно вместе с
тем ей {\em свойственно} и есть ее тождественная с
собою основа; это рефлектированное качество называется поэтому
{\em свойством}. Вещь переходит в нем в некоторую
внешность, но свойство сохраняется в последней. Вещь становится через свои
свойства причиной, а причина заключается в том, что она сохраняет себя как
действие. Однако здесь вещь есть пока что лишь спокойная вещь со многими
свойствами, а еще не определена как действительная причина; она пока что
есть лишь в-себе-сущая рефлексия ее определений, а еще не их полагающая
рефлексия.

{\em Вещь-в-себе}, следовательно, как выяснилось, есть
по существу вещь-в-себе не только таким образом, что ее свойства суть
положенность некоторой внешней рефлексии, а они суть ее собственные
определения, в силу которых она ведет себя определенным образом; она есть
не находящаяся по ту сторону ее внешнего существования, лишенная
определений основа, а наличествует в своих свойствах, как основание, т.~е.
есть тождество с собою в своей положенности; но вместе с тем она имеется в
этих свойствах как {\em обусловленное} основание, т.~е.
ее положенность есть также и внешняя себе рефлексия; она лишь постольку
рефлектирована в себя и есть в себе, поскольку она внешня. "--- Через
существование вещь-в-себе вступает во внешние соотношения, и существование
состоит в этой внешности; оно есть непосредственность бытия, и вещь в силу
этого подвержена изменению; но оно есть также и рефлектированная
непосредственность основания, и тем самым вещь оказывается
{\em в себе} в своем изменении. "--- Это упоминание о
соотношении основания следует, однако, здесь понимать не в том смысле, что
вещь определена вообще как основание своих свойств; сама вещность как
таковая есть определение основания; свойство не отлично от своего
основания; равным образом оно не составляет исключительно только
положенности, а есть перешедшее в свою внешность и тем самым истинно
рефлектированное в себя основание; само свойство как таковое есть
основание, в-себе-сущая положенность, или, иными словами, основание
составляет {\em форму} его (свойства) {\em тождества} с собой;
{\em определенность} свойства есть внешняя себе
рефлексия основания, а целое есть основание, соотносящееся с собою в своем
отталкивании и процессе определения, в своей внешней непосредственности. "---
{\em Вещь-в-себе}, следовательно, существенным образом
{\em существует}, а то обстоятельство, что она
существует, означает, обратно, что существование, как внешняя
непосредственность, есть вместе с тем {\em в-себе-бытие}.

\hegremark[Примечание]%
  {Вещь-в-себе трансцендентального идеализма}%
  {[Вещь-в-себе трансцендентального идеализма]}

Уже выше (I~часть, I~отдел, стр.~\pageref{bkm:bmThingInItself}), говоря о
моменте наличного бытия, в-себе-бытии, мы упомянули о
{\em вещи-в-себе} и при этом заметили, что вещь-в-себе
как таковая есть не что иное, как пустая абстракция от всякой
определенности; об этой вещи-в-себе, разумеется, нельзя
{\em ничего знать} именно потому, что она есть
абстракция от всякого определения. "--- После того как вещь-в-себе
предположена, таким образом, как нечто неопределенное, всякое определение
получает место вне ее, во внешней ей рефлексии, в которой она безразлична.
Для {\em трансцендентального идеализма} эта внешняя
рефлексия есть {\em сознание}. Так как эта философская
система переносит всякую определенность вещей как по форме, так и по
содержанию, в сознание, то согласно этой точке зрения
{\em во мне}, в субъекте, кроется причина того, что я
вижу листья дерева не черными, а зелеными, вижу солнце круглым, а не
четырехугольным, что для моего вкуса сахар сладок, а не горек, что я
определяю первый и второй удар часов как последовательные, а не как
рядоположные, что я не определяю первый удар ни как причину, ни как
действие второго удара и~т.~д. "--- Этому кричащему изображению, даваемому
субъективным идеализмом, непосредственно противоречит сознание свободы,
согласно которому я знаю себя скорее чем-то всеобщим и неопределенным,
отделяю от себя те многообразные и необходимые определения и познаю их как
нечто внешнее для меня, присущее лишь вещам. "--- <<Я>> в этом сознании своей
свободы есть для себя то истинное, рефлектированное в себя тождество,
которым по указанному учению служит вещь-в-себе. "--- В другом месте я
показал, что этот трансцендентальный идеализм не выходит за пределы
ограниченности <<я>> объектом и вообще за пределы конечного мира, а
единственно только изменяет {\em форму} этого предела,
который остается для него чем-то абсолютным, так как он (трансцендентальный
идеализм) лишь перемещает его из объективного облика в субъективный и
делает определенностями <<я>> и совершающеюся в последнем, как в некоторой
вещи, дикой сменой этих определенностей то, что обычное сознание знает как
многообразие и изменение, принадлежащие лишь внешним ему вещам. "--- Здесь же
в предлежащем рассмотрении противостоят друг другу лишь вещь-в-себе и
ближайшим образом внешняя ей рефлексия; последняя еще не определила себя
как сознание, равно как и вещь-в-себе еще не определила себя как <<я>>. Из
рассмотрения природы вещи-в-себе и внешней рефлексии получился тот вывод,
что само это внешнее определяет себя в вещь-в-себе или, наоборот,
становится собственным определением той первой вещи-в-себе. Существенная
недостаточность той точки зрения, на которой останавливается указанная
философия, состоит в том, что эта точка зрения упорно держится
{\em абстрактной вещи-в-себе} как некоторого
{\em последнего} определения, и противопоставляет
вещи-в-себе рефлексию или определенность и многообразие свойств, между тем
как на самом деле вещь-в-себе имеет по существу указанную внешнюю рефлексию
в ней же самой и определяет себя в нечто наделенное
{\em собственными} определениями, свойствами,
вследствие чего та абстракция вещи, в силу которой она есть чистая
вещь-в-себе, оказывается неистинным определением.

\subsection[с) Взаимодействие вещей]{с) Взаимодействие вещей}
Вещь-в-себе существенным образом
{\em существует}; внешняя непосредственность и
определенность входят в состав ее в-себе-бытия или в состав ее рефлексии в
себя. Вещь-в-себе есть в силу этого некоторая вещь, обладающая свойствами и
вследствие этого имеется много вещей, которые отличны друг от друга не
через какое-нибудь чуждое им отношение, а через самих себя. Эти многие
разные вещи находятся благодаря их свойствам в существенном взаимодействии;
свойство есть само это взаимосоотношение, и вещь есть ничто вне этого
взаимодействия; взаимное определение, средний термин вещей-в-себе, которые
должны были бы, как крайние термины, оставаться безразличными к этому их
соотношению, само есть тождественная с собою рефлексия и та самая
вещь-в-себе, которой должны были быть те крайние термины. Вещность тем
самым низведена до формы неопределенного тождества с собой, имеющего свою
существенность лишь в своем свойстве. Поэтому если идет речь о некоторой
вещи или о вещах вообще помимо определенного свойства, то их различие
представляет собою лишь безразличное, количественное различие. То же самое,
что рассматривается как {\em одна} вещь, может также
быть превращено во многие вещи, или, иначе говоря, может рассматриваться,
как многие вещи; перед нами некоторое {\em внешнее
разделение} или {\em соединение}. "--- Книга есть вещь, и
каждый из ее листов есть также вещь и точно так же каждый кусочек ее
листов, и так далее до бесконечности. Определенность, благодаря которой
некоторая вещь есть лишь {\em эта} вещь, заключается
исключительно только в ее свойствах. Она отличается этими свойствами от
других вещей, так как свойство есть отрицательная рефлексия и различение;
вещь поэтому лишь в своем свойстве имеет коренящееся в ней самой отличие ее
от других вещей. Свойство есть рефлектированное в себя различие, через
которое вещь в своей положенности, т.~е. в своем соотношении с другим,
вместе с тем безразлична к другому и к своему соотношению. Поэтому на долю
вещи без ее свойств не остается ничего другого, кроме абстрактного
в-себе-бытия, некоторого несущественного объема и внешнего охвата. Истинное
в-себе-бытие есть в-себе-бытие в его положенности; последняя есть свойство.
Тем самым {\em вещность перешла в свойство}.

Вещь должна была бы относиться к свойству, как в-себе-сущий крайний термин,
а свойство должно было составлять средний термин между находящимися в
соотношении вещами. Однако это соотношение представляет собою то, в чем
вещи, {\em как отталкивающаяся от самой себя
рефлексия}, встречаются друг с другом и в чем они различны и соотнесены.
Это их различие и их соотношение есть их единая рефлексия и
{\em единая непрерывность}. Сами вещи, стало быть,
имеют место лишь в этой непрерывности, которая есть свойство, и исчезают
как такие устойчиво наличные крайние термины, которые обладали бы
существованием вне этого свойства.

{\em Свойство}, которое должно было бы составлять
{\em соотношение} самостоятельных крайних терминов,
есть поэтому {\em само самостоятельное}. Вещи же суть,
напротив, несущественное. Они суть некоторое
{\em существенное} лишь как рефлексия, соотносящаяся с
собой в процессе своего саморазличения; но это есть свойство. Последнее не
есть, следовательно, нечто снятое в вещи или простой момент этой вещи, а
вещь есть на самом деле лишь вышеупомянутый несущественный охват, который,
правда, есть отрицательное единство, но лишь подобно единству некоторого
нечто, а именно, как некоторое {\em непосредственное}
одно. Если выше мы определили вещь, как несущественный охват, постольку,
поскольку она сделана таковым через внешнюю абстракцию, опускающую ее
свойство, то теперь эта абстракция совершилась через самый переход
вещи-в-себе в свойство, но с обратным значением, так что если тому, который
совершал первое абстрагирование, еще предносится абстрактная вещь без ее
свойства, как существенное, свойство же рассматривается как некоторое
внешнее определение, то здесь вещь как таковая определяет себя через самое
себя в некоторую безразличную внешнюю форму свойства. "--- Свойство, стало
быть, освобождено теперь от неопределенного и бессильного
{\em сочетания}, которым служит единство вещи; оно есть
то, что составляет {\em устойчивое наличие} вещи,
некоторая {\em самостоятельная материя}. "--- Поскольку
свойство есть простая непрерывность с собой, оно содержит в себе форму
ближайшим образом лишь как {\em разность}; имеются
поэтому {\em многообразные} такого рода самостоятельные
материи, и {\em вещь состоит из них}.


\section[В. Составленность (das Bestehen) вещи из материй]
{В. Составленность (das Bestehen) вещи из материй}

Переход {\em свойства} в
некоторую {\em материю} или в некоторое самостоятельное
{\em вещество} есть тот известный переход, который
совершает относительно чувственной материи химия, пытаясь представлять
{\em свойства} цвета, запаха, вкуса и~т.~д. как
{\em световое, цветовое, пахучее}, кислое, горькое
и~т.~д. {\em вещества} или же прямо принимая лишь
гипотетически существование других веществ, как например,
{\em теплород}, электрическая, магнетическая материя,
причем она убеждена, что этим она трактует свойства в их истинности. "---
Столь же ходяче выражение, что вещи {\em состоят} из
разных материй или веществ. При этом химики остерегаются называть эти
{\em материи} или вещества
{\em вещами}, хотя они и готовы согласиться с тем, что,
например, пигмент есть вещь; но мне неизвестно, чтобы, например, называли
вещами также и световое вещество, теплород или электрическую материю
и~т.~д. Различают вещи и их составные части, не указывая точно, суть ли
последние тоже вещи и в какой мере они суть вещи или, скажем, лишь
полувещи: но во всяком случае они суть по меньшей мере некоторые
{\em существующие} вообще.

Необходимость переходить от свойств к материям или признать, что свойства
поистине суть материи, получилась вследствие того, что они суть
существенное и тем самым истинно-самостоятельное в вещах. "--- Вместе с тем,
однако, рефлексия свойства в себя составляет лишь одну сторону всей
рефлексии, а именно, снятие различия и непрерывность с самим собой
свойства, которое должно было быть некоторым существованием для другого.
Вещность, как отрицательная рефлексия в себя и отталкивающееся от другого
различение, низведена этим на уровень несущественного момента; но вместе с
тем последний тем самым определился далее. Этот отрицательный момент,
{\em во-первых}, {\em сохранился};
ибо свойство стало непрерывным с собой и самостоятельной материей лишь
постольку, поскольку различие вещей {\em сняло} себя;
непрерывная продолжаемость свойства в инобытие сама содержит,
следовательно, момент отрицательного, и его самостоятельность есть вместе с
тем, как это {\em отрицательное единство},
восстановленное {\em нечто} вещности, отрицательная
самостоятельность, противостоящая положительной самостоятельности вещества.
{\em Во-вторых}, вещь этим вышла из своей
неопределенности, дозрела до полной определенности. Как
{\em вещь-в-себе} она есть
{\em абстрактное} тождество,
{\em просто отрицательное} существование, или, иными
словами, существование, {\em определенное} как
{\em неопределенное}; затем она определена своими
свойствами, которыми она должна отличаться от других вещей; но так как она
скорее непрерывна с другими через свойство, то это неполное различие
снимается; вещь благодаря этому возвратилась в себя, и теперь определена
{\em как определенная}; она {\em определена в себе} или есть
<<{\em эта}>> вещь. "---

Но, {\em в-третьих}, это возвращение в себя есть,
правда, соотносящееся с собой определение, однако оно вместе с тем
несущественно; непрерывное с собой {\em устойчивое
наличие} образует самостоятельную материю, в которой различие вещей, их
в-себе-и-для-себя-сущая определенность, снято и есть некоторое внешнее.
Следовательно, вещь как <<{\em эта}>>{\em вещь}, хотя и
есть, правда, совершенная определенность, но это есть определенность в
элементе несущественности.

Рассматриваемый со стороны движения свойства, этот вывод получается
следующим образом. Свойство есть не только {\em внешнее} определение, но и
{\em в-себе-сущее} существование. Это единство
внешности и существенности, так как в нем содержится рефлексия в себя и
рефлексия в другое, отталкивает себя от самого себя и есть, с одной
стороны, определение как {\em простое}, тождественно
соотносящееся с собой самостоятельное, в котором отрицательное единство,
{\em одно} вещи, есть некоторое снятое, а, с другой
стороны, оно есть это определение по отношению к другому, но опять-таки как
рефлектированное в себя, определенное в себе одно; есть, следовательно, с
одной стороны, {\em материи}, а, с другой,
<<{\em эта}>>{\em вещь}. Это "--- два
момента тождественной с собою внешности или рефлектированного в себя
свойства. "--- Свойство было тем, чем должны были различаться вещи; так как
теперь оно освободилось от этой своей отрицательной стороны, от того, чтобы
быть присущим некоторому другому, то этим и вещь также освобождена от своей
определяемости другими вещами и возвратилась в себя из соотношения с
другим; но она есть вместе с тем лишь {\em ставшая себе
другой вещь-в-себе}, так как многообразные свойства стали со своей стороны
самостоятельными и, следовательно, их
{\em отрицательное соотношение} в единстве вещи стало
лишь снятым соотношением; вещь поэтому есть тождественное с собой отрицание
лишь {\em по отношению} к положительной непрерывности вещества.

<<{\em Этость}>> (das Diese) составляет, следовательно,
полную определенность вещи таким образом, что эта определенность есть
вместе с тем внешняя определенность. Вещь состоит из самостоятельных
материй, безразличных к их соотношению в вещи. Это соотношение есть поэтому
лишь некоторое несущественное сочетание означенных материй, и различие
одной вещи от другой основано на том, находится ли в ней несколько
особенных материй и в каком количестве они в ней находятся. Они переходят
{\em за }<<{\em эту}>> вещь,
продолжаются в других вещах, и их принадлежность этой вещи не есть для них
предел. Столь же мало они суть, далее, ограничение друг для друга, так как
их отрицательное соотношение есть лишь бессильная <<этость>>. Поэтому,
сочетаясь в вещи, они не снимают себя; они, как самостоятельные,
непроницаемы друг для друга, соотносятся в своей определенности лишь с
собой и суть некоторое взаимно безразличное многообразие устойчивого
наличия; они способны иметь лишь некоторую количественную границу. "--- Вещь,
как <<{\em эта}>>{\em вещь}, есть
это их чисто количественное соотношение, есть голая коллекция, их <<также>>.
Она {\em состоит} из какого-либо определенного
количества некоторого вещества, состоит {\em также} из
определенного количества некоторого другого вещества, состоит
{\em также} и из других; эту связь, состоящую в том,
что они не имеют никакой связи, единственно и образует собою вещь.


\section[С. Разложение вещи]{С. Разложение вещи}

<<{\em Эта}>> вещь, взятая
так, как она теперь определилась, как просто количественная связь свободных
веществ, совершенно изменчива. Ее изменение состоит в том, что одна или
несколько материй выделяются из этой совокупности или присоединяются к
этому <<также>>, или же в том, что взаимное соотношение их количеств
изменяется. Возникновение и уничтожение <<{\em этой}>>
вещи есть внешнее разложение такого внешнего сочетания или сочетание того,
чему безразлично быть или не быть сочетанным. Ничем не удерживаемые
вещества выходят из <<{\em этой}>> вещи или входят в нее;
сама она есть абсолютная пористость без собственной меры или формы.

Таким образом, вещь в ее абсолютной определенности, через которую она есть
<<{\em эта}>>, безоговорочно разложима. Это разложение
есть некоторая внешняя определяемость, равно как и бытие вещи; но ее
разложение и внешний характер ее бытия есть существенное в этом бытии; она
есть лишь вышеуказанное <<также>>; она состоит лишь в этой внешности. Но она
состоит также и из своих материй; и не только абстрактное <<это>>, как
таковое, но и {\em вся}
<<{\em эта}>>{\em вещь} есть
разложение себя самой. А~именно, вещь определена как внешняя коллекция
самостоятельных материй; эти материи не суть вещи, они не имеют
отрицательной самостоятельности, а суть свойства как нечто самостоятельное,
а именно "--- определенность, рефлектированная как таковая в себя. Поэтому,
хотя материи просты и соотносятся лишь с самими собой, но
{\em их содержание} есть некоторая
{\em определенность}; рефлексия в себя есть лишь
{\em форма} этого содержания, которое как таковое не
рефлектировано в себя, а соотносится по своей определенности
{\em с другим}. Вещь поэтому есть не только их <<также>>,
их соотношение как безразличных друг к другу, а в такой же мере и их
{\em отрицательное} соотношение; в силу своей
определенности материи сами суть эта их отрицательная рефлексия, каковая
рефлексия есть точечность вещи. Со стороны определенности их содержания по
отношению друг к другу одна материя не есть то,
{\em что} есть другая; а со стороны их
самостоятельности одна материя не {\em есть}, поскольку другая есть.

Вещь есть поэтому соотношение друг с другом материй, из которых она состоит,
таким образом, что в ней {\em состоят} (<<состоят>> в
смысле <<наличны>>. "--- Перев.) {\em и} одна
{\em и} другая, но так, что в ней вместе с тем одна
материя {\em не состоит}, поскольку состоит другая.
Следовательно, поскольку в вещи имеется одна материя, другая тем самым
снята; но вещь есть вместе с тем упомянутое <<также>> или устойчивое наличие
других материй. В~состоянии одной материи другая материя поэтому
{\em не} состоит и вместе с тем она также и состоит в
первой, и таким же образом состоят взаимно все эти разные материи. Поэтому
так как в том же отношении, в каком состоит одна, состоят также и другие, в
каковом едином их состоянии и заключается точечность или отрицательное
единство вещи, то они совершенно {\em проникают} одна
другую; а так как вещь вместе с тем есть лишь их <<также>> и материи
рефлектированы в их определенность, то они безразличны друг к другу и в
своем взаимопроникновении {\em не соприкасаются}.
Поэтому материи по существу {\em пористы}, так что одна
состоит {\em в порах} или в несостоянии других; но эти
другие сами пористы; в их порах или в их несостоянии состоят также и первая
и все прочие; их {\em состояние} есть вместе с тем их
{\em снятость} и состояние {\em других}; а это состояние других есть равным
образом их снятость и {\em состояние первой}, а также и
всех прочих. Поэтому вещь есть противоречащее себе опосредствование с собой
самостоятельного состояния через его противоположность, а именно, через его
отрицание, или, иначе, говоря, противоречивое опосредствование
{\em одной} самостоятельной материи через {\em состояние} и {\em несостояние
другой}. "--- Существование достигло в <<{\em этой}>>{\em вещи} своей
полноты, будучи в {\em одном и том же} и сущим-в-себе
бытием или {\em самостоятельным} состоянием, и
{\em несущественным} существованием; истина
существования заключается поэтому в том, что оно имеет свое в-себе-бытие в
несущественности или свое устойчивое наличие в некотором другом, и притом в
абсолютно другом, или, иными словами, имеет своей основой
{\em свою ничтожность}\footnote{<<Ничтожность>> (Nichtigkeit) не
в смысле малозначительности, а в смысле отсутствия всякого
значения. "--- {\em Перев.}}. Поэтому оно есть {\em явление}.

\hegremark[Примечание]%
  {Пористость материи}%
  {[Пористость материи]}

Одно из обычнейших определений вещи, указываемых представлением, заключается
в том, что {\em вещь состоит из многих самостоятельных
материй}. С~одной стороны, вещь рассматривается так, что она обладает
свойствами, {\em устойчивое наличие} которых есть
{\em вещь}. Но, с другой стороны, эти разные
определения понимаются как материи, устойчивое наличие которых не есть
вещь, а, наоборот, {\em вещь состоит} из них; сама она
есть лишь их внешнее сочетание и количественная граница. Оба, свойства и
материи, суть {\em одно и то же определение
содержания}, и различие их заключается лишь в том, что они там суть
моменты, рефлектированные в их отрицательное единство как в некоторую
отличную от них самих основу, в {\em вещность}, здесь
же они суть самостоятельные разные, каждое из которых рефлектировано в свое
собственное единство с собой. Эти материи, далее, определяют себя как
самостоятельное устойчивое наличие; но они также и совмещаются в одной
вещи. Эта вещь обладает двумя определениями: она есть, во-первых, <<эта>> и,
во-вторых, <<также>>. <<Также>> есть то, что во внешнем созерцании выступает
как {\em пространственная протяженность}; а <<эта>>,
отрицательное единство, есть {\em точечность} вещи.
Материи совместны в точечности, и их <<также>> или протяженность есть повсюду
эта точечность; ибо <<также>> как вещность, равным образом определено по
существу как отрицательное единство. Поэтому, где имеется одна из этих
материй, там {\em в одной и той же точке} есть и
другая; дело обстоит не так, что вещь имеет в другом месте свой цвет, а еще
в другом свое пахучее вещество, а в третьем свой теплород и~т.~д., но в той
же точке, в которой она тепла, она также и цветна, кисла, электризована
и~т.~д. Так как эти вещества или материи находятся не вне друг друга, а в
одном и том же <<этом>>, то они предполагаются
{\em пористыми}, так что одна материя существует в
промежутках другой. Но та материя, которая находится в промежутках другой,
сама также пориста; в ее порах существует поэтому, наоборот, другая, но не
только одна эта другая, а также и третья, десятая и~т.~д. Все материи
пористы, и в промежутках каждой из них находятся все другие, равно как и
сама она вместе с прочими материями находится в этих порах каждой из
материй. Они поэтому составляют некоторое множество, взаимно проникающее
друг в друга таким образом, что проникающие в свою очередь проницаются
другими, что, стало быть, каждая материя снова пронизывает свою собственную
пронизанность. Каждая материя положена как свое отрицание, и это отрицание
есть устойчивое наличие другой, но это устойчивое наличие есть в такой же
мере отрицание этой другой и устойчивое наличие первой.

Отговоркой, посредством которой {\em представление} не
допускает {\em противоречия самостоятельного}
устойчивого наличия {\em многих} материй в
{\em одной} вещи или, иными словами, их
{\em безразличия} друг к другу в их
{\em проникании}, служит, как известно,
{\em малость} частиц и пор. Где появляется различие в
себе, противоречие и отрицание отрицания, вообще, где требуется
{\em постижение в понятиях}, представление впадает во
внешнее {\em количественное} различие; касательно
возникновения и прехождения оно прибегает к постепенности, а касательно
бытия "--- к {\em малости}, так что исчезающее понижается
до {\em незаметного} и противоречие до путаницы,
истинное же отношение переделывается в неопределенное представление,
смутностью которого спасается упраздняющее себя.

Если же ближе осветить эту смутность, то окажется, что она есть противоречие
"--- отчасти субъективное противоречие представления и отчасти объективное
противоречие предмета; само представление содержит в себе полностью
элементы этого противоречия. А~именно, то, что оно, во-первых, само делает,
есть противоречие, состоящее в том, что оно хочет, с одной стороны,
держаться {\em восприятия} и вещей
{\em существующих}, а, с другой стороны, приписывает
тому, что {\em недоступно восприятию}, тому, что
определено рефлексией, чувственное существование; малые частицы и поры суть
согласно представлению вместе с тем и чувственное существование, и об их
положенности говорится, как о том же виде
{\em реальности}, который присущ цвету, теплоте и~т.~д.
Если бы, далее, представление рассмотрело ближе этот
{\em предметный} туман, т.~е. поры и малые частицы, то
оно познало бы в них не только некоторую материю и
{\em также} ее отрицание, так что выходило бы, что
{\em вот здесь} находится материя, а
{\em рядом} "--- ее отрицание, пóра, а
{\em рядом} с этим отрицанием снова материя и~т.~д., а
познало бы, что в лице <<{\em этой}>> вещи оно имеет: 1)
{\em самостоятельную} материю, и 2) ее {\em отрицание} или пористость, и
{\em другую самостоятельную} материю
{\em в одной и той же точке}, "--- познало бы, что эта
пористость и самостоятельное существование материй друг в друге, как в
чем-то одном, есть взаимное отрицание и проницание проницания. "--- Новейшие
изложения физики, говоря о распространении водяных паров в атмосферном
воздухе и о распространении различных газов друг в друге, более определенно
выделяют одну сторону того понятия, которое выяснилось здесь относительно
природы вещи. А~именно, они показывают, что, например, известный объем
вбирает в себя одинаковое количество водяных паров, все равно, свободен ли
этот объем от атмосферного воздуха или наполнен им; они выясняют также, что
различные газы так распространяются друг в друге, как будто каждый
представляет собою для другого пустоту; что по крайней мере они не
находятся между собою ни в каком химическом соединении и каждый,
непрерываемый другим, остается {\em непрерывным с
собой} и сохраняет себя {\em безразличным} к ним в
{\em своей пронизанности другими}. "--- Но дальнейший
момент понятия вещи заключается в том, что в
<<{\em этой}>> вещи одна материя находится там, где и
другая; что проникающее есть в одной и той же точке также и проникаемое,
или, иными словами, самостоятельное есть непосредственно самостоятельность
некоторого другого. Это "--- противоречиво, но вещь есть не что иное, как само
это противоречие; поэтому она есть явление.

Сходно с тем, как обстоит дело с этими материями, обстоит дело в области
духа с представлением о {\em душевных силах} или
{\em душевных способностях}. Дух есть в гораздо более
глубоком смысле <<{\em это}>>, т.~е. отрицательное
единство, в котором его определения проникают друг друга. Но когда его
представляют себе как душу, его часто принимают за некоторую
{\em вещь}. Подобно тому, как относительно человека
вообще считают, что он {\em состоит} из души и тела,
каждое из которых признается чем-то самостоятельным, особым, точно так же
признаются, что душа состоит из так называемых
{\em душевных сил}, каждая из которых обладает особой
самостоятельностью, или, другими словами, есть некоторая непосредственная,
действующая особо согласно своей определенности деятельность. Дело
представляют себе так, что вот здесь действует особо рассудок, а там особо
воображение, что мы культивируем порознь рассудок, память и~т.~д. и в это
время оставляем по левую руку бездеятельными другие силы, пока дойдет (а
может быть даже и не дойдет) очередь и до них. Так как их помещают в
материально-простую {\em душу-вещь}, которая как
простая, признается {\em имматериальной}, то
представляют себе способности, правда, не как отдельные материи, но как
{\em силы}, они принимаются столь же
{\em безразличными} друг к другу, как те материи.
Однако дух не есть то же самое противоречие, что и вещь, которая
разлагается и переходит в явление, а он уже в самом себе есть такое
противоречие, которое возвратилось в свое абсолютное единство, а именно, в
понятие "--- такое противоречие, в котором различия следует мыслить уже не как
самостоятельные, а лишь как {\em особенные} моменты в
субъекте, в простой индивидуальности.
