К~этой новой обработке <<Науки логики>>, первый том которой теперь выходит в
свет, я, должен сказать, приступил с полным сознанием как трудности и предмета
самого по себе, а затем и его изложения, так и несовершенства его обработки в
первом издании. Сколько я ни старался после дальнейших многолетних занятий этой
наукой устранить это несовершенство, я все же чувствую, что имею еще достаточно
причин просить читателя быть ко мне снисходительным. Право же на таковое
снисхождение может быть прежде всего обосновано тем обстоятельством, что для
содержания я нашел в прежних метафизике и логике преимущественно только внешний
материал. Хотя эти науки разрабатывались часто и повсюду, "--- последняя из
указанных наук разрабатывается еще и поныне, "--- все же эта разработка мало
касалась спекулятивной стороны, а скорее повторялся тот же самый материал,
который попеременно то разжижался до тривиальной поверхностности, то расширялся
благодаря тому, что снова вытаскивался старый баласт, так что от таких, часто
лишь совершенно механических, стараний философское содержание ничего не могло
выиграть. Изображение царства мысли философски, т.~е. в его собственной
имманентной деятельности, или, что то же самое, в его необходимом развитии,
должно было поэтому явиться новым предприятием, и приходилось начинать все с
самого начала. Вышеуказанный же ранее приобретенный материал "--- уже известные
формы мысли "--- должен рассматриваться как в высшей степени важный подсобный
материал (Vorlage) и даже как необходимое условие, как заслуживающая нашу
благодарность предпосылка, хотя этот материал лишь кое-где дает нам слабую нить
или безжизненные кости скелета, к тому же еще перемешанные между собою в
беспорядке.

Формы мысли выявляются и отлагаются прежде всего в человеческом {\em языке}.
В~наше время мы должны неустанно напоминать, что человек отличается от
животного именно тем, что он мыслит. Во все, чт\'{о} для него (человека)
становится чем-то внутренним, вообще представлением, во все, чт\'{о} он делает
своим, проник язык, а~все то, чт\'{о} человек превращает в язык и выражает в
языке, содержит в себе, в скрытом ли, спутанном или более разработанном виде,
некоторую категорию; в такой мере свойственно его природе логическое, или,
правильнее сказать, последнее есть сама его своеобразная {\em природа}. Но если
вообще противопоставлять природу, как физическое, духовному, то следовало бы
сказать, что логическое есть, наоборот, сверхприродное, проникающее во весь
природный обиход человека, в его чувства, созерцания, вожделения, потребности,
влечения и тем только и превращающее их, хотя лишь формально, в нечто
человеческое, в представления и цели. Если язык обладает богатством логических
выражений, и притом своеобразных и отдельных, для обозначения самих определений
мысли, то это является для него преимуществом перед другими языками. Из
предлогов и членов многие уже выражают отношения, покоящиеся на мышлении;
китайский язык, как нам сообщают, вовсе не выработал таких частей речи или
выработал их очень мало. Указанные частицы, однако, носят совершенно служебный
характер, они только немногим более отделены от других слов, чем глагольные
приставки, знаки склонения и~т.~д. Гораздо важнее, если в данном языке
определения мысли выявлены в форме существительных и глаголов и таким образом
отчеканены так, что получают предметную форму. Немецкий язык обладает в этом
отношении большими преимуществами перед другими новыми языками; многие из его
слов имеют к тому же еще ту особенность, что обладают не только различными, но
и противоположными значениями, так что нельзя не усмотреть в этом даже
некоторого спекулятивного духа этого языка: мышлению может только доставлять
радость, если оно наталкивается на такого рода слова и находит, что соединение
противоположностей, являющееся выводом спекуляции, но представляющее собою для
рассудка бессмыслицу, наивным образом выражено уже лексикально в виде {\em
одного} слова, имеющего противоположные значения. Философия вообще не нуждается
поэтому в особой терминологии; приходится, правда, заимствовать некоторые слова
из иностранных языков; эти слова, однако, благодаря частому употреблению, уже
получили в нашем языке право гражданства, и аффектированный пуризм был бы менее
всего уместен здесь, где более, чем где-либо, важна суть дела. "--- Успехи
образования вообще и, в частности, наук, даже опытных и чувственных, хотя эти
последние в общем движутся в рамках обычнейших категорий (например, категорий
части и целого, вещи и ее свойств и~т.~п.), постепенно выдвигают также и более
высокие отношения мысли, или по крайней мере, поднимают их до большей
всеобщности и тем самым заставляют обращать на них больше внимания. Если,
например, в физике получило преобладание определение мысли <<{\em сила}>>, то в
новейшее время самую значительную роль играет категория {\em полярности,}
которую, впрочем, слишком à~tort et à~travers (без разбору) втискивают во все,
даже в учение о свете\pagenote{Гегель имеет в виду философию природы Шеллинга,
в частности сочинение Шеллинга <<О мировой душе>> (1798~г.; 2-е изд. "---
1806~г.; 3-е изд. "--- 1809~г.), в котором <<закон полярности>> трактуется как
<<всеобщий мировой закон>>.}; полярность есть определение такого различия, в
котором различенные {\em неразрывно} связаны друг с другом. То обстоятельство,
что таким образом отошли от формы абстракции, от того тождества, которое
сообщает некоей определенности, например силе, самостоятельность, и вместо
этого была выдвинута и стала привычным представлением другая форма определения,
форма различия, которое вместе с тем продолжает оставаться в тождестве как
некое неотделимое от него, "--- это обстоятельство бесконечно важно.
Рассмотрение природы благодаря реальности, которую сохраняют ее предметы,
необходимо заставляет фиксировать категории, которых уже нельзя долее
игнорировать в ней, хотя при этом имеет место величайшая непоследовательность
по отношению к другим категориям, за которыми {\em также} сохраняют их
значимость, и это рассмотрение не допускает того, чт\'{о} легче происходит
в~науках о~духе, не допускает именно, чтобы переходили от противоположности
к~абстракциям и всеобщностям.

Но хотя, таким образом, логические предметы, равно как и выражающие их слова,
суть нечто всем знакомое в области образования, тем не менее, как я сказал
в~другом месте\pagenote{А~именно в предисловии к <<Феноменологии духа>>.
В~русском переводе под ред.~Радлова (Спб. 1913) это место гласит: <<Известное
вообще потому, что оно известно, не является познанным>> (стр.~14).}, то,
чт\'{о} {\em известно} ({\em bekannt}), еще не есть поэтому {\em познанное}
({\em erkannt}); между тем предъявление требования к человеку, чтобы он еще
продолжал заниматься тем, чт\'{о} ему уже известно, может даже вывести его из
терпения, "--- а чт\'{о} более известно, чем именно определения мысли, которыми
мы пользуемся постоянно, которые приходят нам на язык в каждом произносимом
нами предложении? Это предисловие и имеет своей целью указать общие моменты
движения познания, исходящего из этого известного, отношение научной мысли к
этому природному мышлению. Этих указаний вместе с~тем, чт\'{о} содержится
в~прежнем {\em введении,} достаточно для того, чтобы дать общее представление о
смысле логического познания, то общее представление, которое желают получить о
науке, к которой приступают, предварительно, до этой науки, которая уже есть
сам предмет (die Sache selbst).

Прежде всего следует рассматривать как бесконечный прогресс то обстоятельство,
что формы мышления были высвобождены из той материи, в которую они погружены в
самосознательном созерцании, представлении, равно как и в нашем вожделении и
волении, или, вернее, также и в представляющем вожделении и волении (а~ведь нет
человеческого вожделения или воления без представления), что эти всеобщности
были выделены в нечто самостоятельное и, как мы это видим у {\em Платона,} а
главным образом, у {\em Аристотеля,} были сделаны предметом самостоятельного
рассмотрения; этим начинается познание их. <<Лишь после того, "--- говорит
Аристотель, "--- как было налицо почти все необходимое и требующееся для
жизненных удобств и сношений, люди стали стараться достигнуть философского
познания>>\pagenote{Гегель цитирует здесь в вольном переводе фразу из
<<Метафизики>> Аристотеля (кн.~I, гл.~2, 982~b.). В~русском переводе
А.~В.~Кубицкого (М.---Л. 1934) это место находится на 22-й стр.}. <<В Египте,
"--- замечает он перед тем, "--- математические науки рано развились, так как
там жреческое сословие было рано поставлено в условия, дававшие ему
досуг>>\pagenote{Там же, кн.~I, гл.~1, 981~b. (в~переводе Кубицкого стр.~20).}.
Действительно, потребность заниматься чистыми мыслями предполагает длинный
путь, который человеческий дух должен был пройти ранее, она является, можно
сказать, потребностью уже удовлетворенной потребности в необходимом,
потребностью отсутствия потребности, которой человеческий дух должен был
достигнуть, потребностью абстрагировать от материи созерцания, воображения
и~т.~д., от материи конкретных интересов вожделения, влечения, воли, в каковой
материи закутаны определения мысли. В~тихих пространствах пришедшего к самому
себе и лишь в себе пребывающего мышления умолкают интересы, движущие жизнью
народов и отдельных людей. <<Со многих сторон, "--- говорит Аристотель в той же
связи, "--- человеческая природа зависима, но эта наука, которой ищут не для
какого-нибудь употребления, есть единственная наука, свободная в себе и для
себя, и потому кажется, что она как бы не является человеческим
достоянием>>\pagenote{Там же, кн.~I, гл.~2, 982~b. (в~переводе Кубицкого
стр.~22). Гегель переставил предложения в этой цитате.}. Философия вообще еще
имеет дело с конкретными предметами "--- богом, природой, духом "--- в их
мыслях; логика же занимается этими предметами, взятыми всецело особо, в их
полнейшей абстрактности. Логика поэтому является обычно предметом изучения для
юношества, каковое еще не вошло в интересы повседневной жизни, пользуется по
отношению к ним досугом и лишь для субъективной своей цели должно заниматься
приобретением средств и возможностей для проявления своей активности в сфере
объектов указанных интересов, причем и эти занятия еще носят теоретический
характер. К~этим {\em средствам,} в противоположность вышеуказанному
представлению Аристотеля, причисляют и науку логики; усердные занятия ею
считаются предварительной работой, местом для этой работы "--- школа, лишь
после окончания которой должны выступить на сцену вся серьезность жизни и
деятельность для достижения действительных целей. В~жизни переходят
к~{\em пользованию} категориями; они понижаются в ранге, лишаются чести
рассматриваться особо, а обрекаются на то, чтобы {\em служить} в деле духовной
выработки живого содержания, в создании и сообщении друг другу относящихся к
этому содержанию представлений. Они благодаря своей всеобщности служат отчасти
{\em сокращениями,} ибо какое бесконечное множество частностей внешнего
существования и деятельности объемлют собою представления: битва, война, народ,
или море, животное и~т.~д.; какое бесконечное множество представлений,
деятельностей, состояний и~т.~д. должно быть сокращено в представлении: бог или
любовь и~т.~д., чтобы благодаря этой концентрации получилась из них
{\em простота} такого единого представления! Отчасти же они служат для более
точного определения и нахождения {\em предметных отношений,} причем, однако,
содержание и цель, правильность и истинность вмешивающегося мышления ставятся в
полную зависимость от самого существующего, и определениям мысли, самим по
себе, не приписывается никакой определяющей содержание действенности. Такое
употребление категорий, которое в прежнее время называлось естественной
логикой, носит бессознательный характер; а если научная рефлексия отводит им в
духе роль служебных средств, то она этим превращает вообще мышление в нечто
подчиненное другим духовным определениям. Ведь о наших ощущениях, влечениях,
интересах мы не говорим, что они нам служат, а считаем их самостоятельными
силами и властями (Mächte); так что мы сами состоим в том, чтобы ощущать
так-то, желать и хотеть того-то, полагать свой интерес в том-то. У~нас,
наоборот, может получиться сознание, что мы скорее служим нашим чувствам,
влечениям, страстям, интересам и тем паче привычкам, а не обладаем ими; ввиду
же нашего внутреннего единства с ними нам еще менее может прийти в голову, что
они нам служат средствами. Мы скоро обнаруживаем, что такого рода определения
души и духа суть {\em особенные} в противоположность {\em всеобщности,} в
качестве каковой мы себя сознаем и в каковой заложена наша свобода, и начинаем
думать, что мы находимся в плену у этих особенностей, что они властвуют над
нами. После этого мы тем менее можем считать, что формы мысли, тянущиеся через
все наши представления, "--- будут ли последние чисто теоретическими или
содержащими материю, принадлежащую области ощущений, влечений, воли "--- служат
нам, что мы обладаем ими, а не скорее они нами. Что остается на {\em нашу} долю
против них, каким образом можем {\em мы,} каким образом могу я возвышать себя
{\em над} ними, как нечто более всеобщее, когда они сами суть всеобщее как
таковое? Когда мы влагаем себя в какое-нибудь чувство, какую-нибудь цель,
какой-нибудь интерес и чувствуем себя в них ограниченными, несвободными, то
областью, в которую мы из них в состоянии выбраться и тем самым выйти назад на
свободу, является эта область самодостоверности, область чистой абстракции,
мышления. Или, если мы хотим говорить о {\em вещах,} то мы равным образом
называем их {\em природу} или {\em сущность} их {\em понятием,} а последнее
существует только для мышления; о понятиях же вещей мы еще гораздо менее
решимся сказать, что мы ими владеем или что определения мысли, комплексом
которых они являются, служат нам; наша мысль должна, напротив, ограничивать
себя сообразно им и наш произвол или свобода не должны переделывать их
по-своему. Стало быть, поскольку субъективное мышление есть наше
наивнутреннейшее делание, а объективное понятие вещей составляет самое вещь, то
мы не можем выходить за пределы указанного делания, не можем стать выше его, и
столь же мало мы можем выходить за пределы природы вещей. Последнее определение
мы можем, однако, оставить в стороне; оно совпадает с первым постольку,
поскольку оно есть некое отношение нашего мышления к вещи, но только оно дало
бы нам нечто пустое, ибо мы этим признали бы вещь правилом для наших понятий, а
между тем вещь не может быть для нас не чем иным, кроме как нашим понятием о
ней. Если критическая философия понимает отношение между этими тремя терминами
так, что мы ставим {\em мысли} между {\em нами} и {\em вещами,} как средний
термин, в том смысле, что этот средний термин скорее отгораживает {\em нас} от
{\em вещей} вместо того, чтобы смыкать нас с ними, то этому взгляду следует
противопоставить то простое замечание, что как раз эти вещи, которые якобы
стоят на другом конце, по ту сторону нас и по ту сторону соотносящихся с ними
мыслей, сами суть вещи, сочиненные мыслью (Gedank\-end\-inge), а как совершенно
неопределенные, они суть лишь {\em одна} сочиненная мыслью вещь (так называемая
вещь в себе), пустая абстракция.

Но сказанного нами будет достаточно для уяснения той точки зрения, с которой
исчезает отношение к определениям мысли, как только к полезностям и к
средствам. Более важное значение имеет находящийся в связи с указанным
отношением дальнейший взгляд, согласно которому их понимают, как внешние формы.
Проникающая все наши представления, цели, интересы и поступки деятельность
мышления происходит, как сказано, бессознательно (естественная логика). Наше
сознание имеет перед собою согласно этому взгляду содержание, предметы
представлений, то, чт\'{о} заполняет интерес; определения же мысли суть
{\em формы,} которые только находятся {\em на содержимом,} а не суть само
содержимое. Но если верно то, что мы указали выше и с чем в общем соглашаются,
а именно, если верно, что {\em природа,} своеобразная {\em сущность,} как
истинно {\em пребывающее} и субстанциальное в многообразии и случайности
явлений и преходящем проявлении есть {\em понятие} вещи, {\em всеобщее в самой
этой вещи} (как например, каждый человеческий индивидуум, хотя и есть нечто
бесконечно своеобразное, все же имеет в себе prius (первичное) всего своего
своеобразия, prius, состоящее в том, что он в этом своеобразии есть
{\em человек} или как каждое отдельное животное имеет prius, состоящее в том,
что оно есть {\em животное}), то нельзя сказать, чт\'{о} осталось бы от такого
индивидуума (какими бы многообразными прочими предикатами он ни был снабжен),
если бы из него была вынута эта основа (хотя последняя тоже может быть названа
предикатом). Непременная основа, понятие, всеобщее, которое и есть сама мысль,
поскольку только при слове <<мысль>> можно отвлечься от представления, "--- это
всеобщее не может рассматриваться {\em лишь} как безразличная форма,
находящаяся {\em на} некотором содержании. Но эти мысли всех природных и
духовных вещей, составляющие само субстанциальное {\em содержание,} суть еще
такое содержание, которое заключает в себе многообразные определенности и еще
имеет в себе различие души и тела, понятия и относительной реальности; более
глубокой основой служит душа, взятая сама по себе, чистое понятие,
представляющее собою наивнутреннейшее в предметах, их простой жизненный пульс,
равно как и жизненный пульс самого субъективного мышления о них. Задача состоит
в том, чтобы осознать эту {\em логическую} природу, одушевляющую дух, орудующую
и действующую в нем. Инстинктообразное делание отличается от руководимого
интеллектом и свободного делания вообще тем, что последнее происходит
сознательно; поскольку содержание побудительного мотива выключается из
непосредственного единства с субъектом и получает характер стоящей перед ним
предметности, постольку возникает свобода духа, который в инстинктообразном
действовании мышления, связанный своими категориями, расщепляется на бесконечно
многообразную материю. В~этой сети завязываются там и сям более прочные узлы,
служащие опорными и направляющими пунктами его (духа) жизни и сознания; эти
узлы обязаны своей прочностью и мощью именно тому, что они, поставленные перед
сознанием, суть в себе и для себя сущие понятия его сущности. Важнейшим
пунктом, уясняющим природу духа, является отношение к тому, чт\'{о} он есть
{\em в действительности} не только того, чт\'{о} он есть {\em в~себе,} но и
того, чем он {\em себя знает;} так как дух есть по существу сознание, то это
знание себя есть основное определение его {\em действительности}. Вот,
следовательно, высшая задача логики: она должна очистить категории, действующие
сначала лишь инстинктообразно, как влечения, и осознаваемые духом порозненно,
стало быть, как изменчивые и путающие друг друга, доставляющие ему таким
образом порозненную и сомнительную действительность, и этим очищением возвысить
его в них, поднять его к свободе и истине.

То, на чт\'{о} мы указали, как на начальный пункт науки, высокая ценность
которого, взятого сам по себе, и вместе с тем как условия истинного познания,
признано было уже ранее, а именно, рассматривание понятий и вообще моментов
понятия, определений мысли, прежде всего в качестве форм, отличных от материи и
лишь находящихся на ней, "--- это рассматривание тотчас же являет себя в самом
себе неадэкватным отношением к истине, признаваемой предметом и целью логики.
Ибо, беря их как простые формы, как отличные от содержания, принимают, что им
присуще определение, характеризующее их, как конечные, и делающее их
неспособными схватить истину, которая бесконечна в себе. Пусть истинное в свою
очередь сочетано помимо этого в каком бы то ни было отношении с ограниченностью
и конечностью, "--- это есть аспект его отрицания, его неистинности и
недействительности, как раз его конца, а не утверждения, каковое оно есть, как
истинное. По отношению к пустоте чисто формальных категорий инстинкт здравого
разума почувствовал себя, наконец, столь окрепшим, что он презрительно
предоставляет их познание школьной логике и школьной метафизике, пренебрегая
вместе с тем той ценностью, которую рассмотрение этих нитей имеет уже само по
себе, и не сознавая того, что, когда он ограничивается инстинктообразным
действием естественной логики, а тем более когда он обдуманно (reflectiert)
отвергает изучение и познание самих определений мысли, он рабски служит
неочищенному и, стало быть, несвободному мышлению. Простым основным
определением или общим определением формы собрания таких форм служит
{\em тождество,} которое в логике этого собрания форм признается законом, как
$A\hm=A$, как закон противоречия. Здравый разум в такой мере потерял свое
почтительное отношение к школе, которая обладает такими законами истины и в
которой их продолжают разрабатывать, что он из-за этих законов насмехается над
нею и считает невыносимым человека, который, руководясь такими законами, умеет
высказывать такого рода истины: растение есть растение, наука есть наука
и~т.~д. {\em до бесконечности}. Относительно формул, служащих правилами
умозаключения, которое на самом деле представляет собою одно из главных
употреблений рассудка, также упрочилось столь же справедливое сознание, что они
по меньшей мере суть безразличные средства, средства, которые приводят также и
к заблуждению и которыми пользуется софистика; что, как бы мы ни определяли
истину, для высшей, например, религиозной, истины они непригодны, что они
вообще касаются лишь правильности познания, а не его истинности, хотя было бы
несправедливо отрицать, что в познании есть такая область, где они должны
обладать значимостью, и что вместе с тем они представляют собою существенный
материал для мышления разума.

Неполнота этого способа рассмотрения мышления, оставляющего в стороне истину,
может быть устранена лишь привлечением к мыслительному рассмотрению не только
того, чт\'{о} обыкновенно причисляется к внешней форме, но вместе с тем также и
содержания. Тогда вскоре само собою обнаруживается, что то, чт\'{о} в ближайшей
обычной рефлексии отделяют от формы, как содержание, в самом деле не должно
быть бесформенным, лишенным определений внутри себя, ибо в таком случае оно
было бы пустотой, скажем абстракцией вещи в себе; что оно, наоборот, обладает в
самом себе формой и, даже больше того, только благодаря ей одушевлено и
обладает содержимым (Gehalt), и что это она же сама превращается в видимость
некоего содержания, равно как, стало быть, и в видимость чего-то внешнего на
этой видимости содержания. С~этим введением содержания в логическое
рассмотрение предметом логики становятся уже не {\em вещи} (Dinge),
а~{\em суть} (die Sache), {\em понятие} вещей. Однако при этом нам могут также
напомнить о~том, что {\em имеется} множество понятий, множество сутей. Но ответ
на вопрос, чем ограничивается это множество, уже отчасти дан в том, чт\'{о} мы
сказали выше, а именно, что понятие, как мысль вообще, как всеобщее, есть
беспредельное сокращение по сравнению с единичными вещами, как они, их
множество, предносятся неопределенному созерцанию и представлению. Отчасти же
{\em некоторое} понятие есть вместе с тем, во-первых, понятие в самом себе, а
последнее имеется только в единственном числе и есть субстанциальная основа;
но, во-вторых, оно есть некоторое {\em определенное} понятие, каковая
определенность в нем есть то, чт\'{о} выступает как содержание; на самом же
деле определенность понятия есть определение формы указанного субстанциального
единства, момент формы, как целостности, момент {\em самого понятия,}
составляющего основу определенных понятий. Последнего мы не созерцаем и не
представляем себе чувственно; оно есть только предмет, продукт и содержание
{\em мышления} и в себе и для себя сущая суть, логос, разум того, чт\'{о} есть,
истина того, чт\'{о} носит название вещей. Логос-то уж менее всего д\'{о}лжно
оставлять вне науки логики. Поэтому не может зависеть от каприза, вводить ли
его в науку или оставлять его за ее пределами. Если те определения мысли,
которые суть только внешние формы, рассматриваются нами истинно в них самих, то
из этого может получиться в качестве вывода только их конечный характер,
неистинность их якобы самостоятельного бытия и, как их истина, понятие.
Поэтому, имея дело с определениями мысли, проходящими в нашем духе
инстинктообразно и бессознательно и остающимися беспредметными, незамеченными
даже тогда, когда они проникают в язык, логическая наука будет одновременно
также и реконструкцией тех определений мысли, которые выделены рефлексией и
фиксированы ею, как субъективные, внешние формы, формы на материи и на
содержимом.

Нет ни одного предмета, который, сам по себе взятый, столь поддавался бы такому
строгому, имманентно пластическому изложению, как развитие мышления в его
необходимости; нет ни одного предмета, который в такой мере требовал бы такого
изложения; наука о нем должна была бы в этом отношении превосходить даже
математику, ибо ни один предмет не имеет в самом себе этой свободы и
независимости. Такое изложение требовало бы, как это в своем роде имеет место в
последовательном движении математики, чтобы ни на одной ступени развития мысли
не встречались определения мысли и размышления, которые не порождались бы
непосредственно на этой ступени, а переходили бы в нее из предшествующих
ступеней. Но, конечно, приходится в общем отказаться от такого абстрактного
совершенства изложения. Уже одно то обстоятельство, что наука должна начинать с
совершенно простого и, стало быть, наиболее всеобщего и пустого, заставляет
требовать, чтобы изложение ее допускало как раз только такие выражения для
уяснения простого, которые и сами являются простыми, без всякого дальнейшего
добавления хотя бы одного слова; единственно, что по существу дела было бы
допустимо, "--- это отрицательные размышления, которые старались бы не
подпускать и удалить все то прочее, чт\'{о} представление или
недисциплинированное мышление могли бы сюда привнести. Но такие вторжения в
простой, имманентный ход развития мысли, однако, сами по себе случайны, и
старания отразить их, стало быть, сами страдают таким случайным характером; да
и помимо того было бы напрасно стремиться отразить {\em все} такого рода вторжения
именно потому, что они не касаются сущности дела, и для удовлетворения
требования систематичности здесь было бы по крайней мере желательно не гоняться
за полнотой. Но свойственные нашему современному сознанию беспокойство и
разбросанность не допускают, чтобы мы не принимали также более или менее во
внимание напрашивающиеся размышления и вторжения. Пластическое изложение
требует к тому же пластической способности воспринимания и понимания; но таких
пластических юношей и мужей, каких придумывает Платон, таких слушателей, столь
спокойно следящих лишь за ходом рассуждения, самоотреченно отказываясь от
высказывания {\em собственных} рефлексий и взбредших на ум соображений,
которыми доморощенное мышление нетерпеливо торопится показать себя, нельзя было
бы выставить в современном диалоге; еще того менее можно было бы рассчитывать
на таких читателей. Мне, напротив, слишком часто встречались такие яростные
противники, которые не хотели сообразить такой простой вещи, что взбредшие им в
голову мысли и возражения содержат в себе категории, которые сами суть
предположения и которые нужно подвергнуть критическому рассмотрению, прежде чем
пользоваться ими. Неспособность осознавать это заходит невероятно далеко; она
приводит к основному недоразумению, к тому плохому, т.~е. необразованному
способу рассуждения, который, встречаясь с рассмотрением определенной
категории, мыслит {\em нечто другое,} а не самоё эту категорию. Такая неспособность
осознавать тем более не может быть оправдана, что такое {\em другое} представляет собой
мыслительные определения и понятия, а в системе логики эти другие категории
как раз должны были тоже найти себе место, и там быть
самостоятельно рассмотрены. Более всего это бросается в глаза в
преобладающем числе возражений и нападок, вызванных первыми понятиями или
положениями логики, "--- {\em бытием, ничто} и {\em становлением,} каковое
последнее, хотя оно и само есть простое определение, тем не менее неоспоримо
"--- простейший анализ показывает это "--- содержит в себе указанные два
определения как моменты. Основательность, по-видимому, требует, чтобы прежде
всего было вполне исследовано начало как то, на чем строится все остальное, и
даже требует того, чтобы не шли дальше, прежде чем не будет доказано, что оно
прочно, и чтобы, напротив, если окажется, что это не так, было отвергнуто все
следующее за ним. Эта основательность обладает также и тем преимуществом, что
она необычайно облегчает дело мышления: она имеет перед собою все дальнейшее
развитие заключенным в этом зародыше и считает, что покончила со всем
исследованием, если покончила с этим зародышем, а с ним легче всего справиться
потому, что он есть наипростейшее, само простое; малостью требуемой работы
главным образом и подкупает эта столь самодовольная основательность. Это
ограничение [критики] простым оставляет свободный простор произвольному
мышлению, которое само не желает оставаться простым, а приводит относительно
этого простого свои соображения. Будучи вполне вправе сначала заниматься
{\em только} основоначалом (Prinzip) и, стало быть, не вдаваться в рассмотрение
{\em дальнейшего,} эта основательность сама действует в своем рассмотрении как
раз обратно этому, привлекает к рассмотрению {\em дальнейшее,} т.~е. другие
категории, чем ту, которая представляет собою только {\em основоначало,} другие
предпосылки и предрассудки. Для поучения критикуемого автора излагаются
предпосылки вроде того, что бесконечность отлична от конечности, содержание
есть нечто другое, чем форма, внутреннее есть не то, что внешнее, а
опосредствование также не есть непосредственность, как будто кто-то этого не
знает, и притом эти предпосылки не доказываются, а их рассказывают и уверяют,
что они справедливы. В~таком поучении, как в способе поведения, есть
"--- иначе это нельзя назвать "--- нечто глупое. По существу же здесь
отчасти неправомерно то, что такого рода положения только служат предпосылкой
и сразу принимаются; отчасти же и в еще большей мере здесь имеется
незнание того, что потребность и дело логического мышления -- "именно
исследовать, действительно ли конечное без бесконечного есть нечто
истинное, и суть ли {\em что-либо истинное,} а~также
{\em что-либо действительное} такая абстрактная бесконечность, бесформенное
содержание и лишенная содержания форма, такое внутреннее само по себе, которое не
имеет никакого внешнего проявления, внешнее без внутреннего и~т.~д. Но эти культура и дисциплина
мышления, благодаря которым достигается его пластическое отношение [к~предмету
научного рассмотрения] и преодолевается нетерпение вторгающейся рефлексии,
приобретаются единственно только движением вперед, изучением и проделыванием
всего пути развития.

При упоминании о платоновском изложении тот, кто в новейшее время работает над
самостоятельным построением философской науки, может ожидать, что ему напомнят
рассказ, согласно которому Платон семь раз перерабатывал свои книги
о~государстве. Напоминание об этом и сравнение, поскольку можно подумать, что
это напоминание заключает в себе таковое, могли бы только еще в большей мере
вызвать желание, чтобы автору произведения, которое, как принадлежащее
современному миру, имеет перед собою более глубокий принцип, более трудный
предмет и материал более широкого объема, был предоставлен свободный досуг для
перерабатывания его не семь, а семьдесят семь раз. Но, принимая во внимание,
что труд писался в условиях, диктовавшихся внешней необходимостью, что широта и
многосторонность присущих нашему времени интересов неизбежно отрывали от работы над ним,
что автору приходило даже в голову сомнение, оставляют ли еще повседневная
суета и оглушающая болтливость самомнения, довольная тем, что она
ограничивается этой суетой, место для соучастия в бесстрастной
тишине исключительно только мыслящего познания, "--- принимая во внимание
все это, автор, рассматривая свой труд под углом зрения величия задачи, должен
довольствоваться тем, чем этот труд мог стать в таких условиях.

{\em Берлин,} 7~ноября 1831~г.

\bigskip
