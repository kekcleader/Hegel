\chapter[{\em Третья глава} Существенное Отношение]
{Третья глава. Существенное Отношение}

Истиной явления служит {\em существенное отношение}. Содержание последнего
обладает непосредственной самостоятельностью и притом {\em сущей}
непосредственностью и {\em рефлектированной} непосредственностью или
тождественной с собой рефлексией. Вместе с тем оно в этой самостоятельности
есть некоторое относительное содержание, будучи безоговорочно лишь
рефлексией в свое другое или единством соотношения со своим другим. В~этом
единстве самостоятельное содержание есть некоторое положенное, снятое; но
как раз это единство и составляет его существенность и самостоятельность;
эта рефлексия в другое есть рефлексия в себя само. Отношение имеет стороны,
так как оно есть рефлексия в другое; таким образом, оно имеет различие
самого себя в себе же; и стороны суть самостоятельное устойчивое наличие,
так как они в своей безразличной разности друг относительно друга преломлены
в самих себе, так что устойчивое наличие каждой из них вместе с тем имеет
свое значение лишь в соотношении с другой или в их отрицательном единстве.

Поэтому, хотя существенное отношение еще не есть истинное {\em третье}
к {\em сущности} и {\em существованию}, но оно уже содержит в себе
определенное соединение обоих. Сущность реализована в нем таким образом,
что она имеет своим устойчивым наличием самостоятельно существующие, и
последние возвратились из своего безразличия в свое существенное единство,
так что они имеют своим устойчивым наличием лишь это единство. Рефлексивные
определения положительного и отрицательного равным образом оказываются
рефлектированными в себя, лишь как рефлектированные в свою
противоположность; но они не имеют никакого другого определения, кроме
этого их отрицательного единства; напротив, существенное отношение имеет
своими сторонами такие, которые положены как самостоятельные тотальности.
Оно есть то же самое противоположение, что и противоположение
положительного и отрицательного, но вместе с тем как вывороченный наизнанку
мир. Сторона существенного отношения есть некоторая тотальность, которая,
однако, как представляющая собою по существу некоторое противоположное,
имеет нечто {\em потустороннее} по отношению к самой
себе; эго противоположное есть лишь явление; его существование есть скорее
существование не его, а его другого. Оно есть поэтому некоторое
надломленное в самом себе; но эта его снятость состоит в том, что оно есть
единство себя самого и своего другого, есть, следовательно, некоторое
целое, и именно поэтому оно обладает самостоятельным существованием и есть
существенная рефлексия в себя.

Таково {\em понятие} отношения. Но пока что содержащееся
в нем тождество еще не полно; целокупность, которую каждое относительное
представляет собою в себе самом, есть сперва лишь нечто внутреннее; сторона
отношения ближайшим образом положена в {\em одном} из
определений отрицательного единства; собственная самостоятельность каждой
из двух сторон есть то, чт\'{о} составляет форму отношения. Его тождество есть
поэтому лишь некоторое {\em соотношение}, и
самостоятельность этого тождества имеет место вне самого этого соотношения,
а именно, в сторонах; налицо еще нет рефлектированного единства этого
тождества и самостоятельных существований, еще нет
{\em субстанции}. "--- Поэтому, хотя и выяснилось, что
понятие отношения состоит в том, что последнее есть единство
рефлектированной и непосредственной самостоятельности, однако сначала это
{\em понятие} само еще {\em непосредственно}, его моменты суть поэтому
непосредственные друг относительно друга, и единство есть их существенное
соотношение, которое лишь тогда есть истинное, соответствующее понятию
единство, когда и поскольку оно реализовало себя, а именно,
{\em положило} себя через свое движение, как то единство.

Существенное отношение есть поэтому непосредственно отношение
{\em целого} и {\em частей}
"--- соотношение рефлектированной и непосредственной самостоятельности, так
что обе они вместе с тем имеют бытие как взаимно обусловливающие и
предполагающие друг друга.

В этом отношении ни одна из сторон еще не положена как момент другой, их
тождество поэтому само есть некоторая сторона, или, иначе говоря, оно не
есть их отрицательное единство. Поэтому это отношение,
{\em во-вторых}, переходит в такое отношение, где одна
сторона есть момент другой и имеет бытие в ней, как в своем основании, в
истинно самостоятельном (субстрате) обоих; это "--- отношение
{\em силы и ее проявления}.

{\em В-третьих}, еще имеющееся неравенство этого
соотношения снимает себя, и последнее отношение есть отношение
{\em внутреннего и внешнего}. "--- В этом ставшем
совершенно формальным различии само отношение идет ко дну, и выступает
{\em субстанция} или {\em действительное}, как
{\em абсолютное} единство непосредственного и рефлектированного существования.


\section[А. Отношение целого и частей]{А. Отношение целого и частей}

1. Существенное отношение содержит в себе, {\em во-первых},
{\em рефлектированную в себя} самостоятельность
существования; таким образом, оно есть {\em простая
форма}, определения которой, хотя и суть также существования, но вместе с
тем суть положенные существования, моменты, удерживаемые в единстве. Эта
рефлектированная в себя самостоятельность есть вместе с тем рефлексия в ее
противоположное, а именно, в {\em непосредственную}
самостоятельность; и ее устойчивое наличие есть по существу в такой же
мере, в какой оно есть собственная самостоятельность, также и это тождество
со своим противоположным. "--- Как раз тем самым непосредственно положена,
{\em во-вторых}, также и другая сторона, "--- положена
непосредственная самостоятельность, которая, определенная как
{\em другое}, представляет собою некоторое сложное
многообразие внутри себя, но таким образом, что это многообразие содержит в
себе по существу {\em также} и соотношение другой
стороны, единство рефлектированной самостоятельности. Первая сторона,
{\em целое}, есть та самостоятельность, которая
составляла в-себе-и-для-себя-сущий мир; другая сторона,
{\em части}, есть непосредственное существование,
которое было являющимся миром. В~отношении целого и частей обе стороны суть
такого рода самостоятельности, но таким образом, что каждая имеет другую
светящейся в ней, и вместе с тем имеет бытие только как это тождество
обеих. Так как существенное отношение есть пока что лишь первое,
непосредственное отношение, то отрицательное единство и положительная
самостоятельность соединены между собой посредством некоторого
<<{\em также}>>; обе стороны, хотя и положены как
{\em моменты}, но вместе с тем положены {\em также и} как существующие
{\em самостоятельности}. "--- Что обе стороны положены как
моменты, распределено поэтому таким образом, что, во-первых,
{\em целое}, рефлектированная самостоятельность,
выступает как существующее, а в ней другая, непосредственная
самостоятельность выступает как момент; здесь
{\em целое} составляет единство обеих сторон,
{\em основу}, а непосредственное существование имеет
бытие как {\em положенность}. "--- Наоборот, на другой
стороне, т.~е. на стороне {\em частей}, самостоятельной
основой служит непосредственное, внутри себя многообразное существование,
тогда как рефлектированное единство, целое, есть лишь внешнее соотношение.

2. Это отношение содержит, стало быть, в себе как самостоятельность сторон,
так и их снятость, содержит и то, и другое безоговорочно в едином
соотношении. Целое есть самостоятельное, части суть лишь моменты этого
единства; но в той же мере они суть тоже нечто самостоятельное, а их
рефлектированное единство лишь момент; и каждая из обеих сторон есть в
своей {\em самостоятельности} безоговорочно {\em относительное} своего
иного\pagenote{В~немецком
тексте: <<das Relative eines Ändern>>. По-видимому, это опечатка вместо: <<das
Relative seines Ändern>>.}.
Это отношение есть поэтому в себе самом непосредственное противоречие и
снимает себя.

При более близком рассмотрении оказывается, что
{\em целое} есть рефлектированное единство, которое
само по себе обладает самостоятельным устойчивым наличием; но это его
устойчивое наличие также и оттолкнуто от него; как отрицательное единство,
целое есть отрицательное соотношение с самим собой; таким образом, оно
отчудилось от себя; оно имеет свое {\em устойчивое
наличие} в своем противоположном, в многообразной непосредственности,
{\em в частях}. {\em Целое состоит
поэтому из частей}, так что оно без них не есть нечто. Оно, следовательно,
есть все отношение и самостоятельная тотальность; но как раз поэтому оно
есть лишь некоторое относительное, ибо то, что делает его тотальностью,
есть, наоборот, его {\em другое}, части, и оно имеет
свое устойчивое наличие не в самом себе, а в своем другом.

Части равным образом представляют собою все отношение. Они суть
непосредственная самостоятельность {\em против}
рефлектированной и не имеют своего устойчивого наличия в целом, а суть сами
по себе. Они, далее, содержат в себе это целое как свой момент; оно
составляет их соотношение; без целого нет частей. Но так как они суть нечто
самостоятельное, то это соотношение есть лишь некоторый внешний момент, к
которому они сами по себе безразличны. Однако вместе с тем части, как
многообразное существование, рушатся в самих себе, ибо это многообразное
существование есть лишенное рефлексии бытие; они имеют свою
самостоятельность лишь в рефлектированном единстве, которое есть как это
единство, так и существующее многообразие; это означает, что они обладают
самостоятельностью лишь {\em в целом}, которое, однако,
вместе с тем есть {\em другая} по отношению к частям самостоятельность.

Целое и части поэтому взаимно {\em обусловливают} друг
друга; но рассматриваемое здесь отношение вместе с тем стоит выше
соотношения {\em обусловленного} и
{\em условия} друг с другом, как оно определилось в
предшествующем. Эго соотношение здесь {\em реализовано;} а именно,
{\em положено}, что условие есть существенная
самостоятельность обусловленного таким образом, что оно
{\em предполагается} последним. Условие как таковое
есть лишь {\em непосредственное}, и {\em предположено} лишь
{\em в~себе}. Целое же есть, правда, условие частей, но само оно вместе
с тем непосредственно подразумевает, что и оно само имеется лишь постольку,
поскольку оно имеет своей предпосылкой части. Ввиду того, что обе части
отношения положены, таким образом, как взаимнообусловливающие друг друга,
то каждая из сторон есть непосредственная самостоятельность в себе же
самой, но ее самостоятельность вместе с тем также и опосредствована или
положена через другую. Все {\em отношение} есть
благодаря этой взаимности возвращение обусловливания в себя само, есть
не-относительное, {\em безусловное}.

Поскольку теперь каждая из сторон отношения обладает своей
самостоятельностью не в самой себе, а в своей другой, постольку имеется
налицо лишь одно тождество обеих, в котором обе суть лишь моменты;
поскольку же каждая из них самостоятельна в себе самой, постольку они суть
два самостоятельных существования, безразличных друг к другу.

С первой точки зрения, с точки зрения существенного тождества этих сторон,
{\em целое равно частям} и {\em части равны целому}. Нет ничего в целом, чего
не было бы в частях, и нет ничего в частях, чего нет в целом. Целое не есть
абстрактное единство, а единство некоторого
{\em разного многообразия;} но это единство как то, в чем
{\em многообразное} соотносится одно с другим, есть
та {\em определенность} этого многообразия, в силу
которой оно есть часть. Отношение, следовательно, обладает одним
нераздельным тождеством и лишь {\em единой} самостоятельностью.

Но, далее, целое равно частям; однако оно равно {\em им}
не как частям; целое есть рефлектированное единство, части же составляют
определенный момент или {\em инобытие} единства, и суть
разное многообразное. Целое равно им не как этому самостоятельному разному,
а как им, {\em вместе} взятым. Это их <<вместе>> есть
однако не~что иное, как их единство, целое как таковое. Целое,
следовательно, равно в частях лишь себе самому, и равенство его и частей
служит выражением лишь той тавтологии, что {\em целое
как целое равно} не частям, а {\em целому}.

Обратно, части равны целому; но так как они суть момент инобытия в них же
самих, то они равны ему но как единству, а так, что
{\em одно} из его многообразных определений приходится
на часть, или, иначе говоря, они равны {\em ему} как
{\em многообразному;} это означает, что они равны ему
как {\em разделенному на части целому}, т.~е. как
частям. Здесь, следовательно, имеется та же самая тавтология, а именно, что
{\em части как части} равны не {\em целому} как таковому, а в этом целом
"--- {\em самим себе}, т.~е. {\em частям}.

Целое и части, таким образом, безразлично выпадают друг из друга; каждая из
этих сторон соотносится лишь с собой. Но удерживаемые таким образом одно
вне другого, они разрушают сами себя. Такое целое, которое безразлично к
частям, есть {\em абстрактное}, неразличенное внутри
себя {\em тождество;} последнее есть целое лишь как
внутри {\em самого себя различенное}, и притом
различенное внутри себя так, что эти многообразные определения
рефлектированы в себя и обладают непосредственной самостоятельностью.
А~рефлексивное тождество явило себя через свое движение как то, что имеет
своей истиной эту {\em рефлексию в свое другое}. "---
Равным образом и части как безразличные к единству целого суть лишь
несоотнесенное многообразное, {\em другое внутри себя},
которое как таковое есть другое самого себя и лишь упраздняющее себя. "--- Это
соотношение с собой каждой из двух сторон есть ее самостоятельность; но эта
ее самостоятельность, которой каждая обладает
{\em особо}, есть скорее отрицание ее самой. Поэтому
каждая обладает своей самостоятельностью не в себе самой, а в другой; эта
другая, составляющая устойчивое наличие, есть ее предположенное
непосредственное, которое {\em должно} быть первым и ее
началом; но это первое каждой из сторон само есть лишь нечто такое, что не
есть первое, а имеет свое начало в другом.

Истина отношения состоит, следовательно, в
{\em опосредствовании;} его сущностью служит
отрицательное единство, в котором сняты как рефлектированная, так и сущая
непосредственность. Отношение есть противоречие, возвращающееся в свое
основание, в единство, которое, как возвращающееся, есть рефлектированное
единство, но поскольку это последнее положило себя вместе с тем и как
снятое, оно соотносится отрицательно с собой самим, снимает себя и делает
себя сущей непосредственностью. Но это его отрицательное соотношение,
поскольку оно есть нечто первое и непосредственное, опосредствовано лишь
через свое другое и есть вместе с тем нечто положенное. Это другое, т.~е.
сущая непосредственность, равным образом есть лишь снятое; ее
самостоятельность есть некоторое первое, но лишь для того, чтобы исчезнуть,
и обладает таким наличным бытием, которое положено и опосредствовано.

В этом определении отношение уже больше не есть отношение
{\em целого} и {\em частей;}
непосредственность, которой обладали его стороны, перешла в положенность и
опосредствование; каждая сторона, поскольку она непосредственна, положена,
как снимающая себя и переходящая в другую, а поскольку она сама есть
отрицательное соотношение, она положена так, что она вместе с тем
обусловлена другой стороной как ее положительным; равно как ее
непосредственный переход есть вместе с тем нечто опосредствованное, а
именно, снятие, полагаемое другой стороной. "--- Таким образом, отношение
целого и частей перешло в отношение {\em силы и ее проявления}.

\hegremark[Примечание]%
  {Бесконечная делимость}%
  {[Бесконечная делимость]}

Выше, по поводу понятия количества, мы рассмотрели
{\em антиномию бесконечной делимости материи}\pagenote{См. часть первая,
стр.~\pageref{bkm:bm88a}---\pageref{bkm:bm88b}.}.
Количество есть единство непрерывности и дискретности; оно содержит в
{\em самостоятельном} <<одном>> его
{\em слиянность} с другими <<одними>>, а в этом
непрерывно {\em продолжающемся тождестве} с собой
содержит также и {\em отрицание этого тождества}. Когда
непосредственное соотношение этих моментов количества получает выражение в
виде существенного отношения целого и частей, причем количественное
<<{\em одно}>> выступает как {\em часть}, а его
{\em непрерывность} как {\em целое}, cоставленное из частей, то антиномия
состоит в том противоречии, которое встретилось нам при рассмотрении
отношения целого и частей и было там разрешено. "--- А именно, целое и части
столь же существенно соотнесены друг с другом и составляют лишь
{\em одно} тождество, сколь и безразличны друг к другу
и обладают самостоятельным устойчивым наличием. Отношение представляет
собой поэтому ту антиномию, что один момент тем самым, что он освобождается
от другого момента, непосредственно приводит за собой последний.

Следовательно, если определить существующее как целое, то оно обладает
частями, и части составляют его устойчивое наличие; единство целого есть
лишь некоторое положенное соотношение, некоторая внешняя
{\em составность}, не касающаяся самостоятельно
существующего. Поскольку последнее есть часть, оно не есть целое, не есть
составное, а есть тем самым {\em простое}. Но так как
соотношение с некоторым целым внешне ему, то это соотношение его не
касается; самостоятельное, стало быть, не есть часть также и в себе; ибо
оно есть часть лишь через вышеуказанное соотношение. Но так как оно не есть
часть, то оно есть целое, ибо здесь имеется лишь это отношение целого и
частей, и самостоятельное есть одно из этих двух. А~раз оно есть целое,
то оно опять-таки составно; оно опять состоит из частей и
{\em так далее до бесконечности}. "--- Эта бесконечность
состоит не в чем ином, как в вековечном чередовании обоих определений
отношения, в каждом из которых непосредственно возникает другое, так что
положенность каждого из них есть исчезновение его самого. Если определить
материю как целое, то она состоит из частей, а в последних целое становится
несущественным соотношением и исчезает. Но и часть, взятая аналогичным
образом отдельно, также не есть часть, а есть целое. "--- Антиномия этого
умозаключения в совершенно сжатой форме получает, собственно говоря,
следующее выражение: так как целое не есть самостоятельное, то часть есть
самостоятельное; но так как она самостоятельна лишь
{\em без целого}, то она самостоятельна
{\em не}~как часть, а наоборот,
{\em как целое}. Бесконечность получающегося здесь
прогресса есть неспособность свести вместе обе мысли, содержащиеся в этом
опосредствовании, а именно неспособность понять, что каждое из этих двух
определений благодаря своей самостоятельности и отделению от другого
переходит в несамостоятельность и в другое определение.


\section[В. Отношение силы и ее проявления во-вне]
{В. Отношение силы и ее проявления во-вне}

{\em Сила} есть
отрицательное единство, в которое разрешилось противоречие целого и частей,
есть истина того первого отношения. Целое и части есть то чуждое мысли
отношение, на которое представление набредает прежде всего; или, взятое
объективно, это отношение есть мертвый, механический агрегат, который хотя
и обладает определениями формы, благодаря чему многообразие его
самостоятельной материи соотносится в некотором единстве, но обладает ими
таким образом, что это единство остается внешним для многообразия. "---
Отношение же {\em силы} есть более высокое возвращение
в себя, в которое единство целого, составлявшее соотношение
самостоятельного инобытия, перестает быть чем-то внешним для этого
многообразия и безразличным к нему.

Существенное отношение определилось теперь таким образом, что
непосредственная и рефлектированная самостоятельность положены в
нем\pagenote{В~немецком
тексте: <<in derselben>>. По-видимому, это опечатка вместо: <<in demselben>>.}
как снятые или как моменты, тогда как в предшествующем отношении они были
сторонами или крайними терминами, устойчиво наличными сами по себе. Это
означает, {\em во-первых}, что рефлектированное
единство и его непосредственное наличное бытие, поскольку оба суть первые и
непосредственные, снимают себя в самих себе и переходят в свое другое;
первое, т.~е. {\em сила}, переходит в свое
{\em проявление во-вне}, и внешнее есть нечто
исчезающее, возвращающееся в силу, как в свое основание, и имеет бытие лишь
как носимое и положенное этой силой. {\em Во-вторых},
этот переход есть не только становление и исчезновение, а отрицательное
соотношение с собой, или, иначе говоря, {\em изменяющее
свое определение} рефлектировано вместе с тем в этом изменении внутрь себя
и сохраняет себя; движение силы есть не столько некоторый
{\em переход}, сколько то, что она сама себя
{\em переправляет на другую сторону} (über setzt) и в
этом положенном ею самою изменении остается тем, что она есть. "---
{\em В-третьих}, это {\em рефлектированное}, соотносящееся с собой единство,
само также снято и представляет собою момент; оно опосредствовано через
свое другое и имеет последнее {\em условием;} его
отрицательное соотношение с собой, которое есть нечто первое и которое
начинает движение своего {\em спонтанного} (aus sich) перехода,
точно так же имеет некоторую предпосылку, которой оно
{\em возбуждается}, и некоторое другое, с которого оно начинает.


\subsection[а) Обусловленность силы]{а) Обусловленность силы}

Рассматриваемая в ее дальнейших определениях,
сила, {\em во-первых}, содержит в себе момент сущей
непосредственности; сама же она, напротив, определена как отрицательное
единство. Но последнее в определении непосредственного бытия есть некоторое
{\em существующее} {\em нечто}. Это
нечто, так как оно есть отрицательное единство в виде чего-то
непосредственного, представляется первым. Напротив, сила, так как она есть
нечто рефлектированное, представляется положенностью и постольку
принадлежащей к существующей вещи или к некоторой материи. Это не означает,
что она есть {\em форма} этой вещи и что вещь
определена ею, а означает, что вещь как нечто непосредственное безразлична
к ней. "--- По этому определению в вещи нет никакого основания для обладания
некоторой силой; напротив, сила как сторона положенности имеет своей
существенной предпосылкой вещь. Поэтому, когда ставится вопрос, каким
образом вещь или материя доходят до того, чтобы
{\em обладать} силой, то последняя представляется
внешне связанной с вещью и {\em втиснутой} в нее
каким-то чужим насилием.

Как такое непосредственное устойчивое наличие сила есть вообще некоторая
{\em спокойная определенность вещи}, не нечто
проявляющее себя во вне, а непосредственно нечто внешнее. Так, например,
силу обозначают также и как материю, и вместо магнетической, электрической
и~т.~д. сил принимаются магнетическая, электрическая и~т.~д. материи, или,
вместо пресловутой {\em притягательной силы},
принимается существование некоторого тонкого
{\em эфира}, связывающего все. "--- В материи, "--- вот во
что разрешается недеятельное, бессильное отрицательное единство вещи, "--- и
эти материи были рассмотрены нами выше.

Но сила содержит в себе непосредственное существование как момент, как нечто
такое, что хотя и есть условие, но преходит и снимается, следовательно, не
как существующую вещь. Сила есть, далее, не отрицание как определенность, а
отрицательное, рефлектирующее себя внутрь себя единство. Вещь, в которой,
как представлялось раньше, пребывает сила, здесь, стало быть, уже не имеет
никакого значения; сама сила есть скорее полагание внешности, которая
выступает в явлении как существование. Она, следовательно, и не есть также
только некоторая определенная материя; такая самостоятельность давно
перешла в положенность и в явление.

{\em Во-вторых}, сила есть единство рефлектированного и
непосредственного устойчивого наличия, или, другими словами, единство
единства формы и внешней самостоятельности. Она есть и то и другое разом;
она есть соприкосновение таких определений, из которых одно имеется
постольку, поскольку нет другого; она есть тождественная с собой
положительная рефлексия и подвергшаяся отрицанию рефлексия. Таким образом,
сила есть отталкивающееся от себя самого противоречие; она
{\em деятельна} или, иначе сказать, она есть
соотносящееся с собой отрицательное единство, в котором рефлектированная
непосредственность или существенное внутри-себя-бытие положено таким, что
оно дано (ist) лишь как снятое или момент и, стало быть, поскольку оно
отличается от непосредственного существования, переходит в последнее.
Следовательно, сила как определение рефлектированного единства целого
положена как то, что {\em само из себя} становится
существующим внешним многообразием.

Но, в-{\em третьих}, сила пока что есть лишь
{\em в-себе-сущая} и непосредственная деятельность; она
есть рефлектированное единство и столь же существенно оказывается
{\em отрицанием его;} так как она разнится от
последнего, но есть лишь тождество самой себя и своего отрицания, то она
существенным образом соотнесена с последним, как с внешней ей
непосредственностью, и имеет последнюю {\em предпосылкой} и {\em условием}.

Эта предпосылка не есть находящаяся рядом с нею вещь; такого рода
безразличная самостоятельность снята в силе; как условие последней она есть
{\em некоторое другое по отношению к ней
самостоятельное}. Но так как она есть не вещь, а самостоятельная
непосредственность определила себя здесь вместо с тем как соотносящееся с
самим собою отрицательное единство, {\em то она сама
есть сила}. "--- Деятельность силы обусловлена самой собой как тем, что есть
другое для себя, обусловлена некоторой силой.

Сила есть, таким образом, такое отношение, в котором каждая сторона есть то
же самое, что и другая сторона. Сторонами отношения оказываются силы,
которые притом существенно соотносятся друг с другом. "--- Они, далее, суть
ближайшим образом лишь разные вообще; единство их отношения есть пока что
лишь {\em внутреннее, в-себе-сущее} единство.
Обусловленность некоторой другой силой есть, таким образом,
{\em в~себе} делание самой силы, или, иначе сказать,
она постольку есть лишь {\em пред}{}-полагающее, лишь
отрицательно соотносящееся {\em с собой} делание; эта
другая сила еще лежит {\em по ту сторону ее полагающей}
деятельности, а именно, по ту сторону рефлексии, непосредственно
{\em возвращающейся} в своем процессе определения
{\em назад в себя}.


\subsection[b) Возбуждение силы]{b) Возбуждение силы}

Сила обусловлена, так как момент непосредственного существования, который она
содержит в себе, имеет бытие лишь как некоторое {\em положенное;} но так как он
есть вместе с тем нечто непосредственное, то он есть некоторое предположенное,
в котором сила подвергает самое себя отрицанию. Имеющаяся для силы внешность
есть поэтому {\em сама ее собственная пред-полагающая деятельность},
которая ближайшим образом положена, как некоторая {\em другая сила}.

Это {\em пред-полагание} есть, далее, взаимное. Каждая
из обеих сил содержит в себе рефлектированное в себя единство, как снятое,
и есть поэтому предполагающая; она полагает самое себя, как внешнюю; этот
момент внешности есть {\em ее собственный} момент; но
так как она есть также рефлектированное в себя единство, то она полагает
вместе с тем эту свою внешность не в себе самой, а как некоторую другую силу.

Но внешнее как таковое есть само себя снимающее; далее, рефлектирующая себя
в себя деятельность существенно соотнесена с тем внешним, как с чем-то
другим для нее, но точно так же и как с чем-то {\em ничтожным в себе} и
{\em тождественным с нею}. Так как пред-полагающая
деятельность есть также и рефлексия в себя, то она есть снятие того ее
отрицания, и она полагает его, как самое себя или как
{\em свое} внешнее. Таким образом, сила, как
обусловливающая, есть взаимно {\em толчок} для другой
силы, по отношению к которому она деятельна. Ее поведение не есть пассивная
определяемость, так что благодаря этому в нее входило бы нечто другое, а
толчок лишь {\em возбуждает} ее. Она есть в ней же
самой отрицательность себя; отталкивание ее от себя есть ее собственное
полагание. Ее делание состоит, следовательно, в том, что она снимает то
обстоятельство, что тот толчок есть некоторое внешнее; она делает его голым
толчком и полагает его как собственное отталкивание себя самой от себя, как
{\em ее собственное проявление во вне}.

Проявляющая себя во вне сила есть, следовательно, то же самое, что сначала
было лишь пред-полагающей деятельностью, а именно, она есть то, чт\'{о} делает
себя внешним; но сила, как проявляющая себя во вне, есть вместе с тем такая
деятельность, которая подвергает отрицанию внешность и
{\em полагает} ее как нечто свое собственное. Если мы в
этом рассмотрении начинаем с силы, поскольку она есть отрицательное
единство себя самой и тем самым предполагающая рефлексия, то это то же
самое, как если бы при рассмотрении проявления силы во вне мы начинали с
возбуждающего толчка. Сила, таким образом, определена сначала
{\em в своем понятии}, как снимающее себя тождество,
{\em а в своей реальности} одна из двух сил определена
как возбуждающая, а другая "--- как возбуждаемая. Но понятие силы есть вообще
тождество полагающей и пред-полагающей рефлексии, или рефлектированного и
непосредственного единства, и каждое из этих определений есть безоговорочно
лишь момент, находится в единстве, и тем самым дано как опосредствованное
другим. Однако вместе с тем в обеих находящихся во взаимоотношении силах не
содержится никакого такого определения, которое указало бы, какая из них
есть возбуждающая и какая возбуждаемая, или, вернее, каждой из этих сил
одинаковым образом присущи оба определения формы. Но это тождество есть не
только внешнее тождество сравнения, а существенное их единство.

А именно, одна сила определена ближайшим образом как
{\em возбуждающая}, а другая как {\em возбуждаемая;} эти определения формы
представляются, таким образом, непосредственными, имеющимися сами по себе
различиями этих двух сил. Но они существенно опосредствованы. Одна сила
возбуждается; этот толчок есть некоторое положенное в нее
{\em извне} определение. Но сила сама есть нечто
пред-полагающее; она по существу рефлектирует себя в себя и снимает то
обстоятельство, что толчок есть нечто внешнее. То обстоятельство, что она
возбуждается, есть поэтому ее собственное действие, или, иначе говоря, ею
самой определено, что другая сила есть вообще другая и возбуждающая.
Возбуждающая сила соотносится со своей другой отрицательно, так что она
снимает внешность последней; она есть постольку полагающая; но она есть
таковая лишь при том предположении, что она имеет рядом с собой другую
силу; т.~е. она сама есть возбуждающая лишь постольку, поскольку она
содержит в себе некоторую внешность и, следовательно, поскольку она
возбуждается. Или, иначе говоря, она есть возбуждающая лишь постольку,
поскольку она возбуждается к тому, чтобы быть возбуждающей. Тем самым
первая, обратно, возбуждается лишь постольку, поскольку она сама возбуждает
другую к тому, чтобы та возбуждала ее, т.~е. первую. Каждая из этих двух
сил получает, следовательно, толчок от другой; но толчок, который она
сообщает как деятельная, состоит в том, что она получает толчок от другой
силы; толчок, который она получает, возбуждается ею же самой. То и другое,
сообщенный и полученный толчок, или деятельное проявление во вне и
пассивная внешность, не есть поэтому некоторое непосредственное, а
опосредствовано, и притом каждая из этих двух сил тем самым сама есть та
определенность, которую другая сила имеет по отношению к ней,
опосредствована другой силой, и это опосредствующее другое есть опять-таки
ее собственное определяющее полагание.

Таким образом, то обстоятельство, что силе сообщается толчок некоторой
другой силой, что она постольку ведет себя
{\em пассивно}, но опять-таки переходит от этой
пассивности в активность, есть возвращение силы в самое себя. Она проявляет
себя во вне. Проявление во вне есть реакция в том смысле, что она полагает
внешность, как свой собственный момент, и тем самым снимает то
обстоятельство, что она была возбуждена некоторой другой силой. То и другое
"--- проявление силы во вне, через которое она сообщает себе своей
отрицательной деятельностью (направленной на себя самое) некоторое наличное
бытие-для-другого, и бесконечное возвращение в этой внешности к себе самой,
так что она этим лишь соотносится с собой, "--- есть поэтому одно и то же.
Пред-полагающая рефлексия, которой принадлежат обусловленность и толчок,
есть поэтому непосредственно также и возвращающаяся в себя рефлексия, и
деятельность есть по существу реагирующая деятельность, направлена
{\em против себя}. Полагание толчка или внешнего само
есть снятие его и, обратно, снятие толчка есть полагание внешности.


\subsection[с) Бесконечность силы]{с) Бесконечность силы}

Сила {\em конечна},
поскольку ее моменты еще имеют форму непосредственности; ее предполагающая
и ее соотносящаяся с собой рефлексии различны в этом определении; первая
выступает как некоторая существующая особо внешняя сила, а другая "--- в
соотношении с ней как пассивная. Сила, таким образом, по форме обусловлена,
а по содержанию равным образом ограничена; ибо определенность по форме
означает также и ограничение содержания. Но деятельность силы состоит в
том, что она {\em проявляет себя во вне}, т.~е., как
это выяснилось, в том, что она снимает внешность и определяет ее как то, в
чем сила тождественна с собой. Следовательно, на самом деле сила проявляет
то обстоятельство, что ее соотношение с другим есть ее соотношение с самой
собой, что ее пассивность состоит в самой ее активности. Толчок, которым
она возбуждается к деятельности, есть ее собственное возбуждение;
внешность, привходящая в нее, есть не нечто непосредственное, а нечто ею
опосредствованное; равно как и ее собственное существенное тождество с
собой не непосредственно, а опосредствовано его отрицанием; или, иначе
говоря, сила проявляет во вне то обстоятельство, что
{\em ее внешность тождественна с ее внутренностью}\pagenote{Ср. замечание
Энгельса о том, что <<сила имеет точно такую же величину, как и ее
проявление во-вне, ибо в них обоих совершается {\em одно и то же
движение}>> ({\em Engels}, Herrn E.~Dührings Umwälzung der Wissen\-schaft.
Moskau "--- Leningrad 1935, S.~64).}.


\section[С. Отношение внешнего и внутреннего]
{С. Отношение внешнего и внутреннего}

1. Отношение целого и частей есть
непосредственное отношение; рефлектированная и сущая непосредственность
имеют поэтому внутри него каждая свою собственную самостоятельность; но так
как они находятся в существенном отношении, то их самостоятельность есть
лишь их отрицательное единство. Это теперь положено в проявлении силы во
вне рефлектированное единство есть по существу становление другим как
перевод себя самого во внешность; но последняя столь же непосредственно
вобрана обратно в рефлектированное единство; различие самостоятельных сил
снимает себя; проявление силы во вне есть лишь некоторое опосредствование
рефлектированного единства с самим собой. Имеется лишь пустое, прозрачное
различие, видимость, но эта видимость есть опосредствование, которое само
есть самостоятельное устойчивое наличие. Не только противоположные
определения снимают себя в них же самих, и их движение есть не только
переход, но отчасти и та непосредственность, с которой начали и от которой
перешли в инобытие, сама оказывается лишь положенной непосредственностью,
отчасти же благодаря этому каждое из определений в своей непосредственности
уже есть единство со своим другим, и переход благодаря этому есть
безоговорочно также и полагающее себя возвращение в себя.

{\em Внутреннее} определено как форма
{\em рефлектированной непосредственности} или сущности
в противоположность {\em внешнему} как форме
{\em бытия}, но оба суть лишь единое тождество. "--- Это
тождество есть, {\em во-первых}, плотное единство
обоих, как содержательная основа или {\em абсолютная
мыслимая вещь}, в которой оба определения суть безразличные, внешние
моменты. Постольку оно есть содержание и целостность, которая есть
внутреннее, становящееся также и внешним, однако таким образом, что этим
оно не оказывается чем-то ставшим или перешедшим, а остается равным самому
себе. Внешнее согласно этому определению не только
{\em одинаково} по содержанию с внутренним, но оба суть
лишь {\em одна мыслимая вещь}. "--- Но эта мыслимая вещь,
как {\em простое тождество} с собой, разнится от
{\em своих определений формы}, или, иначе сказать,
последние внешни ей; постольку она сама есть некоторое внутреннее,
разнящееся от ее внешности. Но эта внешность состоит в том, что ее
составляют эти самые два определения, т.~е. внутреннее и внешнее. Но
мыслимая вещь сама есть не~что иное, как единство обоих. Тем самым обе
стороны суть по содержанию опять то же самое. Но в мыслимой вещи они
имеются, как взаимно проникающее себя тождество, как содержательная основа.
Во внешности же, как формы мыслимой вещи, они безразличны к этому
тождеству, и тем самым и друг к другу.

2. Они, таким образом, суть разные определения формы, имеющие тождественную
основу не в них самих; а в некотором другом, "--- рефлексивные определения,
имеющие бытие сами по себе: внутреннее как форма рефлексии в себя, форма
существенности, а внешнее, как форма рефлектированной в другое
непосредственности или несущественности. Однако природа отношения показала,
что эти определения составляют безоговорочно лишь одно тождество. Сила в ее
проявлении во вне заключается в том, что предполагающий и возвращающийся в
себя процессы определения суть одно и то же. Поэтому, поскольку внутреннее
и внешнее рассматривались ранее как определения формы, они суть,
{\em во-первых}, лишь сама простая форма, а,
{\em во-вторых}, ввиду того, что они при этом
определены вместе с тем и как противоположные, их единство есть чистое
{\em абстрактное опосредствование}, в котором одно
определение есть {\em непосредственно} другое, и притом
именно {\em потому}, что оно есть первое. Таким
образом, внутреннее есть непосредственно {\em лишь}
внешнее, и оно есть {\em определенность внешности}
потому, что оно есть внутреннее; наоборот, внешнее, есть
{\em лишь} внутреннее, так как оно есть
{\em лишь} внешнее. "--- А именно, поскольку это единство
формы содержит в себе свои оба определения как
{\em противоположные} определения, их тождество есть
лишь этот переход, а в последнем "--- лишь {\em другое} их
обоих, а не их {\em содержательное} тождество. Или,
иначе сказать, это удерживание формы есть вообще сторона
{\em определенности}. Положена с этой стороны не
реальная тотальность целого, а тотальность или сама мыслимая вещь лишь в
{\em определенности} формы; так как последняя есть
безоговорочно связанное вместе единство обоих противоположных определений,
то когда берут сначала одно из них "--- и безразлично, какое это из них, "---
следует сказать об основе или о мыслимой вещи, что она именно
{\em поэтому} столь же существенно дана (ist) и в
другой определенности, но равным образом {\em лишь} в
другой, в том же смысле, в котором сперва было сказано, что она дана (ist)
{\em лишь} в первой. "---

Таким образом, нечто, которое {\em пока что есть лишь
некоторое внутреннее}, именно поэтому есть {\em лишь}
внешнее. Или, наоборот, нечто, которое есть лишь некоторое
{\em внешнее}, именно поэтому есть
{\em лишь} внутреннее. Или, иначе сказать, так как
внутреннее определено как {\em сущность}, а внешнее как
{\em бытие}, то та или иная мыслимая вещь, поскольку
она дана (ist) лишь в своей {\em сущности}, именно
поэтому есть лишь некоторое непосредственное
{\em бытие;} или та мыслимая вещь, которая только
{\em есть}, именно поэтому дана (ist) пока что еще лишь
в своей {\em сущности}. "--- Внешнее и внутреннее суть
определенность, положенная так, что каждое из этих двух определений не
только предполагает другое и переходит в него, как в свою истину, но что
оно, поскольку оно есть эта истина другого, остается
{\em положенным как определенность}, и указует на
тотальность обоих. "--- {\em Внутреннее} есть тем самым
завершение {\em сущности} по форме. А~именно, сущность,
будучи определена как внутреннее, содержит в себе указание на то, что она
недостаточна и имеет бытие лишь как соотношение со своим другим, с внешним;
но и последнее точно так же есть не только бытие или также и существование,
а нечто соотносящееся с сущностью или с внутренним. Но здесь имеется налицо
не просто соотношение обоих друг с другом, а определенное соотношение
абсолютной формы, заключающееся в том, что каждое из них есть
непосредственно своя противоположность, и имеется их общее соотношение
{\em с их третьим}, или, вернее,
{\em с их единством}. Однако их опосредствование еще не
имеет этой содержащей их обоих тождественной основы; их соотношение есть
поэтому непосредственное превращение одного в другое, и это отрицательное
единство, связующее их вместе, есть простая, бессодержательная точка.

\hegremark[Примечание]%
  {Непосредственное тождество внутреннего и внешнего}%
  {[Непосредственное тождество внутреннего и внешнего]}

Движение сущности есть вообще {\em становление
понятием}. В~отношении внутреннего и внешнего выступает тот существенный
момент понятия, что его определения положены быть в отрицательном единстве
таким образом, что каждое есть непосредственно не только свое другое, но
также и тотальность целого. Но в понятии как таковом эта тотальность есть
{\em всеобщее} "--- основа, которой еще нет в отношении
внутреннего и внешнего. "--- В отрицательном тождестве внутреннего и внешнего,
представляющем собою {\em непосредственное превращение}
одного из этих определений в другое, недостает также и той основы, которую
мы выше назвали {\em мыслимой вещью} (Sache).~"---

Очень важно обратить внимание на неопосредствованное
{\em тождество формы} как оно здесь положено еще без
содержательного движения самой мыслимой вещи. Оно встречается в мыслимой
вещи, как последняя есть в своем {\em начале}. Так,
например, {\em чистое бытие} есть непосредственно
{\em ничто}. Вообще, все реальное есть в своем начале
такое лишь непосредственное тождество; ибо в своем начале оно еще не
противопоставило друг другу и не развило моментов; оно, с одной стороны,
еще не вышло из внешности, еще не получило характера
{\em внутреннего}, а, с другой стороны, еще не
выбралось через свою деятельность из внутренности, еще не вывело себя
{\em во-вне} и не продуцировало себя; оно есть поэтому
лишь внутреннее, как {\em определенность} относительно
внешнего, и лишь внешнее, как {\em определенность}
относительно внутреннего. Тем самым оно есть отчасти
{\em лишь} некоторое непосредственное бытие; отчасти
же, поскольку оно есть также и отрицательность, долженствующая стать
деятельностью развития, оно как таковая есть пока что по существу
{\em лишь} внутреннее. "--- Это встречается нам во всяком
вообще природном, научном и духовном развитии, и очень важно убедиться в
том, что всякое первое (das Erste), когда нечто остается только чем-то
{\em внутренним}, или, что то же самое, имеет бытие
только в своем {\em понятии}, именно потому есть лишь
свое непосредственное, пассивное наличное бытие. Так, чтобы обратиться к
ближайшему примеру, рассмотренное здесь
{\em существенное отношение} до того, как оно
продвинулось и реализовалось через опосредствование, через отношение
{\em силы}, есть лишь отношение в себе, его понятие,
или пока что лишь {\em внутренне}. Но именно поэтому
оно есть {\em лишь внешнее}, непосредственное отношение
"--- отношение {\em целого} и
{\em частей}, в котором стороны имеют безразличное друг
к другу устойчивое наличие. В~них самих еще нет их тождества; оно пока что
лишь {\em внутренне}, и поэтому они распадаются,
обладают некоторым непосредственным, внешним устойчивым наличием. "---
Подобным же образом {\em сфера бытия} есть еще вообще
пока что лишь нечто безоговорочно {\em внутреннее}, и
потому она есть сфера сущей непосредственности или внешности. "---
{\em Сущность} есть пока что лишь
{\em внутреннее;} поэтому ее и принимают за совершенно
{\em внешнюю}, чуждую системе общность. Говорят,
например, das Schulwesen, Zeitungs\-wesen (буквально, сущность школы,
сущность газеты; в переводе на русский язык просто "--- школа, пресса, в
собирательном смысле. "--- Перев.) и понимают под этим нечто общее, образуемое
посредством внешнего сочетания существующих предметов, поскольку они не
имеют никакой существенной связи, никакой организации. "--- Или, если
обратиться к конкретным предметам, зародыш растения или ребенок есть пока
что лишь {\em внутреннее} растение,
{\em внутренний} человек. Но именно поэтому растение
как зародыш или человек как зародыш есть нечто непосредственное, нечто
внешнее, еще не сообщившее себе отрицательного соотношения с самим собой,
нечто {\em пассивное, предоставленное} инобытию. "--- Так
и бог в его {\em непосредственном} понятии не есть дух;
дух есть не такое непосредственное, которое противоположно
опосредствованию, а, наоборот, такая сущность, которая вечно полагает свою
непосредственность и вечно возвращается из нее в себя. Поэтому
{\em непосредственно} бог есть
{\em лишь} природа. Или, иначе сказать, природа есть
{\em лишь внутренний}, а не действительный как дух и
тем самым не истинный бог. "--- Или скажем так: бог есть в мышлении, как
{\em первом} мышлении, лишь чистое бытие или же
сущность, абстрактное абсолютное, а не бог как абсолютный дух, каковой
единственно только и есть истинная природа бога.

3. {\em Первое} из рассмотренных нами тождеств
внутреннего и внешнего есть основа, безразличная к различию этих
определений, как к внешней ей форме, или, иначе сказать, тождество как
{\em содержание}. {\em Второе} есть
неопосредствованное тождество их различия, непосредственное превращение
каждого из них в свое противоположное, или, иначе сказать, тождество как
чистая {\em форма}. Но эти два тождества суть лишь
{\em стороны одной и той же тотальности;} или, иначе
сказать, сама эта тотальность есть лишь превращение одного тождества в
другое. Тотальность как основа и содержание есть эта рефлектированная в
себя непосредственность лишь через пред-полагающую рефлексию формы,
снимающей свое различие и полагающей себя по отношению к нему как
безразличное тождество, как рефлектированное единство. Или, иначе сказать,
содержание есть сама форма, поскольку она определяет себя как разность и
делает себя самое одной из своих сторон, как внешность, а другой стороной,
т.~е. внутренним "--- как рефлектированная в себя непосредственность.

Благодаря этому различия формы, внутреннее и внешнее, положены, наоборот,
каждое в себе самом как тотальность себя и своего другого;
{\em внутреннее}, как простое рефлектированное в себя
тождество, есть непосредственное и потому столь же бытие и внешность, сколь
и сущность; а {\em внешнее}, как многообразное,
определенное бытие, есть лишь внешнее, т.~е. положено как несущественное и
возвратившееся в свое основание, стало быть, как внутреннее. Этот переход
обоих друг в друга есть их непосредственное тождество как основа; но он
есть также и их опосредствованное тождество, а именно, каждое есть как раз
через свое другое то, что оно есть в себе, т.~е. тотальность отношения.
Или, обратно, определенность каждой из сторон в силу, того, что она в себе
же есть тотальность, опосредствовала с другой определенностью; таким
образом, тотальность опосредствует себя с самой собою через форму или через
определенность, а определенность опосредствует себя через свое простое
тождество с собою.

Поэтому то, что нечто есть, оно есть целиком в своей внешности; его
внешность есть его тотальность, она есть также и его рефлектированное в
себя единство. Его явление есть рефлексия не только в другое, но и в себя,
и его внешность есть поэтому проявление во вне того, что оно есть в себе; а
так как его содержание и его форма, таким образом, безоговорочно
тождественны, то нечто состоит в себе и для себя не в чем ином, как в том,
что оно {\em проявляется во вне}. Оно есть откровение
своей сущности, так что эта сущность именно и состоит только в том, что она
есть открывающее себя.

В этом тождестве явления с внутренним или сущностью существенное отношение
определило себя как {\em действительность}.

