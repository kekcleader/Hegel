Главный редактор {\em М.~В.~Попов} \\
Корректоры {\em Н.~А.~Гуринович, А.~В.~Кузьмин, В.~А.~Ломов} \\
Художественное оформление {\em Н.~А.~Гуринович, В.~А.~Ломов}

\bigskip
\clearpage

\chapter[Предисловие]{Предисловие}

Эта часть логики, содержащая в~себе {\em учение о понятии} и составляющая
третью часть целого, выпускается под особым названием {\em <<Система
субъективной логики>>;} мы это делаем для удобства тех друзей этой науки,
которые привыкли более интересоваться рассматриваемыми здесь материями,
входящими в состав обычной так называемой логики, чем теми дальнейшими
логическими предметами, о которых мы трактовали в первых двух частях. "---
По отношению к предыдущим частям я мог притязать на снисхождение
справедливых судей ввиду малого числа подготовительных работ, которые могли
бы доставить мне опору, материалы и путеводную нить для
движения вперед. Выпуская настоящую часть, я смею притязать на снисхождение
скорее по противоположному основанию, так как для логики {\em понятия} имеется
налицо вполне готовый и отвердевший, можно сказать, окостеневший материал,
и задача состоит в том, чтобы привести его в текучее состояние и снова
возжечь живое понятие в таком мертвом материале. Если имеются свои
трудности в предприятии построить в пустынной местности новый город, то в
тех случаях, когда дело идет о новой распланировке старого, прочно
построенного города, сохранившегося благодаря тому, что он никогда не
оставался бесхозяйным и необитаемым, нет, правда, недостатка в материале,
но зато встречаются тем большие препятствия другого рода; при этом
приходится, между прочим, решиться не делать никакого употребления из
значительной части того готового материала, который, вообще говоря,
признается ценным.~"---

Но в первую очередь я должен сослаться в качестве извинения
несовершенства выполнения на величие самого предмета. Ибо какой предмет
возвышеннее для познания, чем сама
{\em истина}\pagenote{<<Понятие>> рассматривается Гегелем
как та <<конкретная тотальность>> определений мысли или то <<целое>>, которое и
составляет истину, согласно известному положению Гегеля о том, что <<das
Wahre ist das Ganze>>, причем это <<целое>> мыслится у Гегеля
как некоторый процесс самостановления, как некоторое развитие или
развертывание (см. Предисловие к <<Феноменологии духа>>, стр.~14 в немецком
издании Лассона; стр.~8 в русском переводе под ред. Радлова). Поэтому
именно здесь, в предисловии к <<Учению о понятии>>, Гегель и определяет
предмет логики как самоё истину (die Wahrheit selbst). Ср.~замечание
Энгельса о том, что Гегель в ряде своих произведений и в
особенности в <<Логике>> подчеркивает, что истина {\em есть процесс}
{\em Engels}, L.~Feuerbach, Moskau 1932, S.~19).
Немецкое слово <<Begriff>> очень удобно для того, чтобы толковать его так,
как это делает Гегель, т.~е. в смысле тотальности определений или
совокупности всех моментов рассматриваемого предмета: <<Begriff>> происходит
от глагола <<begreifen>>, который означает <<обнимать собою>>, <<охватывать
воедино>>, а затем <<понимать, постигать умом>>. От глагола <<begreifen>>
происходит, далее, существительное <<Inbegriff>>, означающее
<<совокупность, суть, содержание, сущность>>. У~Гегеля слово <<Begriff>>
нередко употребляется в смысле, близком к значению слова <<Inbegriff>>.
Вообще <<Begriff>> имеет у Гегеля по преимуществу {\em объективное}
значение (объективное в смысле объективного и абсолютного
идеализма), в отличие от того обычного субъективного смысла этого слова,
который имеет в виду <<понятие>> как идею, образуемую человеческим умом, как
умственное отображение предмета в человеческом сознании. Например, на
стр.~\pageref{bkm:bm01a} Гегель пишет: <<понятие есть ушедшая
внутрь себя всеобщая сущность какой-нибудь вещи>>. Ср. замечание Энгельса:
<<в суждении понятия о субъекте высказывается, в какой мере он соответствует
своей всеобщей природе или, как говорит Гегель, своему понятию>> (Engels,
Anti-Duhring und Dialektik der Natur, M.--L. 1935,
S.~662)\label{bkm:bm01}}.
"--- Однако сомнение насчет того, не нуждается ли в извинении
именно этот предмет, не вполне неуместно, если вспомним тот смысл, в
котором Пилат задал вопрос: что есть истина? "--- как говорит
поэт\pagenote{Клопшток в поэме <<Мессиада>>, песнь 7-я.\label{bkm:bm02}},
<<с миной светского человека, близоруко, но с улыбкой
осуждающего серьезное отношение к делу>>. В~таком случае этот вопрос
заключает в~себе тот смысл (на который можно смотреть, как на момент
светскости) и напоминание о том, что цель познания истины есть нечто такое,
от чего, мол, как известно, давно отказались, с чем давно покончили, и
что недостижимость истины есть, дескать, нечто
общепризнанное также и среди профессиональных философов и логиков. "---
Но если в наше время вопрос {\em религии} о ценности
вещей, воззрений и действий, который по своему содержанию имеет тот же
смысл, все более и более отвоевывает обратно свое право на существование,
то и философия должна, конечно, надеяться, что уже больше не будут находить
очень странным, если она снова, прежде всего в своей непосредственной
области, будет настаивать на своей истинной цели и, после того как она
опустилась до уровня других наук, уподобившись им по своим приемам и
отсутствию притязания на истину, будет снова стремиться подняться к этой
цели. Извиняться за эту попытку, собственно говоря, неизвинительно; но для
извинения за несовершенство ее осуществления я разрешу себе еще заметить,
что мои служебные дела и другие личные обстоятельства дозволяли мне лишь
спорадические занятия в области такой науки, которая требует и достойна
непрерываемых и нераздельных усилий.
{\em Нюрнберг}, 21~июля 1816~г.

\clearpage

\chapter[О~понятии вообще]{О~понятии вообще}

Указать непосредственно, в чем состоит {\em природа понятия},
так же мало возможно, как невозможно установить
непосредственно понятие какого бы то ни было другого предмета. Может,
пожалуй, казаться, что для указания понятия какого-либо предмета уже
предполагается в качестве предпосылки логическое и что поэтому последнее
уже не может само в свою очередь ни иметь своей предпосылкой что-нибудь
другое, ни быть чем-то выводным, подобно тому, как в геометрии логические
предложения в том виде, как они выступают в применении к величине и
употребляются в этой науке, предпосылаются ей в форме
{\em аксиом, не выведенных и не выводимых} определений познания.
Но хотя мы должны смотреть на понятие не только как на {\em субъективную
предпосылку}, а как на {\em абсолютную основу}, оно все же может быть
таковой лишь постольку, поскольку оно сделало себя основой.
Абстрактно-непосред-ственное есть, правда, нечто {\em первое;} но, как
абстрактное, оно есть скорее нечто опосредствованное, основу которого,
таким образом, мы, если желаем постигнуть его в его истине, еще должны
отыскать. Эта основа, хотя она и должна поэтому быть чем-то
непосредственным, должна все же быть этим непосредственным лишь таким
образом, что она сделала себя непосредственной через снятие
опосредствования.

Взятое с этой стороны, {\em понятие} должно быть рассматриваемо прежде всего
вообще как {\em третье} к {\em бытию} и {\em сущности}, к
{\em непосредственному} и {\em рефлексии}. Бытие и сущность суть постольку
моменты его {\em становления;} понятие же есть их {\em основа} и
{\em истина}, как то тождество, в котором они затонули и содержатся. Они в
нем содержатся, так как оно есть их {\em результат}, но содержатся уже не
как {\em бытие} и {\em сущность;} это определение они имеют лишь постольку,
поскольку они еще не возвратились в это их единство.

{\em Объективная логика}, рассматривающая {\em бытие} и {\em сущность},
составляет поэтому, собственно говоря, {\em генетическую экспозицию понятия}.
Говоря точнее, уже {\em субстанция} представляет собой {\em реальную сущность}
или {\em сущность}, поскольку она соединилась с {\em бытием} и вступила в
действительность. Поэтому понятие имеет своей непосредственной предпосылкой
субстанцию, она есть {\em в~себе} то, что понятие есть как {\em проявившееся}.
{\em Диалектическое движение субстанции} через причинность и взаимодействие
есть поэтому непосредственный {\em генезис понятия} и изображает ход
{\em становления} последнего. Но его {\em становление}, как и
повсюду становление, имеет то значение, что оно есть рефлексия переходящего
в свое {\em основание} и что кажущееся сперва {\em другим}, в которое перешло
{\em первое}, составляет {\em истину} этого первого. Таким образом, понятие есть
{\em истина} субстанции, а так как {\em необходимость} есть определенный
способ отношения субстанции, то {\em свобода} оказывается {\em истиной
необходимости} и {\em способом отношения понятия}.

Специально свойственным субстанции, необходимым дальнейшим ее определением
служит {\em полагание} того, что есть {\em в~себе и для себя;} и вот
{\em понятие} есть абсолютное единство {\em бытия} и {\em рефлексии},
состоящее в том, что {\em в-себе-и-для-себя-бытие} есть лишь благодаря тому,
что оно есть равным образом {\em рефлексия} или {\em положенность}, и что
{\em положенность} есть {\em в-себе-и-для-себя-бытие}. "--- Этот абстрактный
результат находит себе разъяснение в изображении его конкретного генезиса;
оно содержит в~себе природу понятия; но оно должно было предшествовать
рассмотрению последнего. Главные моменты этой экспозиции (рассмотренной
подробно во второй книге <<Объективной логики>>) должны быть поэтому здесь
вкратце резюмированы.

Субстанция есть {\em абсолютное}, есть сущее в~себе и для себя
действительное: {\em в~себе} "--- как простое тождество возможности и
действительности, абсолютная сущность, содержащая {\em внутри себя}
всяческую действительность и возможность; {\em для себя} "--- это
тождество как абсолютная {\em мощь} или безоговорочно соотносящаяся с собой
{\em отрицательность}. "--- Движение субстанциальности, положенное через эти
моменты, состоит в том, что:

1. Субстанция, как абсолютная мощь или соотносящаяся с собою {\em отрицательность},
диференцирует себя так, что становится некоторым отношением,
в котором они (моменты) суть сперва лишь простые моменты как
{\em субстанции} и как первоначальные {\em предпосылки}. "---
Определенное отношение между ними есть отношение между некоторой {\em пассивной}
и некоторой {\em активной} субстанцией "--- между первоначальностью простого
{\em в-себе-бытия}, которое, лишенное мощи, есть не полагающее само себя,
а лишь первоначальная {\em положенность}, и {\em соотносящеюся с собой}
отрицательностью, которая как таковая положила себя как иное и соотносится с
{\em этим} иным. Это иное именно и есть пассивная субстанция, которую она в
первоначальности своей мощи {\em пред-положила} себе как условие. "--- Это
пред-полагание следует понимать так, что движение самой субстанции
совершается ближайшим образом под формой одного из моментов ее понятия, под
формой {\em в-себе-бытия}, что определенность одной из находящихся между
собой в отношении {\em субстанций} есть вместе с тем и определенность самого
этого {\em отношения},

2. Вторым моментом служит {\em для-себя-бытие}, или, иначе говоря, он
состоит в том, что мощь полагает {\em себя} как соотносящую {\em себя с
самой собой} отрицательность и тем самым вновь снимает
{\em пред-положен-ное}. "--- Активная субстанция есть {\em причина;} она
{\em действует;} это означает, что она есть теперь {\em полагание}, подобно
тому, как перед тем она была {\em пред-полаганием}, означает, что: а) мощи
сообщается также и {\em видимость} мощи, положенности также и {\em видимость}
положенности. То, что в пред-полагании было {\em первоначальным}, становится
в причинности {\em через соотношение с иным} тем, что оно есть в~себе;
причина производит действие и притом в некоторой другой субстанции; она есть
теперь мощь {\em по отношению к некоторому иному;} постольку она {\em являет}
себя как причина, но {\em есть} таковая лишь через этот процесс {\em явления}.
"--- b)~В пассивную субстанцию привходит действие, вследствие чего она теперь
также и являет себя как {\em положенность}, но есть пассивная субстанция
впервые лишь в этом процессе.

3. Но здесь имеется еще нечто большее, чем только это {\em явление}, а именно:
а) причина действует на пассивную субстанцию, {\em изменяет} (делает
иным) ее определение; но последнее есть положенность, в ней нет ничего
иного, что можно было бы изменять; а то другое определение, которое она
получает, есть определение причинности; пассивная субстанция становится,
следовательно, причиной, мощью и деятельностью. b) Действие {\em полагается}
в ней причиной; но положенное причиной есть сама тождественная с собой в
процессе действия причина; эта-то причина и ставит себя на место пассивной
субстанции. "--- Равным образом, что касается активной субстанции, то:
а) воздействие есть перевод причины в действие, в её {\em иное,} в
положенность, и b) в действии причина обнаруживает себя тем, чт\'{о} она есть;
действие тождественно с причиной, а не есть нечто иное; причина,
следовательно, обнаруживает в процессе действия положенность как то, что
она есть по существу. "--- Таким образом, каждое из этих
определений\pagenote{Т.~е. причина и действие, активная субстанция
и пассивная субстанция.\label{bkm:bm03}} становится {\em противоположностью}
самого себя, и это имеет место с обеих сторон, т.~е. как со стороны
тождественного, так и со стороны отрицательного {\em соотнесения с ним его
иного;} но каждое становится этой противоположностью так, что иное,
следовательно и каждое, остается {\em тождественным с самим собою}. "--- Но и
то и другое, и тождественное и отрицательное соотнесение, суть одно и то же;
субстанция тождественна с самой собой лишь в своей противоположности, и это
составляет абсолютное тождество субстанций, положенных как две. Активная
субстанция проявляется как причина или первоначальная субстанциальность
через процесс действия, т.~е. когда она полагает себя как противоположность
самой себя, что вместе с тем есть снятие ее{\em пред-положенного инобытия},
пассивной субстанции. Обратно, через процесс воздействия положенность
проявляется {\em как} положенность, отрицательное "--- {\em как} отрицательное,
и, стало быть, пассивная субстанция "--- как {\em соотносящаяся с собой}
отрицательность; и причина в этом ином самой себя всецело сливается лишь с
собою. Следовательно, через это полагание {\em пред-положенная} или
{\em сущая в~себе} первоначальность становится {\em для себя;} но это
в-себе-и-для-себя-бытие имеется лишь благодаря тому, что это полагание есть
равным образом и {\em снятие} пред-положенного, или, иначе говоря, благодаря
тому, что абсолютная субстанция возвратилась к себе самой лишь {\em из своей
положенности} и {\em в своей положенности} и потому абсолютна. Это
взаимодействие есть тем самым снова снимающее себя явление, заявление той
{\em видимости} причинности, в которой причина имеет бытие {\em как} причина,
{\em что она есть видимость}. Эта бесконечная рефлексия в самого себя,
состоящая в том, что в-себе-и-для-себя-бытие есть лишь благодаря тому, что
оно есть положенность, есть {\em завершение субстанции}. Но это завершение
не есть уже сама {\em субстанция}, оно есть нечто более высокое, {\em понятие},
{\em субъект}. Переход отношения субстанциальности совершается по его
собственной имманентной необходимости и есть не~что иное, как проявление
самой этой необходимости, проявление того, что понятие есть истина
субстанциальности и что свобода есть истина необходимости.

Уже ранее, во второй книге объективной логики (стр.~432 и~сл., примечание),
было упомянуто, что философия, которая становится и продолжает стоять на
точке зрения {\em субстанции}, есть {\em система Спинозы}. Там же мы вместе
с тем вскрыли {\em неудовлетворительность} этой системы как по форме, так и
по содержанию. Но иное дело {\em опровержение} этой системы. Относительно
опровержения какой-либо философской системы мы равным образом сделали в
другом месте то общее замечание, что при этом следует отвергнуть превратное
представление, будто система должна быть изображена как совершенно ложная,
а {\em истинная}mсистема, напротив, как {\em лишь противоположная} ложной.
Из той связи, в которой здесь выступает система Спинозы, само собой
выясняется настоящая точка зрения на нее и на вопрос о том, истинна ли она
или ложна. Отношение субстанциальности породило себя через природу
{\em сущности;} это отношение, равно как и его расширение до целой системы,
есть поэтому {\em необходимая точка зрения}, на которую становится
абсолютное. На такую точку зрения не следует поэтому смотреть, как на
некоторое мнение, как на субъективный, произвольный способ представления и
мышления некоторого индивидуума, как на заблуждение спекуляции; напротив,
последняя на своем пути необходимо перемещается на эту точку зрения, и
постольку эта система совершенно истинна. "--- Но {\em эта точка зрения не есть
наивысшая}. Тем не менее система постольку не может рассматриваться как
{\em ложная}, требующая {\em опровержения} и могущая быть опровергнутой, а в
ней следует рассматривать как {\em ложное} лишь признание ее точки зрения за
наивысшую. {\em Истинная система} не может поэтому и находиться к ней в
таком отношении, что она лишь {\em противоположна} последней; ибо в таком
случае эта противоположная система сама была бы односторонней. Напротив, как
более высокая, она должна содержать в~себе низшую.

Далее, опровержение не должно приходить извне, т.~е. не должно
исходить из допущений, лежащих вне опровергаемой системы, допущений,
которым последняя не соответствует. В~этом случае опровергаемой системе
нужно только не признавать этих допущений; {\em недостаток} есть
недостаток лишь для того, кто исходит из основанных на них потребностей и
требований. В~этом смысле была высказана мысль, что для того, кто не решил
для себя положительно вопроса о свободе и самостоятельности
самосознательного субъекта и не исходит из этой предпосылки, не может иметь
места никакое опровержение спинозизма\pagenote{Гегель имеет в виду высказывания
Фихте, который в <<Первом введении в наукоучение>> (1797) писал: <<Всякий
последовательный догматик "--- неизбежно фаталист; он не
отрицает того факта сознания, что мы считаем себя свободными; ибо это было
бы противно разуму; но он доказывает из своего принципа ложность этого
мнения. Он совершенно отрицает самостоятельность Я, на которой строит свое
учение идеалист, и делает его простым продуктом вещей, случайной
принадлежностью мира; последовательный догматик "--- неизбежно
материалист. Он {\em мог бы быть опровергнут только из постулата свободы и
самостоятельности Я: но это "--- то самое, что он отрицает}>>
(J.~G.~Fichte, Werke, hrsg.~v. Medicus, Bd.~III, S.~14--15; в~русском
издании Избранных сочинений Фихте, т.~I, M.~1916, это место находится на
стр.~421). Что под <<последовательным догматизмом>> Фихте имел в виду
философию Спинозы, видно из многих других мест его сочинений, например, из
следующего места в <<Основах общего наукоучения>> (1794): <<Поскольку
догматизм может быть последовательным, спинозизм является наиболее
последовательным его продуктом>> [{\em Fichte}, Werke, Bd.~I, S.~314; в
русском издании "--- стр.~96). До Фихте о строгой внутренней последовательности
спинозизма и о невозможности его опровергнуть без обращения к другим
принципам говорил Ф.-Г.~Якоби.\label{bkm:bm04}}. Да и, кроме того, такая
высокая и в самой себе уже такая {\em богатая} точка зрения, как отношение
субстанциальности, не игнорирует эти допущения, а тоже содержит их в~себе:
одним из атрибутов спинозовской субстанции служит {\em мышление}. Эта точка
зрения, наоборот, умеет растворить определения, под которыми эти допущения
противоречат ей, и вовлечь их в~себя, так что они выступают {\em в этой же системе},
но в соответствующих ей модификациях. Нерв внешнего
опровержения покоится в таком случае лишь на том, чтобы со своей стороны
упорно и твердо настаивать на противоположных формах указанных допущений,
например, на абсолютном самодовлении мыслящего индивидуума
"--- в противоположность той форме, в какой мышление полагается
тождественным с протяжением в абсолютной субстанции. Истинное опровержение
должно вникнуть в сильную сторону противника и стать в
диапазоне этой силы; нападать же на него вне его сферы и одерживать над ним
верх там, где его нет, не помогает делу. Единственное опровержение
спинозизма может поэтому состоять лишь в том, что его точка зрения
признается, во-первых, существенной и необходимой, но что, во-вторых, эта
точка зрения, исходя {\em из нее же самой}, поднимается на уровень более
высокой точки зрения. Отношение субстанциальности, рассматриваемое всецело
лишь {\em в~себе самом и само по себе}, переводит себя в свою
противоположность, в {\em понятие}. Поэтому содержащаяся в предыдущей книге
экспозиция субстанции, приводящая к {\em понятию}, есть единственное и
истинное опровержение спинозизма. Она есть {\em раскрытие} субстанции, а это
раскрытие есть {\em генезис понятия}, главные моменты которого резюмированы
выше. "--- {\em Единство} субстанции есть ее отношение {\em необходимости;}
но последняя есть, таким образом, лишь {\em внутренняя необходимость;
полагая себя} через момент {\em абсолютной отрицательности}, она
становится {\em проявившимся} или {\em положенным} тождеством и тем самым
{\em свободой}, которая есть тождество понятия. Понятие, как получающаяся из
взаимодействия тотальность, есть единство {\em обеих} взаимодействующих
{\em субстанций}, но такое единство, при котором они отныне принадлежат
свободе, поскольку они теперь уже обладают тождеством не как чем-то слепым,
т.~е. {\em внутренним}, а имеют по существу определение быть {\em видимостью}
или моментами рефлексии; вследствие чего каждая столь же непосредственно
слилась со своим иным или со своей положенностью и каждая содержит свою
положенность {\em внутри себя самой} и, стало быть, положена в своём ином
безоговорочно лишь как тождественная с собой.

В {\em понятии} открылось поэтому царство {\em свободы}. Понятие есть
свободное (das Freie), потому что то {\em в-себе-и-для-себя-сущее тождество},
которое составляет необходимость субстанции, выступает вместе с тем как
снятое или как {\em положенность}, а эта положенность, как соотносящаяся с
самой собой, как раз и есть то тождество. Темнота друг для друга находящихся
в причинном отношении субстанций исчезла, ибо первоначальность их
самостоятельного существования перешла в положенность и благодаря этому
стала прозрачной для себя самой {\em ясностью; первоначальная}
вещь\pagenote{В~немецком тексте: <<die ur\-sprung\-liche Sache>>. Гегель
намекает на этимологию немецкого слова <<Ur-sache>> (причина; буквально:
перво-вещь).\label{bkm:bm05}}
первоначальна лишь постольку, поскольку она есть {\em причина самой себя},
а это и есть {\em субстанция, освобожденная, чтобы стать понятием}.

Из этого для понятия сразу же вытекает следующее, более детальное
определение. Так как в-себе-и-для-себя-бытие непосредственно выступает как
{\em положенность}, то понятие в своем простом соотношении с самим собой есть
абсолютная {\em определенность}, но такая определенность, которая, как
соотносящаяся только с собой, есть вместе с тем непосредственно простое
тождество. Но это {\em соотношение} определенности {\em с самой собой}, как
ее {\em слияние} с собой, есть равным образом и {\em отрицание определенности},
и понятие, как сказанное равенство с самим собой, есть {\em всеобщее}.
Но это тождество имеет в равной мере и определение отрицательности; оно есть
отрицание или определенность, которая соотносится с собой; взятое таким
образом, понятие есть {\em единичное}. Каждое из них есть тотальность, каждое
содержит в~себе определение другого, и поэтому указанные тотальности суть в
такой же мере безоговорочно лишь {\em одна} тотальность, в какой это
единство есть расщепление себя самого, превращение в свободную видимость
указанной двойности, "--- двойности, выступающей в различии {\em единичного}
и {\em всеобщего} как совершенная противоположность, которая, однако, настолько
есть {\em видимость}, что когда постигается и высказывается одно, то вместе
с этим непосредственно постигается и высказывается другое.

Только что изложенное должно быть рассматриваемо как {\em понятие понятия}.
Может показаться, что это понятие отступает от того, что обычно понимают под
понятием, и нам можно было бы предъявить требование, чтобы мы указали, каким
образом то, что здесь получилось как понятие, содержится в других
представлениях или объяснениях. Однако, с одной стороны, здесь не может идти
речь о подтверждении, основанном на {\em авторитете} обычного понимания; в
науке понятия его содержание и определение может быть удостоверено
исключительно только посредством {\em имманентной дедукции}, содержащей его
генезис, и эта дедукция уже лежит позади нас. С~другой же стороны, должно
быть действительно возможно распознать в том, что обычно предлагается как
понятие понятия, дедуцированное здесь понятие. Но не так-то легко найти то,
что другие говорили о природе понятия. Ибо по большей части они вовсе не
занимаются отыскиванием этой природы и предполагают, что когда говорят о
понятии, то каждый уже само собой понимает, о чем идет речь. В~новейшее
время можно было тем более считать себя освобожденными от хлопот с понятием,
что, как некоторое время признавалось хорошим тоном всячески поносить силу
воображения, а затем и память, так теперь в философии уже довольно давно
сделалось привычкой, сохранившейся отчасти еще и поныне, говорить
всевозможные дурные вещи о {\em понятии}, делать его, эту вершину мышления,
предметом презрения и, напротив, считатьm наивысшей вершиной научности и
моральности {\em непостижимое} и {\em отсутствие постижения}\pagenote{Гегель,
по-видимому, имеет в виду иррационализм Фридриха-Генриха Якоби (1743---1819).
Ср. замечания Гегеля об этом философе в <<Малой логике>> ({\em Гегель},
Соч., т.~I, стр.~114---118) и
в <<Истории философии>> (т.~XI, стр.~405---415).\label{bkm:bm06}}.

Я ограничиваюсь здесь одним замечанием, которое может послужить к пониманию
развитых здесь понятий и облегчить читателю ориентироваться в них. Понятие,
поскольку оно достигло такого {\em существования}, которое само свободно,
есть не~что иное, как <<я>> или чистое самосознание. Я, правда, обладаю
понятиями, т.~е. определенными понятиями, но <<я>> есть само чистое понятие,
которое как понятие достигло {\em наличного бытия}. Поэтому, если я напоминаю
об основных определениях, составляющих природу <<я>>, то я имею право
предполагать, что напоминаю о чем-то знакомом, т.~е. привычном для
представления. Но <<я>> есть, {\em во-первых}, это чистое, соотносящееся с собой
единство, и оно таково не непосредственно, а только тогда, когда оно
абстрагирует от всякой определенности и всякого содержания и возвращается к
свободе безграничного равенства с самим собой. Взятое таким образом, оно
есть {\em всеобщность}, "--- единство, которое лишь через то {\em отрицательное}
отношение, которое выступает как абстрагирование, есть единство с собой и
вследствие этого содержит в~себе растворенной всякую определенность.
{\em Во-вторых}, <<я>>, как соотносящаяся: с самой собой отрицательность, есть
столь же непосредственно {\em единичность, абсолютная определенность},
противопоставляющая себя иному и исключающая это иное, {\em индивидуальная
личность}. Указанная абсолютная {\em всеобщность}, которая непосредственно есть
равным образом и абсолютная {\em единичность}, и такое в-себе-и-для-себя-бытие,
которое безоговорочно есть положенность и есть это
{\em в-себе-и-для-себя-бытие} лишь через единство с положенностью, составляют
как природу <<я>>, так и природу {\em понятия;} в том и в другом нельзя ничего
постигнуть, если не мыслить указанных обоих моментов одновременно и в их
абстрактности, и в их совершенном единстве.

Когда, как это обычно принято, говорят о {\em рассудке}, которым
{\em я обладаю}, то под этим понимают некоторую {\em способность} или
{\em свойство}, находящееся в таком отношении к <<я>>, в каком свойство вещи
находится к самой {\em вещи}, "--- к некоторому неопределенному субстрату,
который не есть истинное основание
своего свойства и то, что определяет его. Согласно этому представлению я
{\em обладаю} понятиями и понятием точно так же, как я обладаю сюртуком,
цветом и другими внешними свойствами. "--- {\em Кант} пошел дальше этого
внешнего отношения рассудка (взятого как способность понятий) и самих
понятий к <<я>>. К~глубочайшим и правильнейшим взглядам, находящимся в
<<Критике разума>>, принадлежит взгляд, признающий, что {\em единство},
составляющее {\em сущность понятия}, есть {\em первоначально-синтетическое}
единство {\em апперцепции}, единство <<я мыслю>> (des: Ich denke) или
самосознания. "---
Это положение образует собой так называемую {\em трансцендентальную} дедукцию
категорий; но указанная дедукция издавна считалась одной из труднейших
частей кантовской философии, "--- несомненно не почему-либо иному, как потому,
что она требует от нас, чтобы мы пошли дальше голого {\em представления} об
отношении, в котором {\em <<я>> и рассудок} или {\em понятия} находятся к
какой-нибудь вещи и ее свойствам или акциденциям, "--- чтобы мы перешли к
{\em мысли}. "--- {\em Объект}, говорит Кант (<<Критика чистого разума>>,
стр.~137, 2-е изд.), есть то, {\em в понятии} чего {\em объединено многообразие}
некоторого данного созерцания. А~всякое объединение представлений требует
{\em единства сознания} в их {\em синтезе}. Следовательно, это {\em единство сознания}
есть то, что одно только и составляет отнесение представлений к некоторому
предмету и, стало быть, их {\em объективную значимость}, и то, на чем покоится
сама {\em возможность рассудка}. От этого единства Кант отличает
{\em субъективное единство} сознания, единство представления, "--- вопрос о том,
сознаю ли я некоторое многообразное как {\em одновременное} или как
{\em последовательное}, что, дескать, зависит от эмпирических условий.
Напротив, принципы {\em объективного} определения представлений должны быть,
по Канту, выведены исключительно только из основоположения
{\em трансцендентального единства апперцепции}. Посредством категорий,
представляющих собой эти объективные определения, многообразие данных
представлений определяется так, что оно приводится к {\em единству сознания}.
"--- Согласно этому учению единство понятия есть то, через что нечто есть не
голое {\em определение чувства, созерцание} или голое {\em представление},
а {\em объект}, каковое объективное единство есть единство <<я>> с самим собой. "---
{\em Постижение} какого-либо предмета состоит в самом деле не в чем ином, как
в том, что <<я>> {\em усваивает} себе этот предмет, пронизывает его и приводит его
в {\em свою собственную форму}, т.~е. {\em во всеобщность}, которая есть
непосредственно {\em определенность}, или в определенность, которая есть
непосредственно всеобщность. В~созерцании иди даже в представлении предмет
есть еще нечто {\em внешнее, чуждое}. Через постижение то
{\em в-себе-и-для-себя-бытие}, которым он обладает в процессе созерцания и
представления, превращается в некоторую {\em положенность;} <<я>> пронизывает
его {\em мыслительно}. Но лишь в том виде, в каком он есть в мышлении, он
впервые есть {\em в~себе и для себя;} а в том виде, в каком он выступает в
созерцании или в представлении, он есть {\em явление;} мышление снимает ту
его {\em непосредственность}, с которой он сначала предстает перед нами, и
делает его, таким образом, {\em положенностью;} но эта его {\em положенность}
есть его {\em в-себе-и-для-себя-бытие} или его {\em объективность}. Этой
объективностью предмет, стало быть, обладает в {\em понятии}, и последнее
есть то {\em единство} самосознания, в которое он был вобран; его
объективность или понятие само есть поэтому не~что иное, как природа
самосознания, и не обладает никакими другими моментами или определениями,
кроме тех, какими обладает само <<я>>.

Согласно этому то обстоятельство, что для того чтобы познать, что такое
{\em понятие}, мы напоминаем о природе <<я>>, оправдывается одним из главных
положений кантовской философии. Но и наоборот, для этого необходимо, чтобы
предварительно было постигнуто {\em понятие} <<я>> так, как оно было только что
изложено нами. Если остановиться на голом {\em представлении} о <<я>>, как оно
предносится нашему обычному сознанию, то <<я>> есть лишь простая вещь
(именуемая также и {\em душой}), которой понятие {\em присуще} как некоторое
достояние или свойство. Это представление, не дающее понимания ни <<я>>, ни
понятия, не может служить к тому, чтобы облегчить понимание понятия или
приблизить к нему.

Вышеприведенное учение Канта содержит в~себе еще две стороны,
касающиеся понятия и делающие необходимыми некоторые дальнейшие замечания.
Во-первых, {\em ступени рассудка} у Канта предпосланы {\em ступени чувства} и
{\em созерцания;} и одно из существенных положений кантовой трансцендентальной
философии состоит в том, что {\em понятия без созерцания
пусты} и что они обладают значимостью исключительно только
как {\em соотношения} данного в созерцании {\em многообразия}.
Во-вторых, Кант указывает на понятие как на {\em объективное}
в познании и тем самым как на {\em истину}. Но, с
другой стороны, понятие признается у Канта чем-то {\em только субъективным},
из чего нельзя {\em выколупать реальность},
под которой, ввиду того что она противопоставляется Кантом
субъективности, следует разуметь объективность; и вообще понятие и
логическое объявляются у Канта чем-то только {\em формальным}, которое,
ввиду того что оно отвлекается от содержания, не содержит в~себе истины.

Что касается, во-первых, {\em указанного отношения рассудка или понятия}
к {\em предпосланным ему ступеням}, то все зависит от того, какая наука
занимается определением {\em формы} этих ступеней. В~нашей науке, как
чистой {\em логике}, эти ступени суть {\em бытие} и {\em сущность}.
В~{\em психологии} рассудку предпосылаются {\em чувство} и
{\em созерцание}, а затем вообще {\em представление}. В~{\em феноменологии}
духа, как учении о сознании, мы совершили восхождение к рассудку по
ступеням {\em чувственного сознания}, а затем по ступеням
{\em восприятия}. Кант предпосылает ему лишь чувство и созерцание. В~какой
мере эта лестница, прежде всего, {\em неполна}, это он уже сам дает
понять тем, что присоединяет в виде {\em приложения} к трансцендентальной
логике или учению о рассудке еще {\em рассуждение о рефлективных}
понятиях, "--- область, лежащую между {\em созерцанием} и {\em рассудком}
или между {\em бытием} и {\em понятием}.

Обращаясь к самой сути дела, следует, {\em во-первых},
заметить, что все эти образы {\em созерцания, представления}
и~т.~п. принадлежат {\em самосознательному духу},
который, как таковой, не рассматривается в науке логики.
Чистые определения бытия, сущности и понятия образуют собой, правда, основу
и внутренний простой остов также и форм духа; дух, как
{\em созерцающий}, а равным образом и как {\em чувственное сознание},
имеет определенность непосредственного бытия, а дух, как {\em представляющий},
и также дух, как {\em воспринимающее}
сознание, поднялся от бытия на ступень сущности или
рефлексии. Но эти конкретные образы столь же мало касаются науки логики,
как и те конкретные формы, которые логические определения принимают в
природе, а именно: {\em пространство} и {\em время},
затем наполненное пространство и время, как
{\em неорганическая природа}, и, наконец, {\em органическая природа}.
Равным образом и понятие здесь следует рассматривать не как
акт самосознательного рассудка, не как {\em субъективный рассудок},
а как понятие в~себе и для себя, образующее {\em ступень} как
{\em природы}, так и {\em духа}. Жизнь или
органическая природа есть та ступень природы, на которой выступает понятие,
но как слепое, не постигающее само себя, т.~е. не мыслящее; как мыслящее
оно присуще лишь духу. Но логическая форма понятия независима как от того
недуховного, так и от этого духовного вида понятия; об этом мы уже сделали
необходимое предварительное указание во {\em введении}. Это такое
вразумление, которое не должно быть оправдано впервые в рамках
{\em логики}, а должно быть твердо уяснено еще до нее.

Но каковы бы ни были формы, предшествующие понятию, важно,
{\em во-вторых}, знать, как {\em мыслится отношение} к {\em ним понятия}.
Это отношение как в обычном психологическом представлении,
так и в кантовой трансцендентальной философии понимается так, что
эмпирическая {\em материя},
многообразие созерцания и представления существует сначала
{\em само по себе} и что затем рассудок {\em подходит} к нему, вносит в него
{\em единство} и возводит его посредством {\em абстрагирования} в форму
{\em всеобщности}. Рассудок есть, таким образом, некоторая сама
по себе пустая {\em форма},
которая отчасти получает реальность лишь через указанное
{\em данное} содержание, отчасти же {\em абстрагирует}
от него, а именно, {\em опускает} его как
нечто непригодное, но непригодное лишь для понятия. В~том и другом
действиях понятие не есть нечто независимое, не есть существенное и
истинное содержание этой предшествующей ему материи, представляющей собой,
наоборот, реальность в~себе и для себя, которую, дескать, невозможно
выколупать из понятия.

Правда, нужно согласиться с тем, что {\em понятие как таковое}
еще не полно: оно должно еще возвести себя в {\em идею}, которая
только и есть единство понятия и реальности, как это должно
{\em получиться} в дальнейшем путем рассмотрения природы
{\em самого} понятия. Ибо
реальность, которую оно сообщает себе, должна быть не подобрана, как нечто
внешнее, а выведена согласно требованию науки из него самого. Но, поистине,
не упомянутая выше, данная через созерцание и представление материя должна
быть выдвинута в противоположность понятию как {\em реальное}.
<<{\em Это только понятие}>>, "---
так говорят обыкновенно, противопоставляя понятию, как нечто
более превосходное, не только идею, но и чувственное, пространственное и
временное осязаемое существование. {\em Абстрактное}
считается в таком случае по той причине менее значительным,
чем конкретное, что из него, дескать, опущено так много указанного рода
материи. Абстрагирование получает согласно этому мнению тот смысл, что из
конкретного вынимается (лишь для {\em нашего субъективного
употребления}) {\em тот} или {\em иной признак},
так что с опущением стольких других {\em свойств} и {\em модификаций}
предмета их не лишают ничего из их {\em ценности} и {\em достоинства}, а они
попрежнему оставляются как {\em реальное}, лишь находящееся на другой
стороне, сохраняют попрежнему полное свое значение, и лишь {\em немощь}
рассудка приводит согласно этому взгляду к тому, что ему
невозможно вобрать в~себя все это богатство и приходится довольствоваться
скудной абстракцией. Если данная материя созерцания и многообразие
представления понимаются как реальное в противоположность мыслимому и
понятию, то это есть такой взгляд, предварительный отказ от которого
представляет собой не только условие философствования, но уже
предполагается религией. Каким образом возможна потребность в религии и
чувство религии, если мимолетное и поверхностное явление чувственного и
единичного все еще считается истинным? Философия же дает нам
{\em постигнутое в понятии}
усмотрение того, как обстоит дело с реальностью чувственного
бытия, и предпосылает рассудку вышеуказанные ступени чувства и созерцания,
чувственного сознания и~т.~п. постольку, поскольку они суть
в ходе его становления его условия, причем, однако, они суть его условия
только в том смысле, что {\em из их диалектики} и
{\em ничтожности} понятие возникает как их {\em основание},
а не в том смысле, что оно, мол, обусловлено их {\em реальностью}.
Поэтому абстрагирующее мышление должно рассматриваться не
просто как оставление в стороне чувственной материи, которая при этом не
терпит, дескать, никакого ущерба в своей реальности; оно скорее есть снятие
последней и сведение ее, как простого {\em явления}, к {\em существенному},
проявляющемуся только в {\em понятии}. Конечно,
если та сторона конкретного явления, которую мы, согласно рассматриваемому
воззрению, вбираем в понятие, должна служить лишь
{\em признаком} или {\em знаком}, то она и в
самом деле может быть тоже каким-нибудь лишь чувственным, единичным
определением предмета, которое из-за какого-либо внешнего интереса
избирается среди других, выделяется из них и имеет ту же природу, что и прочие.

Главное недоразумение, имеющее здесь место, заключается в том
мнении, будто {\em естественный} принцип или то {\em начало}, которое
служит исходным пунктом в {\em естественном} развитии или в {\em истории}
развивающегося индивидуума, есть {\em истинное} и {\em первое} также и в
{\em понятии}. Созерцание
или бытие суть, правда, по природе первое или условие для понятия, но это
отнюдь не значит, что они суть безусловное в~себе и для себя. В~понятии
скорее снимается их реальность и, стало быть, вместе с тем снимается и та
видимость, которую они имели как обусловливающее реальное. Если дело идет
не об {\em истине}, а лишь об {\em истории} о
том, как все это происходит в представлении и являющемся мышлении, то
можно, конечно, не идти дальше рассказа о том, что мы начинаем с чувств и
созерцаний и что рассудок из всего их многообразия извлекает некоторую
всеобщность или некоторое абстрактное и, разумеется, нуждается для этого в
вышеупомянутой основе, каковая при этом абстрагировании все еще остается
для представления во всей той реальности, с которой она явила себя вначале.
Но философия не должна быть рассказом о том, что совершается, а должна быть
познанием того, что в этом совершающемся {\em истинно}, и из
истинного она должна постигнуть, далее, то, что в рассказе выступает как
простое событие (Geschehen).

Если при поверхностном представлении о том, что такое понятие,
всякое многообразие стоит {\em вне
понятия}, и последнему присуща лишь форма абстрактной
всеобщности или пустого рефлективного тождества, то можно прежде всего
напомнить уже о том, что и независимо от сказанного всегда определительно
требуется для указания какого-нибудь понятия или для
дефиниции, чтобы к роду, который уже сам, собственно говоря, не есть чисто
абстрактная всеобщность, присоединилась также и
{\em специфическая определенность}.
Если мы только сообразим в некоторой мере мыслительно, чт\'{о}
это означает, то мы убедимся, что тем самым
{\em различение}
рассматривается как столь же существенный момент понятия.
{\em Кант} положил начало
этому рассмотрению той в высшей степени важной мыслью, что существуют
априорные {\em синтетические суждения}.
Этот первоначальный синтез апперцепции представляет собой
один из глубочайших принципов спекулятивного развертывания; он содержит в
себе первый шаг к истинному пониманию природы понятия и совершенно
противоположен вышеупомянутому пустому тождеству или абстрактной
всеобщности, которая не есть внутри себя синтез. "--- Однако
этому первому шагу мало соответствует дальнейшая разработка. Уже выражение
<<синтез>> легко снова приводит к представлению о некотором
{\em внешнем} единстве и {\em голом сочетании} таких элементов, которые
{\em сами по себе раздельны}.
Затем, кантовская философия остановилась только на
психологическом рефлексе понятия и снова возвратилась к утверждению о
непрекращающейся обусловленности понятия некоторым многообразием
созерцания. Эта философия признала рассудочные познания и опыт некоторым
{\em являющимся}
содержанием не потому, что сами категории суть лишь конечные,
а потому, что руководилась психологическим идеализмом, тем соображением,
что они суть {\em лишь}
определения, происходящие из самосознания. К~тому же понятие
согласно учению Канта опять-таки
{\em бессодержательно} и
{\em пусто} без
многообразия созерцания, несмотря на то, что оно есть a priori
некоторый синтез; а ведь поскольку оно есть синтез, оно имеет
определенность и различие внутри себя самого. Поскольку эта определенность
есть определенность понятия и тем самым
{\em абсолютная определенность},
{\em единичность}, понятие
есть основание и источник всякой конечной определенности и всякого
многообразия.

То формальное положение, которое понятие занимает как
рассудок, завершается в кантовом изложении природы
{\em разума}. Можно было
бы ожидать, что в разуме, этой наивысшей ступени мышления, понятие утратит
ту обусловленность, в которой оно еще выступает на ступени рассудка, и
достигнет завершенной истины. Но это ожидание не оправдывается. Так как
Кант определяет отношение разума к категориям как лишь
{\em диалектическое} и
притом безоговорочно понимает результат этой диалектики исключительно
только как {\em бесконечное ничто},
то бесконечное единство разума утрачивает еще
также и синтез, а тем самым и упомянутое выше начало
спекулятивного, истинно бесконечного понятия; оно становится известным,
совершенно формальным, {\em только
регулятивным единством систематического употребления рассудка}.
Кант объявляет злоупотреблением со стороны логики то
обстоятельство, что она, которая должна быть только
{\em каноном логической оценки},
рассматривается как
{\em органон} для
образования {\em объективных}
усмотрений. Понятия разума, в которых мы должны были бы
ожидать более высокую силу и более глубокое содержание, уже не имеют в~себе
ничего {\em конститутивного},
как это еще имело место у категорий; они суть
{\em голые} идеи; их-де,
правда, {\em вполне дозволительно}
употреблять, но эти умопостигаемые сущности, в которых,
согласно докантовским воззрениям, должна была раскрываться вся
{\em истина}, означают не
что иное, как {\em гипотезы},
приписывать которым истину в~себе и для себя было бы полным
произволом и безумным дерзновением, так как они
{\em не могут встретиться ни в каком
опыте}. "--- Можно ли было когда-нибудь подумать, что философия
станет отрицать истину умопостигаемых сущностей потому, что они лишены
пространственной и временн\'{о}й, воспринимаемой чувственностью материи?

С этим непосредственно связана та точка зрения, с которой
следует вообще рассматривать понятие и назначение логики и которая в
философии Канта понимается таким же образом, как это обычно делается: мы
имеем в {\em виду отношение понятия и науки о нем} к самой {\em истине}. Мы уже
привели выше тот пункт кантовской дедукции категорий, который гласит, что
{\em объект}, в котором {\em объединяется}
многообразие созерцания, есть это единство лишь
{\em через единство самосознания}. Здесь, следовательно, определенно высказана
{\em объективность мышления}, то тождество понятия и вещи, которое и есть
{\em истина}. Подобным
образом и обычно все соглашаются с тем, что когда мышление усваивает себе
какой-нибудь данный предмет, то последний вследствие этого претерпевает
некоторое изменение и превращается из чувственного в мыслимый, но что это
изменение не только ничего не изменяет в его существенности, но он,
напротив, {\em истинен}
именно в своем понятии, в непосредственности же, в которой он
дан, он есть лишь {\em явление} и {\em случайность;}
что познание предмета, постигающее его в понятии, есть
познание его таким, каков он {\em в~себе
и для себя}, и что понятие и есть сама его объективность.
Однако, с другой стороны, тут вместе с тем опять-таки
утверждается\pagenote{Имеется в виду трансцендентальная
философия Канта.\label{bkm:bm07}},
что {\em мы все же не можем познавать вещей}, каковы они
{\em в~себе и для себя}, и что {\em истина
недоступна познающему разуму;} что упомянутая выше
истина, состоящая в единстве объекта и понятия, есть все же
лишь явление и притом опять-таки потому, что содержание-де есть лишь
многообразие созерцания. По этому поводу мы уже указали выше, что,
напротив, это многообразие, поскольку оно принадлежит области созерцания в
противоположность понятию, именно и снимается в понятии и что через понятие
предмет приводится обратно к своей неслучайной существенности; последняя
выступает в явлении, и именно поэтому явление есть не просто нечто лишенное
сущности, а проявление сущности. Но ставшее вполне свободным проявление
сущности и есть понятие. "--- Эти положения, о которых мы здесь
напоминаем, не суть догматические утверждения, ибо они представляют собой
результаты, получившиеся сами собой из всего имманентного развития
{\em сущности}. Теперешняя точка зрения, к которой привело это развитие,
состоит в том, что {\em понятие} есть та форма {\em абсолютного},
которая выше бытия и сущности. Так как оказалось, что с этой
стороны оно {\em подчинило себе}
бытие и сущность, к которым при других исходных точках
принадлежит также и чувство, созерцание и представление и которые явились
его предшествующими ему условиями, и что оно есть
{\em их безусловное основание},
то теперь остается еще вторая сторона, изложению которой и
посвящена эта третья книга логики, а именно, остается показать, каким
образом понятие внутри самого себя и из себя образует ту реальность,
которая в нем исчезла\pagenote{Краткая, но чрезвычайно глубокая
критика гегелевского учения о понятии, как о <<конкретной тотальности>>,
<<образующей (или <<порождающей>>) из себя реальность>>, дана Марксом в третьей
главе <<Введения к Критике политической экономии>>. Маркс вышелушивает
рациональное зерно, имеющееся в гегелевском учении о восхождении от
абстрактного к конкретному. Он показывает, что научное мышление
действительно движется от абстрактных определений к <<конкретной
тотальности>> (Маркс даже сам употребляет несколько раз этот гегелевский
термин). Но вместе с тем Маркс с гениальной силой вскрывает фундаментальную
ложь в построениях Гегеля, притом ложь двоякого рода: 1) неверно, будто
<<реальное есть результат мышления, синтезирующего воедино свои определения
внутри самого себя, углубляющегося в~себя и движущегося из самого себя>>: в
действительности <<метод восхождения от абстрактного к конкретному есть лишь
способ, при помощи которого мышление усваивает себе конкретное, {\em воспроизводит}
его духовно как конкретное, а отнюдь не процесс возникновения
самого конкретного>>; 2) неверно, будто конкретная тотальность определений
мысли является продуктом такого <<понятия, которое мыслит {\em вне}
созерцания и представления или над ними и само порождает
себя>>: в действительности эта конкретная тотальность представляет собою
<<продукт {\em переработки представления и созерцания в понятия}>> ({\em Marx},
Zur Kritik der po\-li\-ti\-schen Oekono\-mie, M.--L. 1934, S.~236--237).
Там же Маркс показывает, что соответствие между ходом абстрактного мышления, с
одной стороны, и действительным историческим процессом, идущим от простого
к сложному, с другой стороны, хотя и имеет место в общем и целом, но не
может быть сведено к простому тождеству, так как в действительности дело
обстоит гораздо сложнее.\label{bkm:bm08}}.
Мы поэтому, конечно, согласились с тем, что познание, не
идущее дальше лишь понятия, чисто как такового, еще неполно и дошло пока
что только до {\em абстрактной истины}.
Но его неполнота состоит не в том, что оно лишено той мнимой
реальности, которая, дескать, дана в чувстве и созерцании, а в том, что
понятие еще не сообщило себе своей
{\em собственной}, из
него самого порожденной реальности. В~том-то и состоит выявленная в
противоположность эмпирической материи и на ней, а точнее на ее категориях
и определениях рефлексии, абсолютность понятия, что материя эта
{\em истинна} не в том
виде, в каком она является {\em вне}
и {\em до}
понятия, а исключительно в своей идеальности или в своем
тождестве с понятием. {\em Выведение}
из него реального, если это угодно называть выведением,
состоит по существу ближайшим образом в том, что понятие в своей формальной
абстрактности оказывается незавершенным и через имеющую свое основание в
нем самом диалектику переходит к реальности так, что производит ее из себя,
а не так, что снова возвращается к некоторой готовой, найденной в
противоположность ему реальности и ищет прибежища в чем-то таком, что
показало себя несущественным в явлении, потому что, мол,
понятие, оглядываясь вокруг, искало лучшего, но не нашло
его. "--- Навсегда останется удивительным, что кантовская
философия признала то отношение мышления к чувственному существованию, на
котором она остановилась, лишь за релятивное отношение голого явления, и
хотя и допускала и утверждала их более высокое единство в
{\em идее} вообще и,
например, в идее некоторого созерцающего рассудка, не пошла, однако, дальше
того релятивного отношения и дальше утверждения, что понятие всецело
отделено и остается отделенным от реальности; тем самым она признала
{\em истиной} то, что
сама объявила конечным познанием, а то, что она признала
{\em истиной} и
определенное понятие чего она установила, объявила чем-то непомерным,
недозволительным и лишь мысленным, а не реальным (Gedan\-kendin\-ge).

Так как здесь идет ближайшим образом речь об отношении
{\em логики} (а~не науки
вообще) к истине, то следует далее согласиться также и с тем, что логика,
как {\em формальная наука},
не может и не должна содержать в~себе также и той реальности,
которая составляет содержание дальнейших частей философии,
{\em наук о природе и духе}.
Эти конкретные науки, несомненно, имеют дело с более реальной
формой идеи, чем логика, но вместе с тем не в том смысле, что они
возвращаются опять к той реальности, от которой уже отказалось сознание,
возвысившееся от своего явления до науки, или же к употреблению таких форм
(каковы категории и определения рефлексии), конечность и неистинность
которых выяснились в логике. Напротив, логика показывает, как
{\em идея} поднимается на
такую ступень, где она становится творцом природы и переходит к форме
{\em конкретной непосредственности},
понятие которой, однако, снова разрушает и этот образ для
того, чтобы стать самим собой в виде
{\em конкретного духа}.
По сравнению с этими конкретными науками, имеющими и
сохраняющими, однако, в~себе логическое или понятие в качестве внутреннего
образователя, точно так же как оно [логическое] было их подготовителем и
прообразом, сама логика есть, конечно,
{\em формальная} наука,
но наука об {\em абсолютной форме},
которая есть внутри себя тотальность и содержит в~себе
{\em чистую идею самой истины}.
Эта абсолютная форма имеет в~себе самой свое содержание или
свою реальность; так как понятие не есть тривиальное, пустое тождество, то
оно имеет в моменте своей отрицательности или абсолютного процесса
определения различенные определения; содержание есть вообще не~что иное,
как такие определения абсолютной формы, есть положенное самой этой формой и
потому адекватное ей содержание. "--- Эта форма имеет поэтому
совершенно иную природу, чем обычно приписываемая логической
форме. Она уже {\em сама по себе}
есть {\em истина},
так как это содержание адекватно своей форме или эта
реальность адекватна своему понятию, и притом она есть
{\em чистая истина}, так
как определения этого содержания еще не имеют формы абсолютного инобытия
или абсолютной непосредственности. "--- Когда Кант (<<Критика
чистого разума>>, стр.~83) начинает обсуждать в отношении логики старый и
знаменитый вопрос: {\em что есть
истина}? он
{\em пренебрежительно допускает}
прежде всего, как нечто тривиальное, номинальное
объяснение, гласящее, что она есть согласие познания с его
предметом\pagenote{В~действительности это место
находится на стр.~82 второго издания <<Критики чистого разума>>. См.
{\em Кант}, Критика чистого разума, пер. Н. Лосского, Пгр. 1915,
стр.~64.\label{bkm:bm09}},
"--- дефиницию, имеющую громадную, даже величайшую ценность.
Если мы об этой дефиниции вспомним при чтении основного утверждения
трансцендентального идеализма, что
{\em разумное} познание
не может постигнуть {\em вещей в~себе},
что {\em безоговорочная
реальность} лежит вне
{\em понятия}, то тотчас
же станет ясно, что такой {\em разум},
который не может
{\em привести себя в согласие}
со своим предметом, с вещами в~себе, и такие
{\em вещи в~себе},
которое не согласуются с понятиями разума, такое понятие,
которое не согласуется с реальностью, и такая реальность, которая не
согласуется с понятием, суть
{\em неистинные представления}.
Если бы Кант сопоставил идею
{\em созерцающего рассудка}
с упомянутой дефиницией истины, то он отнесся бы к этой идее,
выражающей требуемое согласие реальности и понятия, не как к чему-то лишь
мысленному, а, наоборот, как к истине.

<<Что тут желают знать, "--- указывает далее Кант,
"--- это {\em всеобщий} и {\em надежный критерий
истины всякого познания;} [\ldots] он должен был бы быть таким,
который имел бы силу в применении ко всем познаниям
{\em без различия их предметов;} [\ldots] но так как в таком случае мы
{\em отвлекаемся от всякого} содержания познания
({\em от отношения к его объекту}), {\em а истина} касается
{\em именно этого содержания} познания, то было бы совершенно
{\em невозможно} и {\em несуразно} спрашивать, в чем заключается признак
{\em истинности этого содержания} познаний>>\pagenote{{\em Кант}
Критика чистого разума, стр.~82---83 по 2-му немецкому изданию.
Курсив принадлежит Гегелю.\label{bkm:bm10}}.
"--- Здесь очень определенно выражено обычное представление о
формальной функции логики, и приведенное рассуждение кажется весьма
убедительным. Но, во-первых, мы должны заметить, что обычная участь такого
рода формального рассуждения "--- забывать в своем словесном
изложении то, что оно сделало своей основой и о чем оно говорит. Было бы
несуразно, слышим мы, спрашивать о критерии {\em истинности содержания}
познания; но согласно приведенной выше дефиниции истину
составляет не {\em содержание}, а его {\em соответствие}
понятию. Такого рода содержание, как то, о котором говорится
здесь, {\em без понятия} есть нечто лишенное понятия и, стало быть,
лишенное и сущности; о критерии истинности такого содержания, конечно,
нельзя спрашивать, но по противоположному основанию: нельзя спрашивать
потому, что оно из-за своей непричастности понятию не есть
{\em требуемое соответствие},
а может быть лишь чем-то принадлежащим области непричастного
истине мнения. "--- Если мы оставим в стороне упоминание о
содержании, создающем здесь путаницу, в которую, однако, формализм
постоянно впадает и которая заставляет его, как только он вдается в
разъяснения, говорить обратное тому, что он хочет сказать, и остановимся
лишь на том абстрактном взгляде, согласно которому логическое есть нечто
лишь формальное и отвлекается от всякого содержания, то в таком случае мы
получим одностороннее знание, которое согласно этому взгляду не содержит в
себе никакого предмета, пустую, лишенную определений форму, которая, стало
быть, столь же мало есть {\em соответствие} (ибо
для соответствия обязательно требуются {\em две} стороны), "---
форму, которая равным образом не есть истина. "---
В лице априорного {\em синтеза} понятия
Кант обладал более высоким принципом, в котором могла быть познана
двойственность в единстве, стало быть, то, что требуется для истины; но
чувственная материя, многообразие созерцания слишком сильно властвовали над
ним, так что он не мог отойти от них и обратиться к рассмотрению понятия и
категорий, {\em взятых сами по себе}, и к спекулятивному философствованию.

Так как логика есть наука об абсолютной форме, то это
формальное для того, {\em чтобы быть
истинным}, должно иметь в~себе самом некоторое {\em содержание},
адекватное своей форме; это тем более верно, что логический
формальный элемент должен быть чистой формой и, следовательно, логическая
истина должна быть сам\'{о}й {\em чистой
истиной}. Вследствие этого указанное формальное должно быть
внутри себя гораздо богаче определениями и содержанием, а также должно
обладать бесконечно большей силой воздействия на конкретное, чем это
обыкновенно принимают\pagenote{Ср. замечание Маркса о том, что
<<до Гегеля логики по профессии упускали из вида формальное содержание (den
Formin\-halt) различных типов суждений и умозаключений>>
({\em Маркс}, Das Kapital, Bd.~1, Hamburg 1867, S.~21), т.~е. то
{\em реальное содержание}, которое имеется в {\em логической форме} суждений и
умозаключений.\label{bkm:bm11}}.
Логические законы сами по себе (если вычесть все то, что и
помимо этого гетерогенно, "--- прикладную логику и прочий
психологический и антропологический материал) сводятся обыкновенно, кроме
предложения о противоречии, еще к нескольким скудным предложениям об
обращении суждений и о формах умозаключений. Даже сюда относящиеся формы,
равно как и их дальнейшие определения, излагаются здесь лишь как бы
исторически, не подвергаются критике с целью установить, истинны ли они в
себе и для себя. Так, например, форма положительного суждения
рассматривается как нечто вполне правильное в~себе, причем
считается, что только от содержания данного суждения всецело зависит,
истинно ли оно. Есть ли эта форма {\em в
себе и для себя} форма истины, не диалектично ли внутри себя
высказываемое ею предложение (<<{\em единичное есть некоторое
всеобщее}>>), "--- об исследовании этого вопроса не думают. Просто
признается, что это суждение само по себе способно содержать в
себе истину и что вышеуказанное предложение, высказываемое всяким
положительным суждением, истинно, хотя непосредственно явствует, что ему не
хватает того, что требуется определением истины, а именно, согласия понятия
со своим предметом. Если принимать сказуемое, которое есть здесь всеобщее,
за понятие, а подлежащее, которое есть единичное, за предмет, то они не
согласуются одно с другим. Если же {\em абстрактное всеобщее},
которое сказуемое представляет собой, еще не составляет
понятия (для понятия, без сомнения, требуется нечто большее) и если такое
подлежащее равным образом имеет еще немногим больше, чем грамматический
смысл, то как может суждение содержать в~себе истину, коль скоро его
понятие и предмет между собой не согласуются или ему даже вообще недостает
понятия и, пожалуй, также и предмета? "--- Поэтому {\em невозможным} и
{\em несуразным} оказывается именно желание облечь истину в такие формы, как
положительное суждение или суждение вообще. Точно так же как философия
Канта не рассматривала категорий в~себе и для себя, а лишь на том неудачном
основании, что они, дескать, суть субъективные формы самосознания, объявила
их конечными определениями, неспособными содержать в~себе истину, так она
еще в меньшей мере подвергла критике формы понятия, составляющие содержание
обычной логики. Эта философия, напротив, использовала часть указанных форм
(а~именно, функции суждений) для определения категорий и признала их
правильными предпосылками. Если даже не видеть в логических формах ничего
другого, кроме формальных функций мышления, то и в таком случае они
заслуживали бы исследования, в какой мере они сами по себе соответствуют
{\em истине}. Логика, не
дающая такого исследования, может изъявлять притязание самое большее на
значение естественно-исторического описания явлений мышления в том виде, в
каком мы их пред-находим. {\em Аристотелю}
принадлежит бесконечная заслуга, которая должна наполнять нас
величайшим уважением перед силой этого ума, впервые предпринявшего такое
описание. Но необходимо идти дальше и познать отчасти систематическую
связь\pagenote{Ср. слова Энгельса:
<<Диалектическая логика, в противоположность старой, чисто формальной
логике, не довольствуется тем, чтобы перечислить и сопоставить без связи
формы движения мышления, т.~е. различные формы суждения и умозаключения.
Она, наоборот, выводит эти формы одну из другой, устанавливает между ними
отношение субординации, а не координации, она развивает высшие формы из
низших>> ({\em Энгельс}\, Диалектика природы, М.~1936, стр.~100).\label{bkm:bm12}},
отчасти же ценность этих форм.

\section[Подразделение]{Подразделение}

Понятие, как оно было рассмотрено выше, оказывается единством {\em бытия} и
{\em сущности}. Сущность есть {\em первое отрицание} бытия, которое вследствие
этого стало {\em видимостью;} понятие есть {\em второе}
отрицание или отрицание этого отрицания; следовательно,
понятие есть восстановленное бытие, но восстановленное как его бесконечное
опосредствование и отрицательность внутри себя самого. Поэтому в понятии
{\em бытие} и {\em сущность} уже не имеют того определения, в котором они суть
{\em бытие} и {\em сущность}, и равным образом не находятся лишь в таком
единстве, при котором каждое {\em светится} в другом.
Понятие поэтому диференцирует себя не на эти определения. Оно есть истина
субстанциального отношения, в котором бытие и сущность достигают друг через
друга своей исполненной самостоятельности и своего определения. Истиной
субстанциальности оказалось {\em субстанциальное тождество},
которое есть равным образом и только {\em положенность}.
Положенность есть {\em наличное бытие} и {\em различение;} поэтому
в-себе-и-для-себя-бытие достигло в понятии адекватного себе и истинного
наличного бытия, ибо указанная положенность есть само
в-себе-и-для-себя-бытие. Эта положенность образует собой различие понятия
внутри его самого; его {\em различия}, ввиду того что
оно\pagenote{Немецкий текст тут испорчен. Напечатано:
<<weil {\em sie} unmit\-tel\-bar das An-und-fursichsein {\em ist}>>.
Между тем, ни в этой, ни в предыдущей фразе нет ни одного
существительного женского рода, к которому могло бы относиться местоимение
<<sie>>. Приходится прибегать к конъектурам. Лассон предлагает читать:
<<es\ldots ist>>. Покойный Б.~Г.~Столпнер предлагал:
<<sie\ldots sind>>. Нам представляется
более обоснованным чтение: <<er (т.~е. der Begriff)\ldots ist>>.
Перевод сделан в соответствии с этой последней конъектурой, в подтверждение
которой можно сослаться на два обстоятельства: 1) под словами <<seine
Unter\-schie\-de>>, непосредственно предшествующими вышеприведенному
придаточному предложению, могут иметься в виду только <<различия
{\em понятия}>>; 2) третья фраза следующего абзаца начинается словами: <<Weil
er (т.~е. der Begriff) das An-und-fursichsein ist>>.\label{bkm:bm13}}
непосредственно есть в-себе-и-для-себя-бытие, суть сами
{\em все понятие целиком;} они суть {\em всеобщие в
своей определенности и тождественны со своим отрицанием}.

Это вот и есть само понятие понятия. Но это {\em пока что} есть
{\em только} его понятие, или, иначе говоря, само оно есть также
{\em только} понятие. Так
как оно есть в-себе-и-для-себя-бытие, поскольку последнее есть
положенность, или, иначе говоря, абсолютная субстанция, поскольку она
обнаруживает {\em необходимость} различенных субстанций как
{\em тождество}, то это
тождество само должно положить то, что оно есть. Моменты движения отношения
субстанциальности, через которые понятие {\em возникло}, и
реальность, которую они представляют собой, находятся пока что лишь в
состоянии перехода к понятию; эта реальность еще не есть
{\em собственное} определение {\em понятия},
происшедшее из него; она принадлежала сфере необходимости,
его же реальность может быть лишь его {\em свободным}
определением, "--- таким наличным бытием, в
котором оно (понятие) тождественно с собой и моменты которого суть понятия
и {\em положены} самим понятием.

Поэтому понятие, {\em во-первых}, есть истина лишь {\em в~себе;}
так как оно есть {\em только} нечто {\em внутреннее}, то оно
есть в такой же степени {\em только} нечто {\em внешнее}.
Оно есть {\em вначале} вообще нечто {\em непосредственное} и
в этом виде его моменты имеют форму {\em непосредственных, неподвижных
определений}. Оно выступает как {\em определенное понятие},
как сфера голого {\em рассудка}. "--- Так как
эта форма непосредственности есть пока что еще не адекватное его природе
наличное бытие (ибо понятие есть соотносящееся только с собой самим
{\em свободное}), то она есть некоторая {\em внешняя}
форма, в которой понятие может иметь значение не чего-то
в-себе-и-для-себя сущего, а чего-то {\em лишь положенного}
или {\em субъективного}. "--- Фаза {\em непосредственного}
понятия образует собой ту стадию, на которой понятие есть
субъективное мышление, некоторая внешняя для {\em предмета} (der Sache)
рефлексия. Эта ступень образует собой поэтому {\em субъективность} или
{\em формальное понятие}. Его внешний характер проявляется в
{\em неподвижном бытии} его {\em определений},
вследствие чего каждое из них выступает особо, как нечто
изолированное, качественное, находящееся лишь во внешнем соотношении со
своим иным. Но тождество понятия, которое именно и составляет их
{\em внутреннюю} или {\em субъективную}
сущность, приводит их в диалектическое движение, посредством
которого снимается их обособленность друг от друга, а вместе с тем и
отделение понятия от предмета, и, как их истина, возникает
{\em тотальность}, образующая собой {\em объективное понятие}.

{\em Во-вторых}, понятие в своей {\em объективности}
есть {\em сам сущий в~себе и для себя предмет}. Своим необходимым дальнейшим
определением {\em формальное}
понятие обращает само себя в предмет и тем самым утрачивает
характер субъективного и внешнего отношения к последнему. Или, обратно,
объективность оказывается {\em выступившим} из своей {\em внутренности}
и перешедшим в наличное бытие {\em реальным понятием}. "---
В этом тождестве с предметом оно имеет тем самым {\em собственное} и
{\em свободное} наличное бытие. Но это еще {\em непосредственная},
еще не {\em отрицательная} свобода. Как единое с предметом, понятие
{\em погружено} в него;
его различия суть объективные существования, в которых оно само снова есть
нечто {\em внутреннее}. Как душа объективного наличного бытия, оно должно
{\em сообщить себе} ту форму {\em субъективности}, которой оно в качестве
{\em формального} понятия обладало {\em непосредственно;} таким образом, оно
{\em в форме} свободного понятия, которой оно еще не обладало в
объективности, выступает рядом с последней и при этом обращает то тождество
с нею, которым оно, как {\em объективное} понятие, обладает
{\em в~себе и для себя}, также и в {\em положенное} тождество.

В этом своем завершении, в котором оно в своей объективности
обладает также и формой свободы, {\em адекватное понятие} есть
{\em идея}. {\em Разум}, который есть сфера идеи, есть {\em раскрывшаяся}
сама себе {\em истина}, в которой понятие имеет безоговорочно адекватную себе
реализацию и свободно постольку, поскольку оно познает этот свой
объективный мир в своей субъективности и последнюю "--- в том объективном мире.

\part[Первый отдел\\ СУБЪЕКТИВНОСТЬ]{Первый отдел. Субъективность}

Понятие есть, прежде всего, {\em формальное}, понятие {\em в начале} или
понятие как {\em непосредственное}. "---
В непосредственном единстве его различенность или
положенность, {\em во-первых}, сама ближайшим образом проста и есть только
{\em некоторая видимость}, так что моменты различия суть непосредственно
тотальность понятия и образуют собой только {\em понятие как таковое}.

Но, {\em во-вторых}, так как оно есть абсолютная отрицательность, то оно
расщепляет себя и полагает себя как {\em отрицательное} или
как {\em иное} себя самого; и притом именно потому, что оно есть пока что лишь
{\em непосредственное} понятие, это полагание или различение имеет то
определение, что моменты оказываются {\em безразличными друг к другу}
и каждый из них становится сам по себе; единство понятия
остается в этом {\em разделении} лишь внешним соотношением. Как
{\em таковое} соотношение своих положенных как {\em самостоятельные} и
{\em безразличные} моментов, оно есть {\em суждение}.

{\em В-третьих}, хотя
суждение и содержит в~себе единство понятия, исчезнувшего в свои
самостоятельные моменты, это единство все же не {\em положено}.
Положенным оно становится через диалектическое движение
суждения, которое тем самым становится {\em умозаключением},
вполне положенным понятием, поскольку в умозаключении
положены столь же его моменты как {\em самостоятельные}
крайние члены, сколь и {\em опосредствующее} их {\em единство}.

Но так как {\em непосредственно} само это {\em единство}, как соединяющий
{\em средний член}, и {\em моменты}, как {\em самостоятельные}
крайние члены, ближайшим образом противостоят друг другу, то
это противоречивое отношение, имеющее место в
{\em формальном умозаключении}, снимает себя, и {\em полнота} понятия
переходит в единство {\em тотальности},
субъективность понятия "--- в его {\em объективность}.

\bigskip

\chapter[{\em Первая глава} Понятие]{Первая глава. Понятие}

Под словом <<{\em рассудок}>> обыкновенно обозначают способность
понятий вообще и постольку различают между
ним и {\em силой суждения} и способностью умозаключения как формальным
{\em разумом}. Но главным образом он противополагается
{\em разуму;} однако постольку он означает способность не понятия вообще, а
{\em определенных} понятий, причем господствует представление, будто понятие
есть {\em только} нечто {\em определенное}. Если
отличать рассудок в этом его значении от формальной силы суждения и
формального разума, то под ним следует понимать способность {\em отдельных}
определенных понятий. Ибо суждение и умозаключение или разум
сами, как формальное, суть лишь нечто {\em рассудочное},
поскольку они подчинены форме абстрактной понятийной
определенности. Но здесь понятие считается вообще не чем-то только
абстрактно определенным; поэтому рассудок от разума следует отличать лишь в
том смысле, что первый есть только способность понятия вообще.

Это всеобщее понятие, подлежащее здесь теперь рассмотрению,
содержит в~себе три момента: {\em всеобщность, особенность} и
{\em единичность}. Различие и те определения, которые понятие сообщает себе
в процессе различения, образуют собой ту сторону, которая раньше была названа
{\em положенностью}. Так
как в понятии последняя тождественна с в-себе-и-для-себя-бытием, то каждый
из этих моментов есть столь же {\em все} понятие {\em целиком}, сколь и
{\em определенное понятие}, а равно и {\em некоторое определение} понятия.

{\em Во-первых}, оно есть {\em чистое понятие} или определение
{\em всеобщности}. Но чистое или всеобщее понятие есть вместе с тем лишь
{\em определенное} или {\em особенное} понятие,
ставящее себя бок о бок рядом с другими. Так как понятие есть тотальность
и, следовательно, в своей всеобщности или в чисто тождественном соотношении
с самим собой представляет собой по существу процесс определения и
различения, то оно в~себе самом имеет то мерило, по которому эта форма его
тождества с собой, пронизывая и объемля собой все моменты, определяет себя
столь же непосредственно к тому, чтобы быть
{\em только всеобщим} в противоположность различенности моментов.

{\em Во-вторых}, понятие выступает вследствие этого как такое-то
{\em особенное} или как {\em определенное} понятие, положенное как
отличное от других.

{\em В-третьих, единичность} есть
понятие, рефлектирующее себя из различия в абсолютную отрицательность. Это
есть вместе с тем тот момент, в котором оно перешло из своего тождества в
свое {\em инобытие} и становится {\em суждением}.

\section[А. Всеобщее понятие]{А. Всеобщее понятие}

Чистое понятие есть абсолютно бесконечное, безусловное и свободное. Здесь,
в начале изложения, имеющего своим {\em содержанием} понятие, нам надлежит
еще раз бросить взгляд на его генезис. {\em Сущность выросла} (ist geworden)
{\em из бытия}, а понятие "--- из сущности, стало быть, также из бытия.
Но это становление имеет значение {\em обратного толчка}, данного самому себе,
так что {\em ставшее} есть скорее {\em безусловное} и {\em первоначальное}.
{\em Бытие} в своем переходе в сущность стало {\em видимостью} или
{\em положенностью}, а {\em становление} или переход
в {\em другое} "--- некоторым {\em полаганием;} и, обратно, {\em полагание}
или рефлексия сущности сняло себя и восстановило себя в виде чего-то
{\em неположенного}, в виде {\em первоначального} бытия. Понятие есть
взаимопроникание этих моментов, так что качественное и первоначально сущее
имеет бытие лишь как полагание и лишь как возвращение внутрь себя, а эта
чистая рефлексия в~себя есть всецело {\em иностановление} или
{\em определенность}, которая именно поэтому есть равным образом
бесконечная, соотносящаяся с собой {\em определенность}.

Поэтому понятие есть, прежде всего, {\em абсолютное тождество с собой}
таким образом, что это тождество таково лишь как отрицание
отрицания или как бесконечное единство отрицательности с самой собой. Это
{\em чистое соотношение} понятия с собой, являющееся этим соотношением
вследствие того, что оно полагает себя через отрицательность, есть
{\em всеобщность} понятия.

Так как {\em всеобщность} есть в высшей степени {\em простое}
определение, то кажется, что она не допускает никакого
объяснения, ибо объяснение должно пускаться в определения и различения и
высказывать предикат о своем предмете, а то, что просто,
таким путем скорее изменяется, чем объясняется. Но природа
всеобщего как раз и состоит в том, что оно есть такое простое, которое
вследствие абсолютной отрицательности содержит {\em внутри себя}
наивысшую степень различия и определенности. {\em Бытие} просто, как
{\em непосредственное;} поэтому оно есть некоторое лишь {\em имеемое в~виду}
(ein nur Gemeintes), и про него нельзя сказать, чт\'{о} оно такое; поэтому
оно непосредственно совпадает со своим иинымс
{\em небытием}\pagenote{Ср. <<Учение о бытии>>, стр.~62---63.\label{bkm:bm14}}.
Его понятие именно и состоит в том, что оно есть такое
простое, которое непосредственно исчезает в своей
противоположности; его понятием служит {\em становление}. {\em Всеобщее же},
напротив, есть такое простое, которое вместе о тем есть
{\em самое богатое внутри себя самого}, ибо оно есть понятие.

Поэтому оно есть, {\em во-первых}, простое соотношение с собой самим; оно
только {\em внутри} себя. Но это тождество есть, {\em во-вторых}, внутри себя
абсолютное {\em опосредствование;} однако не нечто {\em опосредствованное}.
О~таком всеобщем, которое есть опосредствованное, а именно об {\em абстрактном}
всеобщем, противоположном особенному и единичному, может идти речь лишь тогда,
когда мы дойдем до определенного понятия. "--- Но и {\em абстрактное} всеобщее
уже подразумевает, что, для того чтобы получить его, нужно {\em отбросить}
прочие определения конкретного. Эти определения, как вообще детерминации, суть
{\em отрицания;} равным образом и {\em процесс отбрасывания} их есть, далее,
некоторое {\em отрицание}. Таким образом, и в абстрактном всеобщем также имеет
место отрицание отрицания. Но это двойное отрицание представляют себе так,
будто оно {\em внешне} абстрактному всеобщему и будто и отбрасываемые
дальнейшие свойства конкретного отличны от сохраненного, составляющего
содержание абстрактного всеобщего, и эта операция отбрасывания остальных и
удерживания одного совершается вне его. Таким {\em внешним} по отношению к
этому движению всеобщее еще не определило себя; оно само еще есть внутри себя
то абсолютное опосредствование, которое именно и есть отрицание отрицания или
абсолютная отрицательность.

Соответственно этому первоначальному единству первое отрицательное или
{\em определение} не есть, прежде всего, какой-либо предел для всеобщего,
а последнее {\em сохраняется в нем} и положительно тождественно с собой.
Категории бытия были как понятия по существу этими тождествами определений
с самими собой в их пределе или их инобытии; но это тождество было лишь
{\em в~себе} понятием; оно (тождество) еще не было проявлено. Поэтому
качественное определение пропадало как таковое в своем другом и имело своей
истиной некоторое {\em отличное} от него самого определение. Напротив,
всеобщее, даже когда оно влагает себя в некоторое определение, {\em остается}
в~нем тем же, что оно есть. Оно есть {\em душа} того конкретного, в котором оно
обитает, не стесненное и равное самому себе в его многообразии и разности. Оно
не вовлекается вместе с этим многообразием в процесс {\em становления},
а {\em продолжает себя} сквозь него в непомутненной чистоте и обладает силой
неизменно, бессмертно сохранять себя.

Но про него равным образом нельзя сказать, как это имеет место в определениях
рефлексии, что оно {\em светит} только в свое другое. Определение рефлексии,
как нечто {\em релятивное}, но только соотносится с собой, но есть некоторое
<<{\em отношение к}>> (ein Verhalten). Оно {\em дает себя знать} в своем
другом, но {\em светится} только лишь в нем, и свечение каждого в другом или
процесс их взаимного определения имеет при их самостоятельности форму
некоторого внешнего действования. "--- Напротив, {\em всеобщее} положено как
{\em сущность} своего определения, как {\em собственная положительная природа}
последнего. Ибо определение, составляющее его отрицательное, имеет бытие в
понятии безоговорочно только как некоторая {\em положенность} или, по существу,
лишь вместе с тем как некоторое отрицательное отрицательного, и оно имеет бытие
лишь как это тождество отрицательного с собой, каковое тождество и есть
всеобщее. Постольку последнее есть также и {\em субстанция} своих определений,
но так, что то, чт\'{о} для субстанции как таковой было чем-то {\em случайным},
есть собственное {\em опосредствование} понятия с самим собой, его собственная
{\em имманентная рефлексия}. Но это опосредствование, которое поднимает
случайное прежде всего до {\em необходимости}, есть {\em проявленное}
соотношение; понятие не есть бездна бесформенной субстанции или необходимость в
смысле {\em внутреннего} тождества отличных друг от друга и взаимно
ограничивающих друг друга вещей или состояний\pagenote{Имеется в виду
<<философия тожества>> Шеллинга и его последователей. Ср. примечание~93 к т.~I
<<Науки логики>>.\label{bkm:bm15}}, но, как абсолютная отрицательность,
представляет собой некое формирующее и созидающее начало; и так как определение
выступает здесь не как предел, а безоговорочно в такой же мере и как снятое,
как положенность, то свечение (der Schein) есть здесь явление как явление
{\em тождественного}.

Всеобщее есть поэтому {\em свободная} мощь; оно есть оно же само и захватывает
свое иное; но не как нечто {\em насильственное}, а как то, что скорее в этом
ином покоится и находится {\em у себя самого}. Мы его назвали свободной мощью,
но мы также могли бы назвать его {\em свободной любовью} и {\em безграничным
блаженством}, ибо оно есть отношение к {\em различенному} лишь как отношение
к {\em себе самому;} в различенном оно возвратилось к себе самому.

Мы только что упомянули об {\em определенности}, хотя понятие, как пока что
лишь всеобщее и лишь {\em тождественное} с
собой, еще до нее не дошло. Но нельзя говорить о всеобщем, не упомянув об
определенности, которая, строже говоря, есть особенность и единичность; ибо
оно заключает ее в своей абсолютной отрицательности в~себе и для себя;
таким образом, определенность не берется откуда-то извне, когда о ней
говорят по поводу всеобщего. Как отрицательность вообще, или, иначе говоря,
как взятое со стороны {\em первого, непосредственного} отрицания, оно
имеет в~себе самом определенность вообще как {\em особенность;} как
{\em второе} отрицание, отрицание отрицания, оно есть {\em абсолютная
определенность} или {\em единичность} и {\em конкретность}. "--- Всеобщее,
стало быть, есть тотальность понятия, оно есть
конкретное, не нечто пустое, а скорее имеющее {\em содержание}
благодаря своему понятию, "--- такое содержание,
в котором оно не только сохраняет себя, но которое есть его собственное и
имманентно ему. Можно, правда, абстрагировать от содержания, но тогда
получается не всеобщность понятия, а лишь {\em абстрактное},
которое есть изолированный, неполный момент понятия и в котором нет истины.

При ближайшем рассмотрении всеобщее оказывается этой
тотальностью следующим образом. Поскольку оно имеет внутри себя
определенность, последняя есть не только {\em первое} отрицание,
но и рефлексия этого отрицания в~себя. Взятое с этим первым отрицанием
особо, оно есть {\em особенное},
как это будет сейчас рассмотрено; но и в этой определенности
оно по существу еще есть всеобщее; эту сторону мы еще должны рассмотреть
здесь. "--- \label{bkm:bm23a}А именно, эта определенность, как
находящаяся в понятии, есть тотальная рефлексия, есть {\em двоякое свечение}
(Schein): во-первых, свечение {\em во-вне}, рефлексия в иное, и, во-вторых,
свечение {\em во-внутрь}, рефлексия в~себя. Первое, внешнее свечение образует
собой некоторое различие по отношению {\em к другому;} тем самым всеобщее имеет
{\em особенность}, находящую свое разрешение в некотором более высоком
всеобщем. Хотя оно теперь есть лишь некоторое относительно-всеобщее, оно не
утрачивает своего характера всеобщего; оно сохраняет себя в своей
определенности и притом не только так, что оно в соединении с ней остается
лишь безразличным к ней, "--- в таком случае оно было бы лишь {\em сложено}
с~нею, "--- но так, что оно есть именно то, что только что было названо
{\em свечением во-внутрь}. Определенность, как определенное {\em понятие},
{\em повернута} из внешности {\em обратно внутрь себя;} она есть тот
собственный, имманентный {\em характер}, который есть нечто существенное
благодаря тому, что, будучи вобран во всеобщность и проникнут ею, имея равный
с~нею объем и будучи тождественным с нею, он вместе с тем также и пронизывает
ее; этот-то характер и принадлежит {\em роду} как нераздельная с всеобщностью
определенность. Он постольку не есть некий обращенный во-вне {\em предел},
а {\em положителен}, так как он в силу всеобщности находится в свободном
соотношении с самим собой. Таким образом, и определенное понятие остается
внутри себя бесконечно свободным понятием.

Что же касается другой стороны, соответственно которой род
ограничен своим определенным характером, то мы уже заметили, что
этот род, как нижестоящий, находит свое разрешение в
некотором более высоком всеобщем. Последнее может в свою очередь также быть
понимаемо как род, но как более абстрактный; он принадлежит, однако,
опять-таки лишь к той стороне определенного понятия, которая направлена
вовне. Истинно же более высокое всеобщее есть то, в котором эта
направленная во-вне сторона вбирается обратно во-внутрь, то второе
отрицание, в котором определенность выступает всецело только
{\em как} положенность
или, иначе говоря, {\em как}
свечение (видимость). Жизнь, <<я>>, дух, абсолютное понятие не
суть всеобщие только в смысле высших родов, а суть
{\em конкретные},
определенности которых опять-таки не суть только виды или
низшие роды, но которые (конкретные) в своей реальности находятся всецело
только внутри себя и полны собою. Поскольку жизнь, <<я>>, конечный дух также
суть лишь определенные понятия, они находят свое абсолютное разрешение в
том всеобщем, которое должно быть понимаемо как истинно абсолютное понятие,
как идея бесконечного духа,
{\em положенность} коего
есть бесконечная, прозрачная реальность, в которой он созерцает свое
творение и в нем "--- самого себя.

Истинное, бесконечное всеобщее, которое непосредственно внутри
себя есть как особенность, так и единичность теперь должно быть сначала
рассмотрено ближе как
{\em особенность}. Оно
{\em определяет} себя
свободно; получение им характера чего-то конечного не есть переход, имеющий
место лишь в сфере бытия; оно есть
{\em творческая мощь},
как абсолютная отрицательность, соотносящаяся с собой самой.
Оно, как таковая творческая мощь, есть различение внутри себя, а последнее
есть {\em процесс определения}
благодаря тому, что различение едино с всеобщностью. Тем
самым оно есть полагание самих различий, как всеобщих, соотносящихся с
собой. Этим они становятся
{\em фиксированными},
изолированными различиями. Изолированное
{\em устойчивое наличие}
конечного, которое прежде определяло себя как его
для-себя-бытие, а также как вещность, как субстанция, есть в своей истине
всеобщность, каковой формой бесконечное понятие облекает свои различия, "---
формой, которая сама как раз и есть одно из его различий.
В~этом и состоит {\em творческая
деятельность} понятия, которая может быть постигнута лишь в
самой этой его наивнутреннейшей сердцевине.

\section[В. Особенное понятие]{В. Особенное понятие}

{\em Определенность} как таковая принадлежит к бытию и качественному; как
определенность понятия она есть {\em особенность}. Она не есть {\em граница},
т.~е. не относится к чему-то {\em иному}, как к своему {\em потустороннему}, а,
наоборот, как только что обнаружилось, она есть собственный, имманентный момент
всеобщего; последнее находится поэтому в особенности не при чём-то ином, а
всецело остаётся при самом себе.

Особенное содержит в~себе ту всеобщность, которая составляет его субстанцию;
род {\em неизменен} в своих видах; виды разнятся не от всеобщего, а только
{\em друг от друга}. Особенное имеет с другими особенными, к которым оно
относится, одну и ту же всеобщность. В~то же время ввиду их тождества со
всеобщим их разность\pagenote{<<Разность>> (Ver\-schieden\-heit) в смысле
многообразия, как совокупность разных видов одного рода.\label{bkm:bm16}},
{\em как таковая} всеобща; она есть {\em целокупность}. "--- Особенное,
следовательно, не только {\em содержит} всеобщее, но показывает последнее
также и {\em через свою определенность;} тем самым всеобщее образует собой
ту {\em сферу}, которую должно исчерпать особенное. Эта целокупность, поскольку
определенность особенного берется как голая {\em разность}, выступает как
{\em полнота}. В этом смысле виды даны полностью, если именно их
не~{\em имеется} больше, [чем перечислено]. Для них нет никакого внутреннего
мерила или {\em принципа}, так как {\em разность} именно и есть то лишенное
единства различие, в котором всеобщность, представляющая собой, сама по себе,
абсолютное единство, есть лишь внешний рефлекс и некоторая неограниченная,
случайная полнота. Но разность переходит в {\em противоположение},
в {\em имманентное соотношение} разных. Особенность же, как всеобщность, есть
такое имманентное соотношение в~себе и для себя, а не благодаря переходу; она
целокупность в самой себе и {\em простая} определенность, по существу
{\em принцип}. Она не имеет никакой {\em иной} определенности, кроме той,
которая положена самим всеобщим и получается из последнего следующим образом.

Особенное есть само всеобщее, но оно есть его различие или его
соотношение с чем-то {\em иным}, его {\em свечение
вовне;} но налицо нет никакого иного, от которого особенное было бы отлично,
кроме самого всеобщего. "--- Всеобщее определяет {\em себя;} таким
образом, оно само есть особенное; определенность есть {\em его} различие; оно
отлично лишь от самого себя. Его виды суть поэтому лишь (a)
само всеобщее и (b) особенное. Всеобщее как понятие есть оно
же само и его противоположность, которая опять-таки есть оно же само как
его положенная определенность; оно охватывает собой последнюю и находится в
ней у себя. Таким образом, оно есть тотальность и принцип своей разности,
которая всецело определена лишь им самим.

\label{bkm:bm22a}Нет поэтому никакого другого истинного
деления, кроме того, при котором понятие отодвигает само себя в сторону,
как {\em непосредственную}, неопределенную всеобщность; именно это
неопределенное создает его определенность или, иначе говоря,
создает то обстоятельство, что оно есть некоторое
{\em особенное}. {\em И то, и другое} есть особенное, и потому они
{\em соподчинены}. И~то и другое, как особенное, есть вместе с тем
{\em противостоящее} всеобщему {\em определенное;} в
этом смысле они называются {\em подчиненными}
всеобщему. Но именно потому это всеобщее, в {\em противоположность}
которому определено особенное, тем самым скорее само есть
также {\em лишь одно} из противостоящих. Если мы говорим здесь о
{\em двух противостоящих},
то мы должны поэтому также сказать, что оба они составляют
особенное не только {\em вместе}, не только как бы в том смысле,
что они лишь для внешней рефлексии {\em одинаковы}
в том отношении, что оба они суть особенные, а в том смысле,
что их определенность {\em друг против
друга} есть по существу вместе с тем лишь {\em одна}
определенность, та отрицательность, которая во всеобщем
{\em проста и едина} (einfach~ist).

Различие, каковым оно обнаруживает себя здесь, таково, как оно
есть в своем понятии и тем самым в своей истине. Все предыдущие различия
обладают этим единством в
понятии\pagenote{<<В понятии>> означает у Гегеля то
же самое, что <<в~себе>>, т.~е. в неразвернутом виде, в возможности. См.,
например, т.~I <<Науки логики>>, стр.~79 (<<в~себе или в понятии>>) и стр.~153
(<<в~себе или в возможности>>).\label{bkm:bm17}}. Как непосредственное
различие в бытии, оно есть {\em граница} чего-то {\em иного;} в том
виде, в каком оно выступает в рефлексии, оно есть относительное различие и
положено как соотносящееся по существу со своим иным; здесь, стало быть,
единство понятия начинает становиться {\em положенным;} но вначале оно есть
лишь {\em свечение} (der Schein) в некотором ином. "--- Переход и
разложение этих определений имеют лишь тот истинный смысл, что они
достигают своего понятия и своей истины; бытие, наличное бытие, нечто, или
целое и части и~т.~д., субстанция и акциденции, причина и действие суть,
взятые сами по себе, определения, созданные мыслью; как определенные
{\em понятия} они
постигаются постольку, поскольку каждое из них познано в единстве со своим
другим или противоположным. "--- Например, целое и части,
причина и действие и~т.~д. еще не суть такие разные, которые были бы
определены по отношению друг к другу, как {\em особенные}, ибо хотя они
{\em в~себе} и доставляют одно понятие, однако их {\em единство} еще
не~достигло формы {\em всеобщности;} и точно так же то {\em различие}, которое
имеется в этих отношениях, еще не обладает той формой, по которой оно есть
{\em единая} определенность. Например, причина и действие суть не два разных
понятия, а лишь {\em одно определенное} понятие, и причинность, как всякое
понятие, есть нечто {\em простое}.

Что касается полноты, то оказалось, что определенность особенного
{\em полностью} исчерпывается различием {\em всеобщего} и {\em особенного}
и что лишь эти два определения составляют его особенные виды.
В~{\em природе} мы находим,
правда, более двух видов одного какого-нибудь рода, равно как и эти многие
виды не могут иметь вскрытого выше отношения друг к другу. Но в том-то и
состоит бессилие природы, что она не в состоянии сохранить и выразить собой
строгость понятия и растекается в такое чуждое понятию слепое многообразие.
Многообразие ее родов и видов и бесконечная разность ее образований может
вызывать в нас {\em восхищение}, ибо в восхищении {\em нет
понятия}, и его предметом служит то, что лишено разумности.
Так как природа есть вне-себя-бытие понятия, то ей предоставлено
растекаться в этой разности, подобно тому как дух, хотя он и имеет понятие
в образе понятия, пускается также и в представливание и носится по
бесконечному многообразию представлений. Многообразные роды и виды,
встречающиеся в природе, не должны считаться чем-то более высоким, чем
произвольные причудливые мысли духа в его представлениях. И~те и другие,
правда, являют нам повсюду следы и предвосхищения понятия, но не изображают
последнее в верном отображении, так как представляют собой сторону его
свободного вне-себя-бытия; понятие есть абсолютная мощь именно потому, что
оно может свободно отпускать имеющееся в нем различие, дозволять ему, чтобы
оно приняло образ самостоятельной разности, внешней необходимости,
случайности, произвола, мнения, в которых, однако, мы должны видеть не
более, чем абстрактный аспект {\em ничтожности}.

{\em Определенность} особенного, мы это видели, {\em проста} как {\em принцип},
но она проста также и как момент тотальности, как определенность,
противостоящая {\em другой определенности}.
Понятие, поскольку оно определяет или различает себя,
направлено отрицательно на свое единство и сообщает себе форму одного из
своих идеализованных моментов {\em бытия;} как
определенное понятие оно обладает {\em наличным бытием}
вообще. Но это бытие имеет смысл уже не голой {\em непосредственности},
а всеобщности, т.~е. такой непосредственности, которая равна
самой себе благодаря абсолютному опосредствованию и которая вместе с тем
содержит в~себе также и другой момент "--- сущность или
рефлексию. Эта всеобщность, которой облечено определенное, есть
{\em абстрактная} всеобщность. Особенное имеет всеобщность внутри самого себя
как свою сущность; но поскольку определенность различия {\em положена} и тем
самым обладает бытием, {\em всеобщность} есть
в~нем форма, а определенность, как таковая, есть {\em содержание}.
Всеобщность становится формой постольку, поскольку различие
выступает как существенное, точно так же как, наоборот, в чисто всеобщем
оно имеется лишь как абсолютная отрицательность, а {\em не как} такое
различие, которое {\em положено} как таковое.

Определенность есть, правда, {\em абстрактное}, противостоящее {\em другой}
определенности; однако эта другая определенность есть лишь
сама всеобщность; последняя постольку есть также {\em абстрактная}
всеобщность; и определенность понятия или особенность есть
опять-таки и не~что иное, как определенная всеобщность. Понятие в ней
находится {\em вне себя;} поскольку именно {\em понятие} находится в
ней вне себя, абстрактно-всеобщее содержит в~себе все моменты понятия; оно
есть ($\alpha $) всеобщность, ($\beta $) определенность, ($\gamma $)
{\em простое} единство обеих; но это единство {\em непосредственно}, и
поэтому особенность выступает не {\em как} тотальность. {\em В~себе} она есть
также и эта {\em тотальность} и {\em опосредствование;}
она есть по существу {\em исключающее} соотношение {\em с~другим} или
{\em снятие отрицания}, а именно, {\em другой} определенности "--- той
{\em другой}, которая, однако, только предносится уму в виде мнения, ибо
непосредственно она исчезает и обнаруживает себя как то же самое, чем должна
была быть ее {\em другая}. Следовательно, эту всеобщность делает абстрактной
то обстоятельство, что опосредствование есть лишь {\em условие}, или, иначе
говоря, что оно {\em не положено в ней самой}. Так как оно не {\em положено},
то единство абстрактного имеет форму непосредственности, а содержание
"--- форму безразличия к своей всеобщности, ибо оно не выступает
как та тотальность, которую представляет собой всеобщность абсолютной
отрицательности. Абстрактно-всеобщее есть, стало быть, хотя и
{\em понятие}, но как {\em лишенное понятия},
как такое понятие, которое не положено как таковое.

Когда идет речь об {\em определенном понятии},
то обыкновенно имеется в виду исключительно лишь такое
{\em абстрактно-всеобщее}. Равным образом и под
{\em понятием} вообще большей частью разумеется лишь такое
{\em чуждое понятию} понятие, и слово <<{\em рассудок}>>
обозначает способность таких понятий. {\em Доказательство}
принадлежит этому рассудку, поскольку оно {\em движется}
посредством {\em цепи понятий}, т.~е. лишь посредством {\em определений}.
Такое движение по цепи понятий не выходит поэтому за пределы конечности и
необходимости; высшую его ступень представляет собой отрицательное
бесконечное, абстракция высшего существа, которое само есть не~что иное,
как определенность {\em неопределенности}.
Абсолютная субстанция, правда, не есть такая пустая
абстракция, а скорее представляет собой по своему содержанию тотальность.
Но и она абстрактна потому, что лишена абсолютной формы и понятие не
составляет ее наивнутреннейшей истины; хотя она есть тождество всеобщности
и особенности или мышления и внеположности, это тождество,
однако, не есть {\em определенность} понятия; наоборот, {\em вне} ее находится
некоторый рассудок и притом "--- именно потому, что он
находится вне ее, "--- случайный рассудок, в котором и для
которого она существует в различных атрибутах и
модусах\pagenote{Гегель здесь опять дает
неправильное истолкование учению Спинозы об отношении между субстанцией и
атрибутами (а~также и модусами). См. примечание~92 к т.~I <<Науки логики>>,
где приводятся цитаты из сочинений Спинозы, показывающие, что атрибуты, по
учению Спинозы, не только {\em мыслятся} интеллектом (рассудком) как
составляющие сущность субстанции, но и объективно {\em присущи} субстанции
безотносительно к интеллекту.\label{bkm:bm18}}.

Впрочем, абстракция не {\em пуста}, как о ней
обыкновенно выражаются; она есть {\em определенное}
понятие; она имеет содержанием какую-нибудь определенность; и
высшее существо, эта чистая абстракция, также обладает, как мы указали,
определенностью неопределенности; но определенность представляет собой
неопределенность, потому что, как предполагается, она {\em противостоит}
определенному. Однако, когда высказывают, что она такое, то
упраздняется то самое, что, как предполагается, она есть: ее высказывают
как то, что едино с определенностью, и, таким образом, из абстракции
восстановляется понятие и ее истина. "--- Но каждое
определенное понятие, разумеется, {\em пусто} постольку,
поскольку оно содержит в~себе не тотальность, а лишь некоторую одностороннюю
определенность. Если оно даже обладает конкретным содержанием в другом
отношении, например, <<человек>>, <<государство>>, <<животное>> и~т.~п.,
оно все же остается пустым понятием, поскольку его определенность не есть
{\em принцип} его различий; принцип содержит в~себе начало и сущность его
развития и реализации; всякая же иная определенность понятия бесплодна. Поэтому
если вообще обзывают понятие пустым, то при этом упускают из виду ту абсолютную
его определенность, которая есть различие понятия и единственно истинное
содержание в его стихии.

В связи с этим находится то обстоятельство, вследствие которого в новейшее
время относятся с неуважением к рассудку и ставят его столь ниже
разума\pagenote{Имеются в виду Шеллинг и шеллингианцы, а также Якоби и
романтики. См. примечание~78 к т.~I
<<Науки логики>>.\label{bkm:bm19}}; я говорю о той {\em неподвижности}, которую
рассудок сообщает определенностям и, стало быть, конечным предметам. Эта
неподвижность состоит в рассмотренной форме абстрактной всеобщности; через нее
они становятся {\em неизменными}. Ибо качественная определенность, равно как и
определение рефлексии по существу имеют бытие как {\em ограниченные} и через
свой предел находятся в соотношении со своим иным, а тем самым имеют в~себе
{\em необходимость} перехода и прехождения. Но та всеобщность, которой они
обладают в рассудке, сообщает им форму рефлексии в~себя, вследствие чего они
изъяты из соотношения с другим и стали {\em непреходящими}. Если в чистом
понятии эта вечность принадлежит к его природе, то его абстрактные определения
можно было бы назвать вечными сущностями лишь по {\em их форме;} а их
содержание не адекватно этой форме; они поэтому не суть истина и
непреходимость. Их содержание не адекватно форме потому, что оно не есть сама
определенность как всеобщая, т.~е. не есть тотальность различий понятия, или,
иначе говоря, не есть само вся форма целиком; но форма ограниченного рассудка
сама есть поэтому несовершенная, а именно, {\em абстрактная}
всеобщность. "--- Но, далее, следует признать
бесконечной силой рассудка тот факт, что он разделяет конкретное на
абстрактные определенности и постигает ту глубину различия, которая вместе
с тем единственно только и есть мощь, вызывающая их переход. Конкретное,
доставляемое нам {\em созерцанием}, есть {\em тотальность}, но
{\em чувственная} тотальность, "--- некоторая реальная материя,
безразлично существующая {\em внеположно} в
пространстве и времени; это отсутствие единства в данном многообразном,
отсутствие, характерное для содержания созерцания, отнюдь не должно же быть
вменено ему в заслугу и считаться его преимуществом перед содержанием
рассудочной мысли. Та изменчивость, которую многообразное являет в
созерцании, уже предвещает всеобщее; но то, что при этом делается предметом
созерцаний, есть лишь некоторое
{\em другое}, столь же
изменчивое, следовательно, лишь то же самое; занимает его место и является
не всеобщее. Но менее всего следовало бы науке, например, геометрии и
арифметике, вменять в заслугу тот
{\em созерцательный}
момент, который приносит с собой ее материя, и представлять
себе ее теоремы как основанные на этом созерцательном. Наоборот, этот
созерцательный момент служит причиной того обстоятельства, что материя
таких наук имеет низшую природу; созерцание фигур или чисел не помогает
научному их познанию; лишь мышление о них может доставить такое познание.
"--- Но поскольку под созерцанием понимают не только
чувственное, но и {\em объективную
тотальность}, оно представляет собой
{\em интеллектуальное созерцание},
т.~е. оно имеет своим предметом наличное бытие не в его
внешнем существовании, а то в нем, что представляет собой непреходящую
реальность и истину, "--- реальность, лишь поскольку она по
существу определена в понятии и через понятие,
{\em идею}, более
детальная природа которой выяснится позднее. То, что якобы дает созерцанию
как таковому преимущество перед понятием, есть внешняя реальность, нечто
лишенное понятия, получающее ценность лишь через понятие.

Поэтому следует признать, что рассудок представляет собой
бесконечную силу, определяющую всеобщее, или, наоборот, сообщающую
посредством формы всеобщности фиксированную устойчивость тому, что в
определенности само по себе лишено твердости, и не вина
рассудка, если не идут дальше этого. Только субъективное
{\em бессилие разума}
дозволяет этим определенностям иметь значимость в этом виде и
оказывается не в состоянии свести их обратно к единству посредством
противоположной абстрактной всеобщности диалектической силы, т.~е.
посредством их собственной специфической природы, а именно, посредством
понятия указанных определенностей. Рассудок, правда, сообщает им
посредством формы абстрактной всеобщности, так сказать, такую
{\em жесткость} бытия,
какой они не обладают в качественной сфере и в сфере рефлексии; но
посредством этого упрощения он их вместе с тем
{\em одухотворяет} и так
заостряет их, что они, как раз лишь доведенные до этой крайней
заостренности, получают способность раствориться и перейти в свою
противоположность. Наивысшая зрелость и наивысшая ступень, которых что-либо
может достигнуть, есть та ступень, на которой начинается его гибель. Твердо
фиксированный характер определенностей, в которых, как кажется, безысходно
увязает рассудок, иначе говоря, форма непреходящего есть форма
соотносящейся с собой всеобщности. Но она неотъемлемо принадлежит понятию,
и потому в ней самой уже выражено
{\em разложение} "--- и
притом бесконечно близкое "--- конечного. Эта всеобщность
непосредственно {\em изобличает}
определенность конечного и
{\em выражает} его
неадекватность ей. "--- Или, вернее сказать, его адекватность
уже имеется налицо; абстрактное определенное положено как единое с
всеобщностью, и именно поэтому оно положено не как стоящее особо
"--- постольку оно было бы лишь определенным, "--- а
только как единство себя и всеобщего, т.~е. как понятие.

Поэтому надо во всех отношениях отвергнуть обычное
разграничение между рассудком и разумом. Если понятие рассматривается как
чуждое разуму, то на это следует скорее смотреть как на неспособность
разума узнавать себя в понятии. Определенное и абстрактное понятие есть
{\em условие} или, вернее, {\em существенный момент
разума;} оно есть одухотворенная форма, в которой конечное
через ту всеобщность, в каковой оно соотносится с собой, возгорается внутри
себя, положено как диалектическое и, стало быть, есть само
{\em начало} явления разума.

Так как в предыдущем изложении определенное понятие изображено
в своей истине, то остается еще лишь указать, каким оно тем самым уже
положено. "--- Различие, которое есть существенный момент
понятия, но в чисто всеобщем еще не положено как таковое, вступает в
определенном понятии в свои права. Определенность в форме всеобщности
образует в соединении с нею простое; это определенное
всеобщее есть определенность, соотносящаяся с собой самой, "---
определенная определенность или абсолютная отрицательность,
положенная особо. Но соотносящаяся с самой собой определенность есть
{\em единичность}.
Подобно тому как всеобщность непосредственно уже сама по себе
есть особенность, подобно этому особенность столь же непосредственно сама
по себе есть {\em единичность},
на которую следует смотреть ближайшим образом как на третий
момент понятия, поскольку ее фиксируют как {\em противостоящую} двум
первым, но также и как на абсолютное возвращение понятия внутрь себя и
вместе с тем как на положенную утрату понятием самого себя.

\hegremark[Примечание]%
{Обычные виды понятий}%
{[Обычные виды понятий]\pagenote{См. примечание~18 к кн.~I <<Науки логики>>.}}

{\em Всеобщность, особенность} и {\em единичность} суть согласно
вышеизложенному {\em три} определенные понятия, если их именно желают
{\em считать}. Уже ранее было показано, что число есть неподходящая форма для
облечения в нее определений понятия\pagenote{См. т.~I <<Науки логики>>,
стр.~203---207.}. Но особенно не подходит она для выражения определений самого
понятия; число, поскольку оно имеет принципом одно, обращает считаемое в
совершенно обособленные и совершенно безразличные друг к другу [предметы].
В~предыдущем выяснилось, что различные определенные понятия суть безоговорочно
лишь {\em одно} и то же понятие, так что они отнюдь не могут быть облечены
в~форму числа, чтобы выпадать друг из друга.

В обычных изложениях логики мы встречаем различные
{\em подразделения} и {\em виды} понятий. В~них
сразу же бросается в глаза непоследовательность, состоящая в том, что виды
вводятся следующим образом: по количеству, качеству и так далее
{\em имеются} нижеследующие понятия. <<{\em Имеются}>> "--- этим не
выражается никакая другая правомерность, кроме того указания, что мы
{\em преднаходим} такие-то виды и что они являют себя согласно
{\em опыту}. Таким путем получается {\em эмпирическая логика}, "---
странная наука, {\em иррациональное} познание {\em рационального}.
Логика дает этим весьма плохой пример следования своим
собственным учениям; она разрешает ceбe самой делать обратное тому, что она
предписывает как правило, требуя, чтобы понятия были выведены и научные
положения (следовательно, и положение: имеются такие-то и такие-то
различные виды понятий) были доказаны. "--- Философия Канта
совершает в этом отношении дальнейшую непоследовательность: она
{\em заимствует} для {\em трансцендентальной логики}
категории в качестве так называемых основных
понятий из субъективной логики, в которой они были подобраны эмпирически.
Так как она сама признает это последнее обстоятельство, то непонятно,
почему трансцендентальная логика решается на заимствование из такой науки,
а не берется сразу же сама за дело эмпирически.

Приведем несколько примеров: понятия разделяются, главным образом, по их
{\em ясности}, а именно, на {\em ясные} и {\em смутные, отчетливые} и
{\em неотчетливые, адекватные} и {\em неадекватные}. Здесь можно также
упомянуть понятия {\em полные, излишние} и другие подобного рода излишние
вещи. "--- Что касается упомянутого подразделения по {\em ясности}, то сразу
обнаруживается, что эта точка зрения и относящиеся к ней различения
заимствованы из {\em психологических}, а не {\em логических} определений. Так
называемого {\em ясного} понятия, говорят нам, достаточно для того, чтобы
отличать один предмет от другого; но нечто подобное еще нельзя назвать
понятием, а это есть не~что иное, как {\em субъективное представление}. Что
такое {\em смутное} понятие, это должно оставаться его секретом, ибо в
противном случае оно было бы не смутным, а отчетливым понятием. "---
{\em Отчетливым}, говорят нам, является такое понятие, {\em признаки} которого
могут быть указаны. Таким образом, оно есть, собственно говоря,
{\em определенное понятие}. Признак (если только понимать под этим выражением
то, что в нем есть правильного) есть не~что иное, как {\em определенность} или
простое {\em содержание} понятия, поскольку это содержание отличают от формы
всеобщности. Но {\em признак} ближайшим образом не обязательно имеет это более
точное значение, а он есть вообще лишь некоторое определение, посредством
которого некий {\em третий} отмечает себе тот или иной предмет или понятие;
поэтому признаком может служить весьма случайное обстоятельство. Вообще он
выражает собой не столько имманентность и существенность определения, сколько
соотношение последнего с некоторым {\em внешним} рассудком. Если последний
действительно есть рассудок, то он имеет перед собой понятие и отмечает себе
последнее не чем иным, как тем, что {\em содержится в понятии}. Если же
признак отличен от того, что содержится в понятии, то он
есть некоторый {\em значок} или какое-либо иное определение,
принадлежащее к {\em представлению} вещи, а не к ее понятию. "--- Вопрос
о~том, что такое {\em неотчетливое} понятие, мы можем обойти совершенно
без рассмотрения как нечто излишнее.

Адекватное же понятие есть нечто более высокое; в нем,
собственно говоря, предносится уму соответствие между понятием и
реальностью, что уже есть не понятие как таковое, а {\em идея}.

Если бы {\em признак} отчетливого понятия был действительно самим определением
понятия, то логике доставили бы затруднение {\em простые} понятия,
которые согласно другому подразделению противопоставляются {\em сложным}. Ибо
если бы был указан истинный, т.~е. имманентный, признак простого понятия, то
нельзя было бы считать это понятие простым; поскольку же не указали бы
такого признака, понятие не было бы отчетливым. Тут выручает <<{\em ясное}>>
понятие. Единство, реальность и тому подобные определения признаются
{\em простыми} понятиями несомненно только потому, что логики оказались не
в~состоянии найти их определения и потому удовольствовались тем, чтобы иметь
о~них просто {\em ясное} понятие, т.~е. не иметь никакого. Для {\em дефиниции},
т.~е. для указания понятия, обыкновенно требуют указания рода и видового
отличия. Она дает, следовательно, понятие не как нечто простое, а как имеющее
{\em две} могущие быть сосчитанными {\em составные части}. Но такое понятие
не становится еще в силу этого чем-то {\em сложным}. "--- Уму говорящих
о~простом понятии предносится, повидимому, {\em абстрактная простота},
единство, не содержащее внутри себя различия и определенности
и потому не являющееся тем единством, которое свойственно понятию.
Поскольку предмет находится в представлении и, в особенности, в памяти или
поскольку он есть абстрактное определение мысли, он может быть совершенно
прост. Даже самый богатый по своему содержанию предмет, например, дух,
природа, мир, а также бог, если он облекается без всякого понятия в простое
представление о столь же простом выражении: дух, природа, мир, бог, есть
несомненно нечто простое, на чем сознание может остановиться, не выделяя
для себя дальше какого-либо специфического определения или признака; но
предметы сознания не должны оставаться такими простыми, не должны
оставаться представлениями или абстрактными определениями мысли, а должны
быть {\em постигнуты в понятии}, т.~е. их простота должна быть определена их
внутренним различием. "--- {\em Сложное} же понятие есть не более, как
деревянное железо. О~чем-то сложном можно, правда, иметь то или иное понятие,
но сложное понятие было бы чем-то худшим, чем {\em материализм}, который
признает сложным лишь {\em субстанцию души}, а {\em мышление} все же считает
{\em простым}. Необразованная рефлексия прежде всего набредает на сложность,
как на совершенно {\em внешнее} соотношение, на худшую форму, в которой могут
быть рассматриваемы вещи; даже самые низшие природы должны обладать некоторым
{\em внутренним} единством. А~чтобы форма самого неистинного существования была
перенесена на <<я>>, на понятие, "--- это более, чем можно было бы ожидать, это
должно быть рассматриваемо как неприличие и варварство.

Понятия, далее, подразделяются, в особенности, на {\em контрарные} и
{\em контрадикторные}. "--- Если бы при трактовании понятия дело шло о~том,
чтобы указать, какие существуют определенные понятия, то пришлось бы привести
всевозможные определения, "--- ибо {\em все} определения суть понятия и тем
самым определенные понятия, "--- и все категории {\em бытия}, равно как и все
определения {\em сущности}, надлежало бы привести как виды понятий. И~в самом
деле, в логиках "--- по капризу в одних {\em более}, а в других {\em менее},
"--- рассказывается о том, что имеются понятия {\em утвердительные,
отрицательные, тождественные, условные, необходимые} и~т.~д. Так как такие
определения уже лежат позади {\em природы самого понятия} и потому, когда они
приводятся по его поводу, находятся не на своем собственном месте, то они
допускают лишь поверхностные объяснения значений соответствующих слов и не
представляют здесь никакого интереса. "--- В основании {\em контрарных} и {\em
контрадикторных} понятий, "--- различение, которое здесь особенно выдвигается,
"--- лежит рефлексивное определение {\em разности} и {\em противоположности}.
Они рассматриваются как два отдельных {\em вида}, т.~е. каждое как неподвижно
существующее само по себе и безразличное к другому, рассматриваются без всякой
мысли о их диалектике и о внутреннем ничтожестве этих различений; как будто то,
что {\em контрарно}, не должно быть определено вместе с тем и как {\em
контрадикторное}. Природа и существенный переход тех форм рефлексии, которые
ими выражаются, рассмотрены нами в своем месте. В~понятии тождество развито
дальше во всеобщность, различие "--- в особенность, противоположение,
возвращающееся в основание, "--- в единичность. В~этих формах указанные выше
определения рефлексии таковы, каковы они суть в их понятии. Всеобщее оказалось
не только тождественным, но вместе е тем и разным или {\em контрарным} по
отношению к особенному и единичному и, далее, также и противоположным им или
{\em контрадикторным;} но в этом противоположении оно тождественно с ними и
есть их истинное основание, в котором они сняты. То же самое
справедливо об особенном и единичном, которые таким же
образом суть тотальность рефлексивных определений.

Далее, понятия подразделяются на {\em подчиненные} и {\em соподчиненные}, "---
различение, которое ближе касается определения понятия, а именно отношения
всеобщности и особенности, говоря о которых мы мимоходом и употребили эти
выражения\pagenote{См. выше, стр.~\pageref{bkm:bm22a}.\label{bkm:bm22}}.
Только обыкновенно их равным образом
рассматривают как совершенно неподвижные отношения и потому выставляют
относительно них ряд бесплодных положений. Наиболее пространная их
трактовка опять-таки касается отношения контрарности и контрадикторности к
подчинению и соподчинению. Так как {\em суждение} есть {\em соотношение между
определенными понятиями}, то лишь при его рассмотрении должно выясниться
истинное отношение [этих определений]. Указанная манера {\em сравнивать} эти
определения без всякой мысли о их диалектике и о беспрерывном изменении их
определения или, вернее, об имеющемся в них сочетании противоположных
определений делает чем-то бесплодным и бессодержательным все рассуждение о
том, что в них {\em согласно} и что нет, как будто это согласие или несогласие
есть нечто обособленное и постоянное. "--- Великий {\em Эйлер}, бесконечно
плодотворный и остроумный в схватывании и комбинировании глубочайших
отношений алгебраических величин, и, в особенности, сухорассудочный Ламберт
и другие пытались {\em обозначать} линиями, фигурами и~т.~д. этот род отношений
между определениями понятий; вообще имелось в виду {\em возведение} "--- или на
самом деле скорее низведение "--- способов логических отношений в некоторое
{\em исчисление}. Уже самая попытка такого обозначения сразу же являет себя
как сама по себе пустая затея, если сравнить между собой природу знака и того,
что здесь должно быть обозначено. Определения понятия "--- всеобщность,
особенность и единичность "--- несомненно тоже {\em разны}, как и линии или
буквы алгебры; далее, они также и {\em противоположны} и постольку допускают
применение знаков plus и minus. Но сами они, а кроме того и их соотношения,
если даже не идти дальше {\em подчинения} и {\em принадлежности}, имеют по
существу совершенно иную природу, чем буквы, линии и их соотношения, чем
равенство или различие величин, чем plus и minus, чем положение линий по
отношению друг к другу или их соединение в углы и положения замыкаемых ими
пространств. Подобного рода предметы имеют по сравнению с логическими
определениями ту особенность, что они {\em внешни} друг другу и обладают {\em
неизменным} определением. Если же понятия берутся так, что они
соответствуют таким знакам, то они перестают быть понятиями. Их определения
не суть нечто мертвенно-неподвижное, подобно числам и линиям, к составу
которых не принадлежат их соотношения; эти определения суть живые движения;
различенная определенность одной стороны непосредственно внутрення также и
другой стороне; то, что для чисел и линий было бы полным
противоречием, существенно для природы понятия. "--- Высшая
математика, которая тоже доходит до бесконечного и дозволяет себе
противоречия, уже больше не может употреблять для изображения таких
определений свои прежние знаки; для обозначения еще весьма чуждого понятию
представления о {\em бесконечном приближении} двух ординат или для приравнения
дуги бесконечному числу бесконечно малых прямых линий она не может сделать
ничего другого, как начертить указанные две прямые линии {\em друг вне друга}
или вписать в дугу прямые линии, однако как {\em отличные} от нее. Для
постижения бесконечного, в котором здесь главная суть, она отсылает нас
к~{\em представлению}.

Что ближайшим образом соблазнило на указанную попытку, это
преимущественно то {\em количественное}
отношение, в котором якобы находятся друг к другу
{\em всеобщность, особенность} и {\em единичность}.
О~всеобщем говорят, что оно {\em шире}
особенного и единичного, а об особенном, что оно
{\em шире} единичного. Понятие есть {\em конкретное}
и {\em наибогатейшее}, так как оно есть основание и {\em тотальность}
предыдущих определений, т.~е. категорий бытия и определений
рефлексии; поэтому последние, разумеется, выступают также и в нем. Но его
природа понимается совершенно ложно, если их в нем удерживают еще в
указанной абстрактности, если <<{\em более широкий объем}>>
всеобщего понимается так, что оно, дескать, есть некоторое
большее {\em количество}, чем особенное и единичное. Как абсолютное
основание, понятие есть {\em возможность количества},
но равным образом и возможность {\em качества}, т.~е. его
определения различны также и качественно; поэтому они рассматриваются
противно их истине уже в том случае, если их полагают единственно под
формой количества. Подобным же образом, далее, определение рефлексии есть
некое {\em относительное},
в~котором светится его противоположность; оно не находится во
внешнем отношении, как какое-нибудь определенное количество. Но понятие
есть нечто б\'{о}льшее, чем все это; его определения суть определенные
{\em понятия} и сами существенным образом представляют собой {\em тотальность}
всех определений. Поэтому применение числовых и пространственных отношений,
в~которых все определения внеположны между собой, совершенно неподходяще для
формулирования такой внутренней тотальности; они, напротив, суть самое
последнее и самое худшее из всех средств, которые могли бы быть употреблены
для этого. Отношения природы, как, например, магнетизм, отношения цветов,
были бы для этого бесконечно более высокими и более истинными символами.
Так как человек обладает языком, как свойственным разуму средством
обозначения, то является праздной затеей причинять себе
хлопоты этими поисками менее совершенного способа изображения. Понятие как
таковое может по существу быть постигнуто лишь духом, которого оно является
не только достоянием, но и чистой самостью. Тщетно желание фиксировать его
посредством пространственных фигур и алгебраических знаков для того, чтобы
оно стало {\em внешне-зримым} и подходящим для {\em чуждой понятию
механической трактовки}, некоторым {\em счетом}. Также и все
другое, что якобы служит символом, способно самое большее, подобно
символам, обозначающим природу бога, возбуждать чаяния и отзвуки понятия;
но если серьезно стремятся выражать и познавать таким образом понятие, то
мы должны на это сказать, что {\em внешняя природа}
всякого символа неподходяща для этого, и отношение скорее
оказывается обратным: то, что в символе представляет собой отзвук
некоторого более высокого определения, может быть познано только через
понятие и приближено к нему единственно только путем {\em отметания} той
чувственной примеси, которая якобы должна была его выражать.

\section[С. Единичное]{С. Единичное}

{\em Единичность}, как оказалось, положена уже через особенность; последняя
есть {\em определенная всеобщность}, следовательно, соотносящаяся с собой
определенность, {\em определенное определенное}.

1. Поэтому единичность является прежде всего {\em рефлексией} понятия
{\em в~себя само} из своей определенности. Она есть {\em опосредствование}
понятия собой, поскольку его {\em инобытие} вновь сделало себя некоторым
{\em другим}, вследствие чего понятие восстановлено как равное себе самому,
но в определении {\em абсолютной отрицательности}. "--- То отрицательное во
всеобщем, вследствие которого последнее есть некоторое {\em особенное}, мы
определили выше\pagenote{См. выше, стр.~\pageref{bkm:bm23a}.\label{bkm:bm23}}
как двоякое свечение; поскольку оно есть свечение {\em во-внутрь},
особенное остается всеобщим, а через свечение во-вне оно есть
{\em определенное;} возвращение этой последней стороны во всеобщее двояко; это
{\em либо} возвращение через {\em абстракцию},
которая отбрасывает это определенное и восходит к {\em более высокому} и
{\em наивысшему роду, либо} через {\em единичность}, к
которой всеобщее нисходит в сам\'{о}й определенности. "--- Здесь
ответвляется та боковая дорожка, на которой абстракция сбивается с пути
понятия и покидает истину. Ее более высокое и наивысшее всеобщее, к
которому она восходит, есть лишь становящаяся все более и более
бессодержательной поверхность, а презрительно отвергаемая ею единичность
есть та глубина, в которой понятие постигает само себя и положено как понятие.

{\em Всеобщность} и {\em особенность} явились, с одной стороны, моментами
{\em становления} единичного. Но мы уже показали, что они в~себе самих суть
тотальное понятие и тем самым не переходят {\em в единичности} в
нечто {\em иное}, а в единичности лишь положено то, что они суть
в~себе и для себя. {\em Всеобщее есть для себя},
так как оно в~себе самом есть абсолютное опосредствование или
соотношение с собой лишь как абсолютная отрицательность. Оно есть
{\em абстрактное} всеобщее, поскольку это снятие есть {\em внешнее} действие
и вследствие этого {\em отбрасывание}
определенности. Указанная отрицательность поэтому имеется,
правда, в абстрактном, но она остается {\em вне} его, как то,
что есть лишь его {\em условие;}
она есть сама абстракция, держащая свое всеобщее {\em насупротив} себя,
вследствие чего это всеобщее не имеет единичности внутри самого себя и
остается чуждым понятию. "--- Жизни, духа, бога, равно как и
чистого понятия абстракция потому не может постигнуть, что она не
подпускает к своим продуктам единичность, принцип индивидуальности и
личности, и, таким образом, приходит лишь к безжизненным и бездуховным,
бесцветным и бессодержательным всеобщностям.

Но единство понятия так нераздельно, что и эти продукты абстракции,
опуская якобы единичность, сами, собственно говоря, {\em единичны}.
Абстракция возводит конкретное во всеобщность, всеобщее же
она понимает лишь как определенную всеобщность, а это как раз и есть
единичность, которая, как мы видели выше, есть соотносящаяся с собой
определенность. Абстракция есть поэтому {\em разделение}
конкретного и разрознивание (Verein\-zelung) его
определений; посредством абстракции мы схватываем лишь
{\em единичные} свойства
или моменты; ибо ее продукт должен содержать в~себе то, что она есть сама.
Но различие между этой единичностью ее продуктов и единичностью понятия
состоит в том, что в первых единичное, как
{\em содержание}, и всеобщее, как {\em форма},
отличаются друг от друга: содержание не выступает как
абсолютная форма, как само понятие или, иначе говоря, форма не выступает
как тотальность формы. "--- Но это более детальное рассмотрение
показывает нам само абстрактное как единство единичного содержания и
абстрактной всеобщности, стало быть, как
{\em конкретное}, как противоположность тому, чем оно хочет быть.

По тому же самому основанию {\em особенное}, так как
оно есть лишь определенное всеобщее, есть также и {\em единичное}, и,
наоборот, так как единичное есть определенное всеобщее, то оно есть также и
некоторое особенное. Если твердо держаться этой абстрактной определенности,
то мы должны будем сказать, что понятие имеет три особенных
определения "--- всеобщее, особенное и
единичное, между тем как ранее мы указали, как на виды особенного, лишь на
всеобщее и особенное. Так как единичность есть возвращение понятия, как
отрицательного, внутрь себя, то абстракция, которая, собственно говоря,
снята в этом возвращении, может ставить и перечислять самое это
возвращение, как безразличный момент, {\em рядом} с другими моментами.

Если единичность приводится как одно из {\em особенных}
определений понятия, то особенность есть та {\em тотальность},
которая объемлет собой все эти определения; как такая именно
тотальность, она есть их конкретное или сама единичность. Но особенность
есть конкретное также и с отмеченной выше стороны, т.~е. как
{\em определенная всеобщность;}
таким образом, особенность выступает как {\em непосредственное}
единство, в котором ни один из этих моментов не положен как
различенный или как определяющий, и в этой форме она будет составлять
{\em средний термин формального умозаключения}.

Само собой бросается в глаза, что каждое определение,
полученное в предыдущей экспозиции понятия, непосредственно растворялось и
терялось в своем ином. Всякое различение стирается и расплывается в том
самом рассуждении, которое имеет целью их изолировать и фиксировать. Только
голое {\em представление},
для которого их изолировал процесс абстракции, способно
прочно удержать для себя вне друг друга всеобщее, особенное и единичное;
взятые таким образом, они доступны перечислению, а что касается дальнейшего
различия, то представление держится за {\em совершенно внешнее}
различие бытия, за {\em количество}, которому менее всего здесь
место. "--- В~единичности указанное истинное отношение, т.~е.
{\em нераздельность} определений понятия, {\em положено;} ибо, как
отрицание отрицания, она содержит в~себе их противоположность и вместе с
тем эту последнюю в ее основании или единстве, т.~е. происшедшее слияние
каждого из этих определений со своим иным. Так как в этой рефлексии
имеется в~себе и для себя всеобщность, то она есть по существу
отрицательность определений понятия не только таким образом, что она по
отношению к ним есть как бы лишь некое отличное от них третье, а и в том
смысле, что отныне {\em положено}, что положенность есть
{\em в-себе-и-для-себя-бытие}, т.~е. что каждое из принадлежащих к различию
определений само есть {\em тотальность}. Возвращение определенного понятия
в~себя означает, что оно имеет такое определение, по которому оно в
{\em своей определенности} есть {\em все} понятие {\em целиком}.

2. Но единичность есть не только возвращение понятия в~себя
само, а непосредственно и его утрата. Через единичность, подобно тому как
понятие есть в ней {\em внутри себя}, оно
становится {\em вне себя} и вступает в действительность.
{\em Абстракция}, которая, как {\em душа}
единичности, есть соотношение отрицательного с отрицательным,
не есть, как оказалось, нечто внешнее всеобщему и особенному, а имманентна
им, и они благодаря ей суть конкретное, содержание, единичное. Но
единичность, как эта отрицательность, есть определенная определенность,
{\em различение} как таковое; через эту рефлексию различия в~себя различие
становится прочным; процесс определения особенного впервые получает место через
единичность, ибо она есть та абстракция, которая теперь, именно как
единичность, есть {\em положенная абстракция}.

Единичное, как соотносящаяся с собой отрицательность, есть, следовательно,
непосредственное тождество отрицательного с собой; оно есть
{\em для-себя-сущее}. Или, иначе говоря, оно есть абстракция, определяющая
понятие по его идеализованному моменту {\em бытия}, как нечто
{\em непосредственное}. "--- Таким образом, единичное есть некоторое
качественное {\em одно} или <<{\em это}>>. По этому качеству оно есть,
во-первых, отталкивание себя от {\em себя самого}, каковым отталкиванием
предполагаются многие {\em другие} одни; {\em во-вторых}, оно есть
отрицательное соотношение с~этими предположенными {\em иными}, и единичное есть
постольку {\em исключающее}. Всеобщность, соотнесенная с этими единичными, как
с~безразличными одними (а~она непременно должна быть соотнесена с ними, так как
она есть момент понятия единичности), есть лишь {\em общее} (das Gemein\-same)
им. Если под всеобщим понимают то, чт\'{о} {\em обще} многим единичным, то
исходят из их {\em безразличного} устойчивого наличия и примешивают
к~определению понятия непосредственность {\em бытия}. Низшим из всех возможных
представлений о~всеобщем в~его соотношении с~единичным является это
представление о~чисто внешнем отношении всеобщего как чего-то только
{\em общего} многим.

Единичное, которое в рефлексивной сфере существования
выступает как <<{\em это}>>, не имеет того {\em исключающего}
соотношения с другими одними, которое свойственно качественному для-себя-бытию.
<<{\em Это}>>, как {\em рефлектированное в~себя}
одно, само по себе не обладает отталкиванием; или, лучше
сказать, в этой рефлексии отталкивание едино
с~притяжением\pagenote{В~немецком издании 1816~г. стоит
<<Abtrac\-tion>> (sic!). В~изданиях 1834 и 1841~гг., а также у Лассона
напечатано <<Abstrak\-tion>>. Мы считаем более вероятным чтение
<<Attraktion>>.\label{bkm:bm24}} и есть рефлектирующее {\em опосредствование},
которое в нем (в~<<этом>>) таково, что <<это>> есть {\em положенная},
{\em непосредственность} проявляющаяся в чем-то внешнем. <<{\em Это}>>
{\em есть;} оно {\em непосредственно;} но оно есть <<{\em это}>> лишь
постольку, поскольку его показывают. Показывание есть то рефлектирующее
движение, которое собирает себя в~себя и полагает непосредственность, но
как нечто внешнее себе. "--- Единичное же,
правда, есть также и <<это>>, как восстановленное из опосредствования
непосредственное; но единичное имеет это опосредствование не вне себя, а
само оно есть отталкивающий процесс отделения, {\em положенная
абстракция}\pagenote{Латинское слово <<abstrac\-tio>>
означает <<отвлечение>> в смысле <<удаления, оттаскивания, отделения>>. В~этом
смысле Гегель и говорит здесь о <<единичном>> или <<отдельном>> (das Einzelne),
что оно представляет собой процесс самоотделения от других отдельных
предметов. Гегелевский термин <<das Einzelne>> вообще можно было бы всюду
переводить словом <<отдельное>> (как это в ряде мест делает В.~И.~Ленин), но
во многих случаях (особенно в главе об умозаключении, где Гегель применяет
буквенные схемы {\em Е "--- О "--- В} и~т.~д.) удобнее пользоваться
словом <<единичное>>. Поэтому мы и остановились на этом последнем слове
для передачи термина <<das Einzelne>>.\label{bkm:bm25}},
но в самом своем процессе отделения оно есть положительное соотношение.

Это абстрагирование единичного, как рефлексия различия в~себя, есть, во-первых,
полагание различенных как {\em самостоятельных}, рефлектированных в~себя. Они
{\em суть} непосредственно; но этот процесс отделения есть, далее, рефлексия
вообще, {\em свечение одного в~другом;} таким образом, они находятся в
существенном соотношении. Но они, далее, суть по отношению друг к другу не
только {\em сущие} единичные; такая множественность принадлежит области бытия;
{\em единичность}, полагающая себя в виде определенной единичности, полагает
себя не во внешнем, а в понятийном различии; она, следовательно, исключает из
себя всеобщее, но так как всеобщее есть момент ее самой, то оно столь же
существенным образом соотносится с нею.

Понятие, как это соотношение его {\em самостоятельных} определений, потеряло
себя; ибо в таком виде оно уже больше не есть их {\em положенное единство},
и~они выступают уже не как {\em моменты}, не как его {\em свечение} (Schein),
а~как сами по себе существующие. "--- Как единичность, понятие возвращается
в~определенности внутрь себя; тем самым определенное само стало тотальностью.
Возвращение понятия в~себя есть поэтому абсолютное, первоначальное
{\em деление его}, или, иначе говоря, в качестве единичности оно положено как
{\em суждение}\pagenote{Гегель хочет сказать, что
<<суждение>> (das Urteil) этимологически означает в~немецком языке
<<перводеление>> (Ur-teilen). В~действительности это не так. Слово <<Urteil>>
представляет собой существительное, соответствующее глаголу <<erteilen>>, и
первоначально означает, собственно говоря, <<das, was erteilt wird>>, т.~е.
<<то, что предоставляется, присуждается, постановляется>> (судьей,
начальником, {\em законодателем}). См. {\em Kluge}, Etymolo\-gisehes
Worter\-buch der deutschen Sprache, 9-te Aufl., Berlin und
Leipzig 1921, S.~470. Даваемое Гегелем произвольное толкование
значения слова <<Urteil>> (не находящее ни малейшего подтверждения в истории
немецкого языка) нужно Гегелю для того, чтобы облегчить себе переход от
<<понятия>> (в~узком смысле <<понятия как такового>>) к <<суждению>>,
которое он трактует как некоторое {\em объективное}
(объективное в смысле объективного и абсолютного идеализма)
отношение между единичным и всеобщим (или между особенным и всеобщим, или,
наконец, между единичным и особенным). Неправильное толкование этимологии
слова <<Urteil>> в смысле <<перводеления>> встречается также у Шеллинга в его
вышедшей в 1800~г. <<Системе трансцендентального идеализма>>
({\em Schelling}, Werke, hrsg. v.~O.~Weiss, Bd.~II, S.~181).\label{bkm:bm26}}.

\chapter[Вторая глава. Суждение]{Вторая глава\newline Суждение}

Суждение есть {\em положенная} в~самом {\em понятии определенность}
понятия. Определения понятия или (это, как оказалось, одно и
то же) определенные понятия уже были рассмотрены особо; но это рассмотрение
было больше некоторой субъективной рефлексией или субъективной абстракцией.
Но понятие само есть это абстрагирование; противопоставление его
определений друг другу есть его собственный процесс определения. {\em Суждение}
есть это полагание определенных понятий самим же понятием. Процесс суждения
есть {\em постольку} другая функция, чем постижение в понятии (или, вернее,
{\em другая} функция понятия), поскольку он есть {\em процесс
определения} понятия самим собой, и дальнейшее
поступательное движение суждения, переход к разным видам
суждения есть это дальнейшее определение понятия. Какие {\em имеются}
определенные понятия и каким образом эти его определения
получаются с необходимостью, "--- это должно обнаружиться в суждении.

Суждение может поэтому быть названо ближайшей {\em реализацией}
понятия, поскольку слово <<реальность>> вообще обозначает
вступление в {\em наличное бытие}, как в {\em определенное}
бытие. Точнее говоря, природа этой реализации оказалась
состоящей в том, что, {\em во-первых},
моменты понятия в силу его рефлексий в~себя или, иначе
говоря, его единичности суть самостоятельные тотальности, но,
{\em во-вторых}, единство понятия выступает как их {\em соотношение}.
Рефлектированные в~себя определения суть {\em определенные тотальности}
столь же существенно в безразличном, ни с чем другим не
соотносящемся устойчивом наличии, сколь и через взаимное опосредствование
друг другом. Сам процесс определения есть тотальность лишь постольку,
поскольку он содержит в~себе эти тотальности и их соотношение. Эта
тотальность и есть суждение. "--- Оно, следовательно, содержит
в~себе, во-первых, те два самостоятельных, которые носят название
{\em субъекта} и {\em предиката}. Чт\'{о}
собой представляет каждое из этих самостоятельных, "--- это
пока что невозможно сказать; они еще неопределенны, ибо только суждение
должно их определить. Так как суждение есть понятие как определенное, то
имеется лишь в общем виде то различие между ними, что суждение содержит
в~себе {\em определенное} понятие в отличие от понятия, все еще
{\em неопределенного}. Субъект в сопоставлении с предикатом можно,
следовательно, ближайшим образом понимать как единичное по отношению ко
всеобщему, или также как особенное по отношению к всеобщему, или как
единичное по отношению к особенному, поскольку они вообще противостоят
друг другу лишь как более определенное и более всеобщее.

Поэтому является соответственным и потребностью пользоваться
для обозначения определений суждения этими
{\em названиями} "--- <<{\em субъект}>> и <<{\em предикат}>>.
В качестве названий они представляют собой нечто
неопределенное, которому еще только предстоит получить свое определение, и
поэтому они суть не более, как названия. Сами определения понятия не могут
быть употреблены для обозначения этих двух сторон суждения отчасти по этой
причине, но отчасти и еще больше потому, что по своей природе определения
понятия явно не представляют собой чего-то абстрактного и неподвижного, а
имеют свое противоположное определение внутри себя и полагают его в~себе же
самих; так как стороны суждения сами суть понятия и,
следовательно, тотальность своих определений, то они должны пройти и
показать на себе самих (в~абстрактной ли или конкретной форме) все эти
определения. А~для того чтобы при таком изменении их определения можно было
все же фиксировать стороны суждения в общем виде, пригоднее всего названия,
остающиеся в этом изменении равными себе. "--- Но название
противостоит сути или понятию; это различение имеет место в самом суждении
как таковом. Так как субъект выражает собой вообще определенное и потому
преимущественно непосредственно {\em сущее}, а предикат
выражает собой {\em всеобщее},
сущность или понятие, то субъект как таковой есть вначале
лишь некоторого рода {\em название;} ибо {\em то, чт\'{о} он есть},
выражает лишь предикат, содержащий в~себе {\em бытие} в смысле
понятия. <<Что {\em есть} это>> или <<какое это {\em есть} растение?>>
и~т.~д. Под {\em бытием},
о котором здесь спрашивают, часто понимают лишь {\em название}, и, узнав
последнее, чувствуют себя удовлетворенными и уже знают, чт\'{о} такое
{\em есть} эта вещь. Это "--- {\em бытие} в смысле субъекта. Но
{\em понятие}\pagenote{В~немецком тексте всех изданий напечатано:
<<Aber {\em der} Begriff\ldots gibt erst das Pradikat>>. По-видимому, это
опечатка вместо <<Aber {\em den} Begriff\ldots>>.\label{bkm:bm27}},
или, по крайней мере, сущность и всеобщее вообще дается лишь
предикатом, и об этом мы и спрашиваем, если ставим вопрос в смысле
суждения. "--- {\em Бог, дух, природа}, или что бы
там ни было, взятые как субъект некоторого суждения, суть поэтому пока что
только названия; что {\em есть} такого рода субъект согласно понятию, "--- на
это отвечает лишь предикат. Если ищут, какой предикат присущ такому субъекту,
то в основании обсуждения этого вопроса должно было бы уже лежать некоторое
{\em понятие;} но это последнее впервые высказывается лишь
самим предикатом. Предполагаемое значение субъекта есть
поэтому, собственно говоря, лишь голое {\em представление},
и последнее приводит к номинальному объяснению, причем то, чт\'{о} разумеют
или не разумеют под тем или иным названием, является чем-то случайным и
представляет собой исторический факт. Столь многочисленные споры о том,
присущ ли данному субъекту или не присущ тот или иной предикат, являются
потому не более чем спорами о словах, что они исходят из указанной формы;
лежащее в основании (subjectum, \textgreek{ὑποκείμενον}) есть пока
что не более чем название.

Теперь нам нужно рассмотреть ближе, как, {\em во-вторых},
определено соотношение субъекта и предиката в суждении и как
именно в силу этого определены ближайшим образом и они сами. Суждение имеет
вообще своими сторонами тотальности, которые вначале выступают как
существенным образом самостоятельные. Единство понятия есть поэтому пока
что лишь некоторое {\em соотношение} самостоятельных, есть не
{\em конкретное}, возвратившееся из этой реальности в~себя, {\em наполненное}
единство, а такое единство, вне которого они пребывают как {\em не снятые
в нем крайние термины}. "--- Рассмотрение суждения может иметь своим
исходным пунктом либо первоначальное единство понятия, либо
самостоятельность крайних терминов. Суждение есть расщепление понятия самим
собой; {\em это единство} есть поэтому то основание, исходя из которого мы
рассматриваем суждение согласно его истинной {\em объективности}.
Суждение есть постольку {\em первоначальное разделение}
(Teilung) первоначально единого. Слово <<Urteil>> (суждение)
указывает, следовательно, на то, что суждение есть в~себе и для себя. Но
что понятие выступает в суждении как явление, поскольку его моменты
достигли в суждении самостоятельности, "--- за эту {\em внешнюю} сторону
больше держится {\em представление}.

Согласно этому {\em субъективному}
способу рассмотрения субъект и предикат рассматриваются
каждый как нечто, находящееся вне другого и само по себе готовое: субъект
"--- как предмет, который существовал бы также и в том случае,
если бы он не обладал данным предикатом, и предикат "--- как
некоторое всеобщее определение, которое существовало бы и в том случае,
если бы оно не было присуще этому субъекту. С~актом суждения связано
согласно этому размышление о том, можно ли и должно ли
{\em приписывать} тот или иной предикат, находящийся в
{\em голове}, предмету, который имеет бытие {\em вне}
ее, сам по себе; сам акт суждения состоит в том, что лишь им
некоторый предикат {\em приводится}
в связь с субъектом, так что, если бы эта связь не получила
места, то как субъект, так и предикат дались бы, каждый сам по себе, тем,
что они суть: первый "--- существующим предметом, а второй
"--- представлением в голове. Но предикат, приписываемый
субъекту, должен быть также и {\em присущ} ему, т.~е.
должен быть сам по себе тождественен с ним. Этим значением
<<{\em приписывания}>> {\em субъективный} смысл
акта суждения и безразличное внешнее пребывание субъекта и предиката снова
упраздняются; <<это действие {\em есть} хорошее>>; связка <<есть>> указывает
на то, что предикат принадлежит к~{\em бытию} субъекта, а не~приводится лишь
во внешнюю связь с~ним. В~{\em грамматическом} смысле указанное субъективное
отношение, при котором исходным пунктом служит безразличная, внешняя связь
субъекта и предиката, имеет полную силу; ибо здесь внешним образом приводятся
в~связь не~что иное, как {\em слова}. "--- По этому поводу можно также
заметить, что хотя {\em предложение} и имеет субъект и предикат
в~грамматическом смысле, это еще не значит, что оно обязательно есть
{\em суждение}. Для суждения требуется, чтобы предикат относился к субъекту
по типу отношения определений понятия, следовательно, как некоторое
всеобщее к некоторому особенному или единичному. Если то,
что высказывается о единичном субъекте, само есть лишь нечто единичное, то
это "--- простое предложение. Например, <<Аристотель умер на 73-м году своей
жизни\pagenote{Это "--- фактическая ошибка, так как Аристотель умер на 63 году
своей жизни (384---322 до н.~э.).\label{bkm:bm28}}, в~4-м году 115-й
Олимпиады>> "--- есть простое предложение, а не суждение. В~нем было бы нечто
от суждения только в том случае, если бы одно из обстоятельств "--- время ли
смерти или возраст этого философа "--- подвергалось сомнению, но по какому-либо
основанию отстаивались бы приведенные цифры. Ибо в таком случае их брали бы как
нечто всеобщее, как существующее и без сказанного определенного содержания "---
смерти Аристотеля, наполненное другим содержанием или же пустое время. Подобным
же образом известие "--- <<мой друг N умер>> есть предложение; оно было бы
суждением лишь в том случае, если бы вопрос шел о том, действительно ли он умер
или же здесь имеется лишь кажущаяся смерть.

Если суждение обычно объясняется так, что оно есть, дескать,
{\em соединение двух понятий}, то для внешней связки можно, пожалуй, сохранить
неопределенное выражение <<{\em соединение}>> и признать, далее, что
соединяемые члены по крайней мере {\em должны} быть понятиями. Но вообще это
объяснение в высшей степени поверхностно и дело не только в том, что, например,
в разделительном суждении соединено более {\em двух} так называемых
понятий, а больше в том, что объяснение значительно лучше, чем то, что
служит здесь предметом объяснения; ибо то, что здесь имеется в виду, не
есть вообще понятия и едва ли даже определения понятия, а в сущности говоря,
лишь {\em определения представления}. При рассмотрении понятия вообще и
определенного понятия мы уже заметили, что то, чему обычно дается это
название, никоим образом не заслуживает названия понятия; а если так, то
откуда же в суждении могут взяться понятия? "--- Главным же
образом, сказанное объяснение поверхностно потому, что оно упускает из виду
существенную сторону суждения, а именно, различие его определений, и еще
более оно упускает из виду отношение суждения к понятию.

Что касается дальнейшего определения субъекта и предиката, то
уже было указано, что они, собственно говоря, должны получить свое
определение именно лишь в суждении. Но поскольку суждение есть положенная
определенность понятия, ей присущи указанные различия
{\em непосредственно} и {\em абстрактно}, как {\em единичность}
и {\em всеобщность}. "--- Поскольку же суждение есть вообще
{\em наличное бытие} или {\em инобытие} понятия, еще не восстановившего себя
снова, не возвратившегося к тому единству, в~силу которого оно имеет бытие
{\em как понятие}, то здесь выступает также {\em и та} определенность, которая
чужда понятию, "--- противоположность {\em бытия} и рефлексии или
{\em в-себе-бытия}. Но так как понятие составляет существенное {\em основание}
суждения, то указанные определения по крайней мере столь безразличны, что
в~каждом "--- одном, присущем субъекту, и другом, присущем предикату "---
имеет место также и обратное отношение. {\em Субъект}, как {\em единичное},
выступает ближайшим образом как {\em сущее} или {\em для-себя-сущее}
согласно определенной определенности единичного, "---
как некоторый действительный предмет, хотя бы он и был лишь
предметом представления, "--- как, например, храбрость, право,
соответствие и~т.~п., "--- предмет, о котором судят; напротив,
{\em предикат}, как {\em всеобщее}, выступает как эта {\em рефлексия}
о~предмете суждения или же, вернее, как его рефлексия в~себя самого,
выходящая за пределы указанной непосредственности и снимающая определенности
как всего лишь сущие, "--- предикат выступает {\em как его в-себе-бытие}. "---
Постольку в суждении исходят из единичного, как первого, непосредственного, и
{\em возводят} его через суждение {\em во всеобщность}, равно как и обратно
"--- всеобщее, которое есть лишь {\em в~себе}, нисходит в единичном до
наличного бытия или становится некоторым {\em для-себя-сущим}.

\label{bkm:bm31a}Это значение суждения следует брать как его {\em объективный}
смысл и вместе с тем как {\em истину} встречавшихся у нас ранее форм перехода.
Сущее {\em становится} и {\em изменяется}, конечное {\em тонет} в бесконечном;
существующее {\em выходит} из своего {\em основания, вступает} в явление
и {\em идет ко дну, погружается в основание;} акциденция {\em выявляет
богатство} субстанции, равно как и ее мощь; в бытии {\em необходимое}
соотношение обнаруживает себя через {\em переход} в иное, в сущности "--- через
свечение в чём-то ином. Эти переход и свечение теперь перешли
в {\em первоначальное разделение понятия}, которое (понятие), возвращая
единичное во {\em в-себе-бытие} его всеобщности, вместе с тем определяет
всеобщее как {\em действительное}. Эти два процесса, "--- то, что единичность
полагается в ее рефлексию в~себя, а всеобщее полагается как
определенное, "--- суть одно и то же.

Но это объективное значение подразумевает также и то, что
указанные различия, выступая теперь вновь в определенности понятия, вместе
с тем положены лишь как являющиеся, т.~е. что они не суть нечто
неподвижное, а приложимы как к одному определению понятия, так и к другому.
Поэтому следует брать субъект также и как {\em в-себе-бытие},
а~предикат, напротив, также и как {\em наличное бытие}.
{\em Субъект без предиката} представляет собой то же самое, что в явлении
{\em вещь без свойств, вещь-в-себе}, "--- пустое неопределенное
основание; он, таким образом, есть {\em понятие внутри себя самого},
получающее различение и определенность лишь в предикате;
последний, стало быть, составляет сторону {\em наличного бытия}
субъекта. В~силу этой определенной всеобщности субъект
находится в соотношении с внешним, открыт для влияния других вещей и
вступает в действие по отношению к ним. {\em То, что налично}, выходит
из своего {\em внутри-себя-бытия}, вступая во {\em всеобщую}
стихию связи и отношений, в отрицательные соотношения и
взаимовлияние действительности, а это есть {\em продолжение}
единичного в другие единичные и потому всеобщность.

Только что указанное тождество, состоящее в том, что определение субъекта
в~одинаковой мере присуще также и предикату, и обратно, имеет место, однако,
не только в нашем размышлении; оно не только имеется {\em в~себе}, но
также и положено в суждении; ибо суждение есть соотношение обоих; связка
выражает собой, что {\em субъект} есть {\em предикат}.
Субъект есть определенная определенность, а предикат есть эта его
{\em положенная} определенность; субъект определен только в своем предикате,
или, иначе сказать, только в нем он есть субъект, в предикате он возвращен
в~себя и есть в нем всеобщее. "--- Но поскольку субъект есть
самостоятельное, указанное тождество характеризуется тем, что предикат не
имеет сам по себе самостоятельного устойчивого наличия, а имеет свое
устойчивое наличие лишь в субъекте; он {\em принадлежит} последнему. Согласно
этому, поскольку отличают предикат от субъекта, первый есть лишь некоторая
{\em отдельная} (verein\-zelte) определенность последнего, лишь {\em одно} из
его свойств; сам же субъект есть {\em конкретное}, тотальность многообразных
определенностей, из которых предикат содержит в~себе лишь одну; субъект есть
всеобщее. "--- Но, с другой стороны, и предикат есть самостоятельная
всеобщность, а субъект, наоборот, есть лишь одно из его определений. Предикат,
стало быть, {\em подводит под себя} субъект; единичность и особенность не
обладают самодовлеющим бытием, а имеют свою сущность и свою субстанцию во
всеобщем. Предикат выражает субъект в его понятии; единичное и особенное суть
случайные определения в субъекте; предикат есть их абсолютная возможность.
Если при {\em процессе подведения} [под более общее] думают о некотором внешнем
соотношении субъекта и предиката и представляют себе субъект как нечто
самостоятельное, то в таком случае процесс подведения относится к
вышеупомянутому субъективному акту суждения, в котором исходят из
самостоятельности их {\em обоих}. В~этом случае подведение оказывается лишь
{\em применением} всеобщего к некоторому особенному или единичному, которое
ставится {\em под} всеобщим на основании некоторого неопределенного
представления, как нечто, имеющее меньшее количество [меньший объем].

Если в предыдущем изложении тождество субъекта и предиката рассматривалось так,
что {\em иногда} первому присуще одно определение понятия,
а второму "--- другое, {\em иногда же} "--- наоборот, то это тождество тем
самым все еще остается лишь чем-то {\em в-себе-сущим;} ввиду самостоятельной
разности этих двух сторон суждения, их {\em положенное} соотношение также имеет
указанные две стороны, имеет их ближайшим образом как разные. Но {\em истинное}
соотношение субъекта с предикатом образуется, собственно говоря,
{\em таким тождеством, которое свободно от различия}. Определение понятия само
есть по существу {\em соотношение}, ибо оно есть некоторое {\em всеобщее;}
следовательно, теми же самыми определениями, которыми обладают субъект и
предикат, обладает также и само соотношение между ними. Оно {\em всеобще},
так как оно есть положительное тождество обоих, субъекта и предиката; но оно
есть также и {\em особенное}\pagenote{В~немецком тексте всех изданий
напечатано: <<bestimmte>>. Повидимому, это опечатка вместо:
<<besondere>>.\label{bkm:bm29}},
так как определенность предиката есть определенность субъекта; оно, далее, есть
также и {\em единичное}, ибо самостоятельные крайние термины сняты в нем, как
в своем отрицательнее единстве. "--- Но в суждении это тождество еще не
положено; связка выступает как еще неопределенное соотношение {\em бытия}
вообще: $A$ {\em есть} $B$; ибо самостоятельность определенностей понятия или
крайних терминов "--- вот в суждении та {\em реальность}, которой в нем
обладает понятие. Если бы связка <<{\em есть}>> была уже {\em положена} как
указанное определенное и наполненное {\em единство} субъекта и предиката, как
его {\em понятие}, то суждение было бы уже умозаключением.

Восстановить или, вернее, {\em положить} это {\em тождество} понятия есть цель
{\em движения} суждения. Что уже {\em имеется налицо} в суждении,
это "--- отчасти самостоятельность, но вместе с тем и
определенность субъекта и предиката по отношению друг к другу, отчасти же
их, тем не менее, {\em абстрактное} соотношение.
{\em Субъект есть предикат}, "--- вот что ближайшим образом высказывается в
суждении; но так как предикат {\em не}~должен быть тем, чт\'{о}
представляет собой субъект, то получается {\em противоречие}, которое должно
быть {\em разрешено}, должно {\em перейти} в некоторый результат. Но вернее
будет сказать, что так как {\em в~себе и для себя} субъект и предикат
составляют тотальность понятия, а суждение есть реальность понятия, то
поступательное движение суждения есть лишь {\em развитие;} в~нем уже
имеется то, что в нем обнаруживается, и {\em доказательство} есть
постольку лишь {\em показывание}, рефлексия как полагание того, что в крайних
терминах суждения уже {\em имеется налицо;} но и самое это полагание уже
имеется налицо; оно есть {\em соотношение} крайних терминов.

Суждение, каково оно {\em непосредственно}, есть {\em ближайшим образом}
суждение {\em наличного бытия;} его субъект есть непосредственно некоторое
{\em абстрактное, сущее единичное}, а предикат "--- некоторая
{\em непосредственная определенность} или свойство субъекта, нечто
абстрактно всеобщее.

Так как это качественное в субъекте и предикате снимает себя,
то определение одного ближайшим образом {\em светится} в другом;
таким образом, суждение есть, {\em во-вторых}, суждение {\em рефлексии}.

Но это скорее внешнее синтезирование\pagenote{В~изданиях 1816 и 1834~гг.:
<<Zusammen\-fassen>>. В~издании 1841~г., а~также и у Лассона (у~последнего без
указания разночтений): <<Zusaram\-entreffen>>. Мы переводим согласно тексту
изданий 1816 и 1834~гг.\label{bkm:bm30}}
переходит в {\em существенное тождество} некоторой субстанциальной,
{\em необходимой связи;} таким образом, суждение есть, {\em в-третьих},
суждение {\em необходимости}.

{\em В-четвертых}, так как в этом существенном тождестве различие субъекта и
предиката стало {\em формой}, то суждение становится {\em субъективным;} оно
содержит в~себе противоположность {\em понятия} и его {\em реальности} и их
{\em сравнение;} это "--- {\em суждение понятия}\pagenote{В~<<Диалектике
природы>> Энгельс всю главу <<Большой логики>> о суждении,
занимающую в издании 1841~г.
(которым пользовался Энгельс) страницы 63---115 (в~настоящем русском издании
стр.~\pageref{bkm:bm31a} "--- \pageref{bkm:bm31b}), называет
<<гениальной>>, отмечая <<внутреннюю истину и необходимость>> даваемой Гегелем
классификации суждений, несмотря на всю ее <<сухость>> и частичную
<<произвольность на первый взгляд>>. И~Энгельс дает в связи с этой
классификацией блестящий образчик того, {\em как}
надо <<переворачивать>> Гегеля с головы на ноги, извлекая
гениальные зерна истины из насквозь идеалистических, абстрактных и темных
рассуждений Гегеля, не свободных к тому же от того, что Ленин называл
<<данью старой, формальной логике>> ({\em Ленин},
Философские тетради, стр.~171). Дело в том, что Гегель в
главе о Суждении ухитрился втиснуть в свои четыре основных вида суждений
традиционную формально-логическую классификацию суждений по количеству,
качеству, отношению и модальности, "--- ближайшим образом в том
ее виде, в каком она дается у Канта (см. {\em Кант},
Критика чистого разума, пер. Лосского, Пгр. 1915, стр.~70---73;
ср. {\em Кант}, Логика, пер. Маркова, Пгр. 1915, стр.~93---101). Разбирая
гегелевскую классификацию суждений, Энгельс показывает (на конкретном
примере исторического развитая человеческих суждений о превращении одних
форм движения в другие), что <<то, что у Гегеля является развитием
мыслительной формы суждения как такового, выступает здесь перед нами как
развитие наших, покоящихся на {\em эмпирической} основе,
теоретических знаний о природе движения вообще>>
({\em Engels}, Anti-Duhring und Dialektik der Natur, M.--L. 1935, S.~663).
При этом Энгельс сводит четыре основных группы суждений
гегелевской классификации к {\em трем}
основным видам, объединяя гегелевские суждения рефлексии и
суждения необходимости в один вид суждений особенности (или частности).
Соответственно с этим гегелевские суждения наличного бытия характеризуются
у Энгельса как суждения единичности (или отдельности), а гегелевские
суждения понятия "--- как суждения всеобщности. Относительно
расшифровки одного места из этого замечательного отрывка <<Диалектики
природы>> см. рецензию В. Брушлинского на новое немецкое издание
<<Анти-Дюринга>> и <<Диалектики природы>> в журнале <<Под знаменем марксизма>>,
1937, №~7, стр.~171---172.\label{bkm:bm31}}.

Это выступление понятия обосновывает {\em переход суждения в~умозаключение}.

\section[А. Суждение наличного бытия]{А. Суждение наличного бытия}

В~субъективном суждении стремятся видеть {\em один и тот же} предмет
{\em двояким} образом: во-первых, в его единичной действительности, и,
во-вторых, в~его существенном тождестве или в его понятии; это
"--- единичное, возведенное в свою всеобщность, или, что то же
самое, всеобщее, сделанное единичным, перешедшее в свою действительность.
Суждение есть, таким образом, {\em истина}, ибо оно есть согласие понятия и
реальности. Но не таков характер суждения с самого {\em начала;} ибо
{\em сначала} оно {\em непосредственно}, поскольку в нем еще не получилось
рефлексии и движения определений. Эта {\em непосредственность} делает первое
суждение {\em суждением наличного бытия}, которое можно также назвать {\em
качественным суждением}, "--- однако лишь постольку, поскольку {\em качество}
не только принадлежит к определенности {\em бытия}, но также включает в~себя и
абстрактную всеобщность, которая равным образом имеет в силу своей простоты
форму {\em непосредственности}.

Суждение наличного бытия есть также и суждение {\em присущности} (der
Inhärenz); так как непосредственность есть его определение, а в различии
субъекта и предиката субъект есть непосредственное и вследствие этого первое и
существенное в этом суждении, то предикат имеет форму чего-то
несамостоятельного, имеющего свою основу в субъекте,

\subsection[а) Положительное суждение ]{а) Положительное суждение}

1. Субъект и предикат, как было указано, суть ближайшим образом названия,
действительное определение которых получается лишь в ходе суждения. Но как
стороны суждения, которое есть {\em положенное} определенное понятие, они
обладают определениями моментов последнего; однако в силу непосредственности
эти определения еще совсем {\em просты}, отчасти не обогащены
опосредствованием, отчасти же носят ближайшим образом характер абстрактной
противоположности, представляя собой лишь {\em абстрактную единичность} и
{\em всеобщность}. "--- Предикат (начнем с него) есть {\em абстрактное}
всеобщее; но так как абстрактное обусловлено опосредствованием снятия
единичного или особенного, то это опосредствование есть постольку лишь
некоторое {\em пред-положение}. В~сфере понятия не может быть иной
{\em непосредственности}, кроме такой, которая {\em в~себе и для себя} содержит
опосредствование и возникла лишь через его снятие, т.~е. кроме {\em всеобщей}
непосредственности. Таким образом, и само {\em качественное бытие} есть {\em в
своем понятии} некоторое всеобщее; но как {\em бытие} непосредственность еще не
{\em положена} так; лишь как {\em всеобщность} она есть такое определение
понятия, в котором {\em положено}, что ему по существу присуща отрицательность.
Это соотношение имеется в суждении, где оно есть предикат некоторого субъекта.
"--- И точно так же субъект есть некоторое {\em абстрактно}-единичное или,
иначе говоря, такое {\em непосредственное}, которое должно иметь бытие именно
как {\em непосредственное;} поэтому субъект должен представлять собой единичное
как некоторое {\em нечто} вообще. Постольку субъект составляет в суждении тот
абстрактный аспект, по которому понятие перешло в нем во {\em внешность}. "---
Точно так же, как определены оба определения понятия, определено и их
соотношение, <<{\em есть}>>, связка; оно равным образом может иметь значение
лишь непосредственного, абстрактного {\em бытия}. От этого соотношения, не
содержащего еще в~себе никакого опосредствования или отрицания, это суждение
получает название {\em положительного}.

2. Ближайшим чистым выражением положительного суждения служит
поэтому предложение:

<<{\em единичное есть всеобщее}>>.

Это выражение не должно быть облечено в форму <<$A$ есть $B$>>; ибо $A$ и $B$
суть совершенно бесформенные и поэтому лишенные значения названия; суждение же
вообще, и потому уже даже суждение наличного бытия, имеет своими крайними
терминами определения понятия. <<$A$ есть $B$>> может в той же мере
представлять собой любое простое {\em предложение}, как и какое-либо
{\em суждение}. Но в каждом, даже более богато определенном по своей форме
суждении утверждается предложение, имеющее следующее определенное содержание:
<<{\em единичное} есть {\em всеобщее}>>, а именно, поскольку всякое суждение
есть также и абстрактное суждение вообще. Об отрицательном суждении и о том,
в~какой мере оно тоже может быть высказано посредством этого выражения, будет
сейчас идти речь. "--- Если же обыкновенно не думают о том, что каждым, по
крайней мере, положительным суждением (будем пока говорить только о нем)
высказывается утверждение, что единичное есть всеобщее, то это происходит
потому, что отчасти упускается из вида та {\em определенная форма}, которой
субъект отличается от предиката, так как полагают, что суждение есть
просто-напросто соотношение {\em двух} понятий, отчасти же, может быть, и
оттого, что сознанию предносится прочее {\em содержание} суждения: <<Кай учен>>
или <<роза красна>>, и сознание, занятое представлением о {\em Кае} и~т.~п.,
не~размышляет о форме, хотя такое, по крайней мере, содержание, как {\em
логический Кай}, который обычно должен служить примером, представляет собой
весьма мало интересное содержание, и это неинтересное содержание скорее как раз
для того и выбрано, чтобы оно не отвлекало внимания от формы\pagenote{Ср.
приведенное в примечании \ref{bkm:bm11} замечание Маркса о <<формальном
содержании>> суждений и умозаключений.\label{bkm:bm32}}.

По объективному своему значению предложение: <<{\em единичное есть всеобщее}>>,
как мы при случае упомянули выше, выражает отчасти преходящий характер
единичных вещей, отчасти же их положительное наличие в понятии вообще. Само
понятие бессмертно, но то, что выступает из него при его разделении, подвержено
изменению и возвращению в его {\em всеобщую} природу. Но и обратно, всеобщее
сообщает себе некоторое {\em наличное бытие}. Подобно тому как сущность
переходит в своих определениях в {\em видимость}, основание "---
в {\em явление} существования, субстанция "--- в обнаружение, в свои
акциденции, так всеобщее {\em раскрывается}, чтобы стать единичным; суждение
есть это его {\em раскрытие, развитие} той отрицательности, которой оно
уже было в~себе. "--- Последний процесс находит себе выражение в обратном
предложении: <<{\em всеобщее единично}>>, "--- предложении, которое равным
образом высказывается в положительном суждении. Субъект, который ближайшим
образом есть {\em непосредственно единичное}, соотнесен в самом суждении со
своим {\em иным}, а именно, со всеобщим; он, стало быть, положен как
{\em конкретное}, а по своему бытию "--- как некоторое нечто со
{\em многими качествами;} или "--- как конкретное рефлексии, как {\em некоторая
вещь с многообразными свойствами}, как некоторое {\em действительное}
с многообразными {\em возможностями}, как {\em субстанция} с многообразными
{\em акциденциями}. Так как это многообразие принадлежит здесь субъекту
суждения, то нечто или вещь и~т.~п. в своих качествах, свойствах или
акциденциях рефлектировано в~себя, или, иначе говоря, проходя сквозь них,
{\em непрерывно продолжается} в них, сохраняет себя в них, а также и их в~себе.
Положенность или определенность принадлежит к в-себе-и-для-себя-бытию. Поэтому
субъект в~себе самом есть {\em всеобщее}. "--- Напротив, предикат, как эта не
реальная или конкретная, а {\em абстрактная всеобщность}, есть по отношению к
субъекту {\em определенность} и содержит в~себе лишь {\em один момент} его
тотальности, с исключением других. В~силу этой отрицательности, которая вместе
с тем, как крайний термин суждения, соотносится с собой, предикат есть
некоторое {\em абстрактно-единичное}. "--- Например, в предложении: <<роза
благоуханна>> предикат выражает лишь {\em одно} из {\em многих} свойств розы;
он изолирует это свойство, которое в субъекте сращено с и, делает его
единичным, подобно тому как при разложении вещи многообразные присущие ей
свойства {\em изолируются}, становясь отдельными самостоятельными
{\em материями}\pagenote{См. т.~I <<Науки логики>>,
стр.~394---399.\label{bkm:bm33}}. Поэтому предложение суждения (der Satz des
Urteils) гласит с этой своей стороны так: <<{\em всеобщее единично}>>.

Сопоставив это {\em взаимное определение} субъекта и предиката в суждении, мы
получим, стало быть, двоякий результат: (1) субъект есть, правда,
непосредственно сущее или единичное, предикат же "--- всеобщее. Но так как
суждение есть их {\em соотношение} и так как субъект определен предикатом как
всеобщее, то субъект есть всеобщее; (2) предикат определен в субъекте; ибо
предикат есть не некоторое определение {\em вообще}, а определение
{\em субъекта}: <<роза благоуханна>>, "--- это благоухание есть не какое-то
неопределенное благоухание, а благоухание розы; предикат, стало быть, есть
{\em нечто единичное}. "--- А так как субъект и предикат находятся между собой
в~отношении суждения, то они должны по своему определению понятия оставаться
противоположными, подобно тому как во {\em взаимодействии}
причинности прежде, чем оно достигнет своей истины, обе
стороны должны, несмотря на равенство своего определения, все еще
оставаться самостоятельными и противоположными. Поэтому если субъект
определен как всеобщее, то предикат нельзя брать в его определении
всеобщности (ибо в таком случае не было бы никакого
суждения), а он должен быть взят лишь в его определении
единичности; и равным образом, поскольку субъект определен как единичное,
следует брать предикат как всеобщее. "--- Когда мы обращаем
внимание лишь на их голое тождество, то мы получаем следующие два
тождественных предложения:\label{bkm:bm35a}
<<единичное есть единичное>>, <<всеобщее есть всеобщее>>, "--- предложения,
в которых определения суждения совершенно выпадали бы друг из друга и
выражалось бы лишь их соотношение с собой, а их соотношение друг с другом
уничтожалось бы и тем самым упразднялось бы суждение. "--- Из
вышеуказанных двух предложений одно: <<{\em всеобщее единично}>> "---
выражает суждение по его {\em содержанию}, которое
в предикате есть отдельное изолированное определение, а в субъекте
тотальность определений; другое же предложение:
<<{\em единичное всеобще}>> "--- выражает {\em форму},
которая самим им непосредственно указана. "---
В непосредственном положительном суждении крайние термины
еще просты; поэтому форма и содержание еще соединены. Или, иначе говоря,
оно не состоит из двух предложений; получающееся в нем двоякое соотношение
образует собой непосредственно {\em одно} положительное
суждение. Ибо его крайние термины (а) выступают как самостоятельные,
абстрактные определения суждения и (b) каждая из сторон определяется другой
через посредство соотносящей их связки. {\em В~себе} же, как
выяснилось, различие между формой и содержанием в нем именно в силу этого
имеется; и притом то, что содержится в первом предложении (<<единичное
всеобще>>), принадлежит к форме, так как это предложение выражает собой
{\em непосредственную определенность} суждения. Напротив, отношение,
выражаемое вторым предложением (<<всеобщее единично>>), или, иначе говоря,
утверждение, что субъект определен как всеобщее, а предикат, напротив, как
особенное или единичное, касается {\em содержания}, так как его определения
возникают лишь через рефлексию в~себя, вследствие чего непосредственные
определенности снимаются, и тем самым форма превращает себя в некоторое ушедшее
внутрь себя тождество, устойчиво имеющееся как противостоящее различию формы,
т.~е. форма превращает себя в содержание.

3. Если бы оба предложения "--- формы и содержания:\label{bkm:bm34a}

\begin{center}
\begin{tabular}{l~c l}
(субъект) & \  & (предикат) \\
<<единичное & есть & всеобщее>> и \\
<<всеобщее & есть & единичное>>, "---
\end{tabular}
\end{center}

\noindent ввиду того, что они содержатся в {\em одном} положительном
суждении, были бы соединены так, что тем самым оба, и субъект и предикат,
оказались бы определенными как единство единичности и всеобщности, то оба
они были бы {\em особенными}, что {\em в~себе}
должно быть признано их внутренним определением. Однако
отчасти это соединение было бы осуществлено лишь через некоторую внешнюю
рефлексию, отчасти же вытекавшее бы отсюда предложение: <<особенное есть
особенное>> было бы уже не суждением, а пустым тождественным предложением,
подобно тем предложениям, которые уже встретились нам выше: <<единичное есть
единичное>> и: <<всеобщее есть всеобщее>>. "--- Единичность и
всеобщность не могут еще быть соединены в особенность, так как они в
положительном суждении еще положены как {\em непосредственные}. "---
Или, иначе говоря, суждение еще должно быть различаемо по
своей форме и по своему содержанию, потому что как раз субъект и предикат
еще различены как непосредственное и опосредствованное, или, опять-таки
иначе, потому что суждение по своему соотношению есть и то и другое: и
самостоятельность соотносящихся и их взаимное определение или опосредствование.

Итак, суждение, рассматриваемое, {\em во-первых}, по своей {\em форме}, гласит:

<<{\em единичное есть всеобщее}>>.

Но вернее будет сказать, что такое {\em непосредственное} единичное
{\em не}~всеобще: его предикат имеет более широкий объем, и,
следовательно, оно ему не соответствует. Субъект есть нечто
{\em непосредственно, само по себе сущее}, и потому представляет собой
{\em противоположность} этой абстракции, этой положенной через опосредствование
всеобщности, которая должна была быть высказана о нем в суждении.

{\em Во-вторых}, суждение рассматривается по его {\em содержанию} или как
предложение: <<{\em всеобщее единично}>>; в этом случае субъект есть нечто
всеобщее с многообразными качествами, некое конкретное, которое бесконечно
определено; а так как его определенности суть пока что лишь качества, свойства
или акциденции, то его тотальность есть {\em дурно-бесконечное множество} их.
Поэтому такой субъект скорее не есть такого рода {\em единичное} свойство,
какое высказывается его предикатом. Оба предложения должны поэтому
{\em подвергнуться отрицанию}, и положительное суждение должно скорее быть
положено как {\em отрицательное}.

\subsection[b) Отрицательное суждение]{b) Отрицательное суждение}

1. Уже выше была речь о том обычном представлении, согласно
которому лишь от содержания суждения зависит, истинно ли оно или нет, так
как логическая истина касается, дескать, только формы и не требует ничего
другого, кроме того, чтобы это содержание не противоречило самому себе.
Сама же форма суждения согласно этому взгляду состоит лишь в том, что оно
есть соотношение {\em двух} понятий. Но в предыдущем изложении выяснилось, что
эти два понятия не только обладают безотносительным определением некоторого
{\em числа}, а относятся друг к другу как {\em единичное} и {\em всеобщее}.
Эти определения образуют собой истинно логическое {\em содержание}, и притом
они, взятые так абстрактно, составляют содержание положительного суждения;
всякое {\em другое содержание}, встречающееся в суждении (<<солнце кругло>>,
<<Цицерон был великий римский оратор>>, <<теперь день>> и~т.~п.), не касается
суждения как такового; суждение высказывает лишь одно: <<{\em субъект} есть
{\em предикат}>>, или, так как <<субъект>> и <<предикат>> суть лишь названия,
то определеннее: <<{\em единичное есть всеобщее}>>, {\em и наоборот}. "---
Из-за этого {\em чисто логического содержания} положительное суждение
{\em не~истинно}, а имеет свою истину в отрицательном суждении. "--- Ведь
требуют, чтобы содержание суждения не противоречило себе; между тем оказалось,
что оно противоречит себе в этом суждении. "--- Однако будет совершенно
все равно, если мы указанное логическое содержание назовем также и формой,
а под содержанием будем понимать только прочее, эмпирическое наполнение
суждения; в таком случае форма содержит в~себе не только пустое тождество,
вне которого лежало бы определение содержания. Тогда оказывается, что
положительное суждение в силу своей {\em формы} не имеет
истины как положительное суждение; тому, кто называл бы
{\em истиной правильность} некоторого {\em созерцания} или
{\em восприятия}, согласие {\em представления} с~предметом,
по меньшей мере не оставалось бы никакого выражения для
обозначения того, что есть предмет и цель философии. Ему пришлось бы по
меньшей мере назвать последнее <<разумной истиной>>, и, конечно, все
согласятся, что суждения вроде <<Цицерон был великий оратор>>, <<теперь день>>
и~т.~п. "--- не суть разумные истины. Но они не суть таковые не
потому, что они как бы случайно имеют эмпирическое содержание, а потому,
что они суть лишь положительные суждения, не могущие и не долженствующие
иметь никакого иного содержания, кроме некоторого
непосредственно-единичного и некоторой абстрактной определенности.

Положительное суждение имеет свою истину ближайшим образом в отрицательном:
{\em единичное не есть} абстрактно {\em всеобщее}, а предикат единичного, в
силу того, что он есть такой предикат (или же, если рассматривать этот предикат
сам по себе, безотносительно к субъекту, в силу того, что он
{\em есть абстрактно-всеобщее}), сам есть нечто определенное; {\em единичное}
есть поэтому {\em ближайшим образом} некоторое {\em особенное}. Далее, согласно
другому предложению, содержащемуся в положительном суждении, отрицательное
суждение гласит: {\em всеобщее} не есть абстрактно {\em единичное}, а этот
предикат уже потому, что он есть предикат, или, иначе говоря, потому, что он
находится в соотношении с некоторым всеобщим субъектом, есть нечто более
широкое, чем голая единичность, и {\em всеобщее} есть поэтому также
{\em ближайшим образом} некоторое {\em особенное}. "--- Так как это всеобщее,
как субъект, само носит в качестве определения суждения характер единичности,
то оба предложения сводятся к одному:
<<{\em единичное есть некоторое особенное}>>.

Можно заметить (а), что здесь характеристикой предиката оказывается та
{\em особенность}, о которой речь была уже выше\pagenote{См. выше,
стр.~\pageref{bkm:bm34a}.\label{bkm:bm34}}; однако здесь она положена не через
внешнюю рефлексию, а возникла через посредство найденного нами в суждении
отрицательного соотношения. (b) Это определение получается здесь лишь для
предиката. В~{\em непосредственном} суждении, суждении наличного бытия,
субъект есть лежащее в основании; поэтому кажется, что {\em определение}
ближайшим образом {\em протекает в предикате}. На самом же деле это первое
отрицание еще не может быть определением или, собственно говоря,
{\em полаганием единичного}, так как таким полаганием служит лишь второе
отрицание, отрицательное отрицательного.

<<{\em Единичное есть некоторое особенное}>> "--- таково {\em положительное}
выражение отрицательного суждения. Это выражение постольку само не есть
положительное суждение, поскольку последнее в силу своей непосредственности
имеет своими крайними членами лишь абстрактное, а особенное именно через
полагание отношения суждения оказывается первым {\em опосредствованным}
определением. "--- Но это определение следует брать не только как момент
крайнего члена, а также и как то, что оно, собственно говоря, ближайшим образом
и есть, "--- как {\em определение соотношения}, или, иначе говоря, суждение
должно рассматриваться равным образом и как {\em отрицательное}.

Этот переход основывается на отношении между крайними членами и их соотнесением
в суждении вообще. Положительное суждение есть соотношение
{\em непосредственно} единичного и всеобщего, следовательно, таких, из которых
одно вместе с~тем {\em не}~есть то, чт\'{о} есть другое; это соотношение есть
поэтому столь же существенно {\em разделение} или {\em отрицательное}
соотношение; потому следовало положить положительное суждение как отрицательное.
Поэтому логики напрасно подымали столько шума, настаивая на том, чтобы
<<{\em не}>> отрицательного суждения было отнесено к {\em связке}. То, чт\'{о}
в~суждении есть {\em определение} крайнего члена, есть равным образом
и {\em определенное соотношение}. Определение суждения или крайний член
суждения не есть чисто качественное определение {\em непосредственного} бытия,
долженствующее лишь противостоять чему-то {\em иному вне} его. Оно не есть
также и определение рефлексии, которая по своей всеобщей форме ведет себя как
положительное и отрицательное, причем и то и другое положено как исключающее и
тождественно с другим лишь {\em в~себе}. Определение суждения, как определение
понятия, есть в~себе самом нечто всеобщее, положенное как {\em продолжающееся
дальше} в своё иное. Обратно, {\em соотношение} суждения есть то же самое
определение, какое свойственно крайним членам; ибо оно именно и есть эта
всеобщность и продолжаемость их друг в друге; поскольку между крайними членами
имеется различие, соотношение суждения тоже содержит в~себе отрицательность.

Вышеуказанный переход от формы {\em соотношения} к форме {\em определения}
приводит к тому {\em непосредственному выводу}, что <<{\em не}>> связки должно
быть в такой же мере присоединено к предикату, и последний должен быть
определен как {\em не-всеобщее}. Но не-всеобщее путем столь же
непосредственного вывода оказывается {\em особенным}. "--- Если
{\em отрицательное} фиксируется по совершенно абстрактному определению
непосредственного {\em небытия}, то предикат есть лишь {\em совершенно
неопределенное} не-всеобщее. Об этом определении логика обыкновенно трактует
при рассмотрении {\em контрадикторных} понятий, причем она там настоятельно
внушает как нечто важное, что, говоря об {\em отрицании} какого-нибудь понятия,
следует иметь в виду только отрицательное, и последнее следует понимать лишь
как просто {\em неопределенный} объем {\em иного} (des Andern) по отношению к
данному положительному понятию. Так, например, просто {\em небелое} есть
согласно этому взгляду, как красное, желтое, голубое и~т.~д., так и черное.
Но {\em белое} как таковое есть {\em чуждое понятию} определение созерцания;
поэтому <<{\em не}>> в отношении к белому есть столь же чуждое понятию
{\em небытие}, каковая абстракция рассматривалась нами в самом начале логики,
причем ее ближайшей истиной оказалось {\em становление}. Если при рассмотрении
определений суждения в качестве примера пользуются таким чуждым понятию
содержанием, почерпнутым из созерцания и представления, и определения
{\em бытия} и {\em рефлексии} принимаются за определения суждения, то это так
же {\em некритично}, как, например, согласно Канту, было бы некритично
применять понятия рассудка к бесконечной идее разума или к так называемой
{\em вещи-в-себе; понятие}, к которому принадлежит также и исходящее из него
{\em суждение}, есть истинная {\em вещь-в-себе} или {\em разумное}, тогда как
упомянутые выше определения принадлежат к области {\em бытия} или
{\em сущности} и еще не суть формы, развитые до того состояния, в котором они
получают такой вид, какой свойственен им в их истине, в {\em понятии}. "---
Если, как это обычно делается, не идут дальше белого, красного, как
{\em чувственных} представлений, то получает название понятия нечто такое, что
есть лишь определение представления, и в таком случае не-белое, не-красное,
конечно, не есть нечто положительное, точно так же, как и не-треугольное есть
нечто совершенно неопределенное, ибо определение, покоящееся на числе и
определенном количестве вообще, есть по существу {\em определение
безразличное, чуждое понятию}. Но подобно самому {\em небытию} и такого
рода чувственное содержание должно быть {\em постигнуто в понятии} и утратить
то безразличие и ту абстрактную непосредственность, какими оно характеризуется
в слепом, неподвижном представлении. Уже в наличном бытии чуждое мысли
{\em ничто} становится {\em границей}, через посредство которой
{\em нечто соотносится} все же с некоторым {\em иным} вне его. В~рефлексии же
оно есть {\em отрицательное}, которое существенным образом {\em соотносится}
с~некоторым {\em положительным} и тем самым представляет собой
{\em определенное} отрицательное; некое отрицательное уже не есть больше
вышеупомянутое {\em неопределенное небытие}, оно положено так, что оно есть
лишь постольку, поскольку ему противостоит положительное, а третьим служит их
{\em основание;} тем самым отрицательное удерживается в некоторой замкнутой
сфере, в~которой то, чем одно {\em не}~является, есть нечто {\em определенное}.
"--- Но еще в~большей мере в~абсолютно текучей непрерывности понятия и его
определений <<{\em не}>> непосредственно есть нечто
положительное, и {\em отрицание} есть не только определенность,
но вобрано во всеобщность и положено как тождественное ей. Поэтому
не-всеобщее есть сразу же {\em особенное}.

2. Так как отрицание касается соотношения суждения, а
{\em отрицательное суждение} рассматривается еще как таковое, то оно
{\em прежде всего еще} есть {\em суждение;}
тем самым имеется отношение субъекта и предиката или
единичности и всеобщности, а также и их соотношение: {\em форма суждения}.
Субъект, как лежащее в основании непосредственное, остается
незатронут отрицанием и, следовательно, сохраняет свое определение,
заключающееся в том, чтобы иметь некоторый предикат, или, иными словами,
сохраняет свое соотношение со всеобщностью. Поэтому в отрицательном
суждении отрицанию подвергается не вообще всеобщность в предикате, а его
абстрактность или определенность, которая по отношению к той всеобщности
выступала как {\em содержание}. "---
Отрицательное суждение не есть, следовательно, тотальное
отрицание; та всеобщая сфера, которая содержит в~себе предикат, еще
сохраняется; поэтому соотношение субъекта с предикатом еще остается по
существу {\em положительным;} еще сохранившееся {\em определение}
предиката есть в такой же мере {\em соотношение}. "--- Если, например,
говорят, что роза {\em не}~красна, то этим подвергают отрицанию и отделяют
от всеобщности, присущей равным образом и предикату, лишь его
{\em определенность;} всеобщая сфера "--- {\em цвет}
"--- сохраняется; когда говорят <<роза не красна>>, то тем самым
принимают, что она обладает некоторым цветом, и притом другим цветом; со
стороны этой всеобщей сферы суждение еще остается положительным.

<<{\em Единичное есть некоторое особенное}>>, "--- эта положительная форма
отрицательного суждения выражает сказанное непосредственно; особенное содержит
в~себе всеобщность. Эта форма выражает сверх того также и то, что предикат есть
не только нечто всеобщее, но также еще и нечто определенное. Отрицательная
форма содержит в~себе то же самое; ибо поскольку, например, роза не красна, она
не только должна сохранить, как предикат, всеобщую сферу цвета, но должна иметь
также {\em какой-нибудь другой определенный цвет;} упраздняется, следовательно,
лишь {\em единичная} определенность красноты и не только оставляется всеобщая
сфера, но сохраняется также и определенность, превращенная, однако,
в~{\em неопределенную}, во всеобщую определенность, т.~е. в~особенность.

3. {\em Особенность}, оказавшаяся согласно вышесказанному положительным
определением отрицательного суждения, есть то, что опосредствует между
единичностью и всеобщностью; таким образом, отрицательное суждение есть теперь
вообще опосредствующий переход к третьей ступени, к~{\em рефлексии суждения
наличного бытия в~себя само}. Со стороны своего объективного значения оно есть
лишь момент изменения акциденций или (в~сфере наличного бытия) изолированных,
отдельных свойств конкретного. В~силу этого изменения полная определенность
предиката "--- или {\em конкретное} "--- выступает как нечто положенное.

{\em Единичное есть особенное} согласно положительному выражению
отрицательного суждения. Но единичное вместе с тем и
{\em не}~есть особенное; ибо особенность имеет более широкий объем,
чем единичность; она, следовательно, есть предикат, не соответствующий
субъекту, и, стало быть, предикат, в котором субъект еще не имеет своей
истины. {\em Единичное есть только единичное}, отрицательность, соотносящаяся
не с иным, будь последнее положительным или отрицательным, а лишь с самой
собой. "--- Роза не есть {\em какое бы то ни было} цветное, а имеет лишь
определенный цвет, являющийся цветом розы. Единичное есть не некоторое
неопределенно определенное, а определенно определенное.

Если исходить из этой положительной формы отрицательного суждения, то указанное
отрицание последнего представляется опять-таки лишь {\em первым} отрицанием.
Но на самом деле оно не таково. Скорее напротив, отрицательное суждение уже
само по себе есть второе отрицание или отрицание отрицания, и требуется
положить то, что оно есть само по себе. А~именно, оно отрицает
{\em определенность предиката} положительного суждения, его {\em абстрактную}
всеобщность, или, если рассматривать его со стороны содержания, отрицает то
единичное качество субъекта, которое предикат содержит в~себе. Отрицание же
определенности есть уже второе отрицание, стало быть, бесконечное возвращение
единичности в~себя самое. Тем самым {\em восстановлена} конкретная тотальность
субъекта или, вернее сказать, лишь теперь субъект {\em положен} как единичное,
так как через отрицание и снятие этого отрицания его опосредствовали с самим
собой. Со своей стороны, предикат тем самым перешел от первой всеобщности к
абсолютной определенности и сравнялся с субъектом. Суждение постольку гласит:
<<{\em единичное единично}>>. "--- С другой стороны, поскольку следовало брать
субъект также и как {\em всеобщее} и поскольку в отрицательном суждении
предикат, который по отношению к этому определению субъекта есть единичное,
{\em расширился} до {\em особенности} и поскольку, далее, отрицание этой
{\em определенности} есть равным образом и {\em очищение} содержащейся в~нем
всеобщности, то это суждение гласит также и следующим образом:
<<{\em всеобщее есть всеобщее}>>.

В обоих этих суждениях, которые в предшествующем
изложении\pagenote{См. выше, стр.~\pageref{bkm:bm35a}.\label{bkm:bm35}}
получались путем внешней рефлексии, предикат уже выражен в
своей положительности. Но сперва само отрицание отрицательного суждения
должно выступить в форме отрицательного суждения. Мы видели, что в нем еще
остались некоторое {\em положительное соотношение} субъекта с предикатом и
{\em всеобщая сфера} последнего. С~этой стороны, стало быть,
отрицательное суждение содержало в~себе более очищенную от ограниченности
всеобщность, чем положительное суждение, а потому оно тем более должно быть
отрицаемо относительно субъекта как единичного. Этим путем отрицанию
подвергается {\em весь объем} предиката и уже не остается никакого
положительного соотношения между ним и субъектом. Это есть
{\em бесконечное суждение}.

\subsection[c) Бесконечное суждение]{c) Бесконечное суждение}

Отрицательное суждение есть столь же мало истинное суждение,
как и положительное. Но бесконечное суждение, долженствующее служить его
истиной, есть по своему отрицательному выражению {\em отрицательно-бесконечное}
суждение "--- суждение, в котором упразднена также и форма суждения. "--- Но
это "--- {\em бессмысленное суждение}. Оно должно быть {\em суждением}, стало
быть, содержать в~себе некоторое соотношение субъекта с предикатом; но
{\em вместе с тем} в~нем не должно быть такого соотношения. "--- Хотя название
бесконечного суждения и приводится в обычных логиках, но при этом остается
неясным, как с ним обстоит дело. "--- Примеры
отрицательно-бесконечных суждений легко получить, соединяя отрицательным
образом в виде субъекта и предиката такие определения, из которых одно не
содержит в~себе не только определенности другого, но и его всеобщей сферы;
следовательно, например, <<дух не есть красное, не есть желтое>> и~т.~д., <<не
есть кислое, щелочное>> и~т.~д, <<роза не есть слон>>, <<рассудок не есть
стол>> и~т.~п. "--- Эти суждения {\em правильны} или, как выражаются,
{\em истинны}, но, несмотря на такую истинность, бессмысленны и нелепы. "---
Или, вернее, они {\em вовсе не}~суть {\em суждения}. "--- Более
реальным примером бесконечного суждения является
{\em злой} поступок. В~{\em гражданско-правовом}
споре нечто подвергается отрицанию лишь как собственность
противной стороны; но при этом истцы соглашаются с тем, что оно должно было
бы принадлежать противной стороне, если бы она имела на это право, и на это
нечто предъявляется притязание лишь во имя права; следовательно, всеобщая
сфера, право, в этом отрицательном суждении признается и сохраняется.
{\em Преступление} же есть {\em бесконечное суждение}, отрицающее не только
{\em особенное} право, но вместе с тем и всеобщую его сферу, т.~е. отрицающее
{\em право как право}. Оно, правда, {\em правильно} в том
смысле, что это есть некоторый действительный поступок, но так как этот
поступок соотносится совершенно отрицательно с нравственностью,
составляющей его всеобщую сферу, то он бессмысленен.

{\em Положительное} в~бесконечном суждении, этом отрицании отрицания, есть
{\em рефлексия единичности} в~себя самое, благодаря чему она (единичность)
впервые и положена как {\em определенная определенность}.
<<{\em Единичное единично}>>, "--- таково было его выражение согласно этой
рефлексии. Субъект в суждении наличного бытия выступает как
{\em непосредственное} единичное и постольку скорее лишь как {\em нечто}
вообще. Лишь через опосредствование отрицательного и бесконечного суждения
он впервые {\em положен} как единичное.

Тем самым единичное {\em положено} как {\em непрерывно продолжающееся в своем
предикате}, который тождественен с ним; тем самым и всеобщность равным образом
выступает уже не как {\em непосредственная} всеобщность, а как {\em охват}
различенных. Положительно-бесконечное суждение также гласит: <<{\em всеобщее
есть всеобщее}>>. Таким образом и оно положено как возвращение в~себя само.

Этой рефлексией определений суждения в~себя суждение теперь сняло себя; в
отрицательно-бесконечном суждении различие, так сказать, {\em слишком велико}
для того, чтобы оно еще оставалось суждением; субъект и предикат не имеют в нем
никакого положительного соотношения друг с другом; напротив, в
положительно-бесконечном суждении имеется лишь тождество, и оно вследствие
полного отсутствия в нем различия уже не есть суждение.

Строже говоря, сняло себя {\em суждение наличного бытия;} тем самым {\em
положено} то, чт\'{о} содержится в {\em связке} суждения, а~именно,
что качественные крайние члены сняты в этом своем тождестве. Но так
как это единство есть понятие, то оно (единство) опять-таки равным образом
непосредственно расщеплено на свои крайние члены и выступает как суждение,
определения которого, однако, суть уже не непосредственные, а~рефлектированные
в~себя. {\em Суждение наличного бытия} перешло в~{\em суждение рефлексии}.

\section[В. Суждение рефлексии]{В. Суждение рефлексии}

В возникшем теперь суждении субъект есть некоторое единичное как таковое;
равным образом всеобщее уже более не есть {\em абстрактная} всеобщность или
{\em единичное свойство}, а положено как такое всеобщее, которое синтезировало
себя воедино путем соотнесения различенных, или (если его рассматривать со
стороны содержания разных определений вообще) "--- как {\em самособирание}
воедино многообразных свойств и существований. "--- Если нужно давать примеры
предикатов суждений рефлексии, то они должны быть другого рода, чем для
суждений наличного бытия. В~суждении рефлексии, собственно говоря, впервые
имеется некоторое {\em определенное содержание},
т.~е. некоторое содержание вообще; ибо это содержание есть
рефлектированное в тождество определение формы как отличное от формы,
поскольку она есть различенная определенность, каковой она еще продолжает
быть как суждение. В~суждении наличного бытия содержание есть лишь
некоторое непосредственное или абстрактное, неопределенное содержание. "---
Примерами рефлективных суждений могут поэтому служить следующие суждения:
<<человек {\em смертен}>>, <<вещи {\em преходящи}>>, <<эта вещь
{\em полезна}>>, <<{\em вредна}>>; <<{\em твердость}>>, <<{\em упругость}
тел>>, <<{\em счастие}>> и тому подобное представляют собой такие своеобразные
предикаты. Они выражают собой некоторую существенность, которая, однако, есть
некоторое определение в {\em отношении} или некоторая {\em охватывающая}
всеобщность. Эта {\em всеобщность}, которая определится далее в движении
рефлективного суждения, еще отлична от {\em всеобщности понятия} как таковой;
хотя она уже больше не есть абстрактная всеобщность качественного суждения, она
все же находится еще в соотношении с тем непосредственным, из которого она
происходит, и это непосредственное лежит в основании ее отрицательности. "---
Понятие определяет наличное бытие ближайшим образом так, что делает его
определения {\em определениями отношения}, непрерывными продолжениями их самих
в разном многообразии существования, так что истинно всеобщее есть, правда, их
внутренняя сущность, но {\em в~явлении}, и эта {\em релятивная} природа или,
можно также сказать, их {\em признак} еще не есть их в-себе-и-для-себя-сущее.

Можно сказать, что почти напрашивается определить суждение рефлексии как
суждение {\em количества}, подобно тому как мы определили суждение наличного
бытия также и как {\em качественное} суждение. Но подобно тому как
{\em непосредственность} в последнем была не только {\em сущая}, но по существу
также и опосредствованная и {\em абстрактная} непосредственность, так и здесь
эта снятая непосредственность есть не только снятое качество, стало быть, не
только {\em количество;} напротив, подобно тому как качество есть самая внешняя
непосредственность, так и это количество есть таким же образом
{\em самое внешнее} из принадлежащих к опосредствованию {\em определений}.

По поводу {\em определения}, каким оно в своем движении выступает в
рефлективном суждении, следует сделать еще то замечание, что в суждении
наличного бытия {\em движение} этого определения являло себя {\em в предикате},
так как это суждение имело определение непосредственности, и поэтому субъект
выступал как то, что лежит в основании. По такому же основанию в рефлективном
суждении поступательное движение процесса определения протекает
в~{\em субъекте}, так как это суждение имеет своим определением
{\em рефлектированное в-себе-бытие}. Существенным здесь поэтому служит {\em
всеобщее} или предикат; предикат составляет поэтому здесь то {\em лежащее
в~основании}, которым следует мерить субъект, и последний должен быть определен
соответственно этому лежащему в основании. "--- Однако и предикат получает
дальнейшее определение через дальнейшее развитие формы субъекта, но он получает
это развитие {\em косвенно}, развитие же субъекта оказывается по
вышеприведенному основанию {\em прямым} дальнейшим определением.

Что касается объективного значения этого суждения, то в нем
единичное вступает в наличное бытие через свою всеобщность, но как носящее
характер некоторого существенного определения отношения, некоторой такой
существенности, которая, проходя сквозь многообразие явления, сохраняет
себя; субъект {\em должен} быть тем, что определено в~себе и для себя; эту
определенность он имеет в своем предикате. С~другой стороны, единичное
рефлектировано в этот свой предикат, который есть его всеобщая сущность;
постольку субъект есть существующее и являющееся. Предикат в этом суждении
уже больше не {\em присущ} (inhäriert) субъекту; предикат есть скорее
{\em сущее в~себе}, под которое {\em подведено} это единичное как
акцидентальное. Если суждения наличного бытия могут быть определены также
и как {\em суждения присущности} (der Inhärenz), то суждения рефлексии суть
скорее {\em суждения подведения [под более общее]} (Urteile der Subsumtion).

\subsection[а) Сингулярное суждение]{а) Сингулярное\pagenote{Для обозначения
диалектических категорий <<всеобщее>>, <<особенное>> и <<единичное>>,
являющихся, как отмечает Энгельс ({\em Engels}, Dialektik der Natur,
M.--L. 1935, S.~664), теми <<тремя определениями, в которых движется все
Учение о понятии>>, Гегель пользуется немецкими словами <<Аll\-ge\-mei\-nes>>,
<<Beson\-deres>>, <<Ein\-zel\-nes>>. Для обозначения же трех видов суждения
рефлексии, соответствующих принятой в формальной логике классификации
суждений <<по количеству>>, Гегель употребляет заимствованные из латинского
языка прилагательные <<singu\-lär>>, <<parti\-kulär>>, <<univer\-sell>>. Этого
терминологического различения придерживается и Энгельс в цитированном нами
в примечании \ref{bkm:bm31} отрывке о классификации суждений. Поэтому мы сочли
необходимым провести это различение и в русском переводе. Термины
<<сингулярный>>, <<партикулярный>>, <<универсальный>> удобны еще и потому, что
пользование ими облегчает переход к таким выражениям, как
<<партикуляризация>>, <<партикуляризировать>>, служащим для перевода
гегелевских <<Parti\-kulari\-sation>>, <<parti\-kulari\-sieren>>.\\
Что касается немецких слов <<All\-gemei\-nes>> и <<Beson\-deres>>,
то интересное указание на их первоначальное значение
имеется в письме Маркса Энгельсу от 26~марта 1868~г. Маркс пишет: <<Что
сказал бы старый Гегель, если бы он на том свете узнал, что
{\em das All\-ge\-mei\-ne} ({\em общее})
на немецком и на северных наречиях означает не~что иное, как
общинную землю, а что {\em das Sundre} "--- {\em Besondre} ({\em особенное})
есть не~что иное, как выделенное из общей земли особое
владение? Таким образом, ведь совершенно же очевидно, что логические
категории возникают из <<нашего общения>> [из общественных отношений]>>
({\em Маркс и Энгельс}, Письма,
пер. Адоратского, М.---Л. 1931, стр.~232).\label{bkm:bm36}} суждение}

Непосредственное рефлективное суждение теперь гласит
опять-таки: <<{\em единичное всеобще}>>;
но субъект и предикат имеют вышеуказанное значение; поэтому
можно ближайшим образом выразить наше суждение так:
<<{\em это есть некоторое существенным образом всеобщее}>>.

Но некоторое <<это>> {\em не}~есть нечто существенным образом всеобщее.
Указанное по своей всеобщей форме {\em положительное}
вообще суждение должно быть взято отрицательно. Но так как
суждение рефлексии есть не просто положительное суждение, то и отрицание не
касается прямо предиката, который в этом суждении не просто присущ
субъекту, а есть {\em сущее в~себе}.
Субъект же есть, наоборот, изменчивое и подлежащее
определению. Поэтому отрицательное суждение должно быть здесь
сформулировано так: <<{\em не некоторое это} есть некоторое всеобщее
рефлексии>>\pagenote{<<Всеобщее рефлексии>> (das
Allgemeine der Reflexion), т.~е. всеобщее, как оно дано в рефлексии,
противопоставляется у Гегеля <<всеобщему понятия>> (das Allgemeine des
Begriffes), т.~е. всеобщему, как оно дано в
понятии.\label{bkm:bm37}}; такое <<{\em в~себе}>>
обладает более всеобщим существованием, чем существование
лишь в некотором <<этом>>. Сингулярное суждение имеет поэтому свою ближайшую
истину в {\em партикулярном} суждении.

\subsection[b) Партикулярное суждение]{b) Партикулярное суждение}

Неединичность субъекта, которая должна быть положена вместо его сингулярности
в~первом суждении рефлексии, есть особенность. Но единичность определена
в~суждении рефлексии как {\em существенная единичность;} поэтому особенность не
может быть {\em простым}, {\em абстрактным} определением, в~котором единичное
было бы упразднено и существующее погибло бы до основания, а~может выступать
лишь как некоторое расширение его во внешней рефлексии; поэтому субъектом
служат <<{\em некоторые эти}>> или <<некоторое особенное
{\em множество единичных}>>.

Это суждение: <<{\em некоторые единичные суть некоторое всеобщее рефлексии}>>
"--- выступает ближайшим образом как положительное суждение, но оно в такой же
мере также и отрицательно; ибо <<{\em некоторое}>> содержит в~себе всеобщность;
со стороны этой всеобщности оно может быть рассматриваемо как {\em объемлющее;}
но поскольку <<некоторое>> есть особенность, оно вместе с тем и неадекватно ей.
{\em Отрицательное} определение, полученное субъектом в результате перехода
сингулярного суждения [в~партикулярное], есть, как показано выше, также и
определение соотношения, связки. "--- В~суждении <<{\em некоторые} люди
счастливы>> заключается непосредственный вывод: <<{\em некоторые} люди
{\em не}~суть счастливы>>. Если {\em некоторые} вещи полезны, то именно в силу
этого {\em некоторые} вещи {\em не}~полезны. Положительное и отрицательное
суждение уже не оказывается одно вне другого, а партикулярное суждение
непосредственно содержит в~себе оба суждения вместе именно потому, что
оно есть суждение рефлексии. "--- Но в~силу этого партикулярное суждение
{\em неопределенно}.

Если мы, далее, будем в примерах такого суждения рассматривать субъект "---
<<некоторые люди>>, <<некоторые животные>> и~т.~д., то окажется, что, кроме
партикулярного определения формы "--- <<некоторые>>, "--- он содержит в~себе
еще и определение содержания "--- <<человек>> и~т.~д. Субъект сингулярного
суждения мог гласить: <<этот человек>>, "--- некая сингулярность, которая,
собственно говоря, принадлежит области внешнего показывания; правильнее
поэтому, что субъект гласит примерно: <<Кай>>. Субъектом же партикулярного
суждения уже не может служить такое выражение, как <<{\em некоторые Кайи}>>;
ибо Кай должен быть некоторым единичным как таковым. К~выражению
<<{\em некоторые}>> присоединяется поэтому некоторое более всеобщее
{\em содержание}, например, <<{\em люди}>>, <<{\em животные}>> и~т.~д. Это
не только эмпирическое, но и определенное формой суждения содержание; а именно,
последнее есть некоторое {\em всеобщее}, ибо выражение <<некоторые>> содержит
в~себе всеобщность, и эта всеобщность вместе с тем должна быть отделена от
единичных, так как в основании лежит рефлектированная единичность. Говоря
точнее, эта всеобщность есть также и {\em всеобщая природа} или {\em род}
(<<человек>>, <<животное>>), являясь {\em предвосхищением} той всеобщности,
которая есть результат рефлективного суждения, подобно тому как положительное
суждение, имея субъектом {\em единичное}, предвосхищало то определение, которое
есть результат суждения наличного бытия.

Субъект, содержащий в~себе единичные, их отношение к особенности и их всеобщую
природу, постольку уже положен как тотальность определений понятия. Но это
рассмотрение, собственно говоря, носит внешний характер. То, что в субъекте
ближайшим образом уже приведено через его форму во взаимное {\em соотношение},
есть расширение <<этости>> в особенность; но это обобщение не адекватно
обобщаемому; <<{\em это}>> есть нечто вполне определенное, а <<{\em некоторое
это}>> (или <<{\em некоторые эти}>>) неопределенно. Расширение должно касаться
самого <<{\em этого}>>, должно, стало быть, соответствовать ему, быть
{\em вполне определенным;} таковым расширением служит тотальность или
ближайшим образом {\em всеобщность} вообще.

В основании этой всеобщности лежит <<{\em это}>>, ибо единичное есть здесь
рефлектированное в~себя; его дальнейшие определения протекают поэтому в~нем
{\em внешним образом}, и подобно тому как особенность в силу этого определила
себя как <<{\em некоторые}>>, так и та всеобщность, которой достиг субъект,
есть {\em всякость} (Allheit)\pagenote{Слово <<всякость>> не вполне передает
тот смысл, который вкладывается Гегелем в немецкое слово <<Allheit>>:
<<всякость>> образована от слова <<всякий>>, между тем как тут идет речь
о~производном от слова <<все>> (alle). <<Allheit>> означает у Гегеля
эмпирическую {\em совокупность всех}, <<эмпирическую всеобщность>>, как он сам
поясняет на стр.~\pageref{bkm:bm38a}, "--- такую <<форму всеобщности, на
которую обыкновенно раньше всего набредает рефлексия>> ({\em Гегель}, Соч.,
т.~I, стр.~283).\label{bkm:bm38}}, и партикулярное суждение перешло в~{\em
универсальное}.

\subsection[с) Универсальное суждение]{с) Универсальное суждение}

Всеобщность в том виде, в каком она присуща субъекту
универсального суждения, есть внешняя всеобщность рефлексии,
{\em всякость;} <<{\em все}>> суть все {\em единичные;}
единичное остается в этом <<{\em все}>> неизменным.
Поэтому указанная всеобщность есть лишь {\em суммирование}
существующих особо единичных; она есть некоторая
{\em общность} (Gemein\-schaft\-lich\-keit),
присущая им лишь в процессе сравнения. "--- Об
этой общности начинает обыкновенно прежде всего думать субъективное
представление, когда идет речь о всеобщности. Как на
ближайшее основание, почему то или иное определение должно быть
рассматриваемо как всеобщее, указывают на то, что оно
{\em принадлежит многим}. "--- В математическом {\em анализе}
преимущественно предносится уму равным образом это понятие
всеобщности, когда, например, разложение функций в ряд на некотором
{\em многочлене} считается {\em более всеобщим}, чем разложение этой же функции
на некотором {\em двучлене}, так как, дескать, {\em многочлен} представляет
{\em больше единичностей}, чем {\em двучлен}\pagenote{Ср. т.~I <<Науки
логики>>, стр.~222.\label{bkm:bm39}}. Требование, чтобы функция была
изображена в своей всеобщности, было бы удовлетворено, собственно говоря,
только {\em всечленом}, исчерпанной бесконечностью; но здесь сама собой
обнаруживается граница этого требования, и изображение {\em бесконечного}
множества должно удовлетвориться {\em долженствованием} этой бесконечности и
поэтому также и {\em многочленом}. На самом же деле двучлен есть уже всечлен
в~тех случаях, когда {\em метод} или {\em правило} касается лишь зависимости
одного члена от другого и зависимость многих членов от их предшествующих не
партикуляризируется, а имеет своей основой одну и ту же функцию. {\em Метод}
или {\em правило} следует рассматривать как истинно {\em всеобщее;}
в~дальнейшем продолжении разложения в ряд или в разложении в ряд многочлена
это правило лишь {\em повторяется;} путем увеличения количества членов оно,
следовательно, нисколько не выиграет в отношении всеобщности. Уже раньше мы
говорили о~дурной бесконечности и ее обманах; всеобщность понятия есть
{\em достигнутая потусторонность;} указанная же бесконечность остается
обремененной потусторонним как чем-то недостижимым, поскольку она остается
только бесконечным {\em прогрессом}. Если, говоря о всеобщности, уму
предносится лишь {\em всякость}, т.~е. такая всеобщность, которая должна быть
исчерпана в единичных как единичных, то это есть впадение вновь в указанную
дурную бесконечность, или же здесь за всякость принимается то, что есть лишь
{\em множество}. Однако множество, как бы оно ни было велико, безоговорочно
остается лишь партикулярностью и не есть всякость. "--- Но уму смутно
предносится при этом в-себе-и-для-себя-сущая всеобщность {\em понятия}. Понятие
есть то, что насильственно гонит дальше, заставляя идти за пределы пребывающей
единичности (за которую держится представление) и за пределы внешней рефлексии
представления и подставляя всякость {\em как тотальность} или, вернее,
категорическое в-себе-и-для-себя-бытие.

\label{bkm:bm38a}Это и в другом отношении сказывается на всякости, которая
вообще представляет собой {\em эмпирическую} всеобщность. Поскольку единичное
предполагается как некоторое непосредственное и поэтому {\em преднаходится} и
{\em берется} извне, постольку рефлексия, объединяющая его во всякость, ему
столь же внешня. Но так как единичное, как <<{\em это}>>, безоговорочно
безразлично к этой рефлексии, то всеобщность и такого рода единичное не могут
объединиться в одно единство. Эмпирическая всякость {\em остается} поэтому
некоторой {\em задачей}, некоторым {\em долженствованием}, которое, таким
образом, не может быть изображено как бытие. Эмпирически-всеобщее предложение
(ибо такого рода предложения все же выставляются) покоится на молчаливом
соглашении, что если только нельзя указать ни одного {\em примера} чего-нибудь
противоположного, то {\em множество} случаев должно считаться {\em всякостью;}
или, иначе говоря, что {\em субъективную} всякость, а именно всякость
{\em ставших известными} случаев, можно принять за {\em объективную} всякость.

При более близком рассмотрении занимающего нас здесь {\em универсального
суждения} мы убеждаемся, что субъект, который, как было замечено выше, содержит
в-себе-и-для-себя-сущую всеобщность как {\em пред-положенную}, теперь имеет ее
в~себе также и как {\em положенную}. <<{\em Все люди}>> означает,
{\em во-первых}, {\em род} <<человек>>, {\em во-вторых}, этот же род в его
распадении на единичности, но так, что единичные вместе с тем расширены до
всеобщности рода; обратно, всеобщность через эту связанность с единичностью
определена столь же полно, как и единичность; тем самым {\em положенная}
всеобщность стала {\em равной} той, которая {\em пред-положена}.

Но, собственно говоря, следует прежде всего обратить внимание не на
{\em пред-положенное}, а рассмотреть сам по себе тот результат, который
получился касательно определения формы. "--- Единичность, расширившись до
всякости, {\em положена} как такая отрицательность, которая есть тождественное
соотношение с собой. Тем самым она не осталась той первой единичностью, какова,
например, единичность некоего Кайя, но есть определение, тождественное
с~всеобщностью, или абсолютная определенность всеобщего. "--- Та {\em первая}
единичность сингулярного суждения не была {\em непосредственной} единичностью
положительного суждения, а возникла через диалектическое движение суждения
наличного бытия вообще; она была уже определена к тому, чтобы быть
{\em отрицательным тождеством} определений того суждения. Это и есть истинное
пред-положение в суждении рефлексии; по отношению к совершающемуся в последнем
полаганию та {\em первая} определенность единичности была ее <<{\em в-себе}>>
[-бытием]; стало быть, то, что единичность есть {\em в~себе}, теперь
{\em положено} движением рефлективного суждения, а именно, единичность положена
как тождественное соотношение определенного с собой самим. Благодаря этому та
{\em рефлексия}, которая расширяет единичность до всякости, уже не есть внешняя
этой единичности рефлексия, и эта единичность лишь становится {\em для себя}
тем, чем она уже {\em была в~себе}. "--- Таким образом, истинным результатом
оказывается {\em объективная всеобщность}. Постольку субъект совлек с себя
присущее рефлективному суждению определение формы, переходившее от
<<{\em этого}>> через <<{\em некоторое}>> к <<{\em всякости}>>; вместо
<<{\em все люди}>> теперь надо сказать: <<человек>> (der Mensch).

Всеобщность, возникшая благодаря этому, есть {\em род}, "--- такая всеобщность,
которая в самой себе есть конкретное. Род не {\em принадлежит} субъекту или,
иначе говоря, он не есть {\em единичное} свойство и вообще не есть некоторое
свойство субъекта; всякую единичную определенность род содержит растворенной
в~его субстанциальной сплошности (Gedie\-genheit). "--- Будучи положен как это
отрицательное тождество с самим собой, род есть по существу субъект; но он уже
не {\em подчинен} (ist nicht subsumiert) своему предикату. Тем самым теперь
изменяется вообще природа рефлективного суждения.

Последнее было по существу суждением {\em подчинения} или подведения под более
общее (Urteil der Substi\-tution). Предикат был определен по отношению к
своему субъекту как {\em в-себе сущее} всеобщее; по своему содержанию предикат
мог рассматриваться как существенное определение отношения или также как
признак, "--- определение, по которому субъект есть лишь некоторое существенное
{\em явление}. Но определенный как {\em объективная всеобщность} субъект уже
перестает быть подчиненным такому определению отношения или охватывающей
рефлексии; такой предикат есть по отношению к этой всеобщности скорее некоторое
особенное. Тем самым отношение субъекта и предиката здесь стало обратным, и
суждение постольку ближайшим образом сняло себя.

Это снятие суждения совпадает с тем, чем становится {\em определение связки},
которое нам остается еще рассмотреть; снятие определений суждения и переход их
в связку есть одно и то же. "--- А именно, поскольку субъект возвел себя во
всеобщность, он в этом определении стал равен предикату, который, как
рефлектированная всеобщность, объемлет собой также и особенность; поэтому
субъект и предикат тождественны, т.~е., они слились в связку. Это тождество
есть род или в-себе-и-для-себя-сущая природа некоторой вещи. Следовательно,
поскольку это тождество снова расщепляется в некоторое суждение, субъект и
предикат соотносятся друг с другом через {\em внутреннюю природу;} это "---
соотношение {\em необходимости}, в котором указанные определения суждения суть
лишь несущественные различия. "--- {\em <<То, что присуще всем единичным вещам
какого-нибудь рода, в~силу их природы присуще и роду>>} "--- вот
непосредственный вывод и выражение того, что получилось раньше, а именно, того,
что субъект, например, {\em <<все люди>>}, совлекает с себя свое формальное
определение и вместо {\em <<все люди>>} следует сказать <<{\em человек}>>. "---
Эта в-себе-и-для-себя-сущая взаимосвязь составляет основу нового суждения "---
{\em суждения необходимости}.

\section[С. Суждение необходимости]{С. Суждение необходимости}

Определение, до которого доразвилась всеобщность, есть, как
оказалось, {\em в-себе-и-для-себя-сущая} или {\em объективная
всеобщность}, которой в сфере сущности соответствует {\em субстанциальность}.
Первая отличается от последней тем, что она принадлежит
области {\em понятия} и в силу этого есть не только {\em внутренняя}, но и
{\em положенная} необходимость своих определений; или, иначе говоря, тем, что
различие ей {\em имманентно},
между тем как субстанция имеет, напротив, свое различие лишь
в своих акциденциях, а не как принцип внутри себя самой.

В суждении эта объективная всеобщность теперь {\em положена;} тем самым
она дана, {\em во-первых}, с этой ее существенной определенностью как
имманентной ей и, {\em во-вторых}, с этой
же определенностью как отличной от нее в качестве {\em особенности},
субстанциальную основу которой составляет указанная
всеобщность. Таким образом, она определена как {\em род} и {\em вид}.

\subsection[а) Категорическое суждение]{а) Категорическое суждение}

{\em Род разделяется} или существенным образом расталкивает себя,
распадается на {\em виды;} он есть род
лишь постольку, поскольку он объемлет собой виды; вид есть вид лишь
постольку, поскольку он, с одной стороны, существует в единичных вещах, а с
другой "--- поскольку он в роде представляет собой некоторую
более высокую всеобщность. "--- И вот {\em категорическое суждение}
имеет такого рода всеобщность своим предикатом, в котором
субъект находит свою {\em имманентную}
природу. Но само оно есть лишь первое или {\em непосредственное}
суждение необходимости; поэтому та определенность субъекта, в
силу которой он есть по отношению к роду некоторое особенное или по
отношению к виду некоторое единичное, носит постольку характер
непосредственности внешнего существования. "--- Но и
объективная всеобщность также находит здесь пока что лишь
свою {\em непосредственную}
партикуляцию; с одной стороны, она поэтому сама есть
некоторая определенная всеобщность, по отношению к которой имеются более
высокие роды; с другой же стороны, она не есть обязательно {\em ближайшая}
всеобщность, т.~е. ее определенность не есть непременно
принцип специфической особенности субъекта. Но что тут
{\em необходимо} "--- это {\em субстанциальное тождество}
субъекта и предиката, по сравнению с которым то своеобразие,
которым субъект отличается от предиката, выступает лишь как несущественная
положенность или даже лишь как название; субъект в своем предикате
рефлектирован в свое в-себе-и-для-себя-бытие. "--- Такой
предикат не должен быть смешиваем с предикатами вышерассмотренных суждений;
например, если сваливают в одну кучу суждения: <<роза красна>> и
<<роза есть растение>>, или: <<это кольцо желто>> и <<оно есть золото>>,
и такое внешнее свойство, как цвет цветка, признается
предикатом, равнозначащим с растительной природой цветка, то упускается из
виду такое различие, которое должно бросаться в глаза даже самому
обыденному пониманию. "--- Поэтому категорическое суждение
должно быть определенно различено от положительного и отрицательного
суждений; в последних то, что высказывается о субъекте, есть некоторое
единичное, случайное содержание, в первом же оно есть тотальность
рефлектированной в~себя формы. В~нем связка имеет поэтому значение
{\em необходимости}, в них же "--- значение лишь абстрактного,
непосредственного {\em бытия}.

Та {\em определенность} субъекта, в силу которой он есть некоторое
{\em особенное} по отношению к предикату, есть пока что еще нечто
{\em случайное;} субъект и предикат соотнесены как необходимо связанные
не через {\em форму} или {\em определенность;}
необходимость выступает поэтому пока что еще как
{\em внутренняя} необходимость. "--- Субъект же есть субъект лишь
как {\em особенное}, а поскольку он обладает объективной всеобщностью,
он, как предполагается, ею
обладает существенным образом со стороны указанной пока что лишь
непосредственной определенности. Объективно-всеобщее, {\em определяя} себя,
т.~е. помещая себя в суждение, находится по существу в тождественном
соотношении с этой вытолкнутой из него {\em определенностью} как
таковой, т.~е. она должна быть существенным образом положена не как просто
случайное. Категорическое суждение лишь через эту {\em необходимость} его
непосредственного бытия впервые делается соответственным своей объективной
всеобщности, таким образом оно перешло в {\em гипотетическое суждение}.

\subsection[b) Гипотетическое суждение]{b) Гипотетическое суждение}

<<{\em Если есть А, то есть В}>>; или, иначе говоря,
<<{\em бытие А (des А) не есть его
собственное бытие, а бытие некоторого другого, бытие В (des В)}>>. "---
Что положено в этом суждении, это "--- {\em необходимая связь}
непосредственных определенностей, которая в категорическом
суждении еще не положена. "--- Здесь имеются {\em два}
непосредственных или внешне-случайных существования, из
которых в категорическом суждении имеется ближайшим образом лишь одно,
субъект; но так как одно существование внешне другому, то это другое
непосредственно тоже внешне первому. "--- Взятое со стороны
этой непосредственности, {\em содержание} обеих
сторон еще безразлично друг к другу; указанное суждение есть поэтому
ближайшим образом предложение пустой формы. Но непосредственность есть,
правда, {\em во-первых}, как таковая, некоторое самостоятельное, конкретное
{\em бытие;} однако, {\em во-вторых},
существенно именно соотношение последнего; это бытие
выступает поэтому в такой же мере и как голая {\em возможность;}
гипотетическое суждение не означает, что {\em А есть} или что
{\em В есть}, а лишь то, что {\em если} есть одно из них, {\em то} есть и
другое; в качестве сущей положена лишь взаимосвязь крайних членов, а не они
сами. Скорее можно сказать, что в этого рода необходимости каждый из членов
положен как в такой же мере означающий {\em бытие некоторого другого}. "---
Начало тождества гласит: <<{\em А есть лишь А, а не В}>> и <<{\em В есть лишь
В, а~не А}>>; напротив, в гипотетическом суждении бытие конечных
вещей положено понятием согласно их формальной истине, а именно, положено,
что конечное есть свое собственное бытие, но в такой же мере и
{\em не собственное}, а бытие некоторого другого. В~сфере бытия конечное
{\em изменяется}, оно становится чем-то иным; в сфере сущности оно есть
{\em явление}, причем положено, что его бытие состоит в том, что в нем
{\em светится} некоторое другое и что {\em необходимость} есть
{\em внутреннее}, еще не положенное как таковое соотношение. Понятие же
состоит в том, что это тождество {\em положено}
и что сущее есть не абстрактное тождество с собой, а {\em конкретное}
тождество и непосредственно в~себе самом "--- бытие некоторого другого.

Гипотетическое суждение можно посредством применения
рефлективных отношений брать более определенно как отношение
{\em основания} и {\em следствия}, {\em условия} и {\em обусловленного},
{\em причинности} и~т.~д.
Как в категорическом суждении выступила в своей понятийной форме
субстанциальность, так в гипотетическом суждении в этой форме выступает
связь причинности. Это отношение и все остальные подчинены ему, но здесь
они уже перестают быть отношениями
{\em самостоятельных сторон},
а последние выступают по существу лишь как моменты одного и
того же тождества. "--- Однако они в нем еще не
противопоставлены согласно определениям понятия как единичное (или
особенное) и всеобщее, а выступают пока что лишь как
{\em моменты вообще}.
Гипотетическое суждение имеет постольку скорее вид некоторого
предложения; подобно тому как партикулярное суждение имеет неопределенное
содержание, так гипотетическое суждение имеет неопределенную форму,
поскольку его содержание не носит характера отношения субъекта и
предиката. "--- {\em В себе},
однако, бытие, так как оно есть здесь бытие иного, есть
именно поэтому {\em единство самого
себя} и {\em иного}
и тем самым
{\em всеобщность;} этим
самым оно есть вместе с тем, собственно говоря, лишь некоторое
{\em особенное}, так как
оно есть определенное и в своей определенности не только с собой
соотносящееся. Но положена здесь не
{\em простая} абстрактная
особенность, а в силу той
{\em непосредственности},
которой {\em обладают
определенности}, моменты указанной особенности выступают как
различные; вместе с тем, в силу их единства, составляющего их соотношение,
особенность выступает также и как их тотальность. "---
Поэтому то, что поистине положено в этом суждении, "---
это всеобщность как конкретное тождество понятия,
определения которого не имеют самостоятельного устойчивого наличия, а суть
лишь положенные в ней (во всеобщности) особенности. Таким образом, оно есть
{\em разделительное суждение}.

\subsection[с) Разделительное суждение]{с) Разделительное суждение}

В~категорическом суждении понятие выступает как объективная
всеобщность [с~одной стороны] и некоторая внешняя единичность [с~другой
стороны]. В~гипотетическом суждении на фоне этой внешности понятие
выявляется в своем отрицательном тождестве; через это последнее его моменты
получают теперь в разделительном суждении положенной ту определенность,
которой они в непосредственном виде обладают в гипотетическом суждении.
Разделительное суждение есть поэтому объективная всеобщность, положенная
вместе с тем в соединении с формой. Оно содержит в~себе, следовательно,
{\em во-первых}, конкретную всеобщность или род в {\em простой форме}, как
субъект; {\em во-вторых}, содержит в~себе {\em эту
же всеобщность}, но как тотальность ее
различенных определений. <<$A$ есть или $B$ или $C$>>. Это есть
{\em необходимость понятия}, в которой, {\em во-первых},
тождественность обоих крайних членов представляет собой одни
и те же объем, содержание и всеобщность;
{\em во-вторых}, они
различаются по форме определений понятия, но так, что в силу этого
тождества она выступает как {\em голая
форма}. {\em В-третьих},
тождественная объективная всеобщность (в~качестве
рефлектированной в~себя в противоположность несущественной форме) выступает
поэтому как {\em содержание},
которое, однако, обладает в~себе самом определенностью
формы, "--- в одном отношении как простой определенностью
{\em рода}, в другом
отношении как той же определенностью, развитой в ее различие; взятая как
эта последняя, она есть особенность {\em видов} и их {\em тотальность},
всеобщность рода. "--- Особенность в ее развитии
образует {\em предикат}, так как она есть постольку {\em более всеобщее},
поскольку она содержит в~себе не только всю всеобщую сферу
субъекта, но и ее расчленение на особенности, ее обособление.

При ближайшем рассмотрении этого обособления мы убеждаемся,
{\em во-первых}, что род
образует собой субстанциальную всеобщность видов; поэтому субъект есть
{\em как} $B$, {\em так и} $C$. Это <<{\em как "--- так и}>>
обозначает положительное тождество особенного с всеобщим; это
объективное всеобщее сохраняется полностью в своей особенности. Виды,
{\em во-вторых}, {\em взаимно исключают друг друга;} $A$ есть
{\em или} $В$ {\em или} $С$; ибо они составляют {\em определенное
различие} всеобщей сферы. Это <<{\em или-или}>> есть их {\em отрицательное}
соотношение. Но в последнем они столь же тождественны, как и
в первом соотношении; род есть их {\em единство}, как {\em определенных}
особенных. "--- Если бы род был некоторой
абстрактной всеобщностью, как в суждениях наличного бытия, то виды тоже
следовало бы брать только как {\em разные} и
безразличные друг к другу; но он есть не такая внешняя всеобщность,
возникшая лишь через {\em сравнивание} и {\em отбрасывание},
а их имманентная и конкретная всеобщность. "---
Эмпирическое разделительное суждение не обладает
необходимостью: $A$ есть либо $B$, либо $C$, либо $D$ и~т.~д.
потому, что $B$, $C$, $D$ и~т.~д. были {\em преднайдены}: тут
нельзя, собственно говоря, высказать никакого <<{\em или-или}>>; ибо
такие виды образуют собой лишь как бы субъективную полноту;
{\em один} вид, правда, исключает {\em другой;} но <<{\em или-или}>>
исключает {\em всякий дальнейший} вид и замыкает в~себе некоторую
тотальную сферу. Эта тотальность имеет свою {\em необходимость} в
отрицательном единстве объективно-всеобщего, каковое последнее имеет внутри
себя единичность растворенной и имманентной себе в качестве простого
{\em принципа} различия посредством которого виды {\em определяются} и
{\em соотносятся}. Напротив, эмпирические виды находят свои различия в
какой-либо случайности, которая есть некоторый внешний принцип или (в~силу
этого) не {\em их} принцип и, стало быть, не имманентная определенность рода;
они поэтому по своей определенности и не соотнесены друг с другом. "---
Но именно благодаря {\em соотношению} своей
определенности виды образуют собой всеобщность предиката. "--- Так называемые
{\em контрарные} и {\em контрадикторные}
понятия должны были бы, собственно говоря, найти себе место
именно здесь; ибо в разделительном суждении положено существенное различие
понятия; но они вместе с тем имеют в нем также и свою истину, заключающуюся
в том, что сами контрарное и контрадикторное различаются между собой и в
контрарном и в контрадикторном смысле. Виды контрарны, поскольку они лишь
{\em разны}, а именно
имеют некоторое в-себе-и-для-себя-сущее устойчивое наличие через род, как
представляющий собой их объективную природу; они {\em контрадикторны},
поскольку они друг друга исключают. Но каждое из этих
определений, взятое само по себе, односторонне и лишено истины; в
<<{\em или-или}>>
разделительного суждения положено их единство, как их истина,
согласно которой указанное самостоятельное устойчивое наличие, как
{\em конкретная всеобщность}, само есть также и {\em принцип} того
отрицательного единства, в силу которого они взаимно исключают друг друга.

В силу только что вскрытого тождества субъекта и предиката по
их отрицательному единству, род определен в разделительном суждении как
{\em ближайший} род. Это
выражение указывает прежде всего на некоторое только количественное
различие, на {\em большее} или {\em меньшее}
число определений, которые некоторое всеобщее содержит в~себе
по сравнению с обнимаемой им особенностью. Согласно этому остается
случайным, чт\'{о} собственно служит ближайшим родом. Но поскольку род
понимается как такое всеобщее, которое образовано только путем отбрасывания
определений, он, собственно говоря, не может образовать разделительного
суждения; ибо в таком случае является делом случая, осталась ли еще в нем
та определенность, которая составляет принцип <<{\em или-или}>>; род был
бы вообще представлен в видах не по своей {\em определенности}, и
эти виды могли бы обладать лишь случайной полнотой. В~категорическом
суждении род имеет по отношению к субъекту ближайшим образом лишь эту
абстрактную форму и поэтому он не необходимо представляет собой ближайший
для субъекта род и постольку он внешен. Но когда род выступает как
конкретная, существенно {\em определенная}
всеобщность, то он, как простая определенность, есть единство
{\em моментов понятия},
которые в указанной простоте лишь сняты, но
имеют в видах свое реальное различие. Поэтому род постольку есть
{\em ближайший} род
некоторого вида, поскольку последний имеет свое специфическое различение в
существенной определенности первого и виды вообще имеют свое различенное
определение как принцип в природе рода.

Только что рассмотренный пункт составляет тождество субъекта и
предиката со стороны {\em определенности}
вообще "--- стороны, положенной гипотетическим
суждением, необходимость которого есть некоторое тождество непосредственных
и разных и поэтому выступает существенным образом как отрицательное
единство. Это-то отрицательное единство и отделяет вообще субъект и
предикат, но теперь оно само положено как различенное, в субъекте как
{\em простая} определенность, а в предикате как {\em тотальность}.
Упомянутое отделение субъекта и предиката есть {\em различие понятия;}
но и {\em тотальность видов} в предикате равным образом {\em не}~может быть
{\em каким-либо другим} различием. "--- Тем самым получается, стало быть,
{\em определение} терминов разделительного суждения по отношению друг к другу.
Это определение сводится к различию понятия, ибо лишь {\em понятие} разделяется
и в своем определении открывает свое отрицательное единство. Впрочем, здесь
вид принимается в соображение лишь согласно его простой определенности
понятия, а не по тому {\em образу}, который он получает, вступая из идеи
в дальнейшую самостоятельную {\em реальность;} этот образ, конечно,
{\em отпадает} в простом принципе рода; но {\em существенная}
диференциация должна быть моментом понятия. В~рассматриваемом здесь
суждении, собственно говоря, {\em положено} теперь через {\em собственное}
дальнейшее определение понятия самое его разделение, положено
то, что при рассмотрении понятия получилось как его в-себе-и-для-себя-сущее
определение, как его диференциация на определенные понятия. "---
Так как понятие теперь есть всеобщее, представляя собой как
положительную, так и отрицательную тотальность особенных, то оно
{\em само} именно вследствие этого непосредственно есть также и
{\em один из своих разделительных членов;} {\em другим} же
членом служит эта всеобщность, разложенная в {\em свою особенность},
или, иначе говоря, определенность понятия как {\em определенность}, в
которой именно всеобщность являет себя (sich darstellt) как
тотальность. "--- Если разделение какого-либо рода на виды не
достигло еще этой формы, то это служит доказательством того, что оно еще не
возвысилось до определенности понятия и выросло не из него. "---
<<{\em Цвет} бывает или фиолетовый, или темносиний, или голубой, или
зеленый, или желтый, или оранжевый, или красный>>\pagenote{Это "--- семь
основных цветов солнечного спектра согласно Ньютону (см.~{\em Ньютон},
Оптика, пер. с~прим. С.~И.~Вавилова, М.---Л. 1927).\label{bkm:bm40}};
такое разделение сразу же выдает свою, даже
эмпирическую, невыдержанность и нечистоту; рассматриваемое с этой стороны,
оно уже само по себе заслуживает того, чтобы назвать его варварским. Если
цвет постигают как {\em конкретное единство} светлого и
темного\pagenote{Гегель имеет в виду гетевское
учение о цветах, согласно которому <<для порождения цвета нужны свет и мрак,
светлое и темное, или, пользуясь более общей формулой, свет и не-свет>>
[{\em Goethes} Sämtliche Werke, Jubiläums-Ausgabe,
Stuttgart und Berlin, Cotta 1902, Bd.~40, S.~73).
Трактат Гете <<Zur Farbenlehre>>, откуда взята эта цитата,
вышел в 1810~г. В~<<Философии природы>> Гегель развивает теорию цветов,
близкую к гетевской. Об ошибочности этой теории цветов см. примечание
А.~А.~Максимова к стр.~249 <<Философии природы>> ({\em Гегель},
Соч., т.~II, М.---Л. 1934, стр.~601). В~<<Диалектике природы>> Энгельса
мы читаем: <<Гегель построил теорию света и
цветов из чистой мысли и при этом впадает в {\em грубейшую эмпирию} доморощенного
филистерского опыта (хотя, впрочем, с известным основанием,
так как этот пункт тогда еще не был выяснен), например, когда он выдвигает
против Ньютон смешивание красок, практикуемое живописцами>>
({\em Engels}, Anti-Duhring und Dialektik der Natur, M.--L. 1935,
S.~681).\label{bkm:bm41}}, то этот {\em род} имеет в~себе самом ту
{\em определенность}, которая образует {\em принцип} его
разделения на виды. Но из них
один\pagenote{В~немецкой тексте: <<Von diesen aber muss die
eine\ldots>> Под <<diesen>> Гегель, повидимому,
имеет в виду 1) род (die Gattung) в его простом единстве с
самим собой и 2) род в его расчлененности на виды, а под <<die eine>>
"--- род в первом из этих двух его
аспектов.\label{bkm:bm42}}
должен быть безоговорочно простым цветом, содержащим в~себе
противоположность уравновешенной, заключенной в его интенсивности и
подвергнутой в ней отрицанию; а на другой стороне должна явить себя
противоположность отношения между светлым и темным, к каковому отношению,
так как оно касается феномена природы, должна еще прибавиться безразличная
нейтральность
противоположности\pagenote{Согласно гете-гегелевской теории цветов к
двум основным противоположным цветам
"--- {\em желтому} (в~основе которого лежит светлое) и {\em синему}
(в~основе которого лежит темное) присоединяется еще третий
"--- {\em зеленый}, представляющий собой <<простою смесь, обыкновенную
нейтральность желтого и синего>> ({\em Гегель}, Философия природы,
М.---Л. 1934, стр.~266---267).\label{bkm:bm43}}.
"--- Принятие за виды таких смесей, как фиолетовый и оранжевый
цвет, а также степенн\'{ы}х различий вроде темносинего и голубого, может иметь
свое основание лишь в совершенно необдуманном способе рассуждения, который
даже для эмпиризма обнаруживает слишком мало
размышления\pagenote{Это опять выпад против
ньютоновской теории цветов, в которой фиолетовый, оранжевый, темносиний и
голубой цвета рассматриваются как самостоятельные, первоначальные цвета,
занимающие определенные места в спектре.\label{bkm:bm44}}.
"--- Впрочем, здесь не место распространяться о том, какие
различные и еще ближе определенные формы имеет разделение, смотря по тому,
совершается ли оно в стихии природы или в стихии духа.

Разделительное суждение имеет прежде всего в своем предикате
члены разделения; но оно в не меньшей мере и само разделено; его субъект и
предикат суть члены разделения; они суть моменты понятия, положенные в
своей определенности, но вместе с тем и как тождественные, "---
как тождественные ($\alpha $) в объективной всеобщности,
которая в субъекте выступает как простой
{\em род}, а в предикате
"--- как всеобщая сфера и как тотальность моментов понятия, и
($\beta $) в {\em отрицательном}
единстве, в развернутой связи необходимости, по каковой связи
{\em простая определенность}
субъекта расщепилась в
{\em различие видов} и
именно в этом расщеплении есть их существенное соотношение и
себетождество.

Это единство, связка этого суждения, в котором крайние члены в
силу их тождества слились воедино, есть, стало быть, само понятие, и
притом, {\em как положенное;}
простое суждение необходимости тем самым возвело себя в
{\em суждение понятия}.

\section[D. Суждение понятия]{D. Суждение понятия}

Умение высказывать
{\em суждения наличного бытия}:
<<роза красна>>, <<снег бел>> и так далее "--- вряд
ли будет кем-нибудь считаться обнаружением большой силы суждения.
{\em Суждения рефлексии}
суть больше
{\em предложения}, чем
суждения. В~суждении необходимости предмет, правда,
выступает в своей объективной всеобщности; однако лишь в суждении, которое
нам теперь предстоит рассмотреть,
{\em имеется соотношение предмета с
понятием}. В~этом суждении понятие положено в основание, и
так как оно находится в соотношении с предметом, то оно положено в
основание как некоторое
{\em долженствование},
которому реальность может соответствовать или не
соответствовать. "--- Поэтому лишь такое суждение впервые
содержит в~себе истинную оценку; предикаты
<<{\em хороший}>>,
<<{\em дурной}>>,
<<{\em истинный}>>,
<<{\em прекрасный}>>,
<<{\em правильный}>> и так
далее служат выражением того, что к вещи
{\em прикладывается масштаб} ее всеобщего
{\em понятия}, как
безоговорочно пред-положенного
{\em долженствования},
что она
{\em соответствует} ему
или не соответствует.

Суждение понятия получило название суждения
{\em модальности}, и его
рассматривают как содержащее в~себе ту форму, в какой отношение между
субъектом и предикатом выступает в некотором
{\em внешнем рассудке}, и
полагают, что оно касается ценности связки лишь в
{\em отношении к мышлению}.
Согласно этому взгляду
{\em проблематическое}
суждение состоит в признании утверждения или отрицания
{\em произвольным} или
{\em возможным},
{\em ассерторическое}
суждение состоит в признании утверждения или отрицания
{\em истинным}, т.~е.
{\em действительным},
{\em аподиктическое "--- }в
признании этого утверждения или отрицания
{\em необходимым}. "---
Легко усмотреть, почему так напрашивается мысль выйти при
этом суждении за пределы самого суждения и рассматривать его определение
как нечто только {\em субъективное}.
А именно это происходит потому, что здесь в суждении снова
выступает и входит в отношение к некоторой непосредственной
действительности понятие, субъективное. Однако нельзя смешивать это
субъективное с {\em внешней
рефлексией}, которая, конечно, тоже есть нечто субъективное,
но в другом смысле, чем само понятие; последнее, каким оно снова выступает
из разделительного суждения, есть скорее нечто противоположное простому
{\em виду и способу}.
Прежние суждения суть в этом смысле лишь нечто субъективное,
ибо они покоятся на некоторой абстракции и односторонности, в которой
понятие утратилось. Суждение понятия есть, наоборот, по сравнению с ними
объективное суждение и истина именно потому, что в его основании лежит
понятие, но не во внешней рефлексии или в
{\em соотношении} с
некоторым субъективным, т.~е. случайным
{\em мышлением}, а в
своей определенности как понятие.

В разделительном суждении понятие было положено как тождество
всеобщей природы с ее обособлением; тем самым здесь отношение суждения
сняло себя. Это {\em конкретное
единство} всеобщности и особенности есть пока что простой
результат; оно должно теперь развиться далее в тотальность, поскольку
содержащиеся в нем моменты пока что в нем исчезли и еще не противостоят
друг другу в определенной самостоятельности. "---
Недостаточность этого результата можно выразить определеннее
также и следующим образом: хотя в разделительном суждении объективная
{\em всеобщность}
достигла полноты в
{\em своем обособлении},
однако отрицательное единство последнего возвращается лишь
обратно в {\em эту объективную
всеобщность} и еще не определило себя к тому, чтобы быть
третьим, т.~е. {\em единичностью}. "---
Поскольку же самый результат есть
{\em отрицательное единство},
он, правда, есть уже эта
{\em единичность;} но,
взятый таким образом, он есть лишь эта
{\em одна}
определенность, которая теперь должна
{\em положить} свою
отрицательность, расщепиться на
{\em крайние термины} и
таким путем в конце концов развиться в
{\em умозаключение}.

Ближайшим расщеплением этого единства служит такое суждение, в
котором единство это положено, с одной стороны, как субъект, в виде
некоторого
{\em непосредственно-единичного},
а с другой стороны, как предикат, в виде определенного
соотношения моментов этого единства.

\subsection[а) Ассерторическое суждение]{а) Ассерторическое суждение}

Суждение понятия сперва
{\em непосредственно;}
взятое таким образом, оно есть
{\em ассерторическое}
суждение. Субъект есть некоторое конкретное единичное вообще,
предикат же выражает последнее как
{\em соотношение} его
{\em действительности},
определенности или
{\em характера} с его
{\em понятием} (<<этот дом
{\em плох}>>, <<это
действие {\em хорошо}>>).
Если рассматривать его детальнее, в нем содержится,
следовательно, (а), что субъект
{\em должен} быть чем-то;
его {\em всеобщая природа}
положила себя как самостоятельное понятие; (b)
{\em особенность},
которая не только в силу своей непосредственности, но и в
силу решительного различения ее от ее самостоятельной всеобщей природы
выступает как {\em характер}
(Beschaf\-fenheit) и
{\em внешнее существование;}
последнее в силу самостоятельности понятия с своей стороны
тоже безразлично к всеобщему и может как соответствовать, так и не
соответствовать ему. "--- Этот характер есть
{\em единичность},
которая лежит за пределами необходимого
{\em определения}
всеобщего в разделительном суждении, определения, которое
выступает лишь как обособление вида и как отрицательный
{\em принцип} рода.
Постольку конкретная всеобщность, возникшая из разделительного суждения, в
ассерторическом суждении раздвоена в форму таких
{\em крайних терминов},
которым еще недостает самого понятия как
{\em положенного}, их
соотносящего
единства\pagenote{Дело в том, что в разделительном суждении род содержит в~себе
принцип диференциации {\em только для видов}, а не для единичностей
(не для индивидов) Эти последние лежат еще {\em за пределами} того
процесса определения или диференциации, который имеет место
в <<объективной всеобщности>>,
составляющей содержание разделительного суждения. Поэтому, поскольку
ассерторическое суждение (с~его <<конкретной всеобщностью>>) по Гегелю
непосредственно вырастает из разделительного суждения, постольку в нем еще
нет необходимой внутренней связи между единичным (индивидом) и всеобщим
(понятием).\label{bkm:bm45}}.

Поэтому суждение пока что лишь {\em ассерторично; порукой его верности}
служит некоторое субъективное {\em уверение}. Что нечто
хорошо или дурно, правильно, соответственно или нет и так далее, это имеет
свою связь в некотором внешнем третьем. Но что эта связь
{\em положена внешним образом},
это означает то же самое, что она пока что есть лишь
{\em в~себе} или что она лишь {\em внутрення}. "---
Поэтому, когда говорят, что нечто хорошо или дурно и так
далее, то никто, конечно, не подумает, что оно, скажем, хорошо лишь в
{\em субъективном сознании},
а в~себе оно, быть может, дурно, или что хорошее и дурное,
правильное, соответственное и так далее не суть предикаты самих предметов.
Чисто субъективное в характере утверждения этого суждения состоит,
следовательно, в том, что {\em в-себе-}сущая связь субъекта и предиката еще
не~{\em положена}, или, что то же самое, что она лишь {\em внешня;} связка
есть пока что еще только некоторое непосредственное, {\em абстрактное бытие}.

Поэтому уверению ассерторического суждения противостоит с таким
же правом противоположное уверение. Если уверяют: <<это действие
хорошо>>, то противоположное уверение: <<это действие дурно>> имеет еще
одинаковую правомерность. "--- Или, иначе говоря, если будем
рассматривать это суждение {\em само по~себе}, то придем к заключению, что
так как субъект суждения есть {\em непосредственное единичное}, то он в этой
своей абстрактности еще не положил {\em в~самом себе} той {\em определенности},
которая содержала бы его соотношение с всеобщим понятием; таким образом,
для него еще является чем-то случайным как соответствие понятию, так и
несоответствие ему. Суждение поэтому по существу {\em проблематично}.

\subsection[b) Проблематическое суждение]{b) Проблематическое суждение}

{\em Проблематическое} суждение есть ассерторическое суждение, поскольку
последнее должно быть взято и как положительное и как отрицательное. "---
С этой качественной стороны {\em партикулярное}
суждение также проблематично, ибо оно значимо и как
положительное и как отрицательное; равным образом и в~{\em гипотетическом}
суждении бытие субъекта и предиката проблематично; через эту
же качественную сторону положено также, что сингулярное и категорическое
суждение есть еще нечто только субъективное. Но в проблематическом суждении
как таковом это полагание более имманентно, чем в упомянутых суждениях, так
как в этом суждении {\em содержанием предиката} служит {\em соотношение
субъекта с~понятием;} здесь, стало быть, {\em имеется налицо} самое
{\em определение непосредственного как чего-то случайного}.

\label{bkm:bm66a}В~первую очередь является проблематичным лишь
то, должен ли предикат быть связан с известным субъектом или не должен и
постольку неопределенность имеет место в связке. Для {\em предиката} не может
возникнуть отсюда никакого определения, ибо он уже есть объективная,
конкретная всеобщность. Проблематичность касается, следовательно,
непосредственности {\em субъекта}, которая в силу этого определяется как
{\em случайность}. "--- Но,
далее, это не значит, что следует отвлекаться от единичности субъекта;
очищенный вообще от последней, он был бы лишь некоторым всеобщим; предикат
как раз и подразумевает, что понятие субъекта должно быть положено в
соотношении с его единичностью. "--- Нельзя сказать: <<{\em дом} или
{\em некоторый дом} хорош>>, а следует прибавить: {\em <<смотря по тому,
каков его характер>>.} "--- Проблематичность субъекта в нем самом
составляет его {\em случайность} как {\em момент}, "--- составляет
{\em субъективность вещи}, противопоставляемую ее объективной природе
или ее понятию, простой {\em вид и способ} или {\em характер}.

Стало быть, сам {\em субъект} диференцируется на свою всеобщность или
объективную природу (на свое {\em долженствование})
и на особенный характер наличного бытия. Тем самым он
содержит в~себе {\em основание}, от которого зависит, {\em таков} ли он,
каким он {\em должен быть}. Таким путем он уравнивается
с~предикатом. "--- {\em Отрицательность} проблематического суждения,
поскольку она направлена против непосредственности {\em субъекта},
означает, согласно этому, лишь это первоначальное деление субъекта,
который {\em в~себе} уже выступает как единство всеобщего и особенного,
{\em на эти его моменты} "--- деление, которое и есть само суждение.

Можно сделать еще то замечание, что каждая из {\em двух} сторон
субъекта "--- его понятие и его характер "--- могла бы быть названа его
{\em субъективностью}. \label{bkm:bm01a}{\em Понятие}
есть ушедшая внутрь себя всеобщая сущность какой-нибудь вещи,
ее отрицательное единство с самой собой; последнее составляет ее
субъективность. Но вещь по существу также и {\em случайна} и имеет некоторый
{\em внешний характер;} последний также называется ее голой субъективностью
в~противоположность той объективности. Сама вещь именно и состоит в том, что
ее понятие, как отрицательное единство самого себя, отрицает свою
всеобщность и выносит себя во внешность
единичности. "--- В~качестве этого двоякого здесь положен {\em субъект}
суждения; указанные противоположные значения субъективности имеют бытие,
согласно их истине, в~одном и том же. "--- Значение субъективного само
стало проблематичным вследствие того, что субъективное {\em утратило} ту
непосредственную {\em определенность}, которую оно имело в непосредственном
суждении, и свою определенную {\em противоположность к
предикату}. "--- Вышеуказанные, встречающиеся также и
в рассуждениях обычной рефлексии, противоположные значения субъективного
могли бы уже сами по себе заставить обратить внимание по крайней мере на
то, что в каждом из них {\em в отдельности} нет истины. Двоякое значение
есть проявление того, что каждое значение, взятое отдельно, само по себе
односторонне.

\label{bkm:bm66b}Так как проблематичность положена, таким образом, как
проблематическое в~{\em вещи}, как вещь вместе с ее {\em характером}, то само
суждение уже больше не есть проблематическое суждение, но {\em аподиктично}.

\subsection[с) Аподиктическое суждение]{с) Аподиктическое суждение}

Субъект аподиктического суждения (<<дом, устроенный так-то и так-то,
{\em хорош}>>; <<поступок, носящий такой-то и такой-то {\em характер},
{\em справедлив}>>) имеет в нем [в~самом себе], {\em во-первых},
всеобщее "--- то, чем он {\em должен быть, во-вторых}, "--- свой
{\em характер;} последний содержит в~себе {\em основание}, почему
{\em весь субъект} обладает или не обладает некоторым предикатом суждения
понятия, т.~е. соответствует ли субъект или не соответствует своему
понятию. "--- Это суждение теперь {\em истинно} объективно или, иначе говоря,
оно есть {\em истина суждения} вообще. Субъект и предикат соответствуют друг
другу и имеют одно и то же содержание, и это {\em содержание} само есть
положенная {\em конкретная всеобщность;} а именно, оно содержит в~себе два
момента: объективное всеобщее или {\em род} и
оединиченное (das Verein\-zelnte). Здесь имеется, следовательно, такое
всеобщее, которое есть {\em оно само} и продолжается непрерывно через свою
{\em противоположность}, и лишь как {\em единство} с последнею оно и есть
всеобщее. "--- В основании такого всеобщего, как предикаты <<хороший>>,
<<соответственный>>, <<правильный>> и~т.~д., лежит некоторое
{\em долженствование}, и вместе с тем оно содержит в~себе {\em соответствие
наличного бытия;} не указанное долженствование или род сами по себе, а именно
это {\em соответствие} есть та {\em всеобщность}, которая образует собой
предикат аподиктического суждения.

{\em Субъект} равным образом содержит в~себе оба эти момента
в~{\em непосредственном} единстве как {\em вещь}. Но истина последней состоит
в~том, что она {\em надломлена} внутри себя на свое {\em долженствование}
и свое {\em бытие;} это есть {\em абсолютное суждение о всякой
действительности}. "--- То обстоятельство,
что это первоначальное разделение, которое представляет собой всемогущество
понятия, есть в такой же мере и возвращение в единство понятия и абсолютное
соотношение друг с другом долженствования и бытия, "--- это обстоятельство
и делает действительное {\em некоторой вещью;} ее внутреннее соотношение,
это конкретное тождество, составляет {\em душу} вещи.

Переход от непосредственной простоты вещи к {\em соответствию}, которое есть
{\em определенное} соотношение ее долженствования с ее бытием, "--- или, иначе
говоря, связка, "--- оказывается при более близком рассмотрении содержащимся
в~особенной {\em определенности} вещи. Род есть {\em в-себе-и-для-себя-сущее}
всеобщее, которое постольку представляется несоотнесенным; определенность же
есть то, что в этой всеобщности рефлектирует себя {\em в~себя}, но вместе с тем
и {\em в некоторое иное}. Суждение имеет поэтому свое {\em основание}
в~характере субъекта и благодаря этому {\em аподиктично}. Тем самым отныне
имеется налицо {\em определенная} и {\em наполненная связка}, связка, которая
раньше состояла в абстрактном <<{\em есть}>>, теперь же развилась далее
в~{\em основание} вообще. Она выступает прежде всего как {\em непосредственная}
определенность в субъекте, но есть равным образом и {\em соотношение}
с~предикатом, который не имеет никакого другого {\em содержания}, кроме
самог\'{о} этого {\em соответствия} или соотношения субъекта с~всеобщностью.

Таким образом, форма суждения исчезла, во-первых, потому, что
субъект и предикат суть {\em в~себе}
одно и то же содержание; во-вторых же, потому, что субъект
через свою определенность указывает дальше себя и соотносит себя
с~предикатом; но {\em это соотнесение} перешло вместе с тем, в-третьих,
в предикат, составляет лишь его содержание и есть, таким образом,
{\em положенное} соотношение или само суждение. "--- Таким образом,
конкретное тождество понятия, бывшее {\em результатом} разделительного суждения
и составляющее {\em внутреннюю} основу суждения понятия, теперь установлено
{\em в целом}, тогда как вначале оно было положено лишь в предикате.

Рассматривая ближе положительную сторону этого результата,
образующего переход суждения в некоторую другую форму, мы находим, что
субъект и предикат выступают в аподиктическом суждении, как мы видели,
каждый в отдельности как целое понятие. "--- {\em Единство} понятия,
как {\em определенность}, составляющая соотносящую их связку, вместе с тем
{\em отлично} от них. Ближайшим образом эта определенность стоит лишь
на другой стороне субъекта, как его {\em непосредственный характер}. Но так
как она есть по существу {\em соотносящее}, то она есть не только такой
непосредственный характер, но и {\em проходящее сквозь} субъект и предикат
и {\em всеобщее}. "--- В~то время как субъект и предикат имеют одно и то же
{\em содержание}, через эту определенность, напротив, положено
{\em соотношение по форме, определенность в виде некоторого всеобщего} или
{\em особенность}. "--- Таким образом она содержит в~себе оба формальных
определения крайних терминов и есть {\em определенное} соотношение субъекта
и предиката; она есть {\em наполненная или содержательная связка} суждения,
единство понятия, вновь выступившее из {\em суждения}, в крайних терминах
которого оно было утрачено. "--- \label{bkm:bm31b}{\em Через это наполнение
связки} суждение стало {\em умозаключением}\pagenote{Немецкое слово
<<Schluss>> можно переводить трояким образом: 1) <<умозаключение>>,
2) <<заключение>> и 3) <<силлогизм>>. В~настоящем переводе <<Schluss>>
чаще всего передается словом <<умозаключение>>, в отдельных
случаях "--- словом <<силлогизм>> (особенно когда речь идет
о~<<среднем термине силлогизма>>). Всюду пользоваться термином <<силлогизм>> для
перевода немецкого <<Schluss>> неудобно по той причине, что
в~русской философской литературе слово <<силлогизм>> употребляется в более
узком смысле умозаключения от общего к частному, между тем как у Гегеля
речь идет также и об индуктивных умозаключениях, умозаключениях по аналогии
и~т.~д. Что касается термина <<заключение>>, то его пришлось оставить для
перевода немецких терминов <<Schlusssatz>> и <<Konklusion>>, поскольку слово
<<вывод>> не всегда пригодно для передачи этих терминов и служит для
перевода слова <<Folgerung>>. Необходимость переводить <<Schluss>> через
<<умозаключение>> вызывается еще и тем, что глагол <<schliessen>> в~большинстве
случаев можно переводить только через <<умозаключать>>, так как перевод его
через <<заключать>> привел бы к шероховатостям и недоразумениям.

Необходимо, однако, отметить, что
русское слово <<умозаключение>> не вполне соответствует немецкому слову
<<Schluss>>, особенно в том значении этого последнего, которое
ему придает Гегель. Для Гегеля <<умозаключение>> (так же, как и <<понятие>>
и <<суждение>>) имеет прежде всего {\em объективное}
значение (объективное в смысле объективного и абсолютного
идеализма). Он рассматривает <<den Schluss>> не как нечто
такое, что имеет место в~<<уме>>, а как объективное соотношение моментов
самог\'{о} предмета или самог\'{о} понятия (это для Гегеля одно и то же).
Соответственно этому он толкует слово <<Schluss>> как <<Zusammen\-schliessen>>
(<<смыкание воедино>>, <<сключение>>).\label{bkm:bm46}}.

\chapter[Третья глава Умозаключение]{Третья глава\newline Умозаключение}

{\em Умозаключение} оказалось восстановлением {\em понятия} в~{\em суждении}
и, стало быть, единством и истиной их обоих. Понятие как таковое держит свои
моменты снятыми в~{\em единстве;} в~суждении это единство есть нечто внутреннее
или, что то же самое, нечто внешнее, и моменты, хотя и соотнесены, но положены
как {\em самостоятельные крайние термины}. В~{\em умозаключении} определения
понятия так же самостоятельны, как и крайние термины суждения, а вместе с тем
положено и их определенное {\em единство}.

Умозаключение есть, таким образом, полностью положенное понятие; оно поэтому
есть разумное (das Vernünftige). "--- Рассудок признается способностью
{\em определенного} понятия, которое фиксируется {\em для себя} абстракцией
и формой всеобщности. В~разуме же {\em определенные} понятия положены в их
{\em тотальности} и {\em единстве}. Поэтому не только умозаключение есть
разумное, но {\em все разумное есть некоторое умозаключение}. Деятельность
умозаключения издавна приписывается разуму; но, с другой стороны, о~разуме
самом по себе и о~разумных основоположениях и законах говорится так, что
не~видно, как связаны между собой тот вышеупомянутый разум, который
умозаключает, и этот последний разум, служащий источником законов и прочих
вечных истин и абсолютных мыслей. Если признавать, что первый есть лишь
формальный разум, а второй порождает содержание, то именно согласно этому
различению, второму разуму не может недоставать как раз {\em формы} разума,
умозаключения. Тем не менее их обыкновенно держат столь изолированно друг
от друга и, говоря об одном разуме, до такой степени не упоминают
о~другом, что кажется, что разум абсолютных мыслей как бы
стыдится разума умозаключения и что если умозаключение и приводится тоже
как деятельность разума, то это делается почти что только по традиции. Но
очевидно, как мы только что заметили, что если рассматривать логический разум
как {\em формальный}, то должно быть по существу возможно распознать его и
в~разуме, имеющем дело с некоторым содержанием; даже больше того: всякое
содержание может быть разумным лишь через разумную форму. Обратиться здесь
за разъяснением к весьма обыденной болтовне о разуме нельзя потому, что она
воздерживается от указания, что же следует понимать под {\em разумом;} это
претендующее на разумность познание\pagenote{Под этим <<претендующим на
разумность познанием>> (так же как и под <<обыденной болтовней о разуме>> в
предыдущем предложении) имеется в виду <<философия веры>> Фридриха-Генриха
Якоби (1743---1819), центральная мысль которой заключалась в метафизическом
противопоставлении {\em рассудочному} знанию знания непосредственного,
иррационального, мистического, не допускающего обоснования и доказательств.
Это непосредственное иррациональное знание Якоби обозначал терминами <<вера>>,
<<разум>>, <<чувство>>, <<духовное чутье>>, <<откровение>>.\label{bkm:bm47}}
большей частью так занято своими предметами, что забывает
познать самый разум и различает и обозначает его лишь посредством тех
предметов, которыми, как оно уверяет, разум обладает. Если разум есть, как
утверждают, познание, знающее о боге, свободе, справедливости, долге, о
бесконечном, безусловном, сверхчувственном, или хотя бы познание,
сообщающее обо всем этом представления и чувства, то следует сказать, что
отчасти эти его предметы суть лишь отрицательные предметы, отчасти же
вообще остается нерешенным первый вопрос, т.~е. вопрос о том, чт\'{о} во всех
этих предметах имеется такого, в силу чего они разумны? "--- А~в них имеется
то, что бесконечное есть в них не пустая абстракция от конечного, не
бессодержательная и неопределенная всеобщность, а наполненная всеобщность,
понятие, которое {\em определено} и содержит в~себе свою определенность таким
истинным образом, что оно различает себя внутри себя и выступает как единство
этих своих рассудочных и определенных различий. Лишь таким путем разум
{\em поднимается} над конечным, условным, чувственным или как бы его ни
определяли иначе, и в~этой отрицательности он по существу {\em содержателен},
ибо он есть единство как единство определенных крайних терминов; но понимаемое
таким образом, {\em разумное} есть лишь {\em умозаключение}.

Вначале умозаключение, как и суждение, {\em непосредственно;} поэтому его
определения (termini) суть {\em простые}, {\em абстрактные} определенности;
оно, таким образом, есть {\em умозаключение рассудка}. Если не идти дальше
этого его вида, то разумность в нем (хотя она наличествует здесь и положена),
конечно, еще мало заметна. Существенным в нем служит {\em единство} крайних
терминов, связующий их {\em средний термин} и поддерживающее их
{\em основание}. Абстракция, поскольку она фиксирует {\em самостоятельность}
крайних терминов, противополагает им это {\em единство} как некоторую столь же
неподвижную, {\em особо сущую} определенность и таким образом понимает
указанное единство скорее как {\em не-единство}, чем как единство. Выражение:
<<{\em средний термин}>> (medius terminus) заимствовано из области
пространственных представлений и со своей стороны способствует тому, чтобы не
идти дальше {\em внеположности} определений. Но если умозаключение состоит в
том, что в~нем {\em единство крайних терминов положено}, а между тем это
единство безоговорочно понимается, с одной стороны, как некоторое особенное,
существующее само по себе, а с другой стороны, как лишь внешнее соотношение,
так что существенным отношением умозаключения делается {\em не-единство}, "---
то разум, каковым является это умозаключение, еще не дает нам разумности.

{\em Во-первых, умозаключение наличного бытия},
в котором определения определены столь непосредственно и
абстрактно, обнаруживает в самом себе (так как оно, подобно суждению, есть
их {\em соотношение}), что они суть не такие абстрактные определения, а каждое
из них содержит в~себе {\em соотношение с~другим}, и что средний термин не
только содержит в~себе особенность в противоположность определениям крайних
терминов, но и содержит эти последние {\em положенными} в~нем самом.

Через эту свою диалектику оно делает себя
{\em умозаключением рефлексии},
{\em вторым}
умозаключением "--- с такими определениями, в
каждом из которых по существу
{\em светится другое}
определение, или, иначе говоря, которые положены как
{\em опосредствованные},
какими они и должны вообще быть согласно природе
умозаключения.

{\em В-третьих}, так как
это {\em свечение} или
эта опосредствованность рефлектируется в~себя само, то умозаключение
определено как {\em умозаключение
необходимости}, в котором опосредствующим служит объективная
природа вещи. Так как это умозаключение определяет крайние термины понятия
вместе с тем и как тотальности, то
{\em умозаключение}
достигло соответствия между своим понятием (или средним
термином) и своим наличным бытием (или крайними различиями), достигло своей
истины и тем самым перешло из субъективности в
{\em объективность}.

\section[А. Умозаключение наличного бытия]{А. Умозаключение наличного бытия}

1. Умозаключение, каково оно
{\em непосредственно},
имеет своими моментами определения понятия как
{\em непосредственные}.
Они суть, стало быть, абстрактные определенности формы,
которые еще не развились через опосредствование до
{\em конкретности}, но
суть лишь {\em единичные}
определенности. Первое умозаключение поэтому есть,
собственно говоря,
{\em формальное}
умозаключение.
{\em Формализм} процесса
умозаключения состоит в том, что не идут дальше определений этого первого
умозаключения. Понятие, расщепленное на свои
{\em абстрактные}
моменты, имеет своими крайними терминами
{\em единичность} и
{\em всеобщность}, а само
оно выступает как стоящая между ними
{\em особенность}. В~силу
своей непосредственности они, как соотносящиеся лишь с собой
определенности, суть все вместе некоторое
{\em единичное содержание}.
Особенность образует средний термин ближайшим образом
постольку, поскольку она
{\em непосредственно}
соединяет в~себе оба момента, единичность и всеобщность. В~силу своей
определенности она, с одной стороны, подчинена всеобщему, а, с
другой стороны, то единичное, по отношению к которому она обладает
всеобщностью, подчинено ей. Но эта {\em сращенность} или
{\em конкретность}\pagenote{Гегель намекает на то, что латинское слово <<concretus>> происходит от глагола
<<concrescere>>, первоначальное значение которого "--- <<срастаться, сращиваться>>.\label{bkm:bm48}}
есть ближайшим образом лишь
{\em единая двусторонность;}
в силу той непосредственности, которая свойственна среднему
термину в непосредственном умозаключении, он выступает как
{\em простая}
определенность, и образуемое им
{\em опосредствование еще не положено}.
И вот, диалектическое движение умозаключения наличного бытия
состоит в том, чтобы опосредствование, которое одно только и составляет
умозаключение, было положено в моментах умозаключения.

\subsection[а) Первая фигура умозаключения]{а) Первая фигура умозаключения}

{\em Е "--- О "--- В}\pagenote{Т.~е. <<единичное "--- особенное "--- всеобщее>>. В~<<Малой логике>>
Гегель дает такой пример: <<Эта роза красна, красное есть цвет; роза,
следовательно, обладает цветом>> ({\em Гегель}, Соч , т.~I, стр.~291).\label{bkm:bm49}}
есть всеобщая схема определенного умозаключения. Единичность
смыкается через особенность со всеобщностью; единичное не непосредственно
всеобще, а через посредство особенности; точно так же и, наоборот, всеобщее
единично не непосредственно, но нисходит к единичности через особенность.
"--- Эти определения противостоят друг другу как
{\em крайние термины} и
едины в некотором {\em отличном}
от них третьем. Оба они суть определенности; в этом они
{\em тождественны;} эта
их всеобщая определенность есть
{\em особенность}. Но они
равным образом суть и {\em крайние
термины} как по отношению к последней, так и по отношению
друг к другу, ибо каждый из них выступает в своей
{\em непосредственной}
определенности.

Всеобщее значение этого умозаключения состоит в том, что
единичное, которое как таковое есть бесконечное соотношение с собой и,
стало быть, было бы лишь некоторым
{\em внутренним},
выступает благодаря особенности во вне, вступает в
{\em наличное бытие} как
во всеобщность, где оно уже больше не принадлежит лишь самому себе, но
находится во {\em внешней связи;}
обратно, так как единичное отделяется, уходит в свою
определенность как особенность, то оно в этом отделении есть
нечто конкретное, а как соотношение определенности с самой собой оно есть
некоторое {\em всеобщее},
соотносящееся с собой, и тем самым также и некоторое истинно
единичное; оно в крайнем термине всеобщности, выходя из внешности, уходит
{\em внутрь} себя. "---
Объективное значение умозаключения имеется в первом
умозаключении пока что лишь
{\em поверхностно}, так
как в нем определения еще не положены как то единство, которое составляет
сущность умозаключения. Умозаключение постольку еще есть нечто
субъективное, поскольку абстрактное значение, которым обладают его термины,
изолировано так не в~себе и для себя, а лишь в субъективном сознании. "---
Впрочем, отношение единичности, особенности и всеобщности
есть, как мы видели выше,
{\em необходимое} и
{\em существенное отношение}
определений умозаключения со стороны их
{\em формы;} недостаток
состоит не в этой определенности формы, а в том, что
{\em под этой формой}
каждое отдельное определение не становится вместе с тем
{\em более богатым} [по
содержанию]. "--- {\em Аристотель}
больше держался голого отношения присущности (Inhärenz),
излагая природу умозаключения следующим образом:
<<{\em если три определения, относятся
между собой так, что одно крайнее определение целиком содержится в среднем
определении, а это среднее определение целиком содержится в другом крайнем
определении, то оба эти крайние определения необходимо смыкаются в
умозаключение}>>\pagenote{Это "--- известное
место из <<Первой аналитики>> Аристотеля (в~т.~I академического Берлинского
издания 1831~г., под ред. И.~Беккера, стр.~25b, строки
32---35) в несколько вольном переводе Гегеля. Точнее это место гласит: <<Если
три термина так относятся друг к другу, что последний имеется во всем
среднем термине, а этот средний термин либо имеется, либо отсутствует во
всем первом, то в отношении крайних терминов необходимо имеет место полный
силлогизм>>.\label{bkm:bm50}}.
Здесь больше выражено лишь повторение
{\em одного и того же отношения}
принадлежности одного крайнего термина среднему, а этого в
свою очередь другому крайнему термину, нежели взаимная определенность трех
терминов по отношению друг к другу. "--- А так как
умозаключение покоится именно на вышеуказанной взаимной их определенности
по отношению друг к другу, то сразу же явствует, что другие отношения
терминов, образующие прочие фигуры, могут обладать значимостью как
умозаключения рассудка лишь постольку, поскольку они могут быть
{\em сведены} к этому
первоначальному отношению; это "--- не
{\em разные виды} фигур,
стоящие {\em рядом} с
{\em первой}, а, с одной
стороны, поскольку они должны быть правильными умозаключениями, они
покоятся лишь на той существенной форме умозаключения вообще, которой
служит первая фигура; с другой же стороны, поскольку они отклоняются от
нее, они суть видоизменения, в которые необходимо переходит эта первая
абстрактная форма, тем самым давая себе дальнейшее определение и определяя
себя к тотальности. Скоро мы увидим более детально, как обстоит дело с
этими фигурами.

{\em Е "--- О "--- В} есть, таким образом, всеобщая схема
умозаключения, взятого в его определенности. Единичное подчинено
особенному, а последнее всеобщему; поэтому и единичное тоже подчинено
всеобщему. Или, иначе говоря, единичному присуще особенное, особенному же
всеобщее; {\em поэтому}
последнее присуще также и единичному. Особенное, с одной
стороны, именно по отношению к всеобщему, есть субъект, по отношению же к
единичному оно есть предикат; или, иначе говоря, по отношению к всеобщему
оно есть единичное, по отношению же к единичному "--- всеобщее.
Так как в нем соединены обе определенности, то крайние термины смыкаются
через это их единство. Слово
<<{\em поэтому}>>
представляется имеющим место в
{\em субъекте} выводом,
получающимся, дескать, из
{\em субъективного}
уразумения отношения обеих
{\em непосредственных}
посылок. Так как субъективная рефлексия высказывает оба
соотношения среднего термина с крайними в виде отдельных и притом
непосредственных {\em суждений}
или {\em предложений},
то и заключение как
{\em опосредствованное}
соотношение, конечно, тоже есть отдельное
{\em предложение}, и
слово <<{\em поэтому}>> или
<<{\em следовательно}>>
служит выражением того, что оно опосредствовано. Но это
<<{\em поэтому}>> должно
рассматриваться не как некоторое внешнее этому предложению определение,
имеющее, дескать, свое основание и местопребывание лишь в субъективной
рефлексии, а, напротив, как обоснованное в природе самих крайних терминов,
{\em соотношение} которых
высказывается опять в виде {\em простого
суждения} или
{\em предложения} лишь
ради абстрагирующей рефлексии и через нее.
{\em Истинное} же их
{\em соотношение}
положено как средний термин. "--- Что
<<{\em следовательно Е есть В}>>
есть {\em суждение},
это "--- чисто субъективное обстоятельство;
умозаключение состоит именно в том, что это есть не просто
{\em суждение}, т.~е. не
соотношение, произведенное
{\em исключительно} через
связку или пустое <<{\em есть}>>,
а соотношение, осуществленное через определенный,
содержательный средний термин.

Поэтому если умозаключение рассматривается только как
состоящее {\em из трех суждений},
то это "--- формальный взгляд, не принимающий во
внимание того отношения между определениями, которое единственно и важно в
умозаключении. Вообще, лишь субъективная рефлексия разделяет соотношение
терминов на отдельные посылки и отличное от них заключение:

\begin{verse}
Все люди смертны.\\
Кай "--- человек.\\
Следовательно, он смертен.
\end{verse}

На вас сразу же нападет скука, как только вы услышите такое
умозаключение; это проистекает от той бесполезной формы, которая
посредством отдельных предложений создает некую видимость различия, тотчас
же исчезающую в самой вещи. Главным образом вследствие этой субъективной
формы процесс умозаключения представляется какой-то субъективной
{\em уловкой}, к которой разум или рассудок вынужден, дескать, прибегать
в тех случаях, когда они не могут познавать {\em непосредственно}.
"--- Но, конечно, природа вещей (разумное) действует не таким
образом, чтобы сперва устанавливалась некоторая б\'{о}льшая посылка
(соотношение некоторой особенности с некоторым существующим всеобщим), а
затем появлялось бы, во-вторых, некоторое отдельное соотношение некоторой
единичности с особенностью, откуда бы наконец, в-третьих, возникало бы
некоторое новое предложение. "--- \label{bkm:bm52a}Этот
движущийся через отдельные предложения процесс умозаключения есть
не~что иное, как некоторая субъективная форма; природа же дела (der Sache)
состоит в том, что различные понятийные определения вещи объединены в
существенном единстве. Эта разумность есть не уловка, а, напротив, по сравнению
с еще присущей {\em суждению непосредственностью} соотношения, {\em
объективное}, а та непосредственность познания есть, скорее, чисто
субъективное; умозаключение же есть, напротив, истина суждения. "--- Все вещи
суть {\em умозаключение}, некоторое всеобщее, сомкнутое через особенность с
единичностью; но, конечно, они не суть состоящее из {\em трех предложений}
целое.

2. В~{\em непосредственном}
умозаключении рассудка термины имеют форму
{\em непосредственных определений}.
С этой стороны, с которой они суть
{\em содержание}, мы и
должны теперь рассмотреть это умозаключение. Его можно постольку считать
{\em качественным}
умозаключением подобно тому, как в суждении наличного бытия
имеется тот же аспект качественного определения. Термины этого
умозаключения, подобно терминам упомянутого суждения, суть вследствие этого
{\em единичные}
определенности; ибо определенность положена через ее
соотношение с собой как безразличная к
{\em форме}, стало быть,
как содержание. {\em Единичное}
есть какой-либо непосредственный, конкретный предмет,
{\em особенность} "--- одно
из отдельных его определенностей, свойств или отношений,
{\em всеобщность} же есть
опять-таки некоторая еще более абстрактная, еще более отдельная
определенность в особенном. "--- Так как субъект, как некоторое
{\em непосредственно}
определенное, еще не положен в своем понятии, то его
конкретность еще не сведена к существенным определениям понятия; поэтому
его соотносящаяся с собой определенность есть
неопределенное, бесконечное
{\em многообразие}.
Единичное обладает в этой непосредственности бесконечным
множеством определенностей, принадлежащих к составу его особенности, каждая
из которых может поэтому образовать для данного единичного средний термин
умозаключения. Но через каждый другой средний термин оно смыкается с
{\em некоторым другим всеобщим;}
через каждое из своих свойств оно находится в некотором
другом соприкосновении, в некоторой другой связи наличного бытия. "---
Далее, и средний термин тоже есть нечто конкретное по
сравнению с всеобщим; он сам содержит в~себе многие предикаты, и данное
единичное можно через один и тот же средний термин смыкать опять-таки с
многими всеобщими. Поэтому вообще является
{\em совершенно случайным}
и {\em произвольным},
какое из многих свойств вещи мы берем, чтобы, исходя из него,
связать данную вещь с тем или иным предикатом: другие средние термины суть
переходы к другим предикатам, и даже один и тот же средний термин может сам
по себе быть переходом к разным предикатам, потому что он, как особенное,
содержит в~себе сравнительно со всеобщим многие определения.

Но дело не ограничивается тем, что для одного субъекта
одинаково возможно неопределенное множество умозаключений и что каждое
отдельное умозаключение по своему содержанию
{\em случайно}: эти
умозаключения, касающиеся одного и того же субъекта, должны перейти также и
в {\em противоречие}. Ибо
вообще различие, которое ближайшим образом представляет собой безразличную
{\em разность}, есть
столь же существенно и
{\em противоположение}.
Конкретное уже больше не есть нечто просто являющееся, а оно
конкретно через единство противоположностей в понятии
"--- противоположностей, определивших себя в моменты понятия.
Так как со стороны качественной природы терминов конкретное берется в
формальном умозаключении по какому-нибудь одному из присущих ему отдельных
определений, то умозаключение наделяет данное конкретное соответствующим
этому среднему термину предикатом; но так как, исходя из некоторой другой
стороны, умозаключают к противоположной определенности, то тем самым первое
заключение оказывается ложным, хотя сами по себе его посылки и вывод из них
совершенно правильны. "--- Если, исходя из среднего термина,
гласящего, что стена была выкрашена синей краской, умозаключают, что она,
стало быть, синяя, то это умозаключение правильно; но, несмотря на это
умозаключение, стена может быть зеленой, если она сверх того была покрыта
еще и желтой краской, причем из этого последнего обстоятельства, взятого
отдельно, равным образом вытекало бы заключение, что она
желтая. "--- Если от среднего термина <<чувственное существо>>
умозаключают, что человек не добр и не зол, так как о чувственном нельзя
высказать ни того, ни другого, то умозаключение правильно, а заключение
ложно; ибо человеку как конкретному в такой же мере присущ и средний термин
<<духовное существо>>. "--- Из среднего термина <<тяготение
планет, их спутников и комет к солнцу>> правильно следует, что эти тела
падают на солнце; но они не падают на него, так как они вместе с тем сами
по себе суть собственные центры тяготения или, как это говорится, потому,
что ими движет центробежная сила. "--- Подобным же образом из
среднего термина <<социальность>> можно вывести общность имущества граждан;
из среднего же термина <<индивидуальность>>, если проследить его столь же
абстрактно, вытекает распадение государства, что и последовало, например, в
германской империи, когда в ней придерживались последнего среднего
термина. "--- Справедливо считается, что ничто не является
столь недостаточным, как такого рода формальное умозаключение, ибо оно
покоится на случае или произвольном выборе того или иного среднего термина.
Как бы прекрасно ни протекала такая дедукция через ряд умозаключений, и как
бы ни убедительна была ее правильность, это все же еще ни к чему не
приводит, так как всегда остается возможным, что имеются другие средние
термины, из которых можно столь же правильно вывести нечто прямо
противоположное. "--- Кантовские
{\em антиномии} разума
состоят не в чем ином, как в том, что из какого-нибудь понятия в одном
случае берется и кладется в основание одно его определение, а в другом
случае "--- с такой же необходимостью другое. "---
Эту недостаточность и случайность того или иного
умозаключения не следует при этом сваливать исключительно на содержание,
как будто бы она ничуть не зависела от формы, между тем как логика
интересуется, дескать, лишь последней. Напротив, в самой форме формального
умозаключения лежит основание того, что содержанием оказывается столь
одностороннее качество; к этой односторонности содержание определено именно
вышеуказанной {\em абстрактной}
формой. Содержание есть одно из многих единичных качеств или
определений некоторого конкретного предмета или понятия именно потому, что
оно, как предполагается, есть {\em по
форме} не более, чем такая непосредственная, единичная
определенность. Крайний термин <<единичность>>, как
{\em абстрактная единичность},
есть
{\em непосредственное}
конкретное и поэтому бесконечно или неопределимо
многообразное; средний термин есть столь же
{\em абстрактная
}{\em особенность} и,
следовательно, представляет собой одно
{\em единичное} из этих
многообразных качеств; и точно так же другой крайний термин есть
{\em абстрактное всеобщее}.
Поэтому формальное умозаключение в силу своей формы есть
существенным образом нечто совершенно случайное по своему содержанию, и
притом не в том смысле, что для умозаключения случайно, имеет ли оно дело с
таким-то или с каким-нибудь
{\em другим} предметом
"--- от этого содержания логика отвлекается, "---
но, поскольку в основании его лежит какой-либо субъект,
является случайным, {\em какие}
определения содержания будет относительно него выводить
умозаключение.

3. Определения умозаключения суть постольку определения
содержания, поскольку они суть непосредственные, абстрактные,
рефлектированные в~себя определения. Но существенным в них является,
напротив, то, что они не суть такие рефлектированные в~себя, безразличные
друг к другу определения, а суть
{\em определения формы;}
постольку они по существу суть
{\em соотношения}. Эти
соотношения суть, {\em во-первых},
соотношения крайних терминов со средним, соотношения, которые
непосредственны, proposi\-tiones prae\-missae (посылки), а
именно, отчасти соотношение особенного со всеобщим "--- propositio major
(б\'{о}льшая посылка), отчасти единичного с особенным "---
propositio minor (меньшая посылка).
{\em Во-вторых}, имеется
соотношение друг с другом крайних терминов, которое есть
{\em опосредствованное}
соотношение, "--- conclusio (заключение). Те
{\em непосредственные}
соотношения, посылки, суть предложения или суждения вообще и
{\em противоречат природе
умозаключения}, согласно которой различные определения
понятия не должны быть соотнесены непосредственно, а должно быть положено
также и их единство; истиной суждения служит умозаключение. Посылки тем
менее могут оставаться непосредственными соотношениями, что их содержанием
служат непосредственно {\em различные}
определения, и, стало быть, они сами по себе не
непосредственно тождественны, если только эти посылки не суть чисто
тождественные предложения, т.~е. пустые, ни к чему не приводящие
тавтологии.

Поэтому требование, предъявляемое посылкам, обычно гласит, что
они должны быть {\em доказаны},
т.~е. {\em тоже
представлены в виде заключений}. Две посылки требуют, таким
образом, двух дальнейших умозаключений. А~эти
{\em два} новых
умозаключения, вместе взятые, в свою очередь, дают
{\em четыре} посылки,
требующие {\em четырех}
новых умозаключений; в последних имеется
{\em восемь} посылок; в
обосновывающих их {\em восьми}
умозаключениях имеется
{\em шестнадцать}
посылок, для доказательства которых требуется
{\em шестнадцать}
умозаключений, и {\em так
далее} в геометрической прогрессии
{\em до бесконечности}.

Итак, здесь снова появляется
{\em прогресс в бесконечность},
который раньше встретился нам в низшей
{\em сфере бытия} и
которого нельзя уже было ожидать в области понятия (абсолютной рефлексии из
сферы конечного в~себя), в области свободной бесконечности и истины.
Трактуя о сфере бытия, мы показали, что в тех случаях, когда появляется
дурная бесконечность, сводящаяся к бесконечному прогрессу, налицо имеется
противоречие между некоторым
{\em качественным бытием}
и некоторым выходящим за его пределы
{\em бессильным долженствованием;}
сам же прогресс есть повторение предъявляемого к
качественному бытию требования единства и постоянного впадения обратно в
несоответственный этому требованию предел. В~формальном умозаключении
основой служит {\em непосредственное}
соотношение или качественное суждение, а даваемое
умозаключением {\em опосредствование}
является тем, что по сравнению с первым положено как более
высокая истина. Уходящее в бесконечность доказывание посылок не разрешает
указанного противоречия, а только постоянно возобновляет его и представляет
собой повторение одного и того же первоначального недостатка. "---
Истина бесконечного прогресса состоит, напротив, в том, чтобы
и сам он и уже определенная через него, как недостаточная, форма были
сняты. "--- Эта форма есть форма такого опосредствования, как
{\em Е "--- О "--- В}. Оба соотношения {\em Е "--- О} и {\em О
"--- В} должны быть опосредствованы; если это
происходит таким же самым путем, то неудовлетворительная форма {\em Е "--- О
"--- В} только удваивается и так далее до
бесконечности. {\em О} имеет относительно {\em Е} также и
формальное определение некоторого всеобщего, а по отношению к
$B$ "--- формальное определение некоторого
{\em единичного}, ибо эти
соотношения суть вообще суждения. Эти соотношения требуют поэтому
опосредствования, но в силу указанного вида последнего здесь появляется
лишь снова то отношение, которое должно быть снято.

Опосредствование должно поэтому совершиться другим путем. Для
опосредствования соотношения {\em О "--- В} имеется
{\em Е;} опосредствование должно поэтому принять вид

{\centering
{\em О "--- Е
"--- В}.
\par}

А для опосредствования соотношения {\em Е "--- О}
имеется $B$; это опосредствование становится поэтому умозаключением

{\centering
{\em Е "--- В
"--- О}.
\par}

При более детальном рассмотрении этого перехода согласно его
понятию оказывается, что, {\em во-первых},
опосредствование формального умозаключения со стороны его
{\em содержания}, как было показано выше, {\em случайно}.
Непосредственное {\em единичное} имеет в
лице своих определенностей неопределимое множество средних терминов, а
последние в свою очередь имеют столь же много определенностей вообще; так
что всецело от внешнего {\em произвола}
или вообще от некоторого внешнего {\em обстоятельства} и
случайного определения зависит то, с каким всеобщим следует смыкать субъект
умозаключения. Опосредствование не есть поэтому по своему содержанию ни
нечто необходимое, ни всеобщее; оно не имеет своего основания в
{\em понятии предмета; основанием}
умозаключения служит, напротив, то, что внешне в предмете,
т.~е. {\em непосредственное;}
но непосредственным служит среди определений понятия {\em единичное}.

Что касается формы, то {\em опосредствование} точно так же имеет своей
{\em предпосылкой непосредственность
соотношения;} опосредствование поэтому само опосредствовано
и притом через {\em непосредственное}, т.~е. через {\em единичное}. "--- Говоря
точнее, в силу {\em заключения}
первого умозаключения единичное стало опосредствующим.
Заключение есть {\em Е "--- В;} единичное здесь положено как
{\em всеобщее}. В~одной посылке, а именно, в меньшей ({\em Е "--- О}),
оно выступает уже как {\em особенное;} стало
быть, оно выступает как то, в чем соединены оба эти определения. "---
Или, иначе говоря, заключение, взятое само по себе, выражает
единичное как всеобщее, и притом не непосредственным образом, а через
опосредствование, "--- выражает, следовательно, как некоторое
необходимое соотношение. {\em Простая}
особенность была средним термином; в заключении эта
особенность {\em положена развернуто как
соотношение единичного и всеобщности}. Но всеобщее еще есть
пока что некоторая качественная определенность, предикат
{\em единичного;} когда единичное определено как всеобщее, оно
{\em положено} как
всеобщность крайних терминов или, иначе говоря, как средний термин; само по
себе, оно есть крайний термин единичности, но так как оно теперь определено
как всеобщее, то оно вместе с тем есть единство обоих крайних терминов.

\subsection[b) Вторая фигура]

{b) Вторая фигура: {\em О "--- E "--- B}\pagenote{Эта <<вторая фигура>>
умозаключения соответствует <<третьей фигуре>> Аристотеля,
точно так же как <<третья фигура>> Гегеля соответствует
<<второй фигуре>> Аристотеля. В~<<Малой логике>> Гегель пишет
формулу своей <<второй фигуры>> наоборот:
<<{\em B "--- Е "--- О}>> (см. {\em Гегель}, Соч., т.~I, стр.~293).
Такое начертание встречается и в <<Большой логике>> на стр.~137, 138 и
152. Дело в том, что для Гегеля основным и решающим в умозаключении
является именно средний термин как {\em опосредствующий}
крайние термины, тогда как расстановка крайних терминов
(какой из них стоит на первом месте и какой на последнем) не может служить
основанием для классификации силлогизмов.\label{bkm:bm51}}}

1. Истина первого качественного умозаключения состоит в том,
что нечто сомкнуто с некоторой качественной определенностью как со всеобщей
не само по себе, а через некоторую случайность или в некоторой единичности.
В такого рода качестве субъект умозаключения не возвратился в свое понятие,
а постигнут лишь в своей {\em внешности;}
непосредственность составляет основание соотношения и, стало
быть, опосредствование; постольку единичное есть поистине средний термин.

Но, далее, соотношение умозаключения есть {\em снятие}
непосредственности; заключение есть не непосредственное
соотношение, а соотношение, опосредствованное некоторым третьим; оно
поэтому содержит в~себе некоторое {\em отрицательное}
единство; поэтому опосредствование теперь определено так, что
оно содержит в~себе {\em отрицательный} момент.

В этом втором умозаключении посылками служат {\em О "--- E} и {\em Е "--- В;}
лишь первая из этих двух посылок есть еще непосредственная; вторая же
({\em Е "--- В}) уже есть опосредствованная, а именно, она опосредствована
первым умозаключением; второе умозаключение предполагает поэтому первое,
равно как и наоборот, первое предполагает второе. "--- Крайние
термины определены в этом втором умозаключении друг относительно друга как
особенное и всеобщее; последнее постольку сохраняет еще свое {\em место;}
оно есть предикат; но особенное переменило свое место, оно есть субъект,
или, иначе говоря, {\em положено под определением
крайнего термина единичности}, подобно тому как
{\em единичное} положено {\em с определением среднего термина}
или особенности. Поэтому оба уже не суть больше те
абстрактные непосредственности, которыми они были в первом умозаключении.
Однако они еще не положены как конкретные; в силу того обстоятельства, что
каждое из них находится на {\em месте} другого, оно положено в своем
собственном определении и вместе с тем "--- однако лишь
{\em внешним образом} "--- в {\em другом} определении.

{\em Определенный} и {\em объективный смысл}
этого умозаключения заключается в том, что всеобщее есть не
{\em само по себе} некоторое определенное особенное (ибо оно есть, наоборот,
тотальность своих особенных), а {\em такой-то} из его видов существует
{\em через единичность;} другие же из его видов исключены из него
непосредственной внешностью. С~другой стороны, особенное есть всеобщее
точно так же не непосредственно и само по себе, а так, что отрицательное
единство совлекает с него определенность и этим возводит его во
всеобщность. "--- Единичность относится к особенному {\em отрицательно}
постольку, поскольку она должна быть его предикатом; это
{\em не}~есть предикат особенного.

2. Но пока что термины еще суть непосредственные
определенности; они не достигли в своём развитии какого-либо объективного
значения; измененное {\em место},
полученное двумя из них, есть форма, которая пока что имеется
на них лишь внешним образом; они поэтому суть еще, как и в
первом умозаключении, вообще безразличное друг к другу содержание
"--- два качества, связанные друг с другом не сами по себе, а
через некоторую случайную единичность.

Умозаключение первой фигуры было {\em непосредственным}
или также постольку умозаключением, поскольку оно имеет бытие
в своем понятии как {\em абстрактная
форма}, которая еще не реализовала себя в своих
определениях. Так как эта чистая форма перешла в другую фигуру, то это
есть, с одной стороны, начинающаяся реализация понятия, поскольку в
первоначально непосредственной, качественной определенности терминов
полагается {\em отрицательный}
момент опосредствования и тем самым некоторая дальнейшая
определенность формы. "--- Но, вместе с тем, это есть
{\em становление} чистой формы умозаключения чем-то {\em иным}.
Умозаключение уже больше не соответствует ей полностью, и
положенная в его терминах определенность отличается от того первоначального
определения формы. "--- Поскольку умозаключение рассматривается
лишь как субъективное умозаключение, протекающее в некоторой внешней
рефлексии, оно признается некоторым {\em видом}
умозаключения, который должен соответствовать роду, а именно
всеобщей схеме {\em Е "--- О "--- В}. Но оно ближайшим
образом не соответствует этой схеме; две посылки рассматриваемого
умозаключения суть {\em О "--- Е} (или {\em Е "--- О}) и
{\em Е "--- В;} средний термин поэтому оба раза подчинен
или, иначе говоря, есть оба раза субъект, которому, следовательно, два
других термина присущи (inhärieren); следовательно, он не
есть такой средний термин, который один раз подчиняет или является
предикатом, а другой раз подчинен или является субъектом или, иначе говоря,
которому один крайний термин присущ, но который сам присущ другому крайнему
термину. "--- Истинный смысл того обстоятельства, что это
умозаключение не соответствует всеобщей форме умозаключения, заключается в
том, что последняя перешла в него, так как ее истина состоит в том, что она
есть субъективное, случайное смыкание. Если заключение во второй фигуре (не
беря в помощь имеющее быть упомянутым ограничение, которое делает его
чем-то неопределенным) правильно, то оно таково потому, что оно правильно
само по себе, а не потому, что оно есть заключение этого умозаключения. Но
то же самое имеет место и относительно заключения первой фигуры; эта-то его
истина и есть то, что положено второй фигурой. "--- То
воззрение, согласно которому вторая фигура есть лишь
{\em некоторое видоизменение}
[первой], упускает из вида необходимый переход первой фигуры
в эту вторую форму и не идет дальше первой фигуры как
истинной формы. Поэтому, поскольку во второй фигуре (которая, по старой
привычке, без всякого другого основания приводится как
{\em третья} фигура) также должно найти себе место {\em правильное} в этом
субъективном смысле умозаключение, то оно должно было бы быть
соответственным первому умозаключению, и, стало быть, так как одна посылка
({\em Е "--- В}) выражает отношение подчинения среднего термина одному из
крайних терминов, то другая посылка ({\em О "--- Е})
должна была бы получить отношение, противоположное тому,
которое она имеет, и {\em О} должно было бы быть подведено под
{\em Е}. Но такого рода отношение было бы упразднением определенного суждения
<<{\em Е} есть {\em О}>> и могло бы иметь
место лишь в неопределенном суждении "--- в некотором
партикулярном суждении. Поэтому заключение в этой фигуре может быть лишь
партикулярным. Но партикулярное суждение, как было замечено выше, столь же
положительно, сколь отрицательно. Поэтому заключение оказывается здесь
таким суждением, которому не может быть приписана большая ценность. "---
Так как, далее, особенное и всеобщее в этом умозаключении
суть крайние термины и непосредственные, безразличные относительно друг
друга определенности, то их отношение само безразлично. Можно по произволу
принимать одну или другую из этих определенностей за больший или меньший
термин, можно поэтому также принимать и ту и другую посылку за б\'{о}льшую или
меньшую\pagenote{В~этом абзаце Гегель имеет в виду
практикуемое в формальной логике <<сведение>> модусов третьей (а~равно и
второй) фигуры к модусам первой фигуры. Для иллюстрации возьмем
какой-нибудь тривиальный пример умозаключения третьей (по Гегелю
"--- второй) фигуры: <<птицы имеют когти; птицы суть двуногие
существа; следовательно, некоторые двуногие существа имеют когти>>. Средним
термином в этом силлогизме служат <<птицы>>; б\'{о}льшим термином служит
<<обладание когтями>>, а меньшим термином "--- <<двуногость>>. Для
сведения этого силлогизма к первой фигуре надо перевернуть меньшую посылку
(<<птицы суть двуногие существа>>) или, выражаясь языком школьной логики,
<<обратить ее посредством ограничения>>. Тогда силлогизм примет такой вид:
<<птицы имеют когти; некоторые двуногие существа суть птицы; следовательно,
некоторые двуногие существа имеют когти>>. Ввиду того что крайние термины
<<обладание когтями>> и <<двуногость>> находятся во внешнем, безразличном
отношении друг к другу, они могут меняться местами, и заключение может с
таким же правом гласить: <<некоторые снабженные когтями животные имеют две
ноги>>. Чтобы более наглядно выявить характер {\em единичности},
который по Гегелю присущ среднему термину рассматриваемой
фигуры, возьмем еще такой пример: <<Харьков лежит на 50-й параллели;
Харьков "--- большой город; следовательно, некоторые большие
города лежат на 50-й параллели, или: некоторые лежащие на 50-й параллели
города имеют большие размеры>>.

Необходимо, впрочем, отметить, что хотя Гегель и намекает здесь на
формально-логическое <<сведение>> одной фигуры к
другой, но сам он придает силлогистическим фигурам совершенно другой смысл,
чем какой они имеют в формальной логике. Для Гегеля суть дела состоит в
том, какое из трех <<определений понятия>> в том или ином случае служит
<<средним термином>>; т.~е. выполняет функцию опосредствования. Поэтому
приведенные нами примеры (так же как и пример в нижеследующем примечании
\ref{bkm:bm54}) иллюстрируют не гегелевское учение о
фигурах силлогизма, а только гегелевские намеки на формально-логическую
трактовку этих фигур. Гегель указывает, что те три обособленные
предложения, из которых конструируются школьные силлогизмы, представляют
собой лишь внешнюю, субъективную форму (см. в тексте,
стр.~\pageref{bkm:bm52a}). Сам он приводит такие примеры
истинного силлогизма и его трех фигур: 1) взаимоотношения между <<тремя
членами философской науки, т.~е. логической идеей, природой и духом>>
({\em Гегель}, Соч., т.~I, стр.~294---295), 2) взаимоотношения между членами
солнечной системы ({\em Гегель}, Соч., т.~II, стр.~135---136),
3) взаимоотношения между элементами государства ({\em Гегель},
Соч., т.~I, стр.~310; см. также ниже в тексте,
стр.~\pageref{bkm:bm52b} "--- \pageref{bkm:bm52c}) и~т.~д.\label{bkm:bm52}}.

3. Поскольку заключение столь же положительно, сколь и
отрицательно, оно есть некоторое безразличное к этим определенностям и тем
самым {\em всеобщее} соотношение. При более близком рассмотрении мы видим, что
опосредствование первого умозаключения было случайным {\em в~себе;} во втором
же умозаключении эта случайность {\em положена}. Она,
стало быть, есть снимающее само себя опосредствование; опосредствование
имеет определение единичности и непосредственности; а то, что смыкается
через это умозаключение, должно, напротив, быть тождественным
{\em в~себе} и {\em непосредственно}, ибо наш средний термин,
{\em непосредственная единичность},
есть бесконечно многообразная и внешняя определенность.
В~нем, следовательно, положено, напротив, {\em внешнее} себе
опосредствование. Но внешнее единичности есть всеобщность; указанное
опосредствование через непосредственное единичное отсылает дальше самого
себя к {\em другому для него}
опосредствованию, которое, стало быть, происходит через
{\em всеобщее}. "--- Или,
иначе говоря, то, что, как предполагалось, должно быть соединено через
вторую фигуру, на деле должно быть сомкнуто {\em непосредственно;}
через ту {\em непосредственность},
которая лежит в его основании, определенное смыкание не может
быть осуществлено. Та непосредственность, на которую оно указывает, есть
другая непосредственность по отношению к его непосредственности, "---
снятая первая непосредственность бытия, следовательно,
рефлектированная в~себя или, иначе говоря,
{\em в-себе-сущая}
непосредственность, "---
{\em абстрактное всеобщее}.

Переход этого умозаключения представлял собой с
рассматриваемой стороны, подобно переходу бытия, {\em иностановление},
так как в его основании лежит качественное, а именно
непосредственная единичность. Но согласно понятию единичность смыкает
особенное и всеобщее постольку, поскольку она
{\em снимает определенность}
особенного, что представляется как случайность этого
умозаключения. Крайние термины смыкаются не через то их определенное
соотношение, которое они имеют со средним термином. Средний термин поэтому
{\em не}~есть их
{\em определенное единство},
и то положительное единство, которое ему еще присуще, есть
лишь {\em абстрактная всеобщность}.
Но когда средний термин полагается в этом определении,
которое есть его истина, то это есть уже некоторая другая форма
умозаключения.

\subsection[с) Третья фигура]

{с) Третья фигура: {\em Е "--- В "--- O}\pagenote{В~<<Малой логике>> Гегель
пишет формулу своей <<третьей фигуры>> наоборот: {\em О "--- В "--- Е}
(см. {\em Гегель}, Соч., т.~I, стр.~294, а также выше,
примечание \ref{bkm:bm51}). Этой последней формулой пользуется
Маркс при характеристике товарно-денежного обращения. Маркс пишет: в
процессе обращения <<Т "--- Д "--- Т оба крайние
члена Т находятся, под углом зрения формы, не в одинаковом отношении к Д.
Первый Т относится к деньгам как особенный товар к всеобщему товару, между
тем как деньги относятся ко второму Т как всеобщий товар к единичному
товару. Следовательно, абстрактно-логически Т "--- Д "--- Т
может быть сведено к форме силлогизма О "--- В "--- Е,
где особенность образует первый крайний член,
всеобщность "--- связывающий средний член и
единичность "--- последний крайний член>>
({\em Маркс}, К критике политической экономии, Партиздат, 1935, стр.~98).\label{bkm:bm53}}}

1. Это третье умозаключение уже не имеет ни одной
непосредственной посылки; соотношение {\em Е "--- В}
опосредствовано первым умозаключением, а соотношение
{\em О "--- В} "--- вторым. Это умозаключение предполагает поэтому первые два
умозаключения; но и обратно, эти два умозаключения предполагают его, равно
как и вообще каждое умозаключение предполагает два остальных. В~этом
умозаключении, стало быть, завершено вообще определение умозаключения.
"--- Это взаимное опосредствование именно и означает, что каждое
умозаключение, хотя оно само по себе есть опосредствование, вместе с тем не
является в самом себе тотальностью опосредствования, а содержит в~себе
такую непосредственность, опосредствование которой находится вне его.

Умозаключение {\em Е "--- В "--- О}, рассматриваемое в нем
самом, есть истина формального умозаключения, оно выражает собой то, что
опосредствование последнего есть абстрактно всеобщее опосредствование и что
крайние термины содержатся в среднем термине не со стороны своей
существенной определенности, а лишь со стороны своей всеобщности, что,
следовательно, в нем как раз не сомкнуто то, что должно было быть
опосредствовано. Здесь, следовательно, положено то, в чем состоит формализм
такого умозаключения, термины которого имеют
непосредственное, безразличное к форме содержание или, что то же самое,
суть такие определения формы, которые еще не рефлектировали себя так, чтобы
стать определениями содержания.

2. Средний термин этого умозаключения есть, правда, единство
крайних терминов, но такое единство, в котором мы отвлекаемся от их
определенности, {\em неопределенное}
всеобщее. Однако, поскольку это всеобщее вместе с тем как
абстрактное отлично от крайних терминов, как от
{\em определенного}, оно
и само также еще есть нечто
{\em определенное} по
отношению к ним, и целое есть умозаключение, отношение которого к его
понятию надлежит рассмотреть. Средний термин как всеобщее есть по отношению
к {\em обоим} своим
крайним терминам подчиняющее или предикат; ни разу он не является
подчиненным или субъектом. Поэтому, поскольку эта фигура, как
{\em некоторая разновидность}
умозаключения, должна соответствовать требованиям последнего,
это может произойти лишь таким способом, что, поскольку одно соотношение
({\em Е "--- В})
уже имеет требуемую форму отношения, такая же форма
сообщается также и другому соотношению
({\em В "--- O}). А~это
совершается в таком суждении, где отношение между субъектом и предикатом
оказывается безразличным, т.~е. в
{\em отрицательном}
суждении. Таким путем умозаключение становится законным; но
заключение в нем необходимым образом
отрицательно\pagenote{Опять намек на практикуемое в
школьной логике <<сведение>> модусов второй (по Гегелю третьей) фигуры к
модусам первой фигуры (ср. примечание \ref{bkm:bm52}).
Возьмем тривиальный пример: <<рыбы не имеют легких; киты имеют легкие;
следовательно, киты не суть рыбы>>. Для
сведения этого силлогизма к силлогизму первой фигуры нужно перевернуть
большую посылку. Тогда мы получим: <<животные, обладающие легкими, не суть
рыбы; киты обладают легкими; следовательно, киты не рыбы>>.
В~рассматриваемой фигуре заключение всегда имеет форму отрицательного
суждения. Поэтому в нем всегда можно сделать <<обращение>>: субъект поставить
на место предиката, а предикат "--- на место субъекта. Вместо
<<киты не суть рыбы>> получим: <<рыбы не суть киты>>. Об этом безразличном
отношении между субъектом и предикатом {\em заключения} Гегель и говорит
в следующей фразе текста.\label{bkm:bm54}}.

Тем самым теперь оказывается безразличным также и то, какое из
двух определений этого предложения будет считаться предикатом и какое
субъектом, а в умозаключении является безразличным, будет ли то или иное из
этих двух определений считаться крайним термином единичности или крайним
термином особенности, другими словами, будет ли оно считаться меньшим или
б\'{о}льшим термином. Так как, согласно обычному предположению, от этого
зависит, какая из посылок должна быть большей посылкой и какая меньшей, то
и это здесь стало безразличным. "--- Это безразличие составляет
основание обычной {\em четвертой фигуры} умозаключения,
которой Аристотель не знал и которая,
в конце концов, касается совершенно пустого, неинтересного различия.
Непосредственное положение терминов является в четвертой фигуре
{\em обратным} положению
терминов в первой фигуре; так как субъект и предикат отрицательного
заключения, согласно формальному рассмотрению суждения, не имеют между
собой определенного отношения субъекта и предиката, а каждый из них может
занять место другого, то безразлично, какой термин будет браться как
субъект и какой как предикат; поэтому столь же безразлично, какую посылку
будем брать как б\'{о}льшую посылку и какую как меньшую. "---
Это безразличие, которому способствует также и определение
партикулярности (особенно, поскольку делают замечание, что оно может быть
взято в широком смысле), делает указанную четвертую фигуру чем-то
совершенно праздным.

3. Объективное значение умозаключения, в котором всеобщее
составляет средний термин, заключается в том, что опосредствующее, как
единство крайних терминов, есть
{\em существенным образом всеобщее}.
Но так как эта всеобщность есть ближайшим образом лишь
качественная или абстрактная всеобщность, то в ней не содержится
определенность крайних терминов; их смыкание, поскольку оно имеет место,
должно точно так же иметь свое основание в некотором лежащем вне этого
умозаключения опосредствовании и по отношению к этому умозаключению
совершенно так же случайно, как и в предыдущих формах умозаключения. Однако
так как всеобщее определено теперь как средний термин и в последнем
определенность крайних терминов не содержится, то эта определенность
крайних терминов положена как совершенно безразличная и внешняя. "---
Тем самым (ближайшим образом, согласно этой голой
абстракции) действительно возникла
{\em четвертая фигура}
умозаключения, а именно, фигура умозаключения,
{\em лишенного отношения}:
{\em В "--- В
"--- В}, "--- такого умозаключения, которое
абстрагирует от качественного различия терминов и тем самым имеет своим
определением чисто внешнее единство их, а именно, их
{\em равенство}.

\subsection[d) Четвертая фигура]

{d) Четвертая фигура: {\em В "--- В "--- В}, или математическое умозаключение}
\label{bkm:bm110a}
1. Математическое умозаключение гласит:
<<{\em если две вещи или два определения
равны третьему, то они равны между собой}>>. "--- Отношение
присущности или подчинения терминов здесь стерто.

Опосредствующим служит некоторое
{\em третье} вообще. Но
оно не имеет решительно никакого определения по отношению к своим крайним
терминам. Поэтому каждый из трех терминов может с одинаковым правом быть
третьим опосредствующим. Какой из них будет для этого употребляться, какие
из трех соотношений будут соответственно с этим браться как
непосредственные и какое "--- как опосредствованное, "---
это зависит от внешних обстоятельств и прочих условий, а
именно, от того, какие два соотношения из этих трех суть непосредственно
{\em данные}. Но это
определение не касается самого умозаключения и полностью внешне.

2. Математическое умозаключение признается в математике
{\em аксиомой}, т.~е.
{\em само по себе очевидным},
{\em первым} предложением,
не могущим быть доказанным и не нуждающимся ни в каком доказательстве,
т.~е. ни в каком опосредствовании, не предполагающим ничего другого и не
могущим быть выведенным из другого. "--- Если мы рассмотрим
ближе преимущество этого умозаключения, заключающееся в его
непосредственной {\em очевидности},
то окажется, что преимущество это состоит в формальном
характере этого умозаключения, абстрагирующего от всякой качественной
разности определений и имеющего дело только с их количественным равенством
или неравенством. Но по тому же самому основанию оно не обходится без
предпосылки или, иначе говоря, не остается неопосредствованным;
количественное определение, которое одно только и принимается в нем во
внимание, получается лишь
{\em посредством абстрагирования}
от качественного различия и от определений понятия. "---
Линии, фигуры, приравниваемые друг к другу, берутся лишь со
стороны их величины; какой-нибудь треугольник приравнивается к
какому-нибудь квадрату, но не как треугольник к квадрату, а исключительно
только по своей величине и~т.~д. Точно так же и понятие с его определениями
не входит в этот процесс умозаключения. Здесь вообще не
{\em постигают в понятии},
и рассудок не имеет перед собой даже формальных абстрактных
определений понятия. Очевидность этого умозаключения основана поэтому лишь
на том, что оно столь бедно определениями мысли и столь абстрактно.

3. Но {\em результатом
умозаключения наличного бытия} оказывается не только эта
абстракция от всякой определенности понятия. Получающаяся из него
{\em отрицательность}
непосредственных, абстрактных определений имеет еще другую,
{\em положительную сторону},
а именно ту, что в абстрактную определенность
{\em положено её иное},
и она в силу этого стала
{\em конкретной}.

Во-первых, все умозаключения наличного бытия имеют друг друга
{\em предпосылкой}, и
сомкнутые в заключении крайние термины лишь постольку сомкнуты поистине в
себе и для себя, поскольку они {\em помимо этого}
соединены тождеством, имеющим свое основание в чем-то другом;
средний термин, каков он в рассмотренных умозаключениях, {\em должен} быть
их единством понятия, но на самом деле он есть лишь некоторая формальная
определенность, которая не положена как их конкретное единство. Но эта
{\em предпосылка} каждого из указанных опосредствований есть не только
некоторая {\em данная непосредственность}
вообще (как в математическом умозаключении), а она сама есть
некоторое опосредствование, а именно, для каждого из двух других
умозаключений. Следовательно, что имеется поистине, это опосредствование,
основанное не на некоторой данной непосредственности, а на
опосредствовании. Это есть, стало быть, не количественное опосредствование,
абстрагирующее от формы опосредствования, а, напротив,
{\em соотносящееся с опосредствованием
опосредствование}, или, иначе говоря, {\em опосредствование рефлексии}.
Круг взаимного предполагания, образуемый взаимной связью этих
умозаключений, представляет собой возвращение этого предполагания внутрь
самого себя, которое в этом возвращении в~себя образует некоторую
тотальность и имеет то {\em иное},
на которое указывает каждое отдельное умозаключение, не
во-вне посредством абстракции, а охватывает его {\em внутри} круга.

Далее, со стороны {\em отдельных определений формы}
оказалось, что в этом целом формальных умозаключений каждое
отдельное определение занимало {\em место среднего термина}.
Непосредственно этот средний термин был определен как {\em особенность;} в
дальнейшем он определил себя посредством диалектического движения как
{\em единичность} и {\em всеобщность}.
И~точно так же каждое из этих последних определений проходило через
{\em места двух крайних терминов}. {\em Чисто отрицательным результатом}
служит здесь стирание качественных определений формы в чисто
количественном математическом умозаключении. Но что здесь поистине имеется
налицо, это "--- {\em положительный
результат}, состоящий в том, что опосредствование
совершается не через одну {\em отдельную}
качественную определенность формы, а через
{\em конкретное} их {\em тождество}.
Недостаток и формализм рассмотренных трех фигур умозаключения
состоит именно в том, что такого рода отдельная определенность должна
составлять в них средний термин. "--- Опосредствование
определило себя, следовательно, как безразличие непосредственных или
абстрактных определений формы и как положительную
{\em рефлексию} одного из них в другое. Непосредственное
умозаключение наличного бытия тем самым перешло в
{\em умозаключение рефлексии}.\label{bkm:bm110b}

\hegremark[Примечание]%
{Обычный взгляд на умозаключение}%
{[Обычный взгляд на умозаключение]}

В данном здесь изложении природы умозаключения и его различных
форм мы мимоходом обратили внимание также и на то, что в обычном
рассмотрении и трактовке умозаключений составляет главный
интерес, а именно, каким образом в каждой фигуре можно сделать правильное
умозаключение. Однако при этом мы указали лишь главный момент и обошли те
случаи и те комбинации, которые здесь возникают, когда сюда дополнительно
привлекается к рассмотрению различие положительных и отрицательных суждений
вместе с количественным определением и, в особенности, определение
партикулярности. "--- Здесь будут уместны еще некоторые
замечания относительно обычного взгляда на умозаключение и способа его
трактовки в логике. "--- Как известно, учение об умозаключениях
было развито до такой степени подробно, что его так называемые тонкости
сделались предметом всеобщего недовольства и отвращения. Когда
{\em естественный рассудок}
восстал во всех областях духовной культуры против лишенных
субстанциальности форм рефлексии, он обратился также и против указанного
искусственного учения о формах разума и считал, что он может обойтись без
такой науки, на том основании, что он уже сам собой совершает от природы,
без всякого особого изучения, рассматриваемые в этой науке отдельные
операции мышления. И~в самом деле, с человеком дело обстояло бы касательно
разумного мышления очень плохо, если бы условием такого мышления было бы
тягостное изучение формул умозаключения, "--- обстояло бы столь
же плохо, как с ним обстояло бы (мы на это уже указали в
предисловии)\pagenote{См. кн.~I <<Науки логики>>, стр.~9.},
если бы он не мог ходить и переваривать пищу, не изучив
предварительно анатомии и физиологии. Но, подобно тому как изучение
последних наук не остается без пользы для диететического поведения, так и
изучению форм разума надлежит без сомнения приписывать еще более важное
влияние на правильность мышления. Однако, не входя здесь в обсуждение этой
стороны, касающейся культуры субъективного мышления и поэтому, собственно
говоря, педагогики, мы должны будем согласиться с тем, что изучение,
имеющее своим предметом способы операций и законы разума, должно быть само
по себе в высшей степени интересно, по крайней мере, не менее интересно,
чем познание законов природы и ее собственных форм. Если признается
немаловажным делом нахождение шестидесяти с лишком видов попугаев, ста
тридцати семи видов вероники и~т.~д., то надо считать еще гораздо более
важным нахождение форм разума; не является ли фигура умозаключения чем-то
бесконечно более высоким, чем вид попугая или вероники?

Поэтому, хотя мы должны рассматривать только как варварство
презрительное отношение вообще к изучению форм разума, мы все же должны
согласиться с тем, что обычное изложение умозаключения и его особенных форм
не представляет собой {\em разумного}
познания их, не изображает их как
{\em формы разума}, и
силлогистическая премудрость навлекла на себя именно вследствие ее
неценности то пренебрежение, с которым к ней начали относиться. Ее
недостаток состоит в том, что она ни на шаг не идет дальше
{\em рассудочной формы}
умозаключения, согласно которой определения понятия берутся
как {\em абстрактные},
формальные определения. Фиксирование их как абстрактных
качеств тем более непоследовательно, что в умозаключении существенным
являются как раз их {\em соотношения},
и присущность и подчинение уже подразумевают, что единичное,
так как ему присуще всеобщее, само есть всеобщее, а всеобщее, так как оно
подчиняет себе единичное, само есть единичное; указанное фиксирование
определений понятия в виде абстрактных качеств совершенно непоследовательно
ближайшим образом еще потому, что умозаключение явно полагает именно это их
{\em единство} как
{\em средний термин}, и
его (умозаключения) определение [или назначение] как раз и состоит в
{\em опосредствовании},
т.~е. в том, что определения понятия уже больше не имеют, как
в суждении, своей основой свою внешность по отношению друг к другу, а,
наоборот, имеют основой свое единство. "--- Тем самым, понятием
умозаключения высказывается несовершенство формального умозаключения, в
котором средний термин фиксируется не как единство крайних терминов, а как
некоторое формальное, качественно отличное от них, абстрактное
определение. "--- Рассмотрение делается еще более
бессодержательным вследствие того, что здесь все еще принимаются за полные
отношения также и такие соотношения или суждения, в которых даже и
формальные определения становятся безразличными (как, например, в
отрицательном и партикулярном суждениях) и которые поэтому приближаются к
предложениям. "--- Так как качественная форма
{\em Е "--- О "--- В} считается вообще окончательной и
абсолютной, то диалектическое рассмотрение умозаключения совершенно
отпадает, и остальные умозаключения тем самым рассматриваются не как
{\em необходимые изменения}
той формы, а как
{\em виды}. "--- При этом
безразлично, рассматривается ли первое формальное умозаключение само лишь
как некоторый вид {\em наряду}
с прочими или же одновременно и как
{\em род} и как вид;
последнее происходит, когда остальные умозаключения сводятся к первому.
Если это сведение и не происходит явно, то все же в основании [рассмотрения
остальных фигур] всегда лежит то же самое формальное отношение внешнего
подчинения, которое выражает собой первая фигура.

Это формальное умозаключение представляет собой противоречие,
состоящее в том, что средний термин должен быть определенным
единством крайних терминов, но на самом деле он выступает не
как это единство, а как определение, качественно отличное от тех
определений, единством которых оно должно быть. Так как умозаключение
представляет собой это противоречие, то оно в самом себе диалектично. Его
диалектическое движение являет его в полноте моментов понятия, показывая,
что не только вышеуказанное отношение подчинения (или особенность), но
{\em столь же существенным образом}
и отрицательное единство [т.~е. единичность] и всеобщность
служат моментами того смыкания воедино, которое имеет место в
умозаключении. Поскольку каждый из этих двух моментов сам по себе есть
равным образом лишь некоторый односторонний момент особенности, они также
представляют собой несовершенные средние термины, но вместе с тем они
составляют и развитые определения особенности. Весь этот процесс
прохождения через указанные три фигуры показывает средний термин
последовательно в каждом из этих определений, и истинным результатом,
проистекающим из этого процесса, является то, что средний термин есть не
какое-нибудь одно из этих определений, а тотальность их.

Недостаток формального умозаключения состоит поэтому не в
{\em форме умозаключения}
"--- она, напротив, есть форма разумности, "--- а в
том, что она выступает лишь как
{\em абстрактная} и
поэтому чуждая понятию форма. Мы показали, что абстрактное определение в
силу своего абстрактного соотношения с собой может быть рассматриваемо в
такой же мере и как содержание; постольку формальное умозаключение ничего
больше не дает, кроме утверждения, что некоторое соотношение того или иного
субъекта с тем или иным предикатом вытекает или не вытекает
{\em лишь из этого среднего}
термина. То обстоятельство, что то или иное предложение было
доказано посредством такого рода умозаключения, ничуть не помогает делу в
силу абстрактной определенности среднего термина, представляющего собой
некоторое чуждое понятию качество, с таким же правом могут существовать
другие средние термины, из которых вытекает противоположное, и даже больше
того: из одного и того же среднего термина можно, в свою очередь
посредством дальнейших умозаключений, вывести противоположные предикаты. "---
Помимо того, что формальное умозаключение не очень-то много
дает, оно есть также и нечто очень простое; те многочисленные правила,
которые были изобретены силлогистикой, несносны уже потому, что они так
контрастируют с простой природой предмета, а затем также и потому, что они
относятся к таким случаям, где формальное содержание
умозаключения вследствие внешнего определения формы, "---
особенно такого определения, как определение партикулярности
(тем более, что оно должно быть взято для этой цели в широком смысле), "---
окончательно оскудевает и где даже со стороны формы
получаются лишь совершенно бессодержательные результаты. "---
Но самой справедливой и самой важной причиной той немилости,
в которую впала силлогистика, является то, что она есть столь растянутое,
{\em чуждое понятию}
занятие таким предметом, единственным содержанием которого
служит само {\em понятие}. "---
Многочисленные силлогистические правила напоминают образ
действия учителей арифметики, которые равным образом дают множество правил
для арифметических операций, причем все эти правила предполагают отсутствие
у них {\em понятия}
операций. "--- Но числа представляют собой чуждый
понятию материал, счетная операция есть внешнее соединение или разделение,
механический прием, и мы знаем, что в самом деле были изобретены счетные
машины, выполняющие эти операции; напротив, когда с формальными
определениями умозаключения, которые суть понятия, обращаются как с чуждым
понятию материалом, то это является наиболее возмутительным и наиболее
невыносимым.

Доведенный до крайности образчик такой чуждой понятию
трактовки понятийных определений умозаключения, несомненно, дает нам Лейбниц
(Орр. Tom.~II, р.~I), который подверг умозаключение
комбинаторному исчислению и определил посредством него число возможных форм
умозаключения, если именно принимать во внимание различие положительных и
отрицательных, затем всеобщих, партикулярных, неопределенных и сингулярных
суждений; оказывается, что число таких возможных сочетаний 2 048, из
которых по исключении непригодных фигур остается пригодных 24. "---
Лейбниц считает комбинаторный анализ очень полезным для
нахождения не только форм умозаключения, но и сочетаний других понятий.
Служащая для этого операция такая же, как та, посредством которой
вычисляется, сколько комбинаций букв возможны в азбуке, сколько сочетаний
костей "--- при игре в кости, или сколько комбинаций карт при
игре в ломбер и~т.~п. Следовательно, определения умозаключения поставлены
здесь в один ряд с сочетаниями костей или карт при игре в ломбер, разумное
берется как нечто мертвенное и чуждое понятию, и оставляется в стороне
своеобразие понятия и его определений, заключающееся в том, что они
{\em соотносятся} между
собой как духовные сущности и через это соотношение
{\em снимают} свое
{\em непосредственное}
определение. "--- Это лейбницево
применение исчисления комбинаций к умозаключению и к сочетанию других
понятий не отличалось от пресловутого
{\em луллиева искусства}
ничем другим, кроме большей методичности с
{\em арифметической}
точки зрения, вообще же не уступало ему в бессмысленности. "---
С этим у Лейбница была связана любимая его мысль, к которой
он пришел еще в юности и от которой он, несмотря на ее незрелость и
поверхностность, не отказался и впоследствии: мысль о некоторой
{\em всеобщей характеристике}
понятий, о письменном языке, в котором каждое понятие было бы
представлено как соотношение, вытекающее из других понятий, или как
соотношение с другими, как будто в разумной связи, которая существенно
диалектична, какое-либо содержание еще сохраняет те же самые определения,
которые оно имеет, когда его фиксируют отдельно.

{\em Логическое счисление Плукэ}
избрало без сомнения самый последовательный прием для того,
чтобы подчинить вычислению отношение умозаключения. Это счисление основано
на том, что в суждении абстрагируют от различия отношения, т.~е. от
различия между единичностью, особенностью и всеобщностью, и фиксируют
{\em абстрактное тождество}
субъекта и предиката, в силу чего между ними устанавливается
{\em математическое равенство}, "---
соотношение, которое превращает процесс умозаключения в
совершенно бессодержательное и тавтологическое образование предложений. "---
В предложении <<роза красна>> предикат согласно этому учению
должен означать не всеобщий красный цвет, а лишь определенный
{\em красный цвет розы;}
в предложении: <<все христиане суть люди>> предикат должен
означать лишь тех людей, которые суть христиане; из него и из предложения:
<<евреи не христиане>> следует заключение, которое не послужило хорошей
рекомендацией этому силлогистическому счислению в глазах
{\em Мендельсона}:
<<следовательно евреи "--- не люди>> (именно, не те
люди, которые суть христиане). "--- Плукэ указывает в качестве
результата своего изобретения на то, что posse etiam rudes mechanice totam
logicam doceri, uti pueri arithme\-ticam docentur, ita quidem, ut nulla
formidine in ratiociniis suis errandi torqueri, vel fallaciis circum\-veniri
possint, si in calculo non errant
({\em даже невежды могут механически
научиться всей логике} подобно тому, как дети научаются
арифметике, "--- притом так, чтобы их не мучило никакое
опасение ошибиться в своих рассуждениях или быть обманутыми какими-либо
хитростями, если только они не ошиблись в счете). "--- Эта
рекомендация, что невежд можно посредством счисления
{\em механически} научить
всей логике, представляет собой, конечно, наихудшее, что
можно сказать о каком-либо изобретении в области изложения логической
науки.

\section[В. Умозаключение рефлексии]{В. Умозаключение рефлексии}

Развитие качественного умозаключения сняло
{\em абстрактность} его
определений; тем самым термин положил себя как такую определенность, сквозь
которую {\em просвечивает}
также и другая определенность. Кроме абстрактных терминов, в
умозаключении имеется также и их
{\em соотношение}, и в
заключении оно положено как опосредствованное и необходимое; поэтому каждая
определенность положена поистине не особо, как отдельная определенность, но
как соотношение других определенностей, другими словами, как
{\em конкретная} определенность.

{\em Средним термином}
служила абстрактная особенность, которая сама по себе
представляет собой некоторую простую определенность и была средним термином
лишь внешним образом, по отношению к самостоятельным крайним терминам.
Теперь же средний термин положен как
{\em тотальность}
определений; таким образом, он есть
{\em положенное} единство
крайних терминов; но ближайшим образом это есть единство рефлексии,
охватывающей их собой; это "--- такой охват их, который, как
{\em первое} снятие
непосредственности и первое соотнесение определений, еще не есть абсолютное
тождество понятия.

Крайние термины суть определения суждения рефлексии:
{\em единичность} в
собственном смысле и {\em всеобщность}
в смысле определения отношения, или, иначе говоря, рефлексия,
охватывающая собой многообразное. Но единичный субъект, как было показано
при рассмотрении суждения рефлексии, содержит в~себе, кроме принадлежащей к
форме голой единичности, также и определенность, как безоговорочно
рефлектированную в~себя всеобщность, как предположенный, т.~е. здесь еще
непосредственно взятый {\em род}.

Из этой определенности крайних терминов, которая получается в
ходе развития определений суждения, вытекает более детальное содержание
{\em среднего термина},
который имеет капитальную важность в умозаключении, так как
он отличает умозаключение от суждения. Средний термин содержит в~себе 1)
{\em единичность}, 2) но
расширенную до всеобщности единичность, как
{\em все} единичности, и
3) лежащую в основании всеобщность, безоговорочно соединяющую в~себе
единичность и абстрактную всеобщность,
{\em род}. "--- Таким
образом, умозаключение рефлексии впервые получает
{\em настоящую определенность}
формы, поскольку средний термин
{\em положен} как
тотальность определений; непосредственное умозаключение является
по сравнению с умозаключением рефлексии
{\em неопределенным}, так
как средний термин в нем пока что еще есть абстрактная особенность, в
которой моменты его понятия еще не положены. "--- Это первое
умозаключение рефлексии может быть названо
{\em умозаключением
всякости}\pagenote{См. примечание \ref{bkm:bm38}.\label{bkm:bm56}}.

\subsection[а) Умозаключение всякости]{а) Умозаключение всякости}

1. Умозаключение всякости есть умозаключение рассудка в его
завершенности, но не более того. Что средний термин есть в нем не
{\em абстрактная}
особенность, но развернут в свои моменты и потому конкретен,
это, правда, составляет существенное требование для понятия, однако форма
{\em всякости}
синтезирует единичное во всеобщность ближайшим образом лишь
внешне, и, наоборот, во всеобщности она сохраняет единичное все еще как
нечто непосредственно, само по себе существующее. Отрицание
непосредственности определений, которое было результатом умозаключения
наличного бытия, есть лишь {\em первое}
отрицание, а не отрицание отрицания или абсолютная рефлексия
в~себя. Поэтому в основании указанной всеобщности рефлексии, охватывающей
собой отдельные определения, все еще лежат эти отдельные определения, или,
иначе говоря, всякость еще не есть всеобщность понятия, а есть только
внешняя всеобщность рефлексии.

Умозаключение наличного бытия было потому случайно, что его
средний термин, как некоторая единичная определенность конкретного
субъекта, допускает неопределенное множество других подобных же средних
терминов, и тем самым субъект мог быть сомкнут с неопределенным количеством
других и даже противоположных предикатов. Но так как теперь средний термин
содержит в~себе {\em единичность}
и в силу этого сам конкретен, то он может связывать с
субъектом лишь такой предикат, который присущ субъекту как конкретному. "---
Если, например, из среднего термина
<<{\em зеленый}>> нужно
было бы умозаключить, что данная картина приятна, так как зеленое приятно
для глаз, или что стихотворение, здание и~т.~д. прекрасны, так как они
обладают {\em правильностью},
то, несмотря на это, картина и~т.~д. могут быть отвратительны
вследствие других своих определений, от которых можно было бы умозаключить
к предикату <<отвратительный>>. Напротив, когда средний термин обладает
определением {\em всякости},
то он содержит в~себе зеленое и правильность как
{\em нечто конкретное},
которое именно поэтому не есть абстракция некоторого
исключительно только зеленого, только правильного и~т.~д.; с этим
{\em конкретным} могут
быть соединены лишь такие предикаты, которые согласуются с
{\em тотальностью конкретного}. "---
В суждении:
<<{\em зеленое} (или
{\em правильное})
{\em приятно}>> субъект
есть лишь абстракция зеленого (или правильности); напротив, в предложении
<<{\em все зеленое} (или:
{\em все правильное})
{\em приятно}>> субъектом
служат все те действительные конкретные предметы, которые зелены (или
правильны) и которые, следовательно, берутся как
{\em конкретные},
{\em со всеми теми их свойствами},
какими они обладают помимо зелености (или правильности).

2. Но это рефлективное совершенство умозаключения делает его
именно поэтому лишь обманчивым призраком. Средний термин имеет
определенность: <<{\em все}>>; этим <<всем>> {\em непосредственно}
принадлежит в большей посылке тот предикат, который смыкают с
субъектом. Но <<{\em все}>> суть {\em все единичные;}
следовательно, в этой большей посылке единичный субъект уже
непосредственно обладает указанным предикатом,
{\em вместо того чтобы получить его
впервые через умозаключение}. "--- Или, иначе говоря, субъект
получает через заключение предикат как некоторое следствие; но б\'{о}льшая
посылка уже содержит в~себе это заключение;
{\em поэтому б\'{о}льшая посылка правильна не сама
по себе как взятая отдельно} от заключения или, иными
словами, она не есть непосредственное, пред-положенное суждение, но
{\em сама уже предполагает заключение},
основанием которого она должна была служить.

В излюбленном совершенном умозаключении:

\begin{verse}
Все люди смертны,\\
Кай "--- человек,\\
Следовательно, он смертен.
\end{verse}

\noindent б\'{о}льшая посылка правильна лишь потому и постольку, поскольку
{\em правильно заключение}. "---
Если бы Кай случайно не был смертен, то б\'{о}льшая посылка была
бы неправильна. Предложение, долженствовавшее служить заключением, должно
быть правильным уже непосредственно, само по себе, ибо в противном случае
б\'{о}льшая посылка не могла бы охватить собой всех единичных; прежде чем
б\'{о}льшая посылка может быть признана правильной, приходится
{\em предварительно} решить вопрос, не служит ли само это заключение
{\em возражением} против нее.

3. При рассмотрении умозаключения наличного бытия из понятия
умозаключения получился тот вывод, что посылки как {\em непосредственные}
противоречат заключению, а именно, требуемому понятием
умозаключения {\em опосредствованию},
и что поэтому первое умозаключение предполагает другие и,
обратно, эти другие предполагают первое. В~умозаключении рефлексии это
положено в нем же самом, а именно, положено, что б\'{о}льшая
посылка предполагает свое заключение, так как в ней содержится то
соединение единичного с предикатом, которое впервые должно быть установлено
в заключении.

Следовательно, то, что мы здесь имеем на самом деле, может
быть выражено ближайшим образом так: умозаключение рефлексии есть лишь
внешняя пустая {\em видимость
умозаключения;} стало быть, сущность этого умозаключения
покоится на, субъективной {\em единичности;}
последняя тем самым образует собой средний термин и должна
быть положена как средний термин; это "--- такая единичность,
которая имеет бытие как таковая и лишь внешним образом обладает
всеобщностью. "--- Или, иначе говоря, более детальное
установление содержания умозаключения рефлексии показало, что единичное
находится в {\em непосредственном},
а не в умозаключенном соотношении со своим предикатом и что
б\'{о}льшая посылка (соединение некоторого особенного с некоторым всеобщим или,
точнее, некоторого формально всеобщего с некоторым всеобщим в~себе)
опосредствована тем соотношением единичности, которое здесь имеется,
единичности как всякости. Но это есть {\em умозаключение индукции}.

\subsection[b) Умозаключение индукции]{b) Умозаключение индукции}

1. Умозаключение всякости подчинено схеме первой фигуры:
{\em Е "--- О "--- В;} индуктивное умозаключение "--- схеме второй
фигуры: {\em В "--- ~Е "--- О}, так как оно опять имеет средним термином
единичность, но не {\em абстрактную} единичность, а {\em полную}, т.~е.
положенную с противоположным ей определением,
с всеобщностью. "--- {\em Один из крайних терминов} есть какой-либо предикат,
который общ всем этим единичным; его соотношение с ними образует собой
те непосредственные посылки, одна из которых должна была быть заключением
в предшествующем умозаключении. "--- {\em Другой крайний термин}
может быть непосредственным {\em родом}, каков он в среднем термине
предыдущего умозаключения или в субъекте универсального
суждения, родом, который исчерпан в совокупности единичных или же видов
среднего термина. Согласно этому, умозаключение имеет следующий вид:

{\centering
e
\par}

{\centering
В "--- E "--- О
\par}

{\centering
е
\par}

{\centering
е
\par}

{\centering
и~т.~д.
\par}

{\centering
до бесконечности.
\par}

2. Вторая фигура формального умозаключения
({\em В "--- Е "--- О}) потому не соответствовала схеме
умозаключения, что в одной из посылок второй фигуры
{\em Е}, образующее собой
средний термин, не было подчиняющим или предикатом. В~индукции этот
недостаток устраняется; здесь средним термином служит:
<<{\em все} единичные>>;
предложение <<{\em В
"--- E}>>\pagenote{Под <<{\em E}>> Гегель имеет здесь в виду {\em совокупность}
всех единичных какого-нибудь рода. Пользуясь примером,
приводимым Гегелем в следующей фразе, можно вместо
<<{\em В "--- Е}>> подставить такое суждение: <<Четвероногие животные суть: лев,
слон, медведь, лошадь и~т.~д.>>. Для б\'{о}льшей наглядности продолжим этот
пример. Пусть второй посылкой будет суждение: <<лев, слон, медведь, лошадь
и~т.~д. имеют хвост>>. Тогда заключение будет гласить: <<все четвероногие
имеют хвост>>.\label{bkm:bm57}},
которое содержит в~себе в качестве субъекта объективное
всеобщее или род, выделившийся в качестве крайнего термина, имеет такой
предикат, который по меньшей мере обладает равным с субъектом объемом и тем
самым тождественен с ним для внешней рефлексии. Лев, слон, и~т.~д. образуют
собой {\em род} четвероногих животных; различие, состоящее в том,
{\em что то же самое} содержание в одном случае положено в единичности,
а в другом "--- во всеобщности, есть поэтому только
{\em безразличное определение формы;}
"--- безразличие, которое представляет собой положенный в
рефлективном умозаключении результат формального умозаключения и которое
здесь положено равенством объема.

Поэтому индукция не есть умозаключение голого
{\em восприятия} или
случайного наличного бытия, каким была соответствующая ему вторая фигура, а
умозаключение {\em опыта}
"--- субъективного синтезирования единичных в род и смыкания
рода с некоторой всеобщей определенностью, поскольку она встречается во
всех единичных. Умозаключение это имеет также и то объективное значение,
что непосредственный род определяет себя через тотальность единичности к
некоторому всеобщему свойству, имеет свое наличное бытие в некотором
всеобщем отношении или признаке. "--- Однако объективное
значение этого умозаключения, как и других, есть пока что лишь его
внутреннее понятие и здесь еще не положено.

3. Скорее можно сказать, что индукция есть еще по существу
некоторое субъективное умозаключение. Средним термином здесь служат
единичные в их непосредственности; синтезирование их через всякость в род
есть некоторая {\em внешняя}
рефлексия. В~силу пребывающей
{\em непосредственности}
единичных и в силу вытекающего отсюда
{\em внешнего характера}
всеобщность есть лишь полнота или, лучше
сказать, остается {\em некоторой
задачей}. "--- В ней поэтому опять появляется
{\em прогресс} в дурную
бесконечность; {\em единичность}
должна быть положена как
{\em тождественная} со
{\em всеобщностью}, но
так как {\em единичные}
положены вместе с тем и как
{\em непосредственные},
то указанное единство остается лишь постоянным
{\em долженствованием;}
оно есть единство
{\em равенства;}
долженствующие быть тождественными должны вместе с тем и
{\em не}~быть
тождественными. Лишь продолженные до
{\em бесконечности},
{\em a, b, c, d, e} образуют собой род и
дают завершенный опыт. {\em Заключение}
индукции остается постольку
{\em проблематическим}\pagenote{Ср. замечание Энгельса о том, что
постоянные перевороты в индуктивных классификациях животного и
растительного мира служат <<прекрасным подтверждением гегелевского положения
о том, что индуктивное умозаключение по существу является проблематическим>>
({\em Engels}, Dialektik der Natur, M.--L. 1935, S.~653).\label{bkm:bm58}}.

Но выражая собой то обстоятельство, что восприятие для того,
чтобы стать опытом, {\em должно}
быть продолжено {\em до
бесконечности}, индукция предполагает, что род сомкнут со
своей определенностью {\em в~себе и для
себя}. Индукция, собственно говоря, тем самым скорее
предполагает свое заключение как нечто непосредственное, точно так же как
умозаключение всякости предполагает заключение для одной из своих посылок.
"--- Опыт, основанный на индукции, признается значимым,
{\em хотя} восприятие, по
общему признанию, {\em не завершено;}
но полагать, что
{\em против} означенного
опыта не может найтись никакого
{\em противопоказания},
можно лишь постольку, поскольку этот опыт истинен
{\em в~себе и для себя}.
Поэтому умозаключение через индукцию основывается, правда, на
некоторой непосредственности, но не на той непосредственности, на которой
оно, согласно обычному взгляду, должно было бы основываться, т.~е. не на
{\em сущей}
непосредственности
{\em единичности}, а
{\em на в-себе-и-для-себя-сущей},
всеобщей непосредственности. "--- Основное
определение индукции заключается в том, что она есть некоторое
умозаключение; если единичность берется как существенное определение
среднего термина, а всеобщность лишь как его внешнее определение, то
средний термин распался бы на две несвязанные между собой части, и у нас не
было бы никакого умозаключения; этот внешний характер принадлежит скорое
крайним терминам. {\em Единичность}
может быть средним термином только
{\em как непосредственно тождественная
со всеобщностью}. Такая всеобщность есть, собственно говоря,
{\em объективная всеобщность},
{\em род}. "--- Это можно
рассматривать также и следующим образом: в том определении единичности,
которое лежит в основании среднего термина индуктивного умозаключения,
всеобщность имеется {\em внешним},
{\em но существенным образом;}
такое {\em внешнее}
есть столь же непосредственно своя противоположность, т.~е.
{\em внутреннее}. "---
Истиной индуктивного умозаключения служит поэтому
такое умозаключение, которое имеет средним термином такую
единичность, которая непосредственно
{\em в самой себе} есть
всеобщность; это "--- {\em умозаключение по
аналогии}.

\subsection[с) Умозаключение по аналогии]{с) Умозаключение по аналогии}

1. Это умозаключение имеет своей абстрактной схемой третью
фигуру непосредственного умозаключения:
{\em Е "--- В
"--- О}. Но его средним термином служит теперь
уже не какое-либо отдельное качество, а такая всеобщность, которая есть
{\em рефлексия-в-себя некоторого
конкретного} и, стало быть, его
{\em природа;} и обратно,
так как он есть таким образом всеобщность как всеобщность некоторого
конкретного, то он есть вместе с тем в самом себе это
{\em конкретное}. "---
Здесь, следовательно, средним термином служит некоторое
единичное, но со стороны своей всеобщей природы; далее, некоторое другое
единичное образует собой крайний термин, имеющий с первым единичным одну и
ту же всеобщую природу. Например:

{\em Земля} имеет обитателей,
Луна есть {\em некоторая земля},
следовательно, Луна имеет обитателей.

2. Аналогия тем поверхностнее, чем в большей мере то всеобщее,
в котором оба единичных оказываются одним и тем же и согласно которому одно
единичное становится предикатом другого, есть только
{\em качество} или (если
брать качество субъективно) тот или иной
{\em признак}, когда
тождество обоих в отношении этого признака берется просто как
{\em сходство}. Но такого
рода поверхностность, к которой форма рассудка или разума приводится тем,
что ее низводят в сферу голого
{\em представления}, не
должна была бы вообще иметь места в логике. "--- Равным
образом не подобает излагать б\'{о}льшую посылку этого умозаключения так, чтобы
она гласила: <<{\em То, что сходно с
каким-нибудь объектом в некоторых признаках, сходно с ним также и в других
признаках}>>. Таким путем
{\em форма умозаключения}
выражается в виде некоторого содержания, а эмпирическое
содержание, которое, собственно говоря, только и следует называть
содержанием, в своей совокупности переносится в меньшую посылку. Подобным
же образом вся форма, например, первого умозаключения тоже могла бы быть
выражена в виде его б\'{о}льшей посылки:
<<{\em Тому, что подчинено иному, которому присуще нечто третье,
тоже присуще это третье; и так как…}>> и~т.~д. Однако при рассмотрении самого
умозаключения существенно-важным является не эмпирическое содержание, и
обращение его собственной формы в содержание некоторой б\'{о}льшей посылки
столь же безразлично, как если бы мы вместо этого взяли
любое другое эмпирическое содержание. Но подобно тому как в умозаключении
по аналогии при таком способе его формулирования суть дела не должна была
бы состоять в вышеуказанном содержании, не заключающем в~себе ничего, кроме
своеобразной формы умозаключения, точно так же и в первом умозаключении
[наличного бытия] суть дела не состояла бы в этом содержании, т.~е. не
состояла бы в том, что делает умозаключение умозаключением. "---
Что важно и существенно, это всегда форма умозаключения, все
равно, имеет ли оно своим эмпирическим содержанием самое эту форму или
что-нибудь другое. Таким образом, умозаключение по аналогии представляет
собой некоторую своеобразную форму, и нет основания не рассматривать его
как такую форму лишь потому, что его форма может, дескать, быть сделана
содержанием или материей некоторой б\'{о}льшей посылки, а материя-де логики не
касается. "--- В умозаключении по аналогии и, пожалуй, также и
в индуктивном умозаключении на эту мысль может соблазнить то
обстоятельство, что в них средние, а равно и крайние термины имеют б\'{о}льшую
определенность, чем в чисто формальном умозаключении, и поэтому определение
формы, так как оно уже не просто и не абстрактно, необходимо должно
представляться также и {\em определением
содержания}. Но то обстоятельство, что форма, таким образом,
определяет себя как содержание, есть, во-первых, необходимое поступательное
движение формального момента и потому существенно касается природы
умозаключения; но поэтому же,
{\em во-вторых}, такое
определение содержания не может рассматриваться точно так же, как любое
другое эмпирическое содержание, и от него нельзя абстрагировать.

Если рассматривать форму умозаключения по аналогий в виде того
выражения его б\'{о}льшей посылки, которое гласит:
<<{\em если два предмета совпадают в
одном или в нескольких свойствах, то одному из них присуще также и
некоторое дальнейшее свойство, имеющееся, у другого предмета}>>,
то может показаться, что это умозаключение содержит в~себе
{\em четыре определения}, quaterni\-onem
termi\-norum, обстоятельство, затруднившее приведение аналогии
к форме формального умозаключения. "--- В умозаключении по
аналогии [при вышеуказанном способе его формулирования] имеются
{\em два} единичных,
{\em в-третьих}, одно
свойство, непосредственно принимаемое за общее обоим этим единичным
предметам, и, {\em в-четвертых},
то другое свойство, которым одно единичное обладает
непосредственно, а другое единичное получает его лишь через умозаключение.
"--- Это происходит оттого, что, как мы видели, в умозаключении
по аналогии {\em средний термин}
положен как единичность, но непосредственно
{\em также и} как
истинная всеобщность этой единичности. "--- В
{\em индукции} мы, кроме
двух крайних терминов, имеем в среднем термине неопределимое множество
единичных; в этом умозаключении можно было бы поэтому насчитать бесконечное
множество терминов. "--- В умозаключении всякости всеобщность
среднего термина выступает пока что лишь как внешнее формальное определение
всякости; напротив, в умозаключении по аналогии она есть существенная
всеобщность. В~вышеприведенном примере средний термин
<<{\em Земля}>> берется как
такое конкретное, которое по своей истине есть столь же некоторая всеобщая
природа (или род), сколь и нечто единичное.

С этой стороны quaternio terminorum
(учетверение терминов) не делало аналогию несовершенным
умозаключением. Но она становится таковым вследствие этого учетверения с
другой стороны: ибо хотя один субъект имеет ту же всеобщую природу, что и
другой, все же остается неопределенным, присуща ли одному субъекту в силу
его {\em природы} или же
в силу его {\em особенности}
та определенность, присущность которой также и другому
субъекту устанавливается путем умозаключения, например, имеет ли земля
обитателей как небесное тело
{\em вообще}, или же
только как именно это {\em особенное}
небесное тело. "--- Аналогия является еще
умозаключением рефлексии постольку, поскольку единичность и всеобщность
соединены в ее среднем термине
{\em непосредственно}.
Вследствие этой непосредственности здесь еще имеется налицо
{\em внешний характер}
рефлективного единства; единичное есть лишь
{\em в~себе} род; оно не
положено в той отрицательности, через которую его определенность выступала
бы как собственная определенность рода. Поэтому предикат, присущий тому
единичному, которое входит в состав среднего термина, еще не есть
обязательно также и предикат другого единичного, хотя оба эти единичные
принадлежат одному и тому же роду.

3. <<{\em Е "--- О}>>
(<<Луна имеет обитателей>>) есть заключение; но одна из посылок
(<<земля имеет обитателей>>) есть такое же <<{\em E"---О}>>;
поскольку <<{\em Е"---О}>>
должно быть заключением, постольку в этом долженствовании
содержится требование, чтобы и указанная посылка была некоторым
заключением. Это умозаключение есть, стало быть, в~себе самом требование
самого себя в противовес той непосредственности, которую оно содержит в
себе, или, иначе говоря, оно предполагает свое собственное заключение.
Умозаключение наличного бытия имеет свое пред-положение или свою
предпосылку {\em в других}
умозаключениях наличного бытия; в только что
рассмотренных умозаключениях эта предпосылка вдвинута внутрь
их, так как они суть умозаключения рефлексии. Поскольку, стало быть,
умозаключение по аналогии представляет собой требование своего собственного
опосредствования в противовес той непосредственности, которой обременено
его опосредствование, постольку как раз момент
{\em единичности}
оказывается тем, снятия чего оно требует. Таким образом, на
долю среднего термина остается объективное всеобщее
"--- {\em род}, очищенный от
непосредственности. "--- В умозаключении по аналогии род был
моментом среднего термина лишь как
{\em непосредственное пред-положение;}
так как само умозаключение требует снятия пред-положенной
непосредственности, то отрицание единичности и тем самым все общее уже
больше не непосредственно, а
{\em положено}. "---
Умозаключение рефлексии содержало в~себе сперва лишь
{\em первое} отрицание
непосредственности; теперь появилось второе отрицание, и тем самым внешняя
всеобщность рефлексии определена так, что она стала
в-себе-и-для-себя-сущей. "--- Если рассматривать это с
положительной стороны, то заключение оказывается тождественным с посылкой,
опосредствование "--- слившимся со своим пред-положением (со
своей предпосылкой), и, стало быть, имеется такое тождество рефлективной
всеобщности, благодаря которому она становится более высокой
всеобщностью.

Обозревая весь ход умозаключений рефлексии, мы находим, что
опосредствование есть здесь вообще
{\em положенное} или
{\em конкретное} единство
формальных определений крайних терминов: рефлексия состоит в этом полагании
одного определения в другом; опосредствующим служит, таким образом,
{\em всякость}. Однако
существенным основанием последней оказывается
{\em единичность}, а
всеобщность оказывается лишь внешним в ней определением,
{\em полнотой}. Но
всеобщность {\em существенна}
для единичного, раз оно должно быть смыкающим средним
термином; поэтому единичное следует понимать как в-себе-сущее всеобщее.
Однако единичное соединено со всеобщностью не только этим чисто
положительным образом, а снято в ней и представляет собой отрицательный
момент; таким образом, всеобщее есть нечто в-себе-и-для-себя-сущее,
положенный род, а единичное, как непосредственное, есть скорее внешность
последнего, или, иначе говоря, оно есть
{\em крайний термин}. "---
Умозаключение рефлексии, вообще говоря, подчинено схеме
{\em О "--- Е "--- B},
и единичное как таковое еще есть в нем существенное
определение среднего термина; но так как теперь его непосредственность
сняла себя и средний термин определился как
в-себе-и-для-себя-сущая всеобщность, то умозаключение
подчинилось формальной схеме {\em Е
"--- В "--- О}, и умозаключение
рефлексии перешло в {\em умозаключение
необходимости}.

\section[С. Умозаключение необходимости]{С. Умозаключение необходимости}

Опосредствующее определило себя теперь 1) как
{\em простую}
определенную всеобщность, подобно особенности в умозаключении
наличного бытия, но 2) как
{\em объективную}
всеобщность, т.~е. как такую всеобщность, которая содержит в
себе всю определенность различенных крайних терминов, подобно всякости в
умозаключении рефлексии; это
"--- {\em наполненная}, но
{\em простая}
всеобщность, {\em всеобщая
природа} вещи,
{\em род}.

Это умозаключение
{\em содержательно},
потому что
{\em абстрактный} средний
термин умозаключения наличного бытия положил себя так, что он оказывается
{\em определенным различием},
в качестве какового он выступал как средний термин
умозаключения рефлексии, но это различие вновь рефлектировало себя в
простое тождество. "--- Это умозаключение есть поэтому
умозаключение {\em необходимости},
так как его средний термин есть не какое-либо иное
непосредственное содержание, а рефлексия-в-себя определенности крайних
терминов. Последние имеют в среднем термине свое внутреннее тождество,
определения содержания которого суть определения формы крайних терминов. "---
Тем самым то, чем термины отличаются друг от друга, выступает
как {\em внешняя} и
{\em несущественная}
форма, и они оказываются моментами
{\em некоторого необходимого}
наличного бытия.

Вначале это умозаключение есть непосредственное умозаключение
и постольку формальное в том смысле, что
{\em связь} терминов есть
{\em существенная природа}
как {\em содержание},
и последнее выступает в различных терминах лишь в
{\em разной форме}, а
крайние термины сами по себе выступают лишь как некоторое
{\em несущественное}
устойчивое наличие. "--- Реализация этого
умозаключения должно определить его так, чтобы
{\em крайние термины
}тоже были
{\em положены} как та
{\em тотальность},
которую ближайшим образом представляет собой средний термин,
и чтобы {\em необходимость}
соотношения, которая ближайшим образом есть лишь
субстанциальное {\em содержание},
была некоторым соотношением
{\em положенной формы}.

\subsection[а) Категорическое умозаключение]{а) Категорическое умозаключение}

1. Категорическое умозаключение имеет одной или обеими своими
посылками категорическое
суждение\pagenote{Примером категорического
умозаключения может служить такой силлогизм: <<роза есть растение; растение
нуждается во влаге; следовательно, роза нуждается во влаге>>. Или: <<роза
есть растение; растение есть организм; следовательно, роза есть
организм>>.\label{bkm:bm59}}.
"--- Здесь с этим умозаключением, как и с соответствующим
суждением, связывается то более определенное значение, что
его средний термин есть {\em объективная
всеобщность}. При поверхностном рассмотрении и
категорическое умозаключение тоже считается не более, как только
умозаключением присущности.

Категорическое умозаключение есть по своему содержательному
значению {\em первое умозаключение
необходимости}, в котором некоторый субъект смыкается с
некоторым предикатом через {\em свою
субстанцию}. Но субстанция, поднятая в сферу понятия есть
всеобщее, положенное как в-себе-и-для-себя-сущее таким образом, чтобы иметь
формой, способом своего бытия не акцидентальность (как в отношении
субстанциальности), а определение понятия. Ее различиями служат поэтому
крайние термины умозаключения и, определеннее говоря, всеобщность и
единичность. Первая по отношению к
{\em роду}, который
теперь служит более детальным определением
{\em среднего термина},
есть абстрактная всеобщность или всеобщая определенность:
это "--- акцидентальность субстанции, собранная воедино в
простую определенность, которая, однако, есть ее (субстанции) существенное,
{\em специфическое различие}. "---
Единичность же есть действительное; она в~себе есть
конкретное единство рода и определенности, но здесь, как в непосредственном
умозаключении, она ближайшим образом есть непосредственная единичность,
собранная воедино в форму
{\em для-себя-сущего}
устойчивого наличия акцидентальность. "---
Соотношение этого крайнего термина со средним термином
составляет некоторое категорическое суждение; но поскольку и другой крайний
термин, согласно вышеуказанному определению, выражает собой специфическое
отличие рода или его определенный принцип, то и эта другая посылка тоже
категорична.

2. Это умозаключение, как первое и тем самым непосредственное
умозаключение необходимости, подчинено ближайшим образом схеме первого
формального умозаключения {\em Е "--- О "--- В}. Но так как средний
термин есть здесь существенная
{\em природа} единичного,
а не {\em какая-нибудь}
из его определенностей или свойств, и равным образом крайний
термин всеобщности есть не какое-либо абстрактное всеобщее (которое
опять-таки представляет собой лишь некоторое отдельное качество), а
всеобщая определенность,
{\em специфическое в различии}
рода, то отпадает та случайность, которая состоит в том, что
субъект сомкнут с {\em каким-либо
качеством} лишь через посредство
{\em какого-либо}
среднего термина. "--- Так как тем самым и
{\em соотношения} крайних
терминов со средним тоже не обладают той внешней непосредственностью,
которой они обладали в умозаключении наличного бытия, то требование
доказательства выступает здесь не в том смысле, в каком оно
имело место там, приводя к бесконечному прогрессу.

Далее: это умозаключение не предполагает, подобно
умозаключению рефлексии, истинности своего заключения для установления
истинности своих посылок. По своему субстанциальному содержанию термины
находятся в тождественном,
{\em в-себе-и-для-себя-сущем}
соотношении друг с другом; здесь имеется
{\em одна} проходящая
через три термина сущность, в которой определения единичности, особенности
и всеобщности суть лишь
{\em формальные}
моменты.

Поэтому категорическое умозаключение постольку уже больше не
субъективно; вместе с вышеуказанным тождеством начинается объективность;
средний термин есть содержательное тождество своих крайних терминов,
которые содержатся в нем со стороны их самостоятельности; ибо их
самостоятельность и есть указанная субстанциальная всеобщность, род.
Субъективность же этого умозаключения состоит в безразличном устойчивом
наличии крайних терминов перед лицом понятия или среднего термина.

3. Но в этом умозаключении субъективно еще то, что указанное
тождество имеется здесь еще как субстанциальное тождество или как
содержание, а не как вместе с тем и
{\em тождество формы}.
Поэтому тождество понятия есть пока что еще
{\em внутренняя} связь, и
тем самым оно, как соотношение, еще есть
{\em необходимость}.
Всеобщность среднего термина есть плотное,
{\em положительное}
тождество и не выступает вместе с тем и как
{\em отрицательность его крайних
терминов}.

Точнее говоря, непосредственность этого умозаключения, которая
еще не {\em положена} как
то, что она есть {\em в~себе},
имеет место следующим образом. Собственно непосредственным в
умозаключении является {\em единичное}.
Последнее подчинено своему роду как среднему термину; но
этому же роду подчинены еще и другие
{\em неопределенно многие}
единичные; поэтому является
{\em случайным}, что лишь
{\em это} единичное
положено как подчиненное этому роду. "--- Но далее, эта
случайность не принадлежит только области
{\em внешней рефлексии},
которая, {\em сравнивая}
положенное в умозаключении единичное с другими, находит его
случайным; напротив, тем, что это единичное само соотнесено со средним
термином, как со своей объективной всеобщностью, оно положено как
{\em случайное}, как
некоторая субъективная действительность. С~другой стороны, так как субъект
есть некоторое {\em непосредственное}
единичное, то он содержит в~себе такие определения, которые
не содержатся в среднем термине как во всеобщей природе; тем самым субъект
имеет также и некоторое безразличное к последней, само по себе определенное
существование, обладающее собственным своеобразным
содержанием. Тем самым и наоборот, этот другой
термин\pagenote{Речь идет о среднем термине категорического умозаключения.\label{bkm:bm60}}
тоже обладает некоторой безразличной непосредственностью и
отличным от первого существованием. "--- Это же отношение имеет
также место между средним и другим крайним термином; ибо последний равным
образом обладает определением непосредственности, следовательно,
определением некоторого случайного бытия по отношению к своему среднему
термину.

Итак, что же положено в категорическом умозаключении? Это,
{\em с одной стороны},
крайние термины в таком отношении к среднему, что они
{\em в~себе} обладают
объективной всеобщностью или самостоятельной природой и вместе с тем
выступают как непосредственные, следовательно, как
{\em безразличные} друг к
другу {\em действительности}.
С {\em другой}
же {\em стороны},
они выступают в такой же мере и как
{\em случайные}, или,
иначе говоря, их непосредственность определена как
{\em снятая} в их
тождестве. Но последнее в силу сказанной самостоятельности и тотальности
действительности есть лишь формальное, внутреннее тождество; тем самым
умозаключение необходимости определило себя как
{\em гипотетическое}.

\subsection[b) Гипотетическое умозаключение]{b) Гипотетическое умозаключение}

1. Гипотетическое суждение содержит в~себе лишь необходимое
{\em соотношение} без непосредственности соотнесенных.
<<{\em Если есть А, то есть В}>>; или, иначе говоря, бытие $B$ есть в такой же
мере и бытие {\em чего-то иного}, $B$; этим еще не сказано, {\em есть} ли
$A$, и {\em есть} ли $B$. Гипотетическое умозаключение прибавляет эту
{\em непосредственность} бытия:

\begin{verse}
Если есть $A$, то есть $B$,\\
но $A$ {\em есть},\\
следовательно, есть $B$.
\end{verse}

Меньшая посылка сама по себе высказывает непосредственное бытие $A$.

Но не только это прибавилось к гипотетическому суждению.
Умозаключение содержит в~себе соотношение между субъектом и предикатом не
как абстрактную связку, но как наполненное, {\em опосредствующее}
единство. {\em Бытие} $A$ следует поэтому понимать
{\em не как голую непосредственность}, а существенным образом как
{\em средний термин умозаключения}. Это должно быть рассмотрено ближе.

2. Прежде всего, то соотношение, которое имеет место в
гипотетическом суждении, есть {\em необходимость} или,
иначе говоря, внутреннее {\em субстанциальное тождество}
при внешней разности в существовании или при
взаимном безразличии являющегося бытия; это "--- некоторое
внутренне лежащее в основании тождественное {\em содержание}. Обе
стороны суждения суть поэтому не некоторое непосредственное бытие, а бытие,
удерживаемое в необходимости, следовательно, вместе с тем {\em снятое} или лишь
являющееся бытие. Далее, как стороны суждения, они относятся между собой
как {\em всеобщность} и {\em единичность;}
поэтому одна из них есть сказанное содержание как
{\em тотальность условий}, а другая "--- как {\em действительность}.
Однако безразлично, какая из сторон принимается за
всеобщность и какая за единичность. А~именно, поскольку условия суть еще
{\em внутреннее}, {\em абстрактное}
содержание некоторой действительности, они суть
{\em всеобщее}, и вступили-то они в {\em действительность}
лишь через {\em совокупление} их в некоторую {\em единичность}.
И обратно, условия суть {\em единичное}, {\em разбросанное}
явление, которое лишь в {\em действительности} приобретает {\em единство} и
значение, а также имеющее {\em всеобщую значимость наличное бытие}.

То более детальное отношение между двумя сторонами
гипотетического суждения, которое мы здесь рассматривали как отношение
условия к обусловленному, можно, однако, принимать также и за отношение
причины и действия, основания и следствия; это здесь безразлично; но
отношение обусловленности постольку более соответствует имеющемуся в
гипотетическом суждении и умозаключении соотношению, поскольку условие
выступает по существу как некоторое безразличное существование, основание
же и причина суть, напротив, нечто переходящее [в~иное] в силу своей
собственной природы; кроме того, условие есть более всеобщее определение,
так как оно охватывает собой обе стороны указанных отношений, потому что
действие, следствие и~т.~д. суть в такой же мере условия причины, основания
и~т.~д., в какой эти последние суть условия первых.

В гипотетическом умозаключении $A$ есть
{\em опосредствующее} бытие, поскольку оно, {\em во-первых}, есть
некоторое непосредственное бытие, некоторая безразличная действительность;
но, {\em во-вторых}, поскольку оно в такой же мере выступает и как некоторое
{\em в самом себе случайное},
снимающее себя бытие. То, что переводит условия в
действительность того нового образа, условиями которого они являются, это
"--- то обстоятельство, что они суть не бытие в смысле
абстрактного непосредственного бытия, а
{\em бытие в его понятии}, ближайшим образом {\em становление;} но так
как понятие уже не есть переход, то они суть, говоря более определенно,
{\em единичность} как
соотносящееся с собой {\em отрицательное} единство. "--- Условия
представляют собой некоторый разбросанный материал,
ожидающий и требующий своего применения; эта {\em отрицательность}
есть опосредствующее, представляет собой свободное единство
понятия. Она определяет себя как {\em деятельность}, так
как этот средний термин есть противоречие между
{\em объективной всеобщностью}
(или тотальностью тождественного содержания) и
{\em безразличной непосредственностью}.
"--- Поэтому указанный средний термин есть уже не только
внутренняя, но {\em сущая необходимость;} объективная всеобщность содержит
в~себе соотношение с самой собой как {\em простую непосредственность},
как бытие. В~категорическом умозаключении этот момент есть
ближайшим образом определение крайних терминов; но по отношению к
объективной всеобщности среднего термина он определяет себя как
{\em случайность} и тем самым как нечто лишь {\em положенное},
а также и снятое, т.~е. возвратившееся в понятие или в
средний термин как в единство, каковой средний термин сам в своей
объективности есть теперь также и бытие.

Заключение "--- <<{\em следовательно, В есть}>>
"--- выражает собой то же противоречие, а именно, что
$B$ есть некоторое {\em непосредственно}
сущее, но вместе с тем имеет бытие через нечно иное или
{\em опосредствовано}\pagenote{Момент непосредственности выражен
здесь словом <<есть>>,
момент опосредствования "--- словом <<следовательно>>.\label{bkm:bm61}}.
По своей форме оно есть поэтому то же самое понятие, которым
является средний термин. Оно лишь отличается от него, как
{\em необходимое} от {\em необходимости},
отличается совершенно поверхностным образом в форме
единичности, противостоящей всеобщности. Абсолютное
{\em содержание} $A$ и $B$ одно и то же;
это "--- лишь два различных названия одной и той же основы,
которыми пользуется {\em представление},
поскольку оно фиксирует явление разных образов наличного
бытия и отличает от необходимого его необходимость; но если бы эта
последняя была отделена от $B$, то $B$ не было бы необходимым.
Тут, стало быть, имеется тождество {\em опосредствующего} и
{\em опосредствованного}\pagenote{Для б\'{о}льшей наглядности возьмем
простенький пример, аналогичный тому примеру, которым Гегель иллюстрирует
тождество причины и действия в <<Учении о сущности>>: <<если идет дождь, то на
улице мокро; дождь идет; следовательно, на улице мокро>>. Гегель указывает,
что <<та же самая вода, которая составляет дождь, и есть мокротá>> (см. т.~I
<<Науки логики>>, стр.~453). Поэтому тут и получается тождество
опосредствующего (<<дождь идет>>) и опосредствованного (<<на улице
мокро>>).\label{bkm:bm62}}.

3. Гипотетическое умозаключение изображает ближайшим образом
{\em необходимое соотношение}
как связь через {\em форму} или {\em отрицательное единство},
подобно тому как категорическое умозаключение через
положительное единство изображало плотное
{\em содержание}, объективную всеобщность. Но {\em необходимость}
конденсируется в {\em необходимое; формальная деятельность}
перевода обусловливающей действительности в обусловленную есть {\em в~себе} то
единство, в котором определенности противоположности, перед тем
превратившиеся в свободное, безразличное наличное бытие,
{\em сняты} и различие между $A$ и $B$ стало пустым
словом. Она есть поэтому рефлектированное в~себя единство,
тем самым некоторое {\em тождественное}
содержание, и притом не только {\em в~себе}, но это также и {\em положено} в
рассматриваемом умозаключении, так как бытие $A$ есть вместе с тем
и не его собственное бытие, а бытие $B$ и наоборот,
вообще бытие одного есть бытие другого, и в заключении непосредственное
бытие или безразличная определенность прямо выступает как опосредствованная
определенность, "--- следовательно, внешность сняла себя, и ее
{\em ушедшее внутрь себя единство положено}.

Опосредствование умозаключения определило себя в силу этого
как {\em единичность}, {\em непосредственность} и
как {\em соотносящуюся с собой
отрицательность} или, иначе говоря, как различающее себя и
выходящее из этого различения, концентрирующее себя внутрь себя тождество,
"--- как абсолютную форму, и именно в силу этого "--- как объективную
{\em всеобщность}, тождественное с собой {\em содержание}. В~этом
определении умозаключение есть {\em разделительное умозаключение}.

\subsection[с) Разделительное умозаключение]{с) Разделительное умозаключение}

Подобно тому как гипотетическое умозаключение подчинено вообще
схеме второй фигуры {\em В "--- Е "--- О}, так разделительное
умозаключение подчинено схеме третьей фигуры формального умозаключения
{\em Е "--- В "--- О}. Но средний термин есть здесь
{\em наполненная формой всеобщность;}
он определил себя как {\em тотальность}, как {\em развернутую}
объективную всеобщность. Средний термин есть поэтому столь же
всеобщность, сколь и особенность и единичность. Как всеобщность он есть,
во-первых, субстанциальное тождество рода, но, во-вторых, такое тождество,
в которое {\em вобрана особенность}, но {\em как равная ему},
следовательно, как всеобщая сфера, содержащая свое целокупное
разложение на особенности; это "--- род, разложенный на свои
виды "--- такое $A$,
которое есть {\em как} $B$, {\em так и} $C$, {\em так и} $D$. Но
разложение на особенности, как различение, есть в такой же мере и
<<{\em либо-либо}>> наших $B$, $C$ и $D$, {\em отрицательное}
единство, {\em взаимное} исключение определений. "--- Далее, это
исключение есть не только взаимное, и определение не есть только
относительное, но столь же существенным образом и {\em соотносящееся с собой}
определение; особенное как {\em единичность}, с исключением {\em других}.

$A$ есть или $B$ или $C$ или $D$,
Но $A$ есть $B$, следовательно, $A$ не есть ни $C$, ни $D$>>.

Или также:

\begin{verse}
$A$ есть или $B$ или $C$ или $D$.\\
Но $A$ не есть ни $C$, ни $D$,\\
Следовательно, оно есть $B$.
\end{verse}

$A$ есть субъект не только в обеих посылках, но и в заключении.
В~первой посылке $A$ есть всеобщее, а
в своем предикате оно есть разложенная на тотальность своих особенностей,
своих видов, {\em всеобщая сфера;} во второй посылке $A$
выступает как {\em определенное} или
как некоторый вид; в заключении оно положено как исключающая, {\em единичная}
определенность. "--- Или же оно уже в меньшей
посылке положено как исключающая единичность, а в заключении оно положено
положительно как то определенное, что оно есть.

Стало быть, то, что здесь вообще выступает как {\em опосредствованное},
это {\em всеобщность} нашего $A$, которая опосредствована с его
{\em единичностью}. {\em Опосредствующим} же служит это же $A$,
которое составляет {\em всеобщую сферу}
своих обособлений и определено как некоторое {\em единичное}. Таким
образом, то, что является истиной гипотетического умозаключения, единство
опосредствующего и опосредствованного, {\em положено} в
разделительном умозаключении, которое поэтому вместе с тем уже и
{\em не есть умозаключение}.
А именно, средний термин, который положен в нем как
тотальность понятия, сам содержит в~себе оба крайних термина в их полной
определенности. Крайние термины в их отличии от этого среднего термина
выступают только как некоторая положенность, которая уже больше не обладает
никакой собственной, особо ей принадлежащей определенностью по сравнению со
средним термином.

Если рассматривать сказанное, принимая более определенно в
соображение гипотетическое умозаключение, то оказывается, что в последнем
имелось некоторое {\em субстанциальное тождество}, как {\em внутренняя} связь
необходимости, и некоторое, отличное от него,
{\em отрицательное единство}, "---
именно, деятельность или форма, переводившая одно наличное
бытие в другое. Разделительному же умозаключению свойственно вообще
определение {\em всеобщности;} его средним термином служит $A$,
как {\em род} и как совершенно {\em определенное;} в
силу этого единства вышеуказанное субстанциальное содержание, прежде бывшее
внутренним, теперь также и {\em положено}, и
обратно, положенность или форма есть уже не внешнее отрицательное единство
по отношению к некоторому безразличному наличному бытию, но тождественна с
тем плотным содержанием. Все формальное определение понятия положено в
своем определенном различии и вместе с тем в простом тождестве понятия.

Этим снял себя теперь {\em формализм процесса умозаключения},
и, стало быть, сняла себя также и субъективность
умозаключения и понятия вообще. Это формальное или субъективное состояло в
том, что опосредствующим для крайних терминов служит понятие как
{\em абстрактное} определение и потому оно {\em отлично} от самих
этих крайних терминов, чье единство оно составляет. Напротив, в завершении
умозаключения, когда объективная всеобщность положена вместе с тем и как
тотальность определений формы, различие опосредствующего и
опосредствованного отпало. То, что опосредствовано, само есть существенный
момент своего опосредствующего, и каждый момент выступает как тотальность
опосредствованных.

Фигуры умозаключения представляют каждую определенность
понятия {\em в отдельности}
как средний термин, который вместе с тем есть понятие как
{\em долженствование},
как требование, чтобы опосредствующее было тотальностью
понятия. Разные же роды умозаключения представляют собой ступени
{\em наполнения} или
конкретизации среднего термина. В~формальном умозаключении средний термин
полагается как тотальность лишь тем путем, что все определенности, но
каждая {\em в отдельности},
выполняют поочередно функцию опосредствования.
В~умозаключениях рефлексии средний термин выступает как единство,
{\em внешним образом} охватывающее собой определения крайних терминов.
В~умозаключении необходимости он определил себя так, что стал столь же
развернутым и целостным, сколь и простым единством, и этим форма
умозаключения, состоявшего в различии среднего термина от его крайних
терминов, сняла себя.

Тем самым понятие вообще реализовалось; выражаясь
определеннее, оно приобрело такую реальность, которая есть
{\em объективность}. {\em Ближайшая реальность} состояла в том, что
{\em понятие}, как отрицательное внутри себя единство, расщепляет себя и как
{\em суждение} полагает
свои определения в определенном и безразличном различии, а в умозаключении
противопоставляет им само себя. Если оно здесь, таким образом, еще есть
внутреннее этой своей внешности, то через ход развития умозаключений эта
внешность уравнивается с внутренним единством; разные определения через
опосредствование, в котором они оказываются чем-то единым первоначально
лишь в некотором третьем, теперь возвращаются в это единство, и внешность
тем самым на самой себе выявляет понятие, которое поэтому точно так же уже
больше не отличается от нее как внутреннее единство.

Но и наоборот, указанное определение понятия, которое мы
только что рассматривали как {\em реальность}, есть в
такой же мере и некоторая {\em положенность}. Ибо
не только в этом результате истиной понятия оказалось тождество его
внутренности и внешности, но уже и в суждении моменты понятия даже в своем
безразличии друг к другу остаются всё же такими определениями, которые
имеют значение лишь в своем соотношении друг с другом. Умозаключение есть
опосредствование, полное понятие в своей {\em положенности}. Его
движение есть снятие этого опосредствования, в котором ничто не есть само
по себе, а каждое определение имеет бытие лишь через посредство иного.
Поэтому как результат получается некоторая {\em непосредственность},
происшедшая через {\em снятие опосредствования}, некоторое {\em бытие},
которое вместе с тем тождественно с опосредствованием и есть
понятие, создавшее само себя из своего инобытия и в своем инобытии. Это
{\em бытие} есть поэтому некоторая {\em вещь}, сущая {\em в~себе и для
себя}, "--- {\em объективность}.

\bigskip
\clearpage

\part[Второй отдел\newline ОБЪЕКТИВНОСТЬ]{Второй отдел\newline Объективность}

В первой книге объективной логики абстрактное {\em бытие} было
изображено как переходящее в {\em наличное бытие}, но
также и возвращающееся в {\em сущность}. Во второй
книге мы видели, что сущность определяет себя так, что становится
{\em основанием}, в силу этого вступает в {\em существование}
и реализует себя в виде {\em субстанции}, но
снова возвращается в {\em понятие}.
Относительно же понятия мы прежде всего показали, что оно
определяет себя так, что становится {\em объективностью}.
Само собой явствует, что этот последний переход по своему
определению тождественен с тем, что в прежнее время выступало в
{\em метафизике} как {\em умозаключение} от {\em понятия}, а именно,
от {\em понятия бога} к {\em его существованию} или как так называемое
{\em онтологическое доказательство бытия
божия}. "--- Известно также, что возвышеннейшая мысль Декарта,
согласно которой бог есть то, {\em понятие чего заключает в~себе его
бытие}, после того как она опустилась до плохой формы
формального умозаключения, а именно, до формы упомянутого доказательства,
была в конце концов побеждена критикой разума и той мыслью, что невозможно
{\em выколупать существование из
понятия}. Кое-что из касающегося этого доказательства было
рассмотрено уже ранее. В~первой части, на стр.~67---72, трактуя о том, что
{\em бытие} исчезло в своей ближайшей противоположности "--- в
{\em небытии} и что истиной обоих оказалось {\em становление}, мы
обратили внимание читателя на смешение, получающееся в том случае, когда,
говоря о каком-нибудь определенном наличном бытии, фиксируют не его
{\em бытие}, а его {\em определенное содержание}
и потому воображают, что если такое-то {\em определенное содержание}
(например, сто талеров) сравнивается с некоторым другим
{\em определенным содержанием}
(например, с контекстом моего восприятия, с состоянием моего
имущества) и при этом обнаруживается различие между тем случаем, когда
первое содержание прибавляется ко второму, и тем случаем, когда оно не
прибавляется ко второму, то здесь речь идет якобы о различии бытия и
небытия или даже о различии бытия и понятия. Далее, мы там же, на стр.~95,
и во второй части, на стр.~462, осветили встречающееся в онтологическом
доказательстве определение {\em совокупности всех реальностей}. "---
Но существенного предмета этого доказательства, т.~е.
{\em связи понятия и существования},
касается только что законченное рассмотрение {\em понятия} и всего
хода развития, через который оно определяет себя к {\em объективности}.
Понятие, как абсолютно тождественная с собой отрицательность,
есть определяющее само себя; выше мы указали, что уже поскольку оно
раскрывается в единичности, чтобы стать {\em суждением}, оно
полагает себя как {\em реальное, сущее;} эта еще
абстрактная реальность завершает себя в {\em объективности}.

Если может показаться, что переход понятия в объективность
есть нечто иное, нежели переход от понятия бога к его существованию, то
надлежит, с одной стороны, принять в соображение, что определенное
{\em содержание}, бог, ни
в чем не изменяет хода логического развития и что онтологическое
доказательство есть лишь применение этого хода логического развития к тому
частному содержанию. С~другой же стороны, существенно важно вспомнить
сделанное выше замечание о том, что субъект лишь в своем предикате
приобретает определенность и содержание, а до этого, чем бы субъект ни был
для чувства, созерцания и представления, он для познания через понятие есть
лишь одно {\em название;} в предикате же вместе с определенностью начинается
также и {\em реализация} вообще.
"--- Но предикаты следует понимать как нечто такое, что само еще
замкнуто в понятии, стало быть, как нечто субъективное, с которым мы еще не
вышли к существованию; постольку, с одной стороны, {\em реализация} понятия,
конечно, еще не завершена в суждении. Но, с другой стороны, и голое
определение некоторого предмета через предикаты, если оно вместе с тем не
является реализацией и объективированием понятия, тоже есть нечто столь
субъективное, что оно еще далее не есть истинное познание и
{\em определение понятия} предмета, "--- есть нечто субъективное в смысле
абстрактной рефлексии и непостигнутых в понятии представлений. "---
Бог, как живой бог, а еще более, как абсолютный дух,
познается лишь в своем {\em действовании}.
Человеку уже рано дано было наставление познать его в его
{\em делах;} лишь из них могут проистекать те {\em определения},
которые именуются его {\em свойствами}, равно
как в них же содержится также и его {\em бытие}. Таким
образом, постигающее в понятии познание его {\em действования}, т.~е.
его самого, схватывает {\em понятие} бога в его {\em бытии}
и его бытие в его понятии. {\em Бытие} само по себе
или даже {\em наличное бытие}
есть столь скудное и ограниченное определение, что трудность
найти последнее в понятии несомненно могла возникнуть лишь оттого, что не
подвергли рассмотрению вопрос о том, что же такое само это
{\em бытие} или {\em наличное бытие}. "--- {\em Бытие}, как
совершенно {\em абстрактное,
непосредственное соотношение с самим собою}, есть не~что иное,
как абстрактный момент понятия, который есть
абстрактная всеобщность, дающая нам также и то, чего обыкновенно требуют от
бытия, а именно, нахождение {\em вне}
понятия; ибо насколько она есть момент понятия, настолько же
она есть и его различие или абстрактное суждение, в котором понятие
противопоставляет себя самому себе. Понятие, даже и как формальное, уже
непосредственно содержит в~себе
{\em бытие} в {\em более истинной} и {\em более богатой}
форме, поскольку понятие, как соотносящаяся с собой
отрицательность, есть {\em единичность}.

Но, конечно, трудность найти {\em бытие} в понятии
вообще и равным образом в понятии бога становится непреодолимой, если это
бытие должно быть таким бытием, которое встречается в
{\em контексте внешнего опыта} или выступает в {\em форме
чувственного восприятия}, подобно {\em ста талерам в составе моего
имущества}, чем-то схватываемым только рукой, а не духом,
видимым по существу внешнему, а не внутреннему оку, "--- если
бытием, реальностью, истиной именуется то, чем обладают вещи как
чувственные, временные и преходящие. "--- Если
философствование, рассуждая о бытии, не поднимается выше чувственности, то
к этому обстоятельству присоединяется еще и то, что, говоря о понятии, оно
не покидает чисто абстрактной мысли; эта последняя и противополагается бытию.

Привычка принимать понятие лишь за нечто столь одностороннее,
как абстрактная мысль, сделает для многих уже затруднительным признать то,
о чем говорилось выше, когда мы предложили рассматривать переход от
{\em понятия бога} к его {\em бытию} как {\em применение}
изображенного нами логического процесса объективирования
понятия. Однако, если, как это делается обыкновенно, соглашаются с тем, что
логическое, как формальное, составляет форму для познания всякого
определенного содержания, то следовало бы допустить по крайней мере и это
отношение {\em применения},
если только вообще не думают остановиться, как на чем-то
окончательном, на противоположности между понятием и объективностью, на
неистинном понятии и столь же неистинной реальности. "---
Однако, давая экспозицию {\em чистого понятия}, мы
указали еще далее, что оно есть само абсолютное, божественное понятие, так
что поистине тут имело бы место не отношение некоторого {\em применения}, а
указанный логический процесс представлял бы собою непосредственное
изображение самоопределения бога к бытию. Но по этому поводу следует
заметить, что, поскольку понятие должно быть изображено как понятие бога,
его следует понимать уже как принятое в {\em идею}. Упомянутое
чистое понятие проходит через конечные формы суждения и
умозаключения потому, что оно еще не положено как в~себе и для себя единое
с объективностью, а только еще находится в стадии становления последнею.
Таким образом, и эта объективность также еще не есть божественное
существование, еще не есть светящаяся в идее реальность. Но все же
объективность настолько же богаче и выше {\em бытия} или {\em существования},
фигурирующего в онтологическом доказательстве, насколько
чистое понятие богаче и выше, чем указанная метафизическая пустота
{\em совокупности всех реальностей}. "---
Однако я откладываю до другого раза предприятие осветить
ближе то многообразное недоразумение, которое внесено логическим
формализмом в онтологическое, равно как и в остальные так называемые
доказательства существования бога, а также и кантовскую критику их и
посредством восстановления их истинного значения возвратить лежащим в их
основании мыслям присущую им ценность и
достоинство\pagenote{Это намерение Гегель выполнил в
1829~г. в виде <<Лекций о доказательствах бытия божия>>. Перед самой своей
смертью, осенью 1831~г., он собирался издать их отдельной книгой, но не
успел сделать этого. Любопытна первая фраза этих лекций: <<Эти лекции могут
рассматриваться как дополнение к логике: они трактуют о некоторой
своеобразной форме тех основных моментов, которые выступают в логике>>
({\em Hegel}, Die Beweise
vom Dasein Gottes, Neu hrsg. v.~Lasson, Leipzig 1930, S.~1).
Здесь, пожалуй, ярче всего проявляется связь гегелевской
логики с самой настоящей поповщиной.\label{bkm:bm63}}.

Как уже было упомянуто, перед нами проходило уже несколько
форм непосредственности, но в разных определениях. В~сфере бытия она есть
само бытие и наличное бытие; в сфере сущности
"--- существование, а затем действительность и
субстанциальность; в сфере же понятия, кроме непосредственности как
абстрактной всеобщности, мы имеем теперь еще и объективность. "---
Там, где не нужна точность философских различений понятия,
можно употреблять эти выражения как синонимы: упомянутые определения
возникли ведь из необходимости понятия. {\em Бытие} есть вообще
{\em первая} непосредственность, а {\em наличное бытие} есть
она же с первой определенностью. {\em Существование}
вместе с вещью есть непосредственность, возникающая из {\em основания}, из
снимающего себя опосредствования простой рефлексии сущности.
{\em Действительность} же и {\em субстанциальность}
есть непосредственность, происшедшая из снятого различия
между еще несущественным существованием в качестве явления и его
существенностью. Наконец, {\em объективность} есть
такая непосредственность, к которой понятие определяет себя через снятие
своей абстрактности и опосредствования. "--- Философия имеет
право выбирать из языка повседневной жизни, созданного для мира
представлений, такие выражения, которые {\em кажутся близко подходящими}
к определениям понятия. Не может быть речи о том, чтобы {\em доказать}, что
с тем или иным выбранным из языка повседневной жизни словом также и в этой
повседневной жизни люди связывают то же самое понятие, для обозначения
которого его употребляет философия; ибо повседневная жизнь
обладает не понятиями, а представлениями, и уже сама философия должна
познать понятие того, что вне ее есть только представление. Поэтому мы
должны быть уже довольны, если представлению при употреблении тех его
выражений, которыми пользуются для обозначения философских определений,
предносится смутная мысль о различии последних, а это, как я полагаю, имеет
место относительно указанных выражений; в них можно распознать оттенки
представления, имеющие более близкую связь с соответствующими понятиями. "---
Быть может, многим труднее будет согласиться с тем, что нечто
может {\em быть} без того, чтобы {\em существовать;}
но по меньшей мере они не станут, например, смешивать <<{\em бытие}>>
в смысле связки суждения с выражением <<{\em существовать}>>, не
скажут: <<этот товар {\em существует}
дорогой, пригодный>> и так далее, <<деньги {\em существуют} металл
или металлические>> вместо: <<этот товар {\em есть} дорогой,
пригодный>>, <<деньги {\em суть} металл и~т.~д.>>\footnote{В~одном
французском донесении, в котором начальник экспедиции
сообщает, что он ожидает поднимающегося у острова обычно около утра ветра
для того, чтобы направиться к материку, встречается выражение: <<le vent
{\em ayant été} longtemps sans
{\em exister}>>\pagenote{Т.~е., <<так как ветра долгое время не было>>
(буквально: <<так как ветер долгое время {\em был} без того, чтобы
{\em существовать}>>).}\label{bkm:bm64};
здесь различие между глаголами être (быть) и
exister (существовать) возникло просто по аналогии с такими
оборотами речи, как <<il~a été longtemps sans
m'écrire>>\pagenote{Т.~е. <<он долгое время не писал мне>> (буквально:
<<он долгое время был без того, чтобы писать мне>>).\label{bkm:bm65}}.}
А такие выражения, как <<{\em быть}>> и <<{\em являться}>>,
<<{\em явление}>> и <<{\em действительность}>>, равно как и голое
<<{\em бытие}>> в противоположность <<{\em действительности}>>,
употребляются в различном смысле также и за пределами
философии; и еще большее смысловое различие имеется между всеми этими
выражениями и словом {\em <<объективность>>}. "---
Но даже если бы эти слова употреблялись как синонимы, то
философия и помимо этого должна иметь свободу пользоваться такого рода
пустым преизбытком языка для обозначения своих различений.

Говоря об аподиктическом суждении, в котором, как в завершении
суждения, субъект теряет свою определенность по отношению к предикату, мы
упомянули о возникающем отсюда всяком значении {\em субъективности}:
значении в смысле понятия и значении в смысле обыкновенно
противостоящей понятию внешности и
случайности\pagenote{См.~выше,
стр.~\pageref{bkm:bm66a} "--- \pageref{bkm:bm66b} (в~конце
параграфа о проблематическом суждении).\label{bkm:bm66}}.
Подобным же образом оказывается, что и объективность имеет
двоякое значение "--- значение чего-то {\em противостоящего}
самостоятельному {\em понятию}, но также и значение чего-то
{\em в-себе-и-для-себя-сущего}.
Поскольку объект в первом смысле противостоит тому <<Я~=~Я>>,
которое в субъективном идеализме высказывается как абсолютная истина, он
есть многообразный мир в его непосредственном наличном
бытии, мир, с которым <<Я>> или понятие вступает в бесконечную борьбу лишь
для того, чтобы через отрицание этого {\em ничтожного в~себе}
иного придать первичной самодостоверности означенного <<Я>>
{\em действительную истину} его равенства с самим
собою\pagenote{Имеется в виду субъективный идеализм Фихте.\label{bkm:bm67}}.
"--- В более неопределенном смысле объект означает вообще такого
рода предмет для какого-нибудь интереса и деятельности субъекта.

В противоположном же смысле слово <<объективное>> означает такое
{\em в-себе-и-для-себя-сущее},
которое свободно от ограничения и противоположности. Разумные
основоположения, совершенные произведения искусства и~т.~д. называются
{\em объективными},
поскольку они свободны и выше всякой случайности. Хотя
разумные (теоретические ли или нравственные) основоположения принадлежат
лишь сфере субъективного сознания, тем не менее то, что в них есть
в-себе-и-для-себя-сущего, называется объективным; познание истины видят в
том, чтобы объект познавался таким, каков он есть как объект, свободный от
примеси субъективной рефлексии, а праведное действие видят в следовании
объективным законам, которые не имеют субъективного происхождения, не могут
быть произвольными и не допускают трактовки, извращающей их необходимость.

На достигнутой теперь стадии нашего исследования объективность
имеет ближайшим образом значение {\em в-себе-и-для-себя-сущего бытия
понятия}, того понятия, которое сняло положенное в его
самоопределении {\em опосредствование} и сделало его {\em непосредственным}
соотношением с самим собой. Эта непосредственность тем самым
сама непосредственно и всецело проникнута понятием, равно как и тотальность
понятия непосредственно тождественна с его бытием. Но, далее, так как
понятие должно также и восстановить свободное для-себя-бытие своей
субъективности, то появляется такое отношение понятия как {\em цели} к
объективности, в котором непосредственность объективности становится чем-то
отрицательным по отношению к понятию и долженствующим подвергаться
определению через его деятельность и тем самым получает указанное выше
другое значение "--- чего-то самого по себе ничтожного,
поскольку оно противостоит понятию.

Итак, {\em во-первых}, объективность выступает в своей непосредственности,
моменты которой, в силу тотального характера всех моментов, в своем
самостоятельном безразличии друг к другу существуют {\em как объекты вне друг
друга}, а в своем отношении друг к другу обладают {\em субъективным единством}
понятия лишь в смысле {\em внутреннего} или, иначе говоря, {\em внешнего}
единства: это "--- {\em механизм}.

Но так как в нем, {\em во-вторых}, это единство оказывается {\em имманентным}
законом самих объектов, то их отношение становится их {\em своеобразным},
основанным на их законе, различием и таким соотношением, в
котором их определенная самостоятельность снимает себя: это "--- {\em химизм}.

{\em В-третьих}, это
существенное единство объектов именно поэтому положено как отличное от их
самостоятельности; оно есть субъективное понятие, но положенное как само по
себе соотносящееся с объективностью, "--- как
{\em цель}: это "--- {\em телеология}.

Так как цель есть понятие, положенное как соотносящееся в
самом себе с объективностью и снимающее через себя свой недостаток,
состоявший в том, что она была субъективной, то целесообразность, которая
была сперва {\em внешней} целесообразностью, становится благодаря реализации
цели {\em внутренней} целесообразностью и {\em идеей}.

\chapter[Первая глава Механизм]{Первая глава\newline Механизм}

Так как объективность есть возвратившаяся в свое единство
тотальность понятия, то этим положено некое непосредственное, которое само
по себе есть указанная тотальность, а также и {\em положено} как
таковая, но в котором отрицательное единство понятия еще не отделилось от
непосредственности этой тотальности, "--- или, иначе говоря, в
котором объективность еще не положена как {\em суждение}. Поскольку
она имманентно имеет внутри себя понятие, в ней имеется различие
последнего; но в силу характера объективной тотальности различенные суть
{\em полные} и {\em самостоятельные объекты},
которые поэтому даже и в своем соотношении ведут себя по
отношению друг к другу только как {\em самостоятельные} и
остаются во всяком соединении {\em внешними} друг другу. "--- Характер
{\em механизма} заключается в том, что, какое бы соотношение ни имело место
между теми, которые соединены, это соотношение есть {\em чуждое} им
соотношение, не касающееся их природы, и хотя бы даже это соотношение и
было связано с видимостью некоторого единого, оно все же остается не чем
иным, как {\em сложением, смесью, кучей} и~т.~д. Подобно {\em материальному}
механизму, {\em духовный}
механизм также состоит в том, что соотнесенные в духе моменты
остаются внешними и по отношению друг к другу и по отношению к нему самому.
{\em Механический способ представления}, {\em механическая
память}, {\em привычка}, {\em механический образ действия} означают, что в
том, что дух воспринимает или делает, недостает его
своеобразного проникновения и присутствия. Хотя его
теоретический или практический механизм не может иметь места без его
самодеятельности, без некоторого стремления и сознания, здесь, однако,
отсутствует свобода индивидуальности, и так как она тут не обнаруживается,
то такое действие является чисто внешним.

\section[А. Механический объект]{А. Механический объект}

Объект, как выяснилось, есть такое {\em умозаключение},
опосредствование которого выравнилось и потому стало непосредственным
тождеством. Он поэтому есть в~себе и для себя всеобщее; всеобщность не в
смысле некоторой общности свойств, а такая всеобщность, которая пронизывает
собой особенность и есть в ней непосредственная единичность.

1. Поэтому объект, во-первых, не различает себя на {\em материю} и
{\em форму}, из которых первая была бы самостоятельным всеобщим в объекте,
а вторая "--- его особенностью и единичностью; такого абстрактного
различия единичности и всеобщности в нем согласно его понятию не имеется;
если его рассматривают как материю, то его следует брать как в самой себе
оформленную материю. Подобным же образом его можно определять как вещь,
обладающую свойствами, как целое, состоящее из частей, как субстанцию,
обладающую акциденциями, и по прочим отношениям рефлексии; но эти отношения
уже вообще потонули в понятии; объект поэтому не обладает ни свойствами, ни
акциденциями, ибо они отделимы от вещи или субстанции, в объекте же
особенность безоговорочно рефлектирована в тотальность. В~частях целого
имеется, правда, та самостоятельность, которая присуща различиям объекта,
но эти различия сами суть по существу с самого начала объекты, тотальности,
которые не имеют, подобно частям, этого рода определенности по отношению к
целому.

Поэтому объект ближайшим образом {\em неопределенен}
постольку, поскольку он не заключает в~себе никакой
определенной противоположности; ибо он есть опосредствование, слившееся в
непосредственное тождество. Поскольку же {\em понятие существенным образом
определено}, объект обладает определенностью как некоторым,
хотя и полным, но помимо этого {\em неопределенным},
т.~е. {\em лишенным отношения, многообразием}, образующим такую тотальность,
которая тоже ближайшим образом не имеет дальнейших определений;
{\em стороны, части},
которые могут быть различены в объекте, принадлежат некоторой
внешней рефлексии. Указанное совершенно неопределенное различие состоит
поэтому лишь в том, что имеются {\em многие} объекты,
каждый из которых содержит в~себе свою определенность рефлектированной
только в свою всеобщность и не обладает свечением {\em во-вне}. "--- Так как
для него эта неопределенная определенность существенна, то он есть в самом
себе такое {\em множество} и должен поэтому рассматриваться как нечто
{\em составное}, как {\em агрегат}. "--- Он,
однако, не состоит из {\em атомов},
ибо последние не суть объекты, так как они не суть
тотальности. {\em Лейбницевская монада}
имела бы большее право считаться объектом, так как она есть
тотальность представления о мире, но, будучи замкнутой в своей
{\em интенсивной субъективности}, она должна быть по крайней мере существенно
{\em единой} внутри себя. Однако монада как {\em исключающее
одно} есть лишь {\em допущенный рефлексией}
принцип. Но монада есть объект отчасти постольку, поскольку
основание ее многообразных представлений (развитых, т.~е.
{\em положенных} определений ее лишь {\em в-себе}-сущей
тотальности) лежит {\em вне ее},
отчасти же постольку, поскольку для монады точно так же
безразлично, будет ли она или не будет составлять {\em вместе с другими}
некоторый объект; стало быть, на самом деле это не есть нечто
{\em исключающее, определенное само по себе}.

2. А~так как объект есть тотальность {\em определенности} (des
Bestim\-mtseins), но в силу своей неопределенности и
непосредственности не есть {\em отрицательное единство}
этой определенности, то объект {\em безразличен} к
{\em определениям} (как {\em единичным},
определенным в~себе и для себя), равно как и сами эти
определения {\em безразличны}
друг к другу. Эти определения поэтому не могут быть
постигнуты ни из объекта, ни друг из друга; его тотальность есть форма
всеобщей рефлектированности его многообразия в самоё по себе неопределенную
единичность вообще. Следовательно, определенности, которыми объект
обладает, ему, правда, присущи, но {\em форма}, образующая
собой их различие и связывающая их в некоторое единство, есть внешняя,
безразличная форма; все равно, есть ли она некоторая смесь, или, далее,
некоторый {\em порядок}, известное
{\em расположение} частей
и сторон, все это "--- соединения, которые безразличны
соотнесенным таким путем моментам.

Объект, стало быть, подобно какому-либо наличному бытию
вообще, имеет определенность своей тотальности
{\em вне себя}, в {\em других} объектах,
эти последние в свою очередь имеют эту определенность точно так же
{\em вне себя} и так
далее до бесконечности. Возвращение в~себя этого выхода в бесконечное
должно быть, правда, равным образом допущено, и его следует представлять
себе как некоторую {\em тотальность}, как некоторый {\em мир;}
однако этот мир есть не~что иное, как всеобщность, замкнутая
внутри себя через неопределенную единичность, "--- некоторая {\em вселенная}.

Следовательно, поскольку объект в своей определенности вместе,
с тем безразличен к ней, он через себя самого отсылает, что касается своей
определяемости, {\em за свои пределы},
указует опять-таки на объекты, которым, однако, подобным же
образом {\em безразлично то
обстоятельство, что они суть определяющие}. Здесь поэтому
нигде нет принципа самоопределения;
{\em детерминизм} "--- та
точка зрения, на которой стоит познание, поскольку ему объект (в~том виде,
в каком он здесь пока что получился) представляется истиной, "---
указывает для каждого определения объекта определение
некоторого другого объекта, но этот другой объект равным образом
безразличен как к своей определяемости (детерминированности), так и к
своему активному поведению. "--- Детерминизм в силу этого сам
тоже столь неопределенен, что ему приходится шествовать вперед до
бесконечности; он может по произволу остановиться на любом пункте и при
этом чувствовать удовлетворение, потому что тот объект, к которому он
перешел, как некоторая формальная тотальность, замкнут внутри себя и
безразличен к определяемости через другой объект. Поэтому {\em объяснение}
определения какого-нибудь объекта и совершаемое для этой цели
движение вперед этого представления есть лишь {\em пустое слово}, так
как в другом объекте, к которому оно переходит, нет никакого самоопределения.

3. Так как {\em определенность} какого-либо объекта {\em лежит}
{\em в некотором другом объекте},
то не имеется никакой определенной разницы между ними;
определенность лишь {\em удвоена},
выступая сперва в одном объекте, а затем в другом; она
безоговорочно есть лишь нечто {\em тождественное}, и
объяснение или постижение постольку {\em тавтологично}. Эта
тавтология представляет собой внешнее, пустое блуждание туда и сюда; так
как определенность не получает от безразличных к ней объектов никакого
своеобразного различия и потому лишь тождественна, то имеется лишь {\em одна}
определенность, и то обстоятельство, что она двойная, именно
и выражает собой эту внешность или ничтожество какого-либо различия. Но
вместе с тем объекты {\em самостоятельны} по
отношению друг к другу, и поэтому они остаются в том тождестве всецело
{\em внешними} друг другу. "--- Тем самым имеется {\em противоречие} между
полным {\em безразличием} объектов друг к другу и {\em тождеством} их
{\em определенности} или, иначе говоря, противоречие их полной
{\em внешности} в {\em тождестве} их определенности. Это противоречие есть,
таким образом, {\em отрицательное единство} многих безоговорочно
отталкивающихся внутри его объектов "--- {\em механический процесс}.

\section[В. Механический процесс]{В. Механический процесс}

Если объекты рассматриваются лишь как замкнутые внутри себя
тотальности, то они не могут действовать друг на друга. В~этом определении
они суть то же самое, что {\em монады},
которые именно поэтому мыслились как не оказывающие ни
малейшего воздействия друг на друга. Но именно вследствие этого понятие
монады есть неудовлетворительная рефлексия. Ибо, во-первых, она есть
некоторое {\em определенное} представление о своей лишь {\em в-себе-}сущей
тотальности; как {\em известная степень} развития и {\em положенности} своего
представления о мире, она есть нечто {\em определенное;} но,
будучи замкнутой внутри себя тотальностью, она вместе с тем и безразлична к
этой определенности; эта определенность есть поэтому не ее собственная, а
{\em положенная} через посредство чего-то {\em иного}
объекта. Во-вторых, монада есть некое {\em непосредственное}
вообще, поскольку, по мысли Лейбница, она есть нечто лишь
{\em представляющее;} ее соотношение с собой есть поэтому
{\em абстрактная всеобщность;} в силу этого она есть некоторое
{\em открытое} {\em для других наличное бытие}. "---
Чтобы утвердить свободу субстанции, недостаточно представлять
себе ее как такую тотальность, которая, будучи {\em полной внутри себя},
не нуждается в получении чего-нибудь извне. Напротив, как раз
это чуждое понятию, только представляющее соотношение с собой самой и есть
некоторая {\em пассивность} по отношению к другому. "--- Точно так же и
{\em определенность}, будем ли мы ее понимать как определенность некоторого
{\em сущего} или как определенность некоторого {\em представляющего},
как некоторую {\em степень} собственного, идущего изнутри развития, есть нечто
{\em внешнее;} достигаемая развитием {\em степень} имеет свою
{\em границу} в чём-то {\em ином}. Переносить
взаимодействие субстанций в некоторую
{\em предустановленную гармонию}
означает не~что иное, как превращать его в некоторую
предпосылку, т.~е. делать его чем-то таким, что изымается из-под власти
понятия. "--- Потребность избежать признания {\em воздействия}
субстанций друг на друга основывалась [у~Лейбница] на том,
что в основу был положен момент абсолютной
{\em самостоятельности} и первоначальности. Но так как этому
{\em в-себе-бытию} не соответствует {\em положенность},
степень развития, то оно именно поэтому имеет свое основание в чём-то ином.

Касательно отношения субстанциальности мы в свое время
показали, что оно переходит в отношение причинности. На сущее имеет здесь
определение уже не некоторой {\em субстанции}, a некоторого {\em объекта;}
отношение причинности потонуло в понятии; первоначальность
одной субстанции по отношению к другой оказалась видимостью, а ее
действование оказалось переходом в противоположное. Вследствие этого
отношение причинности не обладает никакой объективностью. Поэтому,
поскольку некоторый объект положен в форме субъективного единства как
действующая причина, то это теперь признается уже не {\em первоначальным}
определением, а чем-то {\em опосредствованным;}
действующий объект имеет это свое определение лишь через
посредство некоторого другого объекта. "--- {\em Механизм}, так как он
принадлежит к сфере понятия, положил в нем (в~механизме) то, что оказалось
истиной отношения причинности, а именно, что причина, долженствующая быть
чем-то в-себе-и-для-себя-сущим, есть по существу столь же и действие,
положенность. Поэтому в механизме причинность объекта есть непосредственно
некоторая не-первоначальность; объект безразличен к этому своему
определению; то обстоятельство, что он есть причина, является для него
поэтому чем-то случайным. "--- Постольку можно было бы сказать,
что причинность субстанций есть {\em лишь нечто представляемое}.
Но именно эта представляемая причинность и есть механизм, так
как об состоит в том, что причинность, как {\em тождественная}
определенность разных субстанций и, стало быть, как исчезание
их самостоятельности в этом тождестве, есть некоторая {\em голая положенность;}
объекты безразличны к этому единству и сохраняют себя в
противовес ему. Но в такой же степени и эта их безразличная
{\em самостоятельность} есть тоже голая {\em положенность;} они
поэтому способны {\em смешиваться} друг с другом и {\em сочетаться в агрегаты}
и в виде {\em агрегата} становиться {\em единым
объектом}. В~силу этого безразличия как к своему переходу,
так и к своей самостоятельности субстанции суть {\em объекты}.

\subsection[а) Формальный механический процесс]{а) Формальный механический процесс}

Механический процесс есть полагание того, что содержится в понятии механизма,
стало быть, ближайшим образом полагание некоторого {\em противоречия}.

1. Воздействие объектов друг на друга, как это вытекает из
вскрытого выше понятия, означает
{\em полагание тождественного}
соотношения объектов. Это полагание состоит лишь в том, что
той определенности, которую
причиняют\pagenote{В~издании 1841~г. (а~также и у Лассона)
вместо <<bewirkt wird>> (как напечатано в изданиях 1816 и 1834 гг.)
по ошибке напечатано <<Bеstimmt wird>>. То обстоятельство, что Лассон не
указывает этого места в своем перечне разночтений, свидетельствует о том,
что он в основу своего издания положил текст издания 1841~г. и недостаточно
тщательно сверил его с изданиями 1816 и 1834~гг., содержащими меньшее
количество ошибок и опечаток. Об этом свидетельствует также и случай,
указанный в примечании \ref{bkm:bm30}, и ряд других мест текста.
Ср. также примечание \ref{bkm:bm72}.\label{bkm:bm68}},
дается форма
{\em всеобщности}, а это
есть {\em сообщение},
которое не имеет перехода в противоположное. "---
{\em Духовное сообщение},
и без того совершающееся в такой стихии, которая есть
всеобщее в форме всеобщности, есть само по себе некоторое
{\em идеализованное}
соотношение, в котором
{\em какая-нибудь определенность
непрерывно продолжается} непомутненной от одного лица к
другому и становится всеобщим достоянием без всякого изменения, подобно
тому как благоухание свободно распространяется в не оказывающей
противодействия атмосфере. Но и при сообщении, имеющем место между
материальными объектами, их определенность, так сказать,
{\em распространяется вширь}
таким же идеализованным образом; ведь личность представляет
собой бесконечно более интенсивную
{\em твердость},{\em
неподатливость}, чем объекты. Та формальная
тотальность объекта вообще, которая безразлична к определенности и, стало
быть, не есть самоопределение, делает его неразличенным от другого объекта
и потому делает его воздействие прежде всего некоторым беспрепятственным
непрерывным продолжением определенности одного в другом.

В сфере духовного имеется бесконечно многообразное содержание,
способное быть сообщенным, когда оно, принятое в интеллект, получает ту
{\em форму} всеобщности,
в которой оно становится сообщимым. Но таким всеобщим, которое всеобще не
только через посредство формы, а также в~себе и для себя, служит как в
духовной, так и в телесной области
{\em объективное} как
таковое, по сравнению с которым единичность как внешних объектов, так и лиц
есть нечто несущественное, не могущее оказать ему никакого сопротивления.
Законы, нравы и вообще разумные представления суть в сфере духовного такое
способное быть сообщаемым содержание, пронизывающее собою индивидуумов
бессознательным образом и проявляющее в них свою силу. В~области телесного
это "--- движение, теплота, магнетизм, электричество и~т.~п.,
которые, даже если их угодно представлять себе как вещества или материи,
все же должны быть определены как
{\em невесомые} деятели
"--- деятели, не обладающие тем из свойств материальности,
которое обосновывает ее
{\em диференциацию на единичности} (ihre Verein\-zelung).

2. Но если в воздействии объектов друг на друга прежде всего
полагается их {\em тождественная}
всеобщность, то столь же необходимо положить и другой момент
понятия, {\em особенность;}
поэтому объекты доказывают также и свою
{\em самостоятельность},
сохраняют себя как внешние друг другу и в указанной
всеобщности восстанавливают
{\em единичность}. Это
восстановление есть {\em реакция}
вообще. Прежде всего ее не следует понимать
как {\em голое упразднение}
действия (der Aktion) и сообщенной
определенности; сообщенное, как всеобщее, положительно наличествует в
особенных объектах и {\em обособляется}
лишь в отношении их разности. Постольку сообщенное остается,
следовательно, тем, что оно есть; оно только
{\em распределяется}
между объектами или, иначе говоря, определяется их частными
различиями. "--- Причина теряется в своем ином, в действии,
активность причиняющей субстанции теряется в ее действовании;
{\em воздействующий же объект}
становится лишь некоторым
{\em всеобщим;} его
действование есть ближайшим образом не потеря его определенности, а
некоторая {\em партикуляризация},
в силу которой он, который первоначально был той целой,
{\em единичной} в нем
определенностью, становится теперь некоторым ее
{\em видом}, и лишь этим
{\em определенность}
положена теперь как некое всеобщее. И~то и другое
"--- возведение единичной определенности во всеобщность (в
сообщении) и ее партикуляризация или низведение ее, которая до того была
только единой, до уровня некоторого вида (в~распределении)
"--- есть одно и то
же\pagenote{Для иллюстрации категорий
<<сообщения>> и <<распределения>> какой-нибудь определенности можно
воспользоваться примером, приводимым у Гегеля в параграфе о
<<специфицирующей мере>> (см. т.~I <<Науки логики>>, стр.~269---270). Нагретый
воздух {\em сообщает} теплоту находящимся в нем предметам. Эти последние
воспринимают температуру окружающего их воздуха в меру своей теплоемкости.
Температура воздуха {\em распределяется} между находящимися в нем предметами.
Она в одно и то же время становится {\em всеобщей}({\em генерализируется}),
распространяясь на все другие находящиеся здесь тела, и
{\em партикуляризируется}, будучи воспринимаема каждым телом по-разному,
в меру своей специфической теплоемкости.\label{bkm:bm69}}.

{\em Реакция} равна
{\em действию} (der Aktion). "---
Это проявляется,
{\em во-первых},
так, что другой объект
{\em вобрал в~себя} все
всеобщее и, таким образом, есть теперь активное по отношению к первому.
Таким образом, его реакция есть то же самое, что действие, "---
{\em обратное отталкивание толчка}.
{\em Во-вторых},
сообщенное есть объективное; оно, следовательно,
{\em остается}
субстанциальным определением объектов при их предположенной
разности; всеобщее, стало быть, вместе с тем специфицируется в них, и
поэтому каждый объект не просто отдает обратно все действие, а получает и
свою специфическую долю. Но,
{\em в-третьих}, реакция
есть постольку {\em совершенно
отрицательное действие}, поскольку каждый объект, в силу
{\em эластичности своей
самостоятельности}, выталкивает положенность в нём иного и
сохраняет свое соотношение с собой. Та специфическая
{\em особенность}
сообщенной определенности в объектах, которую мы выше назвали
видом, возвращается назад к
{\em единичности}, и
объект утверждает свою внешность по отношению к
{\em сообщенной ему всеобщности}.
Действие переходит вследствие этого в
{\em покой}. Оно
оказывается некоторым лишь
{\em поверхностным},
преходящим изменением в замкнутой внутри себя безразличной
тотальности объекта,

3. Это возвращение назад составляет
{\em продукт}
механического процесса. {\em Непосредственно} объект {\em предположен}
как единичный; далее "--- как особенный по отношению к другим, а,
в-третьих, "--- как безразличный к своей особенности, как всеобщий.
{\em Продуктом} служит эта {\em предположенная} тотальность понятия,
теперь уже как {\em положенная}. Это "--- заключение\pagenote{В~немецком
тексте всех изданий напечатано: <<Ег ist der Schlusssatz>>.
Повидимому, это опечатка, вместо <<Es ist\ldots>>\label{bkm:bm70}},
в котором сообщенное всеобщее сомкнуто с единичностью через
особенность объекта; но вместе с тем {\em опосредствование}
положено в покое как такое, которое {\em сняло} себя, или,
иначе говоря, как состоящее в том, что продукт безразличен к этой своей
определяемости, и полученная определенность есть в нем внешняя определенность.

Согласно этому продукт есть то же самое, что впервые входящий
в процесс объект. Но вместе с тем он {\em определен} лишь
через это движение; механический объект есть вообще
{\em объект лишь как продукт}, ибо то, что он {\em есть}, в нем имеется
лишь {\em через опосредствование
чем-то иным}. Таким образом, как продукт, он есть то,
чем он должен был быть сам по себе; а именно, он есть нечто
{\em составное}, {\em смешанное}, известный {\em порядок} и {\em расположение}
частей, вообще нечто такое, определенность чего есть не
самоопределение, а нечто {\em положенное}.

С другой стороны, {\em результат} механического процесса вместе с тем еще
{\em не имеется уже до него самого;} его {\em конец} не наличествует в его
{\em начале}, как это имеет место относительно цели. Продукт есть некоторая
определенность в объекте как положенная {\em извне}. По {\em понятию}
этот продукт поэтому несомненно есть то же самое, что объект
есть уже с самого начала. Но вначале внешняя определенность еще не
выступает как {\em положенная}. Результат есть постольку нечто
{\em совершенно иное}, чем первое наличное бытие объекта, и выступает
как нечто для него всецело случайное.

\subsection[b) Реальный механический процесс]{b) Реальный механический процесс}

Механический процесс переходит в {\em покой}. Ведь та
определенность, которую объект получает через механический процесс, есть
лишь некоторая {\em внешняя}
определенность. Столь же внешним для объекта оказывается и
сам этот покой, так как покой есть определенность, противоположная
{\em действованию}
объекта, но обе эти определенности для объекта безразличны;
покой поэтому тоже можно рассматривать как произведенный некоторой
{\em внешней} причиной,
точно так же как для объекта было безразлично, быть ли действующим или нет.

Далее, так как определенность есть теперь
{\em положенная}
определенность и понятие объекта,
{\em пройдя через опосредствование,
возвратилось к себе самому}, то объект содержит в~себе
определенность как рефлектированную в~себя. Теперь объекты в
механическом процессе и сам этот последний имеют поэтому ближе определенное
отношение. Они суть не просто разные, но и
{\em определенно различные}
между собой. Результатом формального процесса, которым, с
одной стороны, служит лишенный определений покой, является, стало быть, с
другой стороны, в силу рефлектированной в~себя определенности,
{\em распределение той
противоположности}, которую объект вообще содержит в~себе,
между многими механически относящимися друг к другу объектами. Объект,
который, с одной стороны, есть нечто лишенное определений, ведущее себя
{\em неэластично} и
{\em несамостоятельно},
обладает, с другой стороны,
{\em самостоятельностью, не могущей быть
прорванной }другими объектами. Объекты имеют теперь также и
{\em по отношению друг к другу}
эту более определенную противоположность
{\em самостоятельной единичности и
несамостоятельной всеобщности}. "--- Более детальное (nähere)
различие можно понимать как чисто
{\em количественное}
различие в величине
{\em массы} тел или как
различие {\em интенсивности}
или разнообразными другими способами. Вообще же это различие
не следует фиксировать только в той абстрактности, в какой оно было
формулировано выше: ведь обе стороны его суть как объекты также и
{\em положительные}
самостоятельные.

Первым моментом этого реального
{\em процесса} служит,
как и ранее, {\em сообщение}.
{\em Более слабый} объект
может быть охвачен и пронизан {\em более
сильным} лишь постольку, поскольку первый объект
воспринимает его в~себя и образует с ним
{\em одну сферу}. Подобно
тому как в материальной области слабое обеспечено против несоразмерно
сильного (так, например, свободно висящий в воздухе платок не пробивается
ружейной пулей или "--- другой пример "--- слабая
органическая восприимчивость возбуждается не столько сильными, сколько
слабыми раздражителями), "--- точно так же совершенно слабый
дух более обеспечен против сильного, чем такой дух, который ближе стоит к
этому сильному духу; если представить себе нечто совсем глупое или
неблагородное, то высокий ум или благородство не могут произвести на него
никакого впечатления; единственно последовательное средство
{\em против} разума
состоит в том, чтобы совсем не связываться с ним. "---
Поскольку несамостоятельное не может слиться с
самостоятельным и между ними не может иметь места никакое сообщение,
постольку самостоятельное не может оказывать также и никакого
{\em сопротивления},
т.~е. не может специфицировать для себя то всеобщее, которое
ему сообщается. "--- Если бы они не находились в одной сфере,
то их соотношение друг с другом было бы бесконечным
суждением, и между ними не был бы возможен никакой процесс.

{\em Сопротивление} есть
ближайший следующий момент преодолевания одного объекта другим, будучи
начальным моментом распределения сообщенного всеобщего и полагания
соотносящейся с собой отрицательности, т.~е. подлежащей восстановлению
единичности. Сопротивление
{\em преодолевается},
поскольку его определенность не
{\em адекватна}
сообщаемому всеобщему, воспринятому объектом и
долженствующему сингуляризироваться в последнем. Его относительная
несамостоятельность проявляется в том, что его
{\em единичность} не
обладает {\em вместимостью для
сообщаемого} и потому разрывается последним, так как он не
может конституироваться в этом всеобщем как
{\em субъект}, не может
сделать его своим {\em предикатом}. "---
Насилие над объектом есть лишь с этой второй стороны нечто
{\em чуждое} для него.
{\em Мощь} становится
{\em насилием} вследствие
того, что она, будучи некоторой объективной всеобщностью,
{\em тождественна с природой}
объекта, но ее определенность или отрицательность не есть та
его собственная {\em отрицательная
рефлексия} в~себя, по которой он есть нечто единичное.
Поскольку отрицательность объекта не рефлектирует себя в~себя
{\em в мощи} и поскольку
мощь не есть его собственное соотношение с собой, она есть в отличие от
последнего лишь {\em абстрактная}
отрицательность, проявлением которой служит гибель.

Мощь, взятая как
{\em объективная всеобщность}
и как насилие {\em над}
объектом, есть то, что называют
{\em судьбой}. Понятие
судьбы относится к сфере механизма, поскольку судьбу называют
{\em слепой}, т.~е.
поскольку ее {\em объективная
всеобщность} не познается субъектом в ее специфическом
своеобразии. "--- Чтобы сделать кое-какие замечания
по этому поводу, мы должны сказать, что судьбой живущего вообще служит
{\em род}, который
проявляется через преходимость живых индивидуумов, свойственную им в их
{\em действительной единичности},
а не как роду. Просто как объекты, существа, обладающие лишь
жизнью, подобно остальным вещам, стоящим на низшей ступени, не имеют
судьбы; то, что с ними происходит, представляет собой случайность; но они
{\em в своем понятии как объекты}
суть {\em внешние себе;}
чуждая мощь судьбы есть поэтому всецело лишь их
{\em собственная, непосредственная
природа}, сама внешность и случайность. Судьбу в собственном
смысле имеет только самосознание, так как оно
{\em свободно} и поэтому
в {\em единичности}
своего <<я>> выступает безоговорочно
{\em в~себе и для себя},
может противопоставлять себя своей объективной всеобщности и
{\em отчуждаться} от нее.
Но вследствие самого этого обособления оно возбуждает против
себя механическое отношение некоей судьбы. Следовательно, для того чтобы
такого рода судьба могла получить власть над самосознанием, последнее
должно было дать себе какую-нибудь определенность, идущую против
существенной всеобщности, должно было совершить некоторое
{\em деяние}. Этим оно
сделало себя некоторым {\em особенным},
и это наличное бытие, как абстрактная всеобщность,
представляет собой вместе с тем ту сторону, которая открыта для сообщения
ему его отчужденной от него сущности; схваченное за эту сторону, оно и
вовлекается в
процесс\pagenote{Гегель имеет в виду главным
образом то понятие судьбы, которое выступает в античной трагедии. В~т.~III
<<Эстетики>>, в главе о драматической поэзии, Гегель более подробно
останавливается на характеристике этой <<судьбы>>. Он возражает против
понимания этой судьбы как слепого рока. <<Разумность судьбы, "---
говорит Гегель, "--- заключается именно в том,
что высшая сила, властвующая над отдельными богами и людьми, не может
потерпеть, чтобы силы, односторонне делающие себя самостоятельными и этим
преступающие границы своего права, равно как и проистекающие отсюда
конфликты, получили устойчивое существование>>
({\em Hegel}, Sämtliche Werke, hrsg. v.~Glockner, Bd.~XIV, S.~554).
И далее: <<В античной трагедии спасает и отстаивает гармонию нравственной
субстанции против напора делающих себя самостоятельными и потому вступающих
в коллизию частных сил вечная справедливость как {\em абсолютная мощь
судьбы}, и она, в силу внутренней разумности ее управления, доставляет нам
удовлетворение зрелищем самой гибели индивидуумов>> (там же, стр.~572).\label{bkm:bm71}}.
Народ, не совершающий никаких деяний, безупречен; он закутан
в объективную нравственную всеобщность и растворен в ней, лишен той
индивидуальности, которая движет неподвижное, дает себе некоторую
определенность, направленную во-вне, и абстрактную всеобщность, отделенную
от объективной всеобщности, вследствие чего, однако, и сам субъект
становится чем-то отчудившимся от своей сущности, некоторым
{\em объектом} и вступает
в отношение {\em внешности}
к своей природе и тем самым в отношение механизма.

\subsection[с) Продукт механического процесса]{с) Продукт механического процесса}

Продуктом
{\em формального}
механизма является объект вообще, безразличная тотальность, в
которой {\em определенность}
выступает как
{\em положенная}. Так как
вследствие этого объект вступил в процесс как нечто
{\em определенное}, то, с
одной стороны, при уничтожении этого определенного результатом является
{\em покой}, как
первоначальный формализм объекта, отрицательность его
для-себя-определяемости. Но, с другой стороны,
результатом служит
снятие\pagenote{В~изданиях 1816 и 1834~гг.: <<Anderer\-seits aber ist
{\em es} das Aufheben\ldots>>.
В издании 1841~г., а также у Лассона (у~последнего "--- без
указания разночтений)
пропущено слово <<es>>. Ср.~примечания
\ref{bkm:bm30} и~\ref{bkm:bm68}.\label{bkm:bm72}}
определяемости (des Bestimmt\-seins) как
{\em его (объекта) положительная
рефлексия} в~себя, ушедшая в~себя, определенность
(Bestimmt\-heit) или {\em положенная
тотальность понятия;} это
"--- {\em истинная единичность
}объекта. Объект, который сначала выступал в~своей
неопределенной всеобщности, а затем "--- как
{\em особенное},
определен теперь как
{\em объективно единичное;}
так что в нем снята вышеуказанная
{\em видимость той единичности},
которая есть лишь некоторая
{\em противополагающая}
себя субстанциальной всеобщности самостоятельность.

Эта рефлексия в~себя есть теперь, как выяснилось, объективная
единость объектов, представляющая собой индивидуальную самостоятельность,
"--- {\em центр}.
{\em Во-вторых},
рефлексия отрицательности есть такая всеобщность, которая
представляет собой не противостоящую определенности, а определенную внутри
себя, разумную судьбу, "--- такая всеобщность, которая
{\em обособляет себя в
}{\em самой себе},
спокойное, прочное в несамостоятельной особенности объектов и
их процессе различие, {\em закон}.
Этот результат есть истина и тем самым также и основа
механического процесса.

\section[С. Абсолютный механизм]{С. Абсолютный механизм}

\subsection[а) Центр]{а) Центр}

Пустое многообразие объекта теперь, во-первых, собрано в
объективную единичность, в простое определяющее само себя
{\em средоточие}.
Поскольку, во-вторых, объект, как непосредственная
тотальность, сохраняет свое безразличие к определенности, постольку
последняя выступает в нем также и как несущественная или, иначе говоря, как
{\em внеположность}
многих объектов. Первая, существенная определенность образует
собой, напротив, {\em реальный средний
термин} между многими механически действующими друг на друга
объектами, через который они сомкнуты
{\em в~себе и для себя},
и есть их объективная всеобщность. Всеобщность оказалась
сначала, в отношении {\em сообщения},
наличествующей лишь через
{\em полагание;} как
{\em объективная} же, она
есть пронизывающая, имманентная сущность объектов.

В материальном мире такой всеобщностью служит
{\em центральное тело},
которое есть {\em род}
(однако в смысле
{\em индивидуальной}
всеобщности) единичных объектов и их механического процесса.
Несущественные единичные тела ведут себя по отношению друг к другу как
{\em толкающие} и
{\em оказывающие давление;}
такого отношения нет между центральным телом и теми
объектами, сущность которых оно составляет; ибо их внешность уже не
образует собой их основного определения. Их тождеством с центральным телом
служит, стало быть, скорее покой, а именно,
{\em бытие в их центре;}
это единство есть их в-себе-и-для-себя-сущее понятие. Однако
это единство остается лишь некоторым
{\em долженствованием},
так как ему (единству) не соответствует та внешность
рассматриваемых объектов, которая пока что вместе с тем все еще положена
здесь. Свойственное им поэтому
{\em стремление} к центру
есть их абсолютная, а не положенная через
{\em сообщение}
всеобщность; она образует собой истинный, сам по себе
{\em конкретный} (а~не
{\em положенный извне})
покой, в который необходимо должен возвратиться процесс
несамостоятельности. "--- Поэтому, когда в механике
принимается, что приведенное в движение тело продолжало бы вообще двигаться
дальше по прямой линии до бесконечности, если бы оно не теряло своего
движения вследствие внешнего сопротивления, то это "--- пустая
абстракция. {\em Трение}
или какая-бы то ни было другая форма
сопротивления есть лишь проявление
{\em центральности;}
центральность и есть как раз то, что абсолютно приводит
обратно к себе движущееся тело; ибо то, обо что трется движущееся тело,
обладает силой сопротивления исключительно только благодаря своей единости
с
центром\pagenote{Как ни темны эти рассуждения
Гегеля о <<центре>> и <<центральности>>, в них есть определенное рациональное
зерно. Под <<центром>> и <<центральностью>> Гегель имеет в виду прежде всего
центр тяжести небесных тел, например, земли. Всякое движение, совершающееся
на поверхности земли или в окружающей землю атмосфере, находится в
необходимой и существенной зависимости от силы тяготения, направленной к
центру земли. То сопротивление движущемуся телу, которое оказывается
воздухом и поверхностью земли, в свою очередь обусловлено в конечном счете
силой тяготения и является существенно-необходимым моментом всякого
движения, рассматриваемого в земной механике, "--- таким
моментом, от которого нельзя отвлечься, не впадая в <<пустую абстракцию>>.
Рациональное зерно этих рассуждений Гегеля выступает особенно явственно,
если их сопоставить со следующими рассуждениями Энгельса в <<Диалектике
природы>>: <<Возьмем, "--- пишет Энгельс, "---
какую-нибудь телесную массу на самой нашей земле. Благодаря
тяжести она связана с землей, подобно тому как Земля, с своей стороны,
связана с Солнцем; но в отличие от земли эта масса неспособна на свободное
планетарное движение. Она может быть приведена в движение только при помощи
внешнего толчка. Но и в этом случае, по миновании толчка, ее движение
вскоре прекращается либо благодаря действию одной лишь тяжести, либо же
благодаря этому действию в соединении с сопротивлением среды, в которой
движется наша масса. Однако {\em и это сопротивление является в последнем
счете действием тяжести}, без которой Земля не имела бы никакой
сопротивляющейся среды, никакой атмосферы на своей поверхности>>
({\em Энгельс}, Диалектика природы, Партиздат, 1936, стр.~133---134.
Подчеркнуто мной. "--- $B$. {\em Б}.)\label{bkm:bm73}}.
"--- В области {\em духовного} центр и
единость с ним принимают более высокие формы; но то реальное единство
понятия, которое здесь есть пока что механическая центральность, необходимо
должно составлять основное определение также и там.

Постольку центральное тело перестало быть просто
{\em объектом}, так как в
последнем определенность есть нечто несущественное; ибо центральное тело
теперь уже обладает не только
{\em в-себе-бытием}
объективной тотальности, но и ее
{\em для-себя-бытием}.
Его можно поэтому рассматривать как некоторый
{\em индивидуум}. Его
определенность существенно отличается от голого
{\em порядка} или
{\em расположения} и
{\em внешней связи}
частей; она, как в-себе-и-для-себя-сущая определенность, есть
некоторая {\em имманентная}
форма, самоопределяющий принцип, которому объекты присущи и
через который они связаны в некоторое истинное одно.

Однако взятый таким образом, этот центральный индивидуум есть
пока что лишь такой {\em средний
термин}, который еще не имеет подлинных крайних терминов; но
как отрицательное единство тотального понятия, он расщепляет себя на
таковые. Или, иначе говоря: те внешние друг другу объекты, которые раньше
были несамостоятельными, определяются через возвращение понятия таким
образом, что оказываются тоже индивидуумами; тождество центрального тела с
собой, которое пока что еще есть некоторое {\em стремление}, обременено
{\em внешностью}, которой, ввиду того что она вобрана в его
{\em объективную единичность}, сообщена эта последняя.
Благодаря этой собственной
центральности они, поставленные вне того первого центра, сами суть центры
для несамостоятельных объектов. Эти вторые центры и несамостоятельные
объекты сомкнуты воедино вышеуказанным абсолютным средним
термином\pagenote{Это "--- гегелевская <<третья фигура>> силлогизма
({\em Е "--- В "--- О}),
причем в роли единичности (отдельности) выступают вторичные
или относительные центры, в роли всеобщности "--- первичный или
абсолютный центр тяжести, а в роли особенности (частности)
"--- несамостоятельные объекты. В~<<Философии природы>> Гегель
применяет эту логическую схему к взаимоотношению между членами солнечной
системы. Солнце выступает в роли всеобщности, Земля (а~также и другие
планеты) "--- в роли единичности, а <<несамостоятельные тела>>
(Луна и кометы) "--- в роли особенности.
См. {\em Гегель}, Соч., т.~II, стр.~136.\label{bkm:bm74}}.

Но относительные центральные индивидуумы сами также составляют
средний термин некоторого {\em второго
умозаключения}, который, с одной стороны, подведен под
некоторый более высокий крайний термин (а~именно, под объективную
{\em всеобщность} и {\em мощь} абсолютного
центра), а, с другой стороны, подводит под себя несамостоятельные объекты,
являясь носителем их поверхностной или формальной изолированности
(единичности)\pagenote{Это "--- гегелевская <<вторая фигура>>
силлогизма~({\em О "--- Е "--- В}).
В немецком тексте всех изданий конец этой фразы грамматически
искажен. Напечатано: <<deren\ldots Verein\-zelung
von ihr getragen {\em werden}>>.
Здесь надо либо вместо <<werden>> поставить <<wird>> (как это и
сделано в нашем переводе), либо после слова <<Verein\-zelung>> прибавить слова
вроде <<und Unselb\-standig\-keit>>
(или: <<und Aeusser\-lich\-keit>>).\label{bkm:bm75}}.
"--- Эти несамостоятельные объекты тоже образуют
собой средний термин некоторого
{\em третьего},
{\em формального умозаключения},
так как они суть связующее звено между абсолютной и
относительной центральной индивидуальностью постольку, поскольку последняя
имеет в них свою внешность, в силу которой
{\em соотношение с собой}
есть вместе с тем
{\em стремление} к
некоторому абсолютному средоточию. Формальные объекты имеют своею сущностью
тождественную {\em тяжесть}
своего непосредственного центрального тела, которому они
присущи как своему субъекту и как крайнему термину, представляющему собой
единичность; через посредство той внешности, которую они составляют, это
центральное тело подведено под абсолютное центральное тело; они,
следовательно, суть формальный средний термин, представляющий собой
{\em особенность}\pagenote{Это "--- гегелевская <<первая фигура>> силлогизма, ({\em Е "--- О "--- В}),
которую Гегель называет также <<формальным силлогизмом>>.\label{bkm:bm76}}.
"--- Абсолютный же индивидуум есть объективно-всеобщий средний
термин, смыкающий между собой и поддерживающий внутри-себя-бытие
относительного индивидуума и его
внешность\pagenote{Здесь опять речь идет о <<третьей фигуре>> силлогизма, о которой упоминалось
выше (см. примечание \ref{bkm:bm74}).\label{bkm:bm77}}.
"--- \label{bkm:bm52b}Подобным же образом и
{\em правительство},
{\em индивидуумы-граждане
}и {\em потребности}
или {\em внешняя жизнь}
единичных людей суть три термина, каждый из которых есть
средний термин для двух остальных.
{\em Правительство} есть
тот абсолютный центр, в котором один крайний термин
"--- единичные "--- сомкнут с другим крайним
термином "--- с их внешним
существованием\pagenote{Умозаключение <<третьей фигуры>> ({\em Е "--- В "--- О}).\label{bkm:bm78}}.
И точно так же
{\em единичные} суть
средний термин, они активизируют этот всеобщий индивидуум, осуществляя его
во внешнем существовании, и переводят свою нравственную сущность в крайний
термин
действительности\pagenote{Умозаключение <<второй фигуры>> ({\em B"--- Е "--- О}).\label{bkm:bm79}}.
Третье умозаключение есть формальное умозаключение или
умозаключение видимости и состоит в том, что единичные люди через свои
{\em потребности} и через
внешнее существование связаны с этой всеобщей абсолютной индивидуальностью;
это умозаключение, будучи чисто субъективным, переходит в те другие
умозаключения и в них имеет свою
истину\label{bkm:bm52c}\pagenote{Умозаключение <<первой фигуры>> ({\em Е "--- О "--- В}).\label{bkm:bm80}}.

Эта тотальность, моменты которой сами суть полные отношения
понятия, т.~е. сами суть
{\em умозаключения}, в
которых каждый из трех различенных объектов проходит через определение
среднего термина и крайних терминов, "--- эта тотальность
составляет {\em свободный механизм}.
В нем различенные объекты имеют своим основным определением
объективную всеобщность, а именно, тяжесть,
{\em пронизывающую} собой
все и сохраняющую себя
{\em тождественной} в
{\em обособлении}.
Соотношения {\em давления,
толчка, притяжения} и тому подобное, равно как и
{\em агрегаты} или
{\em смеси}, принадлежат
к отношению той внешности, которая обосновывает третье из
сопоставленных выше умозаключений.
{\em Порядок},
представляющий собой чисто внешнюю определенность объектов,
перешел в имманентное и объективное определение; последнее есть
{\em закон}.

\subsection[b) Закон]{b) Закон}

В законе явственно обнаруживается более определенное отличие
{\em идеализованной реальности}
объективности от
{\em внешней} реальности.
Объект, как {\em непосредственная}
тотальность понятия, еще не обладает внешностью, как отличной
от {\em понятия}, которое
еще не положено особо. Когда объект, пройдя через процесс, ушел внутрь
себя, перед нами появилась противоположность между
{\em простой центральностью}
и некоторой
{\em внешностью}, которая
теперь определена {\em как}
внешность, т.~е.
{\em положена} как нечто
не в-себе-и-для-себя-сущее. Вышеуказанная тождественность или
идеализованность индивидуальности есть в силу соотношения с внешностью
некоторое {\em долженствование;}
она есть в-себе-и-для-себя-определенное и самоопределяющее
единство понятия, каковому единству упомянутая внешняя реальность не
соответствует, вследствие чего эта внешняя реальность доходит лишь до
{\em стремления}. Но
индивидуальность {\em сама по себе есть
конкретный принцип отрицательного единства и как таковой}
сама есть
{\em тотальность}, есть
единство, расщепляющее себя на
{\em определенные различия понятия}
и остающееся в своей равной самой себе всеобщности; тем самым
индивидуальность есть средоточие,
{\em расширенное} внутри
своей чистой идеальности {\em через
различие}. "--- Эта соответствующая понятию реальность есть
{\em идеализованная}
реальность, отличная от той лишь стремящейся реальности;
различие, которое вначале представляет собой некоторую множественность
объектов, выступает здесь в своей существенности и вобрано в чистую
всеобщность. Эта реальная идеальность есть
{\em душа} развитой выше
объективной тотальности,
{\em в-себе-и-для-себя-определенное
тождество} системы.

Получается поэтому, что объективное
{\em в-себе-и-для-себя-бытие}
в своей тотальности выступает более определенно как
отрицательное единство центра, которое делит себя на
{\em субъективную индивидуальность}
и {\em внешнюю
объективность}, сохраняет в последней первую и определяет ее
в идеализованном различии. Это самоопределяющее единство, абсолютно
приводящее внешнюю объективность обратно в идеальность, есть принцип
{\em самодвижения;}
{\em определенность} этого
одушевляющего принципа, представляющая собой различие самого понятия, есть
{\em закон}. "--- Мертвый
механизм представлял собой рассмотренный выше механический
процесс объектов, которые непосредственно казались самостоятельными, но
именно поэтому на самом деле несамостоятельны и имеют свой центр вне себя.
Этот процесс, переходящий в
{\em покой}, обнаруживает
либо {\em случайность} и
неопределенную неодинаковость, либо
{\em формальное единообразие}.
Это единообразие есть
{\em правило}, но не
{\em закон}. Лишь
свободный механизм обладает некоторым
{\em законом},
собственным определением чистой индивидуальности или
{\em для-себя-сущего понятия;}
этот закон, как различие в самом себе, есть непреходящий
источник само себя возбуждающего движения; поскольку закон в идеальности
своего различия соотносится лишь с собой, он есть
{\em свободная необходимость}.

\subsection[с) Переход механизма]{с) Переход механизма}

Однако эта душа еще погружена в свое тело;
{\em отныне уже определенное},
но {\em внутреннее}
понятие объективной тотальности есть свободная необходимость
таким образом, что закон еще не противопоставил себя своему объекту; он
есть {\em конкретная}
центральность, как всеобщность,
{\em непосредственно}
распространенная в свою объективность. Вышеуказанная
идеальность имеет поэтому своим определенным различием не
{\em сами объекты;}
последние суть
{\em самостоятельные индивидуумы}
тотальности или же, "--- если бросить взгляд
назад на пройденную формальную ступень, "--- неиндивидуальные,
внешние {\em объекты}.
Закон им, правда, имманентен и составляет их природу и мощь,
но его различие замкнуто в его идеальность, и сами объекты не
диференцированы на идеализованные различия закона. Однако объект имеет свою
существенную самостоятельность исключительно только в идеализованной
центральности и ее законе; объект поэтому не в силах оказывать
сопротивление суждению понятия и сохранять свою абстрактную, неопределенную
самостоятельность и замкнутость. В~силу идеализованного, имманентного ему
различия его наличное бытие есть
{\em положенная}
{\em через понятие определенность}.
Его несамостоятельность перестала, таким образом, быть лишь
некоторым {\em стремлением к центру},
по отношению к которому он (именно потому, что его
соотношение с центром есть лишь некоторое стремление) имеет еще вид
(Erscheinung) некоторого самостоятельного внешнего объекта;
он теперь есть стремление к
{\em определенно ему противоположному
объекту}, равно как и сам центр в силу этого распался, и его
отрицательное единство перешло в
{\em объективированную
противоположность}. Центральность есть теперь
поэтому
{\em соотношение} этих
друг к другу отрицательных и напряженных объективностей. Таким образом,
свободный механизм определяет себя как
{\em химизм}.

\chapter[Вторая глава Химизм]{Вторая глава\newline Химизм}

Химизм составляет в целом объективности момент суждения, т.~е.
момент ставшего объективным различия и процесса. Так как он уже начинает с
определенности и положенности и химический объект есть вместе с тем
объективная тотальность, то его дальнейшее течение отличается простотой и
вполне определено его предпосылкой.

\section[А. Химический объект]{А. Химический объект}

Химический объект отличается от механического тем, что
последний есть тотальность, безразличная к определенности; в химическом же
объекте {\em определенность}
и, стало быть,
{\em соотношение с другим},
а также вид и способ этого соотношения принадлежат, напротив,
к его природе. "--- Эта определенность есть по существу вместе
с тем {\em обособление},
т.~е. она вобрана во всеобщность; она, таким образом, есть
{\em принцип}, "---
{\em всеобщая определенность},
определенность не только
{\em одного единичного объекта},
но и {\em другого}.
В объекте теперь поэтому диференцируется его понятие, как
внутренняя тотальность обеих определенностей, и та определенность, которая
составляет природу единичного объекта в его
{\em внешности} и
{\em существовании}. Так
как объект, таким образом, есть {\em в
себе} все понятие, то он обладает в самом себе
{\em необходимостью} и
{\em влечением} снять
свое противоположное, {\em одностороннее
устойчивое наличие} и сделать себя в наличном бытии тем
{\em реальным целым},
которое он есть по своему понятию.

Относительно выражения
<<{\em химизм}>> для
обозначения отношения того небезразличия объективности, которое у нас здесь
получилось, можно, впрочем, заметить, что его здесь не следует понимать
так, как будто это отношение проявляется только в той форме природы
элементов, которая именуется так называемым химизмом в собственном смысле.
Уже метеорологическое отношение должно рассматриваться как такой процесс,
стороны которого обладают больше природой физических, чем химических
элементов. В~живых существах отношение полов подчиняется этой схеме, и она
точно так же составляет
{\em формальную} основу
духовных отношений любви, дружбы
и~т.~д.\pagenote{Имеется в виду так называемое <<сродство душ>>.
Ср.~роман Гете <<Die Wahl\-ver\-wand\-schaf\-ten>> (<<Сродство душ>>),
написанный великим поэтом в 1809~г.\label{bkm:bm81}}

При более близком рассмотрении оказывается, что химический
объект, как некоторая
{\em самостоятельная}
тотальность вообще, есть сперва рефлектированный в~себя
объект, который постольку отличен от своей рефлектированности во-вне, "---
некоторое безразличное
{\em основание},
индивидуум, еще не определенный как небезразличный; личность
также есть такое пока что лишь с собой соотносящееся
основание\pagenote{Слово <<основание>> (Basis) берется здесь в смысле химического основания,
причем, однако, этот смысл метафорически переносится также и на другие,
не химические предметы и даже на личности. В~химии основаниями называются
такие вещества, которые при соединении с кислотами дают соли. Типическими
представителями оснований в химии начала XIX в. считались щелочи (например, едкое кали, едкий натр),
являющиеся соединениями кислорода с металлами.\label{bkm:bm82}}.
А та имманентная определенность, которая составляет
{\em небезразличие}
химического объекта, рефлектирована в~себя,
{\em во-первых}, так, что
это взятие обратно направленного во-вне соотношения есть лишь формальная
абстрактная всеобщность; таким образом, направленное во-вне соотношение
есть определение его (химического объекта) непосредственности и
существования. Взятый с этой стороны, он не
{\em в~себе самом}
возвращается в индивидуальную тотальность, и отрицательное
единство имеет два момента своей противоположности в двух
{\em особых объектах}.
Согласно этому химический объект не может быть постигнут из
него же самого, и бытие одного объекта есть бытие некоторого другого. "---
Но, {\em во-вторых},
определенность рефлектирована в~себя абсолютным образом и
составляет конкретный момент индивидуального понятия целого, каковое
понятие есть всеобщая сущность,
{\em реальный род}
особенного объекта. Химический объект (а~тем самым и
противоречие между его непосредственной положенностью и его имманентным
индивидуальным понятием) есть
{\em стремление} снять
определенность его наличного бытия и дать существование объективной
тотальности понятия. Он поэтому, хотя тоже есть нечто несамостоятельное, но
таким образом, что он по самой своей природе напряжен против этой
несамостоятельности и начинает
{\em процесс}
самоопределяющим образом.

\section[В. Химический процесс]{В. Химический процесс}

1. Он начинается с пред-положения, что хотя объекты напряжены
по отношению к самим себе, они именно поэтому прежде всего напряжены по
отношению друг к другу; это "--- отношение, которое называется
их {\em сродством}. Так
как каждый объект в силу своего понятия находится в противоречии с
собственной односторонностью своего существования и, стало быть, стремится
снять ее, то здесь непосредственно положено стремление снять
односторонность другого объекта и этим взаимным выравниванием и соединением
положить реальность адекватной понятию, содержащему в~себе оба момента.

Поскольку каждый из этих объектов положен как в самом себе
противоречащий себе и снимающий себя, то они лишь посредством
{\em внешнего насилия}
удерживаются в отделении друг от друга и от их
взаимного дополнения. Средним термином, смыкающим эти
крайние термины, служит,
{\em во-первых},
{\em в-себе-сущая} природа
этих двух крайних терминов, держащее их обоих внутри себя целостное
понятие. Но, {\em во-вторых},
так как они в своем существовании противостоят друг другу, то
их абсолютное единство тоже есть
{\em отлично} от них
{\em существующая}, пока
что еще формальная стихия, "--- стихия
{\em сообщения}, в
которой они вступают во внешнюю
{\em общность} друг с
другом. Так как реальное различие принадлежит крайним терминам, то этот
средний термин есть лишь абстрактная нейтральность крайних терминов, их
реальная возможность, "--- как бы
{\em теоретическая стихия}
существования химических объектов, их процесса и его
результата; в области телесного
{\em вода} исполняет
функцию этой среды; в области духовного, поскольку в ней имеет место нечто
аналогичное такого рода отношению, им должен считаться вообще
{\em знак} и, ближе,
{\em язык}.

Взаимоотношение объектов, как представляющее собой всего лишь
сообщение в этой стихии, есть, с одной стороны, некоторое спокойное
слияние, но, с другой стороны, в равной мере и некоторое
{\em отрицательное отношение},
поскольку в процессе сообщения то конкретное понятие, которое
составляет их природу, полагается в сферу реальности, и тем самым
{\em реальные различия}
объектов приводятся обратно к его единству. Таким образом, их
прежняя самостоятельная
{\em определенность}
снимается в соединении, соответствующем понятию, которое в
обоих одно и то же, их противоположность и напряженность в силу этого
притупляются, вместе с чем стремление достигает в этом взаимном дополнении
своей спокойной {\em нейтральности}.

Процесс, таким образом,
{\em угас;} так как
противоречие между понятием и реальностью теперь выравнено, то крайние
термины умозаключения потеряли свою противоположность и тем самым перестали
быть крайними терминами по отношению друг к другу и к среднему термину.
{\em Продукт} есть нечто
{\em нейтральное}, т.~е.
нечто такое, в чем ингредиенты, которые больше не могут быть названы
объектами, уже потеряли свою напряженность, а, стало быть, и те свойства,
которые были им присущи, как напряженным, но в чем сохранилась
{\em способность} к их
прежней самостоятельности и напряженности. А~именно, отрицательное единство
нейтрального исходит из некоторого
{\em предположенного}
различия;
{\em определенность}
химического объекта тождественна с его объективностью, она
первоначальна. Рассмотренным процессом это различие пока что лишь
{\em непосредственно}
снято; определенность поэтому еще не выступает
как абсолютно рефлектированная в~себя, и, стало быть, продукт процесса есть
лишь некоторое формальное единство.

2. В~этом продукте напряженность противоположенности и
отрицательное единство, как деятельность процесса, теперь, правда, угасли.
Но так как это отрицательное единство существенно для понятия и вместе с
тем само получило существование, то оно все еще имеется налицо, однако
{\em вне} нейтрального
объекта, как выступившее из него. Процесс не загорается сам собой снова,
поскольку он имел различие лишь своей
{\em предпосылкой}, а не
сам {\em положил} его. "---
Эта вне объекта, самостоятельная отрицательность,
существование {\em абстрактной}
единичности, для-себя-бытие которой имеет свою реальность в
{\em безразличном объекте},
теперь напряжена внутри самой себя против своей
абстрактности, есть беспокойная внутри себя деятельность, которая,
потребляя себя, обращается во-вне. Она
{\em непосредственно}
соотносится с объектом, спокойная нейтральность которого есть
реальная возможность ее противоположности; объект этот есть теперь
{\em средний термин}
бывшей прежде лишь формальной нейтральности, он теперь
конкретен внутри себя и определенен.

Более тесное непосредственное соотношение
{\em крайнего термина отрицательного
единства} с объектом состоит в том, что последний через него
определяется и в силу этого расщепляется. На это расщепление можно
ближайшим образом смотреть, как на восстановление той противоположности
напряженных объектов, с которой начался химизм. Однако это определение не
составляет другого крайнего термина умозаключения, а принадлежит к
непосредственному соотношению диференцирующего принципа с тем средним
термином, в котором этот принцип дает себе свою непосредственную
реальность; это "--- та определенность, которой средний термин
разделительного умозаключения обладает еще помимо того, что он есть
всеобщая природа предмета, и в силу которой последний представляет собой
как объективную всеобщность, так и определенную особенность.
{\em Другой крайний термин}
рассматриваемого умозаключения противостоит внешнему
{\em самостоятельному крайнему термину}
единичности; он поэтому есть столь же самостоятельный крайний
термин {\em всеобщности;}
поэтому то расщепление, которому здесь подвергается реальная
нейтральность среднего термина, состоит в том, что она разлагается не на
взаимно различные, а на
{\em безразличные}
моменты. Эти моменты суть тем самым, с одной стороны,
абстрактное, безразличное
{\em основание}, а с
другой стороны, его {\em одушевляющий}
принцип, который вследствие своего отделения
от основания тоже получает форму безразличной объективности.

Это разделительное умозаключение есть тотальность химизма, в
которой одно и то же объективное целое изображено сперва как
самостоятельное {\em отрицательное}
единство, затем, в среднем термине, как
{\em реальное} единство,
"--- а под конец изображена химическая реальность, разложенная
на свои {\em абстрактные}
моменты. В~этих последних определенность достигла своей
{\em рефлексии в~себя} не
в чём-то {\em ином},
как это имеет место в нейтральном объекте, а в самой себе
возвратилась в свою абстрактность и представляет собой некоторый
{\em первоначально определенный
элемент}.

3. Тем самым эти элементарные объекты освобождены от
химической напряженности; в них была
{\em положена} через
реальный процесс первоначальная основа той
{\em предпосылки}, с
которой химизм начался. А~поскольку, далее, с одной стороны, их внутренняя
{\em определенность} как
таковая есть по существу противоречие между их
{\em простым безразличным
существованием} и ею как
{\em определенностью} и
представляет собой влечение во-вне, которое расщепляет себя и полагает
напряженность в своём объекте и в некотором
{\em другом},
{\em дабы иметь нечто такое},
к чему объект мог бы относиться как небезразличный, нечто
такое, в чем он мог бы нейтрализоваться и сообщить своей простой
определенности наличносущую реальность, "--- постольку химизм
возвратился тем самым в свою исходную точку, в которой взаимно напряженные
объекты ищут друг друга, а затем соединяются в нечто нейтральное через
некоторый формальный, внешний средний термин. С~другой же стороны, химизм
через это возвращение в свое
{\em понятие} снимает
себя и перешел в некоторую более высокую сферу.

\section[С. Переход химизма]{С. Переход химизма}

Уже обычная химия доставляет нам примеры таких химических
изменений, при которых, например, некоторое тело сообщает одной части своей
массы более высокую степень окисления и этим понижает степень окисления
другой ее части, при каковой пониженной степени окисления данное тело
только и может вступить в нейтральное соединение с приближаемым к нему
другим, небезразличным к нему телом, тогда как при своей первой,
непосредственной степени окисления оно было бы неспособно к этому
соединению. Здесь происходит следующее: объект соотносится с некоторым
другим объектом не по какой-нибудь непосредственной, односторонней
определенности, а {\em полагает}
сообразно внутренней тотальности некоторого
первоначального {\em отношения}
ту {\em предпосылку},
в которой он нуждается для образования некоторого реального
соотношения, и этим дает себе некоторый средний термин, через посредство
которого он смыкает свое понятие со своей реальностью; он есть
в-себе-и-для-себя-определенная единичность, конкретное понятие
как принцип {\em разделения}
на крайние термины,
{\em воссоединение}
которых есть деятельность
{\em того же самого}
отрицательного принципа, который в силу этого возвращается
обратно к своему первому определению, но уже
{\em объективированным}.

Сам химизм есть {\em первое
отрицание безразличной} объективности и
{\em внешности}
определенности; он, следовательно, еще обременен
непосредственной самостоятельностью объекта и внешностью. Он поэтому сам по
себе еще не есть та тотальность самоопределения, которая происходит из него
и в которой он скорее снимает себя. "--- Получившиеся три
умозаключения составляют его тотальность; первое умозаключение имеет
средним термином формальную нейтральность, а крайними терминами напряженные
объекты; второе имеет средним термином продукт первого, реальную
нейтральность, а крайними терминами расщепляющую деятельность и ее продукт,
безразличный элемент; третье же есть реализующее себя понятие, полагающее
для себя ту предпосылку, которой обусловлен процесс его реализации, "---
умозаключение, имеющее своей сущностью всеобщее. Однако в
силу непосредственности и внешности, определению которых подчинена
химическая объективность, {\em эти
умозаключения все еще оказываются вне друг друга}. Первый
процесс, продуктом которого является нейтральность напряженных объектов,
угасает в своем продукте, и его заставляет вновь загораться лишь
привходящее извне диференцирование; будучи обусловлен некоторой
непосредственной предпосылкой, он в ней и исчерпывает себя. "---
И точно так же выделение небезразличных крайних терминов из
нейтрального объекта, а равно и их разложение на их абстрактные элементы
должно исходить от {\em привходящих
извне условий} и возбуждений деятельности. Но хотя оба
существенных момента процесса "--- с одной стороны,
нейтрализация a, с другой стороны, отделение и редукция, "---
связаны в одном и том же процессе, и
{\em соединение} и
притупление напряженных крайних терминов есть также и
{\em разделение} на
такого рода крайние термины, "--- все же означенные моменты
вследствие еще лежащей в основании внешности составляют
{\em две разные} стороны;
крайние термины, выделяемые в этом же процессе, суть другие объекты или
материи, чем те, которые соединяются в нем; поскольку первые выходят из
этого процесса снова небезразличными, они должны обратиться
вовне; их новая нейтрализация представляет собой другой процесс, чем та
нейтрализация, которая имела место в первом процессе.

Но эти разные процессы, получившиеся у нас необходимым
образом, представляют собой столько же
{\em ступеней}, через
восхождение по которым снимаются
{\em внешность} и
{\em обусловленность}, в
результате чего понятие выступает как определенная в~себе и для себя, не
обусловленная внешностью тотальность. В~первом процессе или в первом
умозаключении снимается внешний характер отношения между составляющими всю
реальность, небезразличными друг к другу крайними терминами или, иначе
говоря, снимается отличность
{\em в-себе}-сущего
определенного понятия от его
{\em налично-сущей}
определенности. Во втором процессе или во втором
умозаключении снимается внешность реального единства, соединение как только
{\em нейтральное}. Говоря
точнее, формальная деятельность снимает себя сначала в столь же формальных
[химических] основаниях или безразличных определенностях,
{\em внутреннее понятие}
которых есть теперь ушедшая в~себя, абсолютная деятельность,
как реализующаяся в самой себе, т.~е. как деятельность,
{\em полагающая} внутри
себя определенные различия и конституирующаяся как реальное единство
благодаря этому {\em опосредствованию},
"--- опосредствованию, которое, стало быть, есть
{\em собственное}
опосредствование понятия, его самоопределение и которое ввиду
того, что понятие рефлектируется из этого опосредствования в~себя,
представляет собой имманентное
{\em пред-полагание}.
Третье умозаключение, которое, с одной стороны, есть
восстановление предшествующих процессов, снимает, с другой стороны, еще и
последний момент безразличных [химических] оснований, "---
снимает совершенно абстрактную,
{\em внешнюю непосредственность},
которая таким образом становится
{\em собственным}
моментом опосредствования понятия самим собой. Понятие,
которое тем самым сняло, как внешние, все моменты своего объективного
наличного бытия и положило их в свое простое единство, благодаря этому
полностью освободилось от объективной внешности, с которой оно теперь
соотносится лишь как с некоторой несущественной реальностью; это
объективное свободное понятие есть
{\em цель}.

\chapter[Третья глава Телеология]{Третья глава\newline Телеология}

Там, где усматривается
{\em целесообразность},
принимают существование некоторого
{\em ума} как ее
создателя; для цели, следовательно, требуют собственного, свободного
существования понятия. {\em Теологию}
противопоставляют преимущественно
{\em механизму}, в
котором положенная в объекте определенность выступает существенным образом
как внешняя, как такая определенность, в которой не проявляется никакого
{\em самоопределения}.
Противоположность между causae efficientes и
causae finales, т.~е. между только
{\em действующими} и
{\em целевыми причинами},
относится к указанному различию, к которому, взятому в
конкретной форме, сводится также и исследование вопроса о том, следует ли
понимать абсолютную сущность мира как слепой природный механизм или как ум,
определяющий себя согласно целям. Антиномия
{\em фатализма} (вместе с
{\em детерминизмом})
{\em и свободы} равным
образом касается противоположности между механизмом и телеологией; ибо
свободное есть понятие в его существовании.

Прежняя метафизика обращалась с этими понятиями таким же
образом, как и со своими другими понятиями: она отчасти предпосылала
некоторое представление о мире и старалась показать, что то или иное
понятие ему соответствует, а противоположное понятие неудовлетворительно,
так как предпосланное ею представление о мире не может быть
{\em объяснено} из него;
отчасти же она при этом не подвергала исследованию понятия механической
причины и цели, не ставила вопроса о том, какое из них истинно
{\em само по себе}. Когда
получен отдельно ответ на этот вопрос, то пусть объективный мир являет нам
механические и целевые причины; их существование не есть мерило
{\em истины}, а,
наоборот, истина есть критерий того, какое из этих существований есть
истинное существование мира. Подобно тому как субъективный рассудок являет
нам в нем (в~рассудке) также и заблуждения, так и объективный мир являет
также и те стороны и ступени истины, которые, взятые сами по себе, лишь
односторонни, неполны и суть только отношения явлений. Если механизм и
целесообразность противостоят друг другу, то именно поэтому их нельзя брать
как {\em равнодушные}
друг к другу понятия, как будто бы каждое из них само по себе
есть правильное понятие и обладает такой же значимостью, как и другое, так
что весь вопрос только в том, {\em где}
можно применять то или другое. Эта равная
значимость обоих понятий основана только на том факте, что
оба они {\em имеют бытие},
а именно, на том факте, что мы
{\em обладаем} обоими. Но
так как они противоположны, то необходимым первым вопросом является вопрос
о том, какое из этих понятий есть истинное; а более высоким подлинным
вопросом является вопрос о том, {\em не
служит ли их истиной некоторое третье понятие или не есть ли одно из них
истина другого}. "--- Но
{\em целевое соотношение}
получилось у нас как истина
{\em механизма}. "--- То,
что выступало как {\em химизм},
постольку берется за одну скобку с
{\em механизмом},
поскольку цель есть понятие в его свободном существовании и
ей вообще противостоит несвобода понятия, его погруженность во внешность;
таким образом, и то и другое "--- как механизм, так и химизм,
"--- берутся вместе под общей рубрикой <<природной
необходимости>>, так как в механизме понятие не существует в объекте, потому
что объект этот, как механический, не содержит в~себе самоопределения, в
химизме же понятие или обладает напряженным, односторонним существованием,
или (поскольку оно выступает как единство, напрягающее нейтральный объект и
расщепляющее его на крайние термины) понятие, упраздняя эту раздельность,
оказывается внешним самому себе.

Чем больше телеологический принцип приводился в связь с
понятием некоторого {\em внемирового}
ума и постольку находился под покровительством благочестия,
тем в большей мере он, казалось, удалялся от истинного исследования
природы, которое стремится познать свойства природы не как чужеродные, а
как {\em имманентные определенности}
и признает лишь такое познание
{\em постижением в понятиях}.
Так как цель есть само понятие в его существовании, то здесь
может показаться странным, что познание объектов из их понятия
представляется, наоборот, неправомерным переходом в некоторую
{\em гетерогенную}
стихию, а, напротив, механизм (для которого определенность
какого-либо объекта выступает как определенность, сообщенная ему извне
некоторым другим объектом) слывет за
{\em более имманентное}
воззрение, чем телеология. Механизм, по крайней мере,
обычный, несвободный, равно как и химизм, действительно должен постольку
рассматриваться как имманентный принцип, поскольку определяющее
{\em внешнее} само
{\em в свою очередь} есть
{\em лишь такого рода объект},
нечто внешне определенное и безразличное к такой
определяемости, или, если иметь в виду химизм, поскольку другой объект есть
равным образом химически определенный и вообще поскольку тот или иной
существенный момент тотальности всегда лежит в некотором внешнем. Эти
принципы остаются поэтому в пределах одной и той же природной формы
конечности; но хотя они не хотят выходить за пределы
конечного и для объяснения явлений ведут лишь к конечным причинам, которые
сами требуют перехода к дальнейшим причинам, они все же вместе с тем
расширяют себя отчасти таким образом, что становятся некоторой формальной
тотальностью в понятиях силы, причины и тому подобных рефлективных
определениях, долженствующих обозначать некоторую
{\em первоначальность},
отчасти же через абстрактную
{\em всеобщность} в виде
некоторой {\em вселенной сил},
некоторого {\em целого}
взаимных причин. Механизм показывает себя стремлением к
тотальности уже тем, что он старается понять природу
{\em самое по себе}, как
некоторое {\em целое}, не
требующее для {\em своего}
понятия ничего другого, "--- тотальность, не
имеющая места в цели и в связанном с нею внемировом
уме\pagenote{Эту фразу (начиная со слов:
<<Механизм показывает себя\ldots>>) цитирует в <<Диалектике природы>> Энгельс,
снабжая ее следующим замечанием: <<Беда однако в том, что механизм (также
материализм 18-го века) не может выбраться из абстрактной необходимости, а
потому также и из случайности. Для него тот факт, что материя развивает из
себя мыслящий мозг человека, есть чистая случайность, хотя и необходимо
обусловленная шаг за шагом там, где это происходит. В~действительности же
материя приходит к развитию мыслящих существ в силу самой своей природы, а
потому это с необходимостью и происходит во всех тех случаях, когда имеются
налицо соответствующие условия (не обязательно везде и всегда одни и те
же)>> ({\em Engels},
Dialektik der Natur, M.--L. 1935, S.~654). Этим замечанием
Энгельс показывает: 1) что под гегелевской категорией <<механизма>>
скрывается главным образом антителеологический детерминизм французских
материалистов XVIII~в. и~2) что этот детерминизм имел абстрактный,
метафизический характер, в чем и состояла его историческая ограниченность и
недостаточность. Любопытно, что идеалист Гегель, провозглашая телеологию
более высокой точкой зрения, чем механизм, все же вынужден признать
известные <<формальные>> (как он выражается) преимущества за
механизмом.\label{bkm:bm83}}.

Целесообразность являет себя прежде всего как нечто
{\em более высокое}
вообще, как некоторый
{\em ум, внешним образом}
определяющий многообразие объектов
{\em через некоторое
в-себе-и-для-себя-сущее единство}, так что безразличные
определенности объектов становятся
{\em благодаря этому соотношению
существенными}. В~механизме они становятся существенными
благодаря {\em голой форме
необходимости}, причем их
{\em содержание}
безразлично, ибо они должны оставаться внешними, и лишь
рассудок как таковой должен чувствовать удовлетворение, познавая в них свою
рассудочную связь, абстрактное тождество. Напротив, в телеологии содержание
становится важным, так как она предполагает некоторое понятие, нечто
{\em в-себе-и-для-себя-определенное}
и, стало быть, самоопределяющее, и, следовательно, от
{\em соотношения}
различий и их определяемости друг другом, от
{\em формы} она отличила
{\em рефлектированное в~себя единство},
{\em нечто
в-себе-и-для-себя-определенное}, т.~е.
{\em некоторое содержание}.
Но если последнее помимо этого есть какое-нибудь
{\em конечное} и
незначительное содержание, то оно противоречит тому, чем оно должно быть,
ибо цель есть по своей форме
{\em бесконечная внутри себя}
тотальность "--- в особенности, если признается,
что действующая согласно целям деятельность есть
{\em абсолютная} воля и
{\em абсолютный} ум.
Телеология потому навлекла на себя такие сильные упреки в несообразности,
что цели, которые она указывала, бывали, смотря по обстоятельствам, то
более значительными, то менее значительными, и целевое соотношение между
объектами потому так часто должно было казаться произвольной игрой, что это
соотношение является столь внешним и потому случайным. Напротив, механизм
оставляет относящимся к содержанию определенностям объектов свойственное им
значение случайных определенностей, к которым объект безразличен и которые
ни для объектов, ни для субъективного рассудка не должны
иметь более высокую значимость. Этот принцип дает поэтому в своей связи
внешней необходимости сознание бесконечной свободы по сравнению с
телеологией, выставляющей все незначительные и даже ничтожные определения
своего содержания как нечто абсолютное, в котором более всеобщая мысль
может чувствовать себя лишь бесконечно стесненной и даже испытывает
отвращение\pagenote{Эту фразу (начиная со слов: <<Этот
принцип\ldots>>) цитирует в <<Диалектике природы>> Энгельс, сопровождая ее
следующим замечанием: <<При этом опять-таки колоссальная расточительность
природы с веществом и движением. В~солнечной системе имеются может быть
самое большее только три планеты, на которых, при теперешних условиях,
возможно существование жизни и мыслящих существ. И~ради них весь этот
громадный аппарат!>>
({\em Engels}, Dialektik der Natur, M.--L. 1935, S.~655).
Этим замечанием Энгельс вскрывает несостоятельность того телеологического
воззрения на природу, согласно которому {\em целью}
природы является существование органической жизни и мыслящих
существ. Ср.~примечание \ref{bkm:bm83}. С~точки
зрения диалектического материализма категория цели имеет место только в
области сознательной деятельности людей.\label{bkm:bm84}}.

Формальная невыгодность позиции, которую на первой стадии
занимает эта телеология, заключается в том, что она доходит лишь до
{\em внешней целесообразности}.
Так как понятие вследствие этого положено здесь как нечто
формальное, то содержание есть для нее (для телеологии) также нечто данное
ему внешним образом в многообразии объективного мира, "---
данное ему как раз в тех определенностях, которые составляют
также и содержание механизма, но как нечто внешнее, случайное. Вследствие
этой общности [содержания] единственно только
{\em форма целесообразности},
взятая сама по себе, и составляет существенную черту
телеологической точки зрения. В~этом отношении, еще без принятия во
внимание различия между внешней и внутренней целесообразностью, целевое
соотношение вообще, взятое само по себе, оказалось
{\em истиной механизма}. "---
Телеология обладает вообще более высоким принципом, понятием
в его существовании, каковое понятие само по себе есть бесконечное и
абсолютное, "--- есть принцип свободы, который, безоговорочно
уверенный в своем самоопределении, абсолютно избавлен от характерной для
механизма {\em внешней определяемости}.

Одна из великих заслуг
{\em Канта} перед
философией состоит в проведенном им различении между
{\em относительной} или
{\em внешней} и
{\em внутренней}
целесообразностью; в последней он раскрыл понятие
{\em жизни},
{\em идею} и этим выполнил
{\em положительно} то,
что критика разума выполняет лишь несовершенным образом, весьма кривыми
путями и лишь {\em отрицательно}, "---
а именно, поднял философию выше определений рефлексии и
релятивного мира метафизики. "--- Мы уже указали, что
противоположность между телеологией и механизмом прежде всего представляет
собой более всеобщую противоположность
{\em свободы} и
{\em необходимости}. Кант
приводит рассматриваемую нами противоположность в этой последней форме
среди {\em антиномий}
разума, а именно, как
{\em третье столкновение
трансцендентальных идей}. "--- Я приведу его изложение, на
которое мы уже указали выше, совершенно кратко, так как существенное в
изложении этой антиномии так просто, что не нуждается в пространном
рассуждении, а характер кантовских антиномий мы уже осветили
более подробно в другом месте.

{\em Тезис} антиномии,
подлежащей здесь нашему рассмотрению, гласит: <<Причинность согласно законам
природы не есть единственная причинность, из которой могут быть выведены
все без исключения явления мира. Для их объяснения необходимо допустить
кроме того причинность, действующую через свободу>>.

{\em Антитезис}: <<Нет
никакой свободы, и все в мире совершается исключительно только по законам
природы>>.

Доказательство, как и в прочих антиномиях, берется, во-первых,
за дело апагогически: оно допускает противное каждому тезису; во-вторых,
чтобы показать противоречивость этого допущения, принимается и признается
значимым противоположное этому допущению, т.~е. принимается, наоборот,
положение, подлежащее доказательству. Можно было поэтому обойтись без всего
этого окольного пути доказывания; последнее состоит не в чем другом, как в
ассерторическом утверждении обоих противостоящих друг другу положений.

А именно, для доказательства
{\em тезиса} нам
предлагают сперва допустить следующее: не существует
{\em никакой другой причинности},
кроме действующей по
{\em законам природы},
т.~е. по механической необходимости вообще, включая в нее и
химизм. Это положение "--- говорится далее
"--- противоречит себе потому, что закон природы состоит как раз
в том, что ничто не происходит {\em без
достаточной, определенной a priori причины}, которая, стало
быть, содержит в~себе абсолютную спонтанность; "--- другими
словами, допущение, противоположное тезису, противоречиво потому, что оно
противоречит тезису.

Для доказательства
{\em антитезиса предлагается}
сделать следующее допущение: существует
{\em свобода} как особый
вид причинности, "--- свобода всецело начинать некоторое
состояние и, стало быть, также и ряд следствий из этого состояния. Но так
как такое начало {\em предполагает}
состояние, не имеющее
{\em никакой причинной связи}
с предшествующим состоянием той же самой причины, то оно
противоречит {\em закону причинности},
согласно которому единственно только и возможно единство
опыта и вообще опыт; "--- другими словами, допущение свободы,
противоречащее антитезису, не может быть сделано потому, что оно
противоречит антитезису.

По существу та же самая антиномия встречается снова в <<{\em Критике
телеологической способности суждения}>> как противоположность между
утверждением, что <<{\em всякое порождение материальных вещей} совершается
{\em по чисто механическим законам}>>, и утверждением, что
{\em <<некоторые виды порождения этих вещей невозможны по таким
законам>>}\pagenote{См. {\em Кант}, Критика способности суждения,
пер. Соколова, Спб. 1898, стр.~274.\label{bkm:bm85}}.
"--- Кантово разрешение этой антиномии таково же, как общее
разрешение им прочих антиномий; оно, именно, заключается в том, что разум
но может доказать ни одного, ни другого положения, так как мы относительно
возможности вещей согласно чисто эмпирическим законам природы
{\em не можем иметь никакого
определяющего принципа a~priori;} что, далее, поэтому оба
положения должны быть рассматриваемы не
{\em как объективные положения}, а {\em как субъективные
максимы;} что я, {\em с одной стороны}, должен всегда {\em размышлять} о всех
событиях природы согласно принципу одного только механизма природы, но что
это не мешает тому, чтобы я, {\em когда представится к этому повод},
{\em исследовал} некоторые формы природы согласно {\em другой
максиме}, а именно, согласно принципу целевых причин, "--- как будто эти
{\em две максимы} (они, впрочем, по мнению Канта, необходимы только
для {\em человеческого разума})
не находятся между собой в том же антагонизме, в котором
находятся вышеуказанные положения. "--- Как мы заметили выше,
развивая всю эту точку зрения, Кант не исследует того вопроса, разрешения
которого единственно только и требует философский интерес, а именно, какой
из этих двух принципов истинен сам по себе. Для этой точки зрения не
составляет разницы, рассматриваются ли принципы как {\em объективные} (это
здесь значит: внешне существующие определения природы) или только как
{\em максимы субъективного познания}.
Скорее, наоборот: субъективность, т.~е. случайность познания,
заключается здесь в том, что познание применяет ту или другую максиму по
{\em случайному поводу}, смотря по тому, какую из них оно находит подходящей
для данных объектов, и что в остальном оно не спрашивает об {\em истинности}
самих этих определений, "--- все равно, представляют ли собой оба они
определения объектов или определения познания.

Поэтому, как бы неудовлетворительно ни было кантовское
изложение телеологического принципа со стороны существенной точки зрения,
все же остается замечательным то место, которое Кант ему отводит.
Приписывая этот принцип {\em рефлектирующей способности суждения},
Кант делает его связующим {\em средним членом} между {\em всеобщим разума}
и {\em единичным созерцания}. Он, далее, различает эту
{\em рефлектирующую} способность суждения от {\em определяющей},
которая только {\em подводит} особенное под всеобщее. Такое всеобщее,
которое только {\em подводит} под себя единичности, есть нечто
{\em абстрактное}, становящееся {\em конкретным} лишь в чём-то {\em ином},
в особенном. Напротив, цель есть {\em конкретное всеобщее},
которое в~себе же самом имеет момент особенности и внешности;
оно поэтому деятельно и есть влечение отталкивать себя от себя самого.
Понятие как цель есть, конечно, {\em объективное суждение},
в котором одним определением служит субъект, а именно,
конкретное понятие, как определенное само через себя, а другим
определением "--- не только некоторый предикат, но внешняя
объективность. Однако это не означает, что целевое соотношению есть
некоторый {\em рефлектирующий} процесс суждения, который лишь рассматривает
внешние объекты согласно некоторому единству, {\em как будто бы} некоторый
рассудок дал нам это соотношение {\em для помощи нашей познавательной
способности;} напротив, это соотношение есть
в-себе-и-для-себя-сущее {\em истинное}, судящее {\em объективно}
и определяющее внешнюю объективность абсолютным образом.
Соотношение цели есть в силу этого более чем {\em суждение}, оно есть
{\em умозаключение} самостоятельного, свободного понятия, которое через
объективность смыкает себя с самим собой.

Цель оказалась
{\em третьим} членом по
отношению к механизму и химизму; она есть их истина. Так как она сама еще
находится внутри сферы объективности или непосредственности тотального
понятия, то она еще испытывает воздействие внешности как таковой и имеет
перед собой объективный мир, с которым она соотносится. С~этой стороны при
рассматриваемом нами {\em целевом
соотношении} (которое есть
{\em внешнее}
соотношение) все еще выступает механическая причинность, к
которой в общем следует причислить также и химизм, но выступает как
{\em подчиненная} ему,
как сама по себе снятая. Что касается более детального отношения, то
механический объект, как непосредственная тотальность, безразличен к своей
определяемости и, значит, и к тому факту, что он есть нечто определяющее.
Эта внешняя определяемость теперь развилась дальше до самоопределения, и
тем самым {\em понятие},
которое в объекте было лишь
{\em внутренним} или, что
то же самое, лишь {\em внешним},
теперь {\em положено;}
цель и есть ближайшим образом именно само это внешнее для
механического объекта понятие. Таким же образом и для химизма цель есть
нечто самоопределяющее, приводящее обратно к единству понятия ту внешнюю
определяемость, которой химизм обусловлен. "--- Отсюда явствует
природа подчинения обеих предыдущих форм объективного процесса; то иное, чт\'{о}
 в них выступало в виде бесконечного прогресса, есть то (положенное
вначале, как внешнее для них) понятие, которое есть цель; не только понятие
есть их субстанция, но и внешность есть существенный для них, составляющий
их определенность момент. Таким образом, механическая или
химическая техника по своему характеру, состоящему в том, что она
определена извне, сама собой отдает себя на службу отношению цели, которое
мы теперь и должны рассмотреть ближе.

\section[А. Субъективная цель]{А. Субъективная цель}

В {\em центральности} объективной сферы, которая есть безразличие
к определенности, {\em субъективное}
понятие снова обрело и положило прежде всего
{\em отрицательную точку единства;}
в химизме же оно снова обрело и положило объективность
{\em определений понятия},
в силу которой оно впервые только и положено как
{\em конкретное объективное понятие}.
Его определенность или его простое различие имеет теперь в
нем самом {\em определенность
внешности}, и его простое единство есть в силу этого
единство, отталкивающееся от самого себя и сохраняющееся в этом
отталкивании. Цель есть поэтому субъективное понятие как существенное
стремление и влечение положить себя во-вне. Она при этом избавлена от
перехода. Она не есть ни сила, которая проявляет себя вовне, ни некоторая
субстанция и причина, проявляющая себя в акциденциях и действиях. Сила есть
лишь нечто абстрактно внутреннее, пока она себя не проявила во-вне; или,
иначе говоря, она обладает наличным бытием лишь в своем внешнем проявлении,
к которому она должна быть возбуждена. То же самое верно также и
относительно причины и субстанции: так как они обладают действительностью
лишь в своих акциденциях и в своих действиях, то их деятельность есть
переход, по отношению к которому они не сохраняют себя в свободном
состоянии. Правда, цель тоже может быть определена как сила и причина,
однако эти выражения покрывают собой лишь несовершенную сторону ее
значения. Если применять их для обозначения цели согласно ее истине, то они
могут это выполнять лишь таким образом, что этим упраздняется их понятие:
цель будет тогда силой, которая сама себя возбуждает к проявлению во-вне,
будет причиной, которая есть причина самой себя или действием которой
служит непосредственно сама причина.

Если целесообразность приписывается, как указывалось выше,
некоторому {\em уму}, то
при этом внимание обращается на то или иное
{\em определенное содержание}.
Но это содержание следует брать вообще как
{\em разумное в его существовании}.
Оно потому проявляет
{\em разумность}, что оно
есть конкретное понятие, держащее
{\em объективное различие внутри своего
абсолютного единства}. Оно поэтому есть по существу
{\em умозаключение} в нем
самом. Оно есть равное себе
{\em всеобщее} и притом,
как содержащее в~себе отталкивающую себя от себя
отрицательность, ближайшим образом всеобщая и постольку пока
что еще {\em неопределенная
деятельность;} но так как эта деятельность есть
отрицательное соотношение с самой собою, то она непосредственно
{\em определяет} себя и сообщает себе момент {\em особенности},
которая, как равным образом {\em рефлектированная в~себя тотальность
формы}, есть {\em содержание по отношению к
положенным} различиям формы. Столь же непосредственно эта
отрицательность есть в силу своего соотношения с самой собой абсолютная
рефлексия формы в~себя и {\em единичность}.
С~одной стороны, эта рефлексия есть {\em внутренняя всеобщность субъекта},
но, с другой стороны, она есть {\em рефлексия во-вне}, и
постольку цель есть еще нечто субъективное, и ее деятельность направлена на
внешнюю объективность.

А именно, цель есть понятие, пришедшее в объективности к себе
самому; определенность, которую цель сообщила себе в объективности, есть
определенность {\em объективного безразличия} и {\em внешности}
определяемости; ее отталкивающая себя от себя отрицательность
есть поэтому такая отрицательность, моменты которой, будучи только
определениями самого понятия, имеют также и форму объективного безразличия
друг к другу. "--- Уже в формальном
{\em суждении субъект} и {\em предикат} определены
как самостоятельные по отношению друг к другу; но их самостоятельность есть
пока что лишь абстрактная всеобщность; теперь же она достигла определения
{\em объективности;} но
как момент понятия эта полная разность включена в простое единство понятия.
И вот, поскольку цель есть эта тотальная {\em рефлексия}
объективности {\em в~себя}, и притом "--- {\em непосредственно},
то, {\em во-первых}, самоопределение или особенность, как
{\em простая} рефлексия в~себя, отлична от {\em конкретной}
формы и представляет собой некоторое {\em определенное содержание}.
Цель согласно этому {\em конечна}, хотя по
своей форме она есть бесконечная субъективность. Во-вторых, так как
определенность цели имеет форму объективного безразличия, то эта
определенность имеет вид некоторой {\em предпосылки}, и
конечность цели состоит с этой стороны в том, что она имеет перед собой
некоторый {\em объективный}, механический и химический {\em мир},
к которому ее деятельность относится как к чему-то {\em наличному;} ее
самоопределяющая деятельность в своей тождественности, таким образом,
непосредственно {\em внешня самой себе}
и столько же есть рефлексия в~себя, сколько и рефлексия
во-вне. Постольку цель еще обладает поистине {\em внемировым}
существованием, а именно, поскольку ей противостоит вышеуказанная
объективность, противостоящая цели как некоторое механическое и химическое
целое, еще не определенное и не пронизанное ею.

Поэтому теперь движение цели можно выразить следующим образом: оно направлено
к тому, чтобы упразднить ее {\em предпосылку}, т.~е. непосредственность
объекта, и {\em положить} объект как
определенный через понятие. Эта отрицательная позиция относительно объекта
есть столь же отрицательная позиция и относительно самой себя, упразднение
субъективности цели. Взятая с положительной стороны, эта позиция есть
реализация цели, а именно, соединение с нею объективного бытия, так что это
бытие, которое, как момент цели, есть непосредственно тождественная с нею
определенность, выступает как {\em внешняя}
определенность и, наоборот, объективное, как
{\em предпосылка}, скорее {\em полагается} как
определенное понятием. "--- Цель есть в самой себе влечение к
своей реализации; определенность моментов понятия есть внешность; но ее
{\em простота} в единстве
понятия не соответствует тому, что она есть, и поэтому понятие отталкивает
себя от себя самого. Это отталкивание есть вообще то
{\em решение} (der Entschluss)
соотношения отрицательного единства с собой, в силу которого
оно есть {\em исключающая} единичность: но через это
{\em исключение} (Aus\-schliessen) оно {\em разрешается}
(ent\-schliesst sie sich) или {\em раскрывается} (schliesst sich
{\em auf}), так как это есть {\em самоопределение}, полагание {\em самого
себя}. С~одной стороны, субъективность, определяя себя,
делает себя особенностью, дает себе некоторое содержание, которое, будучи
включенным в единство понятия, все еще есть некоторое внутреннее
содержание; но [с~другой стороны] это {\em полагание}, простая
рефлексия в~себя, есть, как выяснилось, непосредственно вместе с тем
некоторое {\em пред-полагание;} и в том же самом моменте, в котором субъект
цели определяет {\em себя}, он соотнесен
с некоторой безразличной, внешней объективностью, которая должна быть им
сделана равной той внутренней определенности, т.~е. должна быть положена как
нечто {\em определенное понятием}, ближайшим образом "--- как {\em средство}.

\section[В. Средство]{В. Средство}

Первое, непосредственное полагание в цели есть одновременно
как полагание некоторого {\em внутреннего}, т.~е.
чего-то определенного как {\em положенное}, так и
пред-полагание некоторого объективного мира, который безразличен к
определению цели. Но субъективность цели есть
{\em абсолютное отрицательное единство;} ее {\em второй} акт
определения есть поэтому снятие этого пред-положения вообще; это
снятие есть постольку {\em возвращение внутрь себя},
поскольку этим снимается вышеуказанный {\em момент первого отрицания},
полагание отрицательного по отношению к субъекту, внешний
объект. Однако по отношению к предположенной предпосылке или к
непосредственности процесса определения, по отношению к объективному миру
оно (снятие) пока что есть лишь {\em первое} отрицание, которое само есть
непосредственное и потому внешнее отрицание. Указанное полагание поэтому еще
не есть сама выполненная цель, а пока что только {\em начало} этого.
Определенный таким образом объект есть пока что {\em средство}.

Цель смыкает себя через некоторое средство с объективностью,
а в этой последней "--- с самой собой. Средство есть средний
термин умозаключения. Цель нуждается для своего выполнения в некотором
средстве, так как она конечна; нуждается в некотором средстве, т.~е. в
некотором среднем термине, который вместе с тем имеет вид некоторого
{\em внешнего} наличного
бытия, безразличного к самой цели и к ее выполнению. Абсолютное понятие
имеет опосредствование внутри самого себя таким образом, что первое
полагание этого понятия не есть такое пред-полагание, в объекте
которого основным определением была бы безразличная внешность; напротив
того, мир как творение имеет лишь форму такой внешности, основное же
определение его состоит скорее в его отрицательности и положенности. "---
Конечность цели состоит согласно этому в том, что ее процесс
определения вообще внешен самому себе и, стала быть, ее первое определение,
как мы видели, распадается на некоторое полагание и некоторое
пред-полагание; поэтому {\em отрицание}
этого процесса определения также лишь с одной стороны есть
уже рефлексия в~себя, с другой же стороны она есть, наоборот, лишь
{\em первое} отрицание; или, иначе говоря, рефлексия в~себя сама также внешня
себе и есть рефлексия во-вне.

Средство есть поэтому {\em формальный} средний термин некоторого
{\em формального} умозаключения; оно есть нечто {\em внешнее} по
отношению к первому крайнему термину, к субъективной цели, равно как
поэтому оно есть нечто внешнее и по отношению к другому крайнему термину, к
объективной цели, "--- подобно тому, как особенность в
формальном умозаключении есть какой-нибудь безразличный medius terminus,
место которого могут занять также и другие средние термины.
Далее, подобно тому как особенность есть средний термин только в силу того,
что она по отношению к одному крайнему термину есть определенность, по
отношению же к другому крайнему термину есть всеобщее и, следовательно,
обладает своим опосредствующим определением лишь релятивно, через другие
определения, так и средство есть опосредствующий средний
термин только потому, что, во-первых, оно есть некоторый непосредственный
объект, и, что, во-вторых, оно есть средство через {\em внешнее} ему
соотношение с крайним термином цели, каковое соотношение есть для средства
такая форма, к которой оно безразлично.

Поэтому понятие и объективность связаны в средстве лишь
внешним образом; средство постольку есть исключительно только
{\em механический объект}.
Соотношение объекта с целью есть посылка или, иначе говоря,
то непосредственное соотношение, которое, как было указано, в отношении
цели есть {\em рефлексия в~себя само;}
средство есть присущий цели предикат; его объективность
подведена под определение цели, которое в силу своей конкретности есть
всеобщность. Через это имеющееся в средстве определение цели оно вместе с
тем есть подчиняющее по отношению к другому крайнему термину, к сперва еще
неопределенной объективности. "--- Наоборот, средство, как
{\em непосредственная объективность},
обладает по отношению к субъективной цели
{\em всеобщностью наличного бытия},
которого еще лишена субъективная единичность цели. "---
Таким образом, цель, будучи ближайшим образом лишь внешней
определенностью в средстве, сама выступает как отрицательное единство вне
средства, равно как и средство есть механический объект, содержащий в~себе
цель лишь как некоторую определенность, а не как простую конкретность
тотальности. Но как смыкающее начало средний термин должен сам быть
тотальностью цели. Выше обнаружилось, что в средстве определение цели есть
вместе с тем рефлексия в~себя само; постольку это определение есть
{\em формальное}
соотношение с собой, так как
{\em определенность}
положена как {\em реальное
безразличие}, как
{\em объективность}
средства. Но именно поэтому эта, с одной стороны, чистая
субъективность есть вместе с тем также и
{\em деятельность}. "--- В
субъективной цели отрицательное соотношение с самой собой еще тождественно
с определенностью как таковой, с содержанием и внешностью. Но в
начинающемся объективировании цели, в становлении простого понятия иным
указанные моменты распадаются, выступают отдельно или, скажем наоборот, в
этом распадении и состоит само это становление иным или, иначе говоря, сама
внешность.

Весь этот средний термин, стало быть, сам есть целое
умозаключение, в котором абстрактная деятельность и внешнее средство
составляют крайние термины, а их средний термин образует собой та
определенность объекта через цель, в силу которой он есть средство. "---
Но, далее,
{\em всеобщность} есть
{\em соотношение} целевой
деятельности и средства. Средство есть объект, который
{\em в~себе} есть
тотальность понятия; оно не имеет силы сопротивляться цели, как оно
ближайшим образом сопротивляется какому-либо другому непосредственному
объекту. Таким образом, средство всецело проницаемо для цели (которая есть
положенное понятие) и восприимчиво к этому сообщению, так как в~себе оно
тождественно с целью. Но теперь оно также и
{\em положено} как то,
что проницаемо для понятия, ибо в центральности оно есть нечто стремящееся
к отрицательному единству; равным образом и в химизме оно
"--- и как нечто нейтральное, и как нечто небезразличное
"--- стало чем-то несамостоятельным. "--- Его
несамостоятельность именно и состоит в том, что оно лишь
{\em в~себе} есть
тотальность понятия; последнее же есть для-себя-бытие. Объект поэтому
характеризуется тем, что он бессилен перед целью и должен ей служить; она
есть его субъективность или душа, которая в нем имеет свою внешнюю сторону.

Объект, таким образом
{\em непосредственно}
подчиненный цели, не есть один из крайних терминов
умозаключения, а это соотношение составляет одну из посылок последнего. Но
средство имеет также и такую сторону, с которой оно еще обладает
самостоятельностью по отношению к цели. Связанная с последней в средстве
объективность "--- ввиду того, что она связана с целью лишь
непосредственно, "--- еще внешня последней, и поэтому
{\em предпосылка} все еще
имеется налицо. Деятельность цели через средство все еще поэтому направлена
против этой предпосылки, и цель и есть деятельность (а~не только влечение и
стремление) именно постольку, поскольку в средстве момент объективности
положен в своей определенности как нечто внешнее, и простое единство
понятия теперь содержит ее в~себе
{\em как таковую}.

\section[С. Выполненная цель]{С. Выполненная цель}

1. В~своем соотношении со средством цель уже рефлектирована в
себя; но ее {\em объективное}
возвращение в~себя еще не положено. Деятельность цели через
применяемое ею средство еще направлена против объективности как
первоначальной предпосылки, и заключается
{\em она} именно в том,
чтобы быть безразличной к определенности. Если бы деятельность состояла
опять-таки лишь в том, чтобы определять непосредственную объективность, то
продукт был бы в свою очередь лишь некоторым средством, и так далее до
бесконечности; в результате получилось бы только некоторое целесообразное
средство, а не объективность самой цели. Поэтому деятельная в своем
средстве цель должна определять непосредственный объект не
{\em как нечто внешнее;}
этот объект, стало быть, должен через самого себя слиться в
единство понятия, или, иначе говоря, вышеуказанная внешняя деятельность
цели через свое средство должна определить себя
{\em как опосредствование} и снять самое себя.

Соотношение деятельности цели с внешним объектом через
средство есть ближайшим образом {\em вторая посылка}
умозаключения "--- некоторое {\em непосредственное}
соотношение среднего термина с другим крайним термином. Это
соотношение {\em непосредственно}
потому, что средний термин содержит в~себе некоторый внешний
объект, а другой крайний термин есть точно такой же объект. Средство
действенно и могущественно по отношению к последнему, потому что его объект
связан с самоопределяющей деятельностью, между тем как для этого объекта та
непосредственная определенность, которой он обладает, есть некоторая
безразличная определенность. В~рассматриваемом нами соотношении процесс
изменения этой определенности есть не~что иное, как механический или
химический процесс; в этой объективной внешности изменений выступают
предшествующие отношения, но подчиненные господству цели. "---
Но эти процессы, как оказалось при их рассмотрении, сами
собой возвращаются в цель. Следовательно, если на первый взгляд соотношение
средства с подлежащим обработке внешним объектом есть непосредственное
соотношение, то оно уже ранее явило себя как некоторое умозаключение,
причем цель оказалась их истинным средним термином и единством. Таким
образом, так как средство есть объект, стоящий на стороне цели и содержащий
внутри себя ее деятельность, то имеющий здесь место механизм есть вместе с
тем возвращение объективности в самое себя, в понятие, которое, однако, уже
пред-положено как цель; отрицательная позиция целесообразной деятельности
по отношению к объекту есть постольку не нечто
{\em внешнее}, а
изменение и переход объективности в ней же самой в цель.

То обстоятельство, что цель непосредственно соотносится с
некоторым объектом и делает последний средством, равно как и то, что она
через этот объект определяет некоторый другой объект, можно рассматривать
как {\em насилие},
поскольку цель представляется имеющей совершенно другую
природу, чем объект, и эти два объекта также суть по отношению друг к другу
самостоятельные тотальности. А~то обстоятельство, что цель ставит себя в
{\em опосредствованное}
соотношение с объектом и
{\em вдвигает между}
собой и им некоторый другой объект, может быть рассматриваемо
как {\em хитрость}
разума. Конечность разумности заключает в~себе, как было
замечено выше, ту сторону, что цель имеет дело с
предположенной предпосылкой, т.~е. с внешностью объекта.
В~{\em непосредственном соотношении}
с последним она сама вступила бы в область механизма или
химизма и тем самым она подлежала бы случайности и гибели своего
определения, которое состоит в том, что она есть в-себе-и-для-себя-сущее
понятие. Действуя же так, как она действует, она выставляет некоторый
объект как средство, заставляет последнее вместо себя надрываться во
внешней работе, обрекает его на изнашивание и, прикрытая им, сохраняет себя
против механического насилия.

Так как цель конечна, то она, далее, имеет некоторое конечное
содержание; тем самым она не есть нечто абсолютное или, иначе говоря, не
есть нечто безоговорочно, в~себе и для себя
{\em разумное}.
{\em Средство} же есть
внешний средний термин того умозаключения, которое представляет собой
выполнение цели; в средстве поэтому разумность цели являет себя как
разумность, заключающаяся в сохранении себя в
{\em этом внешнем ином}
и как раз {\em через}
эту внешность. Постольку
{\em средство} есть нечто
{\em более высокое}, чем
{\em конечные} цели
{\em внешней}
целесообразности;
{\em плуг} почтеннее, чем
непосредственно те наслаждения, которые подготовляются им и являются
целями. {\em Орудие}
сохраняется, между тем как непосредственные наслаждения
проходят и забываются. В~своих орудиях человек обладает властью над внешней
природой, хотя по своим целям он скорее подчинен ей.

Но цель не только держится вне механического процесса, но и
сохраняется в нем и есть его определение. Цель, как понятие, которое
существует свободно по отношению к объекту и его процессу и есть самоё себя
определяющая деятельность, сливается в механизме лишь с самой собой, так
как она вместе с тем есть в-себе-и-для-себя-сущая истина механизма. Власть
цели над объектом есть это для-себя-сущее тождество, и ее деятельность есть
проявление этого тождества. Цель как
{\em содержание} есть та
в-себе-и-для-себя-сущая
{\em определенность},
которая в объекте выступает как безразличная и внешняя,
деятельность же этой цели есть, с одной стороны,
{\em истина} процесса, а
как отрицательное единство "--- {\em снятие
видимости внешности}. С~точки зрения
{\em абстракции} дело
обстоит так, что одна безразличная определенность объекта столь же внешним
образом заменяется какой-нибудь другой безразличной определенностью; но
простая {\em абстракция}
определенности в своей
{\em истине} есть
тотальность отрицательного, конкретное понятие, полагающее внутри себя
внешность.

{\em Содержанием} цели
служит ее отрицательность как
{\em простая рефлектированная в~себя
особенность}, отличная от ее тотальности как
{\em формы}. В~силу этой
{\em простоты},
определенность которой есть в~себе и для себя тотальность
понятия, содержание выступает как то, что
{\em остается тождественным}
в реализации цели. Телеологический процесс есть
{\em перевод} понятия,
существующего отчетливо как понятие, в объективность; этот перевод в
в нечто иное, служащее предпосылкой, оказывается слиянием понятия
{\em через себя само с самим собой}.
Содержание цели и есть это существующее в форме
тождественного тождество. При всяком переходе понятие сохраняется;
например, когда причина становится действием, причина сливается в действии
лишь с самой собой; но в телеологическом переходе само понятие как таковое
уже существует {\em как причина},
как абсолютное,
{\em свободное} по
отношению к объективности и ее внешней определяемости, конкретное единство.
Та внешность, в которую переводит себя цель, как мы видели, уже сама
положена как момент понятия, как форма его различения внутри себя. Цель
поэтому имеет во внешности {\em свой
собственный момент;} и содержание, как содержание
конкретного единства, есть ее
{\em простая форма},
которая в различенных моментах цели "--- как
субъективная цель, как средство и опосредствованная деятельность и как
объективная цель "--- остается равной себе не только
{\em в~себе}, но также и
существует как остающаяся равной себе.

Поэтому о телеологической деятельности можно сказать, что в
ней конец есть начало, следствие "--- основание, действие
"--- причина, что она есть становление уже ставшего, что в ней
получает существование только уже существующее и т. д., т. е. что вообще
все те соотносительные определения, которые принадлежат к сфере рефлексии
или непосредственного бытия, потеряли свои различия и что то, чт\'{о}
высказывается как то
{\em иное} "--- например,
конец, следствие, действие и~т.~д., "--- в соотношении цели уже
не имеет определения
{\em иного}, а скорее
положено как тождественное с простым понятием.

2. При более близком рассмотрении продукта телеологической
деятельности оказывается, что он содержит в~себе цель лишь внешним образом,
поскольку он есть абсолютная предпосылка по отношению к субъективной цели,
т.~е. поскольку не идут дальше того, что через средство целесообразная
деятельность ведет себя по отношению к объекту лишь механически и на место
одной безразличной его определенности полагает {\em другую}, столь же
внешнюю ему. Подобного рода определенность, приобретаемая объектом через
цель, отличается в общем от другой, чисто механической
определенности тем, что первая есть момент некоторого
{\em единства} и, стало
быть, хотя она внешня объекту, не есть, однако, в~себе самой нечто чисто
внешнее. Объект, обнаруживающий такого рода единство, есть некоторое целое,
к которому его части "--- его собственная внешность
"--- безразличны; он есть определенное,
{\em конкретное}
единство, соединяющее внутри себя различенные соотношения и
определенности. Это единство, которое не может быть постигнуто из
специфической природы объекта и которое со стороны определенного содержания
есть другое содержание, чем своеобразное содержание объекта,
{\em само по себе} не
есть механическая определенность, но в объекте оно еще механично. Как в
этом продукте целесообразной деятельности содержание цели и содержание
объекта внешни друг другу, так относятся между собой также и в других
моментах умозаключения определения этих моментов: в смыкающем среднем
термине "--- целесообразная деятельность и объект, являющийся
средством, а в субъективной цели (другом крайнем термине)
"--- бесконечная форма, как тотальность понятия, и содержание
цели. По тому {\em соотношению},
которым субъективная цель сомкнута с объективностью, как одна
посылка (а~именно соотношение объекта, определенного как средства, с пока
еще внешним объектом), так и другое соотношение (а~именно, соотношение
субъективной цели с тем объектом, который делают средством) одинаково суть
непосредственные соотношения. Это умозаключение страдает поэтому
недостатком формального умозаключения вообще, а именно тем, что те
соотношения, из которых оно состоит, сами не суть заключения или
опосредствования, а скорее уже предполагают наличие того заключения,
средством для получения которого они должны служить.

Если рассматривать одну
{\em посылку},
непосредственное соотношение субъективной цели с тем
объектом, который в силу этого соотношения становится средством, то
окажется, что цель не может непосредственно соотноситься с этим объектом;
ибо последний есть нечто столь же непосредственное, как и тот служащий
другим крайним термином объект, в котором цель должна быть выполнена
{\em через опосредствование}.
Поскольку они таким образом положены как
{\em разные}, между этой
объективностью и субъективной целью необходимо должно быть вставлено
некоторое средство их соотношения; но это средство точно так же есть
некоторый уже определенный целью объект, между объективностью которого и
телеологическим определением следует вставить некоторое новое средство, и
так далее до бесконечности. Тем самым положен
{\em бесконечный прогресс}
{\em опосредствования}. "---
То же самое имеет место и относительно другой
посылки, "--- относительно соотношения средства с еще
неопределенным объектом. Так как они безоговорочно самостоятельны, то они
могут быть соединены лишь в некотором третьем, и так далее до
бесконечности. "--- Или, скажем наоборот, так как посылки уже
предполагают {\em заключение},
то последнее, каково оно есть через указанные, лишь
непосредственные посылки, может быть лишь несовершенным. Заключение или
{\em продукт} целесообразной деятельности есть не~что иное, как некоторый
определенный некоторой внешней ему целью объект;
{\em он, стало быть, есть то же самое,
что и средство}. Поэтому в самом таком продукте получилось
{\em лишь некоторое средство}, а не {\em выполненная
цель;} или, иначе говоря: цель по существу дела не достигла
в нем никакой объективности. "--- Поэтому совершенно
безразлично, будем ли мы рассматривать определенный внешней целью объект
как выполненную цель или только как средство; это
"--- релятивное, самому объекту внешнее, необъективное
определение. Следовательно, все те объекты, которые воплощают в~себе
выполнение некоторой внешней цели, суть в такой же мере лишь средства к
цели. То, что должно быть употреблено для выполнения некоторой цели и что
по существу должно считаться средством, представляет собой средство именно
по тому своему определению, что оно обречено на изнашивание. Но и тот
объект, который должен содержать в~себе выполненную цель и являть себя как
ее объективность, тоже преходящ; он равным образом выполняет свою цель не
через спокойное, самосохраняющееся наличное бытие, а лишь постольку,
поскольку он изнашивается или потребляется; ибо лишь постольку он
соответствует единству понятия, поскольку его внешность, т.~е. его
объективность, снимается в этом единстве. "--- Дом, часы
и~т.~д. могут представляться целями по отношению к употребленным для их
изготовления орудиям; но камни, балки (или: колесики, оси) и так далее,
составляющие действительность цели, выполняют эту последнюю лишь через
испытываемое ими давление, через те химические процессы, действию которых в
связи с воздухом, светом, водой они предоставлены и от которых они
избавляют человека (в~отношении часов: через трение колесиков и~т.~д.).
Они, следовательно, выполняют свое назначение лишь через их употребление в
дело и изнашивание и соответствуют тому, чем они должны быть, лишь через
свое отрицание. Они не соединены с целью положительно, потому что они
содержат в~себе самоопределение лишь внешним образом и суть лишь
относительные цели или, можно также сказать, суть по существу лишь
средства.

Эти цели, как мы показали, имеют вообще некоторое ограниченное
содержание; их формой служит бесконечное самоопределение понятия, которое
(понятие) через это содержание ограничило себя до внешней единичности.
Ограниченное содержание делает эти цели несоответствующими бесконечности
понятия и неистинными; такая определенность предоставлена власти
становления и изменения уже через сферу необходимости, через бытие, и она
есть нечто преходящее.

3. Тем самым получается, как результат что внешняя
целесообразность, которая пока что имеет лишь форму телеологии, доходит,
собственно говоря, лишь до средств, а не до некоторой объективной цели, так
как субъективная цель остается внешним, субъективным определением, "---
или же, поскольку субъективная цель деятельна и выполняет
себя (хотя бы только в некотором средстве), она еще
{\em непосредственно}
связана с объективностью, погружена в последнюю; она сама
есть некоторый объект, и цель, можно сказать, постольку не доходит до
средства, так как нужно, чтобы цель была выполнена уже до того, как она
получит возможность осуществиться через некоторое средство.

Однако на самом деле результат есть не только некоторое
внешнее целевое соотношение, а истина этого соотношения, "---
внутреннее целевое соотношение и некоторая объективная цель.
Самостоятельная по отношению к понятию внешность объекта, которую цель
делает своим пред-положением,
{\em положена} в этом
пред-положении как некоторая несущественная видимость и уже
снята также и сама по себе; деятельность цели есть поэтому, собственно
говоря, лишь выявление этой видимости и ее снятие. "--- Как
обнаружилось через понятие, первый объект становится через сообщение
средством, так как он есть в~себе тотальность понятия, и его
определенность, которая не есть никакая другая определенность, кроме самой
внешности, положена лишь {\em как}
внешнее, несущественное и потому выступает в самой цели как
ее собственный момент, а не как нечто самостоятельное по отношению к ней.
В~силу этого определение объекта к тому, чтобы быть средством, есть
безоговорочно непосредственное определение. Поэтому для того, чтобы сделать
этот объект средством, субъективная цель не нуждается ни в каком насилии
или другом подтверждении своих прав в отношении объекта, кроме
подтверждения себя самой; {\em решение}
(Entschluss), раскрытие (Aufschluss), это
определение самого себя, есть {\em лишь
положенная} внешность объекта, который непосредственно
выступает здесь как подчиненный цели и не обладает по отношению к ней
никаким другим определением, кроме того, что его
в-себе-и-для-себя-бытие представляет собой нечто ничтожное.

Второе снятие объективности через объективность разнится от
рассмотренного снятия следующим образом: рассмотренное здесь снятие как
первое есть цель в объективной
{\em непосредственности},
второе же есть поэтому не только снятие некоторой первой
непосредственности, а снятие обоих "--- объективного, как лишь
положенного, и непосредственного. Отрицательность, значит, возвращается в
самое себя таким образом, что она есть в такой же мере и восстановление
объективности, но как объективности, тождественной с нею, и в этом
восстановлении она вместе с тем есть также и полагание объективности как
всего лишь определенной целью, внешней. Через последнее этот продукт
остается, как и раньше, тоже средством; через первое же он есть
тождественная с понятием объективность, реализованная цель, в которой тот
аспект, что она есть средство, представляет собой реальность самой цели.
В~выполненной цели средство исчезает потому, что оно было бы лишь
непосредственно подведенной под цель объективностью, между тем как в
реализованной цели эта объективность выступает как возвращение цели в самое
себя; тем самым, далее, и само опосредствование (представляющее собой
некоторое отношение внешнего) также исчезает отчасти в конкретное тождество
объективной цели, отчасти же в тождество как в абстрактное тождество и
непосредственность наличного бытия.

В этом содержится также и то опосредствование, которое
требовалось для первой посылки, для непосредственного соотношения цели с
объектом. Осуществленная цель есть также и средство, и обратно, истина
средства заключается равным образом в том, что оно есть сама реальная цель,
и первое снятие объективности есть уже и второе, точно так же как второе
снятие оказалось содержащим в~себе также и первое. А~именно, понятие
{\em определяет себя;} его определенность есть внешнее безразличие, которое
непосредственно определено в решении как {\em снятое}, а именно,
как {\em внутреннее, субъективное} и вместе с тем как {\em пред-положенный
объект}. Его дальнейшее выхождение за свои пределы, представлявшееся именно
{\em непосредственным}
сообщением и подведением под него пред-положенного объекта,
есть вместе с тем снятие указанной внутренней, {\em замкнутой в понятии},
т.~е. положенной как снятая определенности внешности, и
вместе с тем оно есть снятие пред-положения некоторого объекта; тем самым
это по видимости первое снятие безразличной объективности
есть уже также и второе, некоторая прошедшая сквозь опосредствование
рефлексия в~себя и выполненная цель.

Так как понятие здесь, в сфере объективности, где его определенность
имеет форму {\em безразличной внешности},
находится во взаимодействии с самим собой, то изображение его
движения становится здесь вдвойне трудным и запутанным, потому что само это
движение непосредственно есть нечто двоякое, и первое всегда есть в нем
также и второе. В~понятии самом по себе, т.~е. в его субъективности,
различие его от себя выступает как особая {\em непосредственная},
тождественная тотальность; ну, а так как здесь его
определенность есть безразличная внешность, то тождество в этой
определенности с самим собой есть в свою очередь столь же непосредственно и
отталкивание от себя, так что то, чт\'{о} определено как внешнее ему и
безразличное для него, скорее есть оно же само, а оно, как оно же само, как
рефлектированное в~себя, есть скорее свое иное. Только в том случае, если
не упускать этого из вида, может быть понято объективное возвращение
понятия в~себя, т.~е. его истинное объективирование, "--- может
быть понято, что каждый из отдельных моментов, через которые проходит это
опосредствование, сам есть целое умозаключение, состоящее из всех этих
моментов. Таким образом, первоначальная
{\em внутренняя}
внешность понятия (в~силу которой оно есть отталкивающее себя
от себя единство, цель и ее стремление во-вне, к объективированию) есть
непосредственное полагание или пред-полагание некоторого внешнего объекта;
{\em самоопределение}
есть также и определение некоторого
{\em внешнего} объекта,
как не понятием определенного; и обратно, это определение есть
самоопределение, т.~е. снятая,
{\em положенная} как
{\em внутренняя}
внешность или, иначе говоря,
{\em достоверность несущественности}
внешнего объекта. "--- Касательно второго
соотношения, определения объекта как средства, было только что показано,
как оно в самом себе есть опосредствование цели в объекте с самой собой. "---
И точно так же третий момент, механизм, функционирующий под
господством цели и снимающий объект через объект, есть, с одной стороны,
снятие средства, т.~е. того объекта, который уже положен как снятый, и,
стало быть, второе снятие и рефлексия в~себя, а с другой стороны, первый
процесс определения внешнего объекта. Последний процесс определения, как мы
заметили выше, есть в выполненной цели опять-таки лишь
продуцирование некоторого средства: субъективность конечного понятия,
презрительно отметая средство, не достигла в своей цели ничего лучшего. Но
эта рефлексия, что цель достигнута в средстве и что в выполненной цели
сохранились средство и опосредствование, есть
{\em последний результат внешнего
целевого соотношения}, результат, в котором оно сняло само
себя и который оно явило как свою истину. "--- Рассмотренное
напоследок третье умозаключение отличается от других тем, что оно есть,
во-первых, субъективная целевая деятельность предыдущих умозаключений, но,
во-вторых, также и снятие внешней объективности (и, значит, внешности
вообще) {\em через самое себя},
и тем самым оно есть
{\em тотальность в ее положенности}.

Итак, после того как
{\em субъективность},
{\em для-себя-бытие}
понятия перешло, как мы видели, в его
{\em в-себе-бытие}, в
{\em объективность},
оказалось в дальнейшем, что в последней снова обнаружилась
отрицательность, свойственная его для-себя-бытию; понятие определило себя в
объективности так, что его
{\em особенность} есть
{\em внешняя объективность},
или, иначе говоря, оно определило себя как простое конкретное
единство, внешность которого есть его собственное самоопределение. Движение
цели достигло теперь того, что момент внешности не только положен в понятии
и понятие есть не только некоторое
{\em долженствование} и
{\em стремление}, но, как
конкретная тотальность, тождественно с непосредственной объективностью. Это
тождество есть, с одной стороны, простое понятие и равным образом
{\em непосредственная}
объективность, но, с другой стороны, оно столь же существенно
есть {\em опосредствование},
и лишь через последнее, как снимающее само себя
опосредствование, оно есть эта простая непосредственность; таким образом,
понятие состоит существенно в том, что оно, как для-себя-сущее тождество,
отлично от своей
{\em в-себе-сущей}
объективности и в силу этого обладает внешностью, но в этой
внешней тотальности образует ее самоопределяющее тождество. Таким образом,
понятие есть теперь {\em идея}.

\clearpage

\part[Третий отдел\newline ИДЕЯ]{Третий отдел\newline Идея\pagenote{Гегель
определяет <<идею>> как
адекватное понятие, как субъект-объект, как единство понятия и
действительности (или единство понятия и объективности). О~том, чт\'{о} Гегель
разумеет под <<понятием>>, см.~примечание~1. Для правильного понимания того
содержания, которое скрывается в гегелевской категории <<идеи>>, огромное
значение имеют указания Ленина, который дает
{\em двоякую} расшифровку этой категории: 1) идея~=~познание человека
({\em Ленин}, Философские тетради, М.~1936, стр.~187) и~2) идея~=~сама
природа (там же, стр.~198) или идея~=~объективная действительность
(стр.~192). Этот двуединый характер гегелевской идеи, с одной стороны, таит
в~себе глубочайшую фальшь гегелевского идеализма "--- учение о
тождестве мышления и бытия, сводящее бытие к мышлению, а с другой стороны,
здесь же скрывается гениальная догадка Гегеля относительно
<<{\em общих законов движения мира} и {\em мышления}>> ({\em Ленин},
Философские тетради, стр.~170). Эту вторую сторону
гегелевской <<идеи>> имеет в виду Ленин, когда он по поводу гегелевского
Введения к отделу <<Идея>> замечает, что <<Гегель гениально {\em угадал}
диалектику вещей (явлений, мира, {\em природы})
в диалектике понятий>> (там же, Стр.~189). Этой же стороны
дела касается и Энгельс в своих <<Примечаниях>> к <<Анти-Дюрингу>>. Говоря о
заслугах новейшей <<идеалистической, но вместе с тем и диалектической
философии, в особенности Гегеля>>, Энгельс замечает: <<Несмотря на
бесчисленные произвольные и фантастические построения этой философии,
несмотря на идеалистическую, на голову поставленную форму ее конечного
результата "--- единства мышления и бытия, "---
нельзя отрицать того, что она доказала на множестве примеров,
взятых из самых разнообразных отраслей знания, аналогию между процессами
мышления и процессами в области природы и истории "--- и
обратно, и господство одинаковых законов для всех этих процессов>>
({\em Энгельс}, Анти-Дюринг, Партиздат, 1936, стр.~244). Вот почему Ленин
указывает, что именно во Введении к отделу <<Идея>> "--- этому
завершающему отделу всей гегелевской Логики "--- у Гегеля
<<{\ldots}замечательно гениально показано совпадение, так сказать, логики и
гносеологии>>
({\em Ленин},
Философские тетради, стр.~185), т.~е. логики, как науки о
законах развития всех материальных, природных и духовных вещей, и
гносеологии, как науки о законах развития человеческого
познания.\label{bkm:bm86}}}

Идея есть {\em адекватное понятие}, объективная
{\em истина} или
{\em истина как таковая}.
Если что-либо истинно, оно истинно через свою идею или, иначе
говоря, {\em нечто истинно лишь
постольку, поскольку оно есть идея}. "--- Выражение
<<{\em идея}>> часто
употреблялось как в философии, так и в обыденной жизни также и для
обозначения {\em понятия}
и даже для обозначения простого
{\em представления}: <<Я
еще не имею никакой {\em идеи}
об этой тяжбе, об этом здании, об этой местности>>
"--- означает только, что я о них не имею
{\em представления}. Кант
вновь потребовал, чтобы выражению
<<{\em идея}>> был
возвращен его настоящий смысл
{\em понятия разума}.
"--- Понятие же разума есть, согласно Канту, понятие о
{\em безусловном}, а в
отношении явлений оно-де
{\em трансцендентно},
т.~е. нельзя сделать из этого понятия
{\em адекватного ему эмпирического
употребления}. Понятия разума, согласно Канту, служат для
{\em постижения} (Begreifen),
а понятия рассудка "--- для понимания (Verstehen)
восприятий. "--- На самом же деле, если последние суть
действительно {\em понятия}, то {\em они}
{\em суть понятия} "--- мы посредством их постигаем, и {\em понимание}
восприятий посредством понятий рассудка будет {\em постижением}. Если
же понимание есть только процесс определения восприятий посредством таких
определений, как, например, целое и части, сила, причина и тому подобное,
то оно означает лишь некоторый процесс определения посредством рефлексии,
равно как и под <<{\em пониманием}>>
можно подразумевать всего лишь определенное {\em представление}
совершенно определенного чувственного содержания; так, если,
описывая человеку дорогу, говорят, что он в конце леса должен повернуть
налево, и он примерно отвечает: <<{\em понимаю}>>, то это
<<{\em понимание}>> не означает ничего большего, как усвоение сказанного
представлением и памятью. "--- <<{\em Понятие разума}>>
тоже является несколько неловким выражением; ибо понятие есть
вообще нечто разумное; а поскольку разум отличают от рассудка и от понятия
как такового, он есть тотальность понятия и объективности. "---
В этом смысле идея есть {\em разумное;} она есть
безусловное потому, что лишь то имеет условия, что существенно соотносится
с некоторой объективностью, но не с такой объективностью, которая
определена им самим, а с такой, которая еще выступает в форме внешности ему
и безразличия к нему, как это еще имело место во внешней цели.

Раз выражение <<идея>> резервируется для обозначения
объективного или реального понятия и его отличают от самого понятия, а еще
больше "--- от простого представления, то следует, далее, еще в
большей мере отвергнуть ту оценку идеи, согласно которой ее принимают за
нечто лишь недействительное и об истинных мыслях говорят, что
<<{\em это "--- только
идеи}>>. Если {\em мысли}
суть нечто лишь
{\em субъективное} и
случайное, то они, разумеется, не имеют никакой дальнейшей ценности, но они
в этом отношении стоят не ниже временных и случайных
{\em действительностей},
которые равным образом не имеют никакой дальнейшей ценности,
помимо ценности случайностей и явлений. Если же полагают, что идея,
наоборот, не имеет ценности истины потому, что она, мол, в отношении
явлений {\em трансцендентна},
что ей не может быть дан в чувственном мире совпадающий с нею
предмет\pagenote{<<Weil ihr kein kon\-gruie\-render Gegen\-stand in der
Sinnen\-welt gegeben werden konne>> "--- почти дословное
воспроизведение кантовской формулировки из <<Критики чистого разума>>,
2-е~нем. изд., стр.~383.\label{bkm:bm87}},
то это "--- странное недоразумение, ибо идее
здесь отказывают в объективной значимости потому, что ей, дескать,
недостает того, что составляет собой явление,
{\em неистинное бытие}
объективного мира. В~отношении практических идей Кант
признает, что <<нельзя найти ничего более вредного и более недостойного для
философа, чем {\em вульгарная}
ссылка на якобы противоречащий идее
{\em опыт}. Самого этого
опыта не существовало бы, если бы, например, государственные учреждения
были устроены в свое время сообразно идеям и
{\em грубые понятия}, господствующие вместо этих идей именно на том
основании, что они {\em почерпнуты из опыта}, не сделали бы тщетными
все добрые намерения>>\pagenote{Цитата взята (с~маленькими
несущественными изменениями) из <<Критики чистого
разума>>, 2-е нем. изд., стр.~373. Курсив принадлежит Гегелю.\label{bkm:bm88}}.
Кант смотрит на идею, как на нечто необходимое, как на цель
стремлений, которая должна быть выдвинута как
{\em прообраз} некоторого
максимума и к которой следует стремиться все больше и больше приблизить
состояние действительности.

Но так как у нас в качестве результата получился тот вывод,
что идея есть единство понятия и объективности, иначе говоря
"--- истина, то ее надлежит рассматривать не только как
некоторую цель стремлений, к которой следует приближаться, но которая сама
всегда остается некоторого рода
{\em потусторонним}, а
так, что все действительное {\em имеет
бытие} лишь постольку, поскольку оно имеет внутри себя идею
и выражает ее. Предмет, объективный и субъективный мир, не только
{\em должны} вообще
{\em совпадать}\pagenote{<<Kongruieren>>: опять намек на
Канта (см. примечание \ref{bkm:Ref484779980}).
Самый термин <<Kongruieren>> заимствован из геометрии, где он
обозначает совпадение при наложении друг на друга двух равных и подобных
фигур (<<конгруэнтные фигуры>>).\label{bkm:bm89}}
с идеей, но сами суть совпадение понятия и реальности;
реальность, не соответствующая понятию, есть только
{\em явление},
субъективное, случайное, произвольное, которое не есть
истина. Когда говорят, что в опыте мы не находим ни одного такого предмета,
который вполне совпадал бы с
{\em идеей}, то последняя
противопоставляется действительному как некоторый
субъективный масштаб. Но что поистине представляет собой некоторое
действительное, если его понятие в нем не находится и его объективность
вовсе не соответствует этому понятию, "--- этого никто не может
сказать, ибо такое действительное было бы ничто. В~механическом и
химическом объекте, равно как и в бездуховном (geistlos)
субъекте и в духе, сознающем лишь конечное, а не свою
сущность, понятие, которое они имеют в~себе сообразно их различной природе,
существует, правда, не {\em в своей
собственной свободной форме}. Но они могут быть вообще
чем-то истинным лишь постольку, поскольку они суть соединение их понятия и
реальности, их души и их тела. Такие целые, как государство, церковь,
перестают существовать, когда разрушается единство их понятия и их
реальности; человек (и живое вообще) мертв, когда в нем отделились друг от
друга душа и тело. Мертвая природа "--- механический и
химический мир (если именно мы будем понимать под мертвым неорганическую
природу; в противном случае это слово не имело бы никакого положительного
значения), "--- если ее разделяют на ее понятие и на ее
реальность, есть только субъективная абстракция некоторой мыслимой формы и
некоторой бесформенной материи. Дух, который не был бы идеей, единством
самого понятия с собой, понятием, имеющим своею реальностью само понятие,
был бы мертвым, лишенным духа духом, материальным объектом.

{\em Бытие} достигло
значения {\em истины},
так как идея есть единство понятия и реальности; бытием
обладает теперь, следовательно, лишь то, что представляет собой идею.
Поэтому конечные вещи конечны постольку, поскольку они не имеют реальности
своего понятия полностью в них же самих, а нуждаются для этого в других, "---
или, скажем наоборот, поскольку они предполагаются как
объекты и тем самым содержат в~себе понятие как некоторое внешнее
определение. Самым высоким из того, чего они достигают со стороны этой
конечности, является внешняя целесообразность. То обстоятельство, что
действительные вещи не совпадают (kongruieren) с идеей, есть
аспект их {\em конечности,
неистинности}, по которому они суть
{\em объекты} и каждая
сообразно своей различной сфере и в присущих объективности отношениях
определена механически, химически или некоторой внешней целью. Возможность
того, что идея не вполне обработала свою реальность, не полностью подчинила
ее понятию, покоится на том, что сама она обладает
{\em ограниченным содержанием},
на том, что, как бы существенно она ни была единством понятия
и реальности, она столь же существенно есть также и их различие; ибо
только объект есть непосредственное, т.~е. лишь
{\em в-себе}-сущее
единство. А~если бы какой-нибудь предмет, например, государство,
{\em вовсе не}
соответствовал своей идее, т.~е., лучше сказать, если бы
какое-нибудь государство ни в какой мере не было идеей государства, если бы
его реальность, которою являются самосознательные индивидуумы, совершенно
не соответствовала понятию, то это означало бы, что тут отделились друг от
друга его душа и его тело; первая отлетела бы в отрешенные сферы мысли, а
последнее распалось бы на отдельные индивидуальности. Но так как понятие
государства столь существенно составляет их природу, то это понятие имеет в
них бытие как столь могущественное движущее начало, что они вынуждены
переводить его в реальность (хотя бы только в форме внешней
целесообразности) и мириться с его существованием в данном его виде, или в
противном случае им пришлось бы погибнуть. Самое плохое государство,
реальность которого менее всего соответствует понятию, поскольку оно еще
существует, все еще есть идея; индивидуумы все еще повинуются некоторому
власть имущему понятию.

Но идея имеет не только более общий смысл {\em истинного} {\em бытия}, единства
{\em понятия} и {\em реальности}, но и более определенный смысл единства
{\em субъективного понятия} и {\em объективности}. Ведь понятие как таковое
само уже есть тождество себя и {\em реальности;} ибо неопределенное выражение
<<реальность>> не означает вообще ничего другого, кроме {\em определенного
бытия;} а последним понятие обладает в своей особенности и единичности. Далее,
{\em объективность} равным образом есть вышедшее из своей определенности
в {\em тождество}, слившееся с самим собой, тотальное {\em понятие}.
В~субъективности определенность или различие понятия есть некоторая
{\em видимость}, которая непосредственно снята и вернулась обратно
в для-себя-бытие или отрицательное единство, есть {\em присущий} субъекту
предикат. А~в объективности определенность положена как непосредственная
тотальность, как внешнее целое. Идея теперь обнаружила себя как понятие,
снова освободившееся от той непосредственности, в которую оно было
погружено в объекте, освободившееся, чтобы обрести свою субъективность, и
отличающее себя от своей объективности, которая, однако, вместе с тем
определена им самим и имеет свою субстанциальность лишь в этом понятии. Это
тождество было поэтому справедливо определено как
{\em субъект-объект}\pagenote{Выражение <<субъект-объект>>
встречается как у Фихте, так и у Шеллинга. У~Фихте мы находим его,
например, в его <<System der Sittenlehre>> 1798~г. (см. Fichte, Werke, hrsg.
v. Medicus, Bd.~II, S.~436, 454, 524, 531).
Шеллинг употребляет это выражение, например, в <<Системе
трансцендентального идеализма>> 1800~г.
({\em Schelling}, Werke, hrsg. v.~O.~Weiss, Bd.~II, S.~47, 63) и в диалоге
<<Бруно>>, вышедшем в 1802~г. (См.~{\em Шеллинг},
Философские исследования о сущности человеческой свободы.
Бруно или о божественном и естественном начале вещей, Спб. 1908,
стр.~163).\label{bkm:bm90}}; оно есть {\em в такой же
мере} формальное или субъективное понятие,
{\em в какой} и объект
как таковой. Но это следует понимать более определенным образом. Понятие,
достигнув поистине своей реальности, есть абсолютное суждение,
{\em субъект} которого,
как соотносящееся с собой отрицательное единство, отличает
себя от своей объективности и есть ее в-себе-и-для-себя-бытие, но
существенно соотносится с нею через самого себя и есть поэтому
{\em самоцель} и {\em стремление;}
объективности же именно в силу этого субъект не имеет
непосредственно в самом себе; будь это так, субъект был бы лишь потерянной
в объективность тотальностью объекта как такового; на самом же деле его
объективность есть реализация цели, объективность,
{\em положенная}
деятельностью цели, объективность, которая как
{\em положенность} имеет
свое существование и свою форму лишь как проникнутые ее субъектом. Как
объективность, она имеет в~себе момент
{\em внешности} понятия и
образует собой поэтому вообще сторону конечности, изменчивости и явления,
находящую, однако, свою гибель в возвращении обратно в отрицательное
единство понятия; та отрицательность, в силу которой ее безразличная
внеположность обнаруживает себя чем-то несущественным и положенностью, есть
само понятие. Поэтому идея, несмотря на эту объективность, безоговорочно
{\em проста} и {\em имматериальна;} ибо
внешность имеет бытие лишь как определенная понятием и вобрана в
отрицательное единство понятия; поскольку идея существует как безразличная
внешность, она не только отдана вообще во власть механизма, но имеет бытие
лишь как нечто преходящее и неистинное. "--- Следовательно,
хотя идея имеет свою реальность в некоторой материальности, последняя все
же не есть некоторое абстрактное
{\em бытие},
самостоятельно существующее по отношению к понятию, а
выступает только как {\em становление},
через отрицательность безразличного бытия, как простая
определенность понятия.

Из вышесказанного получаются следующие более детальные
определения идеи. "--- Она есть, {\em во-первых}, простая
истина, тождество понятия и объективности как {\em всеобщее}, в котором
противоречие (Gegensatz) и устойчивое наличие особенного
разрешены в его тождественную с собой отрицательность и выступают как
равенство с самим собой. {\em Во-вторых}, она есть {\em соотношение}
для-себя-сущей субъективности простого понятия и его {\em отличенной} от нее
объективности: первая есть существенно {\em стремление}
уничтожить это разделение, а последняя
"--- безразличная положенность, само по себе ничтожное
устойчивое наличие. Как это соотношение, идея есть {\em процесс} расщепления
себя на индивидуальность и на ее неорганическую природу, приведения этой
последней вновь под власть субъекта и возвращения обратно к первой простой
всеобщности. {\em Тождество} идеи с самой собой едино с {\em процессом;} мысль,
освобождающая действительность от видимости бесцельной
изменчивости и преображающая ее в {\em идею}, не должна
представлять себе эту истину действительности как мертвый покой, как
простой {\em образ}, тусклый и бессильный, без стремления и движения, как
некоторого гения, или число\pagenote{Это, по-видимому, намек на
Шеллинга, причем слово <<гений>> имеет в виду <<Систему трансцендентального
идеализма>> Шеллинга (1800~г.), а слово <<число>> "--- его
натурфилософию (ср.~примечание~41 к~т.~I <<Науки логики>>). В~<<Системе
трансцендентального идеализма>> Шеллинг утверждает, что абсолютное дано
человеку в интеллектуальном и эстетическом созерцании и что для этого
созерцания требуется особого рода <<гениальность>> (das Genie),
понятие которой вводится у Шеллинга следующим образом:
<<Подобно тому, как именуется роком та сила, которая через нашу свободную
деятельность без нашего ведома и даже наперекор нашему желанию осуществляет
цели, {\em о которых люди не думали}, "--- так мы обозначаем темным понятием
{\em гениальности} то непостижимое начало, которое без всякого содействия
свободы и до известной степени даже вопреки последней придает осознанному
объективность, тогда как в области свободы то, что соединено в продукте,
вечно убегает от самого себя>>
({\em Schelling}, Werke, hrsg.~v.~Weiss. Bd.~II, S.~290).
В~другом месте Шеллинг определяет гениальность как <<совпадение
бессознательной и сознательной деятельности>> (там же, стр.~298). Можно
думать, что на Шеллинга (а~именно, на его романтическое превознесение
искусства как <<всеобщего органа философии>>) намекают и слова <<как простой
образ>> (als blosses Bild) двумя строками выше, тем более что
дальнейшее определение этого <<образа>> словами <<без стремления и движения>>
напоминает следующие слова Шеллинга из той же <<Системы трансцендентального
идеализма>>: <<Всякое стремление к продуцированию прекращается, как только
продукт завершен; все противоречия устранены, все загадки разрешены>> (там
же, стр.~289). "--- С другой стороны, гегелевские слова <<как простой {\em образ},
тусклый и бессильный>> напоминают известное место из <<Этики>>
Спинозы, где Спиноза полемизирует с теми, кто <<смотрит на идеи как на немые
изображения на картине и под влиянием этого предрассудка не видит, что
идея, поскольку она есть идея, заключает в~себе утверждение или отрицание>>
(схолия к теореме~49 второй части).\label{bkm:bm91}},
или некоторую абстрактную мысль; в силу свободы, которую
понятие достигает в идее, идея имеет внутри себя также и
{\em жесточайшее противоречие}
(Gegensatz); ее покой состоит в твердости и уверенности, с
которыми она вечно порождает это противоречие и вечно его преодолевает и в
нем сливается с самой собою.

Однако вначале идея опять-таки пока что лишь
{\em непосредственна}
или, иначе говоря, находится лишь в своем
{\em понятии;}
объективная реальность, правда, адекватна понятию, но она еще
не освобождена до того, чтобы стать понятием, и последнее не существует
{\em для себя как понятие}.
Таким образом, хотя понятие есть
{\em душа}, но душа
выступает здесь в виде чего-то
{\em непосредственного},
т.~е. ее определенность выступает не как она сама; она не
ухватила себя как душу, не ухватила в~себе самой своей объективной
реальности; понятие выступает как такая душа, которая еще не
{\em полна души}.

Таким образом, идея есть,
{\em во-первых},
{\em жизнь;} это
"--- понятие, которое, отличенное от своей объективности,
простое внутри себя, пронизывает свою объективность и как самоцель имеет в
ней свое средство и полагает ее как свое средство, но имманентно в этом
средстве и есть в нем реализованная, тождественная с собой цель. Вследствие
своей непосредственности эта идея имеет формой своего существования
{\em единичность}. Но
рефлексия ее абсолютного процесса в~себя самого есть снятие этой
непосредственной единичности; этим понятие, которое, как всеобщность, есть
в ней {\em внутреннее},
делает внешность всеобщностью или, иначе говоря, полагает
свою объективность как равенство с самим собой. Таким образом, идея есть,

{\em во-вторых}, идея
{\em истины} и
{\em добра} как
{\em познание} и
{\em воля}. Вначале она
есть конечное познание и конечная воля, в которых истина и добро еще
отличаются друг от друга и оба выступают пока что только как цель
стремлений. Понятие вначале освободило
{\em себя} в качестве
самого себя и дало себе в реальность пока что лишь некоторую
{\em абстрактную объективность}.
Но процесс этого конечного познавания и действования
превращает первоначально абстрактную всеобщность в тотальность, благодаря
чему она становится {\em законченной
объективностью}. "--- Или, если рассматривать этот процесс с
другой стороны, то можно сказать так: конечный, т.~е. субъективный дух
{\em создает} себе
{\em предпосылку}
некоторого объективного мира, подобно тому как жизнь
{\em обладает} такою
предпосылкою; но его деятельность заключается в том, чтобы
снять эту предпосылку и сделать ее чем-то положенным. Таким образом, его
реальность есть для него объективный мир, или, обратно, объективный мир
есть та идеальность, в которой он познает самого себя.

{\em В-третьих}, дух
познает идею как свою {\em абсолютную
истину}, как ту истину, которая имеет бытие в~себе и для
себя. Это "--- бесконечная идея, в которой познание и действие
сравнялись друг с другом и которая есть
{\em абсолютное знание о самой себе}.

\chapter[Первая глава Жизнь]{Первая глава\newline Жизнь}

Идея жизни касается такого конкретного и, если угодно,
реального предмета, что согласно обычному представлению о логике может
показаться, будто, трактуя об этой идее, мы выходим за пределы логики.
Разумеется, если логика не должна содержать в~себе ничего другого, кроме
пустых, мертвых форм мысли, то в ней не могла бы вообще идти речь о такого
рода содержании, как идея или жизнь. Но если предметом логики служит
абсолютная истина, а {\em истина}
как таковая имеет бытие существенным образом
{\em в познавании}, то
следовало бы по крайней мере рассмотреть {\em познавание}. "--- И в
самом деле, вслед за так называемой чистой логикой обыкновенно дают
{\em прикладную} логику "--- логику, имеющую дело с
{\em конкретным познаванием}, "--- не говоря уже о той большой порции
{\em психологии} и {\em антропологии},
вплетение которой в логику часто считается необходимым. Но
антропологическая и психологическая сторона познавания касается его
{\em явления}, в котором
понятие еще не состоит для самого себя в том, чтобы обладать равной ему
объективностью, т.~е. иметь предметом само себя. Та часть логики, которая
рассматривает это конкретное познавание, не должна входить в
{\em прикладную логику}
как таковую; в противном случае пришлось бы включить в логику
все науки, ибо каждая наука есть постольку прикладная логика, поскольку она
состоит в том, чтобы облекать свой предмет в формы мысли и понятия. "---
Субъективное понятие имеет предпосылки, которые являют себя в
психологической, антропологической и других формах. Но в логику предпосылки
чистого понятия должны входить лишь постольку, поскольку они имеют форму
чистых мыслей, абстрактных сущностей, "--- определения
{\em бытия} и {\em сущности}. И~точно
так же предпосылки {\em познавания}
(постижения понятием самого себя) должны
рассматриваться в логике не во всех своих видах, а лишь та
его предпосылка, которая сама есть идея; но эта последняя предпосылка уже с
необходимостью должна быть рассмотрена в логике. Этой предпосылкой служит
{\em непосредственная}
идея; ибо так как познание есть понятие, поскольку оно имеет
самостоятельное бытие, но как субъективное находится в соотношении с
объективным, то понятие соотносится здесь с идеей как с
{\em пред-положенной} или
{\em непосредственной}.
Но непосредственная идея есть
{\em жизнь}.

Постольку необходимость рассматривать в логике идею жизни
основывалась бы на и помимо этого признаваемой необходимости трактовать
здесь о конкретном понятии познания. Но эта идея ввела себя в наше
изложение в силу собственной необходимости понятия.
{\em Идея}, в~себе и для
себя {\em истинное}, есть
существенно предмет логики; так как она сначала должна быть рассмотрена в
своей непосредственности, то она должна быть ухвачена и познана в той
определенности, в которой она есть
{\em жизнь}, дабы
рассмотрение ее не оказалось чем-то пустым и лишенным определений. Здесь,
пожалуй, можно лишь отметить, насколько логическая картина жизни отличается
от других научных картин ее; однако здесь не место говорить о том, как она
трактуется в нефилософских науках, а следует только указать, чем отличается
логическая жизнь как чистая идея от природной жизни, рассматриваемой в
{\em философии природы},
и от жизни, поскольку она находится в связи с
{\em духом}. "--- Первая как
жизнь природы есть жизнь, поскольку она выброшена во
{\em внешность существования}
и имеет свое
{\em условие} в
неорганической природе, причем моменты идеи суть некоторое многообразие
действительных образований. Жизнь в идее не имеет таких
{\em предпосылок},
выступающих как образы действительности; ее предпосылкой
служит понятие, как мы его рассмотрели выше, "--- понятие, с
одной стороны, как субъективное, а с другой стороны, как объективное.
В~природе жизнь выступает как та наивысшая ступень, которая достигается ее
(природы) внешним характером благодаря тому, что эта внешность ушла внутрь
себя и снимает себя в субъективности. В~логике же именно простое
внутри-себя-бытие достигло в лице идеи жизни своей истинно ему
соответствующей внешности; понятие, выступавшее раньше как субъективное,
есть теперь душа самой жизни; оно есть то движущее начало, которое, проходя
сквозь объективность, опосредствует для себя свою реальность. Когда
природа, беря исходным пунктом свою внешность, достигает этой идеи, она
выходит за свои пределы; ее конец имеет бытие не как ее начало, а как ее
граница, в которой она сама себя снимает. "--- И точно так же
моменты реальности жизни получают в идее жизни не образ
внешней действительности, а остаются заключенными в форму понятия.

В {\em духе} же жизнь выступает отчасти как противостоящая ему, отчасти же
как положенная единой с ним, а это единство "--- как снова
порожденное чисто им. А~именно, жизнь следует брать здесь вообще в
собственном смысле этого слова, брать как
{\em природную жизнь;} ибо то, что называют {\em жизнью духа} как
духа, есть его своеобразие, которое противостоит голой жизни; говорят ведь
также и о {\em природе}
духа, хотя дух есть не нечто природное, а, наоборот,
противоположность природе. Стало быть, жизнь как таковая есть для духа
отчасти {\em средство}
(взятую таким образом, дух противопоставляет ее себе);
отчасти он есть живой индивидуум, и жизнь есть его тело; отчасти же это его
единство с его живой телесностью порождается из него самого как
{\em идеал}. Ни одно из
этих соотношений с духом не касается логической жизни, и ее здесь не
следует рассматривать ни как средство некоторого духа, ни как его живое
тело, ни как момент идеала и красоты. "--- Жизнь в обоих
случаях "--- жизнь как {\em природная} и жизнь
как находящаяся в соотношении с {\em духом} "--- обладает
некоторой {\em определенностью своей
внешности;} в первом случае "--- в силу своих
предпосылок, которые суть другие образования природы, во втором же случае
"--- в силу целей и деятельности духа. Идея жизни, взятая сама
по себе, свободна от служащей предпосылкой и обусловливающей объективности,
равно как и от соотношения с субъективностью духа.

Жизнь, рассматриваемая ближе в ее идее, есть в~себе и для себя
абсолютная всеобщность; та объективность, которой жизнь обладает в~себе
самой, всецело проникнута понятием и имеет субстанцией только его. То, что
различает себя как часть или согласно какому-нибудь другому внешнему
соображению, имеет в самом себе все понятие целиком; понятие есть
вездесущая во всех частях душа, остающаяся простым соотношением с самой
собой и единой в том многообразии, которое свойственно объективному бытию.
Это многообразие как внешняя себе объективность обладает безразличным
устойчивым существованием, которое в пространстве и во времени (если можно
было бы уже здесь упомянуть о них) есть совершенно разная и самостоятельная
внеположность. Но в жизни внешность выступает вместе с тем как
{\em простая определенность}
ее понятия; таким образом, душа вездесуще излита в это
многообразие и вместе с тем безоговорочно остается простой единостью
конкретного понятия с самим собою. "--- При рассмотрении жизни,
этого единства ее понятия во внешней объективности, в абсолютной
множественности атомистической материи, у мышления,
держащегося определений отношений рефлексии и формального понятия,
совершенно иссякают все его мысли; вездесущность простого в многообразной
внешности представляет собой для рефлексии абсолютное противоречие, а
поскольку рефлексии приходится вместе с тем принять эту вездесущность из
восприятия жизни и тем самым признать действительность этой идеи, она есть
для рефлексии {\em непостижимая тайна},
так как рефлексия не ухватывает понятия или ухватывает его не
как субстанцию жизни. "--- Но простая жизнь не только
вездесуща, но безоговорочно есть
{\em устойчивое наличие}
и {\em имманентная
субстанция} своей объективности; а как субъективная
субстанция она существенно есть
{\em влечение}, и притом
{\em специфическое влечение особенного}
различия, и столь же существенно она есть единое и всеобщее
влечение специфичности, которое приводит это свое обособление обратно к
единству и сохраняет его в последнем. Лишь как это
{\em отрицательное единство}
своей объективности и своего обособления жизнь есть
соотносящаяся с собой, для себя сущая жизнь, душа. Она тем самым
существенно есть {\em единичное},
которое соотносится с объективностью, как с некоторым {\em иным},
как с неживой природой. Поэтому изначальное
{\em суждение}\pagenote{<<Das Urteil>> в смысле <<перводеления>> (см. примечание \ref{bkm:bm26}).\label{bkm:bm92}}
жизни состоит в том, что она как индивидуальный субъект
отделяет себя от объективного и, конституируясь как отрицательное единство
понятия, создает {\em предпосылку} некоторой непосредственной объективности.

{\em Жизнь} должна быть поэтому рассматриваема, {\em во-первых}, как {\em живой
индивидуум}, который есть для себя субъективная тотальность и пред-положен как
безразличный к некоторой противостоящей ему, как безразличная, объективности.

{\em Во-вторых}, жизнь есть {\em жизненный процесс}, состоящий в том, чтобы
снять свою предпосылку, положить безразличную к жизни объективность как
отрицательную и осуществить себя как ее (объективности) мощь и отрицательное
единство. Этим жизнь делает себя таким всеобщим, которое есть единство себя
самого и своего иного. Жизнь есть поэтому,

{\em в-третьих}, {\em процесс рода}, заключающийся в том, что она снимает свою
единичность и относится к своему объективному наличному бытию, как к самой
себе. Этот процесс есть, стало быть, с одной стороны, возврат к ее понятию и
повторение первого расщепления, становление некоторой новой индивидуальности и
смерть первой непосредственной индивидуальности; но с другой стороны,
{\em ушедшее в~себя понятие} жизни есть становление относящегося к самому себе
понятия, существующего для себя как всеобщее и свободное, "--- переход
в {\em познание}\pagenote{По поводу этого перехода от жизни к познанию Энгельс
делает следующее замечание в <<Диалектике природы>>: <<Когда Гегель переходит
от жизни к познанию через посредство оплодотворения (размножения), то в этом
находится уже в зародыше теория развития, учение о том, что раз дана
органическая жизнь, то она должна развиться путем развития поколений до породы
мыслящих существ>> ({\em Энгельс}, Диалектика природы, Партиздат, 1936,
стр.~46).\label{bkm:bm93}}.

\section[А. Живой индивидуум]{А. Живой индивидуум}

1. Понятие жизни или всеобщая жизнь есть непосредственная
идея, понятие, которому его объективность соответственна; но последняя
соответственна ему лишь постольку, поскольку оно есть отрицательное
единство этой внешности, т.~е. поскольку оно
{\em полагает} ее
соответственной себе. Бесконечное соотношение понятия с самим собой как
отрицательность есть процесс самоопределения, расщепление понятия на себя
{\em как субъективную единичность и на
себя как безразличную всеобщность.} Идея жизни в своей
непосредственности есть пока что лишь творческая всеобщая душа. Вследствие
этой непосредственности первое отрицательное соотношение идеи внутри самой
себя есть ее определение себя как {\em понятия}, "--- полагание
{\em в~себе}, которое лишь как возвращение в~себя есть {\em для-себя-бытие},
творческое {\em пред-полагание}. Через этот процесс самоопределения
{\em всеобщая} жизнь есть некоторое {\em особенное;}
этим она раздвоила себя на два крайние термина суждения,
которое непосредственно становится умозаключением.

Определения этой противоположности суть всеобщие {\em определения понятия},
ибо именно понятию свойственно раздвоение; но {\em исполнением} этих
определений служит идея. Одной стороной служит {\em единство} понятия и
реальности, которое есть идея, "--- как то {\em непосредственное}
единство, которое раньше выступало как {\em объективность}.
Однако здесь это единство имеет другое определение. Там оно
было единством понятия и реальности, поскольку понятие перешло в
объективность и только потерялось в ней; понятие не противостояло ей, или,
так как понятие есть для нее нечто лишь
{\em внутреннее}, то оно есть лишь {\em внешняя}
ей рефлексия. Та объективность есть поэтому такое
непосредственное, которое само непосредственным образом непосредственно.
Напротив, здесь объективность есть лишь нечто происшедшее из понятия, так
что ее сущностью служит положенность и она выступает как нечто
{\em отрицательное}. "--- Ее следует рассматривать как {\em сторону
всеобщности понятия} и, стало бьть, как {\em абстрактную}
всеобщность, {\em присущую} по
существу лишь субъекту и имеющую форму непосредственного {\em бытия}, которое,
будучи положено само по себе, безразлично по отношению к субъекту.
Тотальность понятия, присущая объективности, постольку есть лишь как бы
{\em заимствованная тотальность;}
та последняя самостоятельность, которую она
имеет по отношению к субъекту, есть то {\em бытие}, которое
согласно своей истине оказывается только вышеназванным моментом понятия,
которое, как {\em пред-полагающее},
находится в стадии первой определенности некоторого
{\em в-себе}-сущего {\em полагания}, еще не выступающего {\em как}
полагание, как рефлектированное в~себя единство. Возникнув
из идеи, самостоятельная объективность есть, следовательно,
непосредственное бытие только как {\em предикат} суждения
самоопределения понятия, "--- есть бытие, хотя и разнящееся от
субъекта, но вместе с тем существенно положенное как {\em момент} понятия.

По содержанию эта объективность есть тотальность понятия,
которой, однако, противостоит его субъективность или отрицательное
единство, составляющее истинную центральность, а именно, его свободное
единство с самим собой. Этот {\em субъект} есть идея в
форме {\em единичности}, как простое, но отрицательное тождество
с собой, "--- {\em живой индивидуум}.

Последний есть, во-первых, жизнь как {\em душа}, как понятие
самого себя, которое совершенно определено внутри себя, как начинающий,
самодвижущий {\em принцип}.
Понятие содержит в своей простоте определенную внешность как
заключенный внутри понятия
{\em простой} момент. "---
Но, далее, эта душа {\em в
ее непосредственности} непосредственно внешня и обладает в
самой себе некоторым объективным бытием; это "--- подчиненная
цели реальность, непосредственное
{\em средство}, которое
ближайшим образом есть объективность как
{\em предикат} субъекта;
но эта объективность есть, далее, также и
{\em средний термин}
умозаключения; телесность души есть то, посредством чего она
смыкает себя с внешней объективностью. "--- Телесностью живое
существо обладает ближайшим образом как непосредственно тождественная с
понятием реальность; постольку оно вообще обладает этой телесностью от
{\em природы}.

И вот, так как эта объективность есть предикат индивидуума и
вобрана в субъективное единство, то ей не свойственны прежние определения
объекта, механическое или химическое отношение, и еще менее ей свойственны
абстрактные рефлективные отношения целого и частей и тому подобное. Как
внешность, она, правда, {\em способна}
к таким отношениям, но постольку она не есть живое наличное
бытие. Если живое существо берут как некоторое целое, состоящее из частей,
как нечто такое, на что воздействуют механические или химические причины,
берут как механический или химический продукт, будь последний определен
чисто как таковой или же также и через некоторую внешнюю
цель, то понятие становится внешним ему, живое берется как нечто
{\em мертвое}. Так как
понятие ему имманентно, то
{\em целесообразность}
живого нужно понимать как
{\em внутреннюю}
целесообразность; понятие имеет бытие в живом как
определенное понятие; отличное от своей внешности и в своем различении
пронизывающее ее и тождественное с собою. Эта объективность живого существа
есть {\em организм;} она есть {\em средство и орудие}
цели, совершенно целесообразна, так как понятие составляет ее
субстанцию; но именно потому само это средство и орудие есть выполненная
цель, в которой субъективная цель постольку непосредственно сомкнута с
самой собой. Со стороны своей внешности организм есть многообразие не
{\em частей}, а
{\em членов}, которые как
таковые (а) существуют только внутри индивидуальности; они отделимы,
поскольку они суть внешние и могут быть ухвачены за эту внешность, но
поскольку их отделяют, они возвращаются под власть механических и
химических отношений обыкновенной объективности. (b) Их внешность
противостоит отрицательному единству живой индивидуальности; последняя есть
поэтому {\em влечение}
положить абстрактный момент определенности понятия как
реальное различие; ввиду того что это различие
{\em непосредственно},
оно есть {\em влечение}
каждого {\em единичного},
{\em специфического момента}
продуцировать себя, а также возвести свою особенность во
всеобщность, снять другие, внешние ему, моменты, продуцировать себя за их
счет, но в равной мере снять самого себя и сделать себя средством для
других\pagenote{К~этому месту относится следующее замечание Энгельса из <<Диалектики
природы>>: <<Внутренняя цель в организме прокладывает себе затем согласно
Гегелю (V,~244) дорогу через посредство {\em влечения}.
Не слишком убедительно это. Влечение должно, по Гегелю, привести отдельное
живое существо более или менее в гармонию с его понятием. Отсюда ясно,
насколько вся эта {\em внутренняя цель} сама представляет собой
идеологическое определение. И~однако в этом суть Ламарка>>
({\em Engels}, Dialektik der Natur, M.--L. 1935, S.~655).
Даваемая Энгельсом ссылка <<V,~244>> означает т.~V немецкого собрания
сочинений Гегеля, 2-е изд. (Берлин 1841), стр.~244. Если Энгельс
пользовался 2-м изданием сочинений Гегеля, то Ленин писал свои <<Философские
тетради>> на основе 1-го издания. В~первом издании V тома (1834~г.) указанное
Энгельсом место находится на стр.~251---252.\label{bkm:bm94}}.

2. Этот {\em процесс} живой индивидуальности ограничивается ею самой и
имеет место еще всецело внутри нее. "--- Выше, говоря об умозаключении
внешней целесообразности, мы рассматривали его первую посылку (а~именно, то
обстоятельство, что цель непосредственно соотносится с объективностью и
делает ее средством) таким образом, что хотя в ней цель остается в этом
процессе равной себе и ушла обратно в~себя, но объективность
{\em в~самой себе} еще не сняла себя и поэтому цель в~ней постольку не есть
{\em в~себе и для себя},
а становится в-себе-и-для-себя-сущей только в заключении.
Процесс живого существа [смыкающий его] с самим собой есть указанная
посылка, но лишь постольку, поскольку она есть вместе с тем заключение и
поскольку непосредственное соотношение субъекта с объективностью (которая в
силу этого соотношения становится средством и орудием) выступает вместе с
тем как {\em отрицательное единство}
понятия в самом себе; цель осуществляет себя в этой своей
внешности в силу того, что она есть субъективная мощь последней и тот
процесс, в котором эта внешность показывает свое саморазложение
и возвращение в это отрицательное единство цели.
Беспокойство и изменчивость внешней стороны живого существа есть проявление
в нем понятия, которое, как отрицательность в самом себе, обладает
объективностью лишь постольку, поскольку ее безразличное устойчивое наличие
оказывается снимающим себя. Понятие, следовательно, продуцирует себя через
свое влечение таким образом, что продукт, поскольку понятие образует его
сущность, сам есть нечто продуцирующее, а именно, таким образом, что он
есть продукт только как такая внешность, которая полагает себя также и
отрицательно, или, иначе говоря, как процесс продуцирования.

3. Только что рассмотренная идея и есть {\em понятие живого субъекта}
и {\em его процесса;} определения, находящиеся здесь в отношении друг к другу,
суть соотносящееся с собой {\em отрицательное единство} понятия и
{\em объективность}, которая есть его {\em средство}, но в которой понятие
{\em возвратилось} в само себя. Однако так как это суть моменты идеи жизни
{\em внутри понятия последней}, то это не суть определенные моменты понятия
{\em живого индивидуума в его реальности}.
Объективность или телесность последнего есть
конкретная тотальность; те моменты суть стороны, из которых конституируется
жизненность; они поэтому не суть моменты этой уже конституированной через
идею жизненности. Но живая {\em объективность}
индивидуума как таковая, ввиду того что она одушевлена
понятием и имеет его своей субстанцией, содержит также в~себе в качестве
существенных различий такие различия, которые суть определения понятия, "---
{\em всеобщность, особенность} и {\em единичность;} тот {\em образ},
в котором они внешне различены, подразделен поэтому или
получает надрезы (insectum) сообразно
им\pagenote{Под <<образом>> (Gestalt) Гегель здесь понимает <<животный субъект как целое,
взятое {\em только в его соотношении с самим собой}>> (см. <<Философию
природы>>, \S~353). Помещая в скобках латинское слово <<insectum>>, которое
означает <<надрезанное>>, а затем <<насекомое>>, Гегель намекает на то, что
характерной особенностью насекомых
является разделение или рассечение их тела на голову, грудь и брюшко
(отсюда и самое название этого класса живых существ). В~<<Философии природы>>
Гегель, трактуя о трех основных функциях организма
"--- чувствительности, раздражимости и воспроизведении
(воспроизведение берется здесь у Гегеля в смысле постоянного
воспроизведения живым организмом всех тканей и соков, входящих в его
состав), "--- отмечает, что эти функции в рассеченном виде
представлены у насекомых, у которых, дескать, <<голова является центром
чувствительности, грудь "--- раздражимости, брюшко "--- воспроизведения>>
({\em Гегель}, Соч., т.~II, стр.~464).\label{bkm:bm95}}.

Живая объективность, стало быть, есть, {\em во-первых, всеобщность},
колыхание жизненности чисто лишь в самой себе, {\em чувствительность}.
Понятие всеобщности, как оно получилось у нас выше, есть
простая непосредственность, которая, однако, такова лишь как абсолютная
отрицательность внутри себя. Это понятие {\em абсолютного различия},
поскольку его отрицательность {\em растворена} в {\em простоте} и равна
самой себе, сделано в чувствительности наглядным. Чувствительность есть
внутри-себя-бытие не как абстрактная простота, а как бесконечная
{\em определимая} восприимчивость, которая в своей {\em определенности} не
становится чем-то многообразным и внешним, а безоговорочно рефлектирована в
себя. {\em Определенность} выступает в этой всеобщности как простой
{\em принцип;} единичная внешняя определенность, так называемое
{\em впечатление}, уходит из своего внешнего и многообразного определения
обратно в эту простоту {\em самочувствия}.
Чувствительность, стало быть, может рассматриваться как
наличное бытие внутри-себя-сущей души, так как она вбирает в~себя всяческую
внешность, но приводит последнюю обратно к совершенной простоте равной себе
всеобщности.

Вторым определением понятия служит
{\em особенность}, момент
{\em положенного}
различия. Это "--- освобождение той
отрицательности, которая заключена в простом самочувствии или, иначе
говоря, есть в нем идеализованная, еще не реальная определенность:
{\em раздражимость}.
Вследствие абстрактности своей отрицательности чувство есть
влечение; оно {\em определяет}
себя; самоопределение живого есть его суждение (перводеление)
или превращение его в конечное, сообразно чему оно соотносится с внешним,
как с некоторой {\em пред-положенной}
объективностью, и находится с нею во взаимодействии. "---
Со стороны своей особенности оно есть отчасти вид наряду с
другими видами живых существ;
{\em формальная}
рефлексия в~себя этой
{\em безразличной разности}
есть формальный {\em род}
и его систематизирование; индивидуальная же рефлексия состоит
в том, что особенность есть отрицательность своей определенности как
некоторой направленности во-вне, соотносящаяся с собой отрицательность
понятия.

Со стороны этого
{\em третьего}
определения живое существо выступает
{\em как единичное}.
Ближе эта рефлексия в~себя определяется так, что живое
существо есть в раздражимости своя же собственная внешность по отношению к
самому себе, по отношению к той объективности, которую оно имеет
непосредственно в самом себе как свое средство и орудие и которая поддается
внешнему определению. Рефлексия в~себя снимает эту непосредственность; она
снимает ее, с одной стороны, как теоретическая рефлексия, а именно,
поскольку отрицательность выступает как тот простой момент
чувствительности, который в ней был нами рассмотрен и который составляет
{\em чувство;} она, с
другой стороны, снимает ее как реальная рефлексия, поскольку единство
понятия полагает себя {\em в своей
внешней объективности} как отрицательное единство:
{\em воспроизведение}. "---
Два первых момента, чувствительность и раздражимость, суть
абстрактные определения; в воспроизведении же жизнь есть
{\em конкретное} и
жизненность; в нем, как в своей истине, жизнь впервые и обладает чувством и
силой сопротивления. Воспроизведение есть отрицательность как простой
момент чувствительности, и раздражимость есть живая сила сопротивления
только благодаря тому, что отношение к внешнему есть воспроизведение и
индивидуальное тождество с собой. Каждый из отдельных
моментов есть по существу тотальность всех; их различие составляет
идеализированная определенность формы, которая (определенность) в
воспроизведении положена как конкретная тотальность целого. Поэтому это
целое, с одной стороны, как нечто третье, а именно, как
{\em реальная}
тотальность, противоположно тем определенным тотальностям,
но, с другой стороны, оно есть их в-себе-сущая сущность и вместе с тем то,
в чем они объединены как моменты и в чем они имеют свой субъект и свое
устойчивое наличие.

Вместе с воспроизведением, как моментом единичности, живое
полагает себя как {\em действительную}
индивидуальность, как соотносящееся с собой для-себя-бытие;
но вместе с тем оно есть реальное
{\em соотношение, направленное во-вне},
рефлексия
{\em особенности} или
раздражимости {\em по отношению к
к чему-то иному}, по отношению к
{\em объективному} миру.
Замкнутый внутри индивидуума процесс жизни переходит в отношение к
пред-положенной объективности как таковой вследствие того, что
индивидуум, полагая себя как
{\em субъективную}
тотальность, становится также и
{\em моментом своей определенности}
как {\em соотношения}
с внешностью, "--- становится
{\em тотальностью}.

\section[В. Процесс жизни]{В. Процесс жизни}

Формируя себя внутри самого себя, живой индивидуум тем самым
вступает в напряженные отношения к своему первоначальному пред-положению и
противопоставляет себя как в-себе-и-для-себя-сущего субъекта
пред-положенному объективному миру. Субъект есть самоцель, понятие, имеющее
в подчиненной ему объективности свое средство и свою субъективную
реальность. В~силу этого он конституирован как в-себе-и-для-себя-сущая идея
и как существенно самостоятельное, по отношению к которому пред-положенный
внешний мир обладает лишь ценностью чего-то отрицательного и
несамостоятельного. В~своем самочувствии живое существо обладает этой
{\em уверенностью}
относительно в-себе-сущей
{\em ничтожности}
противостоящего ему
{\em инобытия}. Его
влечение есть потребность снять это инобытие и сообщить себе истину
указанной уверенности. Индивидуум как субъект есть пока что только
{\em понятие} идеи жизни;
его субъективный процесс внутри себя, в котором он живет самим собой, и
непосредственная объективность, которую как естественное средство он
полагает соответственной своему понятию, опосредствованы тем процессом,
который соотносится с полностью положенной внешностью, с
{\em безразлично} стоящей
рядом с ним объективной тотальностью.

Этот процесс начинается с
{\em потребности}, т.~е.
с того момента, что живое существо,
{\em во-первых},
определяет себя, полагает себя, стало быть, как подвергшееся
отрицанию и этим соотносится с некоторой другой по отношению к нему,
безразличной объективностью, но что оно,
{\em во-вторых}, вместе с
тем не потерялось в этой потере самого себя, сохраняет себя в ней и
остается тождеством равного самому себе понятия; в силу этого оно есть
влечение положить {\em для себя},
положить равным себе указанный
{\em другой} по отношению
к нему мир, снять этот мир и объективировать
{\em себя}. Вследствие
этого самоопределение живого имеет форму объективной внешности, а ввиду
того, что оно вместе с тем тождественно с собой, оно есть абсолютное
{\em противоречие}.
Непосредственное образование есть идея в ее простом понятии,
соответственная понятию объективность; таким образом, оно от природы
{\em хорошо}. Но так как
его отрицательный момент реализует себя в виде объективной особенности,
т.~е., так как каждый из существенных моментов его единства сам по себе
реализован в виде тотальности, то понятие
{\em раздвоено} на
абсолютное неравенство понятия самому себе, а так как понятие есть в равной
мере абсолютное тождество в этом раздвоении, то живое существо есть для
самого себя это раздвоение и обладает чувством этого противоречия, каковое
чувство есть {\em боль}.
{\em Боль} есть поэтому
привилегия живых натур; так как они суть существующее понятие, то они суть
действительность, обладающая такой бесконечной силой, что они внутри себя
суть {\em отрицательность}
самих себя, что эта их
{\em отрицательность}
имеет бытие {\em для них}
и что они сохраняют себя в своем инобытии. "---
Если говорят, что противоречие немыслимо, то нужно сказать,
наоборот, что в боли, испытываемой живыми существами, оно есть даже
некоторое действительное существование.

Это расщепление живого внутри себя есть
{\em чувство}, когда это
расщепление вобрано в простую всеобщность понятия, в чувствительность.
С~болью начинаются {\em потребность}
и {\em влечение},
составляющие переход к тому, чтобы индивидуум, выступая для
себя как отрицание самого себя, в такой же мере выступал для себя также и
как тождество с собой, "--- такое тождество, которое имеет
бытие только как отрицание того отрицания. "--- Тождество,
имеющееся во влечении как таковом, есть субъективная уверенность в самом
себе, согласно которой живое существо относится к своему внешнему,
безразлично существующему миру, как к явлению, как к некоторой, в~себе
чуждой понятию и несущественной действительности. Этот мир
должен получить в~себя понятие только через субъект, который
есть имманентная цель. Безразличие объективного мира к определенности и тем
самым и к цели составляет его внешнюю способность к тому, чтобы быть
соответственным субъекту; какие бы спецификации он ни содержал в~себе в
других отношениях, его механическая определяемость, недостаток свободы
имманентного понятия составляют его бессилие сохранить себя против
живущего. "--- Поскольку объект выступает по отношению к живому
прежде всего как некоторая безразличная внешность, он может механически
воздействовать на него, но тогда он действует на него не как на живое;
поскольку же он имеет отношение к последнему, он действует не как причина,
а лишь {\em возбуждает}
его. Так как живое есть влечение, то внешность доходит до
него и входит внутрь его лишь постольку, поскольку она уже сама по себе
есть в нем; поэтому воздействие на субъект состоит лишь в том, что
последний {\em находит соответственной}
представившуюся ему внешность; если она и не соответственна
его тотальности, она все же необходимо должна соответствовать по крайней
мере некоторой особенной стороне в нем, а эта возможность заложена в том,
что он, именно как относящийся внешним образом, есть некоторое особенное.

Итак, субъект, поскольку он определенно соотносится в своей
потребности с внешним и, стало быть, сам есть нечто внешнее или некоторое
орудие, "--- этот субъект совершает
{\em насилие} над
объектом. Его особенный характер и вообще его конечность обнаруживаются в
более определенном явлении этого отношения. "--- Внешнее в
последнем представляет собой процесс объективности вообще, механизм и
химизм. Но этот процесс непосредственно прерывается, и внешность
превращается во внутреннее. Внешняя целесообразность, порождаемая
деятельностью субъекта в безразличном объекте, снимается вследствие того,
что объект не есть субстанция по отношению к понятию и поэтому понятие не
только может стать ее внешней формой, но соответственно своему
первоначальному тождеству необходимо должно положить себя как его сущность
и имманентное, пронизывающее его определение.

Поэтому вместе с овладением объектом механический процесс
переходит во внутренний процесс, через который индивидуум так
{\em усваивает} себе
объект, что лишает его своеобразного характера, делает его своим средством
и дает ему в субстанцию свою субъективность. Эта ассимиляция, стало быть,
совпадает с вышерассмотренным процессом воспроизведения индивидуума;
последний черпает в этом процессе свое питание прежде всего из себя,
делая для себя объектом свою собственную объективность;
механический и химический конфликт его членов с внешними вещами есть
объективный момент его же самого. Механическая и химическая сторона этого
процесса есть начало разложения живого существа. Так как жизнь представляет
собой истину этих процессов и тем самым, как живое существо, она есть
существование этой истины и ее мощь, то она перехлестывает за их пределы,
пронизывает их как их всеобщность, и их продукт ею полностью определен. Это
их превращение в живую индивидуальность составляет возвращение этой
последней в самоё себя, так что продуцирование, которое как таковое было бы
переходом в нечто иное, становится репродуцированием, в котором живое
полагает себя {\em для себя}
тождественным с собой.

Непосредственная идея есть также и непосредственное, не как
{\em для себя} сущее
тождество понятия и реальности; через объективный процесс живое дает себе
свое {\em самочувствие;} ибо оно {\em полагает}
себя в нем как то, что оно есть в~себе и для себя, полагает
себя как то, что в своем (положенном как безразличное) инобытии оказывается
тождественным с самим собой, оказывается отрицательным единством
отрицательного. В~этом влиянии индивидуума со своей сперва пред-положенной
ему как нечто безразличное объективностью он, с одной стороны,
конституировал себя как действительную единичность, а, с другой стороны,
вместе с тем {\em снял свою
особенность} и возвел себя во всеобщность. Его особенность
состояла в расщеплении, в силу которого жизнь полагала как свои виды
индивидуальную жизнь и внешнюю ей объективность. Через внешний жизненный
процесс жизнь, стало быть, положила себя как реальную всеобщую жизнь, как
{\em род}.

\section[С. Род]{С. Род}

Живой индивидуум, выделившийся из всеобщего понятия жизни,
есть некоторое пред-положение, которое еще не подтверждено самим собой.
Через процесс воздействия на вместе с тем пред-положенный мир он сам себя
положил {\em для себя}
как отрицательное единство своего инобытия, как основу самого
себя; он, таким образом, есть действительность идеи, так что индивидуум
теперь порождает себя из
{\em действительности},
между тем как раньше он происходил лишь из
{\em понятия}, и его
возникновение, которое раньше было некоторым
{\em пред-полаганием},
становится теперь его произведением.

Дальнейшее же определение, которого он достиг через снятие
противоположности, заключается в том, что он есть
{\em род} как тождество
себя со своим прежним безразличным инобытием. Так как эта
идея индивидуума есть указанное существенное тождество, то она есть по
существу самообособление (die Besonderung ihrer selbst). Это
ее расщепление есть сообразно тотальности, из которой она происходит,
удвоение индивидуума, "--- пред-полагание такой объективности,
которая тождественна с ним, и отношение живого существа к себе самому как к
другому живому.

Это всеобщее есть третья ступень, истина жизни, поскольку
жизнь еще замкнута в своей сфере. Эта ступень есть соотносящийся с собой
процесс индивидуума, в каковом процессе внешность есть имманентный момент
индивидуума; {\em во-вторых},
эта внешность, как живая тотальность, сама есть такая
объективность, которая для индивидуума есть он же сам, "---
объективность, в которой "--- не как в снятой, а
как в {\em устойчиво наличной}
"--- он обладает достоверностью самого себя.

И вот, так как отношение рода есть тождество индивидуального
самочувствия в чем-то таком, что вместе с тем есть некоторый другой
самостоятельный индивидуум, то оно есть {\em противоречие;} живое
тем самым снова есть влечение. "--- Род есть, правда,
завершение идеи жизни, однако пока что он еще находится внутри сферы
непосредственности; эта всеобщность поэтому
{\em действительна} в {\em единичном} образе;
это "--- понятие, реальность которого имеет форму
непосредственной объективности. Поэтому, хотя индивидуум есть
{\em в~себе} род, но он не есть род {\em для себя;}
то, что есть для него, есть пока что лишь некоторый другой
живой индивидуум; отличенное от себя понятие имеет тем предметом, с которым
оно тождественно, не себя как понятие, а некоторое такое понятие, которое,
как живое существо, вместе с тем обладает для него внешней объективностью,
"--- форма, которая поэтому непосредственно взаимна.

Тождество с другим, всеобщность индивидуума есть, стало быть,
пока что лишь {\em внутреннее}
или {\em субъективное}
тождество; индивидуум имеет поэтому желание положить это
тождество и реализовать себя как всеобщее. Но это влечение к роду может
реализоваться лишь через снятие единичных индивидуальностей, являющихся еще
особенными по отношению друг к другу. Прежде всего, так как именно
последние, будучи {\em в~себе}
всеобщими, удовлетворяют свое напряженное желание и
разрешаются в свою родовую всеобщность, то их реализованное тождество есть
отрицательное единство рода, который из раздвоения рефлектируется в~себя.
Постольку он есть индивидуальность самой жизни, теперь уже
{\em порожденной} не из
своего понятия, а из
{\em действительной}
идеи. Ближайшим образом сам он есть лишь
понятие, которому только еще предстоит объективировать себя, но
{\em действительное понятие}, "---
{\em зародыш некоторого живого
индивидуума}. В~нем для
{\em обыденного восприятия
наличествует} то, что такое есть понятие, и обнаруживается,
что {\em субъективное понятие}
обладает {\em внешней
действительностью}. Ибо зародыш живого существа есть полная
конкретность индивидуальности, в которой все его разные стороны, свойства и
расчлененные различия содержатся {\em во
всей} их
{\em определенности} и
тотальность, вначале
{\em имматериальная} и
субъективная, выступает в неразвитом, простом и нечувственном виде; зародыш
есть, таким образом, все живое существо во внутренней форме понятия.

Рефлексия рода в~себя есть с этой стороны то, через что он
получает {\em действительность},
поскольку в нем
{\em полагается} момент
отрицательного единства и индивидуальности, "---
{\em размножение} живых
поколений. Идея, которая как жизнь еще имеет форму непосредственности,
постольку впадает обратно в действительность, и эта ее рефлексия есть лишь
повторение и бесконечный прогресс, в котором она не выбирается из
конечности своей непосредственности. Но это возвращение в ее первое понятие
имеет в~себе также и ту более высокую сторону, что идея не только прошла
через все опосредствование своих процессов в пределах непосредственности,
но именно этим и сняла свою непосредственность и вследствие этого поднялась
до более высокой формы своего существования.

А именно, процесс рода, в котором (процессе) единичные
индивидуумы снимают друг в друге свое безразличное, непосредственное
существование и умирают в этом отрицательном единстве, имеет, далее, другой
стороной своего продукта {\em peaлизованный род},
положивший себя тождественным с понятием. "--- В~процессе рода
отдельные единичности индивидуальной жизни гибнут; то
отрицательное тождество, в котором род возвращается в~себя, есть, с одной
стороны, {\em порождение единичности},
а с другой стороны, в такой же мере и {\em ее уничтожение},
есть, стало быть, сливающийся с собой род,
{\em для себя становящаяся всеобщность}
идеи. В~процессе рода умирает непосредственность живой
индивидуальности; смерть этой жизни есть возникновение духа. Идея, которая
как род существует (ist) {\em в~себе},
существует (ist) теперь {\em для себя}, поскольку
она сняла свою особенность, представлявшую собой живые поколения, и этим
сообщила себе такую {\em реальность}, которая {\em сама}
есть {\em простая всеобщность;} таким образом, она есть идея,
{\em относящаяся к себе} как к {\em идее},
всеобщее, имеющее всеобщность своей определенностью и
существованием (Dasein), "--- {\em идея познания}.

\chapter[Вторая глава Идея познания]{Вторая глава\newline Идея познания}

Жизнь есть непосредственная идея или, иначе говоря, идея как
ее, еще не реализованное в самом себе, {\em понятие}. В~своем
{\em суждении} идея есть {\em познание} вообще.

Понятие как понятие имеет бытие для себя, поскольку оно
существует {\em свободно}
как абстрактная всеобщность или как род. Таким образом, оно
есть свое чистое тождество с собой, которое различает себя внутри самого
себя так, что различенное не есть некоторая {\em объективность}, а
тоже освобождено к тому, чтобы быть субъективностью или формой простого
равенства с собой, и, стало быть, предмет понятия есть само понятие. Его
{\em реальностью} вообще служит {\em форма} {\em его наличного бытия;}
все дело в определении этой формы; на этом ее определении
покоится различие между тем, что понятие есть
{\em в~себе} или как {\em субъективное}, и
тем, что оно есть, погруженное в объективность, а затем в идее жизни.
В~последней оно, правда, отлично от своей внешней реальности и положено
{\em для себя;} однако
этим своим для-себя-бытием оно обладает лишь как тождество, которое есть
соотношение с собой, как с погруженным в свою подчиненную ему
объективность, или соотношение с собою, как с имманентной, субстанциальной
формой. Возвышение понятия над жизнью состоит в том, что его реальность
есть форма понятия, освобожденная к всеобщности. Через это суждение
(перводеление) идея удвоена или раздвоена на субъективное понятие,
реальностью которого служит оно же само, и на объективное понятие, которое
выступает как жизнь. "--- {\em Мышление,
дух, самосознание} суть определения идеи, поскольку она
имеет своим предметом самое себя и поскольку
{\em ее наличное бытие},
т.~е. определенность ее бытия, есть ее собственное отличие от
самой себя.

{\em Метафизика духа} или (как чаще говорили в прежнее время)
{\em души} вращалась
вокруг определений субстанции, простоты, имматериальности
"--- определений, говоря о которых она клала в основание в
качестве субъекта {\em представление}
о духе, почерпнутое из {\em эмпирического}
сознания, а затем задавалась вопросом, какие предикаты
согласуются с восприятиями, "--- способ рассуждения, который не
мог идти дальше, чем прием физики сводить мир явления к всеобщим законам и
рефлексивным определениям, так как и в этой метафизике дух лежал в
основании тоже лишь в своем {\em явлении;} она даже
должна была оставаться позади степени научности физики,
потому что дух не только бесконечно богаче, чем природа, но
так как к тому же абсолютное единство противоположностей в
{\em понятии} составляет
его сущность, то он в своем явлении и соотношении с внешностью обнаруживает
противоречие в его самой крайней определенности, и поэтому всегда должно
оказаться возможным привести какой-нибудь опыт в пользу каждого из
противоположных рефлексивных определений или, иначе говоря, исходя из
опытов, прийти согласно правилам формального умозаключения к
противоположным определениям. Так как предикаты, непосредственно
получающиеся при рассмотрении явлений, ближайшим образом еще принадлежат
области эмпирической психологии, то для метафизического рассмотрения
остаются, собственно говоря, лишь совершенно скудные определения
рефлексии. "--- {\em Кант} в
своей критике {\em рациональной
психологии} ловит на слове эту метафизику, настаивая на том,
что, поскольку она согласно своему утверждению есть рациональная наука, то
через самомалейшее содержание, которое мы
{\em прибавили бы} из
восприятия к {\em всеобщему
представлению} о самосознании, эта наука превратилась бы в
{\em эмпирическую} и тем
самым была бы испорчена ее рациональная чистота и независимость от всякого
опыта. "--- Таким образом, продолжает Кант, у нас ничего не
осталось бы, кроме простого, самого по себе совершенно бессодержательного
представления <<я>>, о каковом представлении нельзя даже сказать, что оно
есть {\em понятие}: оно есть лишь {\em голое сознание},
{\em сопровождающее все понятия}. Этим мыслящим <<я>> или даже мыслящим
<<{\em им}>> {\em (вещью)} мы, согласно
дальнейшим выводам Канта, ничего больше не представляем себе, кроме
некоторого трансцендентального субъекта мыслей~=~Х, который познается лишь
через те мысли, которые суть его {\em предикаты}, и о
котором, взятом отдельно от его мыслей, мы {\em никогда} не можем
иметь ни {\em малейшего понятия;}
притом указанное <<я>>, согласно собственному выражению Канта,
страдает тем {\em неудобством}, что {\em мы} всегда {\em уже должны
пользоваться им}, чтобы иметь о нем какое-либо суждение, ибо
оно есть не столько {\em некоторое
представление}, посредством которого мы различали бы
некоторый особенный объект, сколько {\em форма} представления
вообще, поскольку последнее должно быть названо познанием. "--- И вот
{\em паралогизм}, который, дескать, совершает рациональная психология,
состоит, согласно Канту, в том, что {\em модусы}
самосознания в мышлении превращаются нами в {\em понятия рассудка}
будто бы о некотором {\em объекте}, в том, что
указанное <<я мыслю>> берется как некоторое {\em мыслящее существо},
как некоторая {\em вещь-в-себе;} таким
образом, из того обстоятельства, что <<я>> всегда встречается в сознании
как {\em субъект}, и притом как {\em единичный}, при всем
многообразии представления {\em тождественный} и
отличающий меня от этого многообразия как внешнего, "---
делается неправомерный вывод, что <<я>> есть {\em субстанция} и,
далее, нечто качественное {\em простое} некоторое {\em одно}
и нечто {\em существующее независимо} от пространственных и временных
вещей\pagenote{Гегель здесь передает (большею частью словами самого Канта)
содержание кантовской критики <<психологического паралогизма>>, занимающей
страницы 400---411 второго немецкого издания <<Критики чистого разума>>, причем
больше всего выписок Гегель делает со страницы 404. Курсив везде принадлежит
Гегелю. В~<<Диалектике природы>> Энгельс делает следующее замечание по поводу
этого места <<Логики>> Гегеля: <<Ценная самокритика кантовской
{\em вещи-в-себе}, показывающая, что Кант терпит крушение также и по
вопросу о мыслящем~Я, в котором он тоже находит некоторую непознаваемую
вещь-в-себе. Гегель, V, 256 и~сл.>> ({\em Engels}, Dialektik
der Natur, M.--L. 1935, S.~655). Даваемая Энгельсом ссылка
(<<V,~256 и~сл.>>) указывает страницы второго немецкого издания
V~тома Собрания сочинений Гегеля. В~первом издании этого тома
указанное место находится на стр.~264---268.\label{bkm:bm96}}.~"---

Я дал в извлечении более подробное изложение этого рассуждения
Канта потому, что из него можно определенно усмотреть как природу прежней
{\em метафизики души},
так и, в особенности, природу той
{\em критики}, от которой
она погибла. "--- Упомянутая метафизика ставила своей целью
определить {\em абстрактную сущность}
души; она при этом первоначально исходила из восприятия и
превращала его эмпирическую всеобщность и определение рефлексии (вообще
{\em внешнее} для
имеющегося в действительности единичного) в форму указанных
{\em определений сущности}. "---
Кант при этом имеет вообще в виду лишь состояние метафизики
его времени, которая преимущественно не шла дальше таких лишенных всякой
диалектики абстрактных, односторонних определений; истинно же
{\em спекулятивных} идей
более старых философов о понятии духа он не принял во внимание и не подверг
исследованию. В~своей {\em критике}
указанных определений он просто-напросто следовал юмовской
скептической манере; а именно, он твердо придерживается того, каким <<я>>
является в самосознании, но, полагает он, так как мы должны познать его
{\em сущность}, "---
{\em вещь-в- себе}, "--- то
отсюда следует отбросить все эмпирическое; после этого ничего дескать не
остается, кроме этого явления
<<{\em я мыслю}>>,
сопровождающего все представления, того <<я мыслю>>, о котором
мы не имеем {\em ни малейшего понятия}.
"--- Несомненно следует согласиться с тем, что ни о <<я>>, ни о
чем бы то ни было, ни даже о самом понятии мы не имеем ни малейшего
понятия, поскольку мы не {\em постигаем
в понятии}, а останавливаемся только на простом, неподвижном
{\em представлении} и
{\em названии}. "--- Странна
мысль о том, "--- если ее вообще можно назвать мыслью, "---
что для того, чтобы судить о <<я>>, я уже необходимо должен
{\em пользоваться} этим
<<я>>. <<Я>>, которое {\em пользуется}
самосознанием как некоторым средством для того, чтобы судить,
есть, несомненно, некоторый X, о котором, равно как и об отношении такого
пользования, мы не можем иметь ни малейшего понятия. Но ведь смешно
называть {\em неудобством}
и порочным
{\em кругом}\pagenote{Выражения <<неудобство>> (Unbequem\-lich\-keit)
и <<круг>> (Zirkel) в применении к
самосознанию употребляются Кантом на стр.~404 второго издания <<Критики
чистого разума>>.\label{bkm:bm97}}
природу самосознания, заключающуюся в том, что <<я>> мыслит
само себя, что <<я>> не может быть мыслимо без того, чтобы мыслящим было
<<я>>, "--- смешно назвать неудобством то обстоятельство,
благодаря которому в непосредственном эмпирическом
самосознании нам открывается абсолютная, вечная природа самосознания и
понятия, открывается потому, что самосознание именно и есть
{\em существующее} (и,
следовательно, {\em могущее быть
эмпирически воспринимаемым}) чистое
{\em понятие}, абсолютное
соотношение с самим собой, которое, как разделяющее суждение, делает себя
своим предметом и исключительно состоит в том, чтобы этим сделать себя
кругом в рассуждении. "--- Камень не страдает таким
{\em неудобством;} когда
он должен стать предметом мысли или суждения, то он при этом не становится
на пути самому себе; он освобожден от этого обременения, ему не приходится
пользоваться для этого дела самим собой; имеется нечто иное вне его,
которое должно взять на себя этот труд.

Недостаток, который эти представления, достойные быть
названными варварскими, находят в том, что при процессе мышления о нашем
<<я>> последнее не может быть опущено как
{\em субъект}, выступает
затем также в том обратном виде, что, дескать, <<я>> встречается нам
{\em лишь} как
{\em субъект сознания}
или, в иной формулировке, что <<я>> может
{\em употреблять} себя
только в качестве {\em субъекта}
суждения, и недостает
{\em созерцания}, через
которое <<я>> было бы {\em дано}
как некоторый
{\em объект}, понятие же
такой вещи, которая может существовать лишь как субъект, еще вовсе не
приводит за собой объективной
реальности\pagenote{Выражения <<Я как субъект сознания>>, <<Я может употреблять себя только в
качестве субъекта суждения>>, <<созерцание, через которое Я было бы дано
как объект>>, "--- взяты со страницы 411 второго издания <<Критики чистого разума>>.}\label{bkm:bm98}.
"--- Если для признания объективности требуют внешнего,
определенного во времени и пространстве, созерцания и указывают, что
этого-то созерцания здесь нет, то ясно, что под объективностью требующие
этого разумеют лишь ту чувственную реальность, возвышение над которой
является условием мышления и истины. Но само собой разумеется, что если <<я>>
чуждым понятию образом берется как лишь простое представление, как мы
высказываем <<я>> в обыденном сознании, то оно есть абстрактное определение,
а не имеющее своим предметом само себя соотношение самого
себя; "--- взятое таким образом, оно есть лишь
{\em один} из крайних
терминов, односторонний субъект без его объективности, или оно было бы
также лишь объектом без субъективности, если бы при этом не имелось
упомянутого <<неудобства>>, что от <<я>> как объекта нельзя отмыслить мыслящего
субъекта. Но на самом деле то же самое <<неудобство>> имеет место также и
относительно первого определения, относительно <<я>> как субъекта: <<я>> мыслит
{\em нечто}, "--- себя или
нечто другое. Эта неотделимость тех двух форм:, в которых оно
противополагает само себя, принадлежит к наисобственнейшей природе его
понятия и самого понятия как такового; она есть как раз то, чего Кант хочет
не допустить, лишь бы только прочно сохранить не различающее
себя внутри самого себя и, стало быть, на самом деле только
{\em чуждое понятию представление}.
Такое чуждое понятию представление может, разумеется,
противопоставить себя абстрактным определениям рефлексии или категориям
прежней метафизики, ибо по односторонности оно стоит на одной доске с ними,
хотя последние все же представляют собой нечто более высокое в области
мысли; напротив, оно тем более оказывается скудным и бессодержательным по
сравнению с более глубокими идеями старых философов о понятии души или
мышления, например, с истинно спекулятивными идеями Аристотеля. Если
кантовская философия подвергла исследованию указанные определения
рефлексии, то она тем более должна была подвергнуть исследованию
фиксированную ею абстракцию пустого <<я>>, мнимую идею вещи-в-себе, которая
(вещь-в-себе) именно вследствие своей абстрактности оказывается, наоборот,
чем-то совершенно неистинным; испытывание упомянутого <<неудобства>>,
явившегося у Канта предметом жалоб, само есть тот эмпирический факт, в
котором находит свое выражение неистинность указанной абстракции.

Кантовская критика рациональной психологии упоминает лишь о
мендельсоновском доказательстве пребывающего характера души, и я привожу
даваемое указанной критикой опровержение этого доказательства еще и ввиду
примечательности того, чт\'{о} этому доказательству
противопоставляется\pagenote{<<Опровержение мендельсоновского
доказательства пребывающего характера души>> (т.~е. ее неуничтожимости или
бессмертия) занимает страницы 413---415 второго издания <<Критики чистого
разума>>.\label{bkm:bm99}}.
Мендельсоновское доказательство основано на {\em простоте} души, в
силу которой последняя, дескать, неспособна к изменению, к
{\em переходу} во времени {\em в} {\em нечто иное}.
Качественная простота есть рассмотренная выше форма
{\em абстракции} вообще; как {\em качественная}
определенность она была нами исследована при рассмотрении
сферы бытия, и там было доказано, что качественное, как такая абстрактно
соотносящаяся с собой определенность, именно потому, наоборот, диалектично
и есть лишь процесс перехода в некоторое другое. А~при рассмотрении понятия
было показано, что если оно рассматривается в отношении к пребываемости,
неразрушимости и непреходимости, оно, наоборот, есть
в-себе-и-для-себя-сущее и вечное, так как оно есть не
{\em абстрактная}, а {\em конкретная}
простота, не абстрактно соотносящаяся с собой определенность,
а единство {\em себя самого} и {\em своего иного},
в которое оно, следовательно, не может перейти так,
как будто оно изменилось в нем, "--- не может перейти таким
образом именно потому, что {\em иное},
определенность, есть оно же само, и оно поэтому в этом
переходе лишь приходит к самому себе. "--- Кантовская критика
противополагает указанному {\em качественному}
определению единства понятия {\em количественное}
определение. Хотя душа, говорит Кант, не есть многообразная
внеположность и не содержит в~себе {\em экстенсивной}
величины, все же сознание имеет {\em некоторую степень},
и душа, подобно {\em всему существующему}, имеет некоторую
{\em интенсивную величину;} а этим дана возможность перехода в ничто путем
{\em постепенного исчезания}. "---
Что же такое представляет собой это опровержение, как не
применение категорий {\em бытия, интенсивной величины} к духу? А это
"--- применение такого определения, которое не имеет истинности
в~себе и в понятии, наоборот, снято.

Метафизика "--- даже та, которая ограничивалась
неподвижными понятиями рассудка и не поднималась к спекулятивному и к
природе понятия и идеи, "--- имела своей целью {\em познание истины} и
исследовала свои предметы со стороны того, суть ли они нечто
{\em истинное} или нет,
субстанции они или феномены. Победа же, одержанная над нею кантовской
критикой, состоит, наоборот, в том, чтобы устранить исследование, имеющее
своей целью познание {\em истины}
и даже саму эту цель; она вовсе и не ставит единственно
интересного вопроса о том, имеет ли определенный субъект (здесь
"--- {\em абстрактное <<я>> представления})
истинность в~себе и для себя. Но не идти дальше явления и
того, что в обыденном сознании получается для простого представления,
значит отказываться от понятия и от философии. Все, что выходит за эти
пределы, получает в кантовской критике название чего-то такого, что
залетает слишком высоко и на что разум не имеет никаких прав.. И~в самом
деле, понятие залетает выше того, что чуждо понятию, и ближайшим
оправданием указанного выхода за пределы этого чуждого понятию служит,
во-первых, само же понятие, а, во-вторых, с отрицательной
стороны, неистинность явления и представления, равно как и таких
абстракций, как вещи-в-себе и то <<я>>, которое якобы не есть для себя
объект.

В контексте нашего логического изложения именно из
{\em идеи жизни}
произошла идея духа или, что то же самое, истиной идеи жизни
оказалась идея духа. Как таковой результат эта идея имеет в~себе и для себя
самой свою истинность, с которой, если угодно, можно затем сравнить также и
эмпирическое содержание или явление духа, чтобы выяснить, насколько оно
согласуется с нею; однако само эмпирическое может быть постигнуто в свою
очередь только через идею и из нее. Относительно {\em жизни} мы видели,
что она есть идея, но вместе с тем оказалось, что она еще не есть истинное
изображение или истинный вид и способ существования идеи. Ибо в жизни
реальность идеи выступает как {\em единичность;} {\em всеобщность} или род
есть нечто {\em внутреннее}.
Истина жизни как абсолютное отрицательное единство состоит
поэтому в том, что она снимает абстрактную или, что то же самое,
непосредственную единичность и как {\em тождественное}
тождественна с собой, как род равна самой себе. Эта идея и
есть {\em дух}. "--- Но
относительно этого можно сделать еще то замечание, что дух рассматривается
здесь в той форме, которая присуща этой идее как логической. Ведь дело в
том, что идея эта имеет еще и другие образы (их можно здесь указать
мимоходом), в которых она должна рассматриваться в конкретных науках о
духе, а именно, как {\em душа, сознание и дух как таковой}.

Название <<{\em душа}>>
употреблялось обыкновенно для обозначения
вообще единичного, конечного духа, и рациональное или эмпирическое
{\em учение о душе}
должно было означать то же самое, что
<<{\em учение о духе}>>.
При употреблении выражения
<<{\em душа}>> нам
предносится представление, что она есть некоторая
{\em вещь}, подобно
другим вещам; ставится вопрос о ее
{\em местопребывании}, о
том {\em пространственном}
определении, из которого действуют ее
{\em силы;} в еще большей
мере ставится вопрос о том, каким образом эта вещь оказывается
{\em непреходящей}, каким
образом она, будучи подчинена условиям
{\em временности}, все же
свободна от изменения в последней.
{\em Монадологическая}
система возводит материю в нечто похожее на душу; душа есть
по этому представлению такой же атом, как и атомы материи вообще; атом,
поднимающийся вверх, как пар из чашки кофе, способен, дескать, благодаря
счастливому стечению обстоятельств развиться в душу, и лишь
{\em б\'{о}льшая} степень
смутности его представлений отличает его от такой вещи, которая проявляется
как душа. "--- {\em Для-самого-себя-сущее
понятие} необходимо выступает также и в
{\em непосредственном существовании;}
в этом субстанциальном тождестве с жизнью, в своей
погруженности в свою внешность понятие должно рассматриваться в
{\em антропологии}. Но и
последней должна оставаться чуждой та метафизика, в которой эта форма
{\em непосредственности}
становится некоторой
{\em душой-вещью},
некоторым {\em атомом},
похожим на атомы материи. "--- Антропологии
должна быть оставлена лишь та темная область, в которой дух подчинен, как
это называли в прежнее время,
{\em звездным} и
{\em земным} влияниям,
живет как природный дух в
{\em симпатическом}
соприкосновении с природой и узнает о ее изменениях в
{\em сновидениях} и
{\em предчувствиях},
обитает в мозгу, сердце, ганглиях, печени и~т.~д., каковой
последней, согласно Платону, бог (дабы и
{\em неразумная} часть не
была забыта божественной благостью и была причастна высшему) дал дар
{\em предсказывания},
дар, выше которого стоит самосознательный человек. К~этой
неразумной стороне принадлежит, далее, отношение
представления и более высокой духовной деятельности, поскольку последняя
подчинена в отдельном субъекте игре совершенно случайной телесной
конституции, внешних влияний и отдельных обстоятельств.

Этот низший из всех конкретных образов, в котором дух погружен
в материальность, имеет своим непосредственно высшим образом
{\em сознание}. В~этой
форме свободное понятие как
{\em для-себя-сущее <<я>>}
вышло обратно из объективности, но так, что соотносится с
нею, как со {\em своим иным},
как с противостоящим предметом. Так как дух здесь уже больше
не выступает как душа, а ввиду его
{\em самодостоверности
непосредственность бытия} скорее обладает для него значением
{\em чего-то отрицательного},
то тождество с самим собой, в котором он имеет бытие в
предметном, есть вместе с тем еще лишь некоторое
{\em свечение}, поскольку
предметное имеет еще также и форму чего-то
{\em в-себе-сущего}. Эта
ступень есть предмет {\em феноменологии
духа} "--- науки, стоящей посредине между наукой о природном
духе и наукой о духе как таковом и рассматривающей
{\em для-себя}-сущий
дух вместе с тем в его {\em соотношении
со своим иным}, которое вследствие этого определено, как
было указано, и как
{\em в-себе}-сущий
объект, и также как подвергшееся отрицанию, "---
рассматривающей, следовательно, дух как
{\em являющийся},
изображающий себя в противоположности самого себя.

Более высокую истину этой формы представляет собой
{\em дух для себя}, для
которого предмет, носящий для сознания характер
{\em в-себе-}сущего
предмета, получает форму его (духа) собственного определения, т.~е. форму
{\em представления}
вообще; этот дух, который действует на определения, как на
свои собственные определения, "--- на чувства, представления и
мысли, "--- постольку пребывает внутри себя и бесконечен в
своей форме. Рассмотрение этой ступени есть дело
{\em учения о духе} в
собственном смысле, которое обнимало бы собой то, что служит предметом
обычной {\em эмпирической психологии},
но которое, чтобы быть наукой о духе, не должно подходить
эмпирически к трактовке своего предмета, а должно быть понято научно. "---
Дух есть на этой ступени
{\em конечный} дух,
поскольку {\em содержание}
его определенности есть некоторое непосредственное данное
содержание; наука об этом духе имеет своей задачей изобразить путь его
движения, в котором он освобождает себя от этой своей определенности и
шествует вперед к уразумению своей истины, "--- бесконечного
духа.

Напротив, {\em идея духа},
составляющая
{\em логический} предмет,
находится уже внутри чистой науки; эта наука имеет поэтому своей задачей не
обозрение того его пути, на котором он переплетается с природой, с
непосредственной определенностью и с материей или
представлением, "--- это рассматривается в вышеуказанных трех
науках; она же имеет этот путь уже позади себя или, что одно и то же,
скорее перед собой, "--- позади себя, поскольку логика берется
как {\em последняя}
наука, перед собой, поскольку она берется как
{\em первая} наука, из
которой идея лишь впервые переходит в природу. Поэтому в логической идее
духа <<я>> сразу же таково, каковым оно явило себя из понятия природы как ее
истина, сразу же есть свободное понятие, которое в своем суждении
(перводелении) есть для самого себя предмет, "--- сразу же есть
{\em понятие как его идея}.
Но и в этом образе идея еще не завершена.

Хотя она есть здесь свободное понятие, имеющее себя само своим
предметом, однако именно потому, что она непосредственна, она
{\em непосредственным образом}
есть еще идея в ее
{\em субъективности} и,
стало быть, в ее конечности вообще. Она есть
{\em цель}, которая
должна себя реализовать, или, иначе говоря, это "--- сама
{\em абсолютная идея}
пока еще в стадии своего
{\em явления}. Предметом
ее {\em исканий} является
{\em истина}, это
тождество самого понятия и реальности, но она пока что только ищет ее; ибо
она здесь такова, какова она {\em в
самом начале}, т.~е. она представляет собой еще нечто
{\em субъективное}.
Поэтому предмет, имеющий бытие для понятия, есть, правда,
здесь также некоторый данный предмет, но он не вступает в субъект как
воздействующий объект или как предмет, каков он как таковой сам по себе,
или как представление, а субъект превращает его в
{\em некоторое определение понятия;}
именно понятие и действует в предмете, соотносится в нем с собой
и, сообщая себе в объекте свою реальность, находит этим путем {\em истину}.

Идея, следовательно, есть ближайшим образом один крайний
термин некоторого умозаключения как понятие, которое как цель имеет
ближайшим образом само себя своей субъективной реальностью; другим крайним
термином служит предел субъективного, объективный мир. Эти два крайних
термина тождественны в том отношении, что они суть идея; во-первых, их
единство есть единство понятия, которое (понятие) есть в одном из них лишь
{\em для себя}, а в другом "--- лишь {\em в~себе;}
во-вторых, реальность в одном термине абстрактна, а в другом
выступает в своей конкретной внешности. "--- Это
единство теперь {\em полагается}
через познание; так как именно субъективная идея является
здесь тем, что как цель исходит от себя, то это единство имеет бытие
ближайшим образом лишь как {\em средний
термин}. "--- Познающее, правда, соотносится через
определенность своего понятия, а именно, через абстрактное
для-себя-бытие, с некоторым внешним миром, но соотносится с
ним в абсолютной достоверности самого себя, чтобы возвести свою реальность
в~себе самом, эту формальную истину, в реальную истину. Оно обладает в лице
своего понятия {\em всей сущностью}
объективного мира; его процесс состоит в том, что оно
полагает для себя конкретное содержание этого мира тождественным с
{\em понятием} и,
наоборот, понятие тождественным с объективностью.

Непосредственно идея явления есть
{\em теоретическая} идея,
{\em познание} как
таковое. Ибо непосредственно объективный мир имеет форму
{\em непосредственности}
или {\em бытия}
для сущего для себя понятия, равно как это последнее есть для
себя сначала лишь абстрактное, еще заключенное внутри себя понятие самого
себя; оно поэтому выступает лишь как {\em форма;} его
реальность, которой оно обладает в самом себе, составляют лишь его простые
определения {\em всеобщности} и {\em особенности;} единичность же или
{\em определенную определенность}, содержание эта форма получает извне.

\section[А. Идея истины]{А. Идея истины}

Субъективная идея есть ближайшим образом {\em влечение}. Ибо она
есть противоречие понятия, заключающееся в том, что последнее имеет себя
{\em предметом} и есть
для себя реальность, однако без того, чтобы предмет выступал как {\em иное},
самостоятельное по отношению к нему, или, иначе говоря, без
того, чтобы отличие самого себя от себя обладало вместе с тем существенным
определением {\em разности} и безразличного существования. Влечение имеет
поэтому определенность, состоящую в том, что оно снимает свою собственную
субъективность, превращает свою пока что абстрактную реальность
в конкретную и наполняет ее {\em содержанием} мира,
который предположен его (влечения) субъективностью. "--- С другой
стороны, оно определяется в силу этого следующим образом: понятие
есть, правда, абсолютная достоверность самого себя, но его
{\em для-себя-бытию} противостоит его пред-положение некоторого
{\em в-себе}-сущего мира, безразличное {\em инобытие}
которого, однако, имеет для самодостоверности понятия
значение лишь чего-то {\em несущественного;}
понятие есть постольку влечение снять это инобытие и
созерцать в объекте тождество с самим собой. Поскольку эта рефлексия в~себя
есть снятая противоположность и {\em положенная}, добытая
для субъекта {\em единичность}, которая первоначально выступает как
пред-положенное {\em в-себе-бытие}, мы
имеем здесь восстановленное из противоположности тождество формы с самой
собой, "--- тождество, которое тем самым
определено как безразличное к форме в ее различенности и
есть {\em содержание}.

Это влечение есть поэтому влечение к {\em истине}, поскольку
она имеет бытие в {\em познании}, следовательно, к {\em истине} как к
{\em теоретической} идее в ее собственном смысле. "--- Если
{\em объективная} истина
есть сама идея как соответствующая понятию реальность и постольку предмет
может иметь или не иметь в~себе истину, то, напротив, более определенный
смысл истины заключается в том, что она есть истина
{\em для} субъективного понятия или {\em в} нем, {\em в знании}. Она есть
отношение {\em суждения понятия},
которое оказалось формальным суждением истины; ведь в этом
суждении предикат есть не только объективность понятия, но и соотносящее
сравнение понятия вещи и ее действительности. "---
{\em Теоретична} эта реализация понятия постольку, поскольку оно как
{\em форма} есть еще определение некоторого {\em субъективного}, или,
иначе сказать, имеет то определение для субъекта, что оно есть его
определение. Так как познание есть идея как цель или как субъективная идея,
то отрицание мира, пред-положенного как {\em в-себе-сущий}, есть
{\em первое} отрицание;
заключение, в котором объективное положено в субъективное, поэтому тоже
имеет ближайшим образом лишь то значение, что в-себе-сущее выступает лишь
как нечто субъективное или, иными словами, что оно лишь
{\em положено} в
определении понятия, но в силу этого еще не таково в~себе и для себя.
Заключение приходит постольку лишь к некоторому
{\em нейтральному} единству или к некоторому {\em синтезу}, т.~е. к
единству таких, которые первоначально разделены, связаны лишь внешним
образом. "--- Поэтому, когда в этом познании понятие полагает
объект как {\em свой}
объект, то идея дает себе ближайшим образом лишь такое
содержание, основа которого {\em дана}
и в котором была снята лишь форма внешности. Постольку это
познание еще сохраняет в своей осуществленной цели свою
{\em конечность}, оно в этой выполненной цели вместе с тем
{\em не}~достигло ее и {\em в своей истине} еще
{\em не}~пришло к {\em истине}. Ибо
поскольку в результате содержание еще имеет определение чего-то
{\em данного}, постольку
пред-положенное {\em в-себе-бытие},
противостоящее понятию, еще не снято; единство понятия и
реальности, истина, здесь, стало быть, вместе с тем и не содержится. "---
Странным образом в новейшее время эта сторона {\em конечности} была
закреплена в философии и была признана {\em абсолютным} отношением
познания\pagenote{Имеется в виду
<<критическая>> философия Канта.\label{bkm:bm100}};
как будто конечное как таковое и должно было быть абсолютным!
Эта точка зрения приписывает объекту {\em позади} познания
характер некоторой неизвестной {\em вещи-в-себе}, и
последняя, а, стало быть, также и истина, рассматривается
как нечто абсолютно {\em потустороннее}
для познания. Определения мысли вообще, категории,
определения рефлексии, равно как формальное понятие и его моменты получают
в этом понимании положение не таких определений, которые конечны сами по
себе, а конечных в том смысле, что они по сравнению с упомянутой пустой
{\em вещью-в-себе} суть
нечто субъективное; принятие этого неистинного отношения познания за
истинное есть заблуждение, сделавшееся всеобщим мнением новейшего времени.

Из этого определения конечного познания непосредственно
явствует, что познание это есть противоречие, уничтожающее само себя;
"--- противоречие, заключающееся в том, что это есть истина,
которая вместе с тем не есть истина, "--- в том, что это есть
познание того, что {\em есть},
которое вместе с тем не познает вещи-в-себе. В~рушении этого
противоречия рушится, т.~е. оказывается неистинным, и его содержание, "---
субъективное познание и вещь-в-себе. Но познание должно ходом
своего собственного движения разрешить свою конечность и тем самым свое
противоречие; вышеуказанное соображение, которое мы выдвигаем по поводу
него, есть внешняя рефлексия; на самом же деле познание само есть понятие,
которое есть для себя цель и, следовательно, через свою реализацию
выполняет само себя и именно в этом выполнении снимает свою субъективность,
а также и пред-положенное
{\em в-себе-бытие}.
"--- Мы должны поэтому рассмотреть это познание в нем самом, в
его положительной деятельности. Так как эта идея, как было показано, есть
влечение понятия реализовать себя
{\em для себя самого}, то
его деятельность состоит в том, чтобы определить объект и этим процессом
определения тождественно соотноситься в нем с собой. Объект есть вообще то,
что безоговорочно поддается определению, и в идее он имеет ту существенную
черту, что он в~себе и для себя не противоположен понятию. Так как это
познание еще есть конечное, а не спекулятивное познание, то пред-положенная
объективность еще не имеет для него того образа, что она безоговорочно есть
в~себе самой лишь понятие и не содержит в~себе по отношению к нему ничего
самостоятельного. Но тем самым, что она считается в-себе-сущим
потусторонним, она по существу обладает определением
{\em определимости}
{\em через понятие},
потому что {\em идея}
есть для-себя-сущее понятие и всецело бесконечное внутри
себя, в котором объект снят {\em в
себе}, и цель заключается лишь в том, чтобы снять его
{\em для себя;} поэтому
объект, правда, предполагается идеей познания как
{\em в-себе-сущий}, но по
существу он пред-положен так в том отношении, что она,
уверенная в самой себе и в ничтожности этой
противоположности, приходит к тому, чтобы реализовать в нем свое понятие.

В том умозаключении, через которое субъективная идея смыкается
теперь с объективностью, {\em первая
посылка} есть та же самая форма непосредственного
захватывания объекта понятием и соотношения последнего с первым, какую мы
видели в целевом соотношении. Определяющее воздействие понятия на объект
представляет собой непосредственное
{\em сообщение} себя и не
встречающее сопротивления
{\em распространение}
себя на объект. В~этой своей деятельности понятие пребывает в
чистом тождестве с самим собой; но эта его непосредственная рефлексия в
себя имеет равным образом и определение объективной непосредственности; то,
что {\em для него} есть
его собственное определение, есть в равной мере и некоторое
{\em бытие}, ибо это есть
{\em первое} отрицание
пред-положения. Положенное определение считается поэтому вместе
с тем лишь некоторым {\em найденным}
пред-положением,
{\em уловлением} чего-то
{\em данного}, в котором
деятельность понятия состоит скорее лишь в том, чтобы быть отрицательным по
отношению к самому себе, держаться по отношению к наличному сдержанно и
пассивно, дабы последнее могло себя
{\em показать} не как
определенное субъектом, а таким, каково оно в самом себе.

Указанное познание выступает поэтому в этой посылке даже не
как некоторое {\em применение}
логических определений, а как некоторое получение и уловление
их как преднайденных, и его деятельность выступает как ограничивающаяся
лишь удалением от предмета некоторого субъективного препятствия, некоторой
внешней скорлупы. Это познание есть
{\em аналитическое}
познание.

\subsection[а) Аналитическое познание]{а) Аналитическое познание}

Мы встречаем иной раз следующую формулировку различия между
аналитическим и синтетическим познанием: первое-де движется от известного к
неизвестному, а второе "--- от неизвестного к известному. Но
если мы ближе всмотримся в это различение, нам будет трудно открыть в нем
определенную мысль, а тем паче понятие. Можно сказать, что познание
начинается вообще с чего-то такого, что неизвестно, ибо с тем, что нам уже
знакомо, нечего знакомиться. Но верно и обратное: познание начинает с
известного; это "--- тавтологическое предложение: то, с чего
оно начинает, то, следовательно, что оно действительно познает, есть именно
благодаря этому нечто известное; то, что еще не познано и должно быть
познана лишь впоследствии, есть еще нечто неизвестное. Постольку мы
должны сказать, что познание, если только оно уже началось,
всегда движется от известного к неизвестному.

Отличительный признак аналитического познания уже определился
как состоящий в том, что в него, как в первую посылку всего умозаключения,
еще не входит опосредствование, но что оно есть непосредственное, еще не
содержащее в~себе инобытия сообщение понятия, в котором деятельность
отчуждается от своей отрицательности. Однако указанная непосредственность
соотношения сама представляет собой опосредствование, ибо она есть
отрицательное соотношение понятия с объектом, но такое соотношение, которое
уничтожает само себя и этим делает себя простым и тождественным. Эта
рефлексия в~себя есть лишь нечто субъективное, потому что в ее
опосредствовании различие еще имеется лишь как пред-положенное
{\em в-себе-сущее}
различие, как разность
{\em объекта} внутри
себя. Поэтому то определение, которое получается через это соотношение,
есть форма простого {\em тождества,
абстрактной всеобщности}. Аналитическое познание имеет
поэтому вообще своим принципом указанное тождество, и из него самого, из
его деятельности исключены переход в другое и связывание разного.

При более близком рассмотрении аналитического познания мы
устанавливаем, что в нем начинают с некоторого
{\em пред-положенного}
и, стало быть, единичного,
{\em конкретного}
предмета, причем все равно, есть ли он уже
{\em готовый} для
представления предмет или же некоторая
{\em задача}, т.~е. дан
лишь в своих обстоятельствах и условиях, но еще не выделен из них особо и
не изображен в простой самостоятельности. Анализ такого предмета не может
состоять в том, что его просто
{\em разлагают} на те
особенные {\em представления},
которые он, возможно, содержит в~себе; такого рода разложение
и его усвоение есть дело, не принадлежащее к области познания в собственном
смысле, а касающееся лишь более подробного
{\em ознакомления},
некоторого определения внутри сферы
{\em представления}. Так
как анализ имеет своим основанием понятие, то он имеет своими продуктами
существенным образом определения понятия, притом как такие определения,
которые {\em непосредственно
содержатся} в предмете. Из рассмотрения природы идеи
познания получился тот вывод, что деятельность субъективного понятия должна
рассматриваться, с одной стороны, лишь как
{\em развитие} того, что
уже {\em есть в объекте},
потому что сам объект есть не~что иное, как тотальность
понятия. Настолько же односторонне представлять себе анализ так, как будто
в предмете нет ничего такого, что не было бы
{\em вложено} в него,
насколько односторонне полагать, что получающиеся определения
только {\em вынимаются}
из него. Первое представление, как известно, высказывается
субъективным идеализмом, принимающим деятельность познания в анализе
исключительно только за некоторое одностороннее
{\em полагание}, по ту
сторону которого остается скрытой
{\em вещь-в-себе;} другое
представление принадлежит так называемому реализму, который понимает
субъективное понятие как пустое тождество,
{\em принимающее} в~себя
определения мысли {\em извне}. "---
Так как оказалось, что аналитическое познание, превращение
данного материала в логические определения, есть разом и то и другое, есть
{\em полагание}, которое
столь же непосредственно определяет себя как
{\em пред-полагание}, то
в силу последнего логическое может казаться чем-то уже
{\em готовым} в предмете,
равно как в силу первого оно может казаться
{\em продуктом} чисто
субъективной деятельности. Но на самом деле нельзя отделить эти два момента
друг от друга; логическое в той его абстрактной форме, в которую его
выделяет анализ, несомненно имеется лишь в познании, равно как и наоборот,
оно есть не только нечто
{\em положенное}, но и
нечто {\em в-себе-сущее}.

Поскольку аналитическое познание есть указанное нами
превращение [данного материала в логические определения], оно не проходит
ни через какие дальнейшие {\em средние
члены}, а определение постольку
{\em непосредственно} и
имеет как раз тот смысл, что оно присуще предмету и принадлежит ему само по
себе и поэтому может быть схвачено без субъективного опосредствования, лишь
исходя из него. "--- Но познание, далее, должно быть также и
некоторым {\em поступательным
движением}, некоторым
{\em развитием различий}.
Но так как по тому определению, которым оно обладает здесь,
оно чуждо понятию и недиалектично, то оно обладает здесь лишь некоторым
{\em данным различием}, и
его поступательное движение совершается исключительно на определениях
{\em материала}. Оно
кажется обладающим некоторым
{\em имманентным}
поступательным движением лишь постольку, поскольку выводимые
им определения мысли могут быть снова подвергнуты анализу, если они еще
суть нечто конкретное; высшим и последним этапом этого анализа оказывается
абстрактное высшее существо, или, иначе говоря, абстрактное субъективное
тождество и противостоящая ему разность. Это поступательное движение есть,
однако, не~что иное, как лишь повторение одного первоначального дела
анализа, а именно, определение вновь как некоторого
{\em конкретного} того,
что уже вобрано в абстрактную форму понятия, а затем
"--- анализ этого конкретного, и после снова определение того
абстрактного, которое явилось результатом этого анализа, как чего-то
конкретного, и~т.~д. "--- Но определения мысли
представляются содержащими в них самих также и некоторый
переход. Если предмет был определен как целое, то, разумеется, от этого
определения идут дальше к {\em другому}
определению, к
{\em части;} от
{\em причины} "--- к другому
определению, к {\em действию}
и~т.~д. Но это здесь постольку не есть поступательное
движение, поскольку целое и части, причина и действие, суть
{\em отношения} и притом
являются для этого формального познания такими
{\em готовыми}
отношениями, что мы
{\em преднаходим} одно
определение существенно связанным с другим. Предмет, который определен как
{\em причина} или как
{\em часть}, в силу этого
определен {\em всем}
отношением, т.~е. уже обеими его сторонами. Хотя оно
{\em в~себе} есть нечто
синтетическое, все же эта взаимосвязь представляет собой для аналитического
познания в такой же мере лишь нечто
{\em данное}, как и
всякая другая взаимосвязь его материала, и потому не входит в состав его
специфической задачи. Определяют ли вообще такого рода взаимосвязь как
нечто априорное или апостериорное, это здесь безразлично, поскольку ее
понимают как {\em преднайденную}
или (как это также называли) как
{\em факт} сознания,
которое, дескать, с определением
<<{\em целое}>> связывает
определение <<{\em часть}>>,
и так далее. Выдвинув глубокое замечание о
{\em синтетических}
основоположениях à~priori и признав, что их корнем служит
единство самосознания, следовательно, тождество понятия с самим собой, Кант
все же заимствует {\em определенную}
взаимосвязь, т.~е. сами понятия отношений и синтетические
основоположения, из {\em формальной
логики}, берет их как
{\em данные}. Их дедукция
должна была бы быть изображением перехода указанного простого единства
самосознания в эти его определения и различия; но Кант избавил себя от
этого труда и не вскрыл нам это истинно синтетическое поступательное
движение, само себя продуцирующее понятие.

Как известно,
{\em арифметику} и более
всеобщие {\em науки}
{\em о дискретной величине}
по преимуществу называют
{\em аналитической наукой}
и {\em анализом}.
Способ познания в этих науках и в самом деле наиболее
имманентно аналитичен, и мы должны вкратце рассмотреть, в чем заключается
основание этого факта. "--- Прочее аналитическое познание
начинает с некоторого конкретного материала, содержащего в~себе некоторое
случайное многообразие; всякое различие содержания и поступательное
движение к дальнейшему содержанию зависят от указанного конкретного
материала. Напротив, арифметический и алгебраический материал есть нечто
уже сделанное совершенно абстрактным и неопределенным, материал, в котором
истреблено всякое своеобразие отношения и для которого, стало быть, теперь
всякое определение и всякое связывание есть нечто внешнее. Таким
материалом является принцип дискретной величины,
{\em одно}. Эта лишенная
всякого отношения атомистическая единица может быть умножена для
образования некоторого {\em множества},
элементы которого можно внешним образом определить и
соединить в некоторую численность; но этот процесс умножения и ограничения
есть пустое поступательное движение и процесс определения, не идущий дальше
того же самого принципа абстрактного одного. Далее, каким образом
соединяются и разъединяются
{\em числа}, это зависит
исключительно только от полагания познающего. Все эти определения делаются
вообще внутри категории {\em величины},
а она представляет собой ставшую безразличной определенность,
так что предмет не обладает никакой такой определенностью, которая была бы
ему имманентна и, следовательно, была бы
{\em дана} познанию.
Поскольку познание дало себе первоначально некоторое случайное многообразие
чисел, они составляют материал для дальнейшей обработки и многообразных
соотношений. Такие соотношения, их нахождение и обработка кажутся, правда,
отнюдь не имманентными аналитическому познанию, а чем-то
случайным и данным (и в самом деле, эти соотношения и связанные с ними
операции излагаются обычно {\em друг за другом} как {\em разные}, без
указания внутренней связи между ними). Однако здесь нетрудно распознать
движущий принцип, а именно "--- имманентный принцип
аналитического тождества, которое в разнящемся выступает в виде
{\em равенства;} движение
вперед состоит здесь в сведении неравного к все б\'{о}льшему и б\'{о}льшему
равенству. Чтобы иллюстрировать это на примере из первых элементов
математики, укажем, что сложение есть сочетание совершенно случайно
{\em неравных} чисел, умножение же, напротив, "--- {\em равных}
чисел, а затем еще следует отношение {\em равенства} между {\em численностью}
и {\em единицей}, т.~е. степенн\'{о}е отношение\pagenote{В~сложении
соединяются вместе два (если брать простейший случай)
каких-нибудь числа, которые не обязательно равны друг другу: поэтому, как
более общий случай, Гегель берет сложение двух {\em неравных}
чисел (например, $5+7$). В~умножении одно и то же число прибавляется к самому
этому числу: поэтому Гегель рассматривает умножение как сложение
{\em равных} чисел (например, $2\cdot 5=5+5$). При возведении в квадрат
(который, по Гегелю, представляет собою {\em основную} математическую
степень) имеет место равенство между тем числом, которое возводится в
квадрат (по Гегелю это "--- <<{\em единица}>>), и тем числом, которое служит
<<{\em численностью}>>, т.~е. множителем (например,
$5^2=5\cdot 5=5+5+5+5+5$). Подробнее об арифметических действиях
Гегель говорит в I~части <<Науки логики>> (см. т.~I <<Науки логики>>,
стр.~158---164).\label{bkm:bm101}}.

Так как определенность предмета и отношений есть {\em положенная}
определенность, то дальнейшие операции с ними тоже совершенно
аналитичны, и в аналитической науке имеются поэтому не столько
{\em теоремы}, сколько {\em задачи}.
Аналитическая теорема содержит в~себе задачу уже как решенную
самоё по себе, и совершенно внешнее различие, присущее тем двум сторонам
теоремы, которые в ней приравниваются друг к другу, столь несущественно,
что такая теорема должна была бы показаться тривиальным тождеством. Кант,
правда, объявил предложение <<$5+7=12$>> {\em синтетическим}
предложением на том основании, что одно и то же содержание на
одной стороне представлено в форме нескольких чисел, в форме 5 и 7, а на
другой стороне "--- в форме одного числа, в форме~12\pagenote{См. {\em Кант},
Критика чистого разума, 2-е нем. изд., стр.~15---16.\label{bkm:bm102}}.
Однако, если аналитическое предложение не
должно означать совершенно абстрактно тождественное и тавтологическое
<<$12=12$>> и в нем вообще должно быть некоторое движение вперед, то должно
быть налицо какое-нибудь различие, но такое различие, которое не
основывается ни на каком качестве, ни на какой определенности рефлексии и
тем паче ни на какой определенности понятия. <<$5+7$>> и <<12>> суть совершенно
то же самое содержание; в первой стороне равенства выражено также и
{\em требование}, чтобы 5 и 7 были сочетаны в {\em одном}
выражении; а это означает, что, подобно тому как 5 есть нечто
сосчитанное, причем прекращение счета на этом числе было совершенно
произвольным и счет мог бы с таким же успехом быть продолжен и дальше, так
теперь следует считать дальше с условием, чтобы число долженствующих быть
прибавленными единиц равнялось 7. <<12>> есть, следовательно, результат 5 и 7
и такого действия, которое здесь уже положено и по своей природе тоже есть
некоторое совершенно внешнее, чуждое мысли дело, так что этот результат
может поэтому быть осуществлен также и машиной. Здесь нет ни малейшего
перехода к чему-то {\em иному;} это просто процесс продолжения, т.~е.
{\em повторения} того же самого действия, через которое произошли 5~и~7.

{\em Доказательство} такой теоремы "--- она требовала бы доказательства, если
бы она была синтетическим предложением, "--- состояло бы лишь в операции
определенного 7-ью дальнейшего счёта, начиная с 5-ти, и в познании
совпадения результата этого дальнейшего счета с тем, что и в других случаях
называется 12-ью и что в свою очередь есть не~что иное, как именно само это
определенное дальнейшее сосчитывание. Поэтому вместо формы теоремы сразу же
берут форму {\em задачи}, {\em требования} действия,
а именно, высказывается лишь {\em одна}
сторона того уравнения, которое составило бы теорему, другая
же сторона этого уравнения должна быть найдена путем решения этой задачи.
Задача заключает в~себе содержание и указывает то определенное действие,
которое должно быть произведено над ним. Действие не ограничено каким-либо
неподатливым, наделенным специфическими отношениями материалом, а
представляет собой внешнюю субъективную операцию, и материал безразлично
принимает те определения, которые в нем полагаются этим действием. Вся
разница между поставленными в задаче условиями и полученным в {\em решении}
результатом состоит лишь в том, что в последнем {\em действительно}
произведено соединение или разъединение тем определенным
образом, как было указано в задаче.

Применение здесь формы геометрического метода, относящегося к синтетическим
предложениям, и присоединение вслед за {\em решением} задачи также и {\em
доказательства} представляют собой поэтому совершенно излишнее сооружение. Это
доказательство не может выразить ничего другого, кроме той тавтологии, что
решение правильно, потому что действие произведено так, как было задано. Если
задача требует сложить несколько чисел, то решение состоит в том, что их
действительно складывают; доказательство же показывает, что решение правильно,
потому что было задано сложить и было произведено сложение. Если задача
заключает в~себе более сложные определения и действия, скажем, например,
перемножить десятичные числа\pagenote{Под <<десятичными числами>>
(Dezimal\-zahlen) Гегель имеет здесь в виду любые {\em многозначные} числа по
десятичной системе счисления, в том числе и многозначные десятичные
дроби.\label{bkm:bm103}}, а решение не указывает ничего, кроме механического
приема, то в этом случае, действительно, требуется доказательство; но это
доказательство не может состоять ни в чем другом, как только в анализе тех
определений и действий, из которых решение получается само собой. В~силу этого
отделения {\em решения}, как некоторого механического приема, от
{\em доказательства}, как припоминания природы подлежащего действию предмета и
самого действия, как раз утрачивается преимущество аналитической задачи,
заключающееся в том, что {\em построение} непосредственно выводится из задачи и
потому само по себе может быть изложено как {\em понятное для рассудка}, между
тем как, действуя иначе, мы ясно выраженным образом сообщаем построению
недостаток, свойственный синтетическому методу. "--- В~высшем анализе, где,
главным образом в связи со степенн\'{ы}м отношением, появляются качественные и
зависящие от понятийных определенностей отношения дискретных величин, задачи и
теоремы, действительно, содержат в~себе синтетические определения; там
приходится брать в качестве средних членов {\em другие} определения и
отношения, чем те, которые {\em непосредственно указаны} задачей или теоремой.
Однако и эти вспомогательные определения непременно должны быть такого рода,
чтобы они имели свое основание в том, что здесь принимается в соображение и
развивается одна из сторон задачи или теоремы; то обстоятельство, что они
выглядят синтетическими, происходит исключительно оттого, что задача или
теорема сама не называет наперед этой стороны. "--- Задача, например, найти
сумму степеней корней уравнения решается посредством рассмотрения и затем
соединения функций, представляющих собой коэфициенты уравнения корней. Взятое
здесь в помощь определение функций коэфициентов и соединения этих функций не
выражено наперед в задаче, но во всем прочем само развертывание совершенно
аналитично. Подобным же образом решение уравнения $X^m-1=0$ с~помощью синусов,
а~также имманентное, как известно, найденное Гауссом\pagenote{Гегель,
повидимому, имеет в~виду <<Арифметические исследования>> (Dis\-quisi\-tio\-nes
arith\-meticae) Гаусса, вышедшие в 1801~г.\label{bkm:bm104}} алгебраическое
решение при помощи рассмотрения {\em остатка} от делениях $X^{m-1}-1=0$ на $m$
и так называемых первообразных корней "--- одно из важнейших расширений анализа
новейшего времени "--- есть синтетическое решение, так как использованные тут
вспомогательные определения (синусы или рассмотрение остатков) не являются
определениями самой задачи.

О природе того анализа, который рассматривает так называемые
бесконечные разности переменных величин, т.~е. о природе диференциального и
интегрального исчисления, мы говорили более подробно в
{\em первой части} этой
логики. Там мы показали, что в основании этого анализа лежит качественное
определение величин, которое можно уразуметь только через понятие. Переход
от величины как таковой к этому определению уже не аналитичен. Математика
до сего дня не была в состоянии оправдать действия, покоящиеся на этом
переходе, собственными силами, т.~е. математическим путем, именно потому,
что природа этого перехода не математическая.
{\em Лейбниц}, которому
приписывается слава преобразования операций с бесконечными разностями в
{\em вычисление}, сделал
указанный переход, как мы об этом говорили там же, самым
неудовлетворительным образом, столь же чуждым понятию, сколь и
нематематическим; но раз мы предположим этот переход "--- а при
нынешнем состоянии науки он представляет собой не больше, чем
предположение, "--- то все дальнейшее представляет собой
действительно лишь ряд обыкновенных аналитических действий.

Мы упомянули, что анализ становится синтетическим, поскольку
он приходит к таким {\em определениям},
которые уже не
{\em положены} самими
задачами. Но всеобщий переход от аналитического к синтетическому познанию
вызывается необходимостью перехода от формы непосредственности к
опосредствованию, от абстрактного тождества к различию. Аналитическое
познание не идет в своей деятельности дальше определений вообще, поскольку
они соотносятся с самими собой; но в силу их
{\em определенности} их
природа существенно такова, что они
{\em соотносятся} также и
с {\em чем-то иным}.
Мы уже сказали, что если аналитическое познание и переходит к
таким отношениям, которые представляют собой не данный извне материал, а
определения мысли, оно все же остается аналитическим, поскольку для него и
эти отношения также суть {\em данные}.
Но так как абстрактное тождество, которое это познание
признает своим единственным принципом, есть по существу
{\em тождество различенного},
то оно и как таковое должно входить в состав
познания, и для субъективного понятия должна стать положенной им и
тождественной с ним также и
{\em связь}.

\subsection[b) Синтетическое познание]{b) Синтетическое познание}

Аналитическое познание есть первая посылка всего
умозаключения "--- {\em непосредственное}
соотношение понятия с объектом;
{\em тождество} есть
поэтому то определение, которое оно признает своим, и это познание есть
лишь {\em схватывание}
того, что {\em есть}.
Синтетическое познание стремится
{\em постигнуть} то, что
{\em есть}, т.~е.
уразуметь многообразие определений в их единстве. Оно поэтому есть вторая
посылка умозаключения, в
которой\pagenote{В~немецком тексте всех изданий напечатано <<in welchem>>. Повидимому, это
"--- опечатка, вместо <<in welcher>>.\label{bkm:bm105}}
оказывается соотнесенным
{\em разное} как таковое.
Его последняя цель заключается поэтому в
{\em необходимости}
вообще. "--- Соединенные разные соединены отчасти
в некотором {\em отношении;}
в последнем они столь же соотнесены друг с другом, сколь и
безразличны друг к другу, самостоятельны по отношению друг к другу; отчасти
же они соединены в {\em понятии},
и последнее есть их простое, но определенное единство. И~вот,
поскольку синтетическое познание ближайшим образом переходит от
{\em абстрактного тождества}
к {\em отношению}
или от {\em бытия}
к {\em рефлексии},
постольку тем, что понятие познает в своем предмете, еще не
служит абсолютная рефлексия понятия; та реальность, которую понятие
сообщает себе, есть ближайшая следующая ступень, а именно, указанное
тождество разных как таковых, которое поэтому еще есть вместе с тем лишь
{\em внутреннее}
тождество и лишь необходимость, не есть субъективное,
для-себя-сущее тождество и потому еще не есть понятие как таковое. Поэтому,
хотя синтетическое познание имеет своим содержанием также и определения
понятия и хотя объект полагается в этих определениях, однако они пока что
находятся лишь в {\em отношении}
друг к другу или пребывают в
{\em непосредственном}
единстве; а тем самым они не находятся в том единстве,
благодаря которому понятие имеет бытие как субъект.

Это составляет конечность рассматриваемого познания; так как
эта реальная сторона идеи пока что еще обладает в нем тождеством как
{\em внутренним}, то ее
определения еще {\em внешни}
себе; так как она еще не выступает как субъективность, то
тому собственному содержанию, которое понятие имеет в своем предмете, все
еще недостает {\em единичности},
и хотя теперь понятию соответствует в объекте уже не
абстрактная, а {\em определенная}
форма и, следовательно,
{\em особенное} понятие,
однако {\em единичное}
этого объекта есть все еще
{\em некоторое данное}
содержание. Поэтому, хотя указанное познание и превращает
объективный мир в понятия, все же оно дает ему сообразно
определениям понятия лишь форму, а что касается объекта со стороны его
{\em единичности}, его
определенной определенности, то оно должно его
{\em найти;} оно еще не
есть самоопределяющее познание. И~точно так же оно
{\em находит} положения и
законы и доказывает их
{\em необходимость}, но
не как необходимость предмета самого по себе, т.~е. не как необходимость из
понятия, а как необходимость познания, шествующего вдоль данных
определений, различий явления и познающего
{\em для себя} то или
иное положение как единство и отношение или, иначе говоря, познающего из
{\em явления} его
основание.

Теперь следует рассмотреть более детальные моменты синтетического познания.

\subsubsection[1. Дефиниция]{\bfseries 1. Дефиниция}

Прежде всего отметим черту, заключающуюся в том, что пока что
еще данную объективность превращают в простую форму как первую форму, стало
быть, в форму {\em понятия;}
моменты этого схватывания суть потому те же самые, что и
моменты понятия: {\em всеобщность},
{\em особенность} и
{\em единичность}. "---
{\em Единичное} есть сам
объект как {\em непосредственное
представление}, есть то, что должно быть дефинировано.
В~определении объективного суждения или суждения необходимости всеобщий
аспект этого объекта оказался
{\em родом}, и притом
{\em ближайшим}, а
именно, всеобщим, обладающим той определенностью, которая вместе с тем
служит принципом для различия, характеризующего особенное. Этим различием
предмет обладает в лице того
{\em специфического различия},
которое делает его определенным видом и которое служит
основанием его отделения от других видов.

Дефиниция, сводя таким образом предмет к его
{\em понятию}, совлекает
его внешние черты, которые нужны для его существования; она абстрагирует от
того, что прибавляется к понятию в процессе его реализации, благодаря чему
понятие выступает из себя и переходит, во-первых, в идею и, во-вторых, во
внешнее существование. {\em Описание}
обращается к
{\em представлению} и
подхватывает это дальнейшее, принадлежащее к области реальности содержание.
Дефиниция же сводит это богатство многообразных определений созерцаемого
наличного бытия к простейшим моментам; какова форма этих простых элементов
и как они определены относительно друг друга, это содержится в понятии.
Стало быть, предмет, как было указано, понимается здесь как такое всеобщее,
которое вместе с тем по существу есть определенное. Сам предмет есть при
этом нечто третье, единичное, в котором род и особенность
положены воедино, и нечто
{\em непосредственное},
положенное {\em вне}
понятия, так как последнее еще не есть самоопределяющее
понятие.

В указанных определениях, в имеющемся в дефиниции различии
формы понятие обретает само себя и имеет в них соответствующую ему
реальность. Но так как рефлексия моментов понятия в~себя самих,
единичность, еще не содержится в этой реальности (ибо в результате объект,
поскольку он находится в познании, еще не определен как нечто
субъективное), то познание есть по отношению к объекту нечто субъективное и
имеет некоторое внешнее начало; или, иначе говоря, вследствие того, что оно
имеет внешнее начало в единичном, оно есть нечто субъективное. Содержание
понятия есть поэтому нечто данное и случайное. Само конкретное понятие
есть, стало быть, нечто случайное, и притом с двух сторон: во-первых, со
стороны своего содержания вообще и, во-вторых, со стороны того, какие
определения содержания из тех многообразных качеств, которыми предмет
обладает во внешнем существовании, отбираются для понятия и должны
составлять его моменты.

Последнее соображение требует более детального рассмотрения.
Дело в том, что так как единичность, как нечто
в-себе-и-для-себя определенное, лежит вне свойственного
синтетическому познанию определения понятия, то нет принципа, из которого
вытекало бы, какие стороны предмета должны рассматриваться как
принадлежащие к составу его понятийного определения и какие должны
рассматриваться как принадлежащие лишь к составу внешней реальности. Это
создает при дефинициях трудность, неустранимую для этого познания. Мы
должны, однако, проводить при этом следующее различение. "---
{\em Во-первых}, что
касается продуктов самосознательной целесообразности, то легко найти их
дефиницию, ибо цель, которой они должны служить, представляет собой
некоторое определение, порожденное субъективным решением и составляющее
существенную особенность, ту форму существующего, которая здесь единственно
важна. Прочая природа его материала или другие внешние свойства, поскольку
они соответствуют цели, содержатся в ее определении; остальные для нее
несущественны.

{\em Во-вторых},
геометрические предметы суть абстрактные пространственные
определения; лежащая в их основании абстракция (так называемое абсолютное
пространство) потеряла всякие дальнейшие конкретные определения и имеет
теперь лишь такие формы (Qestalten) и фигурации, какие в ней
полагают; {\em они}
поэтому {\em суть}
по существу лишь то, чем они
{\em должны} быть; их
понятийное определение вообще и, ближе, их специфическое
различие имеет в них свою простую, не встречающую помех реальность; они
суть постольку то же самое, что и продукты внешней целесообразности, и
вместе с тем они сходны в этом отношении также и с арифметическими
предметами, в основании которых лежит равным образом лишь то определение,
которое было в них положено. "--- Пространство, правда,
обладает еще и дальнейшими определениями (тройственностью своих измерений,
непрерывностью и делимостью), которые не полагаются в нем впервые через
внешний процесс определения. Однако эти определения принадлежат к взятому
извне материалу и суть непосредственные предпосылки; лишь сочетание и
переплетение вышеуказанных субъективных определений с этой своеобразной
природой той их почвы, на которую они занесены, создает синтетические
отношения и законы. "--- Так как в основании числовых
определений лежит простой принцип
{\em одного}, то их
сочетание и дальнейшее определение есть всецело лишь нечто положенное;
напротив, определения в пространстве, которое само по себе есть некоторая
непрерывная {\em внеположность},
разветвляются еще и дальше и обладают разнящейся от их
понятия реальностью, которая, однако, уже больше не принадлежит к составу
непосредственной дефиниции.

Но, {\em в-третьих},
с дефинициями
{\em конкретных} объектов
как природы, так и духа дело выглядит совершенно иначе. Такие предметы суть
вообще для представления {\em вещи}
{\em со многими свойствами}.
Здесь нужно прежде всего схватить, каков их ближайший род, а
затем установить, в чем состоит их специфическое видовое отличие. Следует
поэтому определить, какое из многих свойств принадлежит предмету как роду и
какое принадлежит ему как виду; далее, какое из этих свойств есть
существенное; а для ответа на последний вопрос требуется познать, в какой
связи они находятся друг с другом, положено ли уже одно из них вместе с
другим. Но для этого еще нет никакого другого критерия, кроме самого
{\em существования}. "---
~Существенностью свойства для дефиниции, в которой свойство
должно быть положено как простая, неразвитая определенность, служит его
всеобщность. Но последняя есть в существовании лишь чисто эмпирическая
всеобщность; это "--- всеобщность во времени, устойчивость того
или иного свойства, в то время как другие свойства оказываются преходящими
при продолжающемся существовании целого; или же это
"--- всеобщность, проистекающая из сравнения с другими
конкретными целыми и постольку не идущая дальше простой общности. Если
сравнение приводит к заключению, что целокупный облик, как
он эмпирически являет себя, есть общая основа, то задача рефлексии состоит
в том, чтобы объединить его в простое определение мысли и схватить простой
характер такой целокупности. Но подтверждением того, что то или иное
определение мысли или то или иное из непосредственных свойств составляет
простую и определенную сущность предмета, может быть лишь
{\em выведение} такого
определения из конкретного характера. А~это потребовало бы анализа,
превращающего непосредственный характер в мысли и сводящего конкретные
черты этого характера к чему-то простому, "--- анализа, который
стоит выше, чем только что рассмотренный, так как он должен был бы быть не
абстрагирующим, а таким, который еще сохраняет во всеобщем определенные
черты конкретного, объединяет последние и показывает их зависимость от
простого определения мысли.

Но соотношения многообразных определений непосредственного
существования с простым понятием были бы теоремами, нуждающимися в
доказательстве. Дефиниция же как первое, еще не развитое понятие, ввиду
того что она должна схватить простую определенность предмета, а это
схватывание должно быть чем-то непосредственным, "--- дефиниция
может пользоваться для этой цели лишь одним из его (предмета)
{\em непосредственных}
так называемых свойств "--- некоторым
определением чувственного существования или представления; изолирование
этого свойства путем абстракции составляет тогда простоту, а для
установления всеобщности и существенности понятие отсылается здесь к
эмпирической всеобщности, к факту сохранения свойства при изменившихся
обстоятельствах и к рефлексии, ищущей определение понятия во внешнем
существовании и в представлении, т.~е. там, где его нельзя найти. "---
Дефинирование поэтому и само отказывается от настоящих
определений понятия, которые были бы по существу принципами предметов, и
довольствуется {\em признаками},
т.~е. такими определениями,
{\em существенность}
которых для самого предмета есть нечто безразличное и которые
скорее имеют своей целью быть только
{\em отметками} для
некоторой внешней рефлексии. "--- Такого рода единичная,
{\em внешняя}
определенность находится в слишком большом несоответствии с
конкретной целокупностью и с природой ее понятия для того, чтобы она сама
по себе могла быть выбрана и признана тем, в чем некоторое конкретное целое
имеет свое истинное выражение и определение. "--- Так,
например, по замечанию
{\em Блюменбаха}, ушная
мочка есть нечто такое, что у всех прочих животных отсутствует, и она,
следовательно, согласно обычным рассуждениям об общих и
отличительных признаках могла бы с полным правом быть
использована в дефиниции физического человека как то, что составляет его
отличительный характер. Но сколь несоответственным оказывается с первого же
взгляда такого рода совершенно внешнее определение представлению о
целокупном облике физического человека и требованию, чтобы определение
понятия было чем-то существенным! Являются ли включенные в дефиницию
признаки просто лишь такого рода средством выйти из затруднения, пускаемым
в ход за неимением лучшего, или же они в большей мере приближаются к
природе некоторого принципа, это "--- целиком дело случая.
Вследствие их внешнего характера по ним и видно, что не с них начали в
познании понятия; наоборот, открытию родов в природе и в духе
предшествовало смутное чувство, неопределенное, но более глубокое чутье,
некоторое предчувствие существенного, и лишь после этого начинали
отыскивать для рассудка ту или иную определенную внешнюю черту. "---
Так как в существовании понятие вступило в область внешнего,
то оно развернуто в свои различия и не может быть безоговорочно связано
лишь с одним единственным из таких свойств. Свойства, как внешнее в вещах,
внешни и самим себе. При рассмотрении сферы явления, там, где говорилось о
вещи со многими свойствами, было показано, что они вследствие этого
становятся по существу даже самостоятельными материями; если рассматривать
дух с той же точки зрения явления, то он превращается в агрегат многих
самостоятельных сил. Сама эта точка зрения, признающая отдельное свойство
или силу безразличными к другим, приводит к тому, что это свойство или сила
перестает быть характеризующим принципом, вследствие чего определенность
как определенность понятия вообще исчезает.

В конкретных вещах наряду с разностью свойств между собой
появляется еще и различие между {\em понятием} и его {\em осуществлением}.
В~природе и в духе понятие имеет некоторое внешнее изображение, в котором его
определенность являет себя как зависимость от внешнего, преходимость и
неадекватность. Поэтому нечто действительное показывает, правда, на себе,
чем оно {\em должно}
быть, но в такой же мере оно согласно отрицательному суждению
понятия может показывать также и то, что его действительность лишь
неполностью соответствует этому понятию, что она {\em дурна}. И~вот, если
дефиниция должна указать в некотором непосредственном свойстве
определенность понятия, то нет такого свойства, против которого нельзя было
бы привести как возражение такой случай, где, хотя весь облик предмета
несомненно позволяет нам признать в нем подлежащее
дефинированию конкретное, но свойство, принимаемое за характерную черту
этого конкретного, оказывается недозревшим или захиревшим. В~плохом
растении, в плохом роде животных, в презренном человеке, в плохом
государстве недостаточно наличествуют или совершенно стерты те стороны
существования, которые в других случаях можно было бы принимать для
дефиниции за отличительную черту и существенную определенность в
существовании такого конкретного. Но плохое растение, животное и~т.~д. все
еще остается растением, животным и~т.~д. Поэтому если признать, что в
дефиницию должно быть включено также и плохое, то от эмпирических поисков
там и сям ускользнут все те свойства, которые эмпирическая точка зрения
пыталась рассматривать как существенные, "--- ускользнут ввиду
особых случаев (уродств), в которых они отсутствуют; так, например,
существенность мозга для физического человека опровергается особыми
случаями безголовых уродов, существенность для государства защиты жизни и
собственности "--- особыми случаями деспотических государств и
тиранических правительств. "--- Если против особого случая
будут отстаивать понятие и, принимая последнее за мерило, будут выдавать
этот случай за плохой экземпляр, то здесь понятие уже больше не имеет
своего удостоверения в явлении. Но самостоятельность понятия противна
смыслу дефиниции, которая должна представлять собой
{\em непосредственное}
понятие и поэтому может заимствовать свои определения для
предметов лишь из непосредственности существования и доказывать свою
правомерность лишь на преднайденном материале. "--- Ответ на
вопрос о том, есть ли ее содержание истина
{\em в~себе и для себя}
или оно есть случайность, лежит вне ее сферы; вопрос же о
формальной истинности, о согласии понятия, субъективно положенного в
дефиниции, и некоторого вне его действительного предмета, не может быть
решен потому, что отдельный предмет может быть также и плохим.

Содержание дефиниции взято вообще из сферы непосредственного
существования, и так как оно непосредственно, оно не имеет оправдания.
Вопрос о его необходимости устранен его происхождением; тем самым, что она
высказывает понятие как нечто лишь непосредственное, она отказывается от
того, чтобы постигнуть его самого. Она поэтому не представляет собой ничего
другого, кроме формального определения понятия на некотором данном
содержании, без рефлексии понятия в~себя само, т.~е.
{\em без его для-себя-бытия}.

Но непосредственность вообще происходит лишь из
опосредствования; она должна поэтому перейти к последнему. Или,
иначе говоря, та определенность содержания, которую содержит
в~себе дефиниция, именно потому, что она есть определенность, представляет
собой не только нечто непосредственное, но и нечто опосредствованное своим
иным; дефиниция может поэтому схватить свой предмет лишь через
противоположное определение и должна поэтому перейти к
{\em подразделению}.

\subsubsection[2. Подразделение]{\bfseries 2. Подразделение}

Всеобщее должно {\em обособить} себя на {\em особенности;}
постольку необходимость деления заключена во всеобщем. Но так
как дефиниция уже сама начинает с особенного, то необходимость для нее
перейти к подразделению заключена в особенном, которое само по себе
указывает на некоторое другое особенное. И~обратно, именно здесь особенное
отделяется от всеобщего, поскольку мы фиксируем определенность, имея
потребность отличить ее от иной по отношению к ней определенности; тем
самым всеобщее {\em предполагается}
для подразделения как предпосылка. Поэтому хотя ход движения
здесь таков, что единичное содержание дефиниции восходит через особенность
к крайнему термину всеобщности, однако последнюю следует отныне принимать
за объективную основу, исходя из которой подразделение оказывается
дизъюнкцией, расчленением всеобщего как чего-то первого.

Тем самым появился переход, который, так как он совершается от
всеобщего к особенному, определен формой понятия. Дефиниция, взятая
изолированно, есть нечто единичное; какое- либо множество дефиниций
относится к многим предметам. Принадлежащее понятию поступательное движение
от всеобщего к особенному составляет основу и возможность некоторой
{\em синтетической науки},
некоторой {\em системы}
и {\em систематического
познания}.

Для этого первым требованием является, как было показано,
чтобы вначале предмет рассматривался в форме некоторого
{\em всеобщего}. Если в
действительности (будь это действительность природы или духа)
субъективному, естественному познаванию дана как нечто первое конкретная
единичность, то, напротив, в том познавании, которое по крайней мере
постольку есть постижение, поскольку оно имеет своей основой форму понятия,
первым должно быть {\em простое,
выделенное} из конкретного, так как лишь в этой форме
предмет имеет форму соотносящегося с собой всеобщего и такого
непосредственного, которое сообразно понятию. Против такого пути развития
науки можно, правда, выдвинуть примерно то возражение, что так как, мол,
созерцать легче, чем познавать, то началом науки мы должны
сделать также нечто поддающееся созерцанию, т.~е. конкретную
действительность, и что этот путь
{\em более сообразен с природой},
чем тот, который начинает с предмета в его абстрактности и
отсюда, наоборот, идет дальше к выделению в нем особенного и
конкретно-единичного. "--- Но так как задача состоит в том,
чтобы {\em познавать}, то
вопрос о сравнении с {\em созерцанием}
уже решен, и мы уже отказались от последнего;
"--- теперь вопрос может заключаться только в том, что должно
быть первым {\em в пределах познания}
и каков должен быть дальнейший порядок движения; теперь уже
требуется путь не {\em сообразный с
природой}, а{\em
сообразный с познанием}. "--- Если ставится
вопрос только о {\em легкости},
то и помимо этого само собой ясно, что познаванию легче
схватить абстрактное простое определение мысли, чем конкретное, которое
есть некоторое многообразное соединение таких определений мысли и их
отношений; а ведь именно таким образом, а не так, как оно дано в
созерцании, должно пониматься конкретное. Взятое само по себе,
{\em всеобщее} есть
первый момент понятия потому, что оно есть
{\em простое}, а
особенное есть только последующее потому, что оно есть опосредствованное; и
обратно, {\em простое}
есть нечто более всеобщее, а конкретное, как внутри себя
различенное и, стало быть, опосредствованное, есть то, что уже предполагает
переход от некоторого первого. "--- Это замечание касается не
только порядка движения в определенных формах дефиниций, подразделений и
теорем, но также и порядка познания вообще, и оно имеет силу только
касательно различия абстрактного и конкретного
вообще\pagenote{Ср. рассуждения Маркса о методе политической экономии во <<Введении к
Критике политической экономии>>. См. примечание \ref{bkm:bm8}.\label{bkm:bm106}}.
"--- Поэтому и при обучении, например,
{\em чтению} разумно
начинают не с чтения целых слов или хотя бы слогов, а c
{\em элементов} слов и
слогов и со знаков {\em абстрактных}
звуков; в буквенном письме анализ конкретного слова на его
абстрактные звуки и их знаки уже произведен, и процесс обучения чтению
именно вследствие этого становится первым занятием абстрактными предметами.
В {\em геометрии} следует
начинать не с некоторого конкретного пространственного образа, а с точки
или линии, а затем с плоских фигур, из последних же не с многоугольников, а
с треугольников, из кривых же линий "--- с круга. В~{\em физике} следует
освободить отдельные свойства природы или отдельные материи от тех их
многообразных переплетений, в которых они находятся в конкретной
действительности, и представить их в их простых, необходимых условиях; они,
как и пространственные фигуры, тоже являются чем-то таким, что можно
созерцать, но созерцание их должно быть подготовлено таким образом, чтобы
они сначала выступили и были фиксированы освобожденными от всяких
видоизменений теми обстоятельствами, которые внешни их
собственной определенности. Магнетизм, электричество, различные виды газов
и~т.~д. суть такие предметы, познание которых получает свою определенность
единственно только благодаря тому, что они нами берутся выделенными из
конкретных состояний, в которых они выступают в действительности.
Эксперимент, правда, представляет их созерцанию в некотором конкретном
случае; однако частью он, чтобы быть научным, должен брать для этого лишь
необходимые условия, частью же он должен быть многообразно повторен, чтобы
показать, что неотделимая конкретная сторона этих условий несущественна,
так как последние, когда разнообразится эксперимент, выступают в
разнообразном конкретном виде и, стала быть, для познания остается лишь их
абстрактная форма. "--- Чтобы привести еще один пример, укажем,
что могло бы казаться естественным и остроумным рассматривать
{\em цвет} сначала в его
конкретном явлении животному субъективному ощущению, затем вне субъекта как
некоторое призракообразное, висящее в воздухе явление и, наконец, во
внешней действительности, как прикрепленное к объектам. Однако для познания
всеобщей и тем самым истинно первой формой оказывается средняя из
названных, цвет, висящий в воздухе между субъективностью и объективностью,
как всем нам известный спектр, еще без всякой переплетенности с
субъективными и объективными обстоятельствами. Последние вначале лишь
мешают чистому рассмотрению природы этого предмета, ибо они ведут себя как
действующие причины и потому оставляют нерешенным вопрос о том, имеют ли
определенные изменения, переходы и отношения цвета свое основание в его
собственной специфической природе, или же их скорее следует приписать
болезненному специфическому характеру этих обстоятельств, здоровым или
болезненным особенным состояниям и действиям органов субъекта или
химическим, растительным и животным силам объектов. "--- Можно
привести много и других примеров из области познания органической природы и
мира духа; повсюду абстрактное должно составлять начало и ту стихию, в
которой и из которой развертываются особенности и богатые образы
конкретного.

И вот нужно сказать, что хотя при подразделении или
рассмотрении особенного появляется в собственном смысле отличие особенного
от всеобщего, это всеобщее, однако, само уже есть нечто определенное и,
стало быть, лишь один из членов некоторого подразделения. Поэтому для него
имеется некоторое более высокое всеобщее; а для этого последнего опять-таки
имеется еще более высокое всеобщее и так далее до
бесконечности. Для рассматриваемого здесь познания нет имманентной границы,
так как оно исходит из данного и его исходной точке присуща форма
абстрактной всеобщности. Итак, какой-нибудь предмет, представляющийся
обладающим некоторой элементарной всеобщностью, делается предметом
некоторой определенной науки и служит абсолютным началом постольку,
поскольку {\em предполагается},
что {\em представление}
уже знакомо с ним, и поскольку сам он берется как не
нуждающийся ни в каком выведении. Дефиниция берет его как нечто
непосредственное.

Дальнейшим движением от него является прежде всего
{\em подразделение}. Для
этого поступательного движения требовался бы только некоторый имманентный
принцип, т.~е. начинание со всеобщего и понятия; но рассматриваемому здесь
познанию недостает такого принципа, потому что оно следит лишь за
формальным определением понятия, взятым без его рефлексии в~себя, и потому
берет определенность содержания из данного. Для появляющегося в
подразделении особенного нет собственного основания ни касательно того, что
должно составлять основание деления, ни касательно того определенного
отношения, в котором члены дизъюнкции должны находиться друг с другом. Дело
познания в этом отношении может поэтому состоять лишь в том, чтобы частью
упорядочить отысканное в эмпирическом материале особенное, а частью также и
найти посредством сравнения его всеобщие определения. Последние тогда
признаются основаниями деления, которые могут быть многообразны, точно так
же, как могут иметь место столь же многообразные деления сообразно этим
основаниям. Отношение членов деления (видов) друг к другу имеет только то
общее определение, что они должны быть определены друг относительно друга
{\em по принятому основанию деления;} если бы их различие основывалось на
каком-нибудь другом соображении, то они не были бы координированы друг
с другом на той же самой плоскости.

Вследствие отсутствия принципа для имманентного определения
членов деления, взятых сами по себе, законы для этой операции деления могут
состоять лишь в формальных, пустых правилах, которые ни к чему не
приводят. "--- Так, например, мы находим, что выставляется в
качестве правила, что деление должно исчерпывать понятие; на самом же деле
каждый отдельный член деления должен исчерпывать {\em понятие}. Но
выставляющие это правило имеют в виду, собственно говоря, {\em определенность}
понятия: она именно и должна быть исчерпана; однако при
эмпирическом многообразии видов, лишенном определения внутри
себя, исчерпанию понятия нисколько не способствует то обстоятельство, что
мы преднаходим большее или меньшее количество этих видов; будет ли,
например, вдобавок к 67 видам попугаев найдена еще одна дюжина видов, это
для исчерпания рода безразлично. Требование исчерпывания может означать
лишь то тавтологическое положение, что все виды должны быть {\em полностью}
перечислены. "--- Очень легко может случиться,
что с расширением наших эмпирических сведений найдутся виды, которые не
подходят под принятое определение рода, потому что этот род часто
устанавливается больше в согласии с некоторым смутным представлением обо
всем облике, чем в согласии; с тем более или менее отдельным признаком,
который, как ясно высказывается дефиницией, должен служить для его
определения. "--- В таком случае нужно было бы изменить род и
следовало бы оправдать, почему мы должны рассматривать новое количество
видов как виды некоторого нового рода, т.~е. род определился бы из того,
что люди объединяют вместе по какому- либо соображению, которое хотят
принять за единство; само это соображение было бы при этом основанием
деления. И~наоборот, если продолжают держаться за ту определенность,
которая первоначально была принята в качестве характеристики рода, то
пришлось бы исключить вышеуказанный новый материал, который желательно было
сочетать воедино с прежними видами как равноправные виды. Эта возня без
понятия, при которой то принимают некоторую определенность в качестве
существенного момента рода и согласно этому подчиняют этому роду особенные
или исключают их из него, то начинают с особенных и руководствуются при их
сочетании опять-таки некоторой другой определенностью, "--- эта
возня представляет нам зрелище игры произвола, которому-де предоставлено
решать, какую часть или сторону конкретного он намерен фиксировать, чтобы
сообразно с этим строить свою классификацию. "--- Физическая
природа сама собой являет нам такую случайность в принципах деления; в силу
ее внешней, зависимой действительности она находится в многообразной, для
нее тоже данной, связи; оттого оказывается налицо множество принципов, с
которыми она должна сообразоваться, и она, стало быть, следует в одном ряду
своих форм одному принципу, а в других рядах "--- другим, а
также производит ублюдочные образования, одновременно идущие в направлении
разных сторон. Отсюда происходит то, что в одном ряду вещей природы
выступают как весьма характерные и существенные такие признаки, которые в
другом ряду становятся незначительными и бесцельными, вследствие
чего удерживание одного принципа деления в указанном смысле
становится невозможным.

Общая {\em определенность}
эмпирических видов может состоять лишь в том, что они вообще
{\em разнятся} друг от
друга, не будучи противоположными.
{\em Дизъюнкцию понятия}
мы показали выше в ее определенности; если особенность берут
без отрицательного единства понятия, как непосредственную и данную, то
различие остается лишь при рассмотренной выше рефлексивной форме разности
вообще. Тот внешний характер, в котором понятие преимущественно выступает в
природе, вносит полнейшее безразличие различия; поэтому часто определение
для классификации заимствуется из области
{\em чисел}\pagenote{По Гегелю число (и вообще
количество, величина) есть нечто внешнее, безразличное для предмета (в
известных границах, конечно). Из области чисел заимствуются определения для
классификации, например, в ботанических классификациях Линнея, где
принципом для подразделения на виды нередко служат такие признаки, как
число тычинок и~т.~д.\label{bkm:bm107}}.

Как бы случайно здесь ни было особенное по отношению к
всеобщему и, стало быть, деление вообще, все же имеется некоторый
{\em инстинкт} разума,
которому можно приписать то обстоятельство, что мы иногда находим в этом
познании такие основания деления и такие деления, которые, поскольку это
допускают чувственные свойства, оказываются более соответственными понятию.
Например, относительно {\em животных}
в классификационных системах употребляются в качестве широко
применимого основания деления орудия принятия пищи, зубы и когти; их берут
ближайшим образом лишь как те стороны, в которых могут быть легче намечены
признаки для субъективной цели познания. На самом же деле в этих органах не
только заключается различение, принадлежащее некоторой внешней рефлексии,
но они суть тот жизненный пункт животной индивидуальности, где она,
отправляясь от своего иного, от внешней ей природы, полагает самое себя
как соотносящуюся с собой единичность, выделяющую себя из непрерывности с
иным. "--- У {\em растений} органы
оплодотворения образуют ту высшую точку растительной жизни, которою
растение указывает на переход к половому различию и тем самым к
индивидуальной единичности. Поэтому классификационная система с полным
правом обратилась к этому пункту, как к хотя и недостаточному, но далеко
идущему основанию деления, и этим положила в основание такую
определенность, которая есть не просто определенность для внешней
рефлексии, для сравнения, но сама по себе есть наивысшая определенность, к
которой способно растение.

\subsubsection[3. Теорема]{\bfseries 3. Теорема}

1. Третью ступень этого познания, движущегося вперед согласно
определениям понятия, представляет собой переход особенности в единичность;
последняя составляет содержание {\em теоремы}. Следовательно,
{\em соотносящаяся с собой
определенность}, различие предмета внутри самого себя и
соотношение различенных определенностей друг с другом "--- вот
что должно быть рассмотрено здесь. Дефиниция содержит в~себе лишь
{\em одну определенность}, деление "--- определенность {\em по отношению
к другим определенностям;} в становлении единичным предмет разошелся
в разные стороны внутри самого себя. Если дефиниция не идет дальше
всеобщего понятия, то в теоремах предмет, напротив, познан в его
реальности, в условиях и формах его реального существования. Взятая вместе
с дефиницией, теорема поэтому изображает собой
{\em идею}, которая есть
единство понятия и реальности. Но рассматриваемое здесь, пребывающее еще в
поисках познание постольку не доходит до этого изображения, поскольку
реальность в нем еще не проистекает из понятия, следовательно, не познана
ее зависимость от последнего и, стало быть, не познано само единство
понятия и реальности.

Согласно указанному определению теорема есть настоящим образом
{\em синтетическое} в
предмете, поскольку отношения его определенностей
{\em необходимы}, т.~е.
обоснованы во {\em внутреннем
тождестве} понятия. Синтетическое в дефиниции и делении есть
принятая извне связь; преднайденному придается форма понятия, но как
преднайденное все содержание лишь
{\em показывается;}
теорема же должна быть
{\em доказана}. Так как
это познание {\em не дедуцирует}
содержания своих дефиниций и определений деления, то кажется,
что оно могло бы избавить себя от труда
{\em доказывания} также и
тех отношений, которые выражаются теоремами, и довольствоваться восприятием
также и в этом отношении. Однако познание отличается от голого восприятия и
представления именно формой понятия вообще, которую оно сообщает
содержанию; это осуществляется им в дефиниции и делении; но так как
содержание теоремы проистекает из понятийного момента
{\em единичности}, то она
состоит в таких определениях реальности, которые уже больше не имеют своими
отношениями только простые и непосредственные определения понятия; в
единичности понятие перешло к
{\em инобытию}, к
реальности, благодаря чему оно становится идеей. Тем самым синтез,
содержащийся в теореме, уже больше не имеет своим оправданием форму
понятия; он есть некоторое соединение как соединение
{\em разных}. Поэтому
пока что еще не положенное этим единство еще следует выявить; доказывание
здесь становится, следовательно, необходимым самому этому познанию.

При этом здесь прежде всего выступает трудность провести
определенное {\em различение}
касательно того, какие из
{\em определении
}{\em предмета} могут
быть включены в {\em дефиниции}
и какие из них должны быть отнесены в
{\em теоремы}.
Относительно этого не может быть никакого принципа. Правда,
может показаться, что такой принцип заключается, например, в том, что
присущее некоторому предмету непосредственно принадлежит к дефиниции,
относительно же остального, как опосредствованного, следует сначала выявить
его опосредствование. Однако содержание дефиниции есть некоторое
определенное вообще содержание и вследствие этого само оно есть по существу
нечто опосредствованное; оно имеет лишь некоторую
{\em субъективную}
непосредственность, т.~е. субъект делает некоторое
произвольное начало и допускает, чтобы некоторый предмет признавался в
качестве предпосылки. А~так как это есть вообще некоторый конкретный внутри
себя предмет и так как он должен подвергнуться также и подразделению, то
получается множество определений, которые по своей природе суть
опосредствованные и принимаются за непосредственные и недоказанные не в
силу какого-нибудь принципа, а лишь согласно субъективному определению. "---
И у {\em Евклида},
который искони справедливо признан мастером в этом
синтетическом способе познания, под названием
{\em аксиомы} фигурирует
{\em предпосылка о параллельных
линиях}, которая считалась требующей доказательства и
относительно которой делались разные попытки восполнить этот пробел.
Некоторые математики думали, что они открыли в некоторых других теоремах
такие предпосылки, которые должны были бы быть не приняты непосредственно,
а доказаны. Что же касается упомянутой аксиомы о параллельных линиях, то
можно относительно этого заметить, что как раз в ней видно правильное чутье
Евклида, точно оценившего как стихию, так и природу своей науки;
доказательство этой аксиомы нужно было бы вести, исходя из
{\em понятия}
параллельных линий; но такой способ доказательства так же
мало входит в задачу его науки, как и дедукция выставляемых им дефиниций,
аксиом и вообще его предмета "--- самого пространства и
ближайших его определений, измерений; так как такую дедукцию можно вести
только из понятия, а последнее лежит вне своеобразного характера евклидовой
науки, то указанные дефиниции, аксиомы и~т.~д. необходимым образом
представляют собой для этой науки некоторые
{\em предпосылки}, нечто
относительно-первое.

{\em Аксиомы}, "--- чтобы
сказать по этому поводу несколько слов и о них, "---
принадлежат к тому же классу. Их обыкновенно несправедливо
принимают за абсолютно-первые, как будто они сами по себе не нуждаются ни в
каком доказательстве. Если бы это было на самом деле так, то они были бы
чистыми тавтологиями, так как только в абстрактном тождестве
нет никакой разности и, следовательно, не требуется также и никакого
опосредствования. Но если аксиомы представляют собой нечто большее, чем
тавтологии, то они суть
{\em положения},
заимствованные из
{\em какой-либо}
{\em другой науки}, так
как для той науки, которой они служат в качестве аксиом, они должны быть
предпосылками. Они поэтому суть, собственно говоря,
{\em теоремы}, и притом
большей частью из
логики\pagenote{Ср. следующие замечания Энгельса
в <<Анти-Дюринге>>: <<Математические аксиомы представляют собой выражения
крайне скудного умственного содержания, которое математика должна
заимствовать у логики. Их можно свести к двум следующим аксиомам:

1. Целое больше части. Это положение есть чистая тавтология\ldots

2. Если две величины равны третьей, то они равны между собой. Это положение,
как показал еще Гегель, представляет
собой умозаключение, за правильность которого ручается логика; оно значит
доказывается, хотя и вне области чистой математики. Прочие аксиомы о
равенстве и неравенстве являются просто логическим развитием этого
умозаключения>> [{\em Энгельс}, Анти-Дюринг, Партиздат, 1936, стр.~27).
То умозаключение, о котором здесь говорит Энгельс, составляет у Гегеля
<<четвертую фигуру>> <<умозаключения наличного бытия>> и называется у Гегеля
<<математическим>> или <<чисто-количественным>> умозаключением. Его схема:
<<{\em B"--- B"--- B}>>.\label{bkm:bm108}}.
Аксиомы геометрии и суть подобного рода
леммы\pagenote{Греческое слово <<лемма>> означает <<заимствованное положение>>, т.~е. такое
положение, обоснование которого дано в другой науке.\label{bkm:bm109}},
логические положения, которые, впрочем, приближаются к
тавтологиям вследствие того, что они касаются лишь величины и поэтому
качественные различия в них стерты; о главной аксиоме, о чисто
количественном умозаключении, речь была
выше\pagenote{См. выше, стр.~\pageref{bkm:bm110a}---\pageref{bkm:bm110b}. Ср. примечание
\ref{bkm:bm107}.\label{bkm:bm110}}.
"--- Поэтому рассматриваемые сами по себе, аксиомы точно так же
нуждаются в доказательстве, как и дефиниции и подразделения, и их не делают
теоремами только потому, что они как относительно-первые принимаются для
известной точки зрения за предпосылки.

Касательно {\em содержания
теоремы} следует теперь провести то более детальное
различение, что, так как это содержание состоит в некотором
{\em соотнесении определенностей}
реальности понятия, то эти соотношения могут быть либо более
или менее неполными и отдельными отношениями предмета, либо же таким
отношением, которое охватывает {\em все
содержание} реальности и выражает его определенное
соотношение. Но {\em единство совокупных
определенностей содержания} равно
{\em понятию;}
предложение, содержащее это единство, само поэтому есть
опять-таки дефиниция, но такая дефиниция, которая выражает не только
непосредственно воспринятое понятие, но понятие, развернутое в свои
определенные, реальные различия, или, иначе говоря, полное существование
понятия. И~то и другое вместе взятое представляет поэтому
{\em идею}.

Если более детально сравнить между собой теоремы какой-нибудь
синтетической науки и {\em в особенности
геометрии}, то мы обнаружим следующее различие: одни из
теорем этой науки заключают в~себе лишь отдельные отношения предмета,
другие же "--- такие отношения, в которых выражена полная
определенность предмета. Очень поверхностен тот взгляд, который
рассматривает все предложения как равноценные на том основании, что вообще
каждое из них содержит, дескать, в~себе некоторую истину и что они в
формальном ходе изложения, в связи доказательства одинаково существенны.
Различие касательно содержания теорем находится в теснейшей связи с самим
этим ходом изложения; некоторые дальнейшие замечания об этом ходе изложения
послужат к тому, чтобы ближе осветить как указанное
различие, так и природу синтетического познания. Прежде всего необходимо
отметить следующее: в евклидовской геометрии, которая должна служить здесь
примером как представительница синтетического метода, наиболее совершенный
образец которого она доставляет, искони являлся предметом прославления
порядок расположения теорем, благодаря которому по отношению к каждой
теореме те предложения, которые требуются для ее построения и
доказательства, всегда уже имеются под рукой как уже доказанные раньше. Это
обстоятельство касается формальной последовательности; как ни важна эта
последовательность, она все же больше касается внешней целесообразности
расположения материала и сама по себе не имеет никакого отношения к
существенному различию понятия и идеи, в котором заключается более высокий
принцип необходимости поступательного движения. "--- А именно,
дефиниции, с которых начинают в геометрии, берут чувственный предмет как
непосредственно данный и определяют его по его ближайшему роду и
специфическому (видовому) отличию, которые тоже суть простые,
{\em непосредственные}
определенности понятия "--- всеобщность и
особенность, "--- отношение между которыми не развертывается
дальше. Начальные теоремы сами не могут касаться ничего другого, кроме
таких непосредственных определений, как те, которые содержатся в
дефинициях; а равно и их взаимная
{\em зависимость} может
ближайшим образом иметь только тот общий характер, что одно определение
вообще {\em определено}
другим. Так, первые теоремы Евклида о треугольниках касаются
лишь {\em совпадения},
т.~е. вопроса о том,
{\em сколько} составных
частей {\em должны быть определены}
в треугольнике, чтобы были
{\em вообще определены}
также и {\em остальные}
составные части того же самого треугольника или, иначе
говоря, весь треугольник в целом. Что тут сравниваются друг с другом два
треугольника и совпадение полагают в
{\em покрытии} одного
треугольника другим, это окольный путь, в котором нуждается метод, по
необходимости долженствующий пользоваться
{\em чувственным покрыванием}
вместо {\em мысли}
об {\em определимости}
как таковой. Помимо этого, рассматриваемые сами по себе, эти
теоремы сами содержат в~себе {\em две}
части, из которых на одну можно смотреть как на
{\em понятие}, а на
другую как на {\em реальность},
как на то, что завершает понятие, доводя его до реальности.
А~именно, то, что вполне определяет треугольник (например, две стороны и
заключенный между ними угол), есть
{\em для рассудка} уже
весь треугольник; для полной определенности последнего ничего больше не
требуется; остальные два угла и третья сторона есть избыток реальности над
определенностью понятия. Поэтому вот что, собственно говоря,
делают эти теоремы: они сводят чувственный треугольник, во всяком случае
нуждающийся в трех сторонах и трех углах, к его простейшим условиям;
дефиниция лишь вообще упомянула о трех линиях, замыкающих плоскую фигуру и
делающих ее треугольником; теорема же впервые точно и ясно указывает
{\em определяемость}
углов через определенность сторон, равно как другие теоремы
указывают зависимость других трех составных частей треугольника от трех
остальных частей. "--- Указание же на полную определенность
величины треугольника по его сторонам
{\em внутри его самого}
содержит в~себе
{\em пифагорова теорема;}
только она впервые является
{\em уравнением} сторон
треугольника, тогда как предшествующие
теоремы\pagenote{В~немецком тексте всех изданий стоит <<Seiten>> (стороны). Повидимому, это
опечатка вместо <<Satze>> или <<Lehrsatze>>.\label{bkm:bm111}}
доходят лишь вообще до установления
{\em определенности} его
частей по отношению друг к другу, а не до
{\em уравнения}. Эта
теорема есть поэтому совершенная,
{\em реальная дефиниция}
треугольника, а именно, прежде всего прямоугольного
треугольника, наиболее простого в своих различиях и потому наиболее
правильного. "--- Этой теоремой Евклид заканчивает первую
книгу, так как она (теорема) и в самом деле представляет собой достигнутую
совершенную определенность. Подобным же образом Евклид, после того как он
предварительно свел к единообразному
началу\pagenote{По Гегелю <<единообразное начало>>
(das Gleich\-förmi\-ge) представлено в данном случае
прямоугольным треугольником, о котором Гегель выше говорит, что он есть
<<наиболее простой в своих различиях и потому наиболее правильный
треугольник>>.\label{bkm:bm112}}
обремененные большим неравенством непрямоугольные
треугольники, заканчивает свою вторую книгу сведением прямоугольника к
квадрату, "--- уравнением между равным самому себе (квадратом)
и\pagenote{В~немецком тексте всех изданий вместо <<und>> напечатано <<mit>>. Это "--- явная
опечатка или описка.\label{bkm:bm113}}
неравным внутри себя (прямоугольником); точно так же и
гипотенуза, соответствующая прямому углу, т.~е. чему-то равному самому
себе, составляет в пифагоровой теореме одну сторону уравнения, а другую
сторону образует неравное себе, а именно
{\em два} катета.
Указанное уравнение между квадратом и прямоугольником лежит в основании
{\em второй} дефиниции
круга, которая опять-таки есть пифагорова теорема, поскольку катеты
принимаются за переменные величины; первое уравнение круга находится в
таком же самом отношении
{\em чувственной}
определенности к
{\em уравнению}, в каком
вообще находятся друг к другу две различных дефиниции конических сечений.

Это истинно синтетическое поступательное движение есть переход
от {\em всеобщего} к
{\em единичности}, а
именно, к {\em в~себе и для себя
определенному} или к единству предмета
{\em в самом себе},
поскольку предмет распался на свои существенные реальные
определенности и подвергся различению. Но в других науках совершенно
неполное обычное поступательное движение таково, что хотя в них начинают с
некоторого всеобщего, однако
{\em переход его в единичное}
и его конкретизация представляют собой лишь
{\em применение}
всеобщего к привходящему из какого-то другого места
материалу; подлинное {\em единичное},
единичное идеи есть при таком подходе некоторый
{\em эмпирический} придаток.

Но какое бы содержание ни имела теорема, более совершенное или
менее совершенное, она должна быть
{\em доказана}. Она есть
некоторое отношение реальных определений, не обладающих отношением
определений понятия; если они и обладают этим отношением, как это может
быть показано относительно предложений, которые мы назвали
{\em вторыми} или реальными {\em дефинициями},
то последние именно поэтому суть, с одной стороны, дефиниции;
но так как их содержание состоит вместе с тем из отношений реальных
определений, а не просто лишь в отношении между некоторым всеобщим и
простой определенностью, то они по сравнению с такой первой дефиницией тоже
нуждаются в доказательстве и могут быть доказаны. Как реальные
определенности, они имеют форму {\em безразлично существующих} и {\em разных}.
Вследствие этого они непосредственно не суть нечто единое;
следует поэтому вскрыть их опосредствование. Между тем непосредственное
единство В первой дефиниции есть то единство, в силу которого особенное
находится во всеобщем.

2. {\em Опосредствование}, которое мы теперь должны рассмотреть ближе, может
быть или простым или проходить через многие опосредствования. Опосредствующие
члены находятся в связи с теми членами, которые должны быть опосредствованы;
но так как в этом познании (которому вообще чужд переход в противоположное)
опосредствование и теорема выводятся не из
понятия\pagenote{В~немецком тексте всех изданий
напечатано: <<aus welchem die Vermittlung\ldots zurück\-gefuhrt wird>>.
Грамматически такая конструкция невозможна. Приходится
либо вместо <<aus welchem>> читать <<auf welchen>>, либо
вместо <<zurück\-geführt>> "--- <<abge\-lei\-tet>>. Перевод
сделан в соответствии со второй конъектурой.\label{bkm:bm114}},
то опосредствующие определения, не опирающиеся на понятие
связи, должны быть принесены, как предварительные материалы для здания
доказательства, откуда-то извне. Эта подготовка есть {\em построение}.

Из соотношений содержания теоремы, могущих быть очень
разнообразными, должны быть выбраны и изложены только те, которые служат
для доказательства. Это подбирание материала получает свой смысл только в
самом доказательстве; само по себе оно представляется слепым и чуждым
понятию. После, при доказательстве, мы, правда, усматриваем, что было
целесообразно провести в геометрической фигуре такие, например, дополнительные
линии, помимо указанных в построении; но само построение должно слепо
выполняться; поэтому сама по себе эта операция лишена смысла, так как
руководящая ею цель еще не высказана. "--- Безразлично,
предпринимается ли эта операция ради доказательства теоремы в собственном
смысле этого слова или ради решения задачи; взятая таковой, каковой она
представляется вначале, {\em до}
доказательства, она есть нечто, не выведенное из данного в
теореме или задаче определения, и поэтому представляет собой некоторое
бессмысленное действие для того, кто еще не знает цели; но она всегда есть
нечто управляемое лишь некоторой внешней целью.

Это первоначально еще тайное делается явным в
{\em доказательстве}.
Последнее заключает в~себе, как было указано,
опосредствование того, что в теореме высказано как связанное вместе; только
через это опосредствование указанная в теореме связь впервые
{\em выступает} как
{\em необходимая}.
Подобно тому как построение, само по себе взятое, лишено
субъективности понятия, так доказательство есть некоторое субъективное
действие, лишенное объективности. А~именно, так как определения содержания
теоремы не положены вместе с тем как определения понятия, а выступают
только как данные {\em безразличные
части}, находящиеся в многообразных внешних отношениях друг
к другу, то необходимость получается лишь в
{\em формальном, внешнем}
понятии. Доказательство не есть некоторый
{\em генезис} отношения,
составляющего содержание теоремы; необходимость существует лишь для нашего
разумения, а все доказательство "--- для
{\em субъективных потребностей
познания}. Доказательство есть поэтому вообще некоторая
{\em внешняя} рефлексия,
{\em идущая извне внутрь},
т.~е. умозаключающая от внешних обстоятельств к внутреннему
характеру отношения. Эти обстоятельства, изображенные построением,
представляют собой некоторое
{\em следствие} природы
предмета; здесь же их, наоборот, делают
{\em основанием} и
{\em опосредствующими}
отношениями. Средний термин, то третье, в котором связанные в
теореме определения выступают в своем единстве и которое составляет нерв
доказательства, есть поэтому лишь нечто такое, в чем эта связь
{\em является} и где она
носит {\em внешний}
характер. Так как то
{\em следствие}, которое
прослеживается доказательством, представляет собой скорее нечто обратное
природе самой вещи, то т\'{о}, чт\'{о} в доказательстве рассматривается как
{\em основание}, есть
субъективное основание, из которого природа вещи проистекает только для
познания.

Из сказанного делается ясной необходимая граница этого
познания, которая очень часто упускалась из вида. Блестящим примером
синтетического метода является наука
{\em геометрии}, но его
неуместно применяли также и к другим наукам и даже к философии. Геометрия
есть наука о {\em величинах},
и поэтому для нее весьма пригоден
{\em формальный} процесс
умозаключения; так как в ней рассматривают исключительно
количественное определение и абстрагируют от качественного, то она может
держаться в пределах {\em формального
тождества}, в пределах чуждого понятию единства, которое
есть {\em равенство} и
принадлежит внешней абстрагирующей рефлексии. Служащие предметом геометрии
пространственные определения суть уже наперед такие абстрактные предметы,
которые предварительно были препарированы для той цели, чтобы они имели
совершенно конечную, внешнюю определенность. Эта наука в силу своего
абстрактного предмета, с одной стороны, возвышенна в том смысле, что в этих
пустых, тихих пространствах цвет погашен и точно так же исчезли и другие
чувственные свойства и что, далее, в ней безмолвствует всякий другой
интерес, ближе затрагивающий живую индивидуальность. С~другой стороны,
рассматриваемый абстрактный предмет все еще есть
{\em пространство}, "---
некоторое
{\em нечувственно-чувственное;}
{\em созерцание} возведено
здесь в свою абстракцию, пространство есть
{\em форма} созерцания,
но все еще есть созерцание, "--- некое чувственное, сама
{\em внеположность}
чувственности, свойственное чувственности чистое
{\em отсутствие понятия}. "---
В новейшее время нам приходилось достаточно слышать о
превосходстве геометрии с этой стороны; тот факт, что в ее основании лежит
чувственное созерцание, был объявлен величайшим ее преимуществом, и даже
высказывалось мнение, что ее высокая научность основывается именно на этом
обстоятельстве и что ее доказательства покоятся на
созерцании\pagenote{Имеются в виду Кант и его сторонники.\label{bkm:bm115}}.
Против этого плоского взгляда необходимо прибегнуть к
плоскому напоминанию о том, что ни одна наука не осуществляется через
созерцание, а исключительно только
{\em через мышление}.
Наглядность, которой геометрия обладает благодаря ее все еще
чувственному материалу, сообщает ей только ту сторону очевидности, которую
{\em чувственное} вообще
имеет для немыслящего духа. Поэтому достойно сожаления то обстоятельство,
что преимуществом геометрии считали как раз ту чувственность материала,
которая скорее обозначает собой низменность ее точки зрения. Только
{\em абстрактности} ее
чувственного предмета она обязана своей способностью к более высокой
научности и своим великим преимуществом перед теми собраниями сведений,
которые многим угодно также называть науками, "--- собраниями
сведений, имеющими своим содержанием конкретное, ощутимое чувственное и
лишь порядком, который они стремятся внести в него, обнаруживающими
некоторое отдаленное предчувствие требований понятия, отдаленный намек на
них.

Лишь вследствие того, что пространство геометрии есть
абстрактная и пустая внеположность, делается возможным, чтобы в
его неопределенность вчерчивались фигурации таким образом,
что их определения остаются прочно покоящимися друг вне друга и не имеют
внутри себя никакого перехода в противоположное. Наука о них ость поэтому
простая наука о {\em конечном},
которое сравнивают по величине и единство которого есть
внешнее единство, {\em равенство}.
Но так как при этом начертании фигур исходят вместе с тем из
разных сторон и принципов и разные фигуры возникают отдельно, то при их
сравнении обнаруживаются все же также и
{\em качественное}
неравенство и
{\em несоизмеримость}.
Тем самым геометрия выводится за пределы
{\em конечности}, в
рамках которой она двигалась вперед столь размеренно и уверенно, и
понуждается перейти к
{\em бесконечности}, "--- к
приравниванию друг другу таких фигурации, которые разнятся между собой
качественно. Здесь прекращается ее очевидность, проистекавшая из того, что
во всех прочих случаях в ее основании лежит неподвижная конечность и ей не
приходится иметь дело с понятием и его явлением, т.~е. с указанным
переходом. Конечная наука здесь дошла до своей границы, так как
необходимость и опосредствование синтетического тут основываются уже не
только на {\em положительном}, но и на {\em отрицательном тождестве}.

Если геометрия, равно как и алгебра, имея дело со своими
абстрактными, чисто рассудочными предметами, скоро натыкается на свою
границу, то для {\em других наук}
синтетический метод оказывается с самого начала еще более
неудовлетворительным, а всего неудовлетворительнее он оказывается в
применении к философии. Относительно дефиниций и подразделений необходимые
замечания были уже сделаны выше; здесь следовало бы еще сказать лишь о
теоремах и доказательствах; но помимо предпосылания дефиниций и
подразделений, которое уже требует доказательства и предполагает его, само
вообще {\em местоположение}
их по отношению к теоремам неудовлетворительно. Это
местоположение особенно достойно внимания в опытных науках (как, например,
в физике), когда они хотят придать себе форму синтетических наук. Их путь в
этом случае состоит в том, что
{\em рефлексивные определения}
об особых {\em силах}
или о каких-нибудь других внутренних и имеющих характер
сущностей формах, проистекающих из того способа, каким анализирует опыт, и
могущих оправдать себя лишь как
{\em результаты}, по
необходимости {\em ставятся во главе},
чтобы иметь в их лице ту всеобщую основу, которую затем
{\em применяют} к
{\em единичному} и
отыскивают в последнем. Так как эти всеобщие основы сами по себе на имеют
никакой опоры, то говорят, что их пока следует
{\em допустить;} но по
выведенным {\em следствиям}
мы начинаем замечать, что последние составляют,
собственно говоря,
{\em основание} тех
{\em основ}. Так
называемое {\em объяснение}
и доказательство вводимого в теоремы конкретного материала
оказывается отчасти тавтологией, отчасти запутыванием истинного положения
вещей; отчасти же оказывается, что это запутывание служило к тому, чтобы
прикрыть обман познания, которое односторонне набрало опыты, благодаря чему
оно только и могло получить свои простые дефиниции и основоположения, и
которое устраняет возражение из опыта тем, что подвергает опыт рассмотрению
и признает значимым не в его конкретной целокупности, а как пример, и
притом с той его стороны, которая благоприятна для этих гипотез и теорий.
В~этом подчинении конкретного опыта определениям, принятым в качестве
предпосылки, основа теории затемняется и показывается лишь с той стороны,
которая согласуется с теорией, равно как и вообще этим очень затрудняется
непредубежденное рассмотрение конкретных восприятий самих по себе. Только в
том случае, если мы радикально перевернем весь ход этого познания, целое
получит правильное отношение, в котором можно обозреть связь между
основанием и следствием и правильность преобразования восприятия в мысли.
Одна из главных трудностей при изучении таких наук состоит поэтому в том,
чтобы {\em войти в них;}
а это может совершиться только тем путем, что мы
{\em слепо допускаем}
предпосылки и, не будучи в состоянии составить себе о них
далее понятие и часто даже "--- определенное представление, а
лишь имея возможность в лучшем случае создать себе о них смутный образ
фантазии, запечатлеваем покамест в памяти определения о допущенных силах,
материях и их гипотетических образованиях, направлениях и вращениях. Если
мы будем требовать в качестве условия допущения и признания предпосылок,
чтобы нам показали их необходимость и их понятие, то мы не подвинемся
дальше начала.

О неуместности применения синтетического метода к строго
аналитической науке мы уже имели случай говорить выше.
{\em Вольф} распространил
это применение на всевозможные виды сведений, отнесенных им к философии и
математике, "--- сведений, которые отчасти имеют всецело
аналитическую природу, отчасти же носят случайный и чисто технический
характер. Контраст между таким материалом, легко понятным, не способным по
своей природе сделаться предметом строго научной трактовки, и неловкими
научными окольными путями и оболочками сам по себе показал
несообразность такого применения и подорвал доверие к
нему\footnote{Например, в <<{\em Основоначалах зодчества}>>
(<<Anfangs\-gründen der Bau\-kunst>>) {\em Вольфа восьмая теорема гласит}:
<<Окно должно иметь такую ширину, чтобы два лица
могли удобно лежать на нем рядом. {\em Доказательство}:
Ибо часто ложатся на окно рядом с другим лицом, чтобы
осмотреться кругом. Так как архитектор должен во всем удовлетворить главным
намерениям хозяина строения (\S~1), то он должен и окно сделать настолько
широким, чтобы два лица могли удобно лежать на нем рядом, что и требовалось
доказать>>. Его же <<{\em Основоначала фортификации}>>
(<<Anfangs\-grün\-den der For\-tifika\-tion>>), {\em вторая теорема}:
<<Если неприятель расположится поблизости и предполагается,
что он будет пытаться освободить крепость от осады посредством подачи ей
помощи, то вся крепость должна быть обведена циркумваллационной
линией\pagenote{Циркумвалляционной линией в старой фортификации называлась
блокадная линия, имевшая своим назначением не допустить противника прийти на
выручку осаждаемой неприятельской крепости (от латинского circum\-vallare "---
окружать, обносить валом).\label{bkm:bm116}}. {\em Доказательство}.
Циркумваллационные линии не позволяют кому бы то ни было
проникать в лагерь извне (\S~311). Но те, которые хотят освободить крепость
от осады, желают проникнуть в лагерь извне. Следовательно, если хотят не
допустить их, то нужно создать вокруг лагеря циркумваллационную линию.
Поэтому, если неприятель располагается поблизости и предполагается, что он
будет пытаться освободить крепость от осады посредством оказания ей помощи,
то лагерь должен быть обведен циркумваллационной линией, что и требовалось
доказать>>.}. Но веры в пригодность и существенность этого
метода для придания философии научной строгости указанное злоупотребление
не могло лишить; пример {\em Спинозы}
в изложении его
{\em философии} еще долго
считался образцом для подражания. Но на самом деле
{\em Кантом} и
{\em Якоби} был
ниспровергнут весь способ философствования прежней метафизики, а стало
быть, и ее метод. Кант по-своему показал относительно содержания этой
метафизики, что оно путем строгого доказательства приводит к
{\em антиномиям},
характер которых во всем остальном был освещен нами в
соответствующих местах; но о самой природе этого способа доказательства,
связанного с некоторым, конечным содержанием, он не размышлял; однако одно
должно падать вместе с другим. В~своих <<Основоначалах
естествознания>>\pagenote{Полное заглавие: <<Метафизические основоначала
естествознания>> (<<Meta\-physi\-sche Anfangs\-gründe
der Natur\-wissen\-schaft>>). Вышли в 1786~г.\label{bkm:bm117}}
он сам дал пример разработки в виде рефлексивной науки и по
ее методу некоторой такой науки, которую он этим путем рассчитывал вернуть
в лоно философии. "--- Если Кант атаковал прежнюю метафизику
больше со стороны ее содержания, то Якоби атаковал ее преимущественно со
стороны ее способа доказательства и самым ясным и глубоким образом выделил
тот пункт, в котором вся суть, а именно, показал, что такой метод
доказательства всецело вынужден оставаться в кругу оцепенелой
необходимости, характеризующей конечное, и что {\em свобода}
(т.~е. {\em понятие} и, стало быть, {\em все истинно-сущее})
лежит по ту сторону этого способа доказательства и
недостижима для него. "--- Согласно тому выводу, к которому
пришел Кант, метафизику приводит к противоречиям именно ее своеобразная
материя, и неудовлетворительность [человеческого] познания состоит в его
{\em субъективности;} согласно же выводу Якоби к этим противоречиям приводит
метод и вся природа самого этого познания, которое схватывает лишь некоторую
{\em связь обусловленности} и {\em зависимости}
и поэтому оказывается несоответствующим тому, что есть в~себе
и для себя и абсолютно истинно. И~в самом деле, так как принципом философии
является {\em бесконечное свободное
понятие} и все ее содержание покоится исключительно только
на нем, то метод чуждой понятию конечности неуместен в применении к этому,
содержанию. Синтез и опосредствование, характеризующие этот метод,
{\em доказывание} не
приводит ни к чему б\'{о}льшему, кроме как к противостоящей свободе
{\em необходимости}, к
некоторому {\em тождеству}
зависимого, каковое тождество есть лишь
{\em в~себе}, все равно,
берется ли оно как {\em внутреннее}
или как {\em внешнее;}
в этом тождестве то, что составляет в нем реальность, "---
различенное и вступившее в существование, "---
безоговорочно остается некоторым
{\em самостоятельно-разным}
и потому {\em конечным}.
Здесь, следовательно, само это
{\em тождество} не
достигает {\em существования}
и остается чем-то
{\em лишь внутренним},
или же оно есть лишь нечто
{\em внешнее}, поскольку
его определенное содержание ему дано; и по тому, и по другому взгляду оно
есть нечто абстрактное, не имеет в самом себе реальной стороны и не
положено как в~себе и для себя
{\em определенное тождество;}
{\em понятие}, которое
единственно важно и которое есть в~себе и для себя бесконечное, исключено,
стало быть, из этого познания.

В синтетическом познании идея, следовательно, достигает своей
цели лишь настолько, что {\em понятие}
по своим {\em моментам
тождества} и
{\em реальным определениям},
или, иначе сказать,
{\em по всеобщности} и
{\em особенным}
различиям, а затем также и
{\em как тождество},
представляющее собой
{\em связь} и
{\em зависимость}
разного, становится
{\em для понятия}. Но
этот его предмет не соответственен ему; ибо понятие не становится
{\em единством себя с самим собой в
своем предмете или в своей реальности;} в необходимости его
тождество имеет бытие для него, но в этом тождестве сама необходимость не
есть его {\em определенность},
а выступает как некоторый внешний ему, т.~е. не понятием
определенный материал, в котором понятие, стало быть, не познает самого
себя. Следовательно, понятие не есть вообще для себя, оно по своему
единству не определено в~себе и для себя. Поэтому из-за несоответствия
предмета субъективному понятию идея все еще не достигает
истины в этом познании. "--- Но сфера необходимости есть
высочайшая вершина бытия и рефлексии; она сама по себе переходит в свободу
понятия, внутреннее тождество переходит в свое проявление, которое есть
понятие как понятие. Каким образом совершается
{\em в~себе} этот
{\em переход} из сферы
необходимости в понятие, мы показали при рассмотрении необходимости, и он
также явил себя в начале этой книги как
{\em генезис понятия}.
Здесь {\em необходимость}
занимает другое положение, чем там: она есть
{\em реальность} или
{\em предмет} понятия,
равно как и то понятие, в которое она переходит, выступает теперь как
предмет понятия. Но сам переход есть тот же самый. Он также и здесь есть
пока что лишь {\em в~себе}
и еще лежит вне познания в нашей рефлексии, т.~е. он есть
сама его все еще внутренняя необходимость. Только результат есть для него.
Идея, поскольку понятие теперь есть
{\em для себя}
в-себе-и-для-себя-определенное понятие, есть
{\em практическая} идея,
{\em действование}.

\section[В. Идея добра]{В. Идея добра}

Так как понятие, которое есть предмет самого себя, определено
теперь в~себе и для себя, то субъект определен по отношению к себе как
{\em единичное}. Как
субъективное, понятие имеет опять-таки предпосылку о некотором в-себе-сущем
инобытии; оно есть {\em стремление}
реализовать себя, цель, которая хочет
{\em через себя самоё}
сообщить себе объективность в объективном мире и выполнить
себя. В~теоретической идее субъективное понятие, как
{\em всеобщее}, само по
себе {\em лишенное определений},
противостоит объективному миру, из которого оно берет себе
определенное содержание и наполнение. В~практической же идее это понятие
противостоит, как действительное, действительному. Но та уверенность в
самом себе, которой субъект обладает в своей
в-себе-и-для-себя-определенности, есть уверенность в свой действительности
и в {\em недействительности}
мира. Для субъекта ничтожно не только инобытие мира как
абстрактная всеобщность, но и его единичность и определения его
единичности. {\em Объективность}
здесь присвоил себе сам субъект; его определенность внутри
себя есть объективное, ибо он есть такая всеобщность, которая вместе с тем
также и всецело определена; напротив, то, что раньше было объективным
миром, есть лишь нечто положенное, такое нечто, которое
{\em непосредственно}
определено разным образом, но которое, так как оно определено
только непосредственно, лишено внутри себя единства понятия и само по себе
ничтожно.

Эта содержащаяся в понятии, равная ему и заключающая в~себе
требование единичной внешней действительности определенность
есть {\em добро}.
Оно выступает с достоинством чего-то абсолютного, ибо оно
есть тотальность понятия внутри себя, объективное, имеющее вместе с тем
форму свободного единства и субъективности. Эта идея выше идеи
рассмотренного нами познания, ибо она обладает достоинством не только
всеобщего, но также и безоговорочно действительного. "--- Она
есть {\em стремление},
поскольку это действительное еще субъективно, полагает само
себя, а не обладает вместе с тем формой непосредственной предпосылки; ее
стремление реализовать себя есть стремление дать себе не объективность, "---
последней она обладает в самой себе, "--- а лишь
эту пустую форму непосредственности. "--- Деятельность цели
направлена поэтому не на себя, чтобы принять в~себя некоторое данное
определение и усвоить его себе, а скорее направлена к тому, чтобы положить
свое собственное определение и посредством снятия определений внешнего мира
дать себе реальность в форме внешней действительности. "---
Идея воли, как то, что определяет само себя, обладает
{\em для себя содержанием}
внутри самой себя. Правда, это содержание есть
{\em определенное}
содержание и постольку нечто
{\em конечное} и
{\em ограниченное;}
самоопределение есть по существу
{\em обособление}, так
как рефлексия воли в~себя, как отрицательное единство вообще, есть также и
единичность в смысле исключения и пред-полагания чего-то
иного. Однако особенность содержания ближайшим образом бесконечна через
форму понятия, собственную определенность которого оно составляет и которое
обладает в нем отрицательным тождеством себя самого с самим собой и,
следовательно, не только некоторым особенным, но и своей бесконечной
единичностью. Упомянутая
{\em конечность}
содержания в практической идее есть поэтому одно и то же с
тем обстоятельством, что она пока что еще есть невыполненная идея. Понятие
есть {\em для него}
в-себе-и-для-себя-сущее; оно есть здесь идея в форме сущей
{\em для самой себя}
объективности; с одной стороны, субъективное поэтому уже
больше не есть лишь некоторое
{\em положенное},
произвольное или случайное, а есть нечто абсолютное; но, с
другой стороны, эта {\em форма
существования (для-себя-бытие)} еще не обладает также и
формой {\em в-себе-бытия}.
То, что, таким образом, по форме как таковой выступает как
противоположность, выступает в рефлектированной в виде
{\em простого тождества}
форме понятия, т.~е. в содержании, как его простая
определенность. В~силу этого добро, хотя оно и значимо в~себе и для себя,
есть какая-нибудь особенная цель, которая, однако, не получает впервые
своей истинности через реализацию, а уже сама по себе есть нечто
истинное.

Самое умозаключение непосредственной
{\em реализации} не
требует здесь более подробного изложения; оно всецело есть лишь
вышерассмотренное умозаключение
{\em внешней целесообразности;}
лишь содержание составляет различие. Во внешней, как
формальной, целесообразности содержание было некоторым неопределенным
конечным содержанием вообще, здесь же оно, хотя также есть некоторое
конечное содержание, но, как таковое, оно есть вместе с тем абсолютно
значимое содержание. Но по отношению к заключению, к выполненной цели,
появляется дальнейшее различие. Конечная цель в своей
{\em реализации} доходит
вместе с тем лишь до {\em средства;}
так как она в своей начальной стадии не есть уже в~себе и для
себя определенная цель, то она, также и в качестве выполненной цели, все
еще остается чем-то таким, что не есть в~себе и для себя. Если добро
опять-таки также фиксировано как нечто
{\em конечное} и
существенно и есть таковое, то оно тоже, несмотря на свою внутреннюю
бесконечность, не может избегнуть судьбы конечного, судьбы, выступающей во
многих формах. Выполненное добро есть добро в силу того, чем оно было уже в
субъективной цели, в своей идее; выполнение сообщает ему внешнее
существование; но так как это существование определено только как сама по
себе ничтожная внешность, то добро обладает в ней лишь случайным,
разрушимым существованием и не достигает соответствующего его идее
выполнения. "--- Далее, так как оно есть по своему содержанию
ограниченное добро, то существует также многоразличное добро; существующее
добро подвержено разрушению не только через внешнюю случайность и через
зло, но и через коллизию и столкновение в сфере самого добра. Со стороны
пред-положенного ему объективного мира, в пред-положении которого состоит
субъективность и конечность добра и который как нечто иное идет своим
собственным путем, само выполнение добра встречает помехи и даже натыкается
на невозможность. Таким образом, добро остается некоторым
{\em долженствованием;}
оно есть {\em в~себе и для
себя;} но {\em бытие}
как последняя, абстрактная непосредственность остается по
отношению к нему определенным {\em также
и} как некоторое
{\em небытие}. Идея
завершенного добра есть, правда,
{\em абсолютный постулат},
но не больше, как постулат, т.~е. абсолютное, обремененное
определенностью субъективности. Два мира еще антагонистичны: один мир есть
царство субъективности в чистых пространствах прозрачной мысли, другой мир
"--- царство объективности в стихии некоторой внешней
многообразной действительности, которая есть неотомкнутое, нераскрытое
царство тьмы. Полное развитие неразрешенного противоречия,
образуемого указанной {\em абсолютной}
целью, которой
{\em непреодолимо}
противостоит {\em барьер}
этой действительности, рассмотрено нами ближе в
<<Феноменологии духа>>, стр.~548 и
сл.\pagenote{Гегель указывает страницы по
1-му немецкому изданию <<Феноменологии духа>> (1807). В~1-м издании II тома
Собрания сочинений Гегеля (1832~г.) указанное Гегелем место находится на
страницах 451---475, в издании Лассона
(1921~г.) "--- на страницах 388---408. В~русском переводе
<<Феноменологии духа>> (Спб. 1913)
этому соответствуют страницы 272---287. В~этом отделе
<<Феноменологии>> Гегель рассматривает <<моральное мировоззрение>>, имея в виду
моральную философию Канта и Фихте. При этом Гегель подвергает ее
обстоятельной имманентной критике, язвительно вскрывая таящиеся в ней
противоречия.\label{bkm:bm118}}.
"--- Так как идея теперь содержит в~себе момент совершенной
определенности, то другое понятие, к которому понятие относится в ней,
обладает в своей субъективности вместе с тем и моментом некоторого объекта;
поэтому идея приобретает здесь вид
{\em само-сознания}
и совпадает, с этой одной стороны, с изображением
последнего.

Но чего еще недостает практической идее, это момента самого
сознания в собственном
смысле\pagenote{Говоря здесь о <<сознании в
собственном смысле>>, Гегель имеет в виду то предметное (направленное на
внешние предметы) сознание, которое он рассматривает в первом отделе своей
<<Феноменологии духа>>. Подобным же образом то <<самосознание>>, о котором
упоминает предыдущая фраза текста, соответствует второму отделу
<<Феноменологии>>, трактующему о самосознании, предметом которого является,
во-первых, оно само, а во-вторых, объекты чувственной достоверности и
чувственного восприятия.\label{bkm:bm119}},
недостает именно того, чтобы момент действительности в
понятии достиг сам по себе определения
{\em внешнего бытия}. "---
Этот недостаток может рассматриваться и так, что
{\em практической} идее
еще недостает момента
{\em теоретической} идеи.
А именно, в последней на стороне субъективного понятия, созерцаемого
понятием внутри себя, стоит лишь определение
{\em всеобщности;}
познание знает себя лишь как усвоение, как само по себе
{\em неопределенное}
тождество понятия с самим собой; наполнение, т.~е. в~себе и
для себя определенная объективность, есть для него нечто
{\em данное}, и
{\em истинно-сущим}
признается наличная, независимо от субъективного полагания,
действительность. Наоборот, практическая идея считает эту действительность
(которая вместе с тем противостоит ей как непреодолимый предел) тем, что
само по себе ничтожно и что должно получить свое истинное определение и
единственную ценность лишь через цели добра. Воля поэтому лишь сама
преграждает себе путь к достижению своей цели тем, что она отделяет себя от
познания и что внешняя действительность не получает для нее формы
истинно-сущего; идея добра может поэтому найти свое дополнение единственно
только в идее истины.

Но она совершает этот переход через самое себя.
В~умозаключении действования первой посылкой служит
{\em непосредственное соотношение доброй
цели с той действительностью}, которой эта цель овладевает и
которую она во второй посылке направляет, как внешнее {\em средство}, против
внешней действительности. Добро есть для субъективного понятия нечто
объективное; действительность в ее существовании противостоит добру как
непреодолимый предел лишь постольку, поскольку она еще имеет определение
{\em непосредственного существования},
а не чего-то объективного в смысле в-себе-и-для-себя-бытия;
она скорее есть либо нравственное зло, либо нечто безразличное, лишь
поддающееся определению, имеющее свою ценность не в~себе самом. Но это
абстрактное бытие, противостоящее добру во второй посылке,
уже снято самой практической идеей; первой посылкой ее действования служит
{\em непосредственная объективность}
понятия, согласно которой цель сообщает себя действительности
без всякого сопротивления со стороны последней и находится в простом,
тождественном соотношении с этой действительностью. Постольку нам нужно,
следовательно, лишь свести воедино мысли ее двух посылок. К~тому, что в
первой посылке непосредственно уже совершено объективным понятием,
присоединяется во второй посылке ближайшим образом лишь то, что оно
полагается через опосредствование и, стало быть,
{\em для понятия}. И~вот,
подобно тому как в целевом соотношении вообще выполненная цель есть,
правда, опять-таки лишь средство, но зато и, наоборот, средство есть также
и выполненная цель, так и в умозаключении добра вторая посылка
непосредственно уже имеется {\em в
себе} в первой посылке; только этой непосредственности здесь
недостаточно, и вторая посылка уже постулируется для первой: выполнение
добра наперекор противостоящей ему, другой по отношению к нему
действительности есть то опосредствование, которое существенным образом
необходимо для непосредственного соотношения и действительного
осуществления добра. Ибо это есть лишь первое отрицание или инобытие
понятия, такая объективность, которая была бы погруженностью понятия во
внешность; второе отрицание есть снятие этого инобытия, благодаря чему
непосредственное выполнение цели впервые становится действительностью добра
как для-себя-сущего понятия, поскольку последнее полагается здесь
тождественным с самим собой, а не с чем-то иным, и, стало быть,
полагается единственно свободным. И~вот, если бы цель добра этим все-таки
не была бы выполнена, то это было бы рецидивом, возвращением понятия к той
точке зрения, на которой понятие стояло до своей деятельности, "---
к точке зрения, с которой действительность определена как
ничтожная и все же пред-положена как реальная; этот рецидив становится
прогрессом в дурную бесконечность и имеет свое основание единственно только
в том обстоятельстве, что при снятии указанной абстрактной реальности это
снятие столь же непосредственно забывается, или же забывается, что эта
реальность, наоборот, уже предположена как сама по себе ничтожная, не
объективная действительность. Это повторение предположения о невыполненной
цели после действительного выполнения цели определяет себя поэтому также и
следующим образом: {\em субъективная
позиция} объективного понятия воспроизводится и
увековечивается, вследствие чего
{\em конечность} добра
как по его содержанию, так и по его форме представляется
пребывающей истиной, равно как и его осуществление безоговорочно
представляется всегда лишь некоторым
{\em единичным}, а не
{\em всеобщим} актом. "---
На самом же деле эта определенность сняла себя в
осуществлении добра; что все еще
{\em ограничивает}
объективное понятие, это его собственное
{\em воззрение} на себя,
исчезающее благодаря его размышлению о том, что такое
{\em в~себе} его
осуществление; этим воззрением понятие лишь само преграждает себе путь и
должно по этому поводу направиться не против некоторой внешней
действительности, а против~себя самого.

А именно, деятельность во второй посылке, производящая лишь
некоторое одностороннее
{\em для-себя-бытие},
вследствие чего продукт представляется чем-то
{\em субъективным} и
{\em единичным},
деятельность, в которой поэтому повторяется первое
предположение, "--- эта деятельность есть на самом деле в такой
же мере и полагание {\em в-себе-сущего}
тождества объективного понятия и непосредственной
действительности. Согласно пред-положению эта действительность
обладает лишь реальностью явления, сама по себе ничтожна и безоговорочно
определима объективным понятием. Так как через деятельность объективного
понятия внешняя действительность изменяется и ее определение тем самым
снимается, то именно этим ее лишают характера исключительно только
являющейся реальности, внешней определимости и ничтожности, и она, стало
быть, {\em полагается}
как в-себе-и-для-себя-сущая. Этим вообще снимается указанное
пред-положение, а именно, снимается определение добра как некоторой чисто
субъективной и по своему содержанию ограниченной цели, необходимость еще
только реализовать ее через субъективную деятельность и сама эта
деятельность. В~результате опосредствование снимает само себя; результат
есть некоторая
{\em непосредственность},
которая есть не восстановление пред-положения, а, наоборот,
его снятость. Идея в-себе-и-для-себя-определенного понятия тем самым теперь
уже положена не только в деятельном субъекте, но также и как некоторая
непосредственная действительность, и, обратно, последняя, как она есть в
познании, положена так, что она есть истинно-сущая объективность.
Единичность субъекта, которою он был обременен через свое предположение,
исчезла вместе с исчезновением последнего; субъект, стало быть, выступает
теперь как {\em свободное, всеобщее
тождество с самим собой}, для которого объективность понятия
есть в такой же мере {\em данная}
(непосредственно
{\em имеющаяся} для
субъекта), в какой он знает себя как в-себе-и-для-себя-определенное
понятие. В~этом результате
{\em познание}, стало
быть, восстановлено и соединено с практической идеей,
преднайденная действительность определена вместе с тем как выполненная
абсолютная цель, но не как в ищущем познании, только как объективный мир,
лишенный субъективности понятия, а как такой объективный мир, внутреннее
основание и действительное устойчивое существование которого есть понятие.
Это "--- абсолютная идея.

\chapter[Третья глава Абсолютная идея]{Третья глава\newline
Абсолютная идея\pagenote{Об <<абсолютной идее>> как завершении
<<Логики>> Гегеля мы имеем высказывания Энгельса и Ленина, на
первый взгляд кажущиеся прямо противоположными, а в действительности
прекрасно дополняющие друг друга. Энгельс указывает, что конечная точка
<<Логики>>, абсолютная идея, <<абсолютна лишь постольку, поскольку Гегель
абсолютно ничего не может сказать о ней>> ({\em Engels}, L.~Feuerbach,
Moskau, 1932, S.~19). Ленин пишет: <<Замечательно,
что вся глава об <<абсолютной идее>> почти ни словечка не говорит о боге
(едва ли не один раз случайно вылезло <<божеское>> <<понятие>>) и кроме того
"--- {\em это} NB "--- почти не содержит специфически {\em идеализма},
а главным своим предметом имеет {\em диалектический метод}>>
({\em Ленин},
Философские тетради, М.~1936, стр.~225). Эти две оценки, как
сказано, прекрасно дополняют друг друга. Дело в том, что об абсолютной идее
{\em как таковой}, т.~е. как о каком-то {\em особом}
абсолютном содержании, Гегель, действительно, не в состоянии
ничего сказать, поскольку содержанием этой абсолютной идеи оказывается не
что иное, как {\em весь процесс развертывания логических категорий},
который служил предметом рассмотрения всех предыдущих глав
<<Логики>>. В~главе об абсолютной идее Гегель бросает, как он сам выражается,
<<{\em ретроспективный взгляд}>> (см. {\em Гегель},
Соч., т.~I, стр.~341) на весь этот уже рассмотренный им
процесс и формулирует общие черты, характеризующие
{\em форму} этого процесса, "--- иначе говоря, формулирует основные черты
{\em диалектического метода}, представляющего собою, по
выражению Гегеля, <<имманентную душу самого содержания>> (см. т.~I <<Науки
логики>>, стр.~10). Поэтому и получается, что глава об абсолютной идее
имеет, как указывает Ленин, <<своим главным предметом {\em диалектический
метод}>>. В~предисловии к
<<Феноменологии духа>> Гегель выставил положение о том, что <<истина
"--- это целое>> (см. примечание \ref{bkm:bm01}),
причем это целое понимается у Гегеля как {\em процесс}.
Последняя глава <<Науки логики>> и подводит итоги этому
процессу, бросая общий взгляд назад на весь пройденный путь и формулируя
основные принципы диалектического движения понятий, являющегося, по Гегелю,
вместе с тем диалектическим движением самой действительности.\label{bkm:bm120}}}

Абсолютная идея, какой она получилась здесь, есть тождество
теоретической и практической идеи, каждая из которых, взятая отдельно, еще
одностороння и имеет внутри себя самоё идею лишь как некоторое искомое
потустороннее и недостигнутую цель; каждая из них есть поэтому некоторый
{\em синтез устремления},
имеет внутри себя идею, равно как и
{\em не}~имеет ее,
переходит от одной мысли к другой, но не сводит воедино этих двух мыслей, а
застревает в их противоречии. Абсолютная идея, как разумное понятие,
которое в своей реальности лишь сливается с самим собой, есть в силу этой
непосредственности его объективного тождества, с одной стороны, возврат к
жизни; но она равным образом и сняла эту форму своей непосредственности и
имеет внутри себя наивысшую противоположность. Понятие есть не только
{\em душа}, но и
свободное субъективное понятие, которое есть для себя и поэтому обладает
{\em личностью}, "--- есть
практическое, в~себе и для себя определенное, объективное понятие, которое
в качестве лица есть непроницаемая, атомическая субъективность, но которое
вместе с тем не есть исключающая единичность, а есть для себя
{\em всеобщность} и
{\em познание} и имеет в
своём ином предметом {\em свою
собственную} объективность. Все остальное есть заблуждение,
смутность, мнение, стремление, произвол и преходимость; единственно только
абсолютная идея есть {\em бытие},
непреходящая {\em жизнь,
знающая себя истина} и представляет собой
{\em всю истину}.

Она есть единственный предмет и содержание философии. Так как
она содержит в~себе {\em всяческую
определенность} и ее сущность состоит в возвращении к себе
через свое самоопределение или обособление, то она имеет различные
формации, и задача философии заключается в том, чтобы познать ее в этих
различных формациях. Природа и дух суть вообще различные способы
изображения {\em ее существования},
искусство и религия суть ее разные способы постигать себя и
давать себе соответствующее существование; философия имеет одинаковое
содержание и одинаковую цель с искусством и религией, но она
есть наивысший способ постижения абсолютной идеи, потому что ее способ есть
наивысший, "--- понятие. Она поэтому объемлет собой упомянутые
выше формации реальной и идеальной (ideellen) конечности,
равно как и формации бесконечности и святости, и постигает в понятиях их и
самоё себя. Выведение и познание этих особенных способов есть дальнейшее
дело особенных философских наук. Логический аспект (das Logische)
абсолютной идеи может тоже быть назван одним из
{\em способов} ее
изображения. Но если <<способ>> обозначает некоторый
{\em особенный} вид,
некоторую {\em определенность}
формы, то логическое изображение, напротив, есть всеобщий
способ раскрытия, в котором все особенные способы сняты и завернуты.
Логическая идея есть сама идея в ее чистой сущности, как она (идея) в
простом тождестве заключена в свое понятие и еще не начала
{\em светиться} в
какой-нибудь определенности формы. Логика поэтому изображает самодвижение
абсолютной идеи лишь как первоначальное
{\em слово}, которое есть
{\em внешнее высказывание},
но такое, которое, как внешнее, непосредственно снова
исчезло, в то время как она (идея) есть; идея, следовательно, имеет здесь
характер лишь этого самоопределения к тому, чтобы
{\em внимать себе;} она
пребывает в сфере {\em чистой мысли},
в которой различие еще не есть
{\em инобытие}, а
остается совершенно прозрачным для себя. "--- Логическая идея,
стало быть, имеет своим содержанием себя как
{\em бесконечную форму}, "---
{\em форму}, составляющую
противоположность к {\em содержанию}
постольку, поскольку последнее есть
{\em таким} образом
ушедшее в~себя и снятое в тождестве определение формы, что это конкретное
тождество противостоит тождеству, развитому как форма; содержание имеет вид
чего-то другого и данного по отношению к форме, которая как таковая всецело
находится в {\em соотношении}
и определенность которой вместе с тем положена как
видимость. "--- Сама абсолютная идея имеет, далее, своим
содержанием лишь то, что определение формы есть ее собственная завершенная
тотальность, чистое понятие. И~вот
{\em определенность} идеи
и весь ход развертывания этой определенности составил предмет логической
науки, из какового хода развертывания сама абсолютная идея произошла
{\em для себя;} для себя
же она явила себя таковой, что определенность не имеет вида некоторого
{\em содержания}, а
безоговорочно выступает как
{\em форма}, и что идея
согласно этому выступает как безоговорочно
{\em всеобщая идея}.
Следовательно, то, что нам предстоит здесь еще рассмотреть,
есть не некоторое содержание как таковое, а всеобщая черта его формы, "---
т.~е. {\em метод}.

{\em Метод} может ближайшим
образом представляться только {\em видом
и способом} познавания, и он в самом деле имеет природу
такового. Но вид и способ как метод есть не только некоторая
{\em в~себе и для себя определенная}
модальность {\em бытия},
но в качестве модальности познания положен как определенный
{\em понятием} и как
форма, поскольку она есть душа всякой объективности и поскольку всякое иным
образом определенное содержание имеет свою истину единственно только в
форме. Если содержание берется как опять-таки данное методу и как
обладающее своеобразной природой, то метод, как и логическое вообще, есть в
таком определении некоторая только
{\em внешняя} форма. Но
не только можно против такого понимания сослаться на основное понятие
логического, но и весь ход развертывания последнего, в котором перед нами
проходили все образы некоторого данного содержания и объектов, показал их
переход и неистинность, и вместо того, чтобы можно было считать, что
некоторый данный объект есть основа, по отношению к которой абсолютная
форма занимает положение только внешнего и случайного определения,
последняя, напротив, оказалась абсолютной основой и окончательной истиной.
Таким образом, метод возник отсюда как
{\em само себя знающее понятие, имеющее
своим предметом себя} как столь же субъективное, сколь и
объективное абсолютное и, стало быть, как чистое соответствие понятия и его
реальности, как некоторое существование, которое есть само же понятие.

Стало быть, мы здесь должны рассмотреть в качестве метода лишь
движение самого {\em понятия},
природа которого (движения) уже познана нами, но,
{\em во-первых}, мы
теперь должны рассмотреть его, имея в виду то его
{\em значение}, что
{\em понятие} есть
{\em все} и что его
движение есть {\em всеобщая абсолютная
деятельность}, самоопределяющееся и самореализующееся
движение. Метод должен быть поэтому признан не терпящим ограничения,
всеобщим, внутренним и внешним способом и безоговорочно бесконечной силой,
которой никакой объект, поскольку он выступает как нечто внешнее, далекое
для разума и независимое от него, не может оказывать сопротивление, не
может иметь другой природы по отношению к ней и не быть проникнут ею. Метод
есть поэтому {\em душа} и {\em субстанция}, и о любом предмете мы должны
сказать, что мы его постигаем в понятии и знаем его в его истине только
постольку, поскольку он {\em полностью подчинен методу;} он есть собственный
метод всякой вещи, так как его деятельность есть понятие. Это есть также более
истинный смысл его (метода) {\em всеобщности;} согласно всеобщности рефлексии
его понимают только как метод для {\em всего;} согласно же всеобщности идеи он
есть столь же вид и способ познания, {\em субъективно} знающего себя понятия,
сколь и {\em объективный} вид и способ (или, вернее, {\em субстанциальность})
{\em вещей}, т.~е. понятий, поскольку последние кажутся {\em представлению}
и {\em рефлексии} прежде всего {\em иными}. Метод есть поэтому не только
наивысшая {\em сила} или, вернее, {\em единственная} и абсолютная {\em сила}
разума, но также наивысшее и единственное его {\em влечение} обрести и познать
{\em себя самого во всем через самого себя~}\pagenote{Это место в сокращенном
виде цитируется у Маркса в <<Нищете философии>>, после чего Маркс замечает:
<<Итак, что же такое абсолютный метод? Абстракция движения. Что такое
абстракция движения? Движение в абстрактном виде. Что такое движение
в абстрактном виде? Чисто логическая формула движения, или движение чистого
разума. В~чем состоит движение чистого разума? В том, что он полагает себя,
противополагает себя самому себе и слагается с самим собою, в том, что он
формулируется в тезис, антитезис и синтезис, или, наконец, в том, что он
себя утверждает, отрицает и отрицает свое отрицание>>. И~т.~д. (См.
{\em Маркс}, Нищета философии, Партиздат, 1937, стр.~76---77.) В своей
полемике против Прудона Маркс касается главным образом внешней стороны
гегелевского <<абсолютного метода>>, поскольку Прудон пытался использовать
(притом весьма неудачно) именно эту внешнюю сторону идеалистической
диалектики Гегеля.\label{bkm:bm121}}. "--- Этим, {\em во-вторых},
указано также и {\em отличие метода от понятия как
такового}, т.~е. указана особенная черта метода. Понятие,
как оно нами рассматривалось само по себе, выступало в своей
непосредственности; {\em рефлексия} или {\em рассматривающее
его понятие }имело место в {\em нашем} знании. Метод
есть само это знание, для которого понятие имеет бытие не только как
предмет, но и как его собственное, субъективное действование, как
{\em орудие} и средство
познающей деятельности, отличное от нее, но как ее собственная сущность
(Wesenheit). В~ищущем познании метод тоже поставлен как {\em орудие}, как
некоторое стоящее на субъективной стороне средство, через которое она
соотносится с объектом. Субъект есть в этом умозаключении один крайний
термин, а объект "--- другой, и первый смыкается через свой
метод со вторым, но этим не смыкается для себя {\em с собой самим}.
Крайние термины остаются разными, так как субъект, метод и
объект не положены как {\em единое
тождественное понятие;} умозаключение есть поэтому всегда
формальное умозаключение; та посылка, в которой субъект полагает форму на
свою сторону как свой метод, есть некоторое {\em непосредственное}
определение и содержит в~себе поэтому, как мы видели, определения
формы "--- дефиниции, деления и~т.~д. "--- как {\em преднайденные в
субъекте} факты. Напротив, в истинном познании метод не есть
только множество известных определений, но есть
в-себе-и-для-себя-определенность понятия, представляющего собой средний
термин только потому, что оно в такой же мере имеет также и значение
объективного, которое поэтому в заключении не просто получает через метод
некоторую внешнюю определенность, но положено в своем тождестве с
субъективным понятием.

1. Стало быть, то, что составляет метод, это
"--- определения самого понятия и их соотношения, которые теперь
должны быть рассмотрены в их значении определений метода. "---
При этом мы должны начать, {\em во-первых}, с рассмотрения {\em начала}.
О нем уже говорилось в начале самой логики, равно как и выше,
при рассмотрении субъективного познания, и там мы показали, что если его не
делают произвольно и с категорической бессознательностью, то, хотя и может
казаться, что оно причиняет много затруднений, оно, однако,
имеет в высшей степени простую природу. Так как оно есть начало, то его
содержание есть нечто {\em непосредственное},
но такое непосредственное, которое имеет смысл и форму
{\em абстрактной всеобщности}.
Будет ли оно помимо этого некоторым содержанием из области
{\em бытия}, или {\em сущности}, или {\em понятия}, оно, все
равно, постольку есть нечто {\em взятое
со стороны, преднайденное, ассерторическое,} поскольку оно
есть нечто {\em непосредственное}. Но, {\em во-первых},
оно не есть непосредственное, доставляемое
{\em чувственным созерцанием} или {\em представлением},
а непосредственное, доставляемое {\em мышлением}, каковое
мышление вследствие его непосредственности можно также назвать
сверхчувственным, {\em внутренним
созерцанием}. Непосредственное, принадлежащее области
чувственного созерцания, есть нечто
{\em многообразное} и {\em единичное}. Но
познание есть мышление, постигающее в понятиях; его начало поэтому также
имеет место {\em только в стихии
мышления;} оно есть нечто {\em простое} и {\em всеобщее}. "--- Об этой
форме была речь выше при рассмотрении дефиниции. При начале конечного
познания всеобщность тоже признается существенным определением, но она
берется лишь как определение мышления и понятия в противоположность бытию.
На самом же деле эта {\em первая} всеобщность есть {\em непосредственная}
всеобщность и поэтому в такой же мере обладает также и значением {\em бытия;}
ибо бытие именно и есть это абстрактное соотношение с собой
самим. Бытие не нуждается ни в каком другом выведении, вроде того, что оно
будто бы входит в абстрактную дефиницию лишь потому, что оно было
заимствовано из чувственного созерцания или из какого-нибудь другого
источника, и входит лишь постольку, поскольку его показывают. Это
показывание и выведение касается такого {\em опосредствования},
которое представляет собой нечто большее, чем голое начало, и
оно есть такое опосредствование, которое не принадлежит области
мыслительного постижения в понятиях, а есть лишь возвышение представления,
эмпирического и рассуждающего сознания на точку зрения мышления. Согласно
обычному противоположению мысли или понятия и бытия кажется важной истиной,
что мысли, взятой сама по себе, еще не присуще бытие и что последнее имеет
некоторое собственное, от самой мысли независимое основание. Но простое
определение {\em бытия}
так скудно само по себе, что уже в силу этого с ним нечего
слишком много носиться. Всеобщее само есть непосредственно эта
непосредственность, так как оно, как абстрактное всеобщее, само тоже есть
лишь абстрактное соотношение с собой, которое и есть бытие. На самом же
деле требование выявить бытие имеет дальнейший внутренний
смысл, заключающий в~себе не просто это абстрактное определение: здесь
имеется в виду требование {\em реализации} вообще
{\em понятия}, которая в самом {\em начале} еще не
дана, а, наоборот, представляет собой последнюю цель и дело всего
дальнейшего развития познания. Далее, так как путем показывания во
внутреннем или внешнем восприятии {\em содержание} начала
должно получить свое оправдание и быть удостоверено как нечто истинное или
правильное, то этим имеется в виду уже не
{\em форма} всеобщности
как таковая, а ее {\em определенность},
о которой необходимо сейчас кое-что сказать. На первый взгляд
кажется, что удостоверение того {\em определенного содержания},
которым здесь начинают, лежит {\em позади} этого начала;
на самом же деле это удостоверение должно рассматриваться как движение
вперед, если только оно принадлежит к постигающему в понятиях познанию.

Начало, стало быть, имеет для метода только ту определенность, что оно
есть нечто простое и всеобщее; это как раз та самая {\em определенность},
вследствие которой оно недостаточно. Всеобщность есть чистое,
простое понятие, и метод как сознание этого понятия знает, что всеобщность
есть лишь момент и что понятие еще не определено в ней в~себе и для себя.
Однако, если бы это сознание желало развивать начало дальше только ради
самого метода, то последний был бы чем-то формальным, положенным во внешней
рефлексии. На так как метод есть объективная, имманентная форма, то
непосредственность начала должна быть недостаточной
{\em в самой себе} и наделенной {\em влечением}
вести себя дальше. Но всеобщее имеет в абсолютном методе
значимость не просто лишь абстрактно-всеобщего, а объективно-всеобщего,
т.~е. того, что {\em в~себе} есть {\em конкретная
тотальность}, но тотальность, которая еще не
{\em положена}, еще не есть {\em для себя}. Даже
абстрактно-всеобщее как таковое, рассматриваемое в понятии, т.~е. согласно
своей истине, есть не только {\em простое}, а, как {\em абстрактное}, оно
уже {\em положено} как обремененное некоторым {\em отрицанием}.
Поэтому-то и {\em нет}, будь это в {\em действительности} или в {\em мысли},
такого простого и такого абстрактного, как это обычно
представляют себе. Такое простое есть лишь некоторое {\em мнение}, имеющее
свое основание только в несознавании того, что на самом деле имеется
налицо. "--- Выше мы определили начало как нечто непосредственное;
{\em непосредственность всеобщего}
есть то же самое, что здесь обозначено как
{\em в-себе-бытие} без {\em для-себя-бытия}. "--- Можно поэтому
сказать, что всякое начало должно быть сделано с {\em абсолютного},
равно как и все дальнейшее движение вперед есть лишь его
изображение, поскольку {\em в-себе-сущее} есть
понятие. Но именно потому, что оно пока что есть только
{\em в~себе}, оно в такой же мере и {\em не}~есть
ни абсолютное, ни положенное понятие, ни идея; ибо последние именно и
означают, что {\em в-себе-бытие}
есть лишь абстрактный, односторонний момент. Поэтому движение
вперед не есть что-то вроде {\em излишества;} оно
было бы таковым, если бы начало уже было в действительности абсолютным;
движение вперед состоит, наоборот, в том, что всеобщее определяет само себя
и есть всеобщее {\em для себя},
т.~е. есть вместе с тем также и единичное и субъект. Лишь в
своем завершении оно есть абсолютное.

Можно напомнить о том, что начало, которое {\em в~себе} есть
конкретная тотальность, может как таковое быть также {\em свободным}, и его
непосредственность может обладать определением некоторого
{\em внешнего существования;}
{\em зародыш живого существа} и {\em субъективная цель}
вообще суть, как мы видели выше, такие начала; оба поэтому
сами суть {\em влечения}.
Напротив, не-духовное и неживое есть конкретное понятие лишь
как {\em реальная возможность;} {\em причина} есть
наивысшая ступень, в которой конкретное понятие как начало в сфере
необходимости обладает некоторым непосредственным существованием; но она
еще не есть субъект, который как таковой сохраняет себя также и в своей
действительной реализации. {\em Солнце}, например, и
вообще все неживое суть определенные существования, в которых реальная
возможность остается некоторой {\em внутренней}
тотальностью, так что моменты этой тотальности не
{\em положены}\pagenote{В~немецком тексте всех изданий
стоит: <<weder\ldots gesetzt sind>>. Так как этому <<weder>>
нигде не соответствует второе отрицание <<noch>>,
то вместо <<weder>> надо читать <<nicht>>.\label{bkm:bm122}}
в них в субъективной форме, и, поскольку они реализуются, они
получают существование через {\em другие} телесные индивидуумы.

2. Конкретная тотальность, образующая начало, имеет как
таковая в самой себе начало поступательного движения и развития. Она как
конкретное {\em различена внутри себя;}
однако вследствие ее {\em первой непосредственности}
первые различенные суть прежде всего {\em разные}. Но
непосредственное, как соотносящаяся с собой всеобщность, как субъект, есть
также и {\em единство} этих разных. "--- Эта рефлексия есть первая
ступень дальнейшего движения, "--- есть выступление
{\em различия}, {\em суждение (перводеление)}, {\em процесс определения}
вообще. Существенно то, что абсолютный метод находит и
познает {\em определение}
всеобщего в самом всеобщем. Рассудочное конечное познание
поступает при этом следующим образом: те черты конкретного, которые были
отброшены им при абстрагирующем порождении указанного всеобщего, оно теперь
столь же внешним образом снова вбирает в~себя. Абсолютный же метод,
напротив, ведет себя не как внешняя рефлексия, а берет определенное из
самого своего предмета, так как сам этот метод есть имманентный принцип и
душа предмета. "--- В этом и состояло то
требование, которое {\em Платон}
предъявлял к познанию:
{\em рассматривать вещи сами по себе},
отчасти "--- в их всеобщности, отчасти же, имея в
виду исключительно только их и осознавая то, что в них имманентно, а не
отклоняться от них, хватаясь за побочные обстоятельства, примеры и
сравнения. "--- Постольку метод абсолютного познания
{\em аналитичен}. То
обстоятельство, что он {\em находит}
дальнейшее определение своего начального всеобщего всецело
только в последнем, есть абсолютная объективность понятия, достоверность
которой он есть. "--- \label{bkm:bmpg224a}Но этот метод также и
{\em синтетичен}, так как
его предмет, определенный непосредственно как {\em простое всеобщее}, в
силу той определенности, которой он обладает в самой своей
непосредственности и всеобщности, являет себя как некоторое
{\em иное}. Это
соотнесение некоторого разного, которое (соотнесение) предмет, таким
образом, есть внутри себя, уже, однако, не есть то, что разумеют под
синтезом, когда говорят о конечном познании; уже вообще в силу того его
(предмета) в такой же мере и аналитического определения, что это
соотнесение есть соотнесение в {\em понятии}, оно
совершенно отличается от указанного синтеза.

Этот столь же синтетический, сколь и аналитический момент {\em суждения}
(перводеления), в силу которого первоначальное всеобщее
определяет себя из себя самого как
{\em иное по отношению} {\em к себе}, должен быть
назван {\em диалектическим}. {\em Диалектика} есть одна
из тех древних наук, которая больше всего игнорировалась в метафизике новых
философов, а затем вообще в популярной философии как античного, так и
нового времени. О~{\em Платоне}
Диоген Лаэрций говорит, что, подобно тому как Фалес был
создателем философии природы, Сократ "--- моральной философии,
так Платон был создателем третьей науки, входящей в состав философии,
{\em диалектики;} это,
значит, была та его заслуга, которую древние считали величайшей, но которая
часто оставляется совершенно без внимания теми, кто больше всего говорит о
нем. Часто рассматривали диалектику как некоторое {\em искусство}, как
будто она покоится на каком-то субъективном {\em таланте}, а не
принадлежит к объективности понятия. Какой вид она получила в кантовской
философии и какой вывод сделала из нее эта философия, это мы уже показали
выше на определенных примерах ее воззрений. Следует рассматривать как
бесконечно важный шаг то обстоятельство, что диалектика снова была признана
необходимой для разума, хотя надо сделать вывод, противоположный тому,
который был сделан из этого признания.

Помимо того что диалектика обычно представляется чем-то
случайным, она обыкновенно получает ту более детальную форму, что
относительно какого-нибудь предмета, например, относительно мира, движения,
точки и~т.~д., показывают, что ему присуще какое-нибудь определение,
например (в~порядке названных предметов), конечность в пространстве или
времени, нахождение в {\em этом}
месте, абсолютное отрицание пространства; но что, далее, ему
столь же необходимо присуще также и противоположное определение, например,
бесконечность в пространстве и времени, ненахождение в данном месте,
отношение к пространству и, следовательно, пространственность. Более
древняя элеатская школа применяла свою диалектику преимущественно против
движения; Платон часто применяет диалектику против представлений и понятий
своего времени, в особенности против воззрений софистов, но также и против
чистых категорий и определений рефлексии; развитый, позднейший скептицизм
распространил ее не только на непосредственные так называемые факты
сознания и максимы обыденной жизни, но также и на все научные понятия.
Выводом, который делают из такой диалектики, является вообще
{\em противоречивость} и {\em ничтожность}
выставленных утверждений. Но это может иметь место в двояком
смысле: либо в том объективном смысле, что {\em предмет}, который в
такой мере противоречив в самом себе, упраздняет себя и ничтожен (таков,
например, был вывод элеатов, согласно которому отрицалась {\em истинность},
например, мира, движения, точки); либо же в том субъективном
смысле, что {\em неудовлетворительным} является {\em познание}.
Под последним выводом в одних случаях понимают то, что лишь
сама эта диалектика проделывает, дескать, фокус, создающий такого рода
ложную видимость. Таков обычный взгляд так называемого здравого
человеческого рассудка, держащегося {\em чувственной} очевидности и
{\em привычных представлений} и {\em высказываний},
причем этот рассудок иногда выступает более спокойно (как, например, у
Диогена-собаки\pagenote{Имеется в виду кинический
философ Диоген из Синопа (414---323~гг. до н.~э). Греческое слово <<киник>>
(или <<циник>>) происходит от <<кион>> "--- собака. Поэтому Гегель
и называет здесь Диогена <<Diogenes der Hund>>, желая этим
вместе с тем подчеркнуть свое отрицательное отношение к этому
философу.\label{bkm:bm123}},
который показывал несостоятельность диалектики движения
молчаливым хождением взад и вперед), иногда же начинает сердиться по поводу
этой диалектики, видя в ней или просто глупость или, если дело идет о
важных для нравственности предметах, святотатство, стремящееся поколебать
нечто существенно-прочное и научающее, как доставлять доводы пороку (таков
взгляд сократовской диалектики, направленной против диалектики софистов, и
таков тот гнев, который в свою очередь стоил жизни самому Сократу).
Вульгарное опровержение, противопоставляющее, как это сделал Диоген,
мышлению {\em чувственное сознание}
и полагающее, что в последнем оно обретает истину, должно
быть предоставлено самому себе; поскольку же диалектика упраздняет
нравственные определения, нужно питать доверие к разуму, не сомневаясь в
том, что он сумеет восстановить их, но в их истине и в сознании их права,
однако, также и их границы. "--- Или же [в~других случаях]
вывод о субъективной ничтожности касается не самой диалектики, а, наоборот,
того познания, против которого она направлена, и (в~смысле скептицизма, а
равным образом в смысле кантовской философии) "--- {\em познания вообще}.

Основной предрассудок состоит здесь в том, будто диалектика
имеет {\em лишь отрицательный
результат;} этот вопрос скоро получит свое более детальное
определение. Но раньше мы должны заметить относительно упомянутой
{\em формы}, в которой
эта диалектика обыкновенно выступает, что она (диалектика) и ее результат
касаются согласно этой форме исследуемого {\em предмета} или же
субъективного {\em познания}, и она объявляет ничтожным последнее или
предмет; напротив, те {\em определения},
которые вскрываются в предмете, как в некотором {\em третьем}, остаются
без специального рассмотрения и предполагаются как значимые сами по себе.
\label{bkm:bm126a}Бесконечной заслугой кантовской философии
является то, что она заставила обратить внимание на этот некритический
образ действия и этим дала толчок к восстановлению логики и диалектики в
смысле рассмотрения {\em определений
мысли, взятых сами по себе}. Предмет, каков он без мышления
и понятия, есть некоторое представление или даже только название; только в
определениях мышления и понятия он {\em есть} то, что он {\em есть}. Поэтому
в действительности дело в них одних; они суть истинный предмет и содержание
разума, и все то, что иной раз понимают под предметом и содержанием в
отличие от них, имеет значимость только через них и в них. Поэтому не нужно
считать виной какого-нибудь предмета или познания то обстоятельство, что
они в силу своего характера и в силу некоторой внешней связи выказывают
себя диалектичными. В~этом случае представляют себе тогда и то и другое как
некоторый субъект, в который {\em определения} в форме
предикатов, свойств, самостоятельных всеобщностей внесены так, что они,
будучи принимаемы за прочные и сами по себе правильные, приводятся впервые
в диалектические отношения и в противоречие только путем чуждого им и
случайного связывания их в некотором третьем и со стороны некоторого
третьего. Между тем, такого рода внешний и неподвижный субъект
представления и рассудка, равно как и абстрактные определения, отнюдь не
могут рассматриваться как {\em последние}, прочно
остающиеся лежать в основании; наоборот, на них следует
смотреть как на нечто непосредственное, а именно, как на такое
пред-положенное и начальное, которое, как мы показали выше, само по себе
должно подпасть диалектике, потому что его следует понимать как понятие
{\em в~себе}. Так, все
принимаемые за нечто прочное противоположности, как, например, конечное и
бесконечное, единичное и всеобщее, образуют противоречие не через внешнее
связывание, а, как показало рассмотрение их природы, сами по себе суть
переход; их синтез и тот субъект, в котором они являются, есть продукт
собственной рефлексии их понятия. Если чуждое понятию рассмотрение не идет
дальше их внешнего отношения, изолирует их и оставляет их как прочные
предпосылки, то, наоборот, понятие пристально вглядывается в них самих,
движет ими как их душа и выявляет их диалектику.

Это та самая вышеуказанная [см. стр.~\pageref{bkm:bmpg224a}]
точка зрения, согласно которой некоторое всеобщее первое,
{\em рассматриваемое} {\em само по себе}, являет
себя как иное по отношению к самому себе. Взятое совершенно обще, это
определение может быть понимаемо так, что, стало быть, здесь то, что
сначала было {\em непосредственным},
положено как {\em опосредствованное}, {\em соотнесенное} с
чем-то иным, или, иначе говоря, что всеобщее положено как некоторое
особенное. То {\em второе},
которое возникло в силу этого, есть, стало быть, {\em отрицательное}
первого и, поскольку мы наперед примем в соображение
дальнейший ход развития, {\em первое
отрицательное}. Согласно этому отрицательному аспекту
непосредственное {\em погибло}
в ином, но это иное есть по существу не
{\em пустое отрицательное}, не {\em ничто},
которое принимается за обычный результат диалектики, а оно
есть {\em иное первого},
{\em отрицательное непосредственного;} оно, следовательно, определено как
{\em опосредствованное}, "--- {\em содержит} в~себе
вообще {\em определение первого}. Первое, стало быть, по существу также
{\em сбережено} и {\em сохранено} в ином. "--- Удерживать положительное в
{\em его} отрицательном, содержание предпосылки "--- в ее
результате\pagenote{Немецкий текст первого издания (1816~г.) испорчен:
<<Das Positive in {\em seinem} Negativen, dem
Inhalt der Vorausset\-zung im Resultate fest\-zuhal\-ten\ldots>>
Издание 1834~г., а за ним и все последующие прибавляют
запятую после слова <<Vorausset\-zung>>, но смысл от этого
не~выигрывает, а~проигрывает. Гораздо более соответствует смыслу всего
контекста такая конъектура: вместо <<dem Inhalt>> читать <<den
Inhalt>>. Перевод сделан соответственно этой последней
конъектуре.\label{bkm:bm124}},
это "--- самое важное в разумном познании; вместе
с~тем нужно лишь простейшее размышление для того, чтобы убедиться в
абсолютной истинности и необходимости этого требования, а что касается
{\em примеров} для доказательства этого, то вся логика состоит из них.

Стало быть, что теперь имеется перед нами, это "--- {\em опосредствованное},
которое, взятое ближайшим образом или, иначе говоря, тоже
непосредственно, есть также некоторое {\em простое}
определение, ибо, так как первое в нем погибло, то налицо
имеется лишь второе. А~так как первое также и {\em содержится} во
втором и это второе есть истина первого, то это единство может быть
выражено в виде такого предложения, в котором
непосредственное поставлено как субъект, опосредствованное же
"--- как его предикат, например, {\em <<конечное бесконечно>>},
{\em <<одно есть многое>>}, {\em <<единичное есть всеобщее>>}.
Но неадекватность формы таких предложений и суждений сама
собой бросается в глаза. Говоря о {\em суждении}, мы
показали, что его форма вообще и в особенности непосредственная форма
{\em положительного}
суждения неспособна вместить в~себя спекулятивное и истину.
Для этого нужно было бы по меньшей мере на равных правах присоединить к
нему также и его ближайшее дополнение,
{\em отрицательное суждение}.
В суждении первое как субъект имеет видимость некоторого
самостоятельного устойчивого существования, тогда как на самом деле оно,
наоборот, снято в своем предикате как в своем ином; это отрицание,
правда, содержится в содержании вышеуказанных предложений, но их
положительная форма противоречит этому содержанию; тем самым в них не
полагается то, что в них содержится, а ведь в этом полагании как раз и
состоит цель употребления предложений.

Второе определение, {\em отрицательное} или
{\em опосредствованное}, есть, далее, вместе с тем {\em опосредствующее}
определение. Ближайшим образом его можно принять за простое
определение, но по своей истине оно есть некоторое
{\em соотношение} или {\em отношение}, ибо оно
есть отрицательное, но отрицательное {\em положительного} и
включает последнее в~себя. Оно, следовательно, есть {\em иное} не как
иное чего-то такого, к чему оно равнодушно, "--- будь это
так, оно не было бы ни иным, ни некоторым соотношением или отношением, "---
а оно есть {\em иное в
себе самом}, {\em иное чего-то иного;} поэтому оно заключает в~себе
{\em свое} собственное иное и есть, стало быть, {\em как
противоречие, положенная диалектика себя самого}. "--- Так как
первое или непосредственное есть понятие {\em в~себе} и поэтому
оказывается отрицательным тоже лишь {\em в~себе}, то
диалектический момент у него состоит в том, что то
{\em различие}, которое в нем содержится {\em в~себе},
полагается внутри его. Напротив, второе само есть нечто
{\em определенное}, само представляет собой {\em различие}
или отношение; диалектический момент состоит у него поэтому в
том, чтобы положить содержащееся в нем {\em единство}. "---
Поэтому, если отрицательное, определенное, отношение,
суждение (перводеление) и все входящие в этот второй момент определения не
представляются уже сами по себе противоречием и диалектическими, то это
только недостаток мышления, не сводящего воедино своих мыслей. Ибо материал
({\em противоположные} определения в {\em одном соотношении}) уже
{\em положен} и наличествует для мышления. Но формальное мышление делает
себе законом тождество и низводит противоречивое содержание,
которое оно имеет перед собой, в сферу представления, в пространство и
время, в которых противоречивое удерживается {\em вне друг друга} в
рядоположности и внешней последовательности и таким образом выступает перед
сознанием без взаимного соприкосновения. Это мышление составляет для себя
об этом определенное основоположение, гласящее, что противоречие немыслимо;
на самом же деле мышление противоречия есть существенный момент понятия.
Формальное мышление и мыслит его фактически, но тотчас же закрывает на него
глаза и переходит от него в вышеуказанном высказывании лишь к абстрактному
отрицанию.

[3]. Рассмотренная отрицательность составляет {\em поворотный пункт} в
движении понятия. Она есть {\em простая
точка отрицательного соотношения} с собой, наивнутреннейший
источник всякой деятельности, живого и духовного самодвижения,
диалектическая душа, которую все истинное имеет в самом себе и через
которую оно только и есть истина; ибо единственно лишь на этой
субъективности покоится снятие противоположности между понятием и
реальностью и их единство, которое есть истина. "--- {\em Второе}
отрицательное, отрицательное отрицательного, к которому мы
пришли, есть указанное снятие противоречия, но оно столь же мало, как и
противоречие, есть {\em действие
некоторой внешней рефлексии}, а представляет собой
{\em наивнутреннейший, наиобъективнейший
момент} жизни и духа, благодаря которому имеет бытие
{\em субъект, лицо, свободное}. "--- {\em Соотношение отрицательного с собой
самим} должно рассматриваться как {\em вторая посылка} всего умозаключения. На
{\em первую} посылку, если пользоваться определениями
{\em <<аналитическое>>} и {\em <<синтетическое>>} в
их противоположении друг другу, можно смотреть как на {\em аналитический}
момент, так как непосредственное здесь {\em непосредственно}
относится к своему иному и поэтому {\em переходит} в него
или, лучше сказать, перешло в него; она аналитична, хотя, как уже было
упомянуто [см. выше, стр.~\pageref{bkm:bmpg224a}], это
соотношение как раз поэтому также и синтетично, так как оно переходит
именно в свое {\em другое}. Рассматриваемую теперь вторую
посылку можно определить как {\em синтетическую}, так
как она есть соотношение {\em различенного как такового} со {\em своим
различенным}. "--- ~Подобно тому как первая посылка есть момент
{\em всеобщности} и {\em сообщения}, так
вторая посылка определена через {\em единичность},
которая ближайшим образом соотносится с иным исключающе и
как существующая особо и разная. В~качестве {\em опосредствующего}
отрицательное выступает потому, что оно заключает в~себе себя
само и то непосредственное, отрицанием которого оно
является. Поскольку эти два определения берутся как внешне соотнесенные по
какому-либо отношению, оно (отрицательное) есть лишь опосредствующее
{\em формальное;} как
абсолютная же отрицательность отрицательный момент абсолютного
опосредствования есть то единство, которое представляет собой субъективность и
душу\pagenote{Тот силлогизм, о котором здесь говорит Гегель, различая его
первую и вторую посылку, можно для большей наглядности представить с помощью
схемы <<{\em П "--- О "--- О}>>, где <<{\em П}>> означает <<положительное>>, а
<<{\em О}>> "--- <<отрицательное>>. Первую посылку образует соотношение между
<<{\em П}>> и <<{\em О}>>, вторую "--- соотношение между <<{\em О}>> и
<<{\em О}>>. Эти два <<{\em О}>> отличаются друг от друга. Первое <<{\em О}>>
есть опосредствующий средний термин, и в нем Гегель в свою
очередь различает два момента, как это более подробно выясняется из
дальнейших рассуждений Гегеля: 1) простое, первое, формальное или
абстрактное отрицание и 2) абсолютное или второе отрицание. Этот второй
момент, имеющий внутри себя противоречие, Гегель характеризует как
<<диалектическую душу>> всего истинного, как <<наивнутреннейший источник>>
всякой деятельности и всякого самодвижения. Что же касается того <<{\em О}>>,
которое образует другой крайний термин рассматриваемого
силлогизма, то оно, как отрицание отрицания, есть восстановление первого
непосредственного (<<{\em П}>>),
однако такое восстановление, которое делает его единством
непосредственного и опосредствованного.\label{bkm:bm125}}.

В этом поворотном пункте метода течение познания возвращается
вместе с тем обратно в~себя само. Эта отрицательность есть, как снимающее
себя противоречие, {\em восстановление
первой непосредственности}, простой всеобщности; ибо иное
иного, отрицательное отрицательного непосредственно есть
{\em положительное, тождественное, всеобщее}. Это {\em второе}
непосредственное есть во всем ходе развития, если вообще угодно {\em считать},
нечто {\em третье} по отношению к первому непосредственному и к
опосредствованному. Но оно есть также третье и по отношению к первому или
формальному отрицательному и к абсолютной отрицательности или ко второму
отрицательному; а поскольку то первое отрицательное есть уже второй термин
в развитии всего процесса, то можно считаемое {\em третьим} считать
также и {\em четвертым} и вместо того, чтобы принимать абстрактную форму за
{\em троичность}, принимать ее за {\em четверичность;} отрицательное или
{\em различие} принимается в этом случае за нечто двойное. "---
Третье (или: четвертое) есть вообще единство первого и
второго моментов, непосредственного и опосредствованного. "---
То обстоятельство, что оно (третье) есть это {\em единство}, равно как
и то, что вся форма метода представляет собой {\em троичность}, есть,
правда, всецело поверхностная, лишь внешняя сторона способа познания.
Однако опять-таки как бесконечную заслугу кантовской
философии\pagenote{Выше, на стр.~\pageref{bkm:bm126a} Гегель говорил о другой
<<бесконечной заслуге>> Канта "--- идее критического исследования
определений мысли. Здесь же Гегель имеет в виду главным образом
триадичность кантовской таблицы категорий, в которой <<третья категория
возникает всегда из соединения второй и первой категории одного и того же
класса>> (см. {\em Кант}, Критика чистого разума, пер. Лосского, Пгр. 1915,
стр.~11). Еще б\'{о}льшую роль играет триада в философии Фихте. Триадическая схема
<<тезис "--- антитезис "--- синтез>> фигурирует уже в написанной в 1794~г.
книге Фихте <<Основа общего наукоучения>> (см. {\em Фихте},
Избр. соч., т.~1, М.~1916, стр.~91 и~др.).\label{bkm:bm126}}
следует рассматривать даже и то, что она указала хотя бы
только на эту сторону, и притом в более определенном применении (ибо сама
эта абстрактная числовая форма была, как известно, выдвинута уже очень
рано, но чуждым понятию способом, и потому эта мысль не имела последствий).
{\em Умозаключение}, которое тоже тройственно, всегда признавалось всеобщей
формой разума, но отчасти оно считалось вообще совершенно внешней формой, не
определяющей природы содержания, отчасти же ему недостает существенного
{\em диалектического} момента, {\em отрицательности},
так как оно в формальном смысле сводится лишь к рассудочному
определению {\em тождества;}
но этот момент появляется в троичности определений, так как
третье есть единство двух первых определений, последние же, ввиду того что
они суть разные, могут находиться в единстве только
{\em как снятые}. "---
Формализм, правда, тоже завладел троичностью и
придерживался ее пустой схемы; поверхностность, скандальность и пустота
современного философского так называемого {\em конструирования},
состоящего не в чем ином, как в том, чтобы повсюду наклеивать
эту формальную схему без понятия и имманентного определения и пользоваться
ею для расположения материала во внешнем порядке, сделали указанную форму
скучной и приобрели ей дурную
славу\pagenote{Гегель имеет ввиду Шеллинга и, в
особенности, его последователей и поклонников. О~Шеллинге Гегель в <<Истории
философии>> замечает, что его философия страдает <<формализмом внешнего
конструирования по некоторой наперед принятой схеме>> ({\em Гегель},
Соч., т.~XI, стр.~504), а именно, по схеме троичности (там
же, стр.~506 и 510). О~поклонниках Шеллинга Гегель отзывается еще более
резко. <<Вся эта манера, "--- говорит он об их манере
философствовать, "--- представляет собой такой жалкий
формализм, такое бессмысленное смешение обыденнейшей эмпирии и
поверхностнейших идеальных определений, какой только когда-либо
существовал\ldots Философия благодаря этому\ldots сделалась предметом
пренебрежения и презрения>> (там же, стр.~511---512).\label{bkm:bm127}}.
Но из-за безвкусности этого употребления она не может
потерять своей внутренней ценности, и все же нужно высоко ценить то, что на
первых порах был найден хотя бы непостигнутый в понятиях облик разумного.

Говоря более определенно, {\em третье} есть
непосредственное, но непосредственное
{\em через снятие опосредствования}, простое через {\em снятие
различия}, положительное через снятие отрицательного,
понятие, реализовавшее себя через инобытие, слившееся с собой через снятие
этой реальности и восстановившее свою абсолютную реальность, свое
{\em простое} соотношение с собой. Этот {\em результат} есть поэтому
{\em истина}. Он есть {\em столь же} непосредственность, {\em сколь} и
опосредствование. Но эти формы суждения: <<третье {\em есть}
непосредственность и опосредствование>> или: <<оно {\em есть} их
{\em единство}>> "--- не в
состоянии охватить указанный результат, так как он есть не некое покоящееся
третье, а именно в качестве этого единства образует опосредствующее себя с
самим собой движение и деятельность. "--- Подобно тому как
начало есть {\em всеобщее}, так результат есть
{\em единичное, конкретное, субъект;} то, что первое есть {\em в
себе}, последнее есть теперь в такой же мере {\em для себя}, всеобщее
{\em положено} в субъекте. Два первых момента троичности суть
{\em абстрактные}, неистинные моменты, которые именно поэтому диалектичны и
через эту свою отрицательность делают себя субъектом. Само понятие есть
(ближайшим образом, {\em для нас}) {\em как} в-себе-сущее
всеобщее, {\em так} и для-себя-сущее отрицательное, а равно и третье
"--- в-себе-и-для-себя-сущее; оно есть такое {\em всеобщее}, которое
проходит сквозь все моменты умозаключения; но третье есть заключение, в
котором понятие опосредствовано с самим собой через свою отрицательность и,
стало быть, положено {\em для себя} как {\em всеобщее}
и {\em тождественное своих моментов}.

Этот результат теперь, как ушедшее в~себя и {\em тождественное} с
собой целое, сообщил себе снова форму {\em непосредственности}.
Таким образом, результат теперь сам таков, каким определило
себя {\em начало}. Как
простое соотношение с собой, он есть некоторое всеобщее, и та
{\em отрицательность},
которая составляла его диалектику и опосредствование,
свернулась в этой всеобщности равным образом в
{\em простую определенность},
могущую снова быть началом. На первый взгляд может казаться,
что это познание результата должно быть анализом его и поэтому должно снова
разобрать те определения и тот ход их движения, через который он
(результат) возник и который был предметом рассмотрения. Но если предмет
действительно трактуется таким аналитическим образом, то такая трактовка
принадлежит рассмотренной выше ступени идеи, ищущему познанию, которое
относительно своего предмета лишь указывает, что {\em есть}, не касаясь
необходимости его конкретного тождества и понятия последнего. Метод же
истины, постигающий предмет в понятии, сам, правда, как мы показали,
аналитичен, так как он безоговорочно остается в пределах понятия, но он в
такой же мере и синтетичен, ибо через понятие предмет становится
диалектичным и определяется как другой. При той новой основе, которую
образует собой результат как то, что отныне служит предметом, метод
остается тем же самым, как и при предыдущем предмете. Различие касается
только отношения основы как таковой; она, правда, есть основа также и
теперь, однако ее непосредственность есть лишь {\em форма}, так как она
вместе с тем была результатом; ее определенность как содержание есть
поэтому теперь уже не нечто просто принятое извне, а нечто
{\em выведенное} и {\em доказанное}.

[4]. Только здесь {\em содержание} познания
как таковое впервые вступает в круг рассмотрения, так как теперь оно, как
выведенное, принадлежит методу. Благодаря этому моменту сам метод
расширяется в {\em систему}. "---
Сперва начало, что касается содержания, должно было быть для
метода совершенно неопределенным; метод представляется постольку лишь
формальной душой, для которой и через которую начало было определено
исключительно только со стороны своей формы, а именно, как непосредственное
и всеобщее. Через показанное нами движение предмет получил для самого себя
такую {\em определенность}, которая есть некоторое {\em содержание}, так как
свернувшаяся в простоту отрицательность есть снятая форма и как простая
определенность противостоит своему развитию и прежде всего
"--- самой своей противоположности к всеобщности.

И вот, так как эта определенность есть ближайшая истина
неопределенного начала, то она представляет собой порицание этого начала
как чего-то несовершенного, равно как и порицание своего метода, который,
отправляясь от этого начала, был только формальным. Это может быть выражено
как отныне определенное требование, чтобы начало (так как оно в своем
отношении к определенности результата само есть нечто определенное)
принималось не за непосредственное, а за опосредствованное и
выведенное; а это может казаться требованием идущего
{\em назад} бесконечного
прогресса в доказывании и выведении; подобным же образом из полученного
нового начала через движение метода опять-таки проистекает некоторый
результат, так что поступательное движение продолжается также и
{\em вперед} до бесконечности.

Уже неоднократно указывалось на то, что бесконечный прогресс
принадлежит вообще области чуждой понятию рефлексии; абсолютный метод,
имеющий понятие своей душой и своим содержанием, не может привести к такому
прогрессу. На первый взгляд может казаться, что уже такие начала, как
{\em бытие}, {\em сущность}, {\em всеобщность},
характеризуются тем, что они в полной мере обладают той
всеобщностью и бессодержательностью, которая требуется для совершенно
формального начала, каким оно должно быть, и потому они как абсолютно
первые начала не требуют и не допускают никакого дальнейшего обратного
движения. Так как они суть чистые соотношения с самими собой, суть
непосредственные и неопределенные, то в них, действительно, нет того
различия, которое в каком-либо другом начале сразу же положено между
всеобщностью его формы и его содержанием. Но та неопределенность, которую
указанные логические начала имеют своим единственным содержанием, сама и
есть то, что составляет их определенность; а именно, последняя состоит в их
отрицательности как снятом опосредствовании; особенность последнего
сообщает и их неопределенности некоторую особенность, в силу которой
{\em бытие}, {\em сущность} и {\em всеобщность}
отличаются друг от друга. Свойственная им определенность есть
их {\em непосредственная
определенность}, поскольку их берут самих по себе; она
постольку такая же непосредственная определенность, как определенность
какого-либо содержания, и поэтому нуждается в некотором выведении; для
метода безразлично, принимается ли определенность за определенность
{\em формы} или за определенность {\em содержания}.
Вот почему методу на самом деле не приходится начать
оперировать по-новому оттого, что через первый его результат определилось
некоторое содержание; он вследствие этого не делается ни более, ни менее
формальным, чем прежде. Ибо, так как он есть абсолютная форма, понятие,
знающее само себя и всё другое как понятие, то нет такого содержания,
которое противопоставляло бы себя ему и определило бы его так, что он стал
бы односторонней, внешней формой. Поэтому, подобно тому как
бессодержательность указанных начал не делает их абсолютными началами, так,
с другой стороны, и содержание как таковое не может быть
тем, что приводило бы метод к бесконечному прогрессу вперед или назад.
С~одной стороны, та {\em определенность},
которую метод порождает для себя в своем результате, есть тот
момент, через который метод есть опосредствование с собой и превращает
{\em непосредственное начало} в {\em опосредствованное}.
Но и обратно, именно через определенность протекает это
присущее методу опосредствование; метод, проходя
{\em через} некоторое {\em содержание}, как через кажущееся {\em иное}
себя самого, возвращается обратно к своему началу таким
образом, что он не только восстанавливает последнее в виде чего-то теперь
{\em определенного}, но
результатом оказывается в такой же мере и снятая определенность и, стало
быть, также и восстановление той первой неопределенности, которая для
метода служила началом. Это метод выполняет как единая
{\em система тотальности}.
Мы еще должны рассмотреть его в этом его определении.

Та определенность, которая была результатом, сама есть, как
было указано, вследствие формы простоты, в которую она свернулась,
некоторое новое начало; так как это начало отличается от своего предыдущего
именно этой определенностью, то познание катится вперед от содержания к
содержанию. Прежде всего это поступательное движение характеризуется тем,
что оно начинает с простых определенностей и что последующие определенности
становятся все {\em богаче и
конкретнее}. Ибо результат содержит в~себе свое начало, и
дальнейшее движение этого начала обогатило его (начало) новой
определенностью. {\em Всеобщее}
составляет основу; поэтому поступательное движение не должно
быть понимаемо как {\em течение} от чего-то {\em иного} к
чему-то {\em иному}. В абсолютном методе понятие {\em сохраняется} в своем
инобытии, всеобщее "--- в своем обособлений, в суждении и
реальности; на каждой ступени дальнейшего определения всеобщее поднимает
выше всю массу своего предыдущего содержания и не только ничего не теряет
вследствие своего диалектического поступательного движения, не только
ничего не оставляет позади себя, но уносит с собой все приобретенное и
обогащается и уплотняется внутри себя.

Это {\em расширение} может рассматриваться как момент содержания, а внутри
целого "--- как первая посылка; всеобщее {\em сообщено} богатству
содержания, непосредственно сохранено в нем. Но отношение имеет также и
вторую, отрицательную или диалектическую сторону. Процесс обогащения
движется вперед вдоль {\em необходимости}
понятия, последнее держит его, и каждое определение есть
некоторая рефлексия в~себя. Каждая новая ступень
{\em выхождения вне себя}, т.~е. {\em дальнейшего
определения}, есть также и некоторый уход внутрь себя, и
большее {\em расширение} есть в такой же мере и
{\em более высокая интенсивность}.
Самое богатое есть поэтому самое конкретное и {\em самое субъективное},
и то, что вбирает себя обратно в наиболее простую глубину,
есть самое могущественное и самое объемлющее. Высшей, наиболее заостренной
вершиной является {\em чистая
личность}, которая единственно только через абсолютную
диалектику, составляющую ее природу, вместе с тем
{\em всё охватывает} и держит {\em внутри себя},
потому что она делает себя тем, что всего свободнее, "---
той простотой, которая есть первая непосредственность и всеобщность.

Именно таким образом происходит, что каждый шаг {\em вперед в поступательном
движении}, каждое дальнейшее определение, удаляясь от неопределенного начала,
представляет собой также и {\em возвратное приближение}
к последнему\pagenote{В~смысле все более и более глубокого понимания этого
начала, его истинной природы и его истинного значения.\label{bkm:bm128}} и что,
стало быть, то, что на первый взгляд может казаться разным, "--- {\em идущее
назад обоснование} начала и {\em идущее вперед дальнейшее} его {\em
определение}, "--- совпадает воедино и есть одно и то же. Но метод, образующий,
таким образом, некоторый круг, не может в своем временн\'{о}м развитии
предвосхитить, что начало уже как таковое есть нечто выведенное; для начала
в его непосредственности достаточно того, что оно есть простая всеобщность.
Поскольку оно таково, оно имеет свое полное условие, и нет нужды извиняться
по поводу того, что это начало дескать можно принимать лишь {\em провизорно}
и {\em гипотетически}\pagenote{Немецкий текст гласит: <<es braucht nicht
depreziert zu werden, dass man ihn nur {\em provi\-sori\-sch} und
{\em hypo\-the\-tisch} gelten lassen möge>>. Слово <<deprezieren>> имеет в
немецком языке два значения: 1) извиняться и 2) обесценивать. Если в этой фразе
вместо <<es>> читать <<еr>> (т.~е. der Anfang), то слово <<deprezieren>> можно
истолковать во втором его значении. Тогда перевод будет гласить: <<нет нужды
обесценивать это начало, утверждая, что его можно принимать лишь {\em
провизорно} и {\em гипотетически}>>.
Ленин, по-видимому, склонялся к этому последнему истолкованию
(см.~<<Ленинский сборник>>, IX, М.---Л.~1929, стр.~296).

Гегель имеет здесь в виду кантианца
Рейнгольда, о котором он упоминает в т.~I <<Науки логики>>, как о защитнике
того взгляда, что <<философия должна начинать лишь с некоторого
{\em гипотетически} и {\em проблематически}
истинного и что философствование поэтому может быть сначала
лишь исканием>> (см. т.~I <<Науки логики>>, стр.~45; ср.~{\em Гегель},
Соч., т.~I, стр.~28). О~Рейнгольде и его <<Материалах для
краткого обзора состояния философии в начале XIX столетия>> (первая тетрадь
вышла в 1801 г.) Гегель подробно говорит в своей первой печатной работе
<<Различие между системами философии Фихте и Шеллинга>> (Иена 1801).\label{bkm:bm129}}.
Какие бы возражения ни приводили против него, "---
например, что человеческое познание ограничено или что,
прежде чем обратиться к предмету, требуется критически исследовать орудие
познания\pagenote{Намек на <<критическую>> философию Канта.\label{bkm:bm130}},
"--- эти возражения сами суть предпосылки, которые как
{\em конкретные определения}
приводят за собой требование, чтобы они были опосредствованы
и обоснованы. Так как они тем самым формально не имеют никакого
преимущества перед тем {\em началом}
с предмета, против которого (начала) они протестуют, а
наоборот, вследствие своего более конкретного содержания нуждаются в
выделении, то они должны быть признаны пустыми притязаниями, будто их
следует принимать во внимание более, чем нечто иное. Они имеют
неистинное содержание, так как они превращают в нечто непреложное и
абсолютное то, что известно как конечное и неистинное, а именно, некоторое
{\em ограниченное} познание, определенное как {\em форма} и {\em орудие}
{\em по отношению} к своему {\em содержанию;}
само это неистинное познание есть тоже такая форма, такое
обоснование, которое идет назад. "--- Метод истины также знает
начало как нечто несовершенное, потому что оно есть начало, но вместе с тем
он знает это несовершенное вообще как нечто необходимое,
потому что истина есть не~что иное, как приход к самому себе через
отрицательность непосредственности. Нетерпение, желающее
{\em лишь} выйти за пределы {\em определенного}
(будет ли последнее называться началом, объектом, конечным,
или будет взято в какой-нибудь другой форме) и оказаться непосредственно в
абсолютном, не имеет как познание ничего перед собой, кроме пустой
отрицательности, абстрактной бесконечности; или, иначе говоря, оно имеет
перед собой некое {\em мнимое} (gemeintes) абсолютное, которое есть мнимое
потому, что оно не {\em положено}, не {\em постигнуто;}
постигнуть его можно лишь через {\em опосредствование}
познавания; всеобщее и непосредственное есть момент этого
опосредствования, сама же истина обретается лишь в широко развернутом ходе
его движения и в конце пути. "--- Для удовлетворения
субъективной потребности незнакомых [с~предметом] и их нетерпения можно,
правда, дать {\em наперед} для рефлексии некоторый обзор {\em целого} путем
подразделения, которое на манер конечного познания, начиная с всеобщего,
указывает особенное как нечто {\em имеющееся налицо} и
как то, появления чего следует ожидать в науке. Однако такой обзор дает
только некоторый образ {\em представления;} ибо
истинный переход от всеобщего к особенному и к
в-себе-и-для-себя-определенному целому, в котором само это первое всеобщее
есть по своему истинному определению в свою очередь момент, чужд этому
способу подразделения и есть исключительно только опосредствование самой науки.

В силу указанной природы метода наука являет себя как некоторый завитый в~себя
{\em круг}, в начало которого (в~простое основание) опосредствование вплетает
обратно его конец; при этом круг этот есть {\em круг кругов;} ибо каждый
отдельный член, как одушевленный методом, есть рефлексия в~себя, которая,
возвращаясь в начало, вместе с тем есть начало некоторого нового члена.
Фрагменты этой цепи суть отдельные науки, из коих каждая имеет некоторое
{\em предыдущее} и некоторое {\em последующее}, "--- или, говоря точнее,
каждая отдельная наука {\em имеет} только предыдущую и {\em показывает}
свою последующую в сам\'{о}м своем заключении.

Таким образом, и логика тоже возвратилась в абсолютной идее к
тому простому единству, которое есть ее начало; чистая непосредственность
бытия, в котором всякое определение представляется сначала погашенным или
отброшенным путем абстракции, есть идея, пришедшая путем опосредствования,
а именно, путем снятия опосредствования, к своему адекватному равенству с
собой. Метод есть чистое понятие, относящееся лишь к себе самому; он есть
поэтому {\em простое соотношение с собой}, каковое соотношение есть {\em
бытие}. Но теперь это есть также и {\em наполненное} бытие, {\em постигающее}
себя {\em понятие}, бытие как {\em конкретная} и также безоговорочно {\em
интенсивная} тотальность. "--- Об этой идее следует в заключение сказать еще
лишь то, что в ней, {\em во-первых}, {\em логическая наука }ухватила свое
собственное понятие. В {\em бытии}, этом начале ее {\em содержания}, ее
понятие представляется внешним этому содержанию знанием в субъективной
рефлексии. Напротив, в идее абсолютного познания понятие это стало ее
собственным содержанием. Она сама есть чистое понятие, которое имеет себя
своим предметом и которое, пробегая в качестве предмета для самого себя всю
тотальность своих определений, развивает себя в целое своей реальности, в
систему науки и кончает тем, что ухватывает это постижение самого себя и
тем самым снимает свой характер содержания и предмета и познает понятие
науки\pagenote{Гегель хочет сказать, что только
в самом конце изложения логика достигает полного познания самой себя,
своего предмета и своего <<понятия>>. Начинается же это самопознание логики
(самопознание познания) с самых первых шагов логической науки, с категорий
бытия и ничто. Во введении к <<Науке логики>> Гегель указывает, что его
<<объективная логика есть подлинная критика>> <<чистых форм мысли>>,
рассматриваемых <<{\em в их особенном содержании}>> (т.~I <<Науки
логики>>, стр.~39---40). В~этом заключается одно из существенных отличий
гегелевской логики от <<критицизма>> Канта, который требует критического
исследования познавательных форм {\em до}
самого познания, рассматривая эти формы как пустые априорные
формы, извне накладываемые нами на внешнее им содержание. По Гегелю,
логические формы составляют <<{\em живой дух действительного}>> ({\em Гегель},
Соч., т.~I, стр.~267). Ленин, выписывая эту формулировку
Гегеля, отмечает на полях: <<общие законы {\em движения мира и мышления}>>
({\em Ленин},
Философские тетради, М.~1936, стр.~170). То же самое отмечает
и Энгельс, усматривая великую заслугу диалектики Гегеля в том, что она
показала наличие {\em одних и тех же} законов в трех различных сферах: в
природе, истории и мышлении (см примечания к <<Анти-Дюрингу>>). Эти общие
законы движения мира и мышления и составляют {\em подлинное}
содержание Логики Гегеля. Отсюда и получается (как говорит
Ленин), что <<в этом {\em самом идеалистическом} произведении Гегеля
{\em всего меньше} идеализма, {\em всего больше} материализма.
<<Противоречиво>>, но факт!>> ({\em Ленин},
Философские тетради, стр.~225). Конечно, для того чтобы
усмотреть и понять этот факт, необходимо предварительно очистить
гегелевскую логику от обволакивающей ее <<мистики идей>>, поповщины, остатков
формализма, пустой игры в диалектику и~т.~д.\label{bkm:bm131}}.
"--- {\em Во-вторых},
эта идея есть еще логическая идея, она замкнута в чистую
мысль, еще есть наука лишь божеского {\em понятия}. Правда,
систематическая разработка сама есть реализация, но реализация, не
выходящая за пределы этой же сферы. Так как чистая идея познания постольку
замкнута в субъективность, то она есть {\em влечение} снять эту
субъективность, и чистая истина как последний результат становится также и
{\em началом некоторой другой сферы и
науки}. Этот переход здесь нужно еще только наметить.

А именно, поскольку идея полагает себя как абсолютное {\em единство} чистого
понятия и его реальности и тем самым собирает себя в непосредственность
{\em бытия}, она, как {\em тотальность} в этой форме, есть
{\em природа}\pagenote{Этот переход от абсолютной идеи
к природе имеет у Гегеля две стороны: 1) мистическую или теологическую и 2)
рациональную. Мистической стороны этого перехода Энгельс касается в
<<Людвиге Фейербахе>>, говоря, что у Гегеля <<сотворение мира принимает еще
гораздо более несуразный и невозможный вид, чем в христианстве>> (вспомним,
что сам Гегель определяет содержание логики как <<изображение бога, каков он
есть в своей вечной сущности {\em до сотворения} природы и конечного
духа>>, "--- т.~I <<Науки логики>>, стр.~28; цитата из Энгельса
взята по немецкому изданию <<Л. Фейрбаха>>, Москва 1932, стр.~28)
Рациональную сторону перехода от идеи к природе отмечает в своем Конспекте
Ленин. Выписав гегелевскую фразу: <<А именно\ldots есть {\em природа}>>,
Ленин замечает: <<Эта фраза на {\em последней}, [213]-ой странице
{\em Логики} архизамечательна. Переход логической идеи к {\em природе}.
Рукой подать к материализму. Прав был Энгельс, что система
Гегеля перевернутый материализм>> ({\em Ленин},
Философские тетради, М.~1936, стр.~224). С~этим указанием
Ленина интересно сопоставить замечания Маркса в его
<<Экономическо-философских рукописях 1844 года>>. Маркс пишет: Абсолютная
идея, как результат всей логики Гегеля, <<в свою очередь снимает самое себя,
если только она не хочет снова проделать сначала весь процесс абстракции и
удовольствоваться тем, чтобы быть тотальностью всех абстракций или
постигающей себя абстракцией. Но абстракция, постигающая себя как
абстракцию, знает, что она есть ничто: она должна отказаться от себя,
отказаться от абстракции, и таким образом она приходит к такой сущности,
которая есть ее прямая противоположность, "--- к {\em природе}.
Вся логика представляет собой, стало быть, доказательство
того, что абстрактное мышление, рассматриваемое само по себе (оторванно от
{\em действительного} духа и от {\em действительной} природы), есть ничто,
что абсолютная идея, взятая сама по себе (как нечто самостоятельное по
отношению к природе и духу), есть ничто и что только {\em природа}
есть нечто>> ({\em Marx--Engels}, Gesamt\-ausgabe, hrsg. v.~Adoratskij, Erste
Abteilung, Bd.~III, Berlin 1932, S.~168--169).\label{bkm:bm132}}.
"--- Но это определение не есть результат некоторого
{\em становления} и {\em перехода}, как,
согласно вышесказанному, субъективное понятие в его тотальности
{\em становится объективностью}, а {\em субъективная цель
становится жизнью}. Чистая идея, в которой определенность
или реальность понятия сама возведена в понятие, есть скорее абсолютное
{\em освобождение}, для
которого больше нет никакого непосредственного определения, которое не было
бы вместе с тем {\em положенным}
и понятием; в этой свободе не имеет поэтому места никакой
переход; то простое бытие, к которому определяет себя идея, остается для
нее вполне прозрачным и есть понятие, остающееся в своем определении у
самого себя. Переход, следовательно, должен быть здесь понимаем скорее так,
что идея сама себя {\em свободно
отпускает}, абсолютно уверенная в~себе и покоящаяся внутри
себя. Вследствие этой свободы {\em форма
ее определенности} также безоговорочно свободна, "---
есть абсолютно сама по себе, без субъективности, существующая
{\em внешность пространства и времени}.
"--- Поскольку эта внешность существует и схватывается
сознанием только в аспекте абстрактной непосредственности
бытия, она выступает как простая объективность и внешняя жизнь; но в идее
она остается в~себе и для себя тотальностью понятия, и наука пребывает в
области отношения божественного познания к природе. Однако это ближайшее
решение чистой идеи определить себя как внешнюю идею полагает себе этим
лишь опосредствование, из которого понятие поднимается ввысь как свободное
существование, ушедшее из внешности внутрь себя, завершает свое
самоосвобождение в {\em науке о духе}\pagenote{Имеется в виду
<<Философия духа>> как третья часть гегелевской системы.\label{bkm:bm133}}
и находит наивысшее понятие самого себя в логической науке
как постигающем себя чистом понятии.

\clearpage

\chapter[Примечания к <<Учению о понятии>>]{Примечания к <<Учению о понятии>>}

\printpagenotes

\clearpage\
\chapter[Перевод важнейших терминов <<Науки логики>> Гегеля]
{Перевод важнейших терминов\newline<<Науки логики>> Гегеля}

\bigskip

\begin{multicols}{2}
Allgemeine, das "--- всеобщее

Allgemein\-heit "--- всеобщность

Allheit "--- всякость (см. прим. \ref{bkm:Ref484772654})

Am-Etwas-Sein "--- бытие-в-нечто

Andere, das "--- другое, иное

Anderssein "--- инобытие, инаковость

Aiujers\-werden "--- иностановление, становление иным

An ihm "--- в нем

Anschauung "--- созерцание

An-sich "--- в~себе

Ansichsein "--- в-себе-бытие

An-und-fur-sich-sein "--- в-себе-и-для-себя-бытие

Anzahl "--- численность

Attraktion "--- притяжение

Aufheben "--- 1) снимание, снятие (как технический термин гегелевской
философии); 2) упразднение, устранение (во всех остальных случаях)

Ausein\-ander\-sein "--- внеположность

Ausser\-ein\-ander, das "--- внеположность

Ausser\-ein\-ander\-sein "--- бытие-вне-друг-друга, внеположность

Ausser-sich-sein "--- вне-себя-бытие

Äusserung "--- проявление во-вне

\bigskip

Begriff "--- понятие (см. прим. \ref{bkm:Ref485481734})

Bei sich "--- у себя

Bei-sich-sein "--- у-себя-бытие (иногда: замыкание в~себя)

Beschaffen\-heit "--- характер

Besondere, das "--- особенное

Bestehen "--- устойчивое наличие, устойчивое существование (в
отдельных случаях: составленность, состояние, существование)

Bestimmt\-heit "--- определенность (иногда: определенный характер)

Bestimmt\-sein "--- определенность, определяемость

Bestimmt\-werden "--- определяемость

Bestimmung "--- определение (иногда в смысле назначения)

Beziehung "--- соотношение (иногда: соотнесение)

Böse, das "--- (нравственное) зло

\bigskip

Dasein "--- 1) наличное бытие (как технический термин); 2)
существование (во всех остальных случаях)

Definition "--- дефиниция

Denk\-bestim\-mun\-gen "--- определения мысли, мыслительные определения

Denkformen "--- формы мысли

Diese, das "--- этость

Differenz "--- различие, небезразличие, диферентность

Ding "--- вещь

Ding-an-sich "--- вещь-в-себе

Dingheit "--- вещность

Diskretion "--- дискретность

\bigskip

Eine, das "--- единое

Eines "--- одно

Einheit "--- 1) единица (как единица измерения и как момент числа); 2) единство

Eins "--- одно, единое

Einteilung "--- деление, подразделение

Einzelne, das "--- единичное

Element "--- стихия

Endlichkeit "--- конечность, конечное

Entäus\-serung "--- отчуждение

Entgegen\-setzung "--- противоположение

Entstehen "--- возникновение

Entwicklung "--- 1) развитие, развертывание; 2) разложение в ряд (в~математике)

Ent\-wick\-lungs\-po\-tenz "--- степенн\'{о}й член разложения
(см.~прим.~53 к т.~I)

Existenz "--- существование

\bigskip

Für-Eines-Sein "--- бытие-для-одного (см.~прим.~33 к т.~I)

Für-es-sein "--- для-него-бытие

Fur-sich "--- 1) для себя (как технический термин); 2) особо,
само по себе, отдельно, самостоятельно (во всех остальных случаях)

Fur-sich-sein "--- для-себя-бытие

\bigskip

Gedank\-ending "--- вещь, сочиненная мыслью, нечто лишь мысленное,
а не реальное; голая абстракция

Gegen\-satz "--- противоположность (в~отдельных случаях:
противоречие)

Gehalt "--- содержимое, содержательность (иногда: содержание)

Geist "--- дух

Gemüt "--- душа

Gesetzi\-sein "--- положенность

Gewiss\-heit "--- достоверность, уверенность

Gleich\-heit "--- равенство, одинаковость

Grad "--- градус, степень

Grenze "--- 1) граница (как технический термин логики Гегеля); 2) предел
(в~математике)

Grösse "--- величина

Grund "--- основание (Zu~Grunde gehen "--- идти ко дну,
погружаться в~основание, см.~прим.~81 к~т.~I)

Grundlage "--- основа

Gute, das "--- добро, благо

\bigskip

Ideelle, das "--- идеализованное

Identität "--- тождество

Indiffe\-renz "--- индиференция (см.~прим.~69 к~т.~I)

Inhalt "--- содержание

Inhärenz "--- присущность

In-sich-Sein "--- внутри-себя-бытие

\bigskip

Jenseits, das "--- потустороннее

\bigskip

Kontinui\-tät "--- непрерывность

\bigskip

Leere, das "--- пустота

Lehrsatz "--- теорема

\bigskip

Masslose, das "--- безмерное

Mehrheit "--- многость

Mitte (die) des Schlusses "--- средний термин силлогизма

\bigskip

Nega\-tivi\-tät "--- отрицательность

Nicht\-da\-sein "--- неимение наличного бытия

Nicht\-iden\-tität "--- нетождество

Nicht\-sein "--- небытие

Nicht\-unter\-schieden\-sein "--- неразличность

\bigskip

Punktu\-ali\-tat "--- точечность

\bigskip

Quantität "--- количество

Quantum "--- определенное количество

\bigskip

Räson\-nement "--- рассуждение (иногда: рассуждательство)

Rason\-nieren "--- рассуждательство, рассуждение

Reflek\-tiertsein "--- рефлектированность

Reflex "--- отражение

Reflexion "--- рефлексия (см.~прим~78 к~т.~I; в отдельных случаях
переводится через <<соображение>>)

Re\-flexions\-bestim\-mungen "--- рефлективные (рефлексивные)
определения, определения рефлексии

Regel "--- правило (см.~прим.~61 к~т.~I)

Repul\-sion "--- отталкивание

\bigskip

Sache "--- 1) мыслимая вещь; 2) суть; 3) вещь, предмет

Satz "--- предложение, положение, начало (например, <<начало противоречия>>)

Schein "--- видимость (в~отдельных случаях переводится через <<свечение>>)

Scheinen "--- свечение, свечение видимостью, излучение видимости
(см.~прим.~78 к~т.~I)

Schluss "--- умозаключение, силлогизм (см. прим.~\ref{bkm:Ref486285580})

Schluss\-satz "--- заключение

Schranke "--- предел

Seele "--- душа

Seelending "--- душа-вещь

Sein "--- бытие

Selbst, das "--- самость

Selbst\-bewegung "--- самодвижение

Setzen "--- полагание

Sollen "--- долженствование

Sprung "--- скачок

Staats\-ökonomie "--- политическая экономия

Stetigkeit "--- непрерывность

Subsumtion "--- подведение под более общее, подчинение более общему

\bigskip

Totalität "--- тотальность, целостность, целокупность

Trieb "--- влечение (в~отдельных случаях: стремление, движущее начало, импульс)

Triplizität "--- троичность, тройственность

\bigskip

Ungleich\-heit "--- неравенство, неодинаковость

Universum "--- вселенная, универсум

Unmittel\-bare, das "--- непосредственное

Unter\-schied "--- различие

Unter\-schie\-densein "--- различность

Urteil "--- суждение (см. прим.~\ref{bkm:Ref484772231})

\bigskip

Vergehen "--- прехождение

Verhältnis "--- отношение

Vermit\-telung "--- опосредствование

Vernunft "--- разум

Ver\-schieden\-heit "--- разность

Verstand "--- рассудок (в~отдельных случаях: смысл)

Vielheit "--- множественность

Voraus\-setzen "--- предполагать, предполагать (в~смысле <<полагать
наперед>>, <<предпосылать>>)

Voraus\-setzung "--- предположение, предпосылка, пред-положение

Vor\-stellen "--- представливание, представление

Vor\-stellung "--- представление

\bigskip

Wahrheit "--- истина, истинность

Werden "--- становление

Wesen "--- сущность

Wesen\-heit "--- 1) определенная сущность (как особая категория);
2) сущность (в~остальных случаях)

Wesent\-lich\-keit "--- существенность

Widers\-pruch "--- противоречие

Wollen, das "--- воля

\bigskip

Ziel "--- цель стремлений

Zusammen\-fallen "--- сжиматься, свертываться

Zusammen\-setzung "--- составность

Zweck "--- цель
\end{multicols}

\bigskip
\clearpage

\chapter[Библиография]{Библиография}

\paragraph%
[I. Немецкие издания большой логики]%
{I. Немецкие издания большой логики}

\begin{enumerate}
\item
Wissen\-schaft der Logik. Von D.~{\em Ge.~Wilh. Friedr. Hegel},
Professor und Rector am Königl. Bayeri\-schen Gymnasium zu Ntirnberg.
Erster Band: Die objective Logik, Nurnberg~1812.
\end{enumerate}

{\em Idem}., Erster Band: Die objective Logik; Zweites Buch: Die Lehre
vom Wesen, Nurnberg~1813.

{\em Idem}., Zweiter Band: Die subjective Logik oder Lehre vom Begriff,
Nurnberg~1816.

\begin{enumerate}
\item
{\em G.~W.~F.~Hegel's} Werke, Vollstän\-dige Ausgabe durch einen Verein von
Freunden des Verewigten.
\end{enumerate}

Band~III: Wissen\-schaft der Logik, Heraus\-gegeben von
Leopold von Henning, Teil~I: Die objective Logik; Abteilung~I: Die Lehre
vom Sein, Berlin~1833.

Band~IV: Die objective Logik; Abteilung~2: Die Lehre vom Wesen, Berlin~1834.

Band~V: Wissen\-schaft der Logik; Teil~2: Die subjective Logik oder die Lehre
vom Begriff, Berlin~1834.

\begin{enumerate}
\item
{\em G.~W.~F.~Hegel's} Werke, Voll\-ständige Ausgabe durch einem Verein von
Freunden des Verewigten.
\end{enumerate}

Band~III: Wissen\-schaft der Logik, Heraus\-gegeben von Leopold von Henning;
Teil~I: Die objective Logik; Abteilung~I: Die Lehre vom Sein, 2-te
unver\-änderte Auflage, Berlin~1841.

Band~IV: Die objective Logik; Abteilung~2: Die Lehre vom Wesen, 2-te
unver\-änderte Auflage, Berlin~1841.

Band~V: Wissen\-schaft der Logik; Teil~2: Die subjective Logik oder die Lehre
vom Begriff, 2-te unver\-änderte Auflage, Berlin~1841.

\begin{enumerate}
\item
{\em Hegel}, Wissen\-schaft der Logik, Heraus\-gegeben von Georg Lasson;
Erster Teil (Die Lehre vom Sein), Leipzig~1923.
\end{enumerate}

{\em Idem}., Zweiter Teil (Die Lehre vom Wesen. Die Lehre vom Begriff),
Leipzig~1923.

\begin{enumerate}
\item
{\em Hegel}, Sämtliche Werke, Jubiläums\-ausgabe, heraus\-gegeben von
H.~Glockner.
\end{enumerate}

Band~IV: Wissen\-schaft der Logik; Teil~I: Die objective Logik,
Stuttgart~1928.

Band~V: Wissen\-schaft der Logik; Teil~2: Die subjective Logik,
Stuttgart~1928.

\begin{enumerate}
\item
{\em Hegel}, Wissen\-schaft der Logik, heraus\-gegeben von Georg Lasson,
Erster Teil (Die Lehre vom Sein), 2-te Auflage, Leipzig~1933.
\end{enumerate}

{\em Idem}., Zweiter Teil (Die Lehre vom Wesen. Die Lehre vom Begriff),
2-te Auflage, Leipzig~1934.

\begin{enumerate}
\item
{\em Hegel}, Sämtliche Werke, Jubiläums\-ausgabe, heraus\-gegeben von
H.~Glockner.
\end{enumerate}

Band~IV: Wissen\-schaft der Logik; Teil~I: Die objective Logik, 2-te Auflage,
Stuttgart~1936.

\paragraph%
[II. Переводы большой логики на другие языки]%
{II. Переводы большой логики на другие языки}

\begin{enumerate}
\item
{\em Гегель}, Наука логики, пер. с немецкого Н.~Г.~Дебольского, ч.~I.
Объективная логика; кн.~1: Учение о бытии, Пг.~1916.
\end{enumerate}

То же, ч.~I., кн.~2: Учение о сущности, Пг.~1916.

То же, ч.~II. Субъективная логика или учение о понятии, Пг.~1916.

\begin{enumerate}
\item
{\em Hegel}, La scienza della logica. Traduzione italiana con note di
Arturo Moni, Vol.~1---3, Bari~1925.

\item
{\em Гегель}, Наука логики, Пер. с немецкого Н.~Г.~Дебольского
(перепечатано на правах рукописи с издания 1916~г.), Москва, Профком
слушателей Института Красной профессуры, 1929 (издано одним томом с
раздельной пагинацией по трем частям).

\item
{\em Hegel}, Science of Logic, Translated by W.~H.~Johnston and
L.~G.~Struthers, With an intro\-duction by Haldane of Cloan, vol.~I
(The doctrine of being) and vol.~2 (The doctrine of essence. The doctrine of
the notion), London~1929.

\item{\em Hegel's} doctrine of formal logic, being a translation of the first
section of the subjective logic, with intro\-duction and notes by
H.~S.~Macran, Oxford~1912.

\item{\em Hegel's} logic of world and idea, being a translation of the 2-nd and
3-d parts of the subjective logic, with intro\-nduction on Idealism limited and
absolute by H.~S.~Macran, Oxford~1929.
\end{enumerate}

\paragraph%
[III. Маркс, Энгельс, Ленин о~логике Гегеля]%
{III. Маркс, Энгельс, Ленин о~логике Гегеля}

{\centering А) Маркс и Энгельс \par}

\begin{enumerate}
\item {\em Маркс}, Из критики философии права Гегеля (Соч., т.~I,
Госполитиздат, 1938, стр.~588).
\item {\em Энгельс}, Шеллинг и откровение (Соч., т.~II, М.---Л. 1931,
стр.~129, 139---141).
\item {\em Маркс} и {\em Энгельс}, Святое семейство (Соч., т.~III, М.---Л.
1929, стр.~114, 168).
\item {\em Маркс}, Подготовительные работы для <<Святого семейства>> (Соч.,
т.~III, М.---Л. 1929, стр.~632, 650---652).
\item {\em Маркс} и {\em Энгельс},
Немецкая идеология (Соч., т.~IV, стр.~100, 102, 213, 222, 246---247, 257).
\item {\em Маркс}, Нищета философии (Соч., т.~V, стр.~361---363).
\item {\em Маркс}, К критике политической экономии (Соч., т.~XII, ч.~1-я,
стр.~80).
\item {\em Маркс}, Введение к <<Критике политической экономии>> (Соч., т.~XII,
ч.~1-я, стр.~191---192).
\item {\em Энгельс}, Рецензия на книгу Маркса <<К критике политической
экономии>> (Соч., т.~XI, ч.~2-я, стр.~358---359).
\item {\em Маркс}, Капитал, т.~I (Соч., т.~XVII, стр.~19---20, 199, 288, 339).
\item {\em Маркс}, Капитал, т.~I, изд. 1-е, СПБ 1872, стр.~13, 15, 17, 19.
\item {\em Маркс}, Капитал, т.~III, Партиздат, 1936, стр.~686.
\item {\em Энгельс}, Карл Маркс о капитале (Соч., т.~XIII, ч.~1-я, стр.~258).
\item {\em Энгельс}, К жилищному вопросу (Соч., т.~XV, стр.~60).
\item {\em Энгельс}, Анти-Дюринг, Госполитиздат, 1938, стр.~30, 34, 37---38,
39, 43, 44, 49, 55, 94, 99, 101, 104, 106, 118, 270, 273, 281, 282, 285, 294,
295, 322.
\item {\em Энгельс}, Диалектика природы, Партиздат, 1936, стр.~3, 6, 7, 8, 9, 10,
46, 47, 70, 73, 81, 82, 85, 100---103, 109, 111, 112, 113, 114, 116, 122,
125, 127, 129, 194.
\item {\em Энгельс}, Л.~Фейербах, Госполитиздат, 1938, стр.~6, 9, 19, 35.
\item {\em Энгельс}, Введение к англ. изданию <<Развития социализма>> (Соч.,
т.~XVI, ч.~2-я, стр.~293).
\end{enumerate}

Кроме того, следующие письма:
\begin{enumerate}
\item Энгельс Марксу от 19.~XI. 1844 (Соч., т.~XXI, стр.~7).
\item Маркс Энгельсу от 13.~XI. 1857 (Соч., т.~XXII, стр.~251).
\item Маркс Энгельсу от 8.~XII. 1857 (Соч. т.~XXII, стр.~266).
\item Маркс Энгельсу от 14.~I. 1858 (Соч., т.~XXII, стр.~290---291).
\item Маркс Энгельсу от 1.~II. 1858 (Соч., т.~XXII, стр.~299).
\item Маркс Лассалю от 31.~V. 1858 (Соч., т.~XXV, стр.~229).
\item Энгельс Марксу от 14.~VII. 1858 (Соч., т.~XXII, стр.~345---346).
\item Маркс Энгельсу от 28.~XI. 1860 (Соч., т.~XXII, стр.~542).
\item Маркс Энгельсу от 27.~II. 1861 (Соч., т.~XXIII, стр.~15).
\item Маркс Энгельсу от 9.~XII. 1861 (Соч., т.~XXIII, стр.~51).
\item Энгельс Альберту Ланге от 29.~III. 1865 (Соч., т.~XXV, стр.~452).
\item Маркс Энгельсу от 19.~VIII. 1865 (Соч., т.~XXIII, стр.~305).
\item Энгельс-Марксу от 16.~VI. 1867 (Соч., т.~XXIII, стр.~414---415).
\item Маркс Энгельсу от 22.~VI. 1867 (Соч., т.~XXIII, стр.~417).
\item Маркс Кугельману от 6.~III. 1868 (Соч., т.~XXV, стр.~516).
\item Маркс Энгельсу от 25.~III. 1868 (Соч., т.~XXIV, стр.~34---35).
\item Маркс Энгельсу от 14.~IV. 1870 (Соч., т.~XXIV, стр.~318).
\item Маркс Кугельману от 27.~VI. 1870 (Соч., т.~XXVI, стр.~58).
\item Маркс Энгельсу от 31.~V. 1873 (Соч., т.~XXIV, стр.~415).
\item Энгельс Марксу от 21.~IX. 1874 (Соч., т.~XXIV, стр.~442).
\item Энгельс Марксу от 18.~VIII. 1881 (Соч., т.~XXIV, стр.~531---532).
\item Энгельс Конраду Шмидту от 27.~X. 1890 (Маркс и Энгельс. Письма,
пер. Адоратского, изд. 4-е, М.---Л. 1931, стр.~386).
\item Энгельс К.~Шмидту от 1.~VII. 1891 (Письма, стр.~386).
\item Энгельс К.~Шмидту от 1.~XI. 1891 (Письма, стр.~292---294).
\item Энгельс К.~Шмидту от 4.~II. 1892 (Письма, стр.~394).
\end{enumerate}

{\centering Б) Ленин \par}

\begin{enumerate}
\item Что такое <<друзья народа>> (Соч., т.~I, стр.~80, 82, 84).
\item Шаг вперед, два шага назад (Соч., т.~VI, стр.~325---326).
\item Марксизм и ревизионизм (Соч., т.~XII, стр.~184---185).
\item Материализм и эмпириокритицизм (Соч., т.~XIII, стр.~103---104, 154, 156,
186---187, 190, 253, 276).
\item Письмо Горькому от 16.~XI. 1909 (Соч., т.~XIV, стр.~186).
\item Памяти Герцена (Соч., т.~XV, стр.~464).
\item Три источника и три составных части марксизма (Соч., т.~XVI, стр.~350).
\item Карл Маркс (Соч., т.~XVIII, стр.~10---12).
\item Крах II Интернационала (Соч., т.~XVIII, стр.~247).
\item Еще раз о профсоюзах (Соч., т.~XXVI, стр.~134---135).
\item О значении воинствующего материализма (Соч., т.~XXVII, стр.~187---188).
\item Конспект книги Гегеля <<Наука логики>> (Философские тетради,
Госполитиздат, 1938, стр.~87---228).
\item Замечания на книгу Ж.~Ноэля <<Логика Гегеля>> (Философские тетради,
стр.~229---236).
\item План диалектики (логики) Гегеля (Философские тетради, стр.~237---242).
\item Конспект книги Гегеля <<Лекции по истории философии>>
(Философские тетради, стр.~274 и~283).
\item К вопросу о диалектике (Философские тетради, стр.~323---328).
\item Заметки по отзывам немецких, французских, английских и итальянских
авторов о~логике Гегеля (Философские тетради, стр.~422---425).
\end{enumerate}

\paragraph%
[IV Из буржуазной литературы о логике Гегеля]%
{IV Из буржуазной литературы о логике Гегеля}

\begin{enumerate}
\item{\em Baillie, J.~B.,} The origine and signi\-ficance of Hegel's Logic,
London 1901.
\item{\em Emge, K.~A.,} Hegels Logik und die Gegenwart, Karlsruhe~1927.
\item{\em Hibben, J.~G.,} Hegel's Logic, an essay of an interpre\-tation,
New~York~1902.
\item{\em Mac-Taggart,} A commentary on Hegel's Logic, Cambridge~1910.
\item{\em Michelet} und {\em Haring}, Historisch-kritische Darstellung
der dialekti\-schen Methode Hegels, Leipzig~1888.
\item{\em Noël, G.,} La logique de Hegel, Paris~1897.
\item{\em Schmitt, E.~H.,} Das Geheimniss der Hegelschen Dialektik, beleuchtet
vom konkret-sinnlichen Stand\-punkte, Halle~1888.
\item{\em Stirling. J.~H.,} The secret of Hegel, two volumes, London~1865.
\item{\em Wallace, W.,} Prolegomena to the study of Hegel's philosophy and
especially of his Logic, Oxford and London~1894.
\end{enumerate}

\clearpage
\bigskip
\printpagenotes
\bigskip
\bigskip

{\centering Гегель <<НАУКА ЛОГИКИ>> T.~II. Субъективная логика \par}

\renewcommand\contentsname{}
\section*[Оглавление]{ОГЛАВЛЕНИЕ}
\tableofcontents

\bigskip
\clearpage
\bigskip
