В~мере соединены абстрактно выраженные качество и количество. {\em Бытие} как
таковое есть непосредственное равенство определенности с самою собою. Эта
непосредственность определенности сняла себя. Количество есть бытие,
возвратившееся в себя таким образом, что оно теперь есть простое равенство
с собою как безразличие к определенности. Но это безразличие есть лишь
внешность, характеризующаяся тем, что количество имеет определенность не в себе
самом, а в другом. Следующее за ним теперь третье есть соотносящаяся с самою
собою внешность; как соотношение с собою, оно вместе с тем есть {\em снятая}
внешность и имеет в ней самой отличие от себя, которое как внешность есть
{\em количественный}, а как вобранная обратно в себя "--- {\em качественный}
момент.

Так как {\em модальность} приводится в числе категорий трансцендентального
идеализма после количества и качества, причем между последними и ею вставляется
отношение, то можно упомянуть о ней здесь. Эта категория имеет там то значение,
что она есть отношение {\em предмета к мышлению}. Согласно смыслу учения
указанного идеализма мышление вообще существенно внешне вещи-в-себе. Поскольку
прочие категории имеют лишь то трансцендентальное определение, что принадлежат
{\em сознанию}, но как то, {\em чт\'{о} в~нем объективно}, постольку
модальность, как категория отношения к субъекту, содержит в~себе
в~относительном смысле определение {\em рефлексии} в себя; т.~е. присущая
прочим категориям объективность недостает категориям модальности; последние, по
выражению {\em Канта}, нисколько не умножают понятия как определение объекта,
а~лишь выражают отношение к способности познания (Kr. d. rein. Vernunft,
изд.~2-е, стр.~99, 266). "--- Категории, которые Кант объединяет под названием
модальности, "--- возможность, действительность и необходимость "--- встретятся
нам в дальнейшем в своем месте. Бесконечно важную форму тройственности, "---
хотя она у Канта появляется пока что лишь как формальный луч света (formeller
Lichtfunken), "--- он применил не к родам своих категорий (количество, качество
и~т.~д.), а также не к ним применил и название <<категории>>, но лишь к их
видам; поэтому он не мог найти третьей категории к качеству и количеству.

У~{\em Спинозы модус} есть также третье, следующее за субстанцией и атрибутом;
он его объявляет {\em состояниями} субстанции или тем, чт\'{о} находится
в~другом, через которое оно и постигается. Это третье есть согласно этому
понятию лишь внешность как таковая, и мы уже указали в другом месте, что
у~Спинозы неподвижной субстанциальности недостает возвращения в~себя самоё.

Сделанное нами здесь замечание в более общем виде распространяется на все те
пантеистические системы, которые были до некоторой степени разработаны мыслью.
Бытие, единое, субстанция, бесконечное, сущность "--- есть первое; по отношению
к этой абстракции второе, всякая определенность, может быть вообще столь же
абстрактно охарактеризовано как лишь конечное, лишь акциденциальное,
преходящее, внесущественное и несущественное и~т.~д., как это обычно на первых
порах происходит в совершенно формальном мышлении. Но мысль о связи этого
второго с первым напрашивается так настойчиво, что приходится понимать вместе
с~тем это второе, как находящееся в единстве с первым; так, например, у~Спинозы
{\em атрибут} есть вся субстанция, но субстанция, как ее постигает рассудок,
который сам есть некоторое ограничение или модус; модус же, т.~е.
несубстанциальное вообще, которое может быть постигаемо лишь из некоторого
другого, составляет, таким образом, другую, противоположную крайность к
субстанции, третье вообще. {\em Индийский} пантеизм в своей чрезвычайной
фантастичности, взятый абстрактно, также получил развитие, которое подобно
умеряющей нити тянется через безмерность его фантазии и которое придаёт ей
некоторый интерес, а именно то, что Брахма, единое абстрактного мышления,
переходит через получение облика Вишны, в особенности в форме Кришны, в третье,
в Шиву. Определением этого третьего служит модус, изменение, возникновение и
прехождение, вообще область внешнего. Если эта индийская троица соблазнила
кое-кого сравнивать ее с христианской, то мы должны сказать, что хотя в них
можно распознать общий им элемент определения понятия, однако вместе с тем
существенно важно более определенно осознать различие между ними; это различие
не только бесконечно, но истинная бесконечность и составляет самое это
различие. Третий принцип индийского пантеизма есть по своему определению распад
субстанциального единства, переход его в свою противоположность,
а~{\em не~возвращение его} к~себе, "--- есть скорее бездуховное, чем дух.
В~истинной же троичности имеется не только единство, но и единение,
умозаключение доведено в ней до {\em содержательного} и {\em действительного}
единства, которое в своем совершенно конкретном определении есть {\em дух}.
Вышеуказанный принцип модуса и изменения, правда, не исключает вообще единства.
Так, в спинозизме именно модус как таковой есть неистинное, и лишь субстанция
есть истинное, все должно быть сведено к этой последней, и это сведение
оказывается погружением всяческого содержания в пустоту, в лишь формальное
бессодержательное единство. Точно так же и Шива есть в свою очередь великое
целое, не отличное от Брахмы, сам Брахма, т.~е. различие и определенность лишь
снова исчезают, но не сохраняются, не снимаются, и единство не сводится к
конкретному единству, раздвоение не приводится обратно к применению. Высшая
цель для человека, ввергнутого в сферу возникновения и прохождения, вообще в
область модальности, есть погружение в бессознательность, единство с Брахмой,
уничтожение; то же самое представляет собою буддистская нирвана, ниббана
и~т.~п.

Если модус есть вообще абстрактная внешность, безразличие как к качественным,
так и к количественным определениям, и внешнее, несущественное считается не
имеющим важности в сфере сущности, то, с другой стороны, касательно многого
признается, что все зависит от {\em вида и способа;} этим сам модус объявляется
существенно принадлежащим к субстанциальной стороне вещи, а это весьма
неопределенное отношение означает по меньшей мере то, что это внешнее не есть
столь абстрактно внешнее.

Здесь модус имеет определенный смысл {\em меры}. Спинозовский модус,
как и индусский принцип изменения, есть безмерное. Греческое еще
неопределенное сознание того, что {\em все имеет меру}, так что даже
Парменид ввел после абстрактного бытия {\em необходимость}, как
{\em всем вещам поставленную древнюю границу}, это сознание
заключает в~себе начаток гораздо более высокого понятия, чем
субстанция и различие от нее модуса.

Более развитая, более рефлектированная мера есть необходимость; судьба,
{\em немезида}, сводится в общем к определенности меры именно в~том смысле,
что то, чт\'{о} {\em дерзновенно превозносится}, чт\'{о} делает себя слишком
высоким, слишком великим, приводится ею к другой крайности, унижается,
доводится до ничтожности и тем самым восстанавливается средняя мера,
посредственность. "--- <<Абсолютное, бог есть {\em мера} всех вещей>>, "--- это
положение не более пантеистично, чем дефиниция: <<абсолютное, бог есть
{\em бытие}>>, но первое бесконечно более истинно. "--- Мера есть, правда,
внешний вид и способ, некоторое <<больше>> или <<меньше>>, но она вместе с~тем
также и рефлектирована в себя, есть не только безразличная и внешняя, но и
в-себе-сущая определенность; она, таким образом, есть {\em конкретная истина
бытия;} народы поэтому почитали в мере нечто неприкосновенное, святое.

В мере уже подготовлена идея {\em сущности}, а именно в ней подготовлено
тождество с самим собой в непосредственной определенности, так что сказанная
непосредственность понижается через это тождество с собою до некоторого
опосредствованного, равно как тождество с собою также опосредствовано лишь
через эту внешность, но есть опосредствование {\em с собою;} это "---
рефлексия, определения которой {\em суть}, но даны (sind) в этом бытии
безоговорочно лишь как моменты ее отрицательного единства. В~мере качественное
количественно; определенность или различие дано (ist) в ней как безразличное;
тем самым оно есть такое различие, которое не есть различие; оно снято; эта
количественность как возвращение в себя, в котором она дана (ist) как
качественное составляет в-себе-и-для-себя-бытие, которое есть {\em сущность}.
Но мера есть сущность пока что лишь {\em в~себе} или, иначе говоря, в понятии;
это {\em понятие} меры еще не {\em положено}. Мера, еще как таковая, сама есть
{\em сущее} единство качественного и количественного; ее моменты суть, как
некоторое наличное бытие, некоторое качество и определенные количества этого
качества, которые (качество и количество) пока что лишь в~себе неотделимы, но
еще не имеют значения этого рефлектированного определения. Развитие меры
заключает в себе различение этих моментов, но вместе с тем и их
{\em соотнесение}, так что то тождество, которое они суть {\em в~себе},
становится их взаимным отношением друг к другу, т.~е. становится
{\em положенным}. Смыслом (die Bedeutung) этого развития является реализация
меры, в которой она полагает себя в отношении к себе самой и тем самым полагает
себя вместе с тем как момент; через это опосредствование она определяется как
снятая; ее непосредственность, как и непосредственность ее моментов, исчезает;
они оказываются рефлектированными; таким образом, выступив как то, чт\'{о} она
есть по своему понятию, она перешла в~{\em сущность}.

Мера есть прежде всего {\em непосредственное} единство количественного
и качественного, так что,

{\em во-первых}, имеется {\em одно определенное количество}, которое имеет
качественное значение и выступает {\em как мера}. Ее дальнейшее определение
заключается в том, что {\em в~ней}, во {\em в~себе} определенном, выступает
различие ее моментов, качественной и количественной определенности. Эти моменты
сами определяются далее в целые меры, которые постольку имеют бытие как
{\em самостоятельные;} поскольку они по существу соотносятся друг с другом,
мера становится,

{\em во-вторых, отношением} специфических определенных количеств
{\em как самостоятельных мер}. Но их самостоятельность вместе с тем покоится по
существу на количественном отношении и различии по величине. Таким образом, их
самостоятельность становится переходом друг в друга. Мера тем самым идет ко
дну, погружается (geht zu Grunde) в~{\em безмерном}. "--- Но это потустороннее
меры есть ее отрицательность лишь в~себе самой; поэтому,

{\em в-третьих}, положена неразличенность (индиференция) определений меры и,
как реальная, мера с содержащейся в этой неразличенности отрицательностью
положена как {\em обратное отношение мер}, которые как самостоятельные качества
существенно покоятся лишь на своем количестве и на своем отрицательном
соотношении друг с другом, и тем самым оказывается, что они суть лишь моменты
их истинно самосостоятельного единства, которое есть их рефлексия в себя и
полагание последней, "--- {\em сущность}.

Развитие меры, как мы его попытались изложить в последующем, есть одна из
труднейших материй; начинаясь с непосредственной, внешней меры, оно должно было
бы, с одной стороны, перейти далее к абстрактному дальнейшему определению
количественного ({\em к математике природы}), а, с другой стороны, вскрыть
связь этого определения меры с {\em качествами} вещей природы "--- по крайней
мере в общем виде, ибо определенное доказательство проистекающей из понятия
конкретного предмета {\em связи} качественного и количественного есть дело
особых наук о конкретном (примеры таких доказательств, касающиеся закона
падения тел и закона свободного движения небесных тел, смотри в Энциклопедии
философских наук, изд.~3-е, \S~267 и 270 и примечания к ним). При этом уместно
заметить вообще, что различные формы, в которых реализуется мера, принадлежат
также различным {\em сферам природной реальности}. Полное, абстрактное
безразличие развитой меры, т.~е. ее {\em законов}, может иметь место только в
сфере {\em механизма}, в котором конкретно телесное есть лишь сама являющаяся
абстрактной материя; качественные различия материи имеют по существу своей
определенностью количественное; {\em пространство} и {\em время} суть сами
чистые внешности, а {\em множество} (die Menge) материй, массы, интенсивность
{\em веса} точно так же суть внешние определения, имеющие свою своеобразную
определенность в количественном. Напротив, такая определенность величины
абстрактно-материального уже в области {\em физики}, а еще больше
в~{\em органической} природе, нарушается множественностью и, значит, конфликтом
качеств. Но здесь не только появляется конфликт между качествами как таковыми,
а мера подчиняется здесь более высоким отношениям, и {\em имманентное развитие}
меры сводится скорее к простой форме непосредственной меры. Члены животного
организма имеют меру, которая как некоторое простое определенное количество
находится в отношении к другим определенным количествам других членов;
пропорции человеческого тела суть прочные отношения таких определенных
количеств; естествознанию еще предстоит задача проникнуть в связь таких величин
с органическими функциями, от которых они целиком зависят. Но ближайшим
примером понижения некоторой имманентной меры на степень исключительно внешним
образом детерминированной величины служит {\em движение}. В~небесных телах оно
есть свободное, определяемое лишь понятием движение, величины которого тем
самым также находятся в зависимости только от понятия (см. выше), но
органическими существами оно понижается до {\em произвольного} или
механически-правильного, т.~е. вообще до абстрактного, формального движения.

Но еще в меньшей степени находит себе место своеобразное, свободное развитие
меры в царстве духа. Легко, например, усмотреть, что такой республиканский
государственный строй, как например афинский или строй аристократический,
смешанный с демократией, может иметь место лишь при известной величине
государства; что в~развитом гражданском обществе количества индивидов, занятых
в различных промыслах, находятся между собою в известном отношении; но~это
не~дает ни законов мер, ни особых форм этого отношения. В~области духовного
как такового мы встречаем различия {\em интенсивности} характера, {\em силы}
воображения, чувств, представлений и~т.~п.; но за пределы этой неопределенной
характеристики <<{\em силы}>> или <<{\em слабости}>> определение не выходит.
Какими тусклыми и совершенно пустыми оказываются так называемые законы,
устанавливаемые касательно отношения силы и слабости ощущений, представлений
и~т.~д., мы убеждаемся, обратившись к руководствам по психологии, старающимся
найти такого рода законы.

\hegchapter[Первая глава]{Специфическое количество}

Качественное количество есть {\em ближайшим образом} непосредственное
{\em специфическое определенное количество}, которое

{\em во-вторых}, как относящееся к другому, становится некоторым количественным
специфицированием, снятием безразличного определенного количества. Постольку
эта мера есть {\em правило} и содержит в себе {\em различенными оба момента}
меры, а именно, в-себе-сущую количественную определенность и внешнее
определенное количество. Но в этом различии эти две стороны становятся
качествами, а правило "--- некоторым отношением этих качеств; мера поэтому
представляется

{\em в-третьих, отношением качеств}, имеющих ближайшим образом одну меру,
которая, однако, затем специфицируется внутри себя в некоторое различие мер.

\hegsection[А. Специфическое определенное количество]%
{А. Специфическое определенное количество}

1. Мера есть простое соотношение определенного количества с собою, его
собственная определенность в самом себе; таким образом, определенное количество
качественно. Ближайшим образом мера, как непосредственная мера, есть некоторое
непосредственное и поэтому некоторое определенным образом определенное
количество; столь же непосредственным является и сопряженное с ним качество,
оно есть какое-нибудь определенное качество. "--- Определенное количество, как
эта уже более не безразличная граница, как соотносящаяся с собою внешность,
само, таким образом, есть качество и, будучи отличным от последнего, оно не
простирается дальше его, равно как и это качество не идет дальше этого
определенного количества. Оно есть, таким образом, возвратившаяся в простое
равенство с собою определенность; оно едино с определенным наличным бытием,
точно так же, как это последнее едино со своим определенным количеством.

Если из полученного теперь определения хотят образовать предложение, то можно
выразиться так: {\em все налично сущее имеет некоторую меру}. Всякое наличное
бытие обладает некоторой величиной, и эта величина принадлежит к самой природе
нечто; она составляет его определенную природу и его внутри-себя-бытие. Нечто
не безразлично к этой величине, не остается тем, чт\'{о} оно есть, если
изменяется эта величина, а изменение последней изменяет его качество.
Определенное количество как мера перестало быть такой границей, которая не~есть
граница; оно есть отныне определение вещи, так что если увеличить или уменьшить
эту вещь за пределы этого определенного количества, она погибнет.

Мера как масштаб в обычном смысле есть некоторое определенное количество,
которое произвольно принимается за {\em в себе определенную} единицу по
отношению к внешней численности. Такого рода единица может, правда, и в самом
деле быть определенной в себе единицей, как например,
{\em фут}\pagenote{Английское слово <<фут>> означает прежде всего <<нога,
ступня>>, а затем уже <<фут>> в смысле меры длины, приблизительно
соответствующей длине ступни человека ($30{,}5$~{\em см}). То же самое имеет
место и в немецком языке со словом <<Fuss>>.} и тому подобные первоначальные
меры; однако, поскольку она употребляется вместе с тем и как масштаб для других
вещей, она для них "--- только внешняя, а не первоначальная мера. "--- Так,
например, диаметр земли или длина маятника могут быть сами по себе
рассматриваемы как специфические определенные количества; но решение брать
именно такую-то часть диаметра земли или длины маятника и употреблять последнюю
как масштаб именно на таком-то градусе широты является произвольным. Но еще в
большей степени такого рода масштаб является чем-то внешним для других вещей.
Последние специфицировали всеобщее специфическое определенное количество
опять-таки на особый лад и этим сделались особыми вещами. Нелепо поэтому
говорить о естественном {\em масштабе} вещей. Да и помимо этого всеобщий
масштаб должен, как полагают, служить лишь для внешнего {\em сравнения;} в~этом
наиболее поверхностном смысле, в котором он принимается {\em за всеобщую меру},
совершенно безразлично, чт\'{о} для этого употребляется. Это "--- не основная
мера в том смысле, что в~ней представлены естественные меры особых вещей и из
нее последние познаются согласно некоторому правилу как спецификации единой
всеобщей меры, меры их всеобщего тела. Но без этого смысла абсолютный масштаб
имеет лишь интерес и значение некоторого {\em общего}
(eines Gemein\-schaft\-li\-chen), а~таковое есть всеобщее
не~{\em в~себе}, а~только по соглашению.

Эта непосредственная мера есть некоторое простое определение величины, как
например величина органических существ, их членов и~т.~д. Но всякое
существующее, чтобы быть тем, чт\'{о} оно есть, и чтобы вообще обладать
существованием, имеет некоторую величину. "--- Как определенное количество, она
есть безразличная величина, открытая внешнему определению и способная
подниматься по лестнице большего и опускаться по лестнице меньшего. Но как
мера, она вместе с тем отлична от себя самой как определенного количества, как
такого безразличного определения, и есть ограничение этого безразличного
движения взад и вперед вдоль границы.

Поскольку, таким образом, количественная определенность оказывается в наличном
бытии двоякой "--- с одной стороны, такой определенностью, с которой связано
качество, а с другой стороны, такой определенностью, вдоль которой, не нанося
ущерба качеству, можно двигаться взад и вперед, "--- постольку гибель того
нечто, которое имеет меру, может произойти в результате того, что изменяется
его определенное количество. Эта гибель представляется, с одной стороны,
{\em неожиданной}, поскольку можно ведь вносить изменения в определенное
количество, не изменяя меры и качества, а с другой стороны, ее делают чем-то
совершенно понятным, посредством понятия {\em постепенности}. К~этой категории
охотно прибегают, чтобы сделать представимым или <<{\em объяснить}>>
прехождение какого-либо качества или нечто, так как кажется, что таким образом
можно почти видеть своими глазами процесс исчезания, потому что определенное
количество берется как внешняя, по своей природе изменчивая граница, и, стало
быть, {\em изменение} как изменение исключительно только определенного
количества само собою понятно. Но на самом деле этим ничего не объясняется;
изменение есть вместе с тем по существу переход одного качества в другое, или
более абстрактный переход от наличного бытия в отсутствие наличного бытия; в
этом заключается иное определение, чем в постепенности, которая есть лишь
уменьшение или увеличение и одностороннее цепляние за величину.

2. Но что изменение, выступающее как чисто количественное, переходит также и
в~качественное, "--- на эту связь обратили внимание уже древние и представили
коллизии, возникающие на почве незнания этого обстоятельства, в популярных
примерах. Относящиеся сюда <<{\em эленхи}>>, т.~е. согласно объяснению
Аристотеля способы, посредством которых вынуждаются сказать противоположное
тому, чт\'{о} утверждали до этого, известны под названием <<лысый>>, <<куча>>.
Задавался вопрос: получается ли лысина, если выдернуть один волос из головы или
из лошадиного хвоста, или: перестает ли куча быть кучей, если возьмем из нее
одно зернышко? Можно не задумываясь согласиться с~тем, что при этом не
получается лысины и что куча не перестает быть кучей, так как такое отнятие
составляет только количественную и притом даже весьма незначительную разницу;
таким образом отнимают один волос, одно зернышко и повторяют это так, что
всякий раз, следуя тому, с~чем согласились, отнимается лишь один волос или одно
зернышко; под конец обнаруживается качественное различие: голова, хвост
становятся лысыми, куча исчезает. Когда соглашались, что отнимание одного
волоса не делает лысым и~т.~д., забывали не только о повторении, но и о~том,
что сами по себе незначительные количества (например, сами по себе
незначительные траты состояния) {\em суммируются} и сумма составляет
качественное целое, так что под конец это целое оказывается исчезнувшим, голова
сделалась лысой, кошелек опустел. Затруднение, противоречие, получающееся
в~результате, не есть нечто софистическое в обычном смысле этого слова, не~есть
ложная уловка, введение в обман. Ложным является то, чт\'{о} совершает
предполагаемый другой собеседник, т.~е. наше обыденное сознание, принимающее
количество лишь за безразличную границу, т.~е. берущее его именно в~том
определенном смысле, в~каком оно есть количество как таковое. Это предположение
изобличается как ложное той истиной, к которой оно приводится, истиной,
заключающейся в~том, что количество есть момент меры и находится в~связи
с~качеством; чт\'{о} здесь опровергается, "--- это одностороннее цепляние за
абстрактную определенность определенного количества. "--- Вышеуказанные
обороты рассуждения и не~суть поэтому пустая или педантическая шутка,
а~внутренне правильны и являются порождениями сознания, интересующегося
явлениями, встречающимися в области мышления.

Определенное количество в том случае, когда его принимают за безразличную
границу, есть та сторона, с которой нечто существующее (ein Dasein)
подвергается неожиданному нападению и неожиданной гибели. В~том-то
и~заключается {\em хитрость} понятия, что она схватывает существующее с~той
стороны, с которой, как ему кажется, его качество нисколько не затрагивается
и~притом настолько не затрагивается, что увеличение государства, состояния
и~т.~д., ввергающее государство, собственника в несчастье, сначала даже кажется
его счастьем.

3. Мера есть в своей непосредственности некоторое обычное качество, обладающее
определенной, принадлежащей ему величиной. От той стороны, по которой
определенное количество есть безразличная граница, вдоль которой можно, не
изменяя качества, ходить взад и вперед, отлична его другая сторона, по которой
оно качественно, специфично. Обе стороны суть определения величины одного и
того же. Но согласно той непосредственности, которая первоначально присуща
мере, мы должны, далее, брать это различие как непосредственное; обе стороны
имеют согласно этому также и разное существование. Тогда то существование меры,
которое есть определенная {\em в~себе} величина, есть в своем отношении к
существованию изменчивой, внешней стороны снятие своего безразличия,
{\em специфицирование} меры.

\hegsection[В. Специфицирующая мера]{В. Специфицирующая мера}

Эта последняя есть,

{\em во-первых}, некоторое правило, некоторая мера, внешняя
по отношению к голому определенному количеству;

{\em во-вторых}, специфическое количество, определяющее собою внешнее
определенное количество;

{\em в-третьих, обе стороны}, как {\em качества}, которым присуща
специфическая количественная определенность, относятся друг к другу, как
{\em единая} мера.

\subsection[а) Правило]{а) Правило\pagenote{Слово <<правило>> (die Regel)
Гегель употребляет здесь в смысле <<мерило>>, <<масштаб>>, <<норма>>,
<<образцовая или указная мера>> (Massregel, Richtmass). В~XVIII~в. слово
<<Regel>> иногда употреблялось в смысле линейки с делениями. Гегель,
по-видимому, и намекает на это старинное значение.}}

Правило, или масштаб, о~котором мы уже говорили, есть прежде всего некоторая в
себе определенная величина, служащая единицей по отношению к некоторому
определенному количеству, которое есть особое существование, существует в
некотором другом нечто, чем нечто, служащее масштабом, и измеряется последним,
т.~е. определяется как численность указанной единицы. Это сравнение есть
некоторое внешнее действие; сама та единица есть произвольная величина, которую
в свою очередь можно положить как численность (фут, например, как определенное
число дюймов). Но мера есть не только внешнее правило, но, как специфическая,
она состоит в том, чтобы в себе самой относиться к своему другому, которое есть
некоторое определенное количество.

\subsection[b) Специфицирующая мера]{b) Специфицирующая мера}

Мера есть специфическое определение {\em внешней}, т.~е. безразличной величины,
полагаемой теперь некоторым другим существованием в то нечто, которое служит
мерой и которое, хотя само оно есть определенное количество, все же в отличие
от такового есть нечто качественное, нечто такое, чт\'{о} определяет
исключительно безразличное, внешнее определенное количество. Нечто имеет в нем
ту сторону бытия-для-другого, которой присуще безразличное увеличение и
уменьшение. Указанное имманентное измеряющее есть такое присущее данному нечто
качество, которому противостоит то же самое качество в другом нечто, но в
последнем это качество имеется ближайшим образом с относительно безмерным
определенным количеством вообще, в противоположность первому качеству, которое
определено как измеряющее.

В нечто, поскольку оно есть мера внутри себя, приходит извне некоторое
изменение величины его качества; оно не принимает оттуда {\em арифметического}
множества. Его мера этому противодействует, держится по отношению к множеству
как некоторое интенсивное и вбирает его своеобразным способом; она изменяет
положенное извне изменение, делает из этого определенного количества некоторое
другое и {\em являет} себя через эту спецификацию для-себя-бытием в этой
внешности. "--- Это {\em специфически-вобранное} множество само есть некоторое
определенное количество, также зависимое от другого множества или, иначе
говоря, от последнего как лишь {\em внешнего множества}. Специфицированное
множество поэтому также изменчиво, но не есть вследствие этого определенное
количество как таковое, а есть внешнее определенное количество,
специфицированное константным образом. Таким образом, мера имеет свое наличное
бытие как {\em отношение} и специфическое в ней есть вообще {\em показатель}
этого отношения.

В {\em интенсивном} и {\em экстенсивном} определенном количестве имеется, как
оказалось при рассмотрении этих определений, {\em одно и то же} определенное
количество, которое в одном случае дано в форме интенсивности, а в другом "---
в форме экстенсивности. Лежащее в основании определенное количество не
подвергается в этом различии никакому изменению, это различие есть лишь
некоторая внешняя форма. Напротив, в специфицирующей мере определенное
количество то берется в его непосредственной величине, то оно, проходя через
показатель отношения, берется в некоторой другой численности.

Показатель, составляющий специфическое, может на первый взгляд показаться
постоянным определенным количеством, как частное отношения между внешним и
качественно-определенным количеством. Но в таком случае он был бы не чем иным,
как некоторым внешним определенным количеством; под словом <<показатель>> здесь
следует понимать не~что иное, как момент самого качественного, специфицирующий
определенное количество, как таковое. Собственным имманентным качественным
моментом определенного количества служит, как это оказалось выше, лишь
{\em степенн\'{о}е определение}. Степенн\'{о}е определение и должно быть тем,
чт\'{о} конституирует рассматриваемое отношение и чт\'{о}, в качестве
в-себе-сущего определения, противостоит здесь определенному количеству как
внешнему характеру. Последнее имеет своим принципом нумерическую единицу,
составляющую его в-себе-определенность; соотношение нумерической единицы есть
внешнее соотношение, и изменение, определяемое лишь природой непосредственного
определенного количества как такового, и состоит само по себе в присовокуплении
такой нумерической единицы и снова такой же единицы и~т.~д. Таким образом, если
внешнее определенное количество изменяется в арифметической прогрессии, то
специфицирующая реакция качественной природы меры порождает другой ряд, который
соотносится с первым, возрастает и убывает вместе с ним, но не в отношении,
определяемом некоторым численным показателем, а в отношении, несоизмеримом
с~каким бы то ни было числом, соотносится согласно некоторому степенн\'{о}му
определению.

\subsubsection[Примечание]{\centering {\lsstyle\mdseries Примечание}}

Чтобы привести пример, укажем на {\em температуру;} она представляет собою
такое {\em качество}, в котором различаются эти обе стороны, "--- то
обстоятельство, что она есть внешнее определенное количество, и то
обстоятельство, что она есть специфицированное определенное количество. Как
определенное количество она есть внешняя температура (и~притом тоже некоторого
тела как некоторой общей среды), относительно каковой температуры принимается,
что ее изменение происходит по шкале арифметической прогрессии и что она
равномерно возрастает или убывает; напротив, разными находящимися в ней
отдельными телами температура эта воспринимается по-разному, так как они
определяют воспринятую извне температуру своей имманентной мерой, и их
температурные изменения не находятся в прямом отношении с изменением
температуры среды или между собою. Сравнение разных тел, подвергающихся
действию одной и той же температуры, дает числовые отношения их специфических
теплот, их теплоемкости. Но эти теплоемкости тел неодинаковы в разных
температурах, и с этим связано изменение специфического состояния. В~увеличении
или уменьшении температуры сказывается, следовательно, некоторая особая
спецификация. Отношение температуры, представляемой как внешняя, к температуре
некоторого определенного тела, находящейся вместе с тем в зависимости от первой
температуры, не имеет неизменного показателя отношения; увеличение или
уменьшение этой последней теплоты не идет равномерно с возрастанием и убыванием
внешней температуры. "--- При этом некоторая температура принимается вообще за
внешнюю, изменение которой лишь внешне или чисто количественно. Она, однако,
сама есть температура воздуха или какая-нибудь другая специфическая
температура. Поэтому при более близком рассмотрении следовало бы, собственно
говоря, брать это отношение не как отношение просто количественного к
некоторому качественно-определенному, а как отношение двух специфических
определенных количеств. И~в самом деле, в непосредственно последующем изложении
специфицирующего отношения тотчас же выяснится, что моменты меры состоят не
только в некоторой количественной и некоторой качественно-количественной
сторонах одного и того же качества, а в отношении двух качеств, которые в самих
себе суть меры.

\subsection[с) Отношение обеих сторон как качеств]%
{с) Отношение обеих сторон как качеств}

1. Качественная, в себе определенная сторона определенного количества дана лишь
как соотношение с внешне количественным; как специфицирование последнего она
есть снятие его внешности, через которую определенное количество имеет бытие
как таковое; она, таким образом, имеет последнее своей предпосылкой и начинает
с него. Но количество само отлично от качества также и качественно. Это их
различие должно быть положено в той {\em непосредственности} бытия вообще, в
стадии которой мера еще находится; взятые таким образом обе стороны качественны
по отношению друг друга, и каждая есть сама по себе такого рода наличное бытие;
и то одно определенное количество, которое ближайшим образом представляет собою
лишь формальное, неопределенное в себе определенное количество, есть
определенное количество некоторого нечто и его качества, а так как теперь
взаимоотношение обеих сторон определилось в меру вообще, то оно есть также и
специфическая величина этих качеств. Согласно определению меры эти качества
находятся во взаимном отношении друг к другу; определение меры есть их
показатель; но они в себе соотнесены друг с другом уже в {\em для-себя-бытии}
меры; определенное количество имеет двойное бытие, есть внешнее и
специфическое, так что каждое из различенных количеств заключает в~себе это
двойственное определение и вместе с тем безоговорочно переплетено с~другим;
именно только в этом и заключается определенность обоих качеств. Они, таким
образом, не суть лишь вообще сущее друг для друга наличное бытие, а положены
нераздельными, и связанная с ними определенность величины есть некоторая
качественная единица, "--- {\em одно} определение меры, в котором они согласно
своему понятию, в себе, связаны друг с другом. Мера есть, таким образом,
{\em имманентное} количественное отношение друг к другу {\em двух} качеств.

2. В~мере появляется существенное определение {\em переменной величины}, ибо
она есть определенное количество как снятое и, стало быть, уже более не есть
то, чем оно должно быть чтобы быть определенным количеством, а есть
определенное количество и вместе с тем нечто другое; этим другим служит
качественное и, как было определено, это "--- не~что иное, как его
степенн\'{о}е отношение. В~непосредственной мере это изменение еще не положено;
качество связано лишь с каким-либо и притом отдельным определенным количеством
вообще. В~специфицировании меры, т.~е. в предыдущем определении, где
исключительно внешнее определенное количество подвергалось изменению со стороны
качественного момента, положена различенность этих двух определений величины и
тем самым вообще множественность мер в некотором общем им внешнем определенном
количестве. Определенное количество впервые являет себя налично сущей мерой
лишь в~такой своей различенности от самого себя, когда оно, будучи одним и тем
же (например, той же самой температурой среды), вместе с~тем выступает как
разное и притом количественное наличное бытие (в~разных температурах
находящихся в~этой среде тел). Эта различность определенного количества
в~разных качествах "--- в~разных телах "--- дает дальнейшую форму, ту форму
меры, в~которой обе стороны относятся друг к другу, как качественно
определенные количества, чт\'{о} можно назвать {\em реализованной мерой}.

Величина как некоторая величина вообще переменна, ибо ее определенность имеет
бытие как некоторая граница, которая вместе с~тем не~есть граница; постольку
изменение затрагивает лишь некоторое особое определенное количество, на место
которого ставится некоторое другое определенное количество; но истинным
изменением является лишь изменение определенного количества как такового;
отсюда получается понимаемое таким образом интересное определение переменной
величины в высшей математике; причем не приходится ни останавливаться на
формальной стороне, на {\em переменности} вообще, ни привлекать другие
определения, кроме того простого определения понятия, по которому
{\em другим определенного количества} служит лишь {\em качественное}. Стало
быть, истинное определение реальной переменной величины заключается в~том, что
она есть величина, определяемая качественно и, следовательно, как мы это
достаточно показали, определяемая степенн\'{ы}м отношением. В~этой переменной
величине {\em положено}, что определенное количество значимо не как таковое,
а~по своему другому для него определению, по качественному определению.

Стороны этого отношения имеют по своей абстрактной стороне, как качества
вообще, какое-нибудь особенное значение, например пространства и времени.
Взятые ближайшим образом вообще в отношении их мер, как определенности
величины, одна из них есть численность, увеличивающаяся и уменьшающаяся во
внешней, арифметической прогрессии, а другая есть численность, специфически
определяемая первой, которая служит для нее единицей. Если бы каждая из них
была лишь некоторым особенным качеством вообще, то между ними не было бы
различия, по которому можно было бы сказать, какая из этих двух должна быть
принимаема в отношении ее количественного определения за чисто внешне
количественную и какая "--- за изменяющуюся при количественной спецификации.
Если они, например, относятся между собою, как квадрат и корень, то
безразлично, в какой из них мы рассматриваем увеличение и уменьшение как чисто
внешнее, нарастающее в арифметической прогрессии, и какая из них
рассматривается, напротив, как специфически определяющая себя в этом
определенном количестве.

Но качества не суть неопределенно разные в отношении друг друга, ибо в них как
моментах меры должно заключаться окачествование последней. Ближайшая
определенность самих качеств заключается в~том, что одно есть
{\em экстенсивное}, внешность в~самой себе, а другое "--- {\em интенсивное},
внутри-себя-сущее, или, иначе сказать, отрицательное по отношению к первому. Из
количественных моментов на долю первого приходится согласно этому численность,
а на долю второго "--- единица; в простом прямом отношении первое должно быть
принимаемо за делимое, а второе "--- за делитель, а в специфицирующем отношении
"--- первое за степень или за становление другим и второе "--- за корень.
Поскольку здесь еще занимаются счетом, т.~е. обращают внимание на внешнее
определенное количество (которое, таким образом, есть совершенно случайная,
эмпирически называемая определенность величины) и, стало быть, изменение также
принимается за нарастающее во внешней, арифметической прогрессии, постольку,
это изменение падает на ту сторону, которая служит единицей, на интенсивное
качество; внешнюю же, экстенсивную, сторону мы, напротив, должны представлять
изменяющейся в специфицированном ряду. Но прямое отношение (как, например,
скорость вообще, $\frac s t$) снижено здесь до формального, не существующего,
принадлежащего лишь абстрагирующей рефлексии определения; и если в отношении
корня и квадрата (как например, в $s\hm=at^2$) мы все еще должны принимать
корень за эмпирическое определенное количество, возрастающее в арифметической
прогрессии, а другую сторону отношения за специфицированную, то высшая, более
соответствующая понятию реализация окачествования количественного состоит
в~том, что обе стороны относятся между собою в высших степенн\'{ы}х
определениях (как это, например, имеет место в~$s^3\hm=at^2$).

\subsubsection[Примечание]{\centering {\lsstyle\mdseries Примечание}}

Данное нами здесь разъяснение касательно связи качественной природы некоторого
существования (eines Daseins) и его количественного определения в мере находит
свое применение в уже указанном вкратце примере движения; это применение
заключается прежде всего в том, что в~{\em скорости}, как прямом отношении
пройденного пространства и протекшего времени, величина времени принимается за
знаменатель, а величина пространства, напротив, "--- за числитель. Если
скорость есть вообще лишь отношение между пространством и временем некоторого
движения, то безразлично, какой из этих двух моментов рассматривается как
численность и какой как единица. Но на самом деле пространство так же, как в
удельной тяжести вес, есть внешнее реальное целое вообще и, стало быть,
численность; время же, точно так же как объем, есть, напротив, идеализованное,
отрицательное, сторона, служащая единицей. "--- Но существенным применением
служит здесь то более важное отношение, что в~{\em свободном движении} "---
прежде всего в еще обусловленном движении {\em падения} тел "--- количество
времени и количество пространства определены друг относительно друга первое как
корень, а второе как квадрат,"--- или в абсолютно свободном движении небесных
тел время обращения и расстояние "--- первое на одну степень ниже, чем второе,
"--- определены друг относительно друга первое как квадрат, второе как куб.
Подобные основные отношения покоятся на природе находящихся в отношении качеств
пространства и времени и на роде соотношения, в котором они находятся, зависят
от того, является ли это отношение механическим движением, т.~е. несвободным,
не определяемым понятием моментов, или падением, т.~е. условно свободным
движением, или, наконец, абсолютно свободным небесным движением, каковые роды
движения, точно так же как и их законы, покоятся на развитии понятия их
моментов, пространства и времени, так как эти качества как таковые оказываются
{\em в~себе}, т.~е. в понятии, {\em нераздельными}, и их количественное
отношение есть {\em для-себя-бытие} меры, есть лишь {\em одно} определение
меры.

По поводу абсолютных отношений меры следует сказать, что
{\em математика природы}, если она хочет быть достойной этого имени, по
существу должна быть наукой о мерах, наукой, для которой эмпирически,
несомненно, сделано очень много, но собственно научно, т.~е. философски,
сделано еще весьма мало. {\em Математические начала философии природы}, как
Ньютон назвал свое сочинение, если они должны выполнять это назначение в более
глубоком смысле, чем тот, в котором это делали он и все пошедшее от Бэкона
поколение философов и ученых, должны были бы содержать в себе нечто совсем
иное, чтобы внести свет в эти еще темные, но в высшей степени достойные
рассмотрения области. Велика заслуга познакомиться с эмпирическими числами
природы, например, с расстояниями планет друг от друга; но бесконечно
б\'{о}льшая заслуга состоит в том, чтобы заставить исчезнуть эмпирические
определенные количества и возвести их во {\em всеобщую форму} количественных
определений так, чтобы они стали моментами некоторого {\em закона} или
некоторой меры, "--- бессмертные заслуги, которые приобрели себе, например,
{\em Галилей} относительно падения тел и {\em Кеплер} относительно движения
небесных тел. Они так {\em доказали} найденные ими законы, что показали, что им
соответствует весь объем подробностей, доставляемых восприятием. Но следует
требовать еще высшего {\em доказывания} этих законов, а именно не чего иного,
как того, чтобы их количественные определения были познаны из качеств или,
иначе говоря, из соотнесенных друг с другом определенных понятий (как,
например, пространство и время). Этого рода доказательств еще нет и следа в
указанных математических началах {\em философии природы}, равно как и в
дальнейших подобного рода работах. Выше, по поводу видимости математических
доказательств встречающихся в природе отношений, "--- видимости, основанной на
злоупотреблении бесконечно малым, "--- мы заметили, что попытка вести такие
доказательства собственно {\em математически}, т.~е. не черпая их ни из опыта,
ни из понятия, есть бессмысленное предприятие. Эти доказательства
{\em предполагают наперед} свои теоремы, т.~е. как раз сказанные законы, исходя
из опыта; и они лишь приводят эти законы к абстрактным выражениям и удобным
формулам. Всю приписываемую {\em Ньютону} реальную заслугу, в которой видят его
преимущество перед {\em Кеплером} по отношению к одним и тем же предметам, если
отвлечься от мнимого здания доказательств, несомненно придется в конце концов
(когда наступит более очищенное соображение относительно того, чт\'{о} сделала
математика и чт\'{о} она в состоянии сделать) ограничить с ясным пониманием
сути дела тем, что он дал известное {\em преобразование выражения}\footnote{См.
<<Энциклопедию философских наук>>, примечание к \S~270 о~преобразовании
Кеплеровой формулы $\frac{S^3}{T^2}$ в Ньютонову $\frac{S^2\cdot S}{T^2}$,
причем $\frac{S}{T^2}$ было названо силой тяготения.} и ввел согласно своим
{\em началам} аналитическую трактовку.

\hegsection[С. Для-себя-бытие в мере]{С. Для-себя-бытие в мере}

1. В~только что рассмотренной специфицированной мере количественное обеих
сторон определено качественно (обе "--- в~степенн\'{о}м отношении); они, таким
образом, суть моменты одной, имеющей качественную природу определенности меры.
Но при этом качества только еще положены лишь как непосредственные, {\em лишь}
как {\em различные}, которые сами не находятся между собой в~том отношении,
в~котором находятся определенности их величин, т.~е. именно в~таком отношении,
{\em вне} которого не~имеет ни смысла, ни наличного бытия то, чт\'{о} содержит
степенн\'{а}я определенность величины. Таким образом, качественное прячется,
как будто оно специфицирует не само себя, а определенность величины; лишь как
находящееся {\em в}~последней, оно {\em положено}, само же по себе оно есть
{\em непосредственное} качество как таковое, которое вне того, что величина
полагается отличной от него, и вне своего соотношения со своим другим, еще
обладает само по себе пребывающим наличным бытием. Так, например, пространство
и время оба значимы вне той спецификации, которую имеет их количественная
определенность в движении падения тел или в абсолютно свободном движении,
значимы как пространство вообще, время вообще: пространство значимо, как
перманентно существующее само по себе вне и помимо времени, а время, как
текущее само по себе, независимо от пространства.

Но эта непосредственность качественного, противостоящая его специфическому
соотношению меры, связана также и с некоторою количественною
непосредственностью и безразличием некоторого {\em количественного} в~нем
к~этому своему отношению; непосредственное качество обладает также и лишь
{\em непосредственным определенным количеством}. Поэтому специфическая мера
и~имеет также сторону ближайшим образом внешнего изменения, поступательное
движение которого чисто арифметично, не нарушается этой мерой, и в~котором
имеет место внешняя и потому лишь эмпирическая определенность величины.
Качество и определенное количество, выступая, таким образом, также и вне
специфической меры, находятся вместе с тем в соотношении с~нею;
непосредственность есть момент тех определений, которые сами принадлежат
к~мере. Таким образом, непосредственные качества оказываются тоже
принадлежащими к мере, равным образом соотносительными и находящимися по
определенности величины в таком отношении, которое, как стоящее вне
специфицированного отношения, вне степенн\'{о}го определения, само есть лишь
прямое отношение и непосредственная мера. Мы должны разъяснить ближе этот
вывод и его связь.

2. Определенное непосредственно определенное количество как таковое, хотя
вообще оно как момент меры само по себе и обосновано в какой-нибудь связи
понятия, все же в соотношении со специфическою мерою представляет собою
некоторое данное извне. Но непосредственность, которая этим положена, есть
отрицание качественного определения меры; это отрицание мы обнаружили выше в
сторонах этого определения меры, которые поэтому выступили как самостоятельные
качества. Такое отрицание и возвращение к непосредственной количественной
определенности заключается в качественно определенном отношении постольку,
поскольку вообще отношение различенных заключает в себе их соотношение как
{\em единую} определенность, которая тем самым здесь в количественном, будучи
отличена от определения отношения, есть некоторое определенное количество. Как
отрицание различенных качественно определенных сторон этот показатель есть
некоторое для-себя-бытие, безоговорочная определенность; но он есть такое
для-себя-бытие лишь {\em в~себе;} как наличное бытие он есть простое,
непосредственное определенное количество "--- частное или показатель как
показатель отношения между сторонами меры, когда это отношение берется как
прямое; а говоря вообще, он есть эмпирически выступающая единица в
количественной стороне меры\pagenote{Гегель рассматривает здесь понятие
физической {\em константы}, т.~е. того эмпирического коэффициента, который в
той или иной форме входит в уравнения механики и физики. В~качестве примера
такой константы Гегель в следующей фразе приводит величину~$а$ в уравнении
движения падения тел $s\hm=at^2$. Гораздо чаще формулу движения падения тел
выражают уравнением $s\hm=\frac 1 2 gt^2$, где константа~$g$ (постоянное для
данного географического пункта ускорение силы тяжести) равна приблизительно
$9{,}8$~{\em м} (в~качестве единицы времени берется при этом секунда).
Следовательно, величина~$а$ в уравнении равна приблизительно $4{,}9$~{\em м}.
Впрочем, надо сказать, что величина~$а$ или~$g$, входящая в формулу движения
падения тел, может быть названа константою лишь в весьма относительном смысле.
Дело в~том, что сама она изменяется с изменением расстояния от центра земного
шара (а~также от расположения тяжелых масс на земной поверхности вблизи того
места, где производятся опыты с падением тел). Но так как эти изменения весьма
незначительны в тех случаях падения тел, которые рассматриваются в элементарной
механике (т.~е. в тех случаях, где расстояния, проходимые падающим телом,
незначительны по сравнению с длиной земного радиуса, причем опыты производятся
в одном и том же месте земной поверхности), то ими вполне можно пренебречь.}.
"--- При падении тел пройденные пространства относятся как квадраты протекших
времен: ; это "--- специфически-определенное, степенн\'{о}е отношение
пространства и времени; другое, прямое отношение, как утверждают, присуще
пространству и времени как безразличным друг к другу качествам; оно якобы есть
отношение пространства к {\em первому} моменту времени; один и тот же
коэфициент $a$ остается во все последующие моменты времени, "--- он есть
{\em единица} как некоторое обыкновенное определенное количество по отношению
к~численности, которая помимо этого определяется еще специфицирующей мерой. Эта
единица считается вместе с тем показателем того прямого отношения, которое
свойственно {\em представляемой} простой, т.~е. формальной скорости, не
определяемой специфически понятием. Такой скорости здесь не существует, как
не~существует той упомянутой ранее скорости, которой тело якобы обладает
в~{\em конце} некоторого момента времени. Указанная простая скорость
приписывается {\em первому} моменту времени падения, но сам этот так называемый
момент времени есть лишь предположенная единица и как таковая атомная точка не
обладает существованием; начало движения "--- утверждаемая якобы малость этого
начала не составляет никакой разницы "--- есть сразу же некоторая величина, и
притом величина, специфицированная законом падения тел. Указанное выше
эмпирическое определенное количество приписывается силе тяготения, так что сама
эта сила не имеет согласно этому представлению никакого отношения к имеющейся
налицо спецификации (к~степенн\'{о}й определенности), к своеобразию определения
меры. {\em Непосредственный} момент, состоящий в~том, что в~движении падения
тел на единицу времени (на секунду, и притом на так называемую {\em первую}
секунду) приходится численность в, приблизительно, пятнадцать пространственных
единиц, каковыми принимаются футы, есть {\em непосредственная мера}, такая же,
как мера величины членов человеческого тела, расстояния планет, их диаметры
и~т.~д. Определение такой меры происходит в какой-то иной области, чем внутри
области качественного определения меры, "--- в~нашем примере, в другой области,
чем в самом законе падения тел; но от чего зависят такие {\em числа},
представляющие собою тот момент меры, который выступает как лишь
непосредственный и потому как эмпирический, "--- на это конкретные науки еще не
дали нам ответа. Здесь мы имеем дело лишь с определенностью понятия; она
состоит в~том, что указанный эмпирический коэфициент составляет
{\em для-себя-бытие} в определении меры, но лишь такой момент для-себя-бытия,
где последнее есть {\em в~себе} и потому непосредственно. Другой момент есть
{\em развитое} для-себя-бытие, специфическая определенность сторон мерою. "---
Тяжесть в отношении, данном в падении тел, представляющем собою, правда,
движение еще наполовину обусловленное и лишь наполовину свободное, мы должны,
исходя из этого второго момента, рассматривать как некоторую силу природы, так
что природа времени и пространства определяет собою их отношение и потому
указанная спецификация (степенн\'{о}е отношение) присуща тяжести; вышеуказанное
же простое прямое отношение выражает собою лишь некоторое механическое
отношение между временем и пространством, формальную, внешним образом
произведенную и детерминированную скорость.

3. Мера определилась так, что теперь она есть специфицированное отношение
величин, которое как количественное заключает в себе обычное, внешнее
определенное количество; но последнее не есть определенное количество вообще,
а~представляет собою по существу момент определения отношения как такового;
оно есть, таким образом, показатель, и как представляющее собою теперь
непосредственную определенность "--- неизменяющийся показатель и, стало быть,
показатель того уже упомянутого прямого отношения этих самых качеств, которым
вместе с тем специфически определяется их количественное отношение друг к
другу. В~употребленном нами примере меры падения тел это прямое отношение как
бы предвосхищено и предположено имеющимся налицо; но, как мы уже сказали, оно
еще не существует в этом движении. "--- Но дальнейшее определение состоит
в~том, что мера теперь {\em реализована} таким образом, что обе ее стороны суть
меры, различенные как непосредственная, внешняя и как специфицированная внутри
себя, и она есть их единство. Как это единство мера содержит в себе такое
отношение, в котором величины определены природой качеств и положены
различными, и поэтому определенность его, будучи совершенно имманентной и
самостоятельной, вместе с тем сжалась в для-себя-бытие непосредственного
определенного количества, в показатель прямого отношения; самоопределение меры
{\em подвергается отрицанию} в этом показателе, так как она имеет в этом своем
другом последнюю, для-себя-сущую определенность; и наоборот, непосредственная
мера, которая должна быть качественною в самой себе, имеет в действительности
качественную определенность лишь в вышеупомянутом отношении. Это отрицательное
единство есть {\em реальное для-себя-бытие}, категория некоторого {\em нечто},
как единства качеств, находящихся в отношении меры, "--- полная
{\em самостоятельность}. Две стороны меры, оказавшиеся двумя разными
отношениями, непосредственно дают в результате также и двоякое наличное бытие;
или, говоря точнее, такое самостоятельное целое есть, как для-себя-сущее
вообще, вместе с чем расталкивание, распадение на {\em различенные},
{\em самостоятельные} нечто, качественная природа и устойчивость
(материальность) которых заключается в их определенности меры.
