Мы уже указали отличие количества от качества. Качество есть первая,
непосредственная определенность, количество же "--- определенность, ставшая
безразличной для бытия, граница, которая вместе с тем и не есть граница,
для-себя-бытие, которое безоговорочно тождественно с бытием-для-другого,
"--- отталкивание многих одних, которое есть непосредственно
неотталкивание, непрерывность их.

Так как для-себя-сущее теперь положено таким образом, чтобы не исключать
другое, а наоборот, утвердительно продолжать себя в последнее, то, поскольку
{\em наличное бытие} снова выступает в этой непрерывности и определенность
этого наличного бытия {\em вместе с тем} уже не стоит более в простом
соотношении с собою, инобытие уже не есть более непосредственная определенность
налично сущего нечто, но положено так, что имеет себя как отталкивающееся от
себя, имеет соотношение с собою как определенность скорее в некотором другом
наличном бытии (в~некотором сущем-для-себя); а так как они {\em вместе с тем}
даны (sind) как безразличные рефлектированные в себя, несоотносительные
границы, то определенность есть вообще {\em вне себя}, некое безоговорочно
{\em внешнее} себе, и нечто есть также внешнее; такая граница, безразличие ее
в самой себе и безразличие данного нечто к ней, составляет {\em количественную}
определенность этого нечто.

Прежде всего надлежит отличать {\em чистое количество} от него же как
{\em определенного} количества, от Quantum. Как чистое количество оно есть,
{\em во-первых}, возвратившееся в себя реальное для-себя-бытие, не имеющее еще
в себе никакой определенности; оно есть сплошное (gediegene), продолжающее себя
внутри себя бесконечное единство.

Последнее, {\em во-вторых}, переходит в определенность, полагаемую в~нем как
определенность, которая вместе с тем не такова, есть лишь внешняя
определенность. Количество становится {\em определенным количеством}.
Определенное количество есть безразличная, т.~е. выходящая за себя, отрицающая
самое себя определенность. Как такое инобытие инобытия оно впадает
в~{\em бесконечный} прогресс. Но бесконечное определенное количество есть
снятая безразличная определенность, оно есть восстановление качества.

{\em В-третьих,} определенное количество в качественной форме есть
количественное {\em отношение}. Определенное количество выходит за себя лишь
вообще; в отношении же оно выходит за себя, переходит в свое инобытие так, что
последнее, в котором оно имеет свое определение, вместе с тем положено, есть
некоторое другое определенное количество; тем самым его возвращенность в~себя
и соотношение с собою дано (ist) как имеющееся в его инобытии.

В основании этого отношения еще лежит внешний характер определенного
количества; здесь относятся друг к другу именно {\em безразличные}
определённые количества, т.~е. они имеют своё соотношение
с самими собою в таком вовне-себя-бытии.
Отношение есть тем самым лишь формальное единство качества
и количества. Диалектика отношения состоит в его переходе в их абсолютное
единство, в~{\em меру}.

\subsubsection[Примечание]{\centering {\lsstyle\mdseries Примечание}}

В нечто его граница как качество есть по существу его определенность. Но если
мы под границей понимаем количественную границу, и, например, поле изменяет эту
свою границу, то оно остается полем как до, так и после этого. Напротив, если
изменяется его качественная граница, то это изменяется та его определенность,
через которую оно есть поле, и оно становится лугом, лесом и~т.~д. "---
Краснота, будь она более интенсивной или более слабой, есть всегда краснота;
но, если она изменяет свое качество, она перестает быть краснотой, она
становится синевой и~т.~д. "--- Определение {\em величины} как определенного
количества, как оно получилось выше, состоящее в том, что в основании лежит
некоторое бытие как сохраняющееся, {\em безразличное к определенности, которой
оно обладает,} подтверждается любым другим примером.

Под словом <<величина>> подразумевается, как в данных нами примерах,
{\em определенное количество} (квант), а не количество [вообще], и главным
образом вследствие этого нам приходится употреблять это заимствованное из
чужого языка название\pagenote{Немецкое слово Quantitat (количество)
образовано из латинского quantitas.}.

Дефиниция {\em величины,} даваемая в математике, также имеет в~виду
определенное количество. Обыкновенно определяют величину как нечто, могущее
{\em увеличиваться} или {\em уменьшаться}. Но увеличивать "--- значит сделать
нечто более {\em великим,} а уменьшать "--- сделать нечто менее {\em великим}.
В~этом состоит {\em отличие} величины вообще от нее же самой, и величиной было
бы, таким образом, то, величина чего может изменяться. Дефиниция оказывается
неподходящей, поскольку в~ней применяется то самое определение, которое должно
быть дефинировано. Поскольку в~ней нельзя употреблять это же самое определение,
постольку <<более>> или <<менее>> должны быть разложены на некоторое
прибавление как утверждение (и~притом, согласно природе определенного
количества, столь же внешнее утверждение) и на некоторое убавление как
некоторое столь же внешнее отрицание. В~такой {\em внешний} характер как
реальности, так и отрицания определяет себя вообще природа {\em изменения}
в определенном количестве. Поэтому и в вышеуказанном несовершенном выражении
нельзя не усмотреть того главного момента, в~котором все дело, а~именно,
безразличия изменения, так что в самом его понятии содержится его собственное
<<меньше>> и <<больше>>, его безразличие к самому себе.

\hegchapter[Первая глава]{Количество}

\hegsection[А. Чистое количество]{А. Чистое количество}

Количество есть снятое для-себя-бытие; отталкивающее одно, относившееся
к~исключенному одному лишь отрицательно, теперь, перешедши в~{\em соотношение}
с последним, относится тождественно к другому и, стало быть, потеряло свое
определение; для-себя-бытие перешло в притяжение. Абсолютная неподатливость
отталкивающего {\em одного} растаяла, перешла в это {\em единство}, которое,
однако, как содержащее в себе это одно, определено вместе с тем через
внутреннее отталкивание, {\em есть единство с самим собой}, как
{\em единство вне-себя-бытия}. Притяжение есть, таким образом, момент
{\em непрерывности} в~количестве.

{\em Непрерывность} есть, следовательно, простое, саморавное соотношение
с~собой, непрерываемое никакой границей и никаким исключением, но она есть
{\em не непосредственное} единство, а единство для-себя-сущих одних. В~ней
еще содержится {\em внеположность множественности}, но содержится вместе
с~тем, как нечто неразличенное, {\em непрерываемое}. Множественность положена
в~непрерывности так, как она есть в себе; многие суть одно, что и другое,
каждое равно другому, и множественность есть поэтому простое, лишенное
различий равенство. Непрерывность представляет собой этот момент
{\em равенства} внеположности {\em самой себе}, самопродолжение различенных
одних в~их отличные от~них.

Непосредственно поэтому величина в непрерывности имеет момент
{\em дискретности} "--- отталкивание в том виде, в каком оно теперь является
моментом в количестве. "--- Непрерывность есть саморавенство, но саморавенство
многого, которое, однако, не становится исключающим; только отталкивание
впервые расширяет саморавенство до непрерывности. Дискретность поэтому есть
со своей стороны сливающаяся дискретность, в которой ее одни имеют своей связью
не пустоту, не отрицательное, а свою собственную непрерывность и не прерывают
во многом этого равенства с самими собою.

Количество есть единство этих моментов, непрерывности и дискретности, но оно
сначала есть это единство {\em в форме} одного из них, {\em непрерывности}, как
результат диалектики для-себя-бытия, которое сжалось в форму самой себе равной
непосредственности. Количество как таковое есть этот простой результат,
поскольку он еще не развил и не положил в нем [в~самом себе] своих моментов.
"--- Оно {\em содержит} их ближайшим образом, будучи положено как
для-себя-бытие, как это последнее есть поистине. Это для-себя-бытие было по
своему определению снимающим себя соотнесением с самим собою, вековечным
выхождением вне себя. Но оттолкнутое есть оно же само; отталкивание есть
поэтому то, что порождает продолжающееся течение самого себя. Благодаря
тождественности отталкиваемого это порождение дискретного (dies
Dis\-zer\-nie\-ren) есть непрерываемая непрерывность, а благодаря выхождению
вне~себя эта непрерывность, не будучи прерываемой, есть вместе с~тем
множественность, которая столь же непосредственно остается в своем равенстве
с~самой собою.

\hegremark[Примечание 1]%
{Представление о чистом количестве}%
{[Представление о чистом количестве]}

Чистое количество еще не имеет границы или, иначе говоря, оно еще не есть
определенное количество, а поскольку оно становится определенным
количеством, граница также не служит его пределом; оно, наоборот, именно и
состоит в том, что граница не служит для него пределом, что оно имеет
для-себя-бытие внутри себя как некоторое снятое. То обстоятельство, что
дискретность есть в нем момент, может быть выражено так, что количество
повсюду и безоговорочно есть {\em реальная возможность} одного, но
что также и обратно, одно столь же безоговорочно дано (ist) как непрерывное.

Для чуждого понятию {\em представления} непрерывность легко превращается
в~{\em складывание}, а~именно во~{\em внешнее} соотношение одних друг с другом,
в~котором одно сохраняет свою абсолютную неподатливость и исключение других
одних. Но рассмотрение одного показало, что оно само по себе переходит
в~притяжение, в~свою идеальность и что поэтому непрерывность не внешня для
него, а~принадлежит ему самому и имеет свое основание в~его сущности. За эту-то
{\em внешность} непрерывности для одних и цепляется атомистика, и отказаться от
нее представлению очень трудно. "--- Напротив, математика отвергает ту
метафизику, которая полагала, что время {\em состоит} из временных точек,
пространство вообще или ближайшим образом линия "--- из пространственных точек,
поверхность "--- из линий, все пространство "--- из поверхностей; она не
допускает таких дискретных одних. Если она и определяет, например, величину
поверхности, как {\em сумму} бесконечно многих линий, то она видит в этой
дискретности лишь представление, принимаемое на один момент, и в представлении
о~{\em бесконечном} множестве линий уже заключается снятость их дискретности,
так как пространство, которое они должны составлять, является ведь
ограниченным.

{\em Спиноза}, которому было преимущественно важно выяснение понятия чистого
количества, имеет в виду противоположность этого понятия голому представлению,
когда он высказывается о количестве следующим образом:

\vspace{3mm}

\begin{footnotesize}\fontsize{10}{13}\selectfont
Quantitas duobus modis a~nobis conci\-pitur, abs\-trac\-te scili\-cet
sive super\-fi\-ci\-ali\-ter, prout nempe ipsam ima\-gina\-mur; vel ut
sub\-stan\-tia, quod а solo intel\-lectu fit. Si itaque ad quanti\-tatem
atten\-di\-mus, prout in imagin\-ati\-one est, quod saepe et faci\-lius
a~nobis fit, repe\-rietur finita, {\em divi\-sibi\-lis} et
{\em ex parti\-bus con\-flata,} si autem ad ipsam, prout in intel\-lectu est,
atten\-di\-mus et eam, quate\-nus sub\-stan\-tia est, con\-cipi\-mus,~--- quod
dif\-ficil\-lime fit,~--- {\em infi\-nita, unica et indi\-vi\-si\-bi\-lis}
re\-pe\-rietur. Quod omni\-bus, qui inter ima\-gi\-na\-tio\-nem et
intel\-lectum dis\-tin\-guere scive\-rint, satis mani\-fes\-tum erit.\par
\end{footnotesize}

\vspace{3mm}

\noindent [<<Величина представляется нам двумя способами: абстрактно или
поверхностно, а~именно, как мы ее воображаем, или же как субстанция, что
возможно только посредством разума. Таким образом, если мы рассматриваем
величину, как она существует в~воображении, что бывает чаще и гораздо
легче, то мы находим ее конечной, {\em делимой и состоящей из частей,}
если~же мы рассматриваем ее, как она существует в~разуме, и представляем
ее как субстанцию, что весьма трудно, то мы находим ее {\em бесконечной,
единой и неделимой}. Это будет достаточно ясно всем, кто научился делать
различие между воображением и разумом>>. "--- <<Этика>>, ч.~I,
теорема 15-я, схолия. Курсив Гегеля].

Если потребуют, чтобы мы дали более определенные примеры чистого количества,
то укажем, что таковыми служат пространство и время, а также материя
вообще, свет и~т.~д. и даже <<я>>; только под количеством, как мы уже
заметили выше, не следует понимать определенного количества. Пространство,
время и~т.~д. суть протяжения, множества, которые суть выхождение вне себя,
истечение, не переходящее, однако, в противоположность, в~качество или
в~одно, а представляющее собою, как выхождение вне себя, вековечное
{\em самопродуцирование} своего единства.

Пространство есть то абсолютное {\em вне-себя-бытие}, которое столь же
безоговорочно непрерывно, оно есть инобытие и все снова и снова инобытие,
тождественное с собою; время есть некое абсолютное {\em выхождение вне себя},
некое порождение одного, момента времени, <<{\em теперь}>>, каковое порождение
непосредственно есть уничтожение этого <<теперь>> и в свою очередь непрерывное
уничтожение этого прехождения, так что это самопорождение небытия есть вместе
с~тем простое равенство и тождество с собою.

Что касается {\em материи} как количества, то в числе {\em семи теорем},
сохранившихся от первой диссертации\pagenote{Имеется в~виду юношеская
философская работа Лейбница <<О~принципе индивидуализации>>, написанная им
в 1663~г.} {\em Лейбница} (1-я страница первого тома его сочинений), есть одна
(а~именно вторая), гласящая следующим образом: Non omnino impro\-babi\-le est,
mate\-riam et quanti\-tatem esse reali\-ter idem (Не совсем невероятно, что
материя и количество суть в~действительности одно и то~же). "--- И~в самом
деле, эти понятия отличаются друг от друга лишь тем, что количество есть
чистое определение мысли, а~материя есть это же определение мысли во внешнем
существовании. "--- <<Я>>~(dem Ich) также присуще определение чистого
количества, поскольку <<я>> есть абсолютное становление другим, некоторое
бесконечное удаление или всестороннее отталкивание к~отрицательной свободе
для-себя-бытия, однако такое отталкивание, которое остается безоговорочно
простой непрерывностью, "--- непрерывностью всеобщности или у-себя-бытия,
не~прерываемой бесконечно многообразными границами, содержанием ощущений,
созерцаний и~т.~д. "--- Что касается тех, которые восстают против того, чтобы
понимать {\em множество как простое единство}, и кроме того {\em понятия}, что
каждое из многих есть то же самое, чт\'{о}~и другое, а~именно, одно из многих
(поскольку здесь не идет речь о далее определенном многом, о зеленом, красном
и~т.~д., а~о~многом, рассматриваемом само по себе), требуют еще, чтобы им дали
{\em представление} об этом единстве, то они найдут такого рода представления,
сколько пожелают, в~тех непрерывностях, которые дают в~простом созерцании
дедуцированное понятие количества, как имеющееся налицо.

\hegremark[Примечание 2]%
{Кантовская антиномия неделимости и бесконечной делимости времени,
пространства, материи}%
{[Кантовская антиномия неделимости\\и бесконечной делимости времени,\\
пространства, материи]}

\label{bkm:bm88a}С~природой количества, заключающейся в~том, что оно есть
указанное простое единство дискретности и непрерывности, находится в~связи спор
или {\em антиномия} касательно {\em бесконечной делимости} пространства,
времени, материи и~т.~д.

Эта антиномия состоит исключительно только в~том, что показывает необходимость
утверждать как дискретность, так и непрерывность. Одностороннее утверждение
дискретности приводит к~признанию бесконечной или абсолютной
{\em разделенности} и, следовательно, к~признанию некоторого неделимого как
первоначала; одностороннее утверждение непрерывности приводит, напротив,
к~признанию бесконечной {\em делимости}.

Кантовская критика чистого разума выставляет, как известно, {\em четыре}
(космологических) {\em антиномии}, из которых {\em вторая} касается той
{\em противоположности}, которую составляют {\em моменты количества}.

Эти кантовские антиномии навсегда останутся важной частью критической
философии; они преимущественно и привели к ниспровержению предшествующей
метафизики и могут быть рассматриваемы как главный переход к новейшей
философии, так как они в особенности способствовали возникновению убеждения
в ничтожности категорий конечности со стороны
{\em содержания}, а это представляет собою более
правильный путь, чем формальный путь субъективного идеализма, согласно
которому их недостаток заключается лишь в том, что они субъективны, а не в
том, что они суть в самих себе. Но при всей своей великой заслуге
кантовское изложение антиномий все-таки весьма несовершенно; отчасти оно в
самом себе страдает связанностью и сбивчивостью, отчасти же оно неправильно
в отношении вывода, который предполагает, что познание не имеет никаких
других форм мышления, кроме конечных категорий. "--- В обоих отношениях эти
антиномии заслуживают более пристальной критики, которая ближе осветит их
точку зрения и метод, равно как и освободит основной пункт, в котором вся
суть, от той ненужной формы, в которую он втиснут.

Прежде всего замечу, что Кант примененным им принципом деления, который он
заимствовал из своей схемы категорий, хотел придать своим четырем
космологическим антиномиям видимость полноты. Однако более глубокое
вникновение в антиномическую или, вернее, в диалектическую природу разума
показывает нам, что вообще {\em всякое} понятие есть
единство противоположных моментов, которым можно было бы, следовательно,
придать форму антиномических утверждений. Становление, наличное бытие
и~т.~д. и всякое другое понятие могли бы, таким образом, доставить нам свои
особые антиномии, и, стало быть, можно выставить столько антиномий, сколько
получается понятий. "--- Античный скептицизм не пожалел труда и обнаружил это
противоречие или эту антиномию во всех понятиях, которые он нашел в науках.

Далее мы должны сказать, что Кант берет антиномию не в самих понятиях, а в
уже {\em конкретной форме} космологических определений.
Чтобы получить антиномию в чистом виде и трактовать ее в ее простом
понятии, следовало бы рассматривать определения мысли не в их применении к
представлению о мире, пространстве, времени, материи и~т.~д. и в смешении с
такими представлениями, а без этого конкретного материала, не имеющего в
этом отношении силы и значения, следовало бы рассматривать их в чистом
виде, сами по себе, так как единственно лишь эти определения мысли
составляют сущность и основание антиномий.

Кант дает следующее понимание антиномий: они <<суть не софистические
ухищрения, а противоречия, на которые разум необходимо должен (по~кантовскому
выражению) {\em наталкиваться}>>\pagenote{Свободное изложение мысли Канта из
<<Критики чистого разума>> (стр.~400).}; это важный взгляд. <<После того как
разум усмотрел основание естественной видимости антиномий, он, хотя уже
не~вводится ею в~обман, все~же сбивается с~толку>>\pagenote{{\em Кант}, Критика
чистого разума, 2-е немецкое издание, стр.~449---450. Гегель несколько
перефразирует это место. Во 2-м издании перевода Лосского (Пгр. 1915) это место
находится на стр.~263.}. "--- Критическое разрешение антиномий при помощи так
называемой трансцендентальной идеальности мира восприятий приводит только
к~тому результату, что превращает так называемое противоречие (Wider\-streit)
в~нечто {\em субъективное}, в котором оно, конечно, все еще остается той же
видимостью, т.~е. остается столь же неразрешенным, как и раньше. Их истинное
разрешение может состоять только в~том, что два определения, будучи
противоположными друг другу и необходимо присущими одному и тому же понятию, не
могут быть значимы в их односторонности, каждое само по себе, а имеют свою
истину лишь в их снятости, в единстве их понятия.

При ближайшем рассмотрении оказывается, что кантовские антиномии не содержат
в себе ничего другого, кроме совершенно простого категорического
утверждения {\em каждого} из двух противоположных
моментов некоторого определения, взятого самого по себе, в его
{\em изолированности} от другого. Но при этом указанное
простое категорическое или, собственно говоря, ассерторическое утверждение
запрятано в сложной сети превратных, запутанных рассуждений, благодаря чему
должна получиться видимость доказательства и должен прикрываться, сделаться
незаметным чисто ассерторический характер утверждения; это обнаружится при
ближайшем рассмотрении этих рассуждений.

Имеющая сюда отношение антиномия касается так называемой
{\em бесконечной делимости материи} и основана на
противоположности моментов непрерывности и дискретности, содержащихся в
понятии количества.

{\em Тезис} этой антиномии в изложении Канта гласит:

{\em <<Всякая сложная субстанция в~мире состоит из простых частей, и~нигде
не~существует ничего другого, кроме простого или составленного
из~него>>}\pagenote{<<Критика чистого разума>>, стр.~410.}.

Здесь простому, атому, противопоставляется {\em сложное}, что по сравнению
с~непрерывным или сплошным представляет собой очень отсталое определение.
Субстрат, данный [Кантом] этим абстракциям, а~именно субстанции в~мире,
не~означает здесь ничего другого, кроме вещей, как они доступны чувственному
восприятию, и не~оказывает никакого влияния на характер самой антиномии; можно
было бы с~тем же успехом взять пространство или время. "--- Так как тезис
говорит лишь о~{\em сложении}, вместо того чтобы говорить
о~{\em непрерывности}, то он, собственно говоря, есть тем самым аналитическое
или {\em тавтологическое} предложение. Что сложное или составное есть само
по~себе не~{\em одно}, а~лишь сочетанное внешним образом и что оно
{\em состоит из другого}, это является его непосредственным определением.
Но иное сложного есть простое. Поэтому является тавтологией сказать, что
сложное или составное состоит из простого. "--- Если уже задают вопрос,
{\em из чего состоит} нечто, то требуют, чтобы указали
{\em некое иное, сочетание} которого составляет это нечто. Если говорят, что
чернила опять-таки состоят из чернил, то это означает, что не понят смысл
вопроса о~составленности из иного; этот вопрос остался без ответа, его лишь
еще раз повторяют. Затем возникает дальнейший вопрос: {\em состоит ли} вообще
то, о~чем идет речь, {\em из чего-то} или нет? Но сложное есть несомненно нечто
такое, чт\'{о} должно быть сочетанным и состоять из иного. "--- Если простое,
которое есть иное сложного, принимают лишь за {\em относительно простое},
которое само по~себе в~свою очередь сложено, то вопрос остается и после
ответа, как до него. Представлению предносится лишь то или другое сложное,
относительно которого можно указать, что то или другое нечто есть
{\em его} простое, которое само по~себе есть опять-таки сложное. Но здесь
речь идет о~{\em сложном как таковом}.

Что касается теперь кантовского {\em доказательства}
тезиса, то оно, как и все кантовские доказательства прочих антиномических
положений, берет {\em окольный путь} доказательства
{\em от противного}, который, как увидим, совершенно излишен.

<<Допустим (начинает он), что сложные субстанции не~состоят из простых частей;
в~таком случае, если бы было {\em устранено} мысленно {\em всякое} сложение, то
не было бы никакой сложной части, а~так как (согласно только что сделанному
нами допущению) никаких простых частей нет, то не осталось бы также и никакой
простой части, иными словами, не осталось бы абсолютно ничего и, значит,
не~было бы дано никакой субстанции>>\pagenote{<<Критика чистого разума>>,
стр.~410. Слова в~скобках и~курсив Гегеля.}.

Этот вывод совершенно правилен. Если нет ничего, кроме сложного, и мы
отмыслим все сложное, то ничего не остается, "--- с этим надо согласиться, но
можно было бы прекрасно обойтись без всего этого тавтологического излишества
и сразу начать доказательство с того, чт\'{о} следует за этим, а~именно:

<<Либо невозможно устранить мысленно всякую сложность, либо после ее устранения
должно оставаться нечто, существующее без сложности, т.~е. нечто простое>>.

<<Но в первом случае сложное не состояло бы в свою очередь из субстанций
({\em так как для последних сложение есть случайное
отношение субстанций}\footnote{К~излишеству в~самом способе
доказательства добавляется здесь еще излишество слов: <<Так как
{\em для последних} (именно для субстанций) сложность есть лишь случайное
отношение {\em субстанций}>>.}{\em , без которого они должны
существовать как самостоятельно пребывающие сущности})>>.

Так как этот случай <<противоречит предположению, то остается
только второй случай, а~именно что субстанциально
сложное в~мире состоит из простых частей>>.

В скобки как бы мимоходом заключен тот довод, который здесь представляет
собою главное и в~сравнении с~которым все предшествующее совершенно излишне.
Дилемма состоит в~следующем: либо сложное есть сохраняющееся, либо не оно,
а~простое. Если бы сохраняющимся было первое, а именно сложное,
то сохраняющееся не было бы субстанциями, ибо {\em для субстанций
сложение есть} лишь {\em случайное отношение}. Но субстанции "--- это то,
чт\'{о} сохраняется; стало быть, то, чт\'{о} сохраняется, есть простое.

Ясно, что можно было бы без окольного пути доказательства от противного дать
в~качестве доказательства указанный выше довод, присоединив его
непосредственно к~тезису, гласящему: <<Сложная субстанция состоит из простых
частей>>, {\em ибо} сложение есть лишь {\em случайное} отношение субстанций,
которое для них, следовательно, внешне и не касается самих субстанций. "---
Если правильно, что сложение есть нечто случайное, то сущностью, конечно,
оказывается простое. Но эта случайность, в~которой вся суть, не доказывается
[Кантом], а~прямо принимается [им] "--- и притом мимоходом, в~скобках "--- как
нечто само собою разумеющееся или побочное. Конечно, само собою понятно, что
сложение есть определение случайного и внешнего. Но если вместо непрерывности
имеется в~виду лишь случайная совместность, то не стоило устанавливать
по этому поводу антиномию или, правильнее сказать, вообще нельзя было
установить антиномию. Утверждение о~простоте частей в~таком случае,
как сказано, лишь тавтологично.

Мы видим, стало быть, что на окольном пути доказательства от
противного в~доказательстве имеется то самое утверждение, которое
должно получиться как вывод из доказательства. Можно поэтому
выразить доказательство короче следующим образом:

Допустим, что субстанции не состоят из простых частей, а лишь сложены.
Но ведь можно устранить мысленно всякое сложение (ибо оно есть лишь случайное
отношение); следовательно, после его устранения не осталось бы никаких
субстанций, если бы они не состояли из простых частей. Но субстанции должны
у~нас быть, так как мы предположили, что они существуют; у~нас не~все должно
исчезнуть, а~кое-что должно остаться, ибо мы предположили существование такого
сохраняющегося, которое мы назвали субстанцией; это нечто, следовательно,
необходимо должно быть простым.

Чтобы покончить полностью с~этим доказательством, мы должны рассмотреть
еще~и заключение. Оно гласит:

<<Из этого непосредственно {\em следует}, что все решительно вещи мира суть
простые сущности, "--- {\em что сложение есть только внешнее состояние их}
и~что разум должен мыслить элементарные субстанции как простые сущности>>.

Здесь мы видим, что внешний характер, т.~е. случайность сложения приводится
как {\em следствие}, после того как ранее она была введена в~доказательство
в~скобках и применялась там [в~качестве довода].

Кант решительно протестует против утверждения, будто в противоречивых
положениях антиномий он стремится к фокусам, чтобы, так сказать (как
обыкновенно выражаются), дать адвокатскую аргументацию. Рассматриваемую
аргументацию приходится обвинять не столько в~фокусничестве, сколько
в~бесполезной вымученной запутанности, служащей лишь тому, чтобы достигнуть
внешнего вида доказательства и~помешать читателю заметить во~всей его
прозрачности то обстоятельство, что~то, чт\'{о} должно появиться как следствие,
составляет в~скобках самое суть доказательства, "--- что вообще здесь
нет доказательства, а~есть лишь предположение.

{\em Антитезис} гласит:

{\em Никакая вещь в мире не состоит из простых частей,
и в~нем вообще не существует ничего простого}.

{\em Доказательство} антитезиса тоже ведется от противного и по-своему
столь же неудовлетворительно, как и предыдущее.

<<Допустим, "--- читаем мы, "--- что сложная вещь как субстанция состоит из
простых частей. Так как всякое {\em внешнее отношение} и, значит, также и
всякое сложение из субстанций возможна лишь в {\em пространстве}, то
и пространство, занимаемое сложной вещью, должно состоять из стольких же
частей, из скольких состоит эта вещь. Но пространство состоит не из простых
частей, а из пространств. Следовательно, каждая часть сложного должна занимать
некоторое пространство>>.

<<Но безусловно первоначальные части всякого сложного просты>>.

<<Следовательно, простое занимает некоторое пространство>>.

<<А так как всякое реальное, занимающее некоторое пространство, заключает
в~себе многообразие, [составные части которого] находятся вне друг друга,
стало быть, есть нечто сложное, и притом состоит из субстанций, то простое
было бы субстанциально сложным, что противоречиво>>.

Это доказательство можно назвать целым {\em гнездом} (употребляя встречающееся
в другом месте выражение Канта) ошибочных способов рассуждения.

Прежде всего оборот доказательства от противного есть ни на чем не основанная
видимость. Ибо допущение, что {\em все субстанциальное пространственно,
пространство же не состоит из простых частей}, есть прямое утверждение,
которое Кант делает непосредственным основанием того, чт\'{о} требуется
доказать, и при наличии которого все доказательство уже готово.

Затем это доказательство от противного начинается с предложения, что <<всякая
сложенность из субстанций есть {\em внешнее} отношение>>, но довольно странным
образом Кант сейчас же снова его забывает. А~именно, далее Кант ведет свое
рассуждение так, что сложное возможно лишь {\em в~пространстве}, а пространство
не состоит из простых частей; следовательно, реальное, занимающее некоторое
пространство, сложно. Если только допущено, что сложность есть внешнее
отношение, то сама пространственность (так же, как и все прочее, чт\'{о} может
быть выведено из определения пространственности), единственно лишь в которой
якобы возможно сложение, есть именно поэтому для субстанций внешнее отношение,
которое их совершенно не касается и не затрагивает их природы. Именно на этом
основании, не~следовало бы принимать, что, субстанции помещены в пространстве.

Здесь, далее, предполагается, что пространство, в которое здесь помещены
субстанции, не состоит из простых частей; ибо оно есть некоторое созерцание,
а~именно, согласно кантовскому определению, представление, которое может быть
дано только лишь одним единственным предметом, а~не так называемое дискурсивное
понятие. "--- Как известно, из этого кантовского различения созерцания
и~понятия возникло весьма неподобающее обращение с~созерцанием, и, чтобы
избавить себя от труда, связанного с~достижением {\em в понятиях} (Begreifen),
стали расширять ценность и область созерцания так, чтобы оно совпадало со
всяким познанием. Здесь требуется только принять, что пространство, как и само
созерцание, должно быть вместе с~тем {\em постигнуто в понятиях}, если, именно,
хотят вообще постигать в понятиях. Таким образом, возник бы вопрос, не должны
ли мы мыслить пространство согласно его понятию как состоящее из простых
частей, хотя как созерцание оно представляет собою простую непрерывность, или,
иначе говоря, пространство оказалось бы пораженным той же антиномией, которая
приписывалась только субстанции. И~в самом деле, если антиномия мыслится
абстрактно, то она, как было указано, касается количества вообще и,
следовательно, также и пространства и времени.

Но так как в доказательстве принимается, что пространство не состоит из
простых частей, то это должно было бы служить основанием для того, чтобы не
помещать простого в этот элемент, не соответствующий определению простого.
"--- Но при этом получается также коллизия непрерывности пространства со
сложностью. Кант смешивает их друг с другом, подставляет первую вместо
второй (это приводит в умозаключении к {\em Quaternio
terminorum}). У~Канта ясно высказанным определением пространства служит то,
что оно есть <<{\em единое} и части его основаны лишь на
ограничениях, так что они не предшествуют единому всеобъемлющему
пространству, не суть как бы его {\em составные части},
из которых его можно было бы {\em сложить}>> (Kr.~d.~r.~Vern
изд.~2-е, стр.~39). Здесь непрерывность очень правильно и определенно
приписана пространству в {\em противоположность}
сложенности из составных частей. Напротив, в аргументации выходит, что
помещение субстанций в пространство влечет за собою некоторое
<<{\em находящееся} друг вне друга многообразие>> и
притом, <<следовательно, некоторое сложное>>. А~между тем, как было указано,
способ, каким многообразие оказывается находящимся в пространстве,
исключает, по определенному высказыванию Канта, сложность этого
многообразия и предшествующие единству пространства составные части.

В примечании к доказательству антитезиса нарочито приводится еще кроме того
другое основное представление критической философии, что мы имеем {\em понятие}
о телах лишь как о {\em явлениях}, но что как таковые они необходимо
предполагают пространство как условие возможности всякого внешнего явления.
Следовательно, если под субстанциями разумеются лишь тела, как мы их видим,
осязаем, вкушаем и~т.~д., то, собственно говоря, о том, что они суть в их
понятии, здесь и не поднимается речь; дело идет только о чувственно
воспринимаемом. Таким образом, нужно было бы формулировать доказательство
антитезиса коротко, а именно следующим образом: весь опыт нашего видения,
осязания и~т.~д. показывает нам лишь составное; даже самые лучшие микроскопы и
тончайшие измерители не {\em натолкнули} нас на что-либо простое. Стало быть,
разум и не должен желать натолкнуться на нечто простое.

Следовательно, если мы пристальнее присмотримся к противоположности этих тезиса
и антитезиса и освободим их доказательства от всякого бесполезного излишества и
запутанности, то доказательство антитезиса содержит в себе "--- тем, что оно
помещает субстанции в пространство "--- ассерторическое допущение
{\em непрерывности}, равно как и доказательство тезиса "--- тем, что оно
допускает составность как способ соотношений субстанций "--- содержит в~себе
ассерторическое допущение {\em случайности этого соотношения} и тем самым
допущение, что субстанции суть {\em абсолютные одни}. Вся антиномия сводится,
следовательно, к разъединению и прямому утверждению двух моментов количества и
притом утверждению их как безоговорочно раздельных. Взятые со стороны одной
только {\em дискретности}, субстанция, материя, пространство, время и~т.~д.
безоговорочно разделены; их принципом служит одно. Взятое-же со стороны
{\em непрерывности}, это одно есть лишь некое снятое; деление остается
делимостью, остается {\em возможность} делить как возможность, никогда не
доводящая в действительности до атома. Если же мы остановимся на том
определении, которое дано в том, что было сказано выше об этих
противоположностях, то мы убедимся, что в самой непрерывности заключается
момент атома, так как она безоговорочно есть возможность деления, а равно, что
та деленность, дискретность упраздняет также всякое различие одних, "--- ибо
каждое из простых одних есть то же самое, что и другое, "--- следовательно,
содержит в себе также их одинаковость и, стало быть, их непрерывность. Так как
каждая из двух противоположных сторон содержит в самой себе свою другую и ни
одна из них не может быть мыслима без другой, то из этого следует, что ни одно
из этих определений, взятое отдельно, не истинно, а истинно лишь их единство.
Это есть истинно диалектический способ рассмотрения этих определений, равно как
и истинный результат.

Бесконечно более остроумными и глубокими, чем рассмотренная кантовская
антиномия, являются диалектические примеры древней {\em элейской школы},
в~особенности примеры, касающиеся движения, которые равным образом основаны
на понятии количества и в нем находят свое разрешение. Рассмотрение здесь
еще и их сделало бы наше изложение слишком пространным; они касаются понятий
пространства и времени и могут быть обсуждены при рассмотрении последних и
в~истории философии. "--- Они делают величайшую честь разуму их изобретателей;
они имеют своим {\em результатом} чистое бытие Парменида, так как они
показывают разложение всякого определенного бытия в нем самом и суть,
следовательно, сами в себе {\em течение} Гераклита. Они поэтому и достойны
более основательного рассмотрения, чем обычное заявление, что это только
софизмы; каковое утверждение держится за эмпирическое восприятие, по примеру
столь ясного для здравого человеческого рассудка прецедента Диогена, который,
когда какой-то диалектик вскрывал перед ним противоречие, содержащееся
в~движении, не счел нужным напрягать далее свой разум, а немым хождением взад
и вперед указал на чувственную очевидность; такое утверждение и опровержение,
разумеется, легче выдвинуть, чем углубиться в мысль, внимательно вдуматься
в те затруднения, к которым приводит мысль, и притом мысль, не притянутая
откуда-нибудь издалека, а~формирующаяся в~сам\'{о}м обыденном сознании,
и затем разрешить эти затруднения с~помощью самой же мысли.

То разрешение этих диалектических построений, которое дает {\em Аристотель},
заслуживает великой похвалы и содержится в его истинно спекулятивных понятиях
о~пространстве, времени и движении. Он противополагает бесконечной делимости
(которая, "--- так как ее представляют себе, как будто она осуществляется, "---
тождественна с бесконечной разделенностью, с атомами), на которой основаны
самые знаменитые из этих доказательств, непрерывность, свойственную одинаково
как времени, так и пространству, так что бесконечная, т.~е. абстрактная
множественность оказывается содержащейся в непрерывности лишь {\em в~себе},
{\em в возможности}. Действительным по отношению к абстрактной множественности,
равно как и по отношению к абстрактной непрерывности, служит их конкретное,
сами время и пространство, как в свою очередь по отношению к последним "---
движение и материя. Абстрактное существует (ist) лишь в себе или лишь в
возможности; оно существует лишь как момент некоторого реального. {\em Бейль},
который в своем <<Dic\-tion\-naire>> (статья <<Зенон>>) находит данное
Аристотелем разрешение зеноновской диалектики pitoyable (жалким), не понимает,
что значит, что материя делима до бесконечности только в возможности; он
возражает, что если материя делима до бесконечности, то она {\em действительно}
содержит в себе бесконечное множество частей; это, следовательно, не
бесконечное en puissance (в~возможности), а такое бесконечное, которое
существует реально и актуально. "--- В~противоположность Бейлю следует,
наоборот, сказать, что уже сама {\em делимость} есть лишь возможность,
{\em а~не~существование частей}, и множественность вообще положена в
непрерывности лишь как момент, как снятое. "--- Остроумного рассудка, в котором
Аристотель, несомненно, также никем не превзойден, недостаточно для того, чтобы
понять и оценить его спекулятивные понятия, точно так же, как грубого
чувственного представления, о котором мы рассказали выше, недостаточно для
того, чтобы опровергнуть аргументацию Зенона. Этот рассудок заблуждается,
принимая за нечто истинное и действительное такие сочиненные мыслью вещи, такие
абстракции, как бесконечное множество частей; указанное же чувственное сознание
нельзя заставить перейти от эмпирии к мыслям.

Кантовское разрешение антиномии также состоит лишь в том, что разум не должен
{\em залетать} за пределы {\em чувственного восприятия}, а~должен брать явления
такими, каковы они есть. Это разрешение оставляет в стороне самое содержание
антиномии; оно не достигает природы {\em понятия} ее определений, каждое из
которых, взятое само по себе, изолированно, не имеет никакой силы (nich\-tig
ist) и есть само в~себе лишь переход в свое другое, имеет своим единством
количество и в этом единстве "--- свою истину.\label{bkm:bm88b}

\hegsection[В. Непрерывная и дискретная величина]%
{В. Непрерывная и дискретная величина}

1. Количество содержит в себе оба момента "--- непрерывность и дискретность.
Оно должно быть положено в обоих моментах как в своих определениях. "--- Оно
уже с самого начала есть {\em их непосредственное} единство, т.~е. само оно
ближайшим образом положено лишь в одном из своих определений, в непрерывности,
и есть, таким образом, {\em непрерывная величина}.

Или, иначе говоря, непрерывность есть, правда, один из моментов количества,
которое завершено лишь в соединении с другим моментом, с дискретностью. Однако
количество есть конкретное единство лишь постольку, поскольку оно есть единство
{\em различенных} моментов. Последние следует поэтому брать также и как
различенные; мы должны, однако, не снова разрешить их в притяжение и
отталкивание, а брать их согласно их истине, каждый в его единстве с другим,
т.~е. так, что каждый остается {\em целым}. Непрерывность есть лишь связное,
компактное единство как единство дискретного; {\em положенное} так, оно уже не
есть только момент, а все количество, {\em непрерывная величина}.

2. {\em Непосредственное} количество есть непрерывная величина. Но количество
не есть вообще некоторое непосредственное. Непосредственность "--- это та
определенность, снятостью которой является само количество. Последнее следует,
стало быть, положить в имманентной ему определенности, которой является одно.
Количество есть {\em дискретная величина}.

Дискретность есть подобно непрерывности момент количества, но она же сама есть
также и все количество, именно потому, что она есть момент в последнем, в~целом
и, следовательно, как различное не выступает из этого целого, из своего
единства с другим моментом. "--- Количество есть бытие вне-друг-друга в~себе,
а~непрерывная величина есть это бытие-вне-друг-друга как продолжающее себя без
отрицания, как в самой себе равная связь. Дискретная же величина есть эта
внеположность как не непрерывная, как прерываемая. Однако с этим множеством
одних у нас не получается снова множество атомов и пустота, вообще
отталкивание. Так как дискретная величина есть количество, то сама ее
дискретность непрерывна. Эта непрерывность в дискретном состоит в том, что одни
суть равное друг другу или, иначе говоря, в том, что они обладают одной и той
же {\em единицей}. Дискретная величина есть, следовательно, внеположность
многих одних, {\em как равных}, не многие одни вообще, а положенные как
{\em многие некоторой единицы}.

\hegremark[Примечание]%
{Обычное разлучение этих величин}%
{[Обычное разлучение этих величин]}

В обычных представлениях о непрерывной и дискретной величинах не принимают во
внимание того обстоятельства, что {\em каждая} из этих величин имеет в~себе оба
момента, как непрерывность, так и дискретность, и их отличие друг от друга
составляет только то, какой из двух моментов есть {\em положенная}
определенность и какой есть только в-себе-сущая определенность. Пространство,
время, материя и~т.~д. суть непрерывные величины, так как они суть отталкивания
от самих себя, изливающееся исхождение из себя, которое вместе с тем не есть
переход или отношение к некоторому качественно другому. Они имеют абсолютную
возможность того, чтобы одно повсюду было положено в них, положено не как
пустая возможность простого инобытия (как, например, говорят, что возможно,
чтобы вместо этого камня стояло бы дерево), а они содержат принцип одного в
самих себе; он есть одно из определений, из которых они конституированы.

Равным образом и обратно, в дискретной величине не следует упускать из вида
непрерывность; этим последним моментом, как показано, служит одно как единица.

Непрерывная и дискретная величины могут быть рассматриваемы как {\em виды}
количества, но лишь постольку, поскольку величина положена не под какой-нибудь
внешней определенностью, а под {\em определенностями ее собственных} моментов.
Обычный переход от рода к виду вводит в первый "--- согласно некоторому
{\em внешнему} ему основанию деления, "--- {\em внешние} определения.
Непрерывная и дискретная величины при этом еще не суть определенные величины;
они суть лишь само количество в каждой из его двух форм. Их называют величинами
постольку, поскольку они имеют вообще то общее с определенным количеством, что
они суть некоторая определенность в количестве.

\hegsection[С. Ограничение количества]{С. Ограничение количества}

Дискретная величина имеет, {\em во-первых}, принципом одно и есть,
{\em во-вторых}, множество одних; {\em в-третьих}, она по существу непрерывна,
она есть одно, вместе с тем как снятое, как {\em единица}, есть продолжение
себя как такового в дискретности многих одних. Она поэтому положена как
{\em единая} величина, и ее определенность есть одно, которое есть в этой
положенности и наличном бытии {\em исключающее} одно, граница в единице.
Дискретная величина как таковая, как предполагается, непосредственно не
ограничена: но как отличная от непрерывной величины она дана как некоторое
такое наличное бытие и нечто, определенность которого есть одно, а как
определенность в некотором наличном бытии есть также первое
отрицание и граница.

Эта граница, помимо того, что она соотнесена с единицей и есть отрицание
{\em в~последней, соотнесена} как одно также и {\em с самой собой;} таким
образом, она есть объемлющая, охватывающая граница. Граница здесь сначала не
отличается от <<нечто>> ее наличного бытия, а как одно, она непосредственно
есть сам этот отрицательный пункт. Но то бытие, которое здесь ограничено, дано
по существу как непрерывность, в силу которой оно выходит за границу и за это
одно, и безразлично к ним. Реальное дискретное количество есть, таким образом,
{\em некоторое} количество или, иначе говоря, определенное количество "---
количество как некоторое наличное бытие и нечто.

Так как то одно, которое есть граница, объемлет собою многие одни дискретного
количества, то она полагает их в такой же мере и как снятые в нем; она есть
граница в непрерывности вообще как таковой, и тем самым различие между
непрерывной и дискретной величинами здесь безразлично; или, правильнее, она
есть граница непрерывности как {\em одной}, так и {\em другой; обе} переходят
в~ней к~тому, чтобы быть определенными количествами.

\bigskip
