\subsection[Общее понятие логики]{Общее понятие логики}
Ни в какой другой науке не чувствуется так
сильно потребность начинать с самой сути дела, без предварительных
размышлений, как в науке логики. В~каждой другой науке рассматриваемый ею
предмет и научный метод отличны друг от друга; равным образом и содержание
этих наук не начинает абсолютно с самого начала, а зависит от других
понятий и находится в связи с облегающим его другим материалом. За этими
науками мы вследствие этого готовы признавать право говорить лишь
лемматически о почве, на которой они стоят, и о ее связи, равно как и о
методе, прямо применять предполагаемые известными и
принятыми формы дефиниций и~т.~п. и пользоваться для установления своих
всеобщих понятий и основных определений обычным способом рассуждения.

Логика же, напротив, не может брать в качестве предпосылки ни одной из этих
форм рефлексии или правил и законов мышления, ибо сами они составляют часть
ее содержания и должны впервые получить свое обоснование лишь в ее рамках.
Но в ее содержание входит не только указание научного метода, но и вообще
само {\em понятие науки} и притом это понятие
составляет ее последний результат; она поэтому не может сказать наперед,
что она такое, а лишь все ее изложение рождает впервые это знание о ней
самой как ее последнее слово и как ее завершение. И~точно так же ее
предмет, {\em мышление} или, говоря определеннее,
{\em мышление, постигающее в понятиях}, рассматривается
по существу внутри нее; понятие этого мышления порождает себя в ходе ее
развертывания и, стало быть, не может быть предпослано. То, что мы
предпосылаем здесь в этом введении, не имеет поэтому своей целью дать,
скажем, обоснование понятия логики или дать наперед научное оправдание ее
содержания и метода, а ставит себе целью путем некоторых разъяснений и
размышлений в рассуждающем и историческом духе растолковать представлению
ту точку зрения, с которой следует рассматривать эту науку.

Если в общем логику признают наукой о мышлении, то под этим понимают, что
это мышление представляет собой лишь {\em голую форму}
некоторого познания, что логика абстрагирует от всякого
{\em содержания}, и так называемая вторая
{\em составная часть} всякого познания,
{\em материя}, должна быть дана откуда-то извне, что,
следовательно, логика, от каковой эта материя всецело независима, может
только указать формальные условия истинного познания, но не может содержать
в себе самой реальной истины, не может даже быть
{\em путем} к реальной истине, так как именно
существенное в истине, содержание, лежит вне ее.

Но, во-первых, неудачно уже то выражение, что логика абстрагирует от всякого
{\em содержания}, что она только учит правилам
мышления, не имея возможности вдаваться в рассмотрение мыслимого и его
характера. Ибо раз утверждают, что ее предметом служат мышление и правила
мышления, то ведь выходит, что она в них непосредственно имеет свое, ей
лишь свойственное содержание; в них она имеет также и вышеназванную вторую
составную часть познания, некую материю, характер которой ее интересует.

А, во-вторых, нужно сказать вообще, что представления, на которых до сих пор
основывалось понятие логики, отчасти уже отошли в прошлое, отчасти им давно
пора полностью сойти со сцены, давно пора, чтобы понимание этой науки
исходило из более высокой точки зрения и чтобы она получила совершенно
измененный вид.

Понятие логики, которого придерживались до сих пор, основано на раз навсегда
принятой обычным сознанием предпосылке о раздельности {\em содержания}
познания и его {\em формы}, или, иначе сказать, {\em истины} и
{\em достоверности}. Предполагается, {\em во-первых}, что материя познания
имеется налицо, как некий готовый мир, вне мышления, сама по себе, что
мышление, взятое само по себе, пусто, привходит к указанной материи как
некая форма извне, наполняется ею, лишь в ней приобретает некоторое
содержание и благодаря этому становится реальным познанием.

{\em Во-вторых}, эти две составные части (ибо
предполагается, что они находятся между собою в отношении составных частей
и познание составляется из них механически или, в лучшем случае, химически)
находятся согласно этому воззрению в следующем иерархическом отношении:
объект есть нечто само по себе завершенное, готовое, могущее для своей
действительности вполне обойтись без мышления, тогда как мышление есть,
напротив, нечто ущербное, которому еще предстоит восполнить себя в
некоторой материи, и притом оно в качестве мягкой неопределенной формы
должно сделать себя соответственным своей материи. Истина есть соответствие
мышления предмету, и для того, чтобы создать такое соответствие, ибо само
по себе оно не дано как нечто наличное, мышление должно подчиняться
предмету, приспособляться к нему.

{\em В-третьих}, так как различие материи и формы,
предмета и мышления не оставляется в указанной туманной неопределенности, а
берется в более определенном смысле, то выходит, что каждая из них
представляет собою отделенную от другой сферу. Поэтому мышление,
воспринимая и формируя материю, не выходит за свои пределы, остается в
своем воспринимании последней и в приспособлении к ней некоторым
собственным видоизменением, не становится вследствие этого своим другим; а
самосознательный процесс определения уж во всяком случае принадлежит лишь
ему. Оно (мышление), стало быть, даже в своем отношении к предмету не
выходит за себя, не переходит к предмету; последний как некая вещь в себе
остается чем-то всецело потусторонним мышлению.

Эти взгляды на отношение между субъектом и объектом служат выражением тех
определений, которые составляют природу нашего обычного, являющегося
сознания. Но когда эти предрассудки переносятся в область разума, как будто
и в нем имеет место то же самое отношение, как будто это отношение истинно
само по себе, "--- когда, говорю я, эти взгляды переносятся в область разума,
они оказываются теми заблуждениями, опровержением которых, проведенным
через все части духовного и природного универсума, является философия или,
правильнее сказать, оказываются заблуждениями, от которых следует
освободиться до того, как приступить к философии, так как они заграждают
вход в нее.

Прежняя метафизика имела в этом отношении более высокое понятие о мышлении,
чем то, которое сделалось ходячим в новейшее время. Она клала именно в
основание своего понимания воззрение, согласно которому то, что познается
мышлением о и в предметах, единственно и есть в них истинно истинное;
следовательно, таким истинным служат не предметы в своей
непосредственности, а предметы, возведенные сначала в форму мышления,
предметы как мыслимые. Эта метафизика, стало быть, считала, что мышление и
определения мышления суть не нечто чуждое предметам, а, наоборот, их
сущность, или, иначе говоря, считала, что {\em вещи} и
{\em мышление} о них сами по себе совпадают (как и наш
язык выражает их сродство\footnote{Dinge (вещи) и Denken (мышление) имеют
некоторое сходство в своем звучании и начертании. На это намекает
Гегель. "--- {\em Перев}.}, что мышление в своих имманентных определениях
и истинная природа вещей суть одно и то же.

Но философией овладел {\em рефлектирующий} рассудок. Мы
должны точно знать, что означает это выражение, которое часто употребляется
просто как эффектное словечко (Schlagwort). Под ним следует вообще понимать
абстрагирующий и, следовательно, разделяющий рассудок, который упорствует в
своих разделениях. Обращенный против разума, он ведет себя как
{\em обыкновенный здравый смысл} и выдвигает свой
взгляд, согласно которому истина покоится на чувственной реальности, мысли
суть {\em только} мысли в том смысле, что только
чувственное восприятие впервые сообщает им содержательность (Gehalt) и
реальность, что разум, поскольку он остается сам по себе, порождает лишь
химерические домыслы\pagenote{Ср. выше
замечание Гегеля об <<экзотерическом учении кантовской философии>>
на стр.~\pageref{bkm:Ref474526580}.}.
В этом отречении разума от самого себя утрачивается понятие истины, его
(разум) ограничивают познанием исключительно только субъективной истины,
только явления, только чего-то такого, чему не соответствует природа самой
вещи; {\em знание} обратилось вспять, упало на степень {\em мнения}.

Однако этот оборот, принятый познанием и представляющийся потерей и шагом
назад, имеет более глубокое основание, на котором вообще покоится
возведение разума в высший дух новейшей философии. А~именно основание
вышеуказанного, ставшего всеобщим, представления следует искать в
правильном усмотрении того, что определения рассудка
{\em необходимо сталкиваются} с самими собою. "---
Вышеупомянутая рефлексия заключается в том, что выходят
{\em за пределы} конкретно непосредственного и
{\em определяют} и {\em разделяют}
его. Но она необходимо должна {\em равным образом}
выходить и {\em за пределы} этих своих
{\em разделяющих} определений и прежде всего
{\em соотносить} их. В~стадии (auf den Standpunkte)
этого соотнесения выступает наружу их столкновение. Это совершаемое
рефлексией соотнесение принадлежит в себе к разуму. Возвышение над
указанными определениями, достигающее усмотрения их столкновения, есть
великий отрицательный шаг к истинному понятию разума. Но это не доведенное
до конца усмотрение создает недоразумение, будто именно разум и впадает в
противоречие с собою; оно не познает, что противоречие как раз и есть
возвышение разума над ограниченностями рассудка и их разрешение. Вместо
того, чтобы сделать отсюда последний шаг ввысь, познание
неудовлетворительности рассудочных определений обратилось в бегство назад,
к чувственному существованию, ошибочно полагая, что в нем оно обладает
устойчивым и единым. Но так как, с другой стороны, это познание признает
себя познанием только явлений, то оно тем самым соглашается, что
чувственное существование неудовлетворительно, но вместе с тем
предполагает, будто, хотя вещи в себе и не познаются, все же внутри сферы
явлений получается правильное познание. Выходило, как будто различны только
{\em роды предметов}, и один род предметов, а именно
вещи в себе, не познается, другой же род предметов, а именно явления,
оказывается познаваемым. Это похоже на то, как если бы мы приписывали
кому-нибудь правильное усмотрение, но при этом прибавили бы, что он,
однако, способен усматривать не истинное, а только ложное. Если признать,
что такое высказывание было бы несуразно, то следует также признать, что не
менее несуразно истинное познание, не познающее предмета, как он есть в себе.

{\em Критика форм рассудка} привела к вышеуказанному
выводу, что эти формы не имеют {\em применения к вещам
в себе}. Это может иметь только тот смысл, что эти формы суть в самих себе
нечто неистинное. Но так как эта критика продолжает считать их значимыми
для субъективного разума и для опыта, то она в них самих не произвела
никакой перемены, а оставляет их для субъекта в том же виде, в каковом они
прежде обладали значимостью для объекта. Но если они недостаточны для
познания вещи в себе, то рассудок, которому, как утверждает эта критика,
они принадлежат, должен был бы еще менее охотно допускать их и
довольствоваться ими. Если они не могут быть определениями вещи в себе, то
они еще того менее могут быть определениями рассудка, за которым мы должны
были бы признать по крайней мере достоинство некоторой вещи в себе.
Определения конечности и бесконечности одинаково сталкиваются между собою,
будем ли мы применять их к времени и пространству, к миру или они будут
признаны определениями внутри духа, точно так же, как черное и белое
образуют все равно серое, смешаем ли мы их вместе на стене или только на
палитре. Если наше представление о {\em мире}
разрушается, когда мы на него переносим определения бесконечного и
конечного, то сам {\em дух}, содержащий в себе эти два
определения, должен в еще большей мере оказаться чем-то противоречивым в
самом себе, чем-то разрушающимся. Характер материй или предметов, к которым
мы стали бы их применять или в которых они находятся, не может составлять
здесь какой бы то ни было разницы, ибо предмет имеет в себе противоречие
только через указанные определения и согласно им.

Вышеуказанная критика, стало быть, удалила формы объективного мышления
только из предметов, но оставила их в субъекте в том виде, в каком она их
нашла. А~именно, она не рассмотрела этих форм, взятых сами по себе,
согласно их своеобразному содержанию, а прямо заимствовала их лемматически
из субъективной логики. Таким образом, не было и речи о выводе их в них
самих или хотя бы о выводе их как субъективно-логических форм, а еще менее
о диалектическом их рассмотрении.

Более последовательно проведенный трансцендентальный идеализм познал
никчемность еще сохраненного критической философией призрака
{\em вещи в себе}, этой абстрактной, оторванной от
всякого содержания тени, и он поставил себе целью окончательно его
уничтожить\pagenote{Гегель имеет в
виду философию И. Г. Фихте, который был еще жив, когда Гегель писал свою
<<Науку логики>> (1812).}.
Эта философия, кроме того, положила начало попытке дать разуму развернуть
свои определения из самого себя. Но субъективная позиция этой попытки не
позволила завершить ее. В~дальнейшем отказались от этой позиции, а с нею и
от той начатой попытки и разработки чистой науки.

Но всецело вне всякого отношения к метафизическому значению рассматривается
то, что обыкновенно понимают под логикой. Эта наука в том состоянии, в
каком она еще находится, не имеет того рода содержания, которое признается
в обычном сознании реальностью и некоей истинной вещью. Но не вследствие
этого она есть формальная, лишенная всякой содержательной истины наука.
Ведь в той материи, которой в ней не находят и отсутствию которой
обыкновенно приписывают ее неудовлетворительность, мы и помимо этого не
должны искать истины. Бессодержательность логических форм получается
единственно только вследствие способа их рассмотрения и трактовки. Так как
они в качестве застывших определений лишены связи друг с другом и не
удерживаются вместе в органическом единстве, то они представляют собою
мертвые формы и в них не обитает дух, составляющий их живое конкретное
единство. Но тем самым им недостает самородного содержания "--- материи,
которая была бы в самой себе содержанием. Содержание, которого мы не
находим в логических формах, есть не~что иное, как некоторая прочная основа
и сращение (Konkretion) этих абстрактных определений, и обычно ищут для них
такой субстанциальной сущности вне логики. Но в действительности сам
логический разум и есть то субстанциальное или реальное, которое сцепляет в
себе все абстрактные определения, и он есть их самородное, абсолютно
конкретное единство. Нет, следовательно, надобности далеко искать то, что
обыкновенно называют материей. Если логика, как утверждают, лишена
содержания, то это вина не предмета логики, а исключительно только способа
его понимания.

Это размышление дает нам возможность приступить к указанию той точки зрения,
с которой мы должны рассматривать логику, поскольку эта точка зрения
отличается от прежней трактовки этой науки и представляет собою ту
единственно истинную точку зрения, на которую она впредь должна быть
поставлена раз навсегда.

В <<{\em Феноменологии духа}>> я изобразил сознание в его
поступательном движении от первой непосредственной противоположности между
ним и предметом до абсолютного знания. Этот путь проходит через все формы
{\em отношения сознания к объекту} и имеет своим
результатом {\em понятие науки}. Это понятие,
следовательно (независимо от того, что оно возникает в рамках самой
логики), не нуждается здесь в оправдании, так как это оправдание получено
уже там; и оно не может иметь никакого другого оправдания, кроме этого его
порождения сознанием, для которого все его собственные образы разрешаются в
это понятие, как в истину. Резонерское обоснование или разъяснение
понятия науки может самое большее дать лишь то, что последнее будет
поставлено перед представлением и о нем будут получены исторические
сведения; но дефиниция науки, или, говоря более определенно, логики, имеет
свое {\em доказательство} исключительно только в
вышеуказанной необходимости ее происхождения. Та дефиниция, которой
какая-либо наука начинает абсолютно с самого начала, не может содержать в
себе ничего другого, кроме как определенного корректного выражения того,
что, {\em как известно} и {\em общепризнано},
{\em представляют себе} под предметом и целью этой
науки. Что под этим предметом и целью представляют себе именно то-то, это
есть историческое уверение, в отношении которого можно сослаться
единственно только на то, что то или другое является призванным или,
собственно говоря, в порядке просьбы предложить читателю, чтобы он считал
то или другое признанным. В~действительности это вовсе не прекращает того,
что то тут, то там отдельные авторы приводят какой-нибудь новый случай или
пример, показывающий, что под таким-то выражением нужно понимать еще нечто
большее и другое и что, следовательно, в его дефиницию следует включить еще
одно более частное или более общее определение и согласно с этим должна
быть перестроена и наука. "--- При этом, далее, только от резонерства
зависит то, до какой границы и в каком объеме те или иные определения
должны быть включены или исключены; само же рассуждательство имеет перед
собою на выбор многообразнейшие и различнейшие мнения, относительно
которых, в конце концов, единственно только произвол может давать решающее
заключение. При этом способе начинать науку с ее дефиниции не заходит и
речи о потребности показать {\em необходимость ее
предмета} и, следовательно, также и ее самой.

Итак, понятие чистой науки и его дедукция берутся в настоящем произведении
как предпосылка постольку, поскольку <<Феноменология духа>> представляет
собою не~что иное, как эту дедукцию. Абсолютное знание есть
{\em истина} всех способов сознания, потому что, как к
этому привело описанное в <<Феноменологии духа>> шествие сознания, лишь в
абсолютном знании полностью растворилась разлученность
{\em предмета} и {\em достоверности} самого себя, и истина стала равной
этой достоверности, равно как и эта достоверность стала равной истине.

Чистая наука, стало быть, предполагает освобождение от противоположности
сознания [и его предмета]. Она содержит в себе мысль, поскольку последняя
есть также и вещь (die Sache) {\em в самой себе}, или
{\em вещь} (die Sache) {\em в самой себе, поскольку последняя есть также
и чистая мысль}. В~качестве {\em науки} истина есть чистое развивающееся
самосознание, имеет образ самости, так что {\em в себе и для себя сущее
есть знаемое понятие, а понятие как таковое есть в себе и для себя сущее}.

Это объективное мышление и есть {\em содержание} чистой
науки. Она поэтому столь мало формальна, столь мало лишена материи для
действительного и истинного познания, что, наоборот, только ее содержание и
есть абсолютно истинное, или, если еще угодно пользоваться словом
<<материя>>, истинная материя, но такая материя, для которой форма не есть
нечто внешнее, так как эта материя есть, наоборот, чистая мысль и,
следовательно, есть сама абсолютная форма. Логику согласно этому следует
понимать как систему чистого разума, как царство чистой мысли.
{\em Это царство есть истина, какова она без покровов,
в себе и для себя самой}. Можно поэтому выразиться так: это содержание есть
{\em изображение бога, каков он есть в своей вечной
сущности до сотворения природы и какого бы то ни было конечного духа}.

{\em Анаксагор} восхваляется как тот, который впервые
высказал ту мысль, что {\em нус},
{\em мысль}, есть основоначало (Prinzip) мира, что мы
должны определять сущность мира как мысль. Он этим положил основание тому
интеллектуальному воззрению на вселенную, чистой формой которого должна
быть {\em логика}. В~ней мы имеем дело не с мышлением о
чем-то таком, что лежало бы в основании и существовало бы особо, вне
мышления, не с формами, которые якобы дают только
{\em признаки} истины, а необходимые формы и
собственные определения мышления суть само содержание, сама высшая истина.

Для того чтобы представление по крайней мере понимало, в чем дело, следует
отбросить в сторону мнение, будто истина есть нечто осязаемое. Такой
характер осязаемости вносят, например, даже в платоновские идеи, имеющие
бытие в мышлении бога, толкуя их так, будто они суть как бы существующие
вещи, но существующие в некотором другом мире или области, вне которой
находится мир действительности, обладающий отличною от этих идей и только
благодаря этой отличности реальною субстанциальностью. Платоновская идея
есть не~что иное, как всеобщее, или, говоря более определенно, не~что иное,
как понятие предмета; лишь в своем понятии нечто обладает
действительностью; поскольку же оно отлично от своего понятия, оно
перестает быть действительным и есть нечто ничтожное; аспект (Seite)
осязаемости и чувственного вне-себя-бытия принадлежит этой ничтожной
стороне (Seite).

Но, с другой стороны, можно сослаться на собственные представления обычной
логики; в ней ведь принимается, что, например, дефиниции содержат в себе не
определения, находящиеся лишь в познающем субъекте, а определения предмета,
составляющие его наисущественнейшую, наисобственнейшую природу. Или другой
пример: когда умозаключают от данных определений к другим, то принимают,
что определения, полученные в результате умозаключения, не суть нечто
внешнее и чуждое предмету, а что, напротив, они принадлежат самому этому
предмету, что этому мышлению соответствует бытие. Вообще при употреблении
форм понятия, суждения, умозаключения, дефиниции, разделения и~т.~д. в
основании лежит предпосылка, что они суть формы не только самосознательного
мышления, но и предметного смысла (Verstandes). "---
<<{\em Мышление}>> есть выражение, под которым
разумеется, что содержащиеся в нем определения приписываются
преимущественно сознанию. Но поскольку говорят, что в
{\em предметном мире есть смысл} (Verstand),
{\em разум}, что дух и природа имеют
{\em всеобщие законы}, согласно которым протекает их
жизнь и совершаются их изменения, постольку признают, что определения мысли
обладают также и объективными ценностью и существованием.

Критическая философия, правда, уже превратила
{\em метафизику} в {\em логику;}
однако она подобно позднейшему
идеализму\pagenote{Имеется в виду
субъективный идеализм И.~Г.~Фихте.}
из страха перед объектом придала, как мы уже сказали выше, логическим
определениям существенно субъективное значение; вследствие этого они вместе
с тем оставались обремененными тем объектом, которого они стремились
избежать, и в них оставалась как некоторое потустороннее вещь в себе,
оставался бесконечный толчок. Но освобождение от противоположности сознания
[и его предмета], которое наука должна иметь возможность предположить,
поднимает определения мысли выше этой робкой, незавершенной точки зрения и
требует, чтобы их рассматривали такими, каковы они суть в себе и для себя,
без такого рода ограничения и отношения, требует, чтобы их рассматривали
как логическое, как чисто разумное.

{\em Кант} в одном месте\pagenote{<<Критика чистого
разума>>, предисловие ко 2-му изд., стр.~VIII (во 2-м изд. перевода
Лосского, Петроград 1915, стр.~9).}
считает логику, а именно тот агрегат определений и положений, который в
обычном смысле носит название логики, счастливой тем, что она сравнительно
с другими науками достигла такого раннего завершения; со времени
{\em Аристотеля} она, по его словам, не сделала ни
одного шага назад, но также и ни одного шага вперед; последнего она не
сделала потому, что она по всем признакам, по-видимому, закончена и
завершена. Но если со времени Аристотеля логика не подверглась никаким
изменениям, "--- и в самом деле при рассмотрении новых учебников логики мы
убеждаемся, что изменения сводятся часто больше всего к сокращениям, "--- то
мы отсюда должны сделать скорее тот вывод, что она тем больше нуждается в
полной переработке; ибо двухтысячелетняя непрерывная работа духа должна
была ему доставить более высокое сознание о своем мышлении и о своей чистой
сущности в самой себе. Сравнение образов, до которых поднялись дух
практического и религиозного миров и научный дух во всякого рода реальном и
идеальном сознании, с образом, который носит логика (его сознание о своей
чистой сущности), являет столь огромное различие, что даже при самом
поверхностном рассмотрении не может не бросаться тотчас же в глаза, что это
последнее сознание совершенно не соответствует тем взлетам и недостойно их.

И в самом деле, потребность в преобразовании логики чувствовалась давно. Мы
имеем право сказать, что в том виде, в каком логика излагается в учебниках,
она как со стороны своей формы, так и со стороны своего содержания
сделалась предметом презрения. Ее еще тащат за собою больше вследствие
смутного чувства, что совершенно без логики нельзя вообще обойтись, и
вследствие продолжающейся привычки к традиционному представлению о ее
важности, чем из убеждения, что то обычное содержание и занятие теми
пустыми формами имеет ценность и приносит пользу.

Расширение, которое она получала в продолжение некоторого времени благодаря
добавлению психологического, педагогического и даже физиологического
материала, было затем признано почти всеми за искажения. Взятые сами по
себе, большая часть этих психологических, педагогических, физиологических
наблюдений, законов и правил все равно, излагались ли они в логике или в
какой-либо другой науке, должны представляться очень пустыми и
тривиальными. А~уж такие, например, правила, что следует продумывать и
подвергать критическому разбору прочитываемое в книгах или слышанное, что
кто плохо видит, должен приходить на помощь своим глазам и надевать очки
(правила, дававшиеся учебниками в так называемой прикладной логике и притом
с серьезным видом разделенные на параграфы, дабы люди достигли истины), "---
уж такие правила должны представляться излишними всем, кроме разве автора
учебника или преподавателей, не знающих, как расширить слишком краткое и
мертвенное содержание логики\pagenote{Гегель имеет в
виду логические сочинения Христиана Вольфа (1679---1754) и его
последователей. К~этому месту текста в 1-м издании <<Науки логики>> (Нюрнберг
1812) имелось следующее примечание Гегеля: <<Одна только что появившаяся
новейшая обработка этой науки "--- <<Система логики>> Фриса "--- возвращается
к антропологическим основам. Крайняя поверхностность лежащего в основе этой
<<Системы логики>> представления или мнения самого по себе, а равно и его
разработки избавляет меня от труда в какой бы то ни было мере считаться с
этим лишенным всякого значения произведением>>. Упоминаемая в этом
примечании <<Система логики>> Я.~Ф.~Фриса (1773---1843) вышла 1-м изданием в
1811~г., 2-м изданием в 1819~г., 3-м "--- в 1837~г.}.

Что же касается этого содержания, то мы уже указали выше, почему оно так
плоско. Даваемые им застывшие определения считаются незыблемыми и
приводятся лишь во внешнее взаимоотношение друг с другом. Вследствие того,
что в суждениях и умозаключениях операции сводятся главным образом к
количественной стороне определений и обосновываются только ею, все
оказывается покоящимся на внешнем различии, на голом сравнении, все
становится совершенно аналитическим способом рассуждения и лишенным понятия
вычислением. Дедукция так называемых правил и законов, в особенности
законов и правил умозаключения, немногим лучше, чем перебирание палочек
неравной длины в целях их сортирования и соединения сообразно их величине
или чем служащее игрой детей подбирание подходящих частей разнообразно
разрезанных картинок. Поэтому не без основания приравнивали это мышление к
счету и в свою очередь счет к этому мышлению. В~арифметике числа берутся
как нечто, лишенное понятия, как нечто такое, что помимо своего равенства
или неравенства, т.~е. помимо своего совершенно внешнего отношения, не
обладает значением, "--- берутся как нечто такое, что ни в самом себе, ни в
своих отношениях не есть мысль. Когда мы механически вычисляем, что три
четверти, помноженные на две трети, дают половину, то это действие содержит
в себе примерно столь же много или столь же мало мыслей, как и соображение
о том, может ли иметь место в данной фигуре тот или другой вид умозаключения.

Дабы эти мертвые кости логики оживотворились духом и получили, таким
образом, содержимое и содержание, ее {\em методом}
должен быть тот, который единственно только и способен сделать ее чистой
наукой. В~том состоянии, в котором она находится, нет даже предчувствия
научного метода. Она носит примерно форму опытной науки. Опытные науки
нашли для того, чем они должны быть, свой особый метод, метод дефинирования
и классифицирования своего материала, насколько это возможно. Чистая
математика также имеет свой метод, который подходит для ее абстрактных
предметов и для тех количественных определений, единственно в которых она
их рассматривает. То существенное, что можно сказать об этом методе и
вообще о низшем характере той научности, который может иметь место в
математике, я высказал в предисловии к <<Феноменологии духа>>, но он будет
рассмотрен нами более подробно в рамках самой логики.
{\em Спиноза}, {\em Вольф} и другие
впали в соблазн применить этот метод также и к философии и сделать внешний
ход чуждого понятию количества ходом понятия, что само по себе
противоречиво. До сих пор философия еще не нашла своего метода. Она
смотрела с завистью на систематическое сооружение математики и, как мы
сказали, пробовала заимствовать у нее ее метод или обходилась методом тех
наук, которые представляют собою лишь смесь данного материала, опытных
положений и мыслей, или, наконец, выходила из затруднения, тем, что грубо
отбрасывала всякий метод. Но раскрытие того, что единственно только и может
служить истинным методом философской науки, составляет предмет самой
логики, ибо метод есть сознание о форме внутреннего самодвижения ее
содержания. Я~в <<{\em Феноменологии духа}>> дал образчик
этого метода в применении к более конкретному предмету, к
{\em сознанию}\footnote{Позднее же "--- в применении
и к другим конкретным предметам и соответственно частям философии.}.
Там я показал образы сознания, каждый из которых в своей реализации вместе
с тем разлагает самого себя, имеет своим результатом свое собственное
отрицание, "--- и тем самым перешел в некоторый высший образ. Единственно
нужным для того, {\em чтобы получить научное
поступательное движение}, "--- и о приобретении совершенно
{\em простого} усмотрения, которого следует главным
образом стараться, "--- является познание логического положения, что
отрицательное вместе с тем также и положительно или, иначе говоря, что
противоречащее себе не переходит в нуль, разрешается не в абсолютное ничто,
а по существу только в отрицание своего
{\em особенного} содержания, или, еще иначе, что такое
отрицание есть не всяческое отрицание, а {\em отрицание
определенной вещи}, которая разлагает себя, что такое отрицание есть,
следовательно, определенное отрицание и что, стало быть, в результате
содержится по существу то, результатом чего он является; и это, собственно
говоря, есть по существу тавтология, ибо в противном случае он был бы
чем-то непосредственным, а не результатом. Так как получающееся в качестве
результата отрицание есть {\em определенное} отрицание,
то оно имеет некоторое {\em содержание}. Оно есть новое
понятие, но более высокое, более богатое понятие, чем предыдущее, ибо оно
обогатилось его отрицанием или противоположностью; оно, стало быть,
содержит в себе старое понятие, но содержит в себе более, чем только это
понятие, и есть единство его и его противоположности. Таким путем должна
вообще образоваться система понятий, "--- и в неудержимом, чистом, ничего не
принимающем в себя извне движении получить свое завершение.

Я, разумеется, не могу полагать, что метод, которому я следовал в этой
системе логики или, вернее, которому следовала в себе самой эта система, не
допускает еще многих усовершенствований, большой обработки в частностях, но
я знаю вместе с тем, что он является единственно истинным. Это само по себе
явствует уже из того, что он не есть нечто отличное от своего предмета и
содержания, ибо движет себя вперед содержание внутри себя,
{\em диалектика, которую оно имеет в самом себе}. Ясно,
что никакие изложения не могут считаться научными, если они не следуют по
пути этого метода и не соответственны его простому ритму, ибо движение
этого метода есть движение самой сути дела.

В соответствии с этим методом я напоминаю, что подразделения и заглавия
книг, отделов и глав, данные в настоящем сочинении, равно как и связанные с
ними объяснения, делаются для предварительного обзора и что они, собственно
говоря, обладают только {\em исторической} ценностью.
Они не входят в состав содержания и корпуса науки, а суть резюме, сделанное
внешней рефлексией, которая уже прошлась по всему целому, знает поэтому
наперед порядок следования его моментов и указывает их еще прежде, чем они
введут себя благодаря самой сути дела.

В других науках такие данные наперед определения и подразделения, взятые
сами по себе, также суть не~что иное, как такие внешние указания; но эти
науки отличаются от философии тем, что даже и внутри их эти данные наперед
определения и подразделения не поднимаются выше такого характера. Даже в
<<Логике>> мы читаем, например: <<Логика имеет два главных отдела, общую часть
и методику>>. А~затем в общей части мы без дальнейших объяснений встречаем
такие {\em заголовки}, как: <<Законы мышления>>, и далее:
{\em первая глава}: <<О понятиях>>. Первый раздел: <<О
ясности понятий>> и~т.~д. Эти данные без всякой дедукции, без всякого
обоснования определения и подразделения образуют систематический остов и
всю связь подобных наук. Такого рода логика видит свое призвание в том,
чтобы говорить, что понятия и истины должны быть
{\em выведены} из принципов; но в том, что она называет
методом, нет и намека на мысль о выведении. Порядок состоит здесь примерно
в том, что однородное сводится вместе, рассмотрение более простого
предпосылается рассмотрению сложного, и в других подобного рода внешних
соображениях. А~в отношении внутренней необходимой связи дело
ограничивается перечнем названий отделов, и переход осуществляется лишь
так, что ставят теперь: <<{\em Вторая глава}>> или пишут:
<<{\em Мы переходим теперь} к суждениям>> и~т.~д.

Заглавия и подразделения, встречающиеся в настоящей системе, также сами по
себе не имеют никакого другого значения, помимо предварительного заявления
о последующем содержании. Но помимо этого при рассмотрении самого ее
предмета должны в ней найти себе место {\em необходимость} связи и
{\em имманентное возникновение} различий, ибо они
входят в собственное поступательное движение понятия.

Тем, с помощью чего понятие ведет само себя дальше, является то
вышенамеченное отрицательное, которое оно имеет в самом себе; это
составляет подлинно диалектическое. {\em Диалектика},
которая рассматривалась, как некоторая обособленная часть логики и
относительно цели и точки зрения которой господствовало, можно сказать,
полнейшее непонимание, получает благодаря этому совсем другое положение.
И~{\em платоновская} диалектика даже в <<Пармениде>>, а в
других произведениях еще прямее, также имеет своей целью отчасти только
разлагание и опровержение ограниченных утверждений через них же самих,
отчасти же имеет вообще своим результатом ничто. Обыкновенно видят в
диалектике лишь внешнее и отрицательное делание, не принадлежащее к составу
самого предмета рассмотрения, вызываемое только тщеславием, как некоторой
субъективной страстью колебать и разлагать прочное и истинное, или видят в
ней по крайней мере нечто, приводящее к ничто, как к тому, что показывает
тщету диалектически рассматриваемого предмета.

{\em Кант} отвел диалектике более высокое место, и эта
сторона его философии принадлежит к величайшим его заслугам. Он освободил
ее от видимости произвола, который согласно обычному представлению присущ
ей, и изобразил ее как некоторое {\em необходимое
делание разума}. Пока ее считали только умением проделывать фокусы и
вызывать иллюзии, до тех пор предполагалось, что она просто ведет фальшивую
игру и вся ее сила состоит в том, что ей удается прикрыть обман, и выводы,
к которым она приходит, получаются хитростью и представляют собою
субъективную видимость. Диалектические рассуждения Канта в отделе об
антиномиях чистого разума оказываются, правда, не заслуживающими большой
похвалы, если присмотреться к ним ближе, как мы в дальнейшем это сделаем в
настоящем произведении более пространно; однако положенная им в основание
общая идея состоит в {\em объективности видимости и в
необходимости противоречия}, принадлежащего к природе определений мысли.
Правда, у Канта противоречие носит такой характер лишь постольку, поскольку
разум применяет эти определения к {\em вещам в себе;}
но ведь как раз то, что они суть в разуме и по отношению к тому, что есть в
себе, и есть их природа. Этот результат,
{\em понимаемый в его положительной стороне}, есть не
что иное, как их внутренняя {\em отрицательность}, их
движущая сама себя душа, вообще "--- принцип всякой природной и духовной
жизненности. Но так как Кант не идет дальше абстрактно-отрицательной
стороны диалектики, то его конечным выводом оказывается лишь известное
утверждение, что разум неспособен познать бесконечное; странный вывод: так
как бесконечное есть разумное, то это значит сказать, что разум не способен
познать разумное.

В этом диалектическом, как мы его берем здесь, и, следовательно, в
постижении противоположностей в их единстве, или, иначе говоря, в
постижении положительного в отрицательном, состоит
{\em спекулятивное}. Это есть важнейшая, но для еще
неискушенной, несвободной силы мышления также и труднейшая сторона. Если
эта сила мышления еще не окончила дела своего освобождения от
чувственно-конкретного представления и от резонерства, то она должна
сначала упражняться в абстрактном мышлении, удерживать понятия в их
{\em определенности} и научиться познавать, исходя из
них. Изложение логики, имеющее в виду эту цель, должно было бы держаться в
своем методе вышеупомянутых подразделений, а в отношении ближайшего
содержания "--- определений, получающихся касательно отдельных понятий, не
вдаваясь пока в диалектическую сторону. Оно стало бы похожим по своему
внешнему облику на обычное изложение этой науки, но, впрочем, отличалось бы
также и от последнего по своему содержанию, и все еще служило бы к тому,
чтобы упражнять хотя и не спекулятивное, но все же абстрактное мышление; а
ведь обычная логика, изложение которой сделалось популярным благодаря
психологическим и антропологическим добавкам, не достигает даже и этой
цели. То изложение логики доставляло бы уму образ методически
упорядоченного целого, хотя сама душа, имеющая свою жизнь в диалектическом,
душа этого построения "--- метод "--- в нем отсутствовала бы.

Касательно {\em образовательного значения логики} и
{\em отношения индивидуума к ней} я сделаю в заключение
еще то замечание, что эта наука, подобно грамматике, выступает в двух видах
и сообразно с этим имеет двоякого рода ценность. Она представляет собою
одно для того, кто впервые приступает к ней и вообще к наукам, и нечто
другое для того, кто возвращается к ней от них. Тот, кто только начинает
знакомиться с грамматикой, находит в ее формах и законах сухие абстракции,
случайные правила и вообще изолированное множество определений,
показывающих только ценность и значение того, что заключается в их
непосредственном смысле; познание не познает в них ближайшим образом ничего
кроме них. Напротив, кто владеет вполне каким-нибудь языком и вместе с тем
знает также и другие языки, которые он сопоставляет с первым, только тот и
может почувствовать дух и образование народа в грамматике его языка; те же
самые правила и формы имеют теперь для него наполненную содержанием, живую
ценность. Он может сквозь грамматику познать выражение духа вообще, логику.
Точно так же, кто только приступает к науке, находит сначала в логике
изолированную систему абстракций, ограничивающуюся самой собой, не
захватывающую других познаний и наук. В~сопоставлении с богатством
представления о мире, с реально выступающим содержанием других наук и в
сравнении с обещанием абсолютной науки раскрыть сущность этого богатства,
{\em внутреннюю природу} духа и мира,
{\em истину}, эта наука в ее абстрактном облике, в
бесцветной, холодной простоте ее чистых определений кажется скорее чем-то
исполняющим все, что угодно, но не это обещание, и являющимся
бессодержательным в сопоставлении с этим богатством. При первом знакомстве
с логикой ее значение ограничивают только ею самой; ее содержание
признается только изолированным рассмотрением определений мысли,
{\em наряду} с которым другие научные занятия
представляют собою особую материю и самостоятельное содержание, на которые
логическое имеет лишь некоторое формальное влияние, да притом такое
влияние, которое больше сказывается само собою и в отношении которого уже
во всяком случае можно в крайности обойтись без научного строя логики и его
изучения. Другие науки отбросили в целом метод, по всем правилам искусства
превращающий их в последовательный ряд дефиниций, аксиом, теорем и их
доказательств и~т.~д.; так называемая естественная логика являет в них свою
силу самостоятельно и обходится без особого, направленного на само мышление
познания. И, наконец, материал и содержание этих наук, взятые сами по себе,
уж во всяком случае остаются независимыми от логического и являются также и
более привлекательными для ощущения, чувства, представления и всякого рода
практических интересов.

Таким образом, логику приходится во всяком случае первоначально изучать как
нечто такое, что мы, правда, понимаем и что является для нас убедительным,
но в чем мы не находим сначала большого объема, большой глубины и более
широкого значения. Лишь благодаря более глубокому знакомству с другими
науками логика поднимается для субъективного духа на подобающую высоту,
выступает не только как абстрактно всеобщее, но как всеобщее, охватывающее
собою также и богатство особенного, подобно тому, как одно и то же
нравоучительное изречение в устах юноши, понимающего его совершенно
правильно, не имеет для него того значения и охвата, которые оно имеет для
ума умудренного жизнью зрелого мужа, видящего в нем выражение всей силы
заключенного в нем содержания. Таким образом, логическое получает полную
оценку своего значения лишь благодаря тому, что оно сделалось результатом
опыта наук. Этот опыт являет духу это логическое как всеобщую истину,
являет его не как некоторое {\em особое} ведение,
стоящее {\em наряду} с другими материями и
реальностями, а как сущность всего этого прочего содержания.

Хотя логика в начале ее изучения не существует для духа в этой сознательной
силе, он все же отнюдь не в меньшей мере воспринимает в себя благодаря
этому изучению ту силу, которая ведет его ко всяческой истине. Система
логики есть царство теней, мир простых сущностей, освобожденный от всякой
чувственной конкретности. Изучение этой науки, длительное пребывание и
работа в этом царстве теней есть абсолютная культура и дисциплинирование
сознания. Последнее занимается здесь делом, далеким от чувственных
созерцаний и целей, от чувств, от мира представлений, носящих лишь характер
мнения. Рассматриваемое со своей отрицательной стороны, это занятие состоит
в недопущении случайности резонирующего мышления и произвольного
взбредания в голову и признания правильными тех или иных из
противоположных оснований.

Но главным результатом этого занятия является то, что мысль приобретает
благодаря ему самостоятельность и независимость. Она привыкает вращаться в
абстракциях и двигаться вперед с помощью понятий без чувственных
субстратов, становится бессознательной мощью, способностью вбирать в себя
все остальное многообразие сведений и наук в разумную форму, понимать и
удерживать их в том, что в них существенно, отбрасывать внешнее и, таким
образом, извлекать из них логическое, или, что то же самое, наполнять
приобретенную прежде посредством изучения абстрактную основу логического
содержимым всяческой истины и сообщать ему (логическому) ценность такого
всеобщего, которое уже больше не стоит как некое особенное наряду с другими
особенными, а возвышается над всеми этими особенностями и составляет их
сущность, абсолютно истинное.

\subsection[\hspace{8mm}Общее подразделение логики]{Общее подразделение логики}
Из того, что нами было сказано о
{\em понятии} этой науки и о том, где должно находить
себе место его оправдание, вытекает, что здесь общее
{\em подразделение} может быть лишь
{\em предварительным}, может быть указано как бы лишь
постольку, поскольку автор уже знаком с этой наукой и потому в состоянии
здесь наперед указать {\em исторически}, в какие
основные различия определит себя понятие в ходе своего развития.

Можно, однако, попытаться наперед сделать в общем понятным то, что требуется
для {\em подразделения}, хотя и для этого приходится
прибегать к методу, делающемуся совершенно понятным и получающему свое
полное оправдание только в рамках самой науки. Итак, прежде всего следует
напомнить, что мы здесь исходим из предпосылки, что
{\em подразделение} должно находиться в связи с
{\em понятием} или, вернее, лежать в нем самом. Понятие
не неопределенно, а {\em определено} в самом себе,
подразделение же выражает в {\em развитом} виде эту его
{\em определенность;} оно есть его суждение\footnote{
Urteil (суждение) буквально перво-часть, urteilen (судить) "--- перво-делить.
"--- {\em Перев}.}, не суждение {\em о} каком-нибудь
внешне взятом предмете, а процесс суждения (das Urteilen), т.~е. процесс
{\em определения} понятия в нем же самом.
Прямоугольность, остроугольность и~т.~д. так же, как и равносторонность
и~т.~д., по каковым определениям делят треугольники, не заключаются в
определенности самого треугольника, т.~е. в том, что обыкновенно называют
понятием треугольника, точно так же, как в том, что принимают за понятие
животного вообще или за понятие млекопитающего, птицы и~т.~д., не
заключаются те определения, по которым животные подразделяются на
млекопитающих, птиц и~т.~д., а эти классы "--- на дальнейшие роды. Такие
определения берутся из другого источника, из эмпирического созерцания; они
привходят к вышеупомянутым так называемым понятиям извне. В~философской же
трактовке деления само понятие должно показать себя содержащим в себе
источник его ({\em деления}) происхождения.

Но самое понятие логики, как указано во введении, есть результат науки,
лежащей по ту сторону ее, и, стало быть, принимается здесь равным образом
как {\em предпосылка}. Логика согласно этому
определилась как наука чистого мышления, имеющая своим принципом
{\em чистое знание}, не абстрактное, а конкретное,
живое единство, полученное благодаря тому, что в нем знаема как
преодоленная противоположность сознания о некоем субъективно
{\em само по себе существующем} и сознания о некоем
втором таком же {\em существующем}, о некоем
объективном, и бытие знаемо как чистое понятие в самом себе, а чистое
понятие "--- как истинное бытие. Это, стало быть, те два
{\em момента}, которые содержатся в логическом. Но их теперь знают как
существующие {\em нераздельно}, а
не, как в сознании, как существующие каждое {\em также
и само по себе}. Только благодаря тому, что их вместе с тем знают как
{\em отличные} друг от друга (однако не сущие сами по
себе), их единство не абстрактно, мертвенно, неподвижно, а конкретно.

Это единство составляет логический принцип вместе с тем и в качестве
{\em стихии} (Element), так что развитие вышеуказанного
различия, которое сразу же имеется в ней, совершается только
{\em внутри} этой стихии. Ибо так как подразделение,
как было сказано, есть {\em суждение} понятия,
полагание уже имманентного ему определения и, стало быть, его различия, то
это полагание не должно пониматься как обратное разложение указанного
конкретного единства на его определения, которые должны были бы считаться
существующими сами по себе, ибо это было бы здесь пустым возвращением к
прежней точке зрения, к противоположности сознания. Нет. Последняя уже
преодолена; вышеуказанное единство остается стихией [логического], и из
него уже больше не выходит то различение, которое составляет неотъемлемую
черту подразделения и вообще развития. Тем самым определения, которые
прежде (на {\em пути к истине}), как бы их ни
определяли в каком-либо другом отношении, были
{\em существующими} сами по себе, как, например, некое
субъективное и некое объективное, или же мышление и бытие, или понятие и
реальность, {\em теперь в их истине}, т.~е. в их
единстве, низведены на степень {\em форм}. Они поэтому
в самом своем различии остаются {\em в~себе} целостным
понятием, и последнее полагается в подразделении только под своими
собственными определениями.

Таким образом, целостное понятие должно рассматриваться, во-первых, как
{\em сущее} понятие и, во-вторых, как {\em понятие;} в первом случае оно
{\em есть} только понятие {\em в
себе}, понятие реальности или бытия; во втором случае оно есть понятие как
таковое, {\em для себя сущее понятие} (каково оно
"--- назовем только конкретные формы "--- в мыслящем человеке,
но уже также, хотя и не как {\em сознательное} и еще того менее как
{\em знаемое} понятие, в чувствующем животном и в
органической индивидуальности вообще; понятием же
{\em в~себе} оно бывает лишь в неорганической природе).
Логику согласно этому следовало бы прежде всего делить на логику
{\em понятия} как {\em бытия} и понятия как {\em понятия}, или, пользуясь
хотя обычными, но и самыми неопределенными, а потому и самыми многозначными
выражениями, на {\em объективную} и {\em субъективную} логику.

Но далее, сообразно с лежащей в основании стихией единства понятия в самом
себе и, следовательно, нераздельности его определений, последние, поскольку
они {\em различны}, поскольку понятие полагается в их
{\em различии}, должны также находиться по крайней мере
в {\em соотношении} друг с другом. Отсюда получается
некая сфера {\em опосредствования}, понятие как система
{\em рефлективных определений}, т.~е. как система
бытия, переходящего во {\em внутри-себя-}бытие понятия,
которое (понятие), таким образом, еще не положено,
{\em как таковое}, для себя, а вместе с тем обременено
непосредственным бытием как неким также и внешним ему бытием. Это
"--- {\em учение о сущности}, помещающееся посередине
между учением о бытии и учением о понятии. "--- В общем подразделении
предлежащего логического произведения оно помещено еще в
{\em объективной} логике, поскольку, хотя сущность и
есть уже внутреннее, но характер {\em субъекта} следует
определенно сохранить за понятием.

В новейшее время {\em Кант}\footnote{
Я напомню, что в настоящем сочинении я потому так часто принимаю во
внимание кантовскую философию (это некоторым читателям может казаться
нелишним), что, как бы ни смотрели другие, а также и мы в настоящем
сочинении на детали, на отдельные ее черты, равно как и на разработку
особенных частей ее, она все же составляет основу и исходный пункт новейшей
немецкой философии; и эта ее заслуга остается неумаленной тем, что в ней
подлежит критике. Ее приходится часто принимать во внимание в объективной
логике также и потому, что она подвергает тщательному рассмотрению важные,
{\em более определенные} стороны логического, между тем как
позднейшие изложения философии, напротив, уделяли ему мало внимания и часто
только высказывали по отношению к нему грубое, но не оставшееся без
возмездия, презрение. Философское учение, пользующееся у нас наиболее
широким распространением, {\em не}~идет дальше кантовских
выводов, согласно которым разум не способен познать никакого истинного
содержания и в отношении абсолютной истины следует отсылать к вере. Но это
философствование непосредственно начинает тем, что у Канта представляет
собою только вывод, и этим наперед отрезывает предшествующие соображения,
из которых вытекает указанный вывод и которые именно и представляют собою
философское познание. Кантовская философия служит, таким образом, подушкой
для лености мысли, успокаивающейся на том, что все уже доказано и порешено.
За познанием и определенным содержанием мышления, которых не найти в таком
бесплодном и сухом успокоении, следует, поэтому, обращаться к указанным
предшествующим соображениям.} поставил наряду с тем,
что обычно называлось логикой, еще одну, а именно,
{\em трансцендентальную логику}. То, что мы здесь
назвали {\em объективной} логикой, частью
соответствовало бы тому, что у него является
{\em трансцендентальной логикой}. Он различает между
нею и тем, что он называет общей логикой, следующим образом:
трансцендентальная логика ($\alpha$) рассматривает те понятия, которые
a~priori относятся к {\em предметам}, и, следовательно,
не абстрагируется от всякого {\em содержания}
объективного познания, или, как он это выражает иначе, она заключает в себе
правила чистого мышления о каком бы то ни было
{\em предмете} и ($\beta$) вместе с тем она исследует
происхождение нашего познания, поскольку оно (познание) не может быть
приписано предметам. На эту-то вторую сторону исключительно направлен
философский интерес Канта. Основная его мысль заключается в том, что
{\em категории} следует признать чем-то принадлежащим
самосознанию, как {\em субъективному}
<<{\em я}>>. Вследствие этой черты кантовского учения оно
застревает в сознании и его противоположности и оставляет существовать
кроме эмпирических данных чувства и созерцания еще нечто такое, что не
положено мыслящим самосознанием и не определено им, "---
{\em вещь в себе}, нечто чуждое и внешнее мышлению,
хотя нетрудно усмотреть, что такого рода абстракция, как
{\em вещь в себе}, сама есть лишь продукт мышления и
притом только абстрагирующего мышления. Если другие
кантианцы\pagenote{Имеются в виду И.~Г.~Фихте и его единомышленники.}
выразились об определении {\em предмета} через <<я>> в
том смысле, что объективирование этого <<я>> должно быть рассматриваемо как
некое первоначальное и необходимое делание сознания, так что в этом
первоначальном делании еще нет представления о самом <<я>>, каковое
представление есть только некое сознание указанного сознания, или даже
объективирование этого сознания, то это освобожденное от противоположности
сознания объективирующее делание оказывается при более близком рассмотрении
тем, что можно считать вообще {\em мышлением} как
таковым\footnote{Если выражение <<{\em объективирующее}
делание>> <<я>> может напомнить о других продуктах духа, например, о продуктах
{\em фантазии}, то следует заметить, что речь идет об
определении предмета, поскольку его содержательные моменты
{\em не}~принадлежат области {\em чувства} и
{\em созерцания}. Такой предмет есть некая
{\em мысль}, и определить его означает частью впервые его
продуцировать, частью же, поскольку он есть нечто предположенное, иметь о
нем дальнейшие мысли, мыслительно развивать его далее.}. Но это делание не
должно было бы больше называться сознанием; сознание заключает в себе
противоположность <<я>> и его предмета, а этой противоположности нет в
указанном первоначальном делании. Название <<сознание>> еще больше
набрасывает тень субъективности на это делание, чем выражение <<мышление>>,
которое, однако, здесь следует понимать вообще в абсолютном смысле как
мышление {\em бесконечное}, не обремененное конечностью
сознания, короче говоря, под этим выражением следует понимать
{\em мышление как таковое}.

Так как интерес кантовской философии был направлен на так называемое
{\em трансцендентальное} в определениях мысли, то
рассмотрение самих этих определений не привело к содержательным
заключениям. Вопрос о том, что они такое сами в себе, помимо их
абстрактного, во всех них одинакового отношения к <<я>>, каковы их
определенность в сравнении друг с другом и их отношение друг к другу, не
сделан у Канта предметом рассмотрения; познание их природы поэтому ни
малейше не было подвинуто вперед указанной философией. Единственно
интересное, имеющее отношение к этому вопросу, мы находим в критике идей.
Но для действительного прогресса философии было необходимо, чтобы интерес
мышления был привлечен к рассмотрению формальной стороны, <<я>>, сознания как
такового, т.~е. абстрактного отношения некоего субъективного знания к
некоему объекту, чтобы таким образом было положено начало познанию
{\em бесконечной формы}, т.~е. понятия. Однако, чтобы
достигнуть этого познания, нужно было еще откинуть ту вышеупомянутую
конечную определенность, в которой форма представлена как <<я>>, сознание.
Форма, продуманная таким образом в ее чистоте, содержит в себе самой
процесс {\em определения} себя, т.~е. сообщения себе
содержания и притом сообщения себе последнего в его необходимости "--- в виде
системы определений мысли.

Объективная логика, таким образом, занимает скорее место прежней
{\em метафизики}, каковая была высившимся над миром
научным зданием, которое, как полагали, воздвигается исключительно только с
помощью {\em мыслей}. Если будем иметь в виду
выступивший в ходе развития этой науки последний ее
образ\pagenote{Имеется в виду метафизика Христиана Вольфа и
его последователей.},
то мы должны сказать, во-первых, что объективная логика непосредственно
занимает место {\em онтологии}, той части указанной
метафизики, которая должна была исследовать природу ens (сущего) вообще;
<<ens>> обнимает собою как {\em бытие}, так и
{\em сущность}, для какового различия наш язык, к
счастью, сохранил разные выражения. Но объективная логика охватывает кроме
того также и остальные части метафизики, поскольку эта последняя стремилась
постигнуть посредством чистых форм мысли особенные субстраты,
заимствованные ею первоначально из области представления, "--- душу, бога,
мир "--- и поскольку {\em определения мышления} составляли
{\em существенную} сторону ее способа рассмотрения. Но
логика рассматривает эти формы вне связи с указанными субстратами, с
субъектами {\em представления}, рассматривает их
природу и ценность, взятые сами по себе. Указанная метафизика не сделала
этого и навлекла на себя справедливый упрек в том, что она пользовалась ими
{\em без критики}, без предварительного исследования,
способны ли они и как они способны быть, по выражению Канта, определениями
вещи в себе или, скажем мы правильнее, разумного. "--- Объективная логика есть
поэтому подлинная критика их, критика, рассматривающая их не согласно
абстрактной форме априорности, противопоставляя ее апостериорному, а их
самих в их особенном содержании.

{\em Субъективная логика} есть логика
{\em понятия}, сущности, которая сняла свое соотношение
с некоторым бытием или, иначе говоря, со своей видимостью и которая теперь
уже не внешняя в своем определении, а есть свободное, самостоятельное,
определяющее себя внутри себя субъективное, или, правильнее, есть сам
{\em субъект}. Так как выражение
<<{\em субъективное}>> приводит к недоразумениям,
поскольку оно может быть понято в смысле чего-то случайного и
произвольного, равно как вообще в смысле определений, входящих в состав
формы {\em сознания}, то не следует здесь придавать
особое значение различию между субъективным и объективным, которое позднее
найдет свое более детальное развитие в рамках самой логики.

Логика, следовательно, хотя и распадается вообще на
{\em объективную и субъективную} логику, все же имеет
более определенно следующие три части:
\begin{enumerate}[~~~~I.]
\item{\em Логику бытия,}
\item{\em Логику сущности} и
\item{\em Логику понятия.}
\end{enumerate}

\bigskip
