Бытие есть неопределенное непосредственное. Оно свободно от определенности
по отношению к сущности, равно как еще свободно от всякой определенности,
которую оно может получить внутри самого себя. Это не имеющее рефлексии
бытие есть бытие, как оно есть непосредственно лишь в самом себе.

Так как оно неопределенно, то оно есть бескачественное бытие. Но
\em{в себе} ему принадлежит характер неопределенности
лишь в противоположность к \em{определенному} или
качественному. Но бытию вообще противостоит
\em{определенное} бытие как таковое, а благодаря этому
сама его неопределенность составляет его качество. Тем самым обнаружится,
что \em{первое} бытие есть определенное в себе и что,
следовательно:

\em{во-вторых}, оно переходит в
\em{наличное бытие}, есть
\em{наличное бытие}, но что последнее как конечное
бытие снимает себя и переходит в бесконечное соотношение бытия с самим
собою,

переходит, \em{в-третьих}, в
\em{для-себя-бытие}.

\chapter*{Первая глава. Бытие.}
\section*{А. Бытие}

\em{Бытие}, \em{чистое бытие} "--- без всякого дальнейшего определения.
В~своей неопределенной непосредственности оно равно лишь
самому себе, и~оно также и~не~неравно по отношению к~другому, не~имеет
никакой разности ни внутри себя, ни по~отношению к~внешнему. Если бы
в~нём было какое-либо определение или
содержание, отличное от~другого определения в~нём~же, или же такое
определение или содержание, которым оно отличается от~некоего другого
бытия, то~такое различие нарушило~бы его чистоту. Бытие есть чистая
неопределенность и~пустота. --- В~нём нечего созерцать, если здесь
может итти речь о~созерцании, или, иначе~говоря, оно есть~только само
это чистое, пустое созерцание.
В~нём столь же мало есть нечто такое, что можно было~бы
мыслить, или, иначе~говоря, оно равным образом есть~лишь это пустое
мышление. Бытие, неопределенное, непосредственное, есть~на~деле
\em{ничто} и~не~более и~не~менее, чем~ничто.

\section*{В. Ничто}
\em{Ничто}, \em{чистое ничто}; оно
есть простое равенство с~самим собою, совершенная пустота, отсутствие
определений и~содержания; неразличенность в~самом~себе. --- Поскольку
здесь можно говорить о~созерцании или мышлении, следует сказать,
что~считается небезразличным, созерцаем~ли мы, а~также мыслим~ли мы
нечто~или \em{ничто}. Выражение «созерцать или мыслить ничто»,
следовательно, что-то означает. Мы проводим различие между этими двумя
случаями; таким образом, ничто \em{есть} (существует)
в нашем созерцании или мышлении; или, вернее, оно~и~есть само~это пустое
мышление и~созерцание; и~оно~есть то~же~самое пустое созерцание или
мышление, что~и чистое бытие. --- Ничто есть, стало быть, то~же~самое
определение или, вернее, то~же~самое отсутствие определений и,~значит,
вообще то~же~самое, что~и чистое \em{бытие}.

\section*{С. Становление}
\subsection{1. Единство бытия и ничто}
\em{Чистое бытие и~чистое ничто есть},
\em{следовательно}, \em{одно и~то~же}.
Истина состоит не~в~бытии и~не~в~ничто, а~в~том, что~бытие "---
не~переходит, а "--- перешло в~ничто, и~ничто "---~не~переходит,
а "---~перешло в~бытие. Но равным образом истина заключается
не~в~их неразличенности, а~в~том, что \em{они не одно~и~то~же}, что
\em{они абсолютно различны}, но~столь~же и~нераздельны
и~неотделимы и~что \em{каждое} из~них непосредственно
\em{исчезает в~своей противоположности}. Их истина
есть, следовательно, это \em{движение}
непосредственного исчезновения одного в другом:
\em{становление}; такое~движение, в~котором они оба
различны, но~таким~различием, которое столь~же непосредственно
растворилось.

\subsection{Примечание 1. Противоположность бытия и ничто
  в представлении [ссылка!]}

\em{Ничто} обыкновенно противополагается [категории]
\em{нечто}; но нечто есть уже некое определенное,
сущее, отличающееся от другого \em{нечто}; таким
образом и ничто, противополагаемое [категории] нечто, есть ничто
какого-либо нечто, некое определенное ничто. Но здесь мы должны брать ничто
в его неопределенной простоте. "--- Если бы кто-нибудь считал более правильным
противополагать бытию \em{небытие} вместо ничто, то мы
не имели бы, что возразить против этого в рассуждении получающегося
результата, ибо в \em{небытии} содержится соотношение
с \em{бытием}; оно есть и то и другое, бытие и его
отрицание, высказанные одним духом, ничто, каково оно в становлении. Но
здесь ближайшим образом идет речь не о форме противоположения,~т.~е.
одновременно и о форме \em{соотношения}, а об
абстрактном, непосредственном отрицании, о ничто, взятом чисто само по
себе, о безотносительном отрицании, "--- что, если угодно, можно было бы
выразить также и простым «\em{не}».

Простую мысль о \em{чистом бытии} как об абсолютном и
как единственную истину высказали впервые \em{элеаты},
преимущественно \em{Парменид}, и последний в
сохранившихся после него фрагментах высказал ее с чистым воодушевлением
мышления, в первый раз постигшего себя в своей абсолютной абстрактности:
\em{лишь бытие есть}, \em{а
небытия вовсе нет}. "--- В восточных системах, главным образом в буддизме,
\em{ничто}, пустота, является, как известно,
абсолютным принципом. "--- Глубокомысленный \em{Гераклит}
выдвинул в противоположность вышеуказанной простой и односторонней
абстракции более высокое, целостное понятие становления и сказал:
\em{бытие столь же мало есть},
\em{как и небытие}, или, выражая эту мысль также и
иначе, говорил: «все\em{ течет}»,~т.~е. все есть
\em{становление}. "--- Популярные, в особенности
восточные, изречения, гласящие, что все, что есть, носит зародыш своего
уничтожения в самом своем рождении, а смерть есть, наоборот, вступление в
новую жизнь, выражают в сущности то же самое единение бытия и ничто. Но эти
выражения предполагают субстрат, в котором совершается переход: бытие и
ничто удерживаются вне друг друга во времени, представляются как бы
чередующимися в нем, а не мыслятся в их абстрактности, и поэтому также и не
мыслятся так, чтобы они сами по себе были одним и тем же.

«Ex nihilo nihil fit» (ничто не происходит из ничего) "---~есть одно из тех
положений, которым некогда приписывалось в метафизике большое значение. В
этом положении можно либо усматривать лишь бессодержательную тавтологию:
ничто есть ничто; либо, если действительным смыслом этого положения
является высказывание о \em{становлении}, приходится
сказать, что так как из \em{ничего становится}
\em{ничто же}, то на самом деле здесь нет речи о
становлении, ибо ничто здесь так и остается ничем. «Становление» означает,
что ничто не остается ничем, а переходит в свое другое, в бытие. "--- Если
позднейшая метафизика, главным образом христианская, отвергла положение,
гласящее, что из ничего ничего не происходит, то она этим утверждала
переход ничто в бытие; как бы синтетически или, иначе сказать, в форме
просто представления она ни брала последнее положение, все же даже в самом
несовершенном соединении имеется точка, в которой бытие и ничто встречаются
и их различие исчезает. "--- Подлинную свою важность положение:
\em{из ничего ничего не происходит},
\em{ничто есть именно ничто}, получает благодаря его
антагонизму к \em{становлению} вообще и,
следовательно, также к сотворению мира из ничего. Те, которые утверждают
положение: ничто есть именно ничто, и даже горячо отстаивают его, не
сознают того, что они тем самым соглашаются с абстрактным
\em{пантеизмом} элеатов и по сути дела также и со
спинозовским пантеизмом. Философское воззрение, которое считает принципом
положение: «бытие есть только бытие, ничто есть только ничто», заслуживает
названия системы тождества; это абстрактное тождество представляет собою
сущность пантеизма.

Если вывод, что бытие и ничто суть одно и то же, взятый сам по себе, кажется
удивительным или парадоксальным, то мы в дальнейшем не должны обращать на
это внимания; скорее приходится удивляться этому удивлению, которое
показывает себя таким новичком в философии и забывает, что в этой науке
встречаются совсем иные определения, чем те, которые имеют место в
обыденном сознании и в так называемом здравом человеческом рассудке,
который как раз не всегда есть здравый, а есть также рассудок, специально
культивированный для абстракций и для веры в них или, вернее, для
суеверного отношения к абстракциям. Было бы нетрудно показать наличие этого
единства бытия и ничто на всяком примере во всякой действительной вещи или
мысли. Относительно \em{бытия} и
\em{ничто} следует сказать то же самое, что мы сказали
выше относительно непосредственности и опосредствования (каковое последнее
заключает в себе некое соотношение \em{друг с другом}
и, значит, \em{отрицание}), а именно, что
\em{нет ничего ни на небе, ни на земле, что не
содержало бы в себе и того и другого, и бытия и ничто}. Так как при этом
начинают говорить о \em{каком-нибудь определенном
нечто и действительном}, то, разумеется, в этом нечто указанные определения
уже больше не наличествуют в той совершенной неистинности, в каковой они
выступают как бытие и ничто, а в некотором дальнейшем определении и
понимаются, например, как \em{положительное} и
\em{отрицательное}; первое есть положенное,
рефлектированное бытие, а последнее есть положенное, рефлектированное
ничто; но положительное и отрицательное содержат в себе как свою
абстрактную основу первое "---~бытие, а второе "---~ничто. "--- Так, например, в
самом боге качество, \em{деятельность},
\em{творение}, \em{могущество}
и~т.~д. содержат в себе по существу определение отрицательного, "--- они суть
продуцирование некоторого \em{другого}. Но
эмпирическое пояснение указанного положения примерами было бы здесь
совершенно излишне. Так как мы теперь знаем раз навсегда, что это единство
бытия и ничто лежит в основании в качестве первой истины и составляет
стихию всего последующего, то помимо самого становления все дальнейшие
логические определения: наличное бытие, качество, да и вообще все понятия
философии служат примерами этого единства. А так называющий себя
обыкновенный или здравый человеческий рассудок, поскольку он отвергает
нераздельность бытия и ничто, мы можем пригласить сделать попытку отыскать
пример, в котором мы могли бы найти одно отделенным от другого (нечто от
границы, предела, или бесконечное, бог, как мы только что упомянули, от
деятельности). Только пустые, сочиненные мыслью вещи (Gedankendinge)
"---~бытие и ничто "---~только сами они и суть такого рода раздельные, и их-то
вышеуказанный рассудок предпочитает истине, нераздельности, которую мы
повсюду имеем перед собой.

Нашим намерением не может быть всесторонне предупредить сбивчивые
возражения, путаные соображения, выдвигаемые обыденным сознанием, когда оно
имеет дело с таким логическим положением, ибо они неисчислимы. Мы можем
упомянуть лишь о некоторых из них. Одной из причин такой путаницы служит,
между прочим, то обстоятельство, что сознание привносит в такие абстрактные
логические положения представления о некотором конкретном нечто и забывает,
что речь идет вовсе не о таковом, а лишь о чистых абстракциях бытия и ничто
и что следует твердо держаться исключительно лишь этих последних.

Бытие и небытие суть одно и то же; \em{следовательно},
все равно, существую ли я или не существую, существует ли или не существует
этот дом, обладаю ли я или не обладаю ста талерами. Это умозаключение или
применение указанного положения совершенно меняет его смысл. В указанном
положении говорится о чистых абстракциях бытия и ничто; применение же
делает из них определенное бытие и определенное ничто. Но об определенном
бытии, как уже сказано, здесь нет речи. Определенное, конечное бытие есть
такое бытие, которое соотносится с чем-либо другим; оно есть содержание,
находящееся в отношении необходимости с другим содержанием, со всем миром.
Имея в виду взаимоопределяющую связь целого, метафизика могла выставить "---~в
сущности говоря, тавтологическое "---~утверждение, что если бы была разрушена
одна пылинка, то обрушилась бы вся вселенная. В примерах, приводимых против
рассматриваемого нами положения, представляется небезразличным, есть ли
нечто или его нет, не из-за бытия или небытия, а из-за его
\em{содержания}, связывающего его с другими
содержаниями. Когда \em{предполагается} некое
определенное содержание, какое-либо определенное существование, то это
существование, именно потому, что оно
"---~\em{определенное}, находится в многообразном
соотношении с другим содержанием. Для него небезразлично, есть ли известное
другое содержание, с которым оно находится в соотношении, или его нет, ибо
только благодаря такому соотношению оно существенно есть то, что оно есть.
То же самое имеет место и в \em{представлении}
(поскольку мы берем небытие в более определенном смысле, в котором оно
означает то, что мы представляем себе, в противоположность тому, что
действительно существует), в связи которого небезразлично, имеется ли бытие
или отсутствие некоторого содержания, которое как определенное
представляется нами в его соотношении с другим содержанием.

Это соображение захватывает то, что составляет один из главных моментов в
кантовской критике онтологического доказательства бытия божия, которого,
однако, мы здесь касаемся лишь в отношении встречающегося в нем различения
между бытием и ничто вообще и между \em{определенными}
бытием и небытием. "--- Как известно, в этом так называемом доказательстве
предпосылается, понятие некоторого существа, обладающего всякой реальностью
и, следовательно, также и существованием, которое также принималось за одну
из реальностей. Кантова критика напирает, главным образом, на то, что
\em{существование} или бытие (которые здесь считаются
равнозначными) не составляет \em{свойства} или
\em{реального предиката},~т.~е. не составляет понятия
чего-то такого, что может прибавиться к \em{понятию}
некоторой вещи\footnote{ «Kants Kritik der r. Vern.», 2-te Aufl., 628
[ссылка!].}. --- Кант хочет этим сказать, что бытие не
есть определение содержания. --- Стало быть, продолжает он дальше,
действительное не содержит в себе чего-либо большего, чем возможное; сто
действительных талеров не содержат в себе ни капельки больше, чем сто
возможных талеров, а именно, первые не имеют другого определения
содержания, чем последние.

Для этого, рассматриваемого как изолированное, содержания в самом деле
безразлично быть или не быть; в нем не имеется различия бытия или небытия,
это различие вообще совсем не затрагивает его: сто талеров не сделаются
меньше, если их нет, и больше, если они есть. Различие должно притти из
другой сферы. "--- «Напротив, "--- напоминает Кант, "--- мое имущественное состояние
больше при ста действительных талерах, чем при голом их понятии, или, иначе
говоря, чем при их возможности. Ибо \em{предмет},
когда он действителен, не содержится только аналитически в моем понятии, а
\em{присоединяется к моему понятию} (которое есть
некоторое \em{определение} моего
\em{состояния}) \em{синтетически}
без того, чтобы через это бытие вне моего понятия сами эти мыслимые сто
талеров хотя бы сколько-нибудь увеличились».

Здесь \em{предполагается} "---~если сохранить выражения
Канта, не свободные от тяжеловесной запутанности "---~двоякого рода состояния:
одно, которое Кант называет понятием, и под которым следует разуметь
представление, и другое "---~состояние имущества. Для одного, как и для
другого, для имущества, как и для представления, сто талеров суть некоторое
определение содержания, или, как выражается Кант, «они присоединяются к
таковому \em{синтетически}». Я как
\em{обладатель} ста талеров или как необладатель их
или также я как \em{представляющий} себе сто талеров
или не представляющий себе их есть, во всяком случае, разное содержание.
Сформулируем в более общем виде: абстракции бытия и ничто перестают быть
абстракциями, когда они получают определенное содержание; бытие есть тогда
реальность, определенное бытие ста талеров, ничто есть отрицание,
определенное небытие последних. Самое же это определение содержания, сто
талеров, рассматриваемое равным образом абстрактно, само по себе, остается
неизменным, тем же самым как в том, так и в другом случае. Но когда, далее,
бытие берется как имущественное состояние, сто талеров вступают в связь с
некоторым состоянием, и для последнего такого рода определенность, которую
они составляют, не безразлична; их бытие или небытие есть лишь
\em{изменение}; они перемещены в сферу
\em{наличного бытия}. Если, поэтому, против единства
бытия и ничто выдвигается возражение, что ведь не все равно, существует ли
то-то (100 талеров) или не существует, то возражающие впадают в
заблуждение, относя различие между моим
\em{обладанием} и
\em{необладанием} ста талерами только за счет бытия
или небытия. Это заблуждение, как мы показали, основано на односторонней
абстракции, опускающей имеющееся в такого рода примерах
\em{определенное наличное бытие} и удерживающей лишь
бытие и небытие, так же как и, наоборот, превращающей то абстрактное бытие
и ничто, которое мы здесь должны мыслить, в некоторое определенное бытие и
ничто, в некоторое наличное бытие. Лишь наличное бытие содержит в себе
реальное различие между бытием и ничто, а именно, некое
\em{нечто} и некое \em{другое}. "---
Это реальное различие предносится представлению вместо абстрактного бытия и
чистого ничто и лишь мнимого различия между ними.

Как выражается Кант, «через существование нечто вступает в контекст
совокупного опыта». «Благодаря этому мы получаем одним предметом
\em{восприятия} больше, но наше
\em{понятие} о предмете этим не
умножается»~[ссылка!]. "--- Это, как вытекает из предыдущего
разъяснения, означает следующее: через существование, именно потому, что
нечто есть определенное существование, оно находится в связи с
\em{другим} и, между прочим, также и с неким
воспринимающим. "--- «Понятие ста талеров, "--- говорит Кант, "--- не умножается
вследствие того, что их воспринимают». \em{Понятием}
Кант здесь называет вышеозначенные представляемые
\em{изолированно} сто талеров. В такой изолированности
они, правда, суть некоторое эмпирическое содержание, но содержание
оторванное, не связанное с \em{другим} и лишенное
определенности в отношении \em{другого}. Форма
тождества с собою лишает их отношения к другому и делает их безразличными к
тому, воспринимаются ли они или нет. Но это так называемое
\em{понятие} ста талеров есть ложное понятие; форма
простого соотношения с собою не принадлежит самому подобного рода
ограниченному, конечному содержанию; она есть форма, прибавленная и
внесенная в него субъективным рассудком; сто талеров суть не некоторое
соотносящееся с собою, а некоторое изменчивое и преходящее.

Мышление или представление, которому предносится лишь некое определенное
бытие "---~наличное бытие, "--- следует отослать к вышеупомянутому первому шагу
науки, сделанному Парменидом, который очистил свое представление и,
следовательно, тем самым также и представление последующих времен, возвысил
его до \em{чистой мысли}, до бытия как такового, и
этим создал стихию науки. "--- То, что является
\em{первым в науке}, должно было явить себя первым
также и \em{исторически}. И элеатское
\em{единое} или \em{бытие} мы
должны рассматривать как первый шаг знания о мысли:
\em{вода\textup{~}}[ссылка!] и тому
подобные материальные начала, хотя и \em{должны}, по
мысли выдвигавших их философов, представлять собою всеобщее, однако как
материи они не суть чистые мысли;
\em{числа}~[ссылка!] же не суть ни первая
простая, ни остающаяся у себя мысль, а мысль, всецело внешняя самой себе.

В отсылке от \em{особенного конечного} бытия к бытию,
как таковому, взятому в его совершенно абстрактной всеобщности, следует
видеть наипервейшее как теоретическое, так даже и практическое требование.
А именно, если так носятся с этими ста талерами, придают такую важность
тому указанию, что для моего имущественного состояния составляет разницу,
\em{обладаю} ли я ими или
\em{нет}, и что еще больше разницы, существую ли я или
нет, существует ли другое или нет, то "---~не говоря уже о том, что могут
существовать такие имущественные состояния, для которых такое обладание ста
талерами будет безразлично, "--- можно напомнить, что человек должен подняться
в своем умонастроении до такой абстрактной всеобщности, стоя на точке
зрения которой ему в самом деле будет все равно, существуют ли или не
существуют эти сто талеров, каково бы ни было их количественное отношение к
его имущественному состоянию, а также ему будет все равно, существует ли он
или нет,~т.~е. существует ли он или нет в конечной жизни (ибо имеется в
виду некое состояние, определенное бытие) и~т.~д. Даже si fractus illabatur
orbis, impavidum ferient ruinae (если бы на него обрушился весь мир, он без
страха встретит смерть под его развалинами), сказал один
римлянин~[ссылка!], а тем паче должно быть присуще такое
безразличие христианину.

Следует еще вкратце отметить непосредственную связь, в которой находится
возвышение над ста талерами и вообще над конечными вещами с онтологическим
доказательством и вышеприведенной кантовской критикой последнего. Эта
критика показалась всем убедительной благодаря приведенному ею популярному
примеру; кто же не знает, что сто действительных талеров отличны от ста
лишь возможных талеров, кто не знает, что это составляет разницу в моем
имущественном состоянии. Так как таким образом на примере ста талеров
явственно видна эта разница, то понятие,~т.~е. определенность содержания
как пустая возможность, и бытие разнятся друг от друга;
\em{стало быть}, понятие бога и его бытие также
различны, и сколь мало я могу из возможности ста талеров вывести их
действительность, столь же мало я могу из понятия бога «выколупать»
(herausklauben) его существование; а в таком выколупывании существования
бога из его понятия и состоит-де онтологическое доказательство. Но если
несомненно правильно, что понятие отлично от бытия, то бог еще более
отличен от ста талеров и других конечных вещей. В том и состоит
\em{дефиниция конечных вещей}, что в них понятие и
бытие различны, понятие и реальность, душа и тело отделены друг от друга, и
они, значит, преходящи и смертны; напротив, абстрактной дефиницией бога
является именно то, что его понятие и его бытие
\em{нераздельны} и
\em{неотделимы}. Истинная критика категорий и разума и
заключается как раз в том, чтобы просветить познание относительно этого
различия и удерживать его от применения определений и отношений конечного к
богу.
